\chapter{Introducción a la Computación}

% TODO: Copiar problema de la parada, es útil para demostrar que una cantidad de programas no pueden existir; ya que su existencia implicaría la existencia del programa de la parada

\begin{definicion}
    Un alfabeto es un conjunto finito.
\end{definicion}

A los elementos de los alfabetos los llamaremos símbolos o letras.  

Típicamente, los alfabetos se notan en letras mayúsculas y las letras en minúscula.

\begin{ejemplo}
    Ejemplos de alfabetos son:

    \begin{gather*}
        A = \{0, 1\} \\
        B = \{ \langle 0,0 \rangle , \langle 0,1 \rangle , \langle 1,0 \rangle , \langle 1,1 \rangle \}
    \end{gather*}
\end{ejemplo}


\begin{definicion}
    Una palabra o cadena sobre un alfabeto $A$ es una sucesión finita de elementos de $A$.
\end{definicion}
\begin{ejemplo}
    Por ejemplo, si nuestro alfabeto es $A = \{0,1\}$, una palabra es $11000$.
\end{ejemplo}

\begin{definicion}
    El conjunto de todas las palabras sobre un alfabeto $A$ se nota como $A^*$.
\end{definicion}
Las palabras se notan por las últimas letras del alfabeto, como $u$ y la longitud de estas por $|u|$.

\begin{definicion}
Existe la palabra vacía, es decir, la palabra de longitud cero.    
\end{definicion}

% Copiar propiedades

Hay una cantidad infinito no numerable de lenguajes

Todas las palabras generadas por el procedimiento forman parte del lenguaje generado por la gramática.

Tipo 1: AL sustituir una palabra por otra, tiene que haber un contexto

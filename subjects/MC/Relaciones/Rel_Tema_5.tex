\newpage
\section{Autómatas con Pila}

\begin{ejercicio}\label{ej:1.5.1}
    Determinar una gramática que acepte el lenguaje $N(M)$ donde,
    \begin{equation*}
        M = \{\{q_0, q_1\}, \{a,b\}, \{Z_0,X\},\delta,q_0,Z_0,\emptyset \}
    \end{equation*}
    y donde
    \begin{align*}
        \delta(q_0,a,Z_0) &= \{(q_0, XZ_0)\} \\
        \delta(q_0,a,X) &= \{(q_0, XX),(q_1,\veps)\} \\
        \delta(q_0,b,Z_0) &= \{(q_0, XZ_0)\} \\
        \delta(q_0,b,X) &= \{(q_0, XX)\} \\
        \delta(q_1,a,X) &= \{(q_1, \veps)\} \\
        \delta(q_1,\veps,Z_0) &= \{(q_1, \veps)\} 
    \end{align*}
    y el resto de  transiciones son el conjunto vacío.
\end{ejercicio}

\begin{ejercicio}\label{ej:1.5.2}
    Construir un autómata con pila que acepte el lenguaje
    \begin{equation*}
        \{a^ib^i \mid i\geq 0\} \cup \{a^i \mid i \geq 0\} \cup \{b^i \mid i \geq 0\}
    \end{equation*}
    \begin{enumerate}[label=\alph*)]
        \item Por el criterio de estados finales.
        \item Por el criterio de pila vacía.
    \end{enumerate}
    Indicar si el autómata es determinístico.
\end{ejercicio}

\begin{ejercicio}\label{ej:1.5.3}
    Obtener a partir de la gramática $G=(\{S,T\},\{a,b,c,d\},P,S)$, con
    \begin{equation*}
        P = \{S\rightarrow abS,\ S\rightarrow cdT,\ T\rightarrow bT,\ T\rightarrow b\}
    \end{equation*}
    un autómata con pila que acepta por el criterio de estados finales el lenguaje generado por esa gramática.
\end{ejercicio}

\begin{ejercicio}\label{ej:1.5.4}
    Demostrar que los siguientes lenguajes son libres del contexto y obtener para cada uno de ellos un autómata con pila no determinista que pueda ser usado como reconocedor:
    \begin{itemize}
        \item $L_1 = \{a^pb^q \mid p,q\geq 1;\ p>q\}$.
        \item $L_2 = \{a^pb^q \mid p,q\geq 1;\ p<q\}$.
        \item $L_3 = \{a^pb^qa^r \mid p+q\geq r\geq 1\}$.
    \end{itemize}
\end{ejercicio}

\begin{ejercicio}\label{ej:1.5.5}
    Dado el lenguaje
    \begin{equation*}
        L = \{a^ib^jc^{i+j}\mid i,j\in \mathbb{N}\}
    \end{equation*}
    \begin{itemize}
        \item y haciendo uso de resultados matemáticos concretos, identifica a que tipos de lenguajes NO pertenece $L$.
        \item además encuentra, si es posible, un reconocedor para las cadenas de ese lenguaje.
    \end{itemize}
\end{ejercicio}

\begin{ejercicio}\label{ej:1.5.6}
    Construir un autómata con pila que por el criterio de estados finales acepte el lenguaje de las palabras sobre el alfabeto $\{0,1\}$ en las que el número de 0 es el doble que el número de 1. 

    Construir a partir del autómata una gramática libre de contexto en forma normal de Chomsky que genere el mismo lenguaje.
\end{ejercicio}

\begin{ejercicio}\label{ej:1.5.7}
    Construir un autómata con pila que acepte el lenguage sobre el alfabeto $A=\{a,b\}$ de todas aquella palabras en las que el número de símbolos $a$ es distinto del número de símbolos $b$. Construir una gramática en forma normal de Chomsky a partir de dicho autómata.
\end{ejercicio}

\begin{ejercicio}\label{ej:1.5.8}
    Dado $L=\{a^i b^j c^k a^i \mid i\geq 1,\ j\geq k \geq 1\}$ construir un autómata con pila que acepte dicho lenguaje por el criterio de pila vacía. Transformar dicho autómata en uno que lo acepte por el criterio de estados finales.
\end{ejercicio}

\begin{ejercicio}\label{ej:1.5.9}
    Construir un autómata con pila que acepte el lenguaje:
    \begin{equation*}
        L = \{a^i b^j c^k d^l \mid i+l = j+k\}
    \end{equation*}
\end{ejercicio}

\begin{ejercicio}\label{ej:1.5.10}
    Sea el alfabeto $A = \{0,1\}$ y para $u\in {\{0,1\}}^{\ast}$, sea $\overline{u}$ la palabra obtenida a partir de $u$ cambiando los 0 por 1 y los 1 por 0. Considerar el lenguaje $L = \{u\in {\{0,1\}}^{\ast} \mid u^{-1} = \overline{u}\}$.
    \begin{itemize}
        \item Dar una gramática en forma normal de Chomsky que acepte $L$.
        \item Dar un autómata con pila que acepte $L$ por el criterio de estados finales.
    \end{itemize}
\end{ejercicio}

\begin{ejercicio}\label{ej:1.5.11}
    Construir un autómata con pila que acepte el siguiente lenguaje:
    \begin{equation*}
        L = \{0^r 1^s \mid r\leq s \leq 2r\}
    \end{equation*}
    \begin{itemize}
        \item Construir, a partir de dicho autómata, una gramática libre de contexto que acepte el mismo lenguaje.
        \item Eliminar símbolos y producciones inútiles de la gramática.
    \end{itemize}
\end{ejercicio}

\begin{ejercicio}\label{ej:1.5.12}
    Construir autómatas con pila que acepten los siguientes lenguajes:
    \begin{enumerate}[label=\alph*)]
        \item El conjunto de todas las palabras $u$ con el mismo número de síbolos $a$ y $b$, y tal que en todo prefijo el número de símbolos $a$ es menor o igual que el número de símbolos $b$.
        \item $L = \{a^i b^j c^k \mid i=j\ \lor\ j = k\}$.
    \end{enumerate}
    Los autómatas deberán de ser determinísticos en caso de que sea posible.
\end{ejercicio}

\begin{ejercicio}\label{ej:1.5.13}
    Dado el autómata con pila $M = (\{q_0,q_1\},\{0,1\},\{Z_0,A,B\},\delta,q_0,Z_0,\emptyset )$, donde
    \begin{align*}
        \delta(q_0,0,Z_0) = \{(q_0,AAZ_0)\} &\delta(q_0, 0, A) = \{(q_0,AAA)\} \\
        \delta(q_0,0,B) = \{(q_1,\veps)\} &\delta(q_1, \veps, B) = \{(q_0,\veps)\} \\
        \delta(q_1,\veps,Z_0) = \{(q_0,A)\} &\delta(q_0, 1, Z_0) = \{(q_0,BZ_0)\} \\
        \delta(q_0,1,A) = \{(q_0,\veps)\} &\delta(q_0, 1, B) = \{(q_0,BB)\} 
    \end{align*}
    Encontrar una gramática libre de contexto que genere el mismo lenguaje que este autómata acepta por el criterio de pila vacía. Se valorará que se haga por el procedimiento explicado en clase.
\end{ejercicio}

\begin{ejercicio}\label{ej:1.5.14}
    Encontrar un autómata con pila que acepte el siguiente lenguaje
    \begin{equation*}
        L_1 = \{uvv^{-1}u^{-1}\mid u,v\in {\{0,1\}}^{\ast}\}
    \end{equation*}
\end{ejercicio}

\begin{ejercicio}\label{ej:1.5.15}
    Construir un autómata con pila determinístico que reconozca el lenguaje $L=L_1\cap L_2$ sobre el alfabeto $A = \{0,1,2\}$, donde
    \begin{itemize}
        \item $L_1$ es el conjunto de todas las palabras $u\in A^\ast$ tales que en todo prefijo $u'$ de $u$, la cantidad de símbolos 0 es mayor que la cantidad de 1.
        \item $L_2$ es el lenguaje de todas las palabras sobre $A$ que contienen la subcadena $0102$.
    \end{itemize}
\end{ejercicio}

\begin{ejercicio}\label{ej:1.5.16}
    Encontrar autómatas con pila para los siguientes lenguajes:
    \begin{itemize}
        \item $L_1 = \{a^i b^j c^k \mid i+k = j,\ i,j,k\geq 0\}$.
        \item $L = \{0^n 1^m 2^p 0^q 1^n \mid q = p + m,\ m\geq 1,\ p\geq 0\}$.
    \end{itemize}
\end{ejercicio}

\begin{ejercicio}\label{ej:1.5.17}
    Dado el alfabeto $A = \{0,1\}$,
    \begin{enumerate}[label=\alph*)]
        \item Construir un autómata con pila que acepte por el criterio de estados finales el conjunto de palabras con el triple de ceros que de unos.
        \item Construir una gramática independiente del contexto asociada al autómata.
    \end{enumerate}
\end{ejercicio}

\begin{ejercicio}\label{ej:1.5.18}
    Construir autómatas con pila (si es posible, deterministas) que acepten por el criterio de pila vacía los siguientes lenguajes sobre el alfabeto $\{0,1\}$:
    \begin{itemize}
        \item $L = \{0^n 1^n \mid n\geq 1\} \cup \{0^n 1^{2n}\mid n\geq 1\}$.
        \item $L = \{0^n 1^m 0^m 1^n \mid n,m\geq 1\}$.
    \end{itemize}
\end{ejercicio}

\begin{ejercicio}\label{ej:1.5.19}
    Dado el autómata con pila dado por las transiciones ($R$ es el símbolo inicial):
    \begin{equation*}
        \begin{array}{lll}
            &\delta(q_1,0,R) = \{(q_1,BR)\} &\delta(q_1,1,R) = \{(q_1,GR)\} &\delta(q_1,0,B) = \{(q_1,BB)\} \\
            &\delta(q_1,1,B) = \{(q_1,GB)\} &\delta(q_1,0,G) = \{(q_1,BG)\} &\delta(q_1,1,G) = \{(q_1,GG)\} \\
            &\delta(q_1,1,G) = \{(q_2,G)\} &\delta(q_2,0,B) = \{(q_2,\veps)\} &\delta(q_2,2,G) = \{(q_2,\veps)\} \\
            &\delta(q_2,\veps,R) = \{(q_2,\veps)\} &\delta(q_1,\veps,R) = \{(q_2,R)\} &\delta(q_1,\veps,B) = \{(q_2,B)\} \\
            &\delta(q_1,\veps,G) = \{(q_2,G)\} &\delta(q_1,0,R) = \{(q_2,R)\} &\delta(q_1,0,B) = \{(q_2,B)\} \\
            &\delta(q_1,0,G) = \{(q_2,G)\} &\delta(q_1,1,R) = \{(q_2,R)\} &\delta(q_1,1,B) = \{(q_2,B)\} 
        \end{array}
    \end{equation*}
    Construir una gramática independiente del contexto (siguiendo el procedimiento explicado en clase) que acepte el mismo lenguaje. Eliminar símbolos y producciones inútiles.
\end{ejercicio}

\begin{ejercicio}\label{ej:1.5.20}
    Construir autómatas con pila (si es posible, deterministas) que acepten por el criterio de estados finales los siguientes lenguajes sobre el alfabeto $\{0,1\}$:
    \begin{itemize}
        \item $L = \{0^i 1^j \mid j\geq i \geq 1\}$.
        \item $L = \{0^i 1^j 0^i \mid i,j\geq 1\}\cup \{1^i 0^j 1^i \mid i,j\geq 1\}$
    \end{itemize}
\end{ejercicio}

\begin{ejercicio}\label{ej:1.5.21}
    Construye un autómata con pila determinista por el criterio de estados finales que reconozca el lenguaje:
    \begin{equation*}
        L = \{ucv\mid u,v\in {\{a,b\}}^{+} \text{\ y nº de subcadenas '}ab \text{' en\ } u \text{\ es igual al nº subcadenas '} ba \text{' en\ }v\}
    \end{equation*}
    ¿Es posible encontrarlo por el criterio de pila vacía?
\end{ejercicio}

\begin{ejercicio}\label{ej:1.5.22}
    Sea el lenguaje sobre el alfabeto $A = \{a,b,c,d\}$, dado por las siguientes reglas:
    \begin{enumerate}[label=\alph*)]
        \item $a$ y $b$ son palabras del lenguaje.
        \item Cualquier sucesión no vacía de palabras del lenguaje es una palabra del lenguaje.
        \item Si $u$ es una palabra del lenguaje, entonces $cudd$ es una palabra del lenguaje.
    \end{enumerate}
    Decid de qué tipo es el lenguaje generado. Según sea el tipo del lenguaje, crear un autómata finito minimal o un autómata con pila que lo acepten.
\end{ejercicio}

\begin{ejercicio}\label{ej:1.5.23}
    Describir autómatas con pila para los siguientes lenguajes sobre el alfabeto $A = \{a,b,c\}$ (si es posible hacerlos determistas e indicar si se ha conseguido):
    \begin{enumerate}[label=\alph*)]
        \item Palabras de longitud impar con una $b$ en el centro.
        \item $L = \{a^n b^m c^k \mid n = m\ \lor\ n \leq 3\}$.
    \end{enumerate}
\end{ejercicio}

\begin{ejercicio}\label{ej:1.5.24}
    Supongamos un operador $\otimes$ que puede aparecer en el código de un lenguaje de programación con la siguiente estructura:
    \begin{equation*}
        \otimes(u,v);
    \end{equation*}
    donde
    \begin{itemize}
        \item $u\in {\{0,1\}}^{\ast}$ es una cadena de símbolos binarios que determina:
            \begin{enumerate}[label=\alph*)]
                \item La operación que se ejecutará. Si el número de 0's en la cadena $u$ es igual al números de 1's se realiza una suma, en caso contrario se hace un producto.
                \item El número de operandos de $\otimes$ (número de ocurrencias de la subcadena '$10$' en $u$).
            \end{enumerate}
        \item $v\in {\{a,b,c\}}^{\ast}$ es una cadena donde cada símbolo representa un operando de $\otimes$.
    \end{itemize}
    Construir, si es posible:
    \begin{enumerate}[label=\alph*)]
        \item Un autómata finito determinista que reciba como entrada la cadena $u$ y le indique al ordenador si realiza una suma (estado final) o si realiza un producto (estado no final).
        \item Construir un autómata con pila determinista por el criterio de pila vacía definido sobre el alfabeto de entrada
            \begin{equation*}
                A = \{'\otimes', '(', ')', ',', ';', '0', '1', 'a'\}
            \end{equation*}
            que reciba como entrada una expresión del conjunto
            \begin{equation*}
                \{\otimes(u,v); \mid u\in {\{0,1\}}^{\ast},\ v\in {\{a,b,c\}}^{\ast}\}
            \end{equation*}
            y nos informe si está bien construida sintácticamente (de acuerdo con lo especificado anteriormente) y si el número de operandos es correcto.
    \end{enumerate}
\end{ejercicio}

\begin{ejercicio}\label{ej:1.5.25}
    Construir un autómata con pila (si es posible, determinista) que reconozca el siguiente lenguaje:
    \begin{equation*}
        L_1 = \{a^i b^j c^k \mid i = 2j\ \lor\ j = 2k\}
    \end{equation*}
\end{ejercicio}

\begin{ejercicio}\label{ej:1.5.26}\ 
    \begin{enumerate}[label=\alph*)]
        \item Construye una gramática libre de contexto que genere el siguiente lenguaje en el alfabeto $\{a,b,c,d\}$:
            \begin{equation*}
                L = \{a^m b^n c^p d^q \mid m+n\geq p+q\}
            \end{equation*}
        \item Construye un autómata con pila determinista que reconozca las cadenas del anterior lenguaje $L$ por el criterio de estados finales.
    \end{enumerate}
\end{ejercicio}

\begin{ejercicio}\label{ej:1.5.27}
    Construye un autómata con pila que acepte el lenguaje sobre el alfabeto $A = \{a,b\}$ de todas aquellas palabras en las que el número de símbolos $a$ es distinto al número de símbolos $b$.
\end{ejercicio}

\begin{ejercicio}\label{ej:1.5.28}
    Dado el siguiente lenguaje libre de contexto: $L = \{{(01)}^{i}{(10)}^{j}\mid j\geq i \geq 1\}$
    \begin{enumerate}[label=(\alph*)]
        \item Encuentra una gramática libre de contexto que lo genere.
        \item Transforma la gramática anterior a un autómata con pila que acepte las cadenas del lenguaje $L$ por el criterio de pila vacía.
        \item Transforma el autómata con pila anterior para que acepte las cadenas por el criterio de estados finales.
        \item Encuentra un autómata con pila determinista que acepte las cadenas del lenguaje $L$ por el criterio de estados finales.
    \end{enumerate}
\end{ejercicio}

\subsection{Preguntas Tipo Test}
Se pide discutir la veracidad o falsedad de las siguientes afirmaciones:
\begin{enumerate}
    \item La clase de los lenguajes aceptados por los autómatas con pila deterministas es igual a la clase de los lenguajes generados por las gramáticas de tipo 2.
    \item Una palabra es aceptada por un autómata con pila por el criterio de pila vacı́a si en algún momento, cuando leemos esta palabra, la pila se queda sin ningún sı́mbolo, con independencia de la cantidad de sı́mbolos que hayamos leı́do de la palabra de entrada.
    \item Un autómata con pila siempre acepta el mismo lenguaje por los criterios de pila vacı́a y de estados finales.
    \item Todo lenguaje aceptado por un autómata con pila determinista por el criterio de estados finales es también aceptado por una autómata con pila determinista por el criterio de pila vacı́a.
    \item Para que un autómata con pila sea determinista es suficiente que desde cada configuración se pueda obtener, a lo más, otra configuración en un paso de cálculo.
    \item Si un lenguaje de tipo 2 verifica la propiedad prefijo y es aceptado por un autómata con pila determinista por el criterio de estados finales, entonces también es aceptado por un autómata con pila determinista por el criterio de pila vacı́a.
    \item Para todo autómata con pila existe otro autómata con pila que acepta el mismo lenguaje y tiene un solo estado.
    \item Si un lenguaje es aceptado por una autómata con pila determinista por el criterio de estados finales, entonces también es aceptado por un autómata con pila determinista por el criterio de pila vacı́a.
    \item En un autómata con pila determinista no puede haber transiciones nulas.
    \item Si $L$ es independiente del contexto determinista y $\$ \notin L$ entonces $L.\{\$\}$ es aceptado por un autómata con pila determinista por el criterio de pila vacı́a.
    \item El conjunto de las palabras $\{u0011u^{-1}\mid u\in {\{0,1\}}^{\ast}\}$ es libre del contexto determinista.
    \item En la construcción de una gramática independiente del contexto a partir de un autómata con pila, la variable $[p,X,q]$ genera todas las palabras que llevan al autómata desde el estado $p$ al estado $q$ sustituyendo $X$ por el sı́mbolo inicial de la pila.
    \item En un autómata con pila determinista no puede haber transiciones nulas.
    \item Todo autómata con pila determinista que acepta un lenguaje por pila vacı́a se puede transformar en otro autómata determinista que acepte el mismo lenguaje por el criterio de estados finales.
    \item Para que un lenguaje independiente del contexto sea determinista ha de verificar la propiedad prefijo.
    \item El lenguaje compuesto por las instrucciones completas del lenguaje SQL cumplen la propiedad prefijo.
    \item En el algoritmo para pasar un autómata con pila a gramática que hemos visto, si el autómata tiene 3 estados, entonces la transición $(p,XYZU) \in \delta(q,\veps, H)$ da lugar a $4^3$ producciones.
    \item El lenguaje $\{0^i 1^k 2^i \id i,j\geq 0\}$ es independiente del contexto determinista.
    \item Si tenemos un lenguaje $L$ aceptado por un Autómata con Pila por el criterio de estados finales, podemos encontrar otro $AP$ que reconozca $L$ por el criterio de pila vacı́a.
    \item La propiedad prefijo no tiene ninguna relación con el hecho de que un lenguaje sea aceptado por un autómata con pila determinista por estados finales.
    \item Para toda gramática libre de contexto $G$ siempre se puede encontrar un autómata con pila que acepte el lenguaje generado por $G$.
    \item Si un lenguaje independiente del contexto cumple la propiedad prefijo, entonces puede ser aceptado por un autómata con pila determinista por el criterio de pila vacı́a.
    \item La descripción instantánea de un autómata con pila nos permite saber el estado activo, lo que queda por leer de la cadena de entrada, lo que se ha consumido de la cadena de entrada y lo que nos queda en la pila.
    \item Un autómata finito determinista se puede convertir en un autómata con pila que acepta el mismo lenguaje por el criterio de pila vacı́a.
    \item El conjunto de cadenas generado por una gramática libre de contexto en forma normal de Greibach puede ser reconocido por un autómata finito no determinista con transiciones nulas.
    \item Los lenguajes independientes del contexto con la propiedad prefijo son siempre reconocidos por un autómata con pila determinista por el criterio de pila vacı́a.
    \item Puede existir un lenguaje con pila determinista que no sea aceptado por un autómata con pila determinista por el criterio de estados finales.
    \item Existe un algoritmo para transformar una gramática regular $G$ en un autómata con pila que acepte las cadenas del lenguaje generado por $G$ por el criterio de pila vacı́a.
    \item Un autómata con pila determinista no puede tener transiciones nulas.
    \item El conjunto de cadenas generadas por una gramática independiente del contexto en forma normal de Chomsky puede ser reconocido por un autómata finito no determinista con transiciones nulas.
    \item Para que un lenguaje sea aceptado por una autómata con pila determinista por el criterio de pila vacı́a tiene que verificar la propiedad prefijo.
    \item Un autómata finito determinista se puede convertir en un autómata con pila que acepta el mismo lenguaje por el criterio de pila vacı́a.
    \item Un autómata con pila determinista no puede tener transiciones nulas.
    \item Todo lenguaje aceptado por un automata con pila determinista por el criterio de estados finales es tambien aceptado por un automata con pila determinista por el criterio de pila vacı́a.
    \item Si tenemos un autómata con pila en el que $(p,\veps)\in \delta(q,a,C)$, entonces para construir una gramática independiente del contexto que genere el mismo lenguaje que acepta el autómata, debemos de añadir la producción $[p,C,q]\rightarrow a$ (según el procedimiento visto en clase).
    \item Para que un autómata con pila sea determinista es necesario que no tenga transiciones nulas.
    \item El lenguaje $L = \{u\in {\{0,1\}}^{\ast} \mid u = u^{-1}\}$ es independiente del contexto, pero no determinista.
\end{enumerate}

\newpage
\section{Autómatas Finitos}

\begin{comment}
\begin{tikzpicture}
    \node[state, initial] (q1) {$q_1$};
    \node[state, accepting, right of=q1] (q2) {$q_2$};
    \node[state, right of=q2] (q3) {$q_3$};
    \draw (q1) edge[loop above] node{0} (q1)
    (q1) edge[above] node{1} (q2)
    (q2) edge[loop above] node{1} (q2)
    (q2) edge[bend left, above] node{0} (q3)
    (q3) edge[bend left, below] node{0, 1} (q2);
\end{tikzpicture}
\end{comment}

\begin{ejercicio} \label{ej:1.2.1}
    Considera el siguiente Autómata Finito Determinista (AFD) dado por $M = (Q, A, \delta, q_0, F)$, donde:
    \begin{itemize}
        \item $Q = \{q_0, q_1, q_2\}$
        \item $A = \{0, 1\}$
        \item La función de transición viene dada por:
        \begin{align*}
            \delta(q_0, 0) &= q_1, & \delta(q_0, 1) &= q_0 \\
            \delta(q_1, 0) &= q_2, & \delta(q_1, 1) &= q_0 \\
            \delta(q_2, 0) &= q_2, & \delta(q_2, 1) &= q_2
        \end{align*}
        \item $F = \{q_2\}$
    \end{itemize}
    Describe informalmente el lenguaje aceptado.\\

    Su representación gráfica está en la Figura \ref{fig:ej:1.2.1}.
    \begin{figure}
        \centering
        \begin{tikzpicture}
            \node[state, initial] (q0) {$q_0$};
            \node[state, right of=q0] (q1) {$q_1$};
            \node[state, accepting, right of=q1] (q2) {$q_2$};

            \draw   (q0) edge[loop above] node{1} (q0)
                    (q0) edge[above, bend left] node{0} (q1)
                    (q1) edge[below, bend left] node{1} (q0)
                    (q1) edge[above] node{0} (q2)
                    (q2) edge[loop above] node{0, 1} (q2);
        \end{tikzpicture}
        \caption{Autómata Finito Determinista del Ejercicio \ref{ej:1.2.1}}
        \label{fig:ej:1.2.1}
    \end{figure}

    Tenemos que el lenguaje aceptado por el autómata es el conjunto de todas las palabras que contienen
    la cadena $00$ como subcadena. Es decir,
    \begin{align*}
        L = \{u_100u_2 \in \{0, 1\}^* \mid u_1, u_2 \in \{0, 1\}^*\}.
    \end{align*}
\end{ejercicio}

\begin{ejercicio} \label{ej:1.2.2}
    Dado el AFD de la Figura \ref{fig:ej:1.2.2}, describir el lenguaje aceptado por dicho autómata.
    \begin{figure}
        \centering
        \begin{tikzpicture}
            \node[state, initial] (q0) {$q_0$};
            \node[state, right of=q0] (q1) {$q_1$};
            \node[state, accepting, right of=q1] (q2) {$q_2$};

            \draw   (q0) edge[loop above] node{$b$} (q0)
                    (q0) edge[above] node{$a$} (q1)
                    (q1) edge[loop above] node{$b$} (q1)
                    (q1) edge[bend left, above] node{$a$} (q2)
                    (q2) edge[bend left, below] node{$a$} (q1)
                    (q2) edge[loop above] node{$b$} (q2);
        \end{tikzpicture}
        \caption{Autómata Finito Determinista del Ejercicio \ref{ej:1.2.2}}
        \label{fig:ej:1.2.2}
    \end{figure}

    El lenguaje aceptado por el autómata es el conjunto de todas las palabras que contienen
    un número par de $a$'s. Es decir,
    \begin{align*}
        L = \{u \in \{a, b\}^* \mid n_a(u) \text{ es par},~n_a(u)>0\},
    \end{align*}
\end{ejercicio}

\begin{ejercicio} \label{ej:1.2.3}
    Dibujar AFDs que acepten los siguientes lenguajes con alfabeto $\{0, 1\}$:
    \begin{enumerate}
        \item El lenguaje vacío,
        
        \begin{figure}[H]
            \centering
            \begin{tikzpicture}
                \node[state, initial] (q0) {$q_0$};

                \draw   (q0) edge[loop above] node{0, 1} (q0);
            \end{tikzpicture}
        \end{figure}
        \item El lenguaje formado por la palabra vacía, es decir, $\{\varepsilon\}$,
        
        \begin{figure}[H]
            \centering
            \begin{tikzpicture}
                \node[state, initial, accepting] (q0) {$q_0$};
                \node[state, right of=q0] (q1) {$q_1$};

                \draw   (q0) edge[above] node{0, 1} (q1);
                \draw   (q1) edge[loop above] node{0, 1} (q1);
            \end{tikzpicture}
        \end{figure}
        \item El lenguaje formado por la palabra $01$, es decir, $\{01\}$,
        
        \begin{figure}[H]
            \centering
            \begin{tikzpicture}
                \node[state, initial] (q0) {$q_0$};
                \node[state, right of=q0] (q1) {$q_1$};
                \node[state, accepting, right of=q1] (q2) {$q_2$};
                \node[state, below of=q1, error] (E) {$E$};

                \draw   (q0) edge[above] node{0} (q1)
                        (q1) edge[above] node{1} (q2);
                \draw   (q0) edge[below] node{1} (E)
                        (q1) edge[left] node{0} (E)
                        (q2) edge[below right] node{0, 1} (E)
                        (E) edge[loop below] node{0, 1} (E);
            \end{tikzpicture}
        \end{figure}
        \item El lenguaje $\{11, 00\}$,
        
        \begin{figure}[H]
            \centering
            \begin{tikzpicture}
                \node[state, initial] (q0) {$q_0$};
                \node[state, above right of=q0] (q1) {$q_1$};
                \node[state, below right of=q0] (q2) {$q_2$};
                \node[state, accepting, right of=q0] (q3) {$q_3$};
                \node[state, right of=q3, error] (E) {$E$};

                \draw   (q0) edge[above] node{1} (q1)
                        (q0) edge[below] node{0} (q2)
                        (q1) edge[above] node{0} (E)
                        (q1) edge[left] node{1} (q3)
                        (q2) edge[left] node{0} (q3)
                        (q2) edge[below] node{1} (E);
                \draw   (E) edge[loop above] node{0, 1} (E);
                \draw   (q3) edge[above left] node{0, 1} (E);
            \end{tikzpicture}
        \end{figure}
        \item El lenguaje $\{(01)^i \mid i \geq 0\}$,
        
        \begin{figure}[H]
            \centering
            \begin{tikzpicture}
                \node[state, initial, accepting] (q0) {$q_0$};
                \node[state, below right of=q0, error] (E) {$E$};
                \node[state, above right of=E] (q1) {$q_1$};


                \draw   (q0) edge[below] node{0} (q1)
                        (q0) edge[below] node{1} (q2)
                        (q1) edge[left] node{0} (q2)
                        (q1) edge[above, bend right] node{1} (q0)
                        (q2) edge[loop below] node{0, 1} (q2);
            \end{tikzpicture}
        \end{figure}
        \item El lenguaje formado por las cadenas con $0$'s y $1$'s donde el número de unos es divisible por $3$.
        
        \begin{figure}[H]
            \centering
            \begin{tikzpicture}
                \node[state, initial, accepting] (q0) {$q_0$};
                \node[state, right of=q0] (q1) {$q_1$};
                \node[state, right of=q1] (q2) {$q_2$};

                \draw   (q0) edge[loop above] node{0} (q0)
                        (q0) edge[above] node{1} (q1)
                        (q1) edge[loop above] node{0} (q1)
                        (q1) edge[above] node{1} (q2)
                        (q2) edge[loop above] node{0} (q2)
                        (q2) edge[below, bend left] node{1} (q0);
            \end{tikzpicture}
        \end{figure}
    \end{enumerate}
\end{ejercicio}

\begin{ejercicio} \label{ej:1.2.4}
    Obtener a partir de la gramática regular $G = (\{S, B\}, \{1, 0\}, P, S)$, con
    \begin{align*}
        P = \left\{
            \begin{aligned}
                S &\to 110B \\
                B &\to 0B \mid 1B \mid \varepsilon
            \end{aligned}
        \right.
    \end{align*}
    un AFND que reconozca el lenguaje generado por esa gramática.

    El autómata obtenido es el de la Figura \ref{fig:ej:1.2.4}.
    \begin{figure}[H]
        \centering
        \begin{tikzpicture}
            \node[state, initial] (q0) {$q_0$};
            \node[state, right of=q0] (q1) {$q_1$};
            \node[state, right of=q1] (q2) {$q_2$};
            \node[state, accepting, right of=q2] (q3) {$q_3$};

            \draw   (q0) edge[above] node{1} (q1)
                    (q1) edge[above] node{1} (q2)
                    (q2) edge[above] node{0} (q3)
                    (q3) edge[loop above] node{0, 1} (q3);
        \end{tikzpicture}
        \caption{Autómata Finito No Determinista del Ejercicio \ref{ej:1.2.4}}
        \label{fig:ej:1.2.4}
    \end{figure}
\end{ejercicio}

\begin{ejercicio} \label{ej:1.2.5}
    Dada la gramática regular $G = (\{S\}, \{1, 0\}, P, S)$, con
    \begin{align*}
        P = \{S &\to S10, S \to 0\},
    \end{align*}
    obtener un AFD que reconozca el lenguaje generado por esa gramática.\\

    El lenguaje es:
    \begin{equation*}
        L = \{0(10)^n \mid n\in \bb{N}\cup \{0\}\}.
    \end{equation*}

    El autómata obtenido es el de la Figura \ref{fig:ej:1.2.5}.
    \begin{figure}[H]
        \centering
        \begin{tikzpicture}
            \node[state, initial] (q0) {$q_0$};
            \node[state, below right of=q0, error] (E) {$E$};
            \node[state, accepting, above right of=E] (q1) {$q_1$};

            \draw   (q0) edge[above, bend left] node{0} (q1)
                    (q0) edge[below] node{1} (E)
                    (q1) edge[above, bend left] node{1} (q0)
                    (q1) edge[below] node{0} (E)
                    (E) edge[loop below] node{0, 1} (E);
        \end{tikzpicture}
        \caption{Autómata Finito Determinista del Ejercicio \ref{ej:1.2.5}}
        \label{fig:ej:1.2.5}
    \end{figure}
\end{ejercicio}

\begin{ejercicio} \label{ej:1.2.6}
    Construir un AFND o AFD (dependiendo del caso) que acepte las cadenas $u \in \{0, 1\}^*$ que:
    \begin{enumerate}
        \item AFND. Contengan la subcadena $010$. \label{ej:1.2.6.1}
        
        El autómata obtenido es el de la Figura \ref{fig:ej:1.2.6.1}.
        \begin{figure}[H]
            \centering
            \begin{tikzpicture}
                \node[state, initial] (q0) {$q_0$};
                \node[state, right of=q0] (q1) {$q_1$};
                \node[state, right of=q1] (q2) {$q_2$};
                \node[state, accepting, right of=q2] (q3) {$q_3$};

                \draw   (q0) edge[above] node{0} (q1)
                        (q1) edge[above] node{1} (q2)
                        (q2) edge[above] node{0} (q3)
                        (q3) edge[loop above] node{0, 1} (q3)
                        (q0) edge[loop above] node{0, 1} (q0);
            \end{tikzpicture}
            \caption{Autómata Finito No Determinista del Ejercicio \ref{ej:1.2.6} apartado \ref{ej:1.2.6.1}.}
            \label{fig:ej:1.2.6.1}
        \end{figure}
        \item AFND. Contengan la subcadena $110$. \label{ej:1.2.6.2}
        
        El autómata obtenido es el de la Figura \ref{fig:ej:1.2.6.2}.
        \begin{figure}[H]
            \centering
            \begin{tikzpicture}
                \node[state, initial] (q0) {$q_0$};
                \node[state, right of=q0] (q1) {$q_1$};
                \node[state, right of=q1] (q2) {$q_2$};
                \node[state, accepting, right of=q2] (q3) {$q_3$};

                \draw   (q0) edge[above] node{1} (q1)
                        (q1) edge[above] node{1} (q2)
                        (q2) edge[above] node{0} (q3)
                        (q3) edge[loop above] node{0, 1} (q3)
                        (q0) edge[loop above] node{0, 1} (q0);
            \end{tikzpicture}
            \caption{Autómata Finito No Determinista del Ejercicio \ref{ej:1.2.6} apartado \ref{ej:1.2.6.2}.}
            \label{fig:ej:1.2.6.2}
        \end{figure}
        \item AFD. Contengan simultáneamente las subcadenas $010$ y $110$. \label{ej:1.2.6.3}
        
        El estado $q_0$ representa que no se ha empezado ninguna de las subcadenas, y el estado $q_F$ representa que se han encontrado ambas cadenas.
        Hay dos opciones:
        \begin{description}
            \item[Opción 1] Primero se lee $010$ y luego $110$. Son los siguientes estados:
            \begin{itemize}
                \item $q_0$: Estado inicial, no ha empezado la subcadena $010$.
                \item $q_1$: Se ha leído el $0$ de la subcadena $010$.
                \item $q_2$: Se ha leído la subcadena $01$ de la subcadena $010$.
                \item $q_3$: Se ha leído la subcadena $010$. No ha empezado la subcadena $110$.
                \item $q_4$: Se ha leído el $1$ de la subcadena $110$.
                \item $q_5$: Se ha leído la subcadena $11$ de la subcadena $110$.
                \item $q_F$: Se ha leído la subcadena $110$. Se han leído ambas subcadenas.
            \end{itemize}

            \item[Opción 2] Primero se lee $110$ y luego $010$. Son los siguientes estados:
            \begin{itemize}
                \item $q_0$: Estado inicial, no ha empezado la subcadena $110$.
                \item $q_1'$: Se ha leído el $1$ de la subcadena $110$.
                \item $q_2'$: Se ha leído la subcadena $11$ de la subcadena $110$.
                \item $q_3'$: Se ha leído la subcadena $110$. Se ha leído el $0$ de la subcadena $010$. Notemos que en este caso podemos agruparlo, puesto que el último carácter de la subcadena $110$ es el mismo que el primero de la subcadena $010$.
                \item $q_4'$: Se ha leído la subcadena $01$ de la subcadena $010$.
                \item $q_F$: Se ha leído la subcadena $010$. Se han leído ambas subcadenas.
            \end{itemize}
        \end{description}

        El autómata obtenido es el de la Figura \ref{fig:ej:1.2.6.3}.
        \begin{figure}
            \centering
            \begin{tikzpicture}[node distance=2.2cm]
                \node[state, initial] (q0) {$q_0$};
                \node[state, above right of=q0] (q1) {$q_1$};
                \node[state, right of=q1] (q2) {$q_2$};
                \node[state, right of=q2] (q3) {$q_3$};
                \node[state, right of=q3] (q4) {$q_4$};
                \node[state, right of=q4] (q5) {$q_5$};
                \node[state, accepting, below right of=q5] (qF) {$q_F$};

                \node[state, below right of=q0] (q1p) {$q_1'$};
                \node[state, right of=q1p] (q2p) {$q_2'$};
                \node[state, right of=q2p] (q3p) {$q_3'$};
                \node[state, right of=q3p, xshift=2.5em] (q4p) {$q_4'$};
                
                % Conexiones directas por arriba
                \draw   (q0) edge[above] node{0} (q1)
                        (q1) edge[above] node{1} (q2)
                        (q2) edge[above] node{0} (q3)
                        (q3) edge[above] node{1} (q4)
                        (q4) edge[above] node{1} (q5)
                        (q5) edge[above] node{0} (qF)
                        (qF) edge[loop above] node{0, 1} (qF);
                
                % Conexiones directas por abajo
                \draw   (q0) edge[below] node{1} (q1p)
                        (q1p) edge[below] node{1} (q2p)
                        (q2p) edge[below] node{0} (q3p)
                        (q3p) edge[below] node{1} (q4p)
                        (q4p) edge[below] node{0} (qF);

                % Completamos los de arriba
                \draw   (q1) edge[loop above] node{0} (q1)
                        (q2) edge[left] node{1} (q2p)
                        (q3) edge[loop above] node{0} (q3)
                        (q4) edge[bend left, below] node{0} (q3)
                        (q5) edge[loop above] node{1} (q5);

                % Completamos los de abajo
                \draw   (q1p) edge[left] node{0} (q1)
                        (q2p) edge[loop below] node{1} (q2p)
                        (q3p) edge[loop below] node{0} (q3p)
                        (q4p) edge[bend right, above] node{1} (q2p);
                
            \end{tikzpicture}
            \caption{Autómata Finito Determinista del Ejercicio \ref{ej:1.2.6} apartado \ref{ej:1.2.6.3}.}
            \label{fig:ej:1.2.6.3}
        \end{figure}
    \end{enumerate}
\end{ejercicio}

\begin{ejercicio} \label{ej:1.2.7}
    Construir un AFD que acepte el lenguaje generado por la siguiente gramática:
    \begin{align*}
        S &\to AB, & A &\to aA, & A &\to c, & B &\to bBb, & B &\to b.
    \end{align*}

    El lenguaje generado por la gramática es:
    \begin{align*}
        L = \{a^ncb^{2m+1} \mid n, m \in \bb{N}\cup \{0\}\}.
    \end{align*}

    El autómata obtenido es el de la Figura \ref{fig:ej:1.2.7}.
    \begin{figure}
        \centering
        \begin{tikzpicture}
            \node[state, initial] (q0) {$q_0$};
            \node[state, right of=q0] (q1) {$q_1$};
            \node[state, right of=q1, accepting] (q2) {$q_2$};
            \node[state, below of=q1, error] (E) {$E$};

            \draw   (q0) edge[loop above] node{$a$} (q0)
                    (q0) edge[above] node{$c$} (q1)
                    (q0) edge[above] node{$b$} (E)
                    (q1) edge[left] node{$a,c$} (E)
                    (q1) edge[above, bend right] node{$b$} (q2)
                    (q2) edge[above, bend right] node{$b$} (q1)
                    (q2) edge[above] node{$a,c$} (E)
                    (E) edge[loop below] node{$a,b,c$} (E);
        \end{tikzpicture}
        \caption{Autómata Finito Determinista del Ejercicio \ref{ej:1.2.7}}
        \label{fig:ej:1.2.7}
    \end{figure}

\end{ejercicio}

\begin{ejercicio} \label{ej:1.2.8}
    Construir un AFD que acepte el lenguaje $L \subseteq \{a, b, c\}^*$ de todas las palabras con un número impar de ocurrencias de la subcadena $abc$.

    El autómata tiene los siguientes estados:
    \begin{itemize}
        \item $q_0$: Llevo un número par de ocurrencias de $abc$, y no he empezado la siguiente.
        \item $q_1$: Acabo de empezar una ocurrencia impar de $abc$, llevo solo una $a$.
        \item $q_2$: Estoy en una ocurrencia impar de $abc$, llevo $ab$.
        \item $q_3$: Llevo un número impar de ocurrencias de $abc$, y no he empezado la siguiente.
        \item $q_4$: Acabo de empezar una ocurrencia par de $abc$, llevo solo una $a$.
        \item $q_5$: Estoy en una ocurrencia par de $abc$, llevo $ab$.
    \end{itemize}

    El autómata obtenido es el de la Figura \ref{fig:ej:1.2.8}.
    \begin{figure}
        \centering
        \begin{tikzpicture}
            \node[state, initial] (q0) {$q_0$};
            \node[state, right of=q0] (q1) {$q_1$};
            \node[state, right of=q1] (q2) {$q_2$};
            \node[state, below of=q2, accepting] (q3) {$q_3$};
            \node[state, left of=q3, accepting] (q4) {$q_4$};
            \node[state, left of=q4, accepting] (q5) {$q_5$};

            % Conexiones básicas
            \draw   (q0) edge[above] node{$a$} (q1)
                    (q1) edge[above] node{$b$} (q2)
                    (q2) edge[right] node{$c$} (q3)
                    (q3) edge[below] node{$a$} (q4)
                    (q4) edge[below] node{$b$} (q5)
                    (q5) edge[left] node{$c$} (q0);

            % Completamos
            \draw   (q0) edge[loop above] node{$b,c$} (q0)
                    (q1) edge[loop above] node{$a$} (q1)
                    (q1) edge[above, bend right] node{$c$} (q0)
                    (q2) edge[above, bend right] node{$a$} (q1)
                    (q2) edge[below, bend left] node{$b$} (q0)
                    (q3) edge[loop below] node{$b,c$} (q3)
                    (q4) edge[bend right, below] node{$c$} (q3)
                    (q4) edge[loop below] node{$a$} (q4)
                    (q5) edge[bend right, below] node{$a$} (q4)
                    (q5) edge[bend left, above] node{$b$} (q3);
        \end{tikzpicture}
        \caption{Autómata Finito Determinista del Ejercicio \ref{ej:1.2.8}}
        \label{fig:ej:1.2.8}
    \end{figure}
\end{ejercicio}

\begin{ejercicio} \label{ej:1.2.9}
    Sea $L$ el lenguaje de todas las palabras sobre el alfabeto $\{0, 1\}$ que no contienen dos $1$s que estén separados por un número impar de símbolos. Describir un AFD que acepte este lenguaje.

    % // TODO: No se ha hecho

    \begin{comment}
    Veamos qué estados hay:
    \begin{itemize}
        \item $q_0$: No se ha introducido ningún $1$.
        \item $q_1$: Se ha introducido un $1$. El número de símbolos después del $1$ es par, por lo que dejo meter otro $0$ o un $1$.
        \item $q_2$: El número de símbolos después del $1$ es impar, por lo que solo puedo meter un $0$.
    \end{itemize}

    El autómata obtenido es el de la Figura \ref{fig:ej:1.2.9}.
    \begin{figure}
        \centering
        \begin{tikzpicture}
            \node[state, initial] (q0) {$q_0$};
            \node[state, above right of=q0] (q1) {$q_1$};
            \node[state, below right of=q0] (q2) {$q_2$};

            \draw   (q0) edge[above] node{0} (q1)
                    (q0) edge[below] node{1} (q2);
        \end{tikzpicture}
        \caption{Autómata Finito Determinista del Ejercicio \ref{ej:1.2.9}}
        \label{fig:ej:1.2.9}
    \end{figure}
    \end{comment}
\end{ejercicio}

\begin{ejercicio} \label{ej:1.2.10}
    Dada la expresión regular $(a + \varepsilon)b^*$, encontrar un AFND asociado y, a partir de este, calcular un AFD que acepte el lenguaje.

    El AFND con transiciones nulas obtenido (siguiendo el algoritmo) es el de la Figura \ref{fig:ej:1.2.10}.
    \begin{figure}
        \centering
        \begin{tikzpicture}
            \node[state, initial] (q0) {$q_0$};
            \node[state, above right of=q0] (q1) {$q_1$};
            \node[state, below right of=q0] (q2) {$q_2$};
            \node[state, right of=q1] (q3) {$q_3$};
            \node[state, below right of=q3, accepting] (q4) {$q_4$};
            \node[state, right of=q4] (q5) {$q_5$};
            \node[state, right of=q5, accepting] (q6) {$q_6$};

            \draw   (q0) edge[above] node{$\varepsilon$} (q1)
                    (q0) edge[below] node{$\varepsilon$} (q2)
                    (q1) edge[above] node{$a$} (q3)
                    (q2) edge[above, bend right] node{$\veps$} (q4)
                    (q3) edge[above] node{$\varepsilon$} (q4)
                    (q4) edge[above] node{$\varepsilon$} (q5)
                    (q5) edge[above] node{$b$} (q6)
                    (q6) edge[below, bend left] node{$\veps$} (q5);
        \end{tikzpicture}
        \caption{Autómata Finito No Determinista algorítmico del Ejercicio \ref{ej:1.2.10}.}
        \label{fig:ej:1.2.10}
    \end{figure}

    Podemos simplificar este autómata para que así la transición al AFD sea más sencilla. El autómata simplificado es el de la Figura \ref{fig:ej:1.2.10.simplified}.
    \begin{figure}
        \centering
        \begin{tikzpicture}
            \node[state, initial] (q0) {$q_0$};
            \node[state, right of=q0, accepting] (q6) {$q_6$};

            \draw   (q0) edge[above] node{$a, \veps$} (q6)
                    (q6) edge[loop above] node{$b$} (q6);
        \end{tikzpicture}
        \caption{Autómata Finito No Determinista simplificado del Ejercicio \ref{ej:1.2.10}.}
        \label{fig:ej:1.2.10.simplified}
    \end{figure}

    A partir de este autómata simplificado, obtenemos el AFD de la Figura \ref{fig:ej:1.2.10.afd}.
    \begin{figure}
        \centering
        \begin{tikzpicture}
            \node[state, initial, accepting] (q0q6) {$\{q_0, q_6\}$};
            \node[state, right of=q0q6, accepting] (q6) {$q_6$};
            \node[state, right of=q6, error] (E) {$E$};

            \draw   (q0q6) edge[above] node{$a,b$} (q6)
                    (q6) edge[loop above] node{$b$} (q6)
                    (q6) edge[above] node{$a$} (E)
                    (E) edge[loop above] node{$a,b$} (E);
        \end{tikzpicture}
        \caption{Autómata Finito Determinista del Ejercicio \ref{ej:1.2.10}.}
        \label{fig:ej:1.2.10.afd}
    \end{figure}

    % // TODO: Lo ha hecho Irina, lo ha copiado Joaquín. Comprobar
\end{ejercicio}

\begin{ejercicio}
    Obtener una expresión regular para el lenguaje complementario al aceptado por la gramática
    \begin{align*}
        S &\to abA \mid B \mid baB \mid \varepsilon, & A &\to bS \mid b, & B &\to aS.
    \end{align*}
    \begin{observacion}
        Construir un AFD asociado.
    \end{observacion}
    % // TODO: No se ha hecho
\end{ejercicio}

\begin{ejercicio} \label{ej:1.2.12}
    Dar expresiones regulares para los lenguajes sobre el alfabeto $\{a, b\}$ dados por las siguientes condiciones:
    \begin{enumerate}
        \item Palabras que no contienen la subcadena $a$,
        \begin{equation*}
            b^*
        \end{equation*}
        \item Palabras que no contienen la subcadena $ab$.
        \begin{equation*}
            b^*a^*
        \end{equation*}
        \item Palabras que no contienen la subcadena $aba$.
        
        % // TODO: No se ha hecho
    \end{enumerate}
\end{ejercicio}

\begin{ejercicio}
    Determinar si el lenguaje generado por la siguiente gramática es regular:
    \begin{align*}
        S &\to AabB, & A &\to aA \mid bA \mid \varepsilon, & B &\to Bab \mid Bb \mid ab \mid b.
    \end{align*}
    En caso de que lo sea, encontrar una expresión regular asociada.\\

    Es directo ver que el lenguaje generado por la gramática tiene como expresión regular asociada:
    \begin{align*}
        (a+b)^*ab(a+b)^*.
    \end{align*}

    Por tanto, el lenguaje es regular.
\end{ejercicio}

\begin{ejercicio}
    Sobre el alfabeto $A = \{0, 1\}$ realizar las siguientes tareas:
    \begin{enumerate}
        \item Describir un autómata finito determinista que acepte todas las palabras que contengan a $011$ o a $010$ (o las dos) como subcadenas.
        \item Describir un autómata finito determinista que acepte todas las palabras que empiecen o terminen (o ambas cosas) por $01$.
        \item Dar una expresión regular para el conjunto de las palabras en las que hay dos ceros separados por un número de símbolos que es múltiplo de $4$ (los símbolos que separan los ceros pueden ser ceros y puede haber otros símbolos delante o detrás de estos dos ceros).
        \item Dar una expresión regular para las palabras en las que el número de ceros es divisible por $4$.
    \end{enumerate}
\end{ejercicio}

\begin{ejercicio}
    Construye una gramática regular que genere el siguiente lenguaje:
    \begin{align*}
        L_1 = \{u \in \{0, 1\}^* \mid \text{el número de $1$'s y de $0$'s es impar}\}.
    \end{align*}
\end{ejercicio}

\begin{ejercicio}
    Encuentra una expresión regular que represente el siguiente lenguaje:
    \begin{align*}
        L_2 = \{0^n1^m \mid n \geq 1, m \geq 0, n \text{ múltiplo de } 3 \text{ y } m \text{ es par}\}.
    \end{align*}
\end{ejercicio}

\begin{ejercicio}
    Diseña un autómata finito determinista que reconozca el siguiente lenguaje:
    \begin{align*}
        L_3 = \{u \in \{0, 1\}^* \mid \text{el número de $1$'s no es múltiplo de } 3 \text{ y el número de $0$'s es par}\}.
    \end{align*}
\end{ejercicio}

\begin{ejercicio} \label{ej:1.2.18}
    Dar una expresión regular para el lenguaje aceptado por el autómata de la Figura \ref{fig:ej:1.2.18}.
    \begin{figure}
        \centering
        \begin{tikzpicture}
            \node[state, initial, accepting] (q0) {$q_0$};
            \node[state, accepting, below right of=q0] (q2) {$q_2$};
            \node[state, above right of=q2] (q1) {$q_1$};

            \draw   (q1) edge[loop above] node{$a$} (q1)
                    (q0) edge[above] node{$a,b$} (q1)
                    (q1) edge[below, bend left] node{$b$} (q2)
                    (q2) edge[below, bend left] node{$b$} (q1)
                    (q2) edge[above] node{$a$} (q0);
        \end{tikzpicture}
        \caption{Autómata Finito Determinista del Ejercicio \ref{ej:1.2.18}}
        \label{fig:ej:1.2.18}
    \end{figure}
\end{ejercicio}


\begin{ejercicio} \label{ej:1.2.19}
    Dado el lenguaje
    \begin{align*}
        L = \{u110 \mid u \in \{1, 0\}^*\},
    \end{align*}
    encontrar la expresión regular, la gramática lineal por la derecha, la gramática lineal por la izquierda y el AFD asociado.
\end{ejercicio}

\begin{ejercicio}
    Dado un AFD, determinar el proceso que habría que seguir para construir una Gramática lineal por la izquierda capaz de generar el Lenguaje aceptado por dicho autómata.
\end{ejercicio}

\begin{ejercicio}
    Construir un autómata finito determinista que acepte el lenguaje de todas las palabras sobre el alfabeto $\{0, 1\}$ que no contengan la subcadena $001$.
    Construir una gramática regular por la izquierda a partir de dicho autómata.
\end{ejercicio}

\begin{ejercicio}
    Sea $B_n = \{a^k \mid k \text{ es múltiplo de } n\}$. Demostrar que $B_n$ es regular para todo $n$.
\end{ejercicio}

\begin{ejercicio}
    Decimos que $u$ es un prefijo de $v$ si existe $w$ tal que $uw = v$. Decimos que $u$ es un prefijo propio de $v$ si además $u \neq v$ y $u \neq \varepsilon$. Demostrar que si $L$ es regular, también lo son los lenguajes
    \begin{enumerate}
        \item $NOPREFIJO(L) = \{u \in L \mid \text{ningún prefijo propio de } u \text{ pertenece a } L\}$,
        \item $NOEXTENSION(L) = \{u \in L \mid u \text{ no es un prefijo propio de ninguna palabra de } L\}$.
    \end{enumerate}
\end{ejercicio}

\begin{ejercicio}
    Si $L \subseteq A^*$, define la relación $\equiv$ en $A^*$ como sigue: si $u, v \in A^*$, entonces $u \equiv v$ si y solo si para toda $z \in A^*$, tenemos que $(uz \in L \Leftrightarrow vz \in L)$.
    \begin{enumerate}
        \item Demostrar que $\equiv$ es una relación de equivalencia.
        \item Calcular las clases de equivalencia de $L = \{a^ib^i \mid i \geq 0\}$.
        \item Calcular las clases de equivalencia de $L = \{a^ib^j \mid i, j \geq 0\}$.
        \item Demostrar que $L$ es aceptado por un autómata finito determinístico si y solo si el número de clases de equivalencia es finito.
        \item ¿Qué relación existe entre el número de clases de equivalencia y el autómata finito minimal que acepta $L$?
    \end{enumerate}
\end{ejercicio}

\begin{ejercicio}
    Dada una palabra $u = a_1 \cdots a_n \in A^*$, se llama $Per(u)$ al conjunto
    \begin{align*}
        \{a_{\sigma(1)}, \ldots, a_{\sigma(n)} \mid \sigma \text{ es una permutación de } \{1, \ldots, n\}\}.
    \end{align*}
    Dado un lenguaje $L$, se llama $Per(L) = \bigcup_{u \in L} Per(u)$.
    Dar expresiones regulares y autómatas minimales para $Per(L)$ en los siguientes casos:
    \begin{enumerate}
        \item $L = (00 + 1)^*$,
        \item $L = (0 + 1)^*0$,
        \item $L = (01)^*$.
    \end{enumerate}
    ¿Es posible que, siendo $L$ regular, $Per(L)$ no lo sea?
\end{ejercicio}
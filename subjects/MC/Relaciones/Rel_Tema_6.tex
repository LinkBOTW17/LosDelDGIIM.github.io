\newpage
\section{Propiedades de Lenguajes Indep. del Contexto}

\begin{ejercicio}\label{ej:1.6.1}
    Proporcione ejemplos de los siguientes lenguajes:
    \begin{enumerate}
        \item Un lenguaje que no es independiente del contexto.
        
        Sea el lenguaje siguiente:
        \begin{equation*}
            L = \{a^n b^n c^n \mid n\geq 0\}
        \end{equation*}

        Veamos que no es independiente del contexto mediante el recíproco del Lema de Bombeo. Para cada $n\in \bb{N}$, consideramos la palabra $z=a^n b^n c^n\in L$, con $|z|\geq n$. Consideramos ahora cualquier descomposición de $z$ en cinco partes $z=uvwxy$, con $u,v,w,x,y\in {\{a,b,c\}}^{\ast}$, de forma que $|vwx|\leq n$ y $|vx|\geq 1$.

        Como $|vwx|\leq n$, entonces no es posible que contenga los tres símbolos $a,b,c$;
        y como $|vx|\geq 1$, $vx$ contiene al menos uno de los tres símbolos, pero no los tres.
        Por tanto, al bombear $vx$ añadiremos alguno de los tres símbolos pero no los tres, rompiendo el equilibrio. En concreto, tenemos que:
        \begin{equation*}
            uv^2wx^2y \notin L
        \end{equation*}

        Por tanto, por el recíproco del Lema de Bombeo, $L$ no es independiente del contexto.
        \item Un lenguaje independiente del contexto pero no determinista.
        % // TODO: Añadir ejemplo indep del contexto pero no determinista
        \item Un lenguaje que es independiente del contexto determinista, pero que no es aceptado por un autómata con pila determinista que tiene que vaciar su pila.
        
        Es necesario que el lenguaje dado no cumpla la propiedad prefijo. Un ejemplo válido es:
        \begin{equation*}
            L = \{ucv\mid u,v\in {\{a,b\}}^{+} \text{\ y n$^\circ$ de subcadenas ``}ab \text{'' en\ } u \text{\ es igual al n$^\circ$ subcadenas ``} ba \text{'' en\ }v\}
        \end{equation*}
        tal y como se vio en el Ejercicio~\ref{ej:1.5.21}.
        \item Un lenguaje que es aceptado por un autómata con pila determinista que tiene que vaciar su pila, pero que no es un lenguaje regular.
        
        Un ejemplo es el lenguaje:
        \begin{equation*}
            L=\{a^n b^n \mid n\geq 1\}
        \end{equation*}

        Veamos en primer lugar que no es regular. Para ello, aplicamos el Lema de Bombeo.
        Para cada $n\in \bb{N}$, consideramos la palabra $z=a^n b^n\in L$, con $|z|=2n\geq n$. Consideramos ahora cualquier descomposición de $z$ en tres partes $z=uvw$, con $u,v,w\in {\{a,b\}}^{\ast}$, de forma que $|v|\geq 1$ y $|uv|\leq n$. De esta forma:
        \begin{equation*}
            u=a^k \quad v=a^l \quad w=A^{n-k-l}b^n\qquad \text{con } k+l\leq n, l\geq 1
        \end{equation*}

        Al bombear $v$ con $i=2$, tenemos que:
        \begin{equation*}
            uv^2w = a^{k+2l}a^{n-k-l}b^n = a^{n+l}b^n \notin L
        \end{equation*}
        ya que $n+l\neq n$ por ser $l\geq 1$. Por tanto, por el Lema de Bombeo, $L$ no es regular.\\

        No obstante, sí es aceptado por el siguiente autómata con pila determinista por el criterio de la pila vacía:
        \begin{equation*}
            M = (\{q_0,q_1\},\{a,b\},\{Z_0,X\},\delta,q_0,Z_0,\emptyset)
        \end{equation*}
        con la función de transición $\delta$ dada por:
        \begin{align*}
            &\delta(q_0,a,Z_0) = \{(q_0,XZ_0)\} \\
            &\delta(q_0,a,X) = \{(q_0,XX)\} \\
            &\delta(q_0,b,X) = \{(q_1,\veps)\} \\
            &\delta(q_1,b,X) = \{(q_1,\veps)\} \\
            &\delta(q_1,\veps,Z_0) = \{(q_1,\veps)\}
        \end{align*}
    \end{enumerate}
\end{ejercicio}

\begin{ejercicio}\label{ej:1.6.2}
    Encontrar cuando sea posible, un autómata con pila que acepte el lenguaje $L$, donde:
    \begin{enumerate}
        \item \label{ej:1.6.2-1}
        $L = \{ww^{-1}\mid w\in {\{a,b\}}^{\ast}\}$.
        
        Sea el autómata con pila $M = (\{q_0,q_1,q_f\},\{a,b\},\{Z_0,A,B\},\delta,q_0,Z_0,\{q_f\})$ que acepta $L$ por ambos criterios (tanto por pila vacía como por estados finales), con la función de transición $\delta$ dada por:
        \begin{align*}
            &\delta(q_0,a,Z_0) = \{(q_0,AZ_0)\} \\
            &\delta(q_0,b,Z_0) = \{(q_0,BZ_0)\} \\
            &\delta(q_0,a,A) = \{(q_0,AA)\} \\
            &\delta(q_0,b,B) = \{(q_0,BB)\} \\
            &\delta(q_0,a,B) = \{(q_0,AB)\} \\
            &\delta(q_0,b,A) = \{(q_0,BA)\} \\
            &\red{\delta(q_0,\veps,i) = \{(q_1,i)\}} \qquad \text{Para } i\in \{Z_0,A,B\} \\
            &\delta(q_1,a,A) = \{(q_1,\veps)\} \\
            &\delta(q_1,b,B) = \{(q_1,\veps)\} \\
            &\delta(q_1,\veps,Z_0) = \{(q_f,\veps)\}
        \end{align*}
        \item $L = \{ww\mid w \in {\{a,b\}}^{\ast}\}$.
        
        Veamos que no es posible demostrando que no es independiente del contexto por el recíproco del Lema de Bombeo. Para cada $n\in \bb{N}$, consideramos la palabra $z=a^nb^na^nb^n\in L$, con $|z|\geq n$. Consideramos ahora cualquier descomposición de $z$ en cinco partes $z=uvwxy$, con $u,v,w,x,y\in {\{a,b\}}^{\ast}$, de forma que $|vwx|\leq n$ y $|vx|\geq 1$.

        Como $|vwx|\leq n$, hay varias posibilidades:
        \begin{enumerate}
            \item $vwx$ está contenido entero en la primera mitad de la palabra (recíprocamente con la segunda). Al bombear $v$ y $x$ con $i=2$, se rompe el equilibrio entre las dos mitades, teniendo así que $uv^2wx^2y\notin L$.
            \item $vwx$ corta a ambas mitades, es decir, contiene la subcadena $ba$. Como $|vwx|\leq n$, tenemos que es de la forma $b^k a^l$ con $k+l\leq n$. Como $|vx|\geq 1$, al menos uno de los dos no es nulo. Supongamos que $v\neq \veps$ (en el caso contrario se haría el mismo razonamiento con $x$). Entonces, $v$ contiene al menos una $b$, por lo que al bombear $v$ y $x$ con $i=2$, se rompe el equilibrio entre las dos mitades, ya que el número de $b$'s en la primera mitad es mayor que en la segunda. Por tanto, $uv^2wx^2y\notin L$.
        \end{enumerate}

        En cualquier caso, con $i=2$ tenemos que $uv^2wx^2y\notin L$, por lo que $L$ no es independiente del contexto.

        \item $L = \{a^l b^m c^n \mid l + m = n\}$.
        
        Sea el autómata con pila $M = (\{q_0,q_1,q_2,q_f\},\{a,b,c\},\{Z_0,X\},\delta,q_0,Z_0,\{q_f\})$ que acepta $L$ por ambos criterios (tanto por pila vacía como por estados finales), con la función de transición $\delta$ dada por:
        \begin{align*}
            &\delta(q_0,a,Z_0) = \{(q_0,XZ_0)\} \\
            &\delta(q_0,a,X) = \{(q_0,XX)\} \\
            &\red{\delta(q_0,\veps,i) = \{(q_1,i)\}} \qquad \text{Para } i\in \{Z_0,X\} \\
            &\delta(q_1,b,Z_0) = \{(q_1,XZ_0)\} \\
            &\delta(q_1,b,X) = \{(q_1,XX)\} \\
            &\red{\delta(q_1,\veps,i) = \{(q_2,i)\}} \qquad \text{Para } i\in \{Z_0,X\} \\
            &\delta(q_2,c,X) = \{(q_2,\veps)\} \\
            &\delta(q_2,\veps,Z_0) = \{(q_f,\veps)\}
        \end{align*}
        \item $L = \{a^m b^n c^m \mid n\leq m\}$.
        
        Demostraremos que no es independiente del contexto por el recíproco del Lema de Bombeo. Para cada $n\in \bb{N}$, consideramos la palabra $z=a^{n}b^nc^{n}\in L$, con $|z|\geq n$. Consideramos ahora cualquier descomposición de $z$ en cinco partes $z=uvwxy$, con $u,v,w,x,y\in {\{a,b,c\}}^{\ast}$, de forma que $|vwx|\leq n$ y $|vx|\geq 1$.

        Como $|vwx|\leq n$, hay varias posibilidades:
        \begin{enumerate}
            \item Si $vwx$ tan solo contiene $a$'s, entonces al bombear $v$ y $x$ con $i=2$, se rompe el equilibrio entre el número de $a$'s y $c$'s, teniendo así que $uv^2wx^2y\notin L$.
            \item Si $vwx$ tan solo contiene $c$'s, entonces al bombear $v$ y $x$ con $i=2$, se rompe el equilibrio entre el número de $a$'s y $c$'s, teniendo así que $uv^2wx^2y\notin L$.
            \item Si $vwx$ tan solo contiene $b$'s, entonces al bombear $v$ y $x$ con $i=2$, se tiene que hay más $b$'s que $a$'s ($n>m$), teniendo así que $uv^2wx^2y\notin L$.
            \item Si $vwx$ contiene $a$'s y $b$'s, entonces al bombear $v$ y $x$ con $i=2$ se da alguno de los siguientes casos:
            \begin{itemize}
                \item Si $v\neq \veps$, entonces $v$ contiene al menos una $a$, por lo que al bombear $v$ y $x$ con $i=2$, se rompe el equilibrio entre el número de $a$'s y $c$'s, teniendo así que $uv^2wx^2y\notin L$.
                \item Si $x\neq \veps$, entonces $x$ contiene al menos una $b$, por lo que al bombear $v$ y $x$ con $i=2$, se tiene que hay más $b$'s que $a$'s ($n>m$), teniendo así que $uv^2wx^2y\notin L$.
            \end{itemize}
            \item Si $vwx$ contiene $b$'s y $c$'s, entonces al bombear $v$ y $x$ con $i=2$ se da alguno de los siguientes casos:
            \begin{itemize}
                \item Si $v\neq \veps$, entonces $v$ contiene al menos una $b$, por lo que al bombear $v$ y $x$ con $i=2$, se tiene que hay más $b$'s que $a$'s ($n>m$), teniendo así que $uv^2wx^2y\notin L$.
                \item Si $x\neq \veps$, entonces $x$ contiene al menos una $c$, por lo que al bombear $v$ y $x$ con $i=2$, se rompe el equilibrio entre el número de $a$'s y $c$'s, teniendo así que $uv^2wx^2y\notin L$.
            \end{itemize}
            \item Como $|vwx|\leq n$, no se puede dar que a la vez contenga $a$'s y $c$'s, por lo que no se da este caso.
        \end{enumerate}

        En cualquier caso, con $i=2$ tenemos que $uv^2wx^2y\notin L$, por lo que $L$ no es independiente del contexto.
    \end{enumerate}
\end{ejercicio}

\begin{ejercicio}\label{ej:1.6.3}
    Demostrar que los siguientes lenguajes no son libres de contexto:
    \begin{enumerate}
        \item $L_1 = \{a^p \mid p \text{\ es primo}\}$.
        
        Demostraremos que no es independiente del contexto por el recíproco del Lema de Bombeo. Para cada $n\in \bb{N}$, sea $p_n$ el $n$-ésimo número primo, que sabemos que existe y, además, $p_n\geq n$. Consideramos la palabra $z=a^{p_n}\in L_1$, con $|z|=p_n\geq n$. Consideramos ahora cualquier descomposición de $z$ en cinco partes $z=uvwxy$, con $u,v,w,x,y\in {\{a\}}^{\ast}$, de forma que $|vwx|\leq n$ y $|vx|\geq 1$.

        Tomemos ahora $i=p_n+1$. Entonces, al bombear $v$ y $x$, tenemos que:
        \begin{equation*}
            |uv^iwx^iy| = |uvwxy| + (i-1)|vx| = p_n + p_n\cdot |vx| = p_n(1+|vx|) > p_n
        \end{equation*}

        Por tanto, tenemos que $p_n$ es un divisor no propio de $|uv^iwx^iy|$, por lo que $uv^iwx^iy\notin L_1$.
        Por tanto, por el recíproco del Lema de Bombeo, $L_1$ no es independiente del contexto.
        \item \label{ej:1.6.3-2}
        $L_2 = \{a^{n^2}\mid n\geq 1\}$.
        
        Demostraremos que no es independiente del contexto por el recíproco del Lema de Bombeo. Para cada $n\in \bb{N}$, consideramos la palabra $z=a^{n^2}\in L_2$, con $|z|=n^2\geq n$. Consideramos ahora cualquier descomposición de $z$ en cinco partes $z=uvwxy$, con $u,v,w,x,y\in {\{a\}}^{\ast}$, de forma que $|vwx|\leq n$ y $|vx|\geq 1$.
        
        Tomemos ahora $i=2$. Entonces, al bombear $v$ y $x$, tenemos que:
        \begin{equation*}
            |uv^iwx^iy| = |uvwxy| + (i-1)|vx| = n^2 + |vx|
        \end{equation*}

        Como $|vwx|\leq n$, tenemos que:
        \begin{equation*}
            |uv^iwx^iy| = n^2 + |vx| \leq n^2 + n < n^2+n+1\leq (n+1)^2
        \end{equation*}

        Además, como $|vx|\geq 1$, tenemos que:
        \begin{equation*}
            n^2<|uv^iwx^iy|<(n+1)^2
        \end{equation*}

        Por tanto, tenemos que $|uv^iwx^iy|$ no es un cuadrado perfecto, por lo que $uv^iwx^iy\notin L_2$.
        Por tanto, por el recíproco del Lema de Bombeo, $L_2$ no es independiente del contexto.
    \end{enumerate}
\end{ejercicio}

\begin{ejercicio}\label{ej:1.6.4}
    Considerar el lenguaje siguiente:
    \begin{equation*}
        L = \{0^n uu^{-1}1^n \mid u\in {\{0,1\}}^{\ast}\}
    \end{equation*}
    \begin{enumerate}
        \item Encontrar un autómata con pila que acepte, por el criterio de pila vacía, el lenguaje $L$.
        
        Sea el autómata con pila $M = (\{q_0,q_1,q_2,q_3,q_f\},\{0,1\},\{Z_0,X,0,1\},\delta,q_0,Z_0,\{q_f\})$ que acepta $L$ por ambos criterios (tanto por pila vacía como por estados finales), con la función de transición $\delta$ dada por:
        \begin{align*}
            &\delta(q_0,0,Z_0) = \{(q_0,XZ_0)\} \\
            &\delta(q_0,0,X) = \{(q_0,XX)\} \\
            &\red{\delta(q_0,\veps,i) = \{(q_1,i)\}} \qquad \text{Para } i\in \{Z_0,X\} \\
            &\delta(q_1,1,i) = \{(q_1,1i)\} \qquad \text{Para } i\in \{Z_0,X,1,0\} \\
            &\delta(q_1,0,i) = \{(q_1,0i)\} \qquad \text{Para } i\in \{Z_0,X,1,0\} \\
            &\red{\delta(q_1,\veps,i) = \{(q_2,i)\}} \qquad \text{Para } i\in \{Z_0,X,1,0\} \\
            &\delta(q_2,1,1) = \{(q_2,\veps)\} \\
            &\delta(q_2,0,0) = \{(q_2,\veps)\} \\
            &\red{\delta(q_2,\veps,i) = \{(q_3,i)\}} \qquad \text{Para } i\in \{Z_0,X\} \\
            &\delta(q_3,1,X) = \{(q_3,\veps)\} \\
            &\delta(q_3,\veps,Z_0) = \{(q_f,\veps)\}
        \end{align*}

        \item Encontrar un autómata que acepte $\ol{L}$.
        % // TODO: Añadir autómata que acepte L^c
    \end{enumerate}
\end{ejercicio}

\begin{ejercicio}\label{ej:1.6.5}
    Considerar la gramática libre de contexto dada por las siguientes producciones:
    \begin{align*}
        S &\rightarrow aABb \mid aBA \mid \veps \\
        A &\rightarrow aS \mid bAAA \\
        B &\rightarrow aABB \mid aBAB \mid aBBA \mid bS
    \end{align*}
    Determinar si las cadenas $aabaab$ y las cadenas $bbaaa$ son generadas por esta gramática
    \begin{enumerate}
        \item Haciendo una búsqueda en el árbol de todas las derivaciones, pasando previamente la gramática a forma normal de Greibach.
        
        En primer lugar, eliminamos las producciones nulas:
        \begin{align*}
            S &\rightarrow aABb \mid aBA \\
            A &\rightarrow aS \mid bAAA \mid a\\
            B &\rightarrow aABB \mid aBAB \mid aBBA \mid bS \mid b
        \end{align*}

        Obtenemos ahora la forma normal de Greibach:
        \begin{align*}
            S &\rightarrow aABC_b \mid aBA \\
            A &\rightarrow aS \mid bAAA \mid a\\
            B &\rightarrow aABB \mid aBAB \mid aBBA \mid bS \mid b\\
            C_b &\rightarrow b
        \end{align*}

        Vemos ahora que la cadena $aabaab$ no es generada por la gramática en la Figura~\ref{fig:1.6.5-1}.
        \begin{figure}
            \centering
            \begin{forest}
                for tree={
                    edge={-}, % Hace que las líneas no terminen en punta
                    fit=rectangle, % Ajusta el tamaño del rectángulo al texto
                }
                [$S$
                    [$aABC_b$
                        [$aaSBC_b$
                            [$\times$]
                        ]
                        [$aaBC_b$
                            [$aabC_b$
                                [$\times$]
                            ]
                            [$aabSC_b$
                                [$aabaABC_bC_b$
                                    [$\times$]
                                ]
                                [$aabaBAC_b$
                                    [$\times$]
                                ]
                            ]
                        ]
                    ]
                    [$aBA$
                        [$aaABBA$
                            [$\times$]
                        ]
                        [$aaBABA$
                            [$\times$]
                        ]
                        [$aaBBAA$
                            [$\times$]
                        ]
                    ]
                ]
            \end{forest}
            \caption{Árbol que muestra que $aabaab$ no es generada por la gramática.}
            \label{fig:1.6.5-1}
        \end{figure}

        Por otro lado, la cadena $bbaaa$ vemos de forma directa que no es generada por la gramática, ya que no se puede llegar a una cadena que empiece por $b$.

        
        \item Mediante el algoritmo de Cocke-Younger-Kasami.
        
        Pasamos ahora la gramática a forma normal de Chomsky:
        \begin{align*}
            S &\rightarrow C_aD_1 \mid C_aD_2 \\
            A &\rightarrow C_aS \mid C_bD_3 \mid a\\
            B &\rightarrow C_aD_4 \mid C_aD_5 \mid C_aD_6 \mid C_bS \mid b\\
            D_1 &\rightarrow AD_7\\
            D_2 &\rightarrow BA\\
            D_3 &\rightarrow AD_8\\
            D_4 &\rightarrow AD_9\\
            D_5 &\rightarrow BD_{10} \\
            D_6 &\rightarrow BD_{2} \\
            D_7 &\rightarrow BC_b \\
            D_8 &\rightarrow AA \\
            D_9 &\rightarrow BB\\
            D_{10} &\rightarrow AB\\
            C_a & \rightarrow a \\
            C_b & \rightarrow b
        \end{align*}

        En la Tabla~\ref{fig:1.6.5-2.1} vemos el algoritmo de Cocke-Younger-Kasami para la cadena $aabaab$, que nos muestra que no es generada por la gramática.
        \begin{table}
            \centering
            \begin{tabular}{cccccc}
                $a$ & $a$ & $b$ & $a$ & $a$ & $b$ \\ \hhline{*{6}{-}}
                \cell{C_a,A} & \cell{C_a,A} & \cell{C_b,B} & \cell{C_a,A} & \cell{C_a,A} & \cell{C_b,B} \\ \hhline{*{6}{-}}
                \cell{D_8} & \cell{D_{10}} & \cell{D_2} & \cell{D_8} & \cell{D_{10}} \\\hhline{*{5}{-}}
                \cell{\emptyset} & \cell{S} & \cell{\emptyset} & \cell{\emptyset} \\ \hhline{*{4}{-}}
                \cell{A} & \cell{\emptyset} & \cell{\emptyset} \\ \hhline{*{3}{-}}
                \cell{D_8} & \cell{\emptyset} \\ \hhline{*{2}{-}}
                \cell{\emptyset} \\ \hhline{*{1}{-}}
            \end{tabular}
            \caption{Algoritmo de Cocke-Younger-Kasami para la cadena $aabaab$.}
            \label{fig:1.6.5-2.1}
        \end{table}

        En la Tabla~\ref{fig:1.6.5-2.2} vemos el algoritmo de Cocke-Younger-Kasami para la cadena $bbaaa$, que nos muestra que no es generada por la gramática.
        \begin{table}
            \centering
            \begin{tabular}{ccccc}
                $b$ & $b$ & $a$ & $a$ & $a$ \\ \hhline{*{5}{-}}
                \cell{C_b,B} & \cell{C_b,B} & \cell{C_a,A} & \cell{C_a,A} & \cell{C_a,A} \\ \hhline{*{5}{-}}
                \cell{D_9} & \cell{D_2} & \cell{D_8} & \cell{D_8} \\ \hhline{*{4}{-}}
                \cell{D_6} & \cell{\emptyset} & \cell{D_3} \\ \hhline{*{3}{-}}
                \cell{\emptyset} & \cell{A} \\ \hhline{*{2}{-}}
                \cell{D_2} \\ \hhline{*{1}{-}}
            \end{tabular}
            \caption{Algoritmo de Cocke-Younger-Kasami para la cadena $bbaaa$.}
            \label{fig:1.6.5-2.2}
        \end{table}
    \end{enumerate}
\end{ejercicio}

\begin{ejercicio}\label{ej:1.6.6}
    Determinar qué lenguajes son regulares y/o libres de contexto:
    \begin{enumerate}
        \item $\{0^{n^2}\mid n\geq 1\}$.
        
        No es independiente del contexto, como ya demostramos en el Ejercicio~\ref{ej:1.6.3}.\ref{ej:1.6.3-2}.
        \item $\{0^n 1^n 0^n 1^n \mid n \geq 0\}$.
        
        Veamos que no es independiente del contexto por el recíproco del Lema de Bombeo. Para cada $n\in \bb{N}$, consideramos la palabra $z=0^n1^n0^n1^n\in L$, con $|z|=4n\geq n$. Consideramos ahora cualquier descomposición de $z$ en cinco partes $z=uvwxy$, con $u,v,w,x,y\in {\{0,1\}}^{\ast}$, de forma que $|vwx|\leq n$ y $|vx|\geq 1$.

        Como $|vwx|\leq n$, considerando que la palabra $z$ está dividida en cuatro partes iguales cada una de longitud $n$, caben dos opciones:
        \begin{enumerate}
            \item $vwx$ está contenido entero en una de las partes.
            \item $vwx$ corta dos partes consecutivas.
        \end{enumerate}

        En cualquier caso, no corta a las cuatro partes. Por tanto, al bombear $v$ y $x$ con $i=2$, se rompe el equilibrio entre las cuatro partes, teniendo así que $uv^2wx^2y\notin L$.

        Por tanto, por el recíproco del Lema de Bombeo, $L$ no es independiente del contexto.
        \item $\{0^n 10^m 10^{n+m}\mid n,m\geq 0\}$.
        
        Veamos que es independiente del contexto. Sea la gramática libre de contexto $G=(\{S,X\},\{0,1\},P,S)$, con $P$ dada por:
        \begin{align*}
            S &\rightarrow 0S0 \mid 1X \\
            X &\rightarrow 0X0 \mid 1
        \end{align*}

        Como $\cc{L}(G)=\{0^n 10^m 10^{n+m}\mid n,m\geq 0\}$, tenemos que $L$ es independiente del contexto. Veamos ahora que no es regular por el Lema de Bombeo.

        Para cada $n\in \bb{N}$, consideramos la palabra $z=0^n10^n10^{2n}\in L$, con $|z|=4n+2\geq n$. Consideramos ahora cualquier descomposición de $z$ de la forma $z=uvw$, con $u,v,w\in {\{0,1\}}^{\ast}$, de forma que $|v|\geq 1$ y $|uv|\leq n$. De esta forma:
        \begin{equation*}
            u=0^k \quad v=0^l \quad w=0^{n-k-l}10^n10^{2n}\qquad \text{con } k+l\leq n, l\geq 1
        \end{equation*}

        Al bombear $v$ con $i=2$, tenemos que:
        \begin{equation*}
            uv^2w = 0^{k+2l}0^{n-k-l}10^n10^{2n} = 0^{n+l}10^{n}10^{2n} \notin L
        \end{equation*}
        ya que $n+l+n\neq 2n$ por ser $l\geq 1$. Por tanto, por el recíproco del Lema de Bombeo, $L$ no es regular.
        \item Conjunto de palabras en las que toda posición impar está ocupada por un 1.
        
        Este lenguaje es regular. Sea el alfabeto del lenguaje $A$, con $1\in A$.
        \begin{equation*}
            A=\{1,a_1,\dots,a_{n-1}\}
        \end{equation*}

        Entonces, el lenguaje tiene como expresión regular:
        \begin{equation*}
            (1~(1+a_1+a_2+\cdots+a_{n-1}))^{\ast}~(1+\veps)
        \end{equation*}
    \end{enumerate}
\end{ejercicio}

\begin{ejercicio}\label{ej:1.6.7}
    Comprobar si las palabras $bba0d1$ y $cba1d1$ pertenecen al lenguaje generado por la gramática
    \begin{align*}
        S &\rightarrow AaB \mid AaC \\
        A &\rightarrow Ab \mid Ac \mid b \mid c \\
        B &\rightarrow BdC \mid 0 \\
        C &\rightarrow CeB \mid 1
    \end{align*}
    usando para ello:
    \begin{enumerate}
        \item El algoritmo de Cocke-Younger-Kasami.
        
        En primer lugar, pasamos la gramática a forma normal de Chomsky:
        \begin{align*}
            S &\rightarrow AD_0 \mid AD_1 \\
            A &\rightarrow AC_b \mid AC_c \mid b \mid c \\
            B &\rightarrow BD_2 \mid 0 \\
            C &\rightarrow CD_3 \mid 1 \\
            C_i &\rightarrow i\qquad \text{Para } i\in \{a,b,c,d,e\} \\
            D_0 &\rightarrow C_aB \\
            D_1 &\rightarrow C_aC \\
            D_2 &\rightarrow C_dC \\
            D_3 &\rightarrow C_eB
        \end{align*}

        En la Tabla~\ref{fig:1.6.7-1} vemos el algoritmo de Cocke-Younger-Kasami para la cadena $bba0d1$, que nos muestra que sí es generada por la gramática, con derivación:
        \begin{equation*}
            S\Rightarrow AD_0\Rightarrow AC_bC_aB\Rightarrow AC_bC_aBD_2\Rightarrow AC_bC_aBC_dC\Rightarrow bba0d1
        \end{equation*}
        \begin{table}
            \centering
            \begin{tabular}{cccccc}
                $b$ & $b$ & $a$ & $0$ & $d$ & $1$ \\ \hhline{*{6}{-}}
                \cell{A,C_b} & \cell{A,C_b} & \cell{C_a} & \cell{B} & \cell{C_d} & \cell{C} \\ \hhline{*{6}{-}}
                \cell{A} & \cell{\emptyset} & \cell{D_0} & \cell{\emptyset} & \cell{D_2} \\ \hhline{*{5}{-}}
                \cell{\emptyset} & \cell{S} & \cell{\emptyset} & \cell{B} \\ \hhline{*{4}{-}}
                \cell{S} & \cell{\emptyset} & \cell{D_0} \\ \hhline{*{3}{-}}
                \cell{\emptyset} & \cell{S} \\ \hhline{*{2}{-}}
                \cell{S} \\ \hhline{*{1}{-}}
            \end{tabular}
            \caption{Algoritmo de Cocke-Younger-Kasami para la cadena $bba0d1$.}
            \label{fig:1.6.7-1}
        \end{table}

        En la Tabla~\ref{fig:1.6.7-2} vemos el algoritmo de Cocke-Younger-Kasami para la cadena $cba1d1$, que nos muestra que no es generada por la gramática.
        \begin{table}
            \centering
            \begin{tabular}{cccccc}
                $c$ & $b$ & $a$ & $1$ & $d$ & $1$ \\ \hhline{*{6}{-}}
                \cell{A,C_c} & \cell{A,C_b} & \cell{C_a} & \cell{C} & \cell{C_d} & \cell{C} \\ \hhline{*{6}{-}}
                \cell{A} & \cell{\emptyset} & \cell{D_1} & \cell{\emptyset} & \cell{D_2} \\ \hhline{*{5}{-}}
                \cell{\emptyset} & \cell{S} & \cell{\emptyset} & \cell{\emptyset} \\ \hhline{*{4}{-}}
                \cell{S} & \cell{\emptyset} & \cell{\emptyset} \\ \hhline{*{3}{-}}
                \cell{\emptyset} & \cell{\emptyset} \\ \hhline{*{2}{-}}
                \cell{\emptyset} \\ \hhline{*{1}{-}}
            \end{tabular}
            \caption{Algoritmo de Cocke-Younger-Kasami para la cadena $cba1d1$.}
            \label{fig:1.6.7-2}
        \end{table}
        \item El algoritmo de Early.
        
        Para la cadena $bba0d1$, tenemos que:
        \begin{enumerate}[label=\arabic*), start=0]
            \item \ul{Registros$[0]$}: $(0,0,S,\veps,AaB)$, $(0,0,S,\veps,AaC)$, $(0,0,A,\veps,Ab)$, $(0,0,A,\veps,Ac)$, $(0,0,A,\veps,b)$, $(0,0,A,\veps,c)$.
            \item \ul{Registros$[1]$}: $(0,1,A,b,\veps)$, $(0,1,S,A,aB)$, $(0,1,S,A,aC)$, $(0,1,A,A,b)$, $(0,1,A,A,c)$.
            \item \ul{Registros$[2]$}: $(0,2,A,Ab,\veps)$, $(0,2,S,A,aB)$, $(0,2,S,A,aC)$, $(0,2,A,A,b)$, $(0,2,A,A,c)$.
            \item \ul{Registros$[3]$}: $(0,3,S,Aa,B)$, $(0,3,S,Aa,C)$, $(3,3,B,\veps,BdC)$, $(3,3,B,\veps,0)$, $(3,3,C,\veps,CeB)$, $(3,3,C,\veps,1)$.
            \item \ul{Registros$[4]$}: $(3,4,B,0,\veps)$, $(0,4,S,AaB,\veps)$, $(3,4,B,B,dC)$.
            \item \ul{Registros$[5]$}: $(3,5,B,Bd,C)$, $(5,5,C,\veps,CeB)$, $(5,5,C,\veps,1)$.
            \item \ul{Registros$[6]$}: $(5,6,C,1,\veps)$, $(3,6,B,BdC,\veps)$, $(5,6,C,C,eB)$, $(0,6,S,AaB,\veps)$, $(3,6,B,B,dC)$.
        \end{enumerate}

        Por tanto, la cadena $bba0d1$ es generada por la gramática, con derivación:
        \begin{equation*}
            S\Rightarrow AaB\Rightarrow AaBdC\Rightarrow Aba0d1 \Rightarrow bba0d1
        \end{equation*}

        Para la cadena $cba1d1$, tenemos que:
        \begin{enumerate}[label=\arabic*), start=0]
            \item \ul{Registros$[0]$}: $(0,0,S,\veps,AaB)$, $(0,0,S,\veps,AaC)$, $(0,0,A,\veps,Ab)$, $(0,0,A,\veps,Ac)$, $(0,0,A,\veps,b)$, $(0,0,A,\veps,c)$.
            \item \ul{Registros$[1]$}: $(0,1,A,c,\veps)$, $(0,1,S,A,aB)$, $(0,1,S,A,aC)$, $(0,1,A,A,b)$, $(0,1,A,A,c)$.
            \item \ul{Registros$[2]$}: $(0,2,A,Ab,\veps)$, $(0,2,S,A,aB)$, $(0,2,S,A,aC)$, $(0,2,A,A,b)$, $(0,2,A,A,c)$.
            \item \ul{Registros$[3]$}: $(0,3,S,Aa,B)$, $(0,3,S,Aa,C)$, $(3,3,B,\veps,BdC)$, $(3,3,B,\veps,0)$, $(3,3,C,\veps,CeB)$, $(3,3,C,\veps,1)$.
            \item \ul{Registros$[4]$}: $(3,4,C,1,\veps)$, $(0,4,S,AaC,\veps)$, $(3,4,C,C,eB)$.
            \item \ul{Registros$[5]$}: No hay registros.
            \item \ul{Registros$[6]$}: No hay registros.
        \end{enumerate}

        Por tanto, la cadena $cba1d1$ no es generada por la gramática.
    \end{enumerate}
\end{ejercicio}

\begin{ejercicio}\label{ej:1.6.8}
    Dada la gramática $G=(\{a,b,c,d\},\{S,A,B,C,D\},S,P)$ con producciones:
    \begin{align*}
        S &\rightarrow AB \mid C \\
        A &\rightarrow aAb \mid ab \\
        B &\rightarrow cBd \mid cd \\
        C &\rightarrow aCd \mid aDd \\
        D &\rightarrow bDc \mid bc
    \end{align*}
    determinar mediante el algoritmo de Cocke-Younger-Kasami si las palabras $abbccd$ y $aabbcd$ son generadas.\\

    En primer lugar, eliminamos las producciones unitarias:
    \begin{align*}
        S &\rightarrow AB \mid aCd\mid aDd \\
        A &\rightarrow aAb \mid ab \\
        B &\rightarrow cBd \mid cd \\
        C &\rightarrow aCd \mid aDd \\
        D &\rightarrow bDc \mid bc
    \end{align*}

    Ahora, obtenemos las producciones en forma normal de Chomsky:
    \begin{align*}
        S &\rightarrow AB \mid C_aX_1\mid C_aX_2 \\
        A &\rightarrow C_aX_3 \mid C_aC_b \\
        B &\rightarrow C_cX_4 \mid C_cC_d \\
        C &\rightarrow C_aX_1 \mid C_aX_2 \\
        D &\rightarrow C_bX_5 \mid C_bC_c \\
        X_1 &\rightarrow CC_d \\
        X_2 &\rightarrow DC_d \\
        X_3 &\rightarrow AC_b \\
        X_4 &\rightarrow BC_d \\
        X_5 &\rightarrow DC_c \\
        C_i &\rightarrow i \qquad \text{Para } i\in \{a,b,c,d\}
    \end{align*}

    En la Tabla~\ref{fig:1.6.8-1} vemos el algoritmo de Cocke-Younger-Kasami para la cadena $abbccd$, que nos muestra que sí es generada por la gramática. La derivación es:
    \begin{equation*}
        S\Rightarrow C_aX_2\Rightarrow aDC_d\Rightarrow aC_bX_5d\Rightarrow abDC_cd \Rightarrow abC_bC_cC_cd \Rightarrow abbccd
    \end{equation*}
    \begin{table}
        \centering
        \begin{tabular}{ccccccc}
            $a$ & $b$ & $b$ & $c$ & $c$ & $d$ \\ \hhline{*{6}{-}}
            \cell{C_a} & \cell{C_b} & \cell{C_b} & \cell{C_c} & \cell{C_c} & \cell{C_d} \\ \hhline{*{6}{-}}
            \cell{A} & \cell{\emptyset} & \cell{D} & \cell{\emptyset} & \cell{B} \\ \hhline{*{5}{-}}
            \cell{X_3} & \cell{\emptyset} & \cell{X_5} & \cell{\emptyset} \\ \hhline{*{4}{-}}
            \cell{\emptyset} & \cell{D} & \cell{\emptyset} \\ \hhline{*{3}{-}}
            \cell{\emptyset} & \cell{X_2} \\ \hhline{*{2}{-}}
            \cell{S} \\ \hhline{*{1}{-}}
        \end{tabular}
        \caption{Algoritmo de Cocke-Younger-Kasami para la cadena $abbccd$.}
        \label{fig:1.6.8-1}
    \end{table}

    En la Tabla~\ref{fig:1.6.8-2} vemos el algoritmo de Cocke-Younger-Kasami para la cadena $aabbcd$, que nos muestra que sí es generada por la gramática. La derivación es:
    \begin{equation*}
        S\Rightarrow AB\Rightarrow C_aX_3C_cC_d
        \Rightarrow aAC_bcd \Rightarrow aC_aC_bbcd \Rightarrow aabbcd
    \end{equation*}
    \begin{table}
        \centering
        \begin{tabular}{ccccccc}
            $a$ & $a$ & $b$ & $b$ & $c$ & $d$ \\ \hhline{*{6}{-}}
            \cell{C_a} & \cell{C_a} & \cell{C_b} & \cell{C_b} & \cell{C_c} & \cell{C_d} \\ \hhline{*{6}{-}}
            \cell{\emptyset} & \cell{A} & \cell{\emptyset} & \cell{D} & \cell{B} \\ \hhline{*{5}{-}}
            \cell{\emptyset} & \cell{X_3} & \cell{\emptyset} & \cell{X_2} \\ \hhline{*{4}{-}}
            \cell{A} & \cell{\emptyset} & \cell{\emptyset} \\ \hhline{*{3}{-}}
            \cell{\emptyset} & \cell{\emptyset} \\ \hhline{*{2}{-}}
            \cell{S} \\ \hhline{*{1}{-}}
        \end{tabular}
        \caption{Algoritmo de Cocke-Younger-Kasami para la cadena $aabbcd$.}
        \label{fig:1.6.8-2}
    \end{table}
\end{ejercicio}

\begin{ejercicio}\label{ej:1.6.9}
    Determinar si son regulares y/o independientes del contexto los siguientes lenguajes:
    \begin{enumerate}
        \item $\{uu^{-1}u \mid u\in {\{0,1\}}^{\ast}\}$.
        
        Veamos que no es independiente del contexto por el recíproco del Lema de Bombeo. Para cada $n\in \bb{N}$, consideramos la palabra $z=0^n1^{n}2^n0^{n}0^n1^n\in L$, con $|z|=6n\geq n$. Consideramos ahora cualquier descomposición de $z$ en cinco partes $z=uvwxy$, con $u,v,w,x,y\in {\{0,1\}}^{\ast}$, de forma que $|vwx|\leq n$ y $|vx|\geq 1$.

        Como $|vwx|\leq n$, considerando que la palabra $z$ está dividida en seis partes, caben dos opciones:
        \begin{enumerate}
            \item $vwx$ está contenido entero en una de las partes.
            \item $vwx$ corta dos partes consecutivas.
        \end{enumerate}

        En cualquier caso, no corta a tres partes. Por tanto, en el caso de que $vwx$ comienze por $0$'s (recíprocamente con $1$'s), no podremos bombear $0$'s de dos partes distintas, rompiendo entonces el equilibrio. Por tanto, tenemos que:
        \begin{equation*}
            uv^2wx^2y\notin L
        \end{equation*}

        Por tanto, por el recíproco del Lema de Bombeo, $L$ no es independiente del contexto.
        \item $\{uu^{-1}ww^{-1}\mid u,w\in {\{0,1\}}^{\ast}\}$.
        
        Veamos que sí es independiente del contexto. Sea la gramática libre de contexto $G=(\{S,A,B,C\},\{0,1\},P,S)$, con $P$ dada por:
        \begin{align*}
            S &\rightarrow AB \\
            A &\rightarrow 0A0 \mid 1A1 \mid C \\
            B &\rightarrow 0B0 \mid 1B1 \mid C \\
            C &\rightarrow 0 \mid 1\mid \veps
        \end{align*}

        Como $\cc{L}(G)=\{uu^{-1}ww^{-1}\mid u,w\in {\{0,1\}}^{\ast}\}$, tenemos que $L$ es independiente del contexto. Veamos ahora que no es regular por el Lema de Bombeo.

        Para cada $n\in \bb{N}$, consideramos la palabra $z=0^{n}1^{2n}0^{n}\in L$, con $|z|=4n\geq n$. Consideramos ahora cualquier descomposición de $z$ de la forma $z=uvw$, con $u,v,w\in {\{0,1\}}^{\ast}$, de forma que $|v|\geq 1$ y $|uv|\leq n$. De esta forma:
        \begin{equation*}
            u=0^k \quad v=0^l \quad w=0^{n-k-l}1^{2n}0^n\qquad \text{con } k+l\leq n, l\geq 1
        \end{equation*}

        Al bombear $v$ $i=2$ veces, tenemos que:
        \begin{equation*}
            uv^2w = 0^{k+2l}0^{n-k-l}1^{2n}0^n = 0^{n+l}1^{2n}0^n \notin L
        \end{equation*}
        ya que $n+l\neq n$. Por tanto, por el recíproco del Lema de Bombeo, $L$ no es regular.

        \item $\{uu^{-1}w \mid u,w\in {\{0,1\}}^{\ast} \text{\ y\ } |u|\leq 3\}$
        
        Consideramos los lenguajes auxiliares:
        \begin{align*}
            L_1 &= \{uu^{-1}\mid u\in {\{0,1\}}^{\ast},~|u|\leq 3\} \\
            L_2 &= \{w\mid w\in {\{0,1\}}^{\ast}\}
        \end{align*}

        Tenemos que $L_1$ es regular por ser finito, y $L_2$ es regular tener expresión regular $(0+1)^{\ast}$. Por tanto, $L$ es regular por ser la concatenación de dos lenguajes regulares.
    \end{enumerate}
\end{ejercicio}

\begin{ejercicio}\label{ej:1.6.10}
    Construir una gramática independiente del contexto para el lenguaje más pequeño que verifica las siguientes reglas:
    \begin{enumerate}
        \item Cualquier sucesión de dígitos de $\{0,1,2,\ldots, 9\}$ de longitud mayor o igual a 1 es una palabra del lenguaje.
        \item Si $u_1,\ldots,u_n$ ($n\geq 1$) son palabras del lenguaje, entonces $(u_1+\cdots + u_n)$ es una palabra del lenguaje.
        \item Si $u_1,\ldots,u_n$ ($n\geq 1$) son palabras del lenguaje, entonces $[u_1 \ast \cdots \ast u_n]$ es una palabra del lenguaje.
    \end{enumerate}
    Comprobar por el algoritmo de Cocke-Younger-Kasami si las palabras: $(0+1)*3$ y $[(0+1)]$ son generadas por la gramática.\\

    Sea la gramática buscada $G=(\{0,1,2,\ldots,9,+,\ast,(,)\},\{S,\Pi,\Sigma,N,D\},S,P)$, con $P$ dada por:
    \begin{align*}
        S &\rightarrow N \mid (~\Sigma~) \mid [~\Pi~] \\
        N &\rightarrow ND \mid D \\
        D &\rightarrow 0 \mid 1 \mid 2 \mid \cdots \mid 9 \\
        \Pi &\rightarrow \Pi \ast \Pi \mid S \\
        \Sigma &\rightarrow \Sigma + \Sigma \mid S
    \end{align*}

    % // TODO: Pasar a FN de Chomsky. Muy largo
\end{ejercicio}

\begin{ejercicio}\label{ej:1.6.11}
    Encuentra una gramática libre de contexto en forma normal de Chomsky que genere el siguiente lenguaje:
    \begin{equation*}
        L = \{ucv \mid u,v\in {\{0,1\}}^{+} \text{\ y nº de subcadenas ``} 01 \text{'' en\ } u \text{\ es igual al nº subcadenas ``}10 \text{'' en\ }v\}
    \end{equation*}
    Comprueba con el algoritmo CYK si la cadena $010c101$ pertenece al lenguaje generado por la gramática.

    % // TODO: Sabemos el autómata. Algoritmo de autómata a gramática.
\end{ejercicio}

\begin{ejercicio}\label{ej:1.6.12}
    Encuentra una gramática libre de contexto en forma normal de Chomsky que genere el siguiente lenguaje definido sobre el alfabeto $\{a,0,1\}$:
    \begin{equation*}
        L = \{auava \mid u,v\in {\{0,1\}}^{\ast} \text{\ y\ } u=v^{-1}\}
    \end{equation*}
    Comprueba con el algoritmo CYK si la cadenas $a0a0a$ y $a1a0a$ pertenecen al lenguaje generado por la gramática.\\

    Sea $u,v\in \{0,1\}^*$, con $u=v^{-1}$. Entonces, $u^{-1}=(v^{-1})^{-1}=v$. Por tanto, tenemos que:
    \begin{align*}
        L &= \{auau^{-1}a \mid u\in {\{0,1\}}^{\ast}\}
    \end{align*}

    Sea por tanto la gramática $G=(\{a,0,1\},\{S,X\},S,P)$, con $P$ dada por:
    \begin{align*}
        S &\rightarrow aXa\\
        X &\rightarrow 0X0 \mid 1X1 \mid a
    \end{align*}

    Pasamos ahora la gramática a forma normal de Chomsky:
    \begin{align*}
        S &\rightarrow C_aD_1\\
        X &\rightarrow C_0D_2 \mid C_1D_3 \mid a\\
        D_1 &\rightarrow XC_a\\
        D_2 &\rightarrow XC_0\\
        D_3 &\rightarrow XC_1\\
        C_i &\rightarrow i \qquad \text{Para } i\in \{a,0,1\}
    \end{align*}

    En la Tabla~\ref{fig:1.6.12-1} vemos el algoritmo de Cocke-Younger-Kasami para la cadena $a0a0a$, que nos muestra que sí es generada por la gramática. La derivación es:
    \begin{equation*}
        S\Rightarrow aXa\Rightarrow a0X0a\Rightarrow a0a0a
    \end{equation*}
    \begin{table}
        \centering
        \begin{tabular}{cccccc}
            $a$ & $0$ & $a$ & $0$ & $a$ \\ \hhline{*{5}{-}}
            \cell{C_a,X} & \cell{C_0} & \cell{C_a,X} & \cell{C_0} & \cell{C_a,X} \\ \hhline{*{5}{-}}
            \cell{D_2} & \cell{\emptyset} & \cell{D_2} & \cell{\emptyset} \\ \hhline{*{4}{-}}
            \cell{\emptyset} & \cell{X} & \cell{\emptyset} \\ \hhline{*{3}{-}}
            \cell{\emptyset} & \cell{D_1} \\ \hhline{*{2}{-}}
            \cell{S} \\ \hhline{*{1}{-}}
        \end{tabular}
        \caption{Algoritmo de Cocke-Younger-Kasami para la cadena $a0a0a$.}
        \label{fig:1.6.12-1}
    \end{table}

    En la Tabla~\ref{fig:1.6.12-2} vemos el algoritmo de Cocke-Younger-Kasami para la cadena $a1a0a$, que nos muestra que no es generada por la gramática.
    \begin{table}
        \centering
        \begin{tabular}{cccccc}
            $a$ & $1$ & $a$ & $0$ & $a$ \\ \hhline{*{5}{-}}
            \cell{C_a,X} & \cell{C_1} & \cell{C_a,X} & \cell{C_0} & \cell{C_a,X} \\ \hhline{*{5}{-}}
            \cell{D_3} & \cell{\emptyset} & \cell{D_2} & \cell{\emptyset} \\ \hhline{*{4}{-}}
            \cell{\emptyset} & \cell{\emptyset} & \cell{\emptyset} \\ \hhline{*{3}{-}}
            \cell{\emptyset} & \cell{\emptyset} \\ \hhline{*{2}{-}}
            \cell{\emptyset} \\ \hhline{*{1}{-}}
        \end{tabular}
        \caption{Algoritmo de Cocke-Younger-Kasami para la cadena $a1a0a$.}
        \label{fig:1.6.12-2}
    \end{table}
\end{ejercicio}

\begin{ejercicio}\label{ej:1.6.13}
    Encuentra una gramática libre de contexto en forma normal de Chomsky que genere los siguientes lenguaje definidos sobre el alfabeto $\{a,0,1\}$:
    \begin{align*}
        L_1 &= \{auava \mid u,v \in {\{0,1\}}^{+} \text{\ y\ } u^{-1} = v\} \\
        L_2 &= \{uvu \mid u\in {\{0,1\}}^{+}\text{\ y\ } u^{-1} = v\}
    \end{align*}
    Comprueba con el algoritmo CYK si la cadena $a0a0a$ pertenece a $L_1$ y la cadena $011001$ pertenece al lenguaje $L_2$.\\

    En primer lugar, y razonando como en el ejercicio anterior, tenemos que:
    \begin{align*}
        L_1 &= \{auau^{-1}a \mid u \in {\{0,1\}}^{+}\} \\
        L_2 &= \{uu^{-1}u \mid u\in {\{0,1\}}^{+}\}
    \end{align*}

    % // TODO: L_2 No es independiente del contexto.
\end{ejercicio}

\begin{ejercicio}\label{ej:1.6.14}
    Sea la gramatica $G = (\{a,b\},\{S,A,B\},S,P)$ siendo $P$:
    \begin{align*}
        S&\rightarrow AabB \\
        A &\rightarrow aA \mid bA \mid \veps \\
        B&\rightarrow Bab \mid Bb \mid ab \mid b
    \end{align*}
    \begin{enumerate}
        \item ¿Es regular el lenguaje que genera $G$?
        
        Veamos el lenguaje generado por $A$ y $B$. Notando por $L(A)$ y $L(B)$ a los lenguajes generados por $A$ y $B$ respectivamente, tenemos que:
        \begin{align*}
            L(A) &= \{a,b\}^{\ast} \\
            L(B) &= \{ab,b\}^{+}
        \end{align*}

        Por tanto, $\cc{L}(G)$ es regular por ser concatenación de lenguajes regulares, con expresión regular:
        \begin{equation*}
            (a+b)^{\ast}ab(b+ab)^{+}
        \end{equation*}
        \item Transforma $G$ a una gramatica equivalente en Forma Normal de Chomsky.
        
        En primer lugar, eliminamos las producciones nulas:
        \begin{align*}
            S&\rightarrow AabB \mid abB \\
            A &\rightarrow aA \mid bA \mid a \mid b \\
            B&\rightarrow Bab \mid Bb \mid ab \mid b
        \end{align*}

        En segundo lugar, como no hay producciones unitarias, pasamos a forma normal de Chomsky:
        \begin{align*}
            S&\rightarrow AX_1 \mid C_aX_2 \\
            A &\rightarrow C_aA \mid C_bA \mid a \mid b \\
            B&\rightarrow BX_3 \mid BC_b \mid C_aC_b \mid b\\
            X_1 &\rightarrow C_aX_2 \\
            X_2 &\rightarrow C_bB \\
            X_3 &\rightarrow C_aC_b \\
            C_i &\rightarrow i \qquad \text{Para } i\in \{a,b\}
        \end{align*}
        \item Aplicando el algoritmo CYK, determinar si las siguientes cadenas pertenecen a $cc{L}(G)$: $aababb$, $aaba$.
        
        En la Tabla~\ref{fig:1.6.14-1} vemos el algoritmo de Cocke-Younger-Kasami para la cadena $aababb$, que nos muestra que sí es generada por la gramática.
        \begin{table}
            \centering
            \begin{tabular}{ccccccc}
                $a$ & $a$ & $b$ & $a$ & $b$ & $b$ \\ \hhline{*{6}{-}}
                \cell{C_a,A} & \cell{C_a,A} & \cell{C_b,A,B} & \cell{C_a,A} & \cell{C_b,A,B} & \cell{C_b,A,B} \\ \hhline{*{6}{-}}
                \cell{A} & \cell{A,B,X_3} & \cell{A} & \cell{A,B,X_3} & \cell{A,B,X_2} \\ \hhline{*{5}{-}}
                \cell{A} & \cell{A} & \cell{A,B,X_2} & \cell{A,B,X_1} \\ \hhline{*{4}{-}}
                \cell{A} & \cell{A,S,B,X_1} & \cell{A,B,S,X_2} \\ \hhline{*{3}{-}}
                \cell{A,S} & \cell{S,A,B,X_1} \\ \hhline{*{2}{-}}
                \cell{A,S} \\ \hhline{*{1}{-}}
            \end{tabular}
            \caption{Algoritmo de Cocke-Younger-Kasami para la cadena $aababb$.}
            \label{fig:1.6.14-1}
        \end{table}

        En la Tabla~\ref{fig:1.6.14-2} vemos el algoritmo de Cocke-Younger-Kasami para la cadena $aaba$, que nos muestra que no es generada por la gramática (notemos que podemos copiarlo de forma directa de la Tabla~\ref{fig:1.6.14-1}).
        \begin{table}
            \centering
            \begin{tabular}{cccccc}
                $a$ & $a$ & $b$ & $a$ \\ \hhline{*{4}{-}}
                \cell{C_a,A} & \cell{C_a,A} & \cell{C_b,A,B} & \cell{C_a,A} \\ \hhline{*{4}{-}}
                \cell{A} & \cell{A,B,X_3} & \cell{A} \\ \hhline{*{3}{-}}
                \cell{A} & \cell{A} \\ \hhline{*{2}{-}}
                \cell{A} \\ \hhline{*{1}{-}}
            \end{tabular}
            \caption{Algoritmo de Cocke-Younger-Kasami para la cadena $aaba$.}
            \label{fig:1.6.14-2}
        \end{table}
        \item Muestra el arbol de derivación para generar las palabras del apartado anterior que pertenecen a $\cc{L}(G)$.\\
        
        El árbol de derivación (aunque no es único) de $aabaab\in \cc{L}(G)$ se muestra en la Figura~\ref{fig:1.6.14-1-1}.
        \begin{figure}
            \centering
            \begin{forest}
                for tree={
                    edge={-}, % Hace que las líneas no terminen en punta
                    fit=rectangle, % Ajusta el tamaño del rectángulo al texto
                }
                [$S$
                    [$A$
                        [$C_a$
                            [$a$]
                        ]
                        [$A$
                            [$C_a$
                                [$a$]
                            ]
                            [$A$
                                [$b$]
                            ]
                        ]
                    ]
                    [$X_1$
                        [$C_a$
                            [$a$]
                        ]
                        [$X_2$
                            [$C_b$
                                [$b$]
                            ]
                            [$B$
                                [$b$]
                            ]
                        ]
                    ]
                ]
            \end{forest}
            \caption{Árbol de derivación para la cadena $aababb$.}
            \label{fig:1.6.14-1-1}
        \end{figure}
    \end{enumerate}
\end{ejercicio}

\begin{ejercicio}\label{ej:1.6.15}
    Dada la gramática $G=(\{a,b,c,d\},\{S,A,B,C,D,E\},S,P)$ con producciones:
    \begin{align*}
        S &\rightarrow AB \mid C \mid BE \\
        A &\rightarrow aAb \mid \veps \\
        B &\rightarrow cBd \mid \veps \\
        C &\rightarrow aCd \mid aDd \\
        D &\rightarrow bDc \mid \veps
    \end{align*}
    determinar mediante el algoritmo de Cocke-Younger-Kasami si las palabras $abbccd$ y $aabbcd$ son generadas por esta gramática.\\

    En primer lugar, eliminamos las producciones y variables inútiles. Como desde $E$ no podemos llegar a símbolos terminales, eliminamos $E$ y las producciones que lo contienen.
    \begin{align*}
        S &\rightarrow AB \mid C \\
        A &\rightarrow aAb \mid \veps \\
        B &\rightarrow cBd \mid \veps \\
        C &\rightarrow aCd \mid aDd \\
        D &\rightarrow bDc \mid \veps
    \end{align*}

    Eliminamos ahora las producciones nulas:
    \begin{align*}
        S &\rightarrow AB \mid C \mid Acd\mid abB \mid abcd\\
        A &\rightarrow aAb \mid ab \\
        B &\rightarrow cBd \mid cd \\
        C &\rightarrow aCd \mid aDd \mid ad\\
        D &\rightarrow bDc \mid bc
    \end{align*}

    Eliminamos ahora las producciones unitarias:
    \begin{align*}
        S &\rightarrow AB \mid Acd\mid abB \mid abcd \mid aCd \mid aDd \mid ad\\
        A &\rightarrow aAb \mid ab \\
        B &\rightarrow cBd \mid cd \\
        C &\rightarrow aCd \mid aDd \mid ad\\
        D &\rightarrow bDc \mid bc
    \end{align*}

    Aplicamos ahora el algoritmo para obtener la Forma Normal de Chomsky:
    \begin{align*}
        S &\rightarrow AB \mid AC_{cd}\mid C_aX_1 \mid C_aC_{bcd} \mid C_aX_2 \mid C_aX_3 \mid C_aC_d\\
        A &\rightarrow C_aX_4 \mid C_aC_b \\
        B &\rightarrow C_cX_5 \mid C_cC_d \\
        C &\rightarrow C_aX_2 \mid C_aX_3 \mid C_aC_d\\
        D &\rightarrow C_bX_6 \mid C_bC_c \\
        C_{cd} &\rightarrow C_cC_d \\
        C_{bcd} &\rightarrow C_bC_{cd} \\
        X_1 &\rightarrow C_bB \\
        X_2 &\rightarrow CC_d \\
        X_3 &\rightarrow DC_d \\
        X_4 &\rightarrow AC_b \\
        X_5 &\rightarrow BC_d \\
        X_6 &\rightarrow DC_c \\
        C_i &\rightarrow i \qquad \text{Para } i\in \{a,b,c,d\}
    \end{align*}

    En la Tabla~\ref{fig:1.6.15-1} vemos el algoritmo de Cocke-Younger-Kasami para la cadena $abbccd$, que nos muestra que sí es generada por la gramática. La derivación es:
    \begin{equation*}
        S\Rightarrow C_aX_3\Rightarrow aDC_d\Rightarrow aC_bX_6d\Rightarrow abDC_cd \Rightarrow abC_bC_ccd \Rightarrow abbccd
    \end{equation*}
    \begin{table}
        \centering
        \begin{tabular}{ccccccc}
            $a$ & $b$ & $b$ & $c$ & $c$ & $d$ \\ \hhline{*{6}{-}}
            \cell{C_a} & \cell{C_b} & \cell{C_b} & \cell{C_c} & \cell{C_c} & \cell{C_d} \\ \hhline{*{6}{-}}
            \cell{A} & \cell{\emptyset} & \cell{D} & \cell{\emptyset} & \cell{B,C_{cd}} \\ \hhline{*{5}{-}}
            \cell{X_4} & \cell{\emptyset} & \cell{X_6} & \cell{\emptyset} \\ \hhline{*{4}{-}}
            \cell{\emptyset} & \cell{D} & \cell{\emptyset} \\ \hhline{*{3}{-}}
            \cell{\emptyset} & \cell{X_3} \\ \hhline{*{2}{-}}
            \cell{S,C} \\ \hhline{*{1}{-}}
        \end{tabular}
        \caption{Algoritmo de Cocke-Younger-Kasami para la cadena $abbccd$.}
        \label{fig:1.6.15-1}
    \end{table}

    En la Tabla~\ref{fig:1.6.15-2} vemos el algoritmo de Cocke-Younger-Kasami para la cadena $aabbcd$, que nos muestra que sí es generada por la gramática. La derivación es:
    \begin{equation*}
        S\Rightarrow AB\Rightarrow C_aX_4C_cC_d \Rightarrow aAC_bcd \Rightarrow aC_aC_bbcd \Rightarrow aabbcd
    \end{equation*}
    \begin{table}
        \centering
        \begin{tabular}{ccccccc}
            $a$ & $a$ & $b$ & $b$ & $c$ & $d$ \\ \hhline{*{6}{-}}
            \cell{C_a} & \cell{C_a} & \cell{C_b} & \cell{C_b} & \cell{C_c} & \cell{C_d} \\ \hhline{*{6}{-}}
            \cell{\emptyset} & \cell{A} & \cell{\emptyset} & \cell{D} & \cell{B,C_{cd}} \\ \hhline{*{5}{-}}
            \cell{\emptyset} & \cell{X_4} & \cell{\emptyset} & \cell{X_1,C_{bcd},X_3} \\ \hhline{*{4}{-}}
            \cell{A} & \cell{\emptyset} & \cell{\emptyset} \\ \hhline{*{3}{-}}
            \cell{\emptyset} & \cell{\emptyset} \\ \hhline{*{2}{-}}
            \cell{S} \\ \hhline{*{1}{-}}
        \end{tabular}
        \caption{Algoritmo de Cocke-Younger-Kasami para la cadena $aabbcd$.}
        \label{fig:1.6.15-2}
    \end{table}
\end{ejercicio}

\begin{ejercicio}\label{ej:1.6.16}
    Demostrar que si $L_1$ es independiente del contexto y $L_2$ es regular, entonces $L_1\cap L_2$ es independiente del contexto.\\

    Sea $L_1$ independiente del contexto y $L_2$ regular, ambos sobre el alfabeto $A$. Por ser $L_1$ independiente del contexto, existe un APND $M_1=(Q_1,A,B,\delta_1,q_0^1,F_1)$ que acepta $L_1$ por el criterio de los estados finales. Por ser $L_2$ regular, existe un AFD $M_2=(Q_2,A,\delta_2,q_0^2,F_2)$ que acepta $L_2$. Construimos el siguiente autómata con píla:
    \begin{equation*}
        M=(Q,A,B,\delta,q_0,F)
    \end{equation*}
    donde:
    \begin{itemize}
        \item $Q=Q_1\times Q_2$.
        \item $q_0=(q_0^1,q_0^2)$.
        \item $F=F_1\times F_2$.
        \item Describamos ahora $\delta$ en función de $\delta_1$ y $\delta_2$:
        \begin{align*}
            \delta((p,q),a,X)&=\{((r,s),\alpha)\mid (r,\alpha)\in \delta_1(p,a,X),~s=\delta_2(q,a)\}\quad \forall (p,q)\in Q,~a\in A,~X\in B\\
            \delta((p,q),\veps,X)&=\{((r,q),\alpha)\mid (r,\alpha)\in \delta_1(p,\veps,X)\}\quad \forall (p,q)\in Q,~X\in B
        \end{align*}
    \end{itemize}

    De esta forma, como $M$ acepta $L_1\cap L_2$ por el criterio de los estados finales, tenemos que $L_1\cap L_2$ es independiente del contexto.
\end{ejercicio}

\begin{ejercicio}\label{ej:1.6.17}
    Si $L_1$ y $L_2$ son lenguajes sobre el alfabeto $A$, entonces se define el cociente $L_1/L_2 = \{u\in A^\ast \mid \exists w\in L_2 \text{\ tal que\ } uw\in L_1\}$ . Demostrar que si $L_1$ es independiente del contexto y $L_2$ regular, entonces $L_1/L_2$ es independiente del contexto.

    % // TODO: Hacer JJ
\end{ejercicio}

\begin{ejercicio}\label{ej:1.6.18}
    Si $L$ es un lenguaje sobre $\{0,1\}$, sea $\operatorname{SUF}(L)$ el conjunto de los sufijos de palabras de $L$: 
    \begin{equation*}
        \operatorname{SUF}(L) = \{u\in {\{0,1\}}^{\ast} \mid \exists v\in {\{0,1\}}^{\ast}, \text{\ tal que\ } vu\in L\}.
    \end{equation*}
    Demostrar que si $L$ es independiente del contexto, entonces $\operatorname{SUF}(L)$ también es independiente del contexto.

    % // TODO: Hacer JJ
\end{ejercicio}

\begin{ejercicio}\label{ej:1.6.19}
    Demostrar que $L=\{0^i 1^i \mid i\geq 0\} \cup \{0^i1^{2i} \mid i\geq 0\}$ es independiente del contexto, pero no es determinista.\\

    Tenemos que es independiente del contexto por ser unión de lenguajes independientes del contexto. Veamos ahora que no es determinista.
    
    % // TODO: Ver que no es determinista.
    

    \begin{comment}
    Para ello, consideramos como lenguaje auxiliar el lenguaje regular $L_1$ con expresión regular $0^*1^*$.

    Por reducción al absurdo, supongamos que $L$ es determinista. Entonces, como los lenguajes independientes del contexto \emph{deterministas} son cerrados por complementarios, tenemos que $\ol{L}$ es independiente del contexto (determinista). Como la intersección de un lenguaje independiente del contexto y un lenguaje regular es independiente del contexto, tenemos que $\ol{L}\cap L_1$ es independiente del contexto:
    \begin{equation*}
        \ol{L}\cap L_1 = \{0^i1^j \mid i\neq j ~\land~ 2i\neq j\}
    \end{equation*}
    Por ser $\ol{L}\cap L_1$ independiente del contexto, cumple el Lema de Bombeo. Sea $n$ la constante que nos da este Lema, y sea $z=0^n1^{2n}$. Por el Lema de Bombeo, podemos escribir $z=uvwxy$ con $|vwx|\leq n$, $|vx|\geq 1$ y $uv^iwx^iy\in \ol{L}\cap L_1$ para todo $i\geq 0$. Consideramos ahora $i=2$. Entonces, $uv^2wx^2y=0^{n+|vx|}1^{2n+|vx|}\in \ol{L}\cap L_1$, lo cual es una contradicción. Por tanto, $L$ no es determinista.
    \end{comment}



\end{ejercicio}

\subsection{Preguntas Tipo Test}
Indicar si son verdaderas o falsas las siguientes afirmaciones:
\begin{enumerate}
    \item La intersección de lenguajes libres de contexto es siempre libre de contexto.
    \item Existe un algoritmo para determinar si una palabra es generada por una gramática independiente del contexto.
    \item El lenguaje $\{a^i b^j c^i d^i \mid i,j\geq 0\}$ es independiente del contexto.
    \item Existe un algoritmo para determinar si una gramática independiente del contexto es ambigua.
    \item Existe un algoritmo para comprobar cuando dos gramáticas libres de contexto generan el mismo lenguaje.
    \item El lenguaje $L = \{0^i 1^j 2^k \mid i\leq i \leq j \leq k\}$ es independiente del contexto.
    \item Si el lenguaje $L$ es independiente del contexto, entonces $L^{-1}$ es independiente del contexto.
    \item Existe un algoritmo que permite determinar si una gramática independiente del contexto genera un lenguaje finito o infinito.
    \item Existe un algoritmo para determinar si una gramática independiente del contexto es ambigua.
    \item En el algoritmo de Earley, la presencia del registro $(2,5,A,CD,adS)$ implica que a partir de $CD$ se puede generar la subcadena de la palrba de entrada que va del carácter 3 al 5.
    \item Existe un algoritmo para comprobar si el lenguaje generado por una gramática libre de contexto es regular.
    \item El algoritmo de Earley se puede aplicar a cualquier gramática independiente del contexto (sin producciones nulas ni unitarias).
    \item El conjunto de palabras $\{a^n b^n c^i \mid i\leq n\}$ es independiente del contexto.
    \item Si $L_1$ y $L_2$ son independientes del contexto, entonces $L_1-L_2$ es siempre independiente del contexto.
    \item Hay lenguajes que no son independientes del contexto y si verifican la condición que aparece en el lema de bombeo para lenguajes independientes del contexto.
    \item El conjunto de palabras $\{u011u\mid u\in {\{0,1\}}^{\ast}\}$ es independiente del contexto.
    \item El conjunto de palabras que contienen la subcadena $011$ es independiente del contexto.
    \item En el algoritmo de Cocke-Younger-Kasami calculamos los conjuntos $V_{ij}$ que son las variables que generan la subcadena de la palabra de entrada que va desde el símbolo en la posición $i$ al símbolo en la posición $j$.
    \item Un lenguaje puede cumplir la negación de la condición que aparece en el lema de bombeo para lenguajes independientes del contexto y ser regular.
    \item Existe un algoritmo para comprobar si el lenguaje generado con una gramática independiente del contexto es finito o infinito.
    \item Si $L_1$ y $L_2$ son lenguajes independientes de contexto, entonces ${(L_1L_2 \cup L_1)}^{\ast}$ es independiente del contexto.
    \item Si $L_1$ y $L_2$ son lenguajes independientes de contexto, entonces $(L_1-L_2)$ es independiente del contexto.
    \item Existe un algoritmo para determinar si una palabra $u$ tiene más de un árbol de derivación en una gramática independiente del contexto $G$.
    \item La intersección de dos lenguajes independientes de contexto con un número finito de palabras produce siempre un lenguaje regular.
    \item El complementario de un lenguaje con un número finitos de palabras es siempre libre de contexto.
    \item Todo lenguaje aceptado por un autómata con pila por el criterio de estados finales cumple la condición que aparece en el lema de bombeo para lenguajes libres de contexto.
    \item No existe algoritmo que para toda gramática libre de contexto $G$ nos indique si el lenguaje generado por esta gramática $L(G)$ es finito o infinito.
    \item Si $L_1$ y $L_2$ son lenguajes independientes de contexto, entonces ${(L_1L_2\cup L_1)}^{\ast}$ puede ser representado por un autómata con pila.
    \item Existe un algoritmo para determinar si un autómata con pila es determinista.
    \item La demostración del lema de bombeo para lenguajes independientes del contexto se basa en que si las palabras superan una longitud determinada, entonces en el árbol de derivación debe de aparecer una variable como descendiente de ella misma.
    \item La unión de dos lenguajes independientes contexto puede ser siempre aceptada por un autómata con pila.
    \item El complementario de un lenguaje libre de contexto con una cantidad finita de palabras no tiene porque producir otro lenguaje libre de contexto.
    \item El lema de bombeo para lenguajes libres de contexto es útil para demostrar que un lenguaje determinado no es libre de contexto.
    \item La intersección de dos lenguajes independientes del contexto da lugar a un lenguaje aceptado por un autómata con pila determinista.
    \item No existe algoritmo que reciba como entrada una gramática independiente del contexto y nos devuelva si el lenguaje generado por esta gramática es finito o infinito.
    \item En el algoritmo de Cocke-Younger-Kasami si $A\in V_{1,2}$ y $B\in V_{3,2}$ y $C\rightarrow AB$, podemos deducir que $C\in V_{1,4}$.
    \item Si $L$ es independiente del contexto, entonces $L^{-1}$ es independiente del contexto.
    \item No existe un algoritmo que nos diga si son iguales los lenguajes generados por dos gramáticas independientes del contexto $G_1$ y $G_2$.
    \item La intersección de dos lenguajes infinitos da lugar a un lenguaje independiente del contexto.
    \item La unión de dos lenguajes independientes del contexto puede ser aceptado por un autómata con pila.
    \item El lenuaje $L=\{0^i 1^j 2^k \mid 1 \leq i \leq j \leq k\}$ es independiente del contexto.
    \item Si $L_1$ y $L_2$ son independientes del contexto, no podemos asegurar que $L_1 \cap L_2$ también lo sea.
    \item Si un lenguaje satisface la condición necesaria del lema de bombeo para lenguajes regulares, entonces también tiene que satisfacer la condición necesaria del lema de bombeo para lenguajes independientes del contexto.  
\end{enumerate}

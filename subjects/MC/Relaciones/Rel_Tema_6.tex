\newpage
\section{Propiedades de los Lenguajes Independientes del Contexto}

\begin{ejercicio}\label{ej:1.6.1}
    Proporcione ejemplos de los siguientes lenguajes:
    \begin{enumerate}[label=\alph*)]
        \item Un lenguaje que no es independiente del contexto.
        \item Un lenguaje independiente del contexto pero no determinista.
        \item Un lenguaje que es independiente del contexto determinista, pero que no es aceptado por un autómata con pila determinista que tiene que vaciar su pila.
        \item Un lenguaje que es aceptado por un autómata con pila determinista que tiene que vaciar su pila, pero que no es un lenguaje regular.
    \end{enumerate}
\end{ejercicio}

\begin{ejercicio}\label{ej:1.6.2}
    Encontrar cuando sea posible, un autómata con pila que acepte el lenguaje $L$, donde:
    \begin{itemize}
        \item $L = \{ww^{-1}\mid w\in {\{a,b\}}^{\ast}\}$.
        \item $L = \{ww\mid w \in {\{a,b\}}^{\ast}\}$.
        \item $L = \{a^l b^m c^n \mid l + m = n\}$.
        \item $L = \{a^m b^n c^m \mid n\leq m\}$.
    \end{itemize}
\end{ejercicio}

\begin{ejercicio}\label{ej:1.6.3}
    Demostrar que los siguientes lenguajes no son libres de contexto:
    \begin{itemize}
        \item $L_1 = \{a^p \mid p \text{\ es primo}\}$.
        \item $L_2 = \{a^{n^2}\mid n\geq 1\}$.
    \end{itemize}
\end{ejercicio}

\begin{ejercicio}\label{ej:1.6.4}
    Encontrar un autómata con pila que acepte, por el criterio de pila vacía el lenguaje
    \begin{equation*}
        L = \{0^n uu^{-1}1^n \mid u\in {\{0,1\}}^{\ast}\}
    \end{equation*}
    Encontrar un autómata que acepte el lenguaje complementario.
\end{ejercicio}

\begin{ejercicio}\label{ej:1.6.5}
    Considerar la gramática libre de contexto dada por las siguientes producciones:
    \begin{align*}
        S &\rightarrow aABb\ |\ aBA\ |\ \veps \\
        A &\rightarrow aS\ |\ bAAA \\
        B &\rightarrow aABB\ |\ aBAB\ |\ aBBA\ |\ bS
    \end{align*}
    Determinar si las cadenas aabaab y las cadenas bbaaa son generadas por esta gramática
    \begin{enumerate}[label=\alph*)]
        \item Haciendo una búsqueda en el árbol de todas las derivaciones, pasando previamente la gramática a forma normal de Greibach.
        \item Mediante el algoritmo de Cocke-Younger-Kasami.
    \end{enumerate}
\end{ejercicio}

\begin{ejercicio}\label{ej:1.6.6}
    Determinar qué lenguajes son regulares y/o libres de contexto:
    \begin{itemize}
        \item $\{0^{n^2}\mid n\geq 1\}$.
        \item $\{0^n 1^n 0^n 1^n \mid n \geq 0\}$.
        \item $\{0^n 10^m 10^{n+m}\mid n,m\geq 0\}$.
        \item Conjunto de palabras en las que toda posición impar está ocupada por un 1.
    \end{itemize}
\end{ejercicio}

\begin{ejercicio}\label{ej:1.6.7}
    Comprobar, usando el algoritmo de Cocke-Younger-Kasami y el algoritmo de Early si las palabras $bba0d1$ y $cba1d1$ pertenecen al lenguaje generado por la gramática:
    \begin{align*}
        S &\rightarrow AaB\ |\ AaC \\
        A &\rightarrow Ab\ |\ Ac\ |\ b\ |\ c \\
        B &\rightarrow BdC\ |\ 0 \\
        C &\rightarrow CeB\ |\ 1
    \end{align*}
\end{ejercicio}

\begin{ejercicio}\label{ej:1.6.8}
    Dada la gramática:
    \begin{equation*}
        \begin{array}{llll}
            &S\rightarrow AB &S\rightarrow C & & \\
            &A\rightarrow aAb &A\rightarrow ab &B\rightarrow cBd &B\rightarrow cd \\
            &C\rightarrow aCd &C\rightarrow aDd &D\rightarrow bDc &D\rightarrow bc
        \end{array}
    \end{equation*}
    determinar mediante el algoritmo de Cocke-Younger-Kasami si las palabras $abbccd$ y $aabbcd$ son generadas.
\end{ejercicio}

\begin{ejercicio}\label{ej:1.6.9}
    Determinar si son regulares y/o independientes del contexto los siguientes lenguajes:
    \begin{enumerate}[label=\alph*)]
        \item $\{uu^{-1}u \mid u\in {\{0,1\}}^{\ast}\}$.
        \item $\{uu^{-1}ww^{-1}\mid u,w\in {\{0,1\}}^{\ast}\}$.
        \item $\{uu^{-1}w \mid u,w\in {\{0,1\}}^{\ast} \text{\ y\ } |u|\leq 3\}$
    \end{enumerate}
    Justificar las respuestas.
\end{ejercicio}

\begin{ejercicio}\label{ej:1.6.10}
    Construir una gramática independiente del contexto para el lenguaje más pequeño que verifica las siguientes reglas:
    \begin{enumerate}[label=\alph*)]
        \item Cualquier sucesión de dígitos de $\{0,1,2,\ldots, 9\}$ de longitud mayor o igual a 1 es una palabra del lenguaje.
        \item Si $u_1,\ldots,u_n$ ($n\geq 1$) son palabras del lenguaje, entonces $(u_1+\cdots + u_n)$ es una palabra del lenguaje.
        \item Si $u_1,\ldots,u_n$ ($n\geq 1$) son palabras del lenguaje, entonces $[u_1 \ast \cdots \ast u_n]$ es una palabra del lenguaje.
    \end{enumerate}
    Comprobar por el algoritmo de Cocke-Younger-Kasami si las palabras: $(0+1)*3$ y $[(0+1)]$ son generadas por la gramática.
\end{ejercicio}

\begin{ejercicio}\label{ej:1.6.11}
    Encuentra una gramática libre de contexto en forma normal de Chomsky que genere el siguiente lenguaje:
    \begin{equation*}
        L = \{ucv \mid u,v\in {\{0,1\}}^{+} \text{\ y nº de subcadenas '} 01 \text{' en\ } u \text{\ es igual al nº subcadenas '}10 \text{' en\ }v\}
    \end{equation*}
    Comprueba con el algoritmo CYK si la cadena $010c101$ pertenece al lenguaje generado por la gramática.
\end{ejercicio}

\begin{ejercicio}\label{ej:1.6.12}
    Encuentra una gramática libre de contexto en forma normal de Chomsky que genere el siguiente lenguaje definido sobre el alfabeto $\{a,0,1\}$:
    \begin{equation*}
        L = \{auava \mid u,v\in {\{0,1\}}^{\ast} \text{\ y\ } u=v^{-1}\}
    \end{equation*}
    Comprueba con el algoritmo CYK si la cadenas $a0a0a$ y $a1a0a$ pertenecen al lenguaje generado por la gramática.
\end{ejercicio}

\begin{ejercicio}\label{ej:1.6.13}
    Encuentra una gramática libre de contexto en forma normal de Chomsky que genere los siguientes lenguaje definidos sobre el alfabeto $\{a,0,1\}$:
    \begin{gather*}
        L_1 = \{auava \mid u,v \in {\{0,1\}}^{+} \text{\ y\ } u^{-1} = v\} \\
        L_2 = \{uvu \mid u\in {\{0,1\}}^{+}\text{\ y\ } u^{-1} = v\}
    \end{gather*}
    Comprueba con el algoritmo CYK si la cadena $a0a0a$ pertenece a $L_1$ y la cadena $011001$ pertenece al lenguaje $L_2$.
\end{ejercicio}

\begin{ejercicio}\label{ej:1.6.14}
    Sea la gramatica $G = (\{a,b\},\{S,A,B\},S,P)$ siendo $P$:
    \begin{align*}
        S&\rightarrow AabB \\
        A &\rightarrow aA\ |\ bA\ |\ \veps \\
        B&\rightarrow Bab\ |\ Bb\ |\ ab\ |\ b
    \end{align*}
    \begin{enumerate}[label=\alph*)]
        \item ¿Es regular el lenguaje que genera $G$?.
        \item Transforma $G$ a una gramatica equivalente en Forma Normal de Chomsky.
        \item Aplicando el algoritmo CYK, determinar si las siguientes cadenas pertenecen a $L(G)$: $aababb$, $aaba$.
        \item Muestra el arbol de derivacion para generar las palabras del apartado anterior que pertenecen a $L(G)$.
    \end{enumerate}
\end{ejercicio}

\begin{ejercicio}\label{ej:1.6.15}
    Dada la gramática:
    \begin{equation*}
        \begin{array}{lll}
            &S\rightarrow AB &S\rightarrow C &S\rightarrow BE \\
            &A\rightarrow aAb &A\rightarrow \veps &\\
            &B\rightarrow cBd &B\rightarrow\veps & \\
            &C\rightarrow aCd &C\rightarrow aDd & \\
            &D\rightarrow bDc &D\rightarrow\veps & 
        \end{array}
    \end{equation*}
    determinar mediante el algoritmo de Cocke-Younger-Kasami si las palabras $abbccd$ y $aabbcd$ son generadas por esta gramática.
\end{ejercicio}

\begin{ejercicio}\label{ej:1.6.16}
    Demostrar que si $L_1$ es independiente del contexto y $L_2$ es regular, entonces $L_1\cap L_2$ es independiente del contexto.
\end{ejercicio}

\begin{ejercicio}\label{ej:1.6.17}
    Si $L_1$ y $L_2$ son lenguajes sobre el alfabeto $A$, entonces se define el cociente $L_1/L_2 = \{u\in A^\ast \mid \exists w\in L_2 \text{\ tal que\ } uw\in L_1\}$ . Demostrar que si $L_1$ es independiente del contexto y $L_2$ regular, entonces $L_1/L_2$ es independiente dl contexto.
\end{ejercicio}

\begin{ejercicio}\label{ej:1.6.18}
    Si $L$ es un lenguaje sobre $\{0,1\}$, sea $SUF(L)$ el conjunto de los sufijos de palabras de $L$: 
    \begin{equation*}
        SUF(L) = \{u\in {\{0,1\}}^{\ast} \mid \exists v\in {\{0,1\}}^{\ast}, \text{\ tal que\ } vu\in L\}. Demostrar que si $L$ es independiente del contexto, entonces $SUF(L)$ también es independiente del contexto.
    \end{equation*}
\end{ejercicio}

\begin{ejercicio}\label{ej:1.6.19}
    Demostrar que $L=\{0^i 1^1 \mid i\geq 0\} \cup \{0^i1^{2i} \mid i\geq 0\}$ es independiente del contexto, pero no es determinista.
\end{ejercicio}

\subsection{Preguntas Tipo Test}
Indicar si son verdaderas o falsas las siguientes afirmaciones:
\begin{enumerate}
    \item La intersección de lenguajes libres de contexto es siempre libre de contexto.
    \item Existe un algoritmo para determinar si una palabra es generada por una gramática independiente del contexto.
    \item El lenguaje $\{a^i b^j c^i d^i \mid i,j\geq 0\}$ es independiente del contexto.
    \item Existe un algoritmo para determinar si una gramática independiente del contexto es ambigua.
    \item Existe un algoritmo para comprobar cuando dos gramáticas libres de contexto generan el mismo lenguaje.
    \item El lenguaje $L = \{0^i 1^j 2^k \mid i\leq i \leq j \leq k\}$ es independiente del contexto.
    \item Si el lenguaje $L$ es independiente del contexto, entonces $L^{-1}$ es independiente del contexto.
    \item Existe un algoritmo que permite determinar si una gramática independiente del contexto genera un lenguaje finito o infinito.
    \item Existe un algoritmo para determinar si una gramática independiente del contexto es ambigua.
    \item En el algoritmo de Earley, la presencia del registro $(2,5,A,CD,adS)$ implica que a partir de $CD$ se puede generar la subcadena de la palrba de entrada que va del carácter 3 al 5.
    \item Existe un algoritmo para comprobar si el lenguaje generado por una gramática libre de contexto es regular.
    \item El algoritmo de Earley se puede aplicar a cualquier gramática independiente del contexto (sin producciones nulas ni unitarias).
    \item El conjunto de palabras $\{a^n b^n c^i \mid i\leq n\}$ es independiente del contexto.
    \item Si $L_1$ y $L_2$ son independientes del contexto, entonces $L_1-L_2$ es siempre independiente del contexto.
    \item Hay lenguajes que no son independientes del contexto y si verifican la condición que aparece en el lema de bombeo para lenguajes independientes del contexto.
    \item El conjunto de palabras $\{u011u\mid u\in {\{0,1\}}^{\ast}\}$ es independiente del contexto.
    \item El conjunto de palabras que contienen la subcadena $011$ es independiente del contexto.
    \item En el algoritmo de Cocke-Younger-Kasami calculamos los conjuntos $V_{ij}$ que son las variables que generan la subcadena de la palabra de entrada que va desde el sı́mbolo en la posición $i$ al sı́mbolo en la posición $j$.
    \item Un lenguaje puede cumplir la negación de la condición que aparece en el lema de bombeo para lenguajes independientes del contexto y ser regular.
    \item Existe un algoritmo para comprobar si el lenguaje generado con una gramática independiente del contexto es finito o infinito.
    \item Si $L_1$ y $L_2$ son lenguajes independientes de contexto, entonces ${(L_1L_2 \cup L_1)}^{\ast}$ es independiente del contexto.
    \item Si $L_1$ y $L_2$ son lenguajes independientes de contexto, entonces $(L_1-L_2)$ es independiente del contexto.
    \item Existe un algoritmo para determinar si una palabra $u$ tiene más de un árbol de derivación en una gramática independiente del contexto $G$.
    \item La intersección de dos lenguajes independientes de contexto con un número finito de palabras produce siempre un lenguaje regular.
    \item El complementario de un lenguaje con un número finitos de palabras es siempre libre de contexto.
    \item Todo lenguaje aceptado por un autómata con pila por el criterio de estados finales cumple la condición que aparece en el lema de bombeo para lenguajes libres de contexto.
    \item No existe algoritmo que para toda gramática libre de contexto $G$ nos indique si el lenguaje generado por esta gramática $L(G)$ es finito o infinito.
    \item Si $L_1$ y $L_2$ son lenguajes independientes de contexto, entonces ${(L_1L_2\cup L_1)}^{\ast}$ puede ser representado por un autómata con pila.
    \item Existe un algoritmo para determinar si un autómata con pila es determinista.
    \item La demostración del lema de bombeo para lenguajes independientes del contexto se basa en que si las palabras superan una longitud determinada, entonces en el árbol de derivación debe de aparecer una variable como descendiente de ella misma.
    \item La unión de dos lenguajes independientes contexto puede ser siempre aceptada por un autómata con pila.
    \item El complementario de un lenguaje libre de contexto con una cantidad finita de palabras no tiene porque producir otro lenguaje libre de contexto.
    \item El lema de bombeo para lenguajes libres de contexto es útil para demostrar que un lenguaje determinado no es libre de contexto.
    \item La intersección de dos lenguajes independientes del contexto da lugar a un lenguaje aceptado por un autómata con pila determinista.
    \item No existe algoritmo que reciba como entrada una gramática independiente del contexto y nos devuelva si el lenguaje generado por esta gramática es finito o infinito.
    \item En el algoritmo de Cocke-Younger-Kasami si $A\in V_{1,2}$ y $B\in V_{3,2}$ y $C\rightarrow AB$, podemos deducir que $C\in V_{1,4}$.
    \item Si $L$ es independiente del contexto, entonces $L^{-1}$ es independiente del contexto.
    \item No existe un algoritmo que nos diga si son iguales los lenguajes generados por dos gramáticas independientes del contexto $G_1$ y $G_2$.
    \item La intersección de dos lenguajes infinitos da lugar a un lenguaje independiente del contexto.
    \item La unión de dos lenguajes independientes del contexto puede ser aceptado por un autómata con pila.
    \item El lenuaje $L=\{0^i 1^j 2^k \mid 1 \leq i \leq j \leq k\}$ es independiente del contexto.
    \item Si $L_1$ y $L_2$ son independientes del contexto, no podemos asegurar que $L_1 \cap L_2$ también lo sea.
    \item Si un lenguaje satisface la condición necesaria del lema de bombeo para lenguajes regulares, entonces también tiene que satisfacer la condición necesaria del lema de bombeo para lenguajes independientes del contexto.  
\end{enumerate}

\newpage
\section{Propiedades de los Lenguajes Regulares}

\begin{comment}
\begin{tikzpicture}
    \node[state, initial] (q1) {$q_1$};
    \node[state, accepting, right of=q1] (q2) {$q_2$};
    \node[state, right of=q2] (q3) {$q_3$};
    \draw (q1) edge[loop above] node{0} (q1)
    (q1) edge[above] node{1} (q2)
    (q2) edge[loop above] node{1} (q2)
    (q2) edge[bend left, above] node{0} (q3)
    (q3) edge[bend left, below] node{0, 1} (q2);
\end{tikzpicture}
\end{comment}

\begin{ejercicio}\label{ej:1.3.1}
    Determinar si los siguientes lenguajes son regulares o libres de contexto. Justificar las respuestas.
    \begin{enumerate}
        \item $\{0^i b^j \mid i = 2j \text{\ ó\ } 2i=j\}$
        \item $\{uu^{-1} \mid u \in {\{0,1\}}^\ast, |u|\leq 1000\}$
        \item $\{uu^{-1} \mid u \in {\{0,1\}}^\ast, |u|\geq 1000\}$
        \item $\{0^i 1^j 2^k \mid i = j \text{\ ó\ } j=k\}$
    \end{enumerate}
\end{ejercicio}

\begin{ejercicio}\label{ej:1.3.2}
    Determinar qué lenguajes son regulares o libres de contexto de los siguientes:
    \begin{enumerate}[label=\alph*)]
        \item $\{u0u^{-1}\mid u \in {\{0,1\}}^\ast\}$
        \item Números en binerio que sean múltiplos de 4
        \item Palabras de ${\{0,1\}}^\ast$ que no contienen la subcadena $0110$.
    \end{enumerate}
\end{ejercicio}

\begin{ejercicio}\label{ej:1.3.3}
    Determinar qué lenguajes son regulares y qué lenguajes son libres de contexto entre los siguientes:
    \begin{enumerate}[label=\alph*)]
        \item Conjunto de palabras sobre el alfabeto $\{0,1\}$ en las que cada 1 va precedido por un número par de ceros.
        \item Conjunto $\{0^i 1^2 j0^{i+j} \mid i,j\geq 0\}$
        \item Conjunto $\{0^i 1^j 0^{i\ast j}\mid i,j\geq 0\}$
    \end{enumerate}
\end{ejercicio}

\begin{ejercicio}\label{ej:1.3.4}
    Determina si los siguientes lenguajes son regulares. Encuentra una gramática que los genere o un reconocedor que los acepte.
    \begin{enumerate}[label=\alph*)]
        \item $L_1 = \{0^i 1^j \mid j < i\}$.
        \item $L_2 = \{001^i 0^j \mid i,j \geq 1\}$.
        \item $L_3 = \{010u \mid u \in {\{0,1\}}^{\ast}, u \text{ no contiene la subcadena } 010\}$.
    \end{enumerate}
\end{ejercicio}

\begin{ejercicio}\label{ej:1.3.5}
    Sea el alfabeto $A=\{0,1,+,=\}$, demostrar que el lenguaje
    \begin{equation*}
        ADD = \{x=y+z \mid x,y,z \text{ son números en binario, y } x \text{ es la suma de  } y \text{ y } z\}
    \end{equation*}
    no es regular.
\end{ejercicio}

\begin{ejercicio}\label{ej:1.3.6}
    Determinar si los siguientes lenguajes son regulares o no:
    \begin{enumerate}[label=\alph*)]
        \item $L=\{uvu^{-1} \mid u,v \in {\{0,1\}}^{\ast}\}$.
        \item $L$ es el lenguaje sobre el alfabeto $\{0,1\}$ formado de las palabras de la forma $u0v$ donde $u^{-1}$ es un prefijo de $v$.
        \item $L$ es el lenguaje sobre el alfebeto $\{0,1\}$ formado por las palabres en las que el tercer símbolo empezando por el final es un 1.
    \end{enumerate}
\end{ejercicio}

% // TODO: Ejercicio 7 está tachado

\begin{ejercicio}\label{ej:1.3.8}
    Dar una expresión regular para la intersección de los lenguajes asociados a las expresiones regulares ${(01+1)}^{\ast}0$ y ${(10+0)}^{\ast}$. Se valorará que se construya el autómata que acepta la intersección de estos lenguajes, se minimice y, a partir del resultado, se construya la expresión regular.
\end{ejercicio}

% // TODO: Ejercicio 9 está tachado

\begin{ejercicio}\label{ej:1.3.10}
    Encontrar un AFD minimal para el lenguaje
    \begin{equation*}
        {(a+b)}^{\ast}(aa+bb){(a+b)}^{\ast}
    \end{equation*}
\end{ejercicio}

\begin{ejercicio}\label{ej:1.3.11}
    Para cada uno de los siguientes lenguajes regulares, encontrar el autómata minimal asociado, y a partir de dicho autómata minimal, determinar la gramática regular que genera el lenguaje:
    \begin{enumerate}
        \item $a^+ b^+$
        \item $a{(a+b)}^{\ast}b$
    \end{enumerate}
\end{ejercicio}

% // TODO: Ejercicio 12 está tachado

\begin{ejercicio}\label{ej:1.3.13}
    Determinar autómatas minimales para los lenguajes $L(M_1) \cup L(M_2)$ y $L(M_1)\cap \overline{L(M_2)}$ donde,
    \begin{enumerate}
        \item $M_1 = (\{q_0, q_1, q_2, q_3\}, \{a,b,c\},\delta_1,q_0, \{q_2\})$ donde
            \begin{equation*}
                \begin{array}{c|cccc}
                    \delta_1 & q_0 & q_1 & q_2 & q_3 \\ 
                    \hline
                    a & q_1 & q_1 & q_3 & q_3 \\ 
                    b & q_2 & q_1 & q_1 & q_3 \\ 
                    c & q_3 & q_3 & q_0 & q_3 
                \end{array}
            \end{equation*}
        \item $M_2 = (\{q_0, q_1, q_2, q_3\}, \{a,b,c\},\delta_2,q_0, \{q_2\})$ donde
            \begin{equation*}
                \begin{array}{c|cccc}
                    \delta_2 & q_0 & q_1 & q_2 & q_3 \\ 
                    \hline
                    a & q_1 & q_1 & q_3 & q_3 \\ 
                    b & q_1 & q_2 & q_2 & q_3 \\ 
                    c & q_3 & q_3 & q_0 & q_3 
                \end{array}
            \end{equation*}
    \end{enumerate}
\end{ejercicio}

\begin{ejercicio}\label{ej:1.3.14}
    Dado el conjunto regular representado por la expresión regular $a^\ast b^\ast + b^\ast a^\ast$, construir un autómata finido determinístico minimal que lo acepte.
\end{ejercicio}

\begin{ejercicio}\label{ej:1.3.15}
    Sean los lenguajes:
    \begin{enumerate}
        \item $L_1={(01+1)}^{\ast}00$
        \item $L_2=01{(01+1)}^{\ast}$
    \end{enumerate}
    construir un autómata finito determinístico minimal que acepte el lenguaje $L_1 \setminus L_2$, a partir de autómatas que acepten $L_1$ y $L_2$.
\end{ejercicio}

\begin{ejercicio}\label{ej:1.3.16}
    Dados los alfabetos $A=\{0,1,2,3\}$ y $B=\{0,1\}$ y el homomorfismo $f$ de $A^\ast$ en $B^\ast$ dado por:
    \begin{enumerate}
        \item $f(0)=00$, $f(1)=01$, $f(2)=10$, $f(3)=11$
    \end{enumerate}
    Sea $L$ el conjunto de las palabras de $B^\ast$ en las que el número de símbolos 0 es par y el de símbolos 1 no es múltiplo de 3. Construir un autómata finito determinista que acepte el lenguaje $f^{-1}(L)$.
\end{ejercicio}

% // TODO: Ejercicio 17 está tachado

\begin{ejercicio}\label{ej:1.3.18}
    Determinar si las expresiones regulares siguientes representan el mismo lenguaje:
    \begin{enumerate}[label=\alph*)]
        \item ${(b+(c+a)a^\ast (b+c))}^{\ast} (c+a)a^\ast$
        \item $b^\ast (c+a) {((b+c)b^\ast (c+a))}^{\ast}a^\ast$
        \item $b^\ast (c+a){(a^\ast (b+c)b^\ast (c+a))}^{\ast}a^\ast$
    \end{enumerate}
    Justificar la respuesta.
\end{ejercicio}

\begin{ejercicio}\label{ej:1.3.19}
    Construir un autómata finito determinista minila que acepte el conjunto de palabras sobre el alfabeto $A=\{0,1\}$ que representen números no divisibles por dos ni por tres.
\end{ejercicio}

% // TODO: Ejercicios 20 y 21 tachados

\begin{ejercicio}\label{ej:1.3.22}
    \begin{enumerate}
        \item Construye una gramática regular que genere el siguiente lenguaje:
            \begin{equation*}
                L_1 = \{u\in {\{0,1\}}^{\ast} \mid \text{ el número de 1's y el número de 0's en } u \text{ es par }\}
            \end{equation*}
        \item Construye un autómata que reconozca el siguiente lenguaje:
            \begin{equation*}
                L_2 = \{0^n 1^m \mid n\geq 1, m\geq 0, n \text{ múltiplo de 3, } m \text{ par }\}
            \end{equation*}
            \item Diseña el AFD mínimo que reconoce el lenguaje $(L_1 \cup L_2)$.
    \end{enumerate}
\end{ejercicio}

% // TODO: Ejercicios 23, 24 y 25 tachados

\begin{ejercicio}\label{ej:1.3.26}
    Construir autómatas finitos para los siguientes lenguajes sobre el alfabeto $\{a,b,c\}$:
    \begin{enumerate}[label=\alph*)]
        \item $L_1$: palabras del lenguaje ${(a+b)}^{\ast}{(b+c)}^{\ast}$.
        \item $L_2$: palabras en las que nunca hay una 'a' posterior a una 'c'.
        \item $(L_1 \setminus L_2)\cup (L_2 \setminus L_1)$
    \end{enumerate}
    ¿Qué podemos concluir sobre $L_1$ y $L_2$?
\end{ejercicio}

\begin{ejercicio}\label{ej:1.3.27}
    Si $f:{\{0,1\}}^{\ast}\rightarrow{\{a,b,c\}}^{\ast}$ es un homomorfismo dao por
    \begin{equation*}
        f(0) = aab \qquad f(1) = bbc
    \end{equation*}
    dar autómatas finitos deterministas minimales para los lenguajes $L$ y $f^{-1}(L)$ donde $L\subseteq {\{a,b,c\}}^{\ast}$ es el lenguaje en el que el número de símbolos $a$ no es múltiplo de 4.
\end{ejercicio}

\begin{ejercicio}\label{ej:1.3.28}
    Si $L_1$ es el lenguaje asociadoa a la expresión regular $01{(01+1)}^{\ast}$ y $L_2$ el lenguaje asociado a la expresión ${(1+10)}^{\ast}01$, encontrar un autómata minimal que acepte el lenguaje $L_1\setminus L_2$.
\end{ejercicio}

\begin{ejercicio}\label{ej:1.3.29}
    Sean los alfabetos $A_1=\{a,b,c,d\}$ y $A_2=\{0,1\}$ y el lenguaje $L\subseteq A^\ast_2$ dado por la expresión regular ${(0+1)}^{\ast}0(0+1)$, calcular una expresión regular para el lenguaje $f^{-1}(L)$ donde $f$ es el homomorfismos entre $A^\ast_1$ y $A^\ast_2$ dado por
    \begin{equation*}
        f(a)=01 \qquad f(b) = 1 \qquad f(c)=0 \qquad f(d)=00
    \end{equation*}
\end{ejercicio}

% // TODO: EJercicio 30 está tachado

\begin{ejercicio}\label{ej:1.3.31}
    Dado el lenguaje $L$ asociado a la expresión regular ${(01+011)}^{\ast}$ y el homomorfismo $f:{\{0,1\}}^{\ast}\rightarrow{\{0,1\}}^{\ast}$ dado por $f(0)=01$, $f(1)=1$, construir una expresión regular para el lenguaje $f^{-1}(L)$.
\end{ejercicio}

\begin{ejercicio}\label{ej:1.3.32}
    Dar expresiones regulares para los siguientes lenguajes sobre el alfabeto $A_1=\{0,1,2\}$:
    \begin{enumerate}[label=\alph*)]
        \item $L$ dado por el conjunto de palabras en las que cada 0 que no sea el último de la palabra va seguido por un 1 y cada 1 que no sea el último símbolo de la palabra va seguido por un 0.
        \item Considera el homomorfismo de $A_1$ en $A_2=\{0,1\}$  dado por $f(0)=001$, $f(1)=100$, $f(2)=0011$. Dar una expresión regular para $f(L)$.
        \item Dar una expresión regular para $LL^{-1}$.
    \end{enumerate}
\end{ejercicio}

\begin{ejercicio}\label{ej:1.3.33}
    Dados los lenguajes
    \begin{equation*}
        L_1 = \{0^i 1^j \mid i\geq 1, j\text{ es par y } j\geq 2\}
    \end{equation*}
    y
    \begin{equation*}
        L_2 = \{1^j 0^k \mid k\geq 1, j\text{ es impar y } j\geq 1\}
    \end{equation*}
    encuentra:
    \begin{enumerate}[label=\alph*)]
        \item Una gramática regular que genere el lenguaje $L_1$.
        \item Una expresión regular que represente al lenguaje $L_2$.
        \item Un automata finito determinista que acepte las cadenas de la concatenación de los lenguajes $L_1L_2$. Aplica el algoritmos para minimizar este autómata.
    \end{enumerate}
\end{ejercicio}

\section{Introducción a la Computación}


\begin{ejercicio}
    Sea la gramática $G=\left(V,T,P,S\right)$ dada por:
    \begin{align*}
        V &= \{S, X, Y\} \\
        T &= \{a,b\} \\
        S &= S
    \end{align*}
    \begin{enumerate}
        \item Describe el lenguaje generado por la gramática teniendo en cuenta que $P$ viene descrito por:
        \begin{align*}
            S &\rightarrow XYX \\
            X &\rightarrow aX \mid bX \mid \veps \\
            Y &\rightarrow bbb
        \end{align*}

        Sea $L=\{ubbbv\mid u,v\in\{a,b\}^\ast\}$. Demostraremos mediante doble inclusión que $L=\cc{L}(G)$.
        \begin{description}
            \item[$\subset)$] Sea $w\in L$. Entonces, $w=ubbbv$ con $u,v\in\{a,b\}^\ast$. Veamos que
            $S \stackrel{\ast}{\Longrightarrow} w$:
            \begin{equation*}
                S \Longrightarrow XYX \Longrightarrow XbbbX
            \end{equation*}

            Además, es fácil ver que la regla de producción $X\rightarrow aX \mid bX \mid \veps$ nos permite generar cualquier palabra $u\in\{a,b\}^\ast$. Por tanto, tenemos que $X \stackrel{\ast}{\Longrightarrow} u$ y $X \stackrel{\ast}{\Longrightarrow} v$; teniendo así que $S \stackrel{\ast}{\Longrightarrow} ubbbv$.

            \item[$\supset)$] Sea $w\in\cc{L}(G)$. Veamos la forma de $w$:
            \begin{equation*}
                S \Longrightarrow XYX \Longrightarrow XbbbX \Longrightarrow ubbbv \mid u,v\in\{a,b\}^\ast
            \end{equation*}
            donde en el último paso hemos empleado lo visto en el apartado anterior de la regla de producción $X\rightarrow aX \mid bX \mid \veps$. Por tanto, $w\in L$.
        \end{description}


        \item Describe el lenguaje generado por la gramática teniendo en cuenta que $P$ viene descrito por:
        \begin{align*}
            S &\rightarrow aX \\
            X &\rightarrow aX \mid bX \mid \veps
        \end{align*}

        Sea $L=\{au \mid u\in\{a,b\}^\ast\}$. Demostraremos mediante doble inclusión que $L=\cc{L}(G)$.
        \begin{description}
            \item[$\subset)$] Sea $w\in L$. Entonces, $w=au$ con $u\in\{a,b\}^\ast$. Veamos que
            $S \stackrel{\ast}{\Longrightarrow} w$:
            \begin{equation*}
                S \Longrightarrow aX \Longrightarrow au
            \end{equation*}
            donde en el último paso hemos empleado lo visto respecto a la regla de producción $X\rightarrow aX \mid bX \mid \veps$. Por tanto, $w\in \cc{L}(G)$.

            \item[$\supset)$] Sea $w\in\cc{L}(G)$. Veamos la forma de $w$:
            \begin{equation*}
                S \Longrightarrow aX \Longrightarrow au \mid u\in\{a,b\}^\ast
            \end{equation*}
            donde en el último paso hemos empleado lo visto respecto a la regla de producción $X\rightarrow aX \mid bX \mid \veps$. Por tanto, $w\in L$.
        \end{description}

        \item Describe el lenguaje generado por la gramática teniendo en cuenta que $P$ viene descrito por:
        \begin{align*}
            S &\rightarrow XaXaX \\
            X &\rightarrow aX \mid bX \mid \veps
        \end{align*}

        Sea $L=\{uavaw \mid u,v,w\in\{a,b\}^\ast\}$. Demostraremos mediante doble inclusión que $L=\cc{L}(G)$.
        \begin{description}
            \item[$\subset)$] Sea $z\in L$. Entonces, $z=uavaw$ con $u,v,w\in\{a,b\}^\ast$. Veamos que
            $S \stackrel{\ast}{\Longrightarrow} z$:
            \begin{equation*}
                S \Longrightarrow XaXaX \Longrightarrow uavaw
            \end{equation*}
            donde en el último paso hemos empleado lo visto respecto a la regla de producción $X\rightarrow aX \mid bX \mid \veps$. Por tanto, $z\in \cc{L}(G)$.

            \item[$\supset)$] Sea $z\in\cc{L}(G)$. Veamos la forma de $z$:
            \begin{equation*}
                S \Longrightarrow XaXaX \Longrightarrow uavaw \mid u,v,w\in\{a,b\}^\ast
            \end{equation*}
            donde en el último paso hemos empleado lo visto respecto a la regla de producción $X\rightarrow aX \mid bX \mid \veps$. Por tanto, $z\in L$.
        \end{description}

        \item Describe el lenguaje generado por la gramática teniendo en cuenta que $P$ viene descrito por:
        \begin{align*}
            S &\rightarrow SS \mid XaXaX \mid \veps \\
            X &\rightarrow bX \mid \veps
        \end{align*}

        Sea el lenguaje $L=\{b^i a b^j a b^k \mid i,j,k\in \bb{N}\cup \{0\}\}$. Demostraremos mediante doble inclusión que $L^\ast=\cc{L}(G)$.
        \begin{description}
            \item[$\subset)$] Sea $z\in L^\ast=\bigcup\limits_{i\in \bb{N}} L^i$.
            Sea $n$ el menor número natural tal que $z\in L^n$.
            Notando por $n_a(z)$ al número de $a$'s en $z$, tenemos que $n_a(z)=2n$.
            Entonces, $z\in L\cdot \ldots \cdot L$ ($n$ veces), por lo que existen
            $i_1,j_1,k_1,\ldots,i_n,j_n,k_n\in \bb{N}\cup \{0\}$ tales que $z=b^{i_1} a b^{j_1} a b^{k_1} \cdot \ldots \cdot b^{i_n} a b^{j_n} a b^{k_n}$. Veamos que
            $S \stackrel{\ast}{\Longrightarrow} z$:
            \begin{itemize}
                \item Para conseguir el número de $a$'s deseado, empleamos la regla de producción $S \rightarrow SS$ y reemplazamos una de las $S$ por $XaXaX$. Esto lo hacemos $n$ veces.
                \item Posteriormente, cada $X$ la sustituiremos tantas veces como sea necesario por $bX$ para conseguir el número de $b$'s deseado en cada posición, y finalizaremos con $X\rightarrow \veps$.
            \end{itemize}

            \item[$\supset)$] Sea $z\in\cc{L}(G)$, y sea $n_a(z)$ el número de $a$'s en $z$. Entonces, como el número de $a$ siempre aumenta de dos en dos, tenemos que $n_a(z)=2n$ para algún $n\in \bb{N}\cup \{0\}$.
            Veamos la forma de $z$:
            \begin{itemize}
                \item Para llegar a $z$, hemos tenido que emplear la regla de producción $S \rightarrow SS\rightarrow SXaXaX$ $n$ veces. Una vez llegados aquí, para eliminar la $S$ (ya que habremos llegado a $n_a(z)$ $a$'s), empleamos la regla de producción $S\rightarrow \veps$.
                \item Posteriormente, para cada $X$, tan solo podemos emplear la regla de producción $X\rightarrow bX \mid \veps$ para conseguir el número de $b$'s deseado en cada posición.
            \end{itemize}
            Por tanto, es directo ver que $z\in L^n\subseteq L^\ast$.
        \end{description}
    \end{enumerate}
\end{ejercicio}



\begin{ejercicio} \label{ej:1.2}
    Sea la gramática $G=\left(V,T,P,S\right)$. Determinar en cada caso el lenguaje generado por la gramática.
    \begin{enumerate}
        \item Tenga en cuenta que:
        \begin{align*}
            V &= \{S,A\}\\
            T &= \{a,b\}\\
            S &= S \\
            P &= \left\{
                \begin{array}{rcl}
                    S &\rightarrow & abAS \mid a \\
                    abA &\rightarrow & baab \\
                    A &\rightarrow & b
                \end{array}
            \right\}
        \end{align*}

        Sea $L=\{ua \mid u\in \{abb, baab\}^\ast\}$. Demostraremos mediante doble inclusión que $L=\cc{L}(G)$.
        \begin{description}
            \item[$\subset)$] Sea $w\in L$. Entonces, $w=ua$ con $u\in \{abb, baab\}^\ast$. Veamos que
            $S \stackrel{\ast}{\Longrightarrow} w$. Para ello, sabemos que $u\in \{abb, baab\}^\ast=\bigcup\limits_{i\in \bb{N}} \{abb, baab\}^i$.
            Sea $n$ el menor número natural tal que $u\in \{abb, baab\}^n$, es decir, es una concatenación de $n$ subcadenas, cada una de las cuales es o bien $abb$ o bien $baab$. Veamos que $S$ produce ambas subcadenas:
            \begin{itemize}
                \item Para producir $abb$, tenemos que $S\rightarrow abAS \rightarrow abbS$.
                \item Para producir $baab$, tenemos que $S\rightarrow abAS \rightarrow baabS$.
            \end{itemize}
            Como vemos, en cada caso podemos concatenar la subcadena necesaria, pero siempre nos quedará una $S$ al final. Usamos la regla de producción $S\rightarrow a$ para eliminarla, llegando así a $w$, por lo que $S \stackrel{\ast}{\Longrightarrow} w$ y $w\in \cc{L}(G)$.

            \item[$\supset)$] Sea $w\in\cc{L}(G)$. Veamos la forma de $w$, para lo cual hay dos opciones:
            \begin{itemize}
                \item $S\rightarrow a$: En este caso, habremos finalizado la palabra con $a$, por lo que habremos añadido la subcadena $a$ a la palabra al final.
                \item $S \rightarrow abAS$: En este caso, también hay dos opciones:
                \begin{itemize}
                    \item $S \rightarrow abAS \rightarrow baabS$: En este caso, habremos concatenado $baab$ con $S$, por lo que habremos añadido la subcadena $baab$ a la palabra.
                    \item $S \rightarrow abAS \rightarrow abbS$: En este caso, habremos concatenado $abb$ con $S$, por lo que habremos añadido la subcadena $abb$ a la palabra.
                \end{itemize}
            \end{itemize}
            Por tanto, $w$ es de la forma $ua$ con $u$ una concatenación de $abb$'s y $baab$'s, es decir, $u\in\{abb, baab\}^\ast$.
            Por tanto, $w\in L$.
        \end{description}

        \item \label{ej:1.2.b} Tenga en cuenta que:
        \begin{align*}
            V &= \{\langle \text{número} \rangle, \langle \text{dígito} \rangle\} \\
            T &= \{0,1,2,3,4,5,6,7,8,9\} \\
            S &= \langle \text{número} \rangle \\
            P &= \left\{
                \begin{array}{rcl}
                    \langle \text{número} \rangle &\rightarrow & \langle \text{número} \rangle \langle \text{dígito} \rangle \\
                    \langle \text{número} \rangle &\rightarrow & \langle \text{dígito} \rangle \\
                    \langle \text{dígito} \rangle &\rightarrow & 0 \mid 1 \mid 2 \mid 3 \mid 4 \mid 5 \mid 6 \mid 7 \mid 8 \mid 9
                \end{array}
            \right\}
        \end{align*}

        Tenemos que $\cc{L}(G)$ es el conjunto de los números naturales, permitiendo
        tantos ceros a la izquierda como se quiera. Es decir (usando la notación de potencia y concatenación vista para lenguajes):
        \begin{equation*}
            L = \{0^i n \mid i\in \bb{N}\cup\{0\},~n\in \bb{N}\cup \{0\}\}
        \end{equation*}
        Demostrémoslo mediante doble inclusión que $L=\cc{L}(G)$.
        \begin{description}
            \item[$\subset)$] Sea $w\in L$. Entonces, $w=0^i n$ con $i\in \bb{N}\cup\{0\}$ y $n\in \bb{N}\cup \{0\}$. Veamos que
            $\langle \text{número} \rangle \stackrel{\ast}{\Longrightarrow} w$:
            \begin{itemize}
                \item En primer lugar, aplicamos $|w|-1$ veces la regla de producción $\langle \text{número} \rangle \rightarrow \langle \text{número} \rangle \langle \text{dígito} \rangle$ y la regla
                que lleva de $\langle \text{dígito} \rangle$ a uno de los símbolos terminales, consiguiendo así en cada etapa reemplazar
                la última variable presente en la cadena por un dígito.
                \item Finalmente, aplicamos la regla de producción $\langle \text{número} \rangle \rightarrow \langle \text{dígito} \rangle$ para reemplazar la última variable por un dígito, que será el primero del número formado.
            \end{itemize}
            Por tanto, $\langle \text{número} \rangle \stackrel{\ast}{\Longrightarrow} w$, teniendo que $w\in \cc{L}(G)$.

            \item[$\supset)$] Sea $w\in\cc{L}(G)$. Como la única regla que
            aumenta la longitud es la regla de producción $\langle \text{número} \rangle \rightarrow \langle \text{número} \rangle \langle \text{dígito} \rangle$, tenemos que $w$ tiene la forma:
            \begin{align*}
                \langle \text{número} \rangle &\Longrightarrow \langle \text{número} \rangle \langle \text{dígito} \rangle \stackrel{|w|-1\text{\ veces}}{\Longrightarrow} \\
                &\Longrightarrow
                \langle \text{número} \rangle \langle \text{dígito} \rangle \langle \text{dígito} \rangle \stackrel{|w|-1\text{\ veces}}{\cdots} \langle \text{dígito} \rangle
                \Longrightarrow \\& \Longrightarrow
                \langle \text{dígito} \rangle \stackrel{|w|\text{\ veces}}{\cdots} \langle \text{dígito} \rangle
            \end{align*}
            Por tanto, tenemos que se trata una sucesión de $|w|$ dígitos, lo que nos lleva a que $w\in L$.
        \end{description}

        \item Tenga en cuenta que:
        \begin{align*}
            V &= \{A,S\} \\
            T &= \{a,b\} \\
            S &= S \\
            P &= \left\{
                \begin{array}{rcl}
                    S &\rightarrow & aS \mid aA \\
                    A &\rightarrow & bA \mid b
                \end{array}
            \right\}
        \end{align*}

        Sea $L=\{a^nb^m \in \{a,b\}^\ast \mid n, m \in \bb{N}\}$. Demostraremos mediante doble inclusión que $L=\cc{L}(G)$.
        \begin{description}
            \item[$\subset)$] Sea $w\in L$. Entonces, $w=a^nb^m$ con $n,m\in \bb{N}$. Veamos que
            $S \stackrel{\ast}{\Longrightarrow} w$:
            \begin{itemize}
                \item En primer lugar, aplicamos $n-1$ veces la regla de producción $S \rightarrow aS$ para obtener $a^{n-1}S$,
                \begin{equation*}
                    S \stackrel{\ast}{\Longrightarrow} a^{n-1}S
                \end{equation*}

                \item Para cambiar a la etapa de añadir $b$'s, aplicamos la regla de producción $S \rightarrow aA$, obteniendo así $a^{n}A$,
                \item Después, aplicamos $m-1$ veces la regla de producción $A \rightarrow bA$ para obtener $a^nb^{m-1}A$.
                \item Para finalizar, aplicamos la regla de producción $A \rightarrow b$ para obtener $a^nb^m$.
            \end{itemize}
            Por tanto, $S \stackrel{\ast}{\Longrightarrow} w$, teniendo que $w\in \cc{L}(G)$.

            \item[$\supset)$] Sea $w\in\cc{L}(G)$. Vemos que en la palabra siempre
            va a haber tan solo una variable (ya sea $S$ o $A$). Se empezará con la $S$, y en cierto momento se cambiará a la $A$,
            sin poder entonces volver a la $S$.
            \begin{itemize}
                \item Cuando se está en la etapa en la que hay $S$, tan solo se pueden añadir $a$'s,
                o bien cambiar a la $A$.
                \item Cuando se está en la etapa en la que hay $A$, tan solo se pueden añadir $b$'s.
            \end{itemize}
            Por tanto, tenemos que $w$ estará formada por una sucesión de
            $a$'s seguida de una sucesión de $b$'s, lo que nos lleva a que $w\in L$.
        \end{description}
    \end{enumerate}
\end{ejercicio}

\begin{ejercicio}
    Encontrar gramáticas de tipo 2 para los siguientes lenguajes sobre el alfabeto $\{a, b\}$. En cada caso determinar si los lenguajes generados son de tipo 3, estudiando si existe una gramática de tipo 3 que los genera.
    \begin{enumerate}
        \item Palabras en las que el número de $b$ no es tres.
        
        Tenemos varias opciones:
        \begin{itemize}
            \item Que no tenga $b$'s.
            \item Que tenga una $b$.
            \item Que tenga dos $b$'s.
            \item Que tenga $4$ o más $b$'s.
        \end{itemize}

        Sea la gramática $G=\left(V,T,P,S\right)$ dada por:
        \begin{align*}
            V &= \{S, A, X\} \\
            T &= \{a,b\} \\
            S &= S \\
            P &= \left\{
                \begin{array}{rcl}
                    S &\rightarrow & A \mid AbA \mid AbAbA \mid XbXbXbXbX \\
                    A &\rightarrow & aA \mid \veps \\
                    X &\rightarrow & aX \mid bX \mid \veps
                \end{array}
            \right\}
        \end{align*}

        Esta gramática no obstante es de tipo $2$. Busquemos otra que sea de tipo 3.
        Sea la gramática $G'=\left(V',T',P',S'\right)$ dada por:
        \begin{align*}
            V' &= \{S, X,Y,Z, W\} \\
            T' &= \{a,b\} \\
            S' &= S \\
            P' &= \left\{
                \begin{array}{rcl}
                    S &\rightarrow & \veps \mid aS \mid bX \\
                    X &\rightarrow & \veps \mid aX \mid bY \\
                    Y &\rightarrow & \veps \mid aY \mid bZ \\
                    Z &\rightarrow & aZ \mid bW \\
                    W &\rightarrow & \veps \mid aW \mid bW
                \end{array}
            \right\}
        \end{align*}

        Esta sí es de tipo $3$, y genera el lenguaje deseado.



        \item Palabras que tienen 2 ó 3 $b$.
        
        Sea la gramática $G=\left(V,T,P,S\right)$ dada por:
        \begin{align*}
            V &= \{S, A, B\} \\
            T &= \{a,b\} \\
            S &= S \\
            P &= \left\{
                \begin{array}{rcl}
                    S &\rightarrow & AbAbABA \\
                    A &\rightarrow & aA \mid \veps \\
                    B &\rightarrow & b \mid \veps
                \end{array}
            \right\}
        \end{align*}

        Esta gramática no obstante es de tipo $2$. Busquemos otra que sea de tipo 3.
        Sea la gramática $G'=\left(V',T',P',S'\right)$ dada por:
        \begin{align*}
            V' &= \{S, X,Y,Z,W,V,T\} \\
            T' &= \{a,b\} \\
            S' &= S \\
            P' &= \left\{
                \begin{array}{rcl}
                    S &\rightarrow & aS \mid X \\
                    X &\rightarrow & bY \\
                    Y &\rightarrow & aY \mid Z \\
                    Z &\rightarrow & bW \\
                    W &\rightarrow & aW \mid \veps \mid V \\
                    V &\rightarrow & bT \\
                    T &\rightarrow & aT \mid \veps
                \end{array}
            \right\}
        \end{align*}

        Esta gramática ya es de tipo $3$, pero contiene un número elevado de variables. Veamos si podemos reducirlo:
        Sea la gramática $G''=\left(V'',T'',P'',S''\right)$ dada por:
        \begin{align*}
            V'' &= \{S, X,Y,Z\} \\
            T'' &= \{a,b\} \\
            S'' &= S \\
            P'' &= \left\{
                \begin{array}{rcl}
                    S &\rightarrow & aS \mid bX \\
                    X &\rightarrow & aX \mid bY \\
                    Y &\rightarrow & aY \mid \veps \mid bZ \\
                    Z &\rightarrow & aZ \mid \veps
                \end{array}
            \right\}
        \end{align*}

        Notemos que, en esta gramática de tipo $3$, ya hemos conseguido el menor número de variables posibles, que representan las $4$ etapas. Como la última es opcional, está la regla $Y\rightarrow \veps$, para así no agregar la tercera $b$.

    \end{enumerate}
\end{ejercicio}

\begin{ejercicio}
    Encontrar gramáticas de tipo 2 para los siguientes lenguajes sobre el alfabeto $\{a, b\}$. En cada caso determinar si los lenguajes generados son de tipo 3, estudiando si existe una gramática de tipo 3 que los genera.
    \begin{enumerate}
        \item Palabras que no contienen la subcadena $ab$.
        
        Sea la gramática $G=\left(V,T,P,S\right)$ dada por:
        \begin{align*}
            V &= \{S, A\} \\
            T &= \{a,b\} \\
            S &= S \\
            P &= \left\{
                \begin{array}{rcl}
                    S &\rightarrow & aA \mid bS \mid \veps \\
                    A &\rightarrow & aA \mid \veps \\
                \end{array}
            \right\}
        \end{align*}

        Notemos además que esta gramática es de tipo $3$, y se tiene que:
        \begin{equation*}
            \cc{L}(G) = \{b^i a^j \mid i,j\in \bb{N}\cup \{0\}\}
        \end{equation*}


        \item Palabras que no contienen la subcadena $baa$.
        
        Sea la gramática $G=\left(V,T,P,S\right)$ dada por:
        \begin{align*}
            V &= \{S, B\} \\
            T &= \{a,b\} \\
            S &= S \\
            P &= \left\{
                \begin{array}{rcl}
                    S &\rightarrow & aS \mid bB \mid \veps \\
                    B &\rightarrow & bB \mid abB \mid a \mid \veps
                \end{array}
            \right\}
        \end{align*}
        Notemos además que esta gramática es de tipo $3$.
    \end{enumerate}
\end{ejercicio}

\begin{ejercicio}
    Encontrar una gramática libre de contexto que genere el lenguaje sobre el alfabeto $\{a, b\}$ de las palabras que tienen más $a$ que $b$ (al menos una más).

    Sea la gramática $G=\left(V,T,P,S\right)$ dada por:
    \begin{align*}
        V &= \{S, S'\} \\
        T &= \{a,b\} \\
        S &= S \\
        P &= \left\{
            \begin{array}{rcl}
                S &\rightarrow & S'aS'\\
                S' &\rightarrow & S'aS' \mid aS'bS' \mid bS'aS' \mid \veps
            \end{array}
        \right\}
    \end{align*}
\end{ejercicio}

\begin{ejercicio}
    Encontrar, si es posible, una gramática regular (o, si no es posible, una gramática libre del contexto) que genere el lenguaje $L$ supuesto que $L \subset \{a, b\}^\ast$ y verifica:
    \begin{enumerate}
        \item $u \in L$ si, y solamente si, verifica que $u$ no contiene dos símbolos $b$ consecutivos.
        
        Sea la gramática $G=\left(V,T,P,S\right)$ dada por:
        \begin{align*}
            V &= \{S\} \\
            T &= \{a,b\} \\
            S &= S \\
            P &= \left\{
                \begin{array}{rcl}
                    S &\rightarrow & aS \mid baS \mid b\mid \veps
                \end{array}
            \right\}
        \end{align*}
        \item $u \in L$ si, y solamente si, verifica que $u$ contiene dos símbolos $b$ consecutivos.
        
        Sea la gramática $G=\left(V,T,P,S\right)$ dada por:
        \begin{align*}
            V &= \{S, B, F\} \\
            T &= \{a,b\} \\
            S &= S \\
            P &= \left\{
                \begin{array}{rcl}
                    S &\rightarrow & aS \mid bB \\
                    B &\rightarrow & bF \mid aS \\
                    F &\rightarrow & aF \mid bF \mid \veps
                \end{array}
            \right\}
        \end{align*}
        Notemos que, en este caso, tenemos tres estados:
        \begin{itemize}
            \item $S$: No hemos encontrado dos $b$'s consecutivas.
            \item $B$: Hemos encontrado una $b$, y puede ser que nos encontremos la segunda $b$.
            \item $F$: Hemos encontrado dos $b$'s consecutivas; ya hay libertad.
        \end{itemize}

        Sí es cierto que usamos tres variables. Para usar solo dos variables,
        podemos hacer lo siguiente.
        Sea la gramática $G'=\left(V',T',P',S'\right)$ dada por:
        \begin{align*}
            V' &= \{S, X\} \\
            T' &= \{a,b\} \\
            S' &= S \\
            P' &= \left\{
                \begin{array}{rcl}
                    S &\rightarrow & aS \mid bS \mid bbX \\
                    X &\rightarrow & aX \mid bX \mid \veps
                \end{array}
            \right\}
        \end{align*}
    \end{enumerate}
\end{ejercicio}

\begin{ejercicio}
    Encontrar, si es posible, una gramática regular (o, si no es posible, una gramática libre del contexto) que genere el lenguaje $L$ supuesto que $L \subset \{a, b\}^\ast$ y verifica:
    \begin{enumerate}
        \item $u \in L$ si, y solamente si, verifica que contiene un número impar de símbolos $a$.
        
        Sea la gramática $G=\left(V,T,P,S\right)$ dada por:
        \begin{align*}
            V &= \{S, X\} \\
            T &= \{a,b\} \\
            S &= S \\
            P &= \left\{
                \begin{array}{rcl}
                    S &\rightarrow & aX \mid bS\\
                    X &\rightarrow & aS \mid bX \mid \veps
                \end{array}
            \right\}
        \end{align*}
        \item $u \in L$ si, y solamente si, verifica que no contiene el mismo número de símbolos $a$ que de símbolos $b$.
        
        Sea la gramática $G=\left(V,T,P,S\right)$ dada por:
        \begin{align*}
            V &= \{S, A, B, X\} \\
            T &= \{a,b\} \\
            S &= S \\
            P &= \left\{
                \begin{array}{rcl}
                    S &\rightarrow & AaA \mid BbB \\
                    A &\rightarrow & AaA\mid X \\
                    B &\rightarrow & BbB \mid X \\
                    X &\rightarrow & aXbX \mid bXaX \mid \veps
                \end{array}
            \right\}
        \end{align*}
    \end{enumerate}
\end{ejercicio}


\begin{ejercicio}
    Dado el alfabeto $A = \{a, b\}$ determinar si es posible encontrar una gramática libre de contexto que:
    \begin{enumerate}
        \item Genere las palabras de longitud impar, y mayor o igual que 3, tales que la primera letra coincida con la letra central de la palabra.
        
        Sea la gramática $G=\left(V,T,P,S\right)$ dada por:
        \begin{align*}
            V &= \{S, X, A, B, C, D\} \\
            T &= \{a,b\} \\
            S &= S \\
            P &= \left\{
                \begin{array}{rcl}
                    S &A\mid B \\
                    A &\rightarrow & aCX \\
                    C & \rightarrow & a \mid XCX \\
                    B &\rightarrow & bDX \\
                    D & \rightarrow & b \mid XDX \\
                    X &\rightarrow & a\mid b
                \end{array}
            \right\}
        \end{align*}
        \item Genere las palabras de longitud par, y mayor o igual que 2, tales que las dos letras centrales coincidan.
        
        Sea la gramática $G=\left(V,T,P,S\right)$ dada por:
        \begin{align*}
            V &= \{S, X\} \\
            T &= \{a,b\} \\
            S &= S \\
            P &= \left\{
                \begin{array}{rcl}
                    S &\rightarrow & XSX\mid C \\
                    C &\rightarrow & aa \mid bb \\
                    X &\rightarrow & a \mid b
                \end{array}
            \right\}
        \end{align*}
    \end{enumerate}
\end{ejercicio}

\begin{ejercicio}
    Sea la gramática $G=\left(V,T,P,S\right)$ dada por:
    \begin{align*}
        V &= \{S, X\} \\
        T &= \{a,b\} \\
        S &= S\\
        P &= \left\{
            \begin{array}{rcl}
                S &\rightarrow & SS \\
                S &\rightarrow & XXX \\
                X &\rightarrow & aX \mid Xa \mid b
            \end{array}
        \right\}
    \end{align*}
    Determinar si el lenguaje generado por la gramática es regular. Justificar la respuesta.\\

    Sea la siguiente gramática regular $G'=\left(V',T',P',S'\right)$ dada por:
    \begin{align*}
        V' &= \{S, X,Y,Z\} \\
        T' &= \{a,b\} \\
        S' &= S \\
        P' &= \left\{
            \begin{array}{rcl}
                S &\rightarrow & aS \mid bX \\
                X &\rightarrow & aX \mid bY \\
                Y &\rightarrow & aY \mid bZ \\
                Z &\rightarrow & aZ \mid bX \mid \veps
            \end{array}
        \right\}
    \end{align*}

    Tenemos que $\cc{L}(G) = \cc{L}(G')$, y como $G'$ es una gramática regular, tenemos que $\cc{L}(G)$ es regular.
    Sí es cierto que en el tema $2$ aprendemos otras maneras de demostrarlo más sencillas, como buscar un autómata finito que lo genere.
\end{ejercicio}

\begin{ejercicio}
    Dado un lenguaje $L$ sobre un alfabeto $A$, ¿es $L^{\ast}$ siempre numerable? ¿nunca lo es? ¿o puede serlo unas veces sí y otras, no? Pon ejemplos en este último caso.\\

    $L^{\ast}$ es siempre numerable, veámos por qué. $L^{\ast}$ es un lenguaje sobre el alfabeto $A$, por lo que $L^{\ast}\subseteq A^{\ast}$ y $A^{\ast}$ es numerable (visto en teoría), luego $L^{\ast}$ también lo es.
\end{ejercicio}

\begin{ejercicio}
    Dado un lenguaje $L$ sobre un alfabeto $A$, caracterizar cuando $L^{\ast} = L$. Esto es, dar un conjunto de propiedades sobre $L$ de manera que $L$ cumpla esas propiedades si y sólo si $L^{\ast} = L$.

    \begin{equation*}
        L = L^{\ast} \Longleftrightarrow \left\{
            \begin{array}{cl}
                \veps \in L \\ \land \\ u,v \in L & \Longrightarrow uv\in L
            \end{array}
        \right.
    \end{equation*}
    Es decir, $L=L^{\ast}$ si y solo si la cadena vacía está en $L$ y además es cerrado para concatenaciones.

    \begin{proof} Demostramos mediante doble implicación.
        \begin{description}
            \item [$\Longleftarrow)$] La inclusión $L\subseteq L^{\ast}$ es obvia, por lo que solo falta demostrar la otra inclusión.\\

                Sea $v\in L^{\ast}$:
                \begin{enumerate}
                    \item Si $v = \veps \Longrightarrow v\in L$ por hipótesis.
                    \item Si $v\neq \veps$, $\exists n\in \mathbb{N}$ tal que 
                        \begin{equation*}
                            v = a_1 a_2 \ldots a_n
                        \end{equation*}
                        con $a_i \in L$ $\forall i \in \{1, \ldots, n\}$, de donde tenemos que $v\in L$, por ser cerrado para concatenaciones. Luego $L^{\ast}\subseteq L$.
                \end{enumerate}
            \item [$\Longrightarrow)$] Hemos de probar dos cosas:
                \begin{enumerate}
                    \item $\veps \in L^{\ast}=L$.
                    \item Sean $u,v\in L=L^{\ast} \Longrightarrow uv\in L^{\ast}=L$.
                \end{enumerate}
        \end{description}
    \end{proof}
\end{ejercicio}

\begin{ejercicio}
    Dados dos homomorfismos $f : A^{\ast} \rightarrow B^{\ast}$, $g : A^{\ast} \rightarrow B^{\ast}$, se dice que son iguales si $f(x) = g(x)$, $\forall x \in A^{\ast}$. ¿Existe un procedimiento algorítmico para comprobar si dos homomorfismos son iguales?\\

    Sí, basta probar que su imagen coincide sobre un conjunto finito de elementos, los de $A$:
    \begin{equation*}
        f(x) = g(x) \quad \forall x\in A^{\ast} \Longleftrightarrow f(a)=g(a) \quad \forall a\in A
    \end{equation*}
    \begin{proof}\ 
        \begin{description}
            \item [$\Longleftarrow)$] Sea $v\in A^{\ast}$, $\exists n\in \mathbb{N}$ tal que $v=a_1a_2\ldots a_n$ con $a_i \in A$ $\forall i \in \{1,\ldots, n\}$
                \begin{equation*}
                    f(v) = f(a_1)f(a_2)\ldots f(a_n) = g(a_1)g(a_2)\ldots g(a_n) = g(v)
                \end{equation*}
            \item [$\Longrightarrow)$] Sea $a\in A \Longrightarrow a\in A^{\ast}\Longrightarrow f(a)=g(a)$.
        \end{description}
    \end{proof}
\end{ejercicio}

\begin{ejercicio}
    Sea $L \subseteq A^{\ast}$ un lenguaje arbitrario. Sea $C_0 = L$ y definamos los lenguajes $S_i$ y $C_i$, para todo $i \geq 1$, por $S_i = C_{i-1}^+$ y $C_i = \ol{S_i}$. 
    \begin{enumerate}
        \item ¿Es $S_1$ siempre, nunca o a veces igual a $C_2$? Justifica la respuesta.
        \item Demostrar que $S_2 = C_3$, cualquiera que sea $L$.
        \begin{observacion}
            Demuestra que $C_2$ es cerrado para la concatenación.
        \end{observacion}
    \end{enumerate}
    % // TODO: Hacer JJ
\end{ejercicio}

\begin{ejercicio}
    Demuestra que, para todo alfabeto $A$, el conjunto de los lenguajes finitos sobre dicho alfabeto es numerable.

    Sea $A=\{a_1, a_2, \ldots, a_n\}$, con $n\in \mathbb{N}$. Definimos el siguiente conjunto:
    \begin{equation*}
        \Gamma = \{L\subseteq A^{\ast} \mid L \text{\ es finito}\}
    \end{equation*}

    Dado un símbolo $z\notin A$, definimos el conjunto $B=\{z\}\cup A$. Sea $B^{\ast}$ numerable, y buscamos una inyección de $\Gamma$ en $B^{\ast}$.
    Dado un lenguaje $L\in \Gamma$, sea $L=\{l_1, l_2, \ldots, l_m\}$, con $m\in \mathbb{N}$ y $l_i\in A^{\ast}$ $\forall i\in \{1, \ldots, m\}$. Definimos la siguiente función:
    \Func{f}{\Gamma}{B^\ast}{L}{zl_1zl_2\ldots zl_mz}

    Veamos que $f$ es inyectiva. Sean $L_1, L_2\in \Gamma$ tales que $f(L_1)=f(L_2)$. Entonces,
    \begin{equation*}
        zl_1zl_2\ldots zl_kz = zl'_1zl'_2\ldots zl'_{k'}z
    \end{equation*}
    Por ser ambas palabras iguales, tenemos que $k=k'$ y $l_i=l'_i$ $\forall i\in \{1, \ldots, k\}$, de donde $L_1=L_2$. Por tanto, $f$ es inyectiva, por lo que $\Gamma$ es inyectivo con un subconjunto de $B^{\ast}$, que es numerable. Por tanto, $\Gamma$ es numerable.
\end{ejercicio}




\subsection{Cálculo de gramáticas}

\begin{ejercicio}[Complejidad: Sencilla]
    Calcula, de forma razonada, gramáticas que generen cada uno de los siguientes lenguajes:
    \begin{enumerate}
        \item $\{ u\in \{0,1\}^\ast \mid |u|\leq 4 \}$
        
        Sea la gramática $G=\left(V,T,P,S\right)$ dada por:
        \begin{align*}
            V &= \{S, X\} \\
            T &= \{0,1\} \\
            S &= S \\
            P &= \left\{
                \begin{array}{rcl}
                    S &\rightarrow & XXXX \\
                    X &\rightarrow & 0 \mid 1 \mid \veps
                \end{array}
            \right\}
        \end{align*}

        No obstante, esta gramática es de tipo $2$. Busquemos una de tipo $3$.
        Sea la gramática $G'=\left(V',T',P',S'\right)$ dada por:
        \begin{align*}
            V' &= \{S, X, Y, Z\} \\
            T' &= \{0,1\} \\
            S' &= S \\
            P' &= \left\{
                \begin{array}{rcl}
                    S &\rightarrow & 0X \mid 1X \mid \veps \\
                    X &\rightarrow & 0Y \mid 1Y \mid \veps \\
                    Y &\rightarrow & 0Z \mid 1Z \mid \veps \\
                    Z &\rightarrow & 0 \mid 1
                \end{array}
            \right\}
        \end{align*}
        Tenemos que $\cc{L}(G) = \cc{L}(G')$, y es igual al lenguaje deseado. Tenemos por tanto que es un lenguaje regular.


        \item Palabras con 0's y 1's que no contengan dos 1's consecutivos y que empiecen por un 1 y que terminen por dos 0's.
        
        Sea la gramática $G=\left(V,T,P,S\right)$ dada por:
        \begin{align*}
            V &= \{S, X, Y\} \\
            T &= \{0,1\} \\
            S &= S \\
            P &= \left\{
                \begin{array}{rcl}
                    S &\rightarrow & 1X00 \\
                    X &\rightarrow & 0Y \mid \veps \\
                    Y &\rightarrow & 0Y \mid 1X \mid \veps \\
                \end{array}
            \right\}
        \end{align*}

        Notemos que esta gramática es de tipo 2 debido a la primera regla de producción. Busquemos una de tipo 3. 
        Sea la gramática $G'=\left(V',T',P',S'\right)$ dada por:
        \begin{align*}
            V' &= \{S, X, Y\} \\
            T' &= \{0,1\} \\
            S' &= S \\
            P' &= \left\{
                \begin{array}{rcl}
                    S &\rightarrow & 1X \\
                    X &\rightarrow & 0Y \mid F \\
                    Y &\rightarrow & 0Y \mid 1X \mid F \\
                    F &\rightarrow & 00
                \end{array}
            \right\}
        \end{align*}

        Tenemos que $\cc{L}(G) = \cc{L}(G')$, y es igual al lenguaje deseado. Tenemos por tanto que es un lenguaje regular. En esta última gramática, tenemos los siguientes estados:
        \begin{itemize}
            \item $S$: Es el estado inicial, empezamos con un $1$.
            \item $X$: Acabamos de escribir un $1$, por lo que ahora tan solo podemos escribir $0$'s.
            \item $Y$: Acabamos de escribir un $0$, por lo que ahora podemos escribir tanto $0$'s como $1$'s.
            \item $F$: Ya hemos terminado, y escribimos los dos $0$'s finales por la restricción impuesta.
        \end{itemize}
        

        \item El conjunto vacío.
        
        Sea la gramática $G=\left(V,T,P,S\right)$ dada por:
        \begin{align*}
            V &= \{S\} \\
            T &= \emptyset \\
            S &= S \\
            P &= \left\{
                \begin{array}{rcl}
                    S &\rightarrow & S
                \end{array}
            \right\}
        \end{align*}

        \item El lenguaje formado por los números naturales.
        
        Sea la gramática $G=\left(V,T,P,S\right)$ dada por:
        \begin{align*}
            V &= \{\langle \text{número no iniciado} \rangle, \langle \text{dígito no cero} \rangle, \langle \text{dígito} \rangle, \langle \text{número iniciado} \rangle\} \\
            T &= \{0,1,2,3,4,5,6,7,8,9\} \\
            S &= \langle \text{número no iniciado} \rangle \\
            P &= \left\{
                \begin{array}{rcl}
                    \langle \text{número no iniciado} \rangle &\rightarrow & \langle \text{dígito no cero} \rangle \mid \langle \text{dígito no cero} \rangle \langle \text{número iniciado} \rangle \\
                    \langle \text{número iniciado} \rangle &\rightarrow & \langle \text{dígito} \rangle \mid \langle \text{dígito} \rangle \langle \text{número iniciado} \rangle \\
                    \langle \text{dígito no cero} \rangle &\rightarrow & 1 \mid 2 \mid 3 \mid 4 \mid 5 \mid 6 \mid 7 \mid 8 \mid 9 \\
                    \langle \text{dígito} \rangle &\rightarrow & 0 \mid \langle \text{dígito no cero} \rangle
                \end{array}
            \right\}
        \end{align*}

        Notemos que esta gramática es similar a la descrita en el Ejercicio~\ref{ej:1.2}.\ref{ej:1.2.b}, pero adaptada para que los números naturales no puedan empezar por $0$.
        No obstante, esta gramática es de tipo $2$. Busquemos una de tipo $3$.
        Sea la gramática $G'=\left(V',T',P',S'\right)$ dada por:
        \begin{align*}
            V' &= \{S, X, Y, Z\} \\
            T' &= \{0,1,2,3,4,5,6,7,8,9\} \\
            S' &= S \\
            P' &= \left\{
                \begin{array}{rcl}
                    S &\rightarrow & 0 \mid 1N \mid 2N \mid 3N \mid 4N \mid 5N \mid 6N \mid 7N \mid 8N \mid 9N\\
                    N &\rightarrow & 0N\mid 1N \mid 2N \mid 3N \mid 4N \mid 5N \mid 6N \mid 7N \mid 8N \mid 9N \mid \veps
                \end{array}
            \right\}
        \end{align*}
        \item $\{ a^n \in \{a,b\}^\ast \mid n\geq 0 \} \cup \{ a^nb^n \in \{a,b\}^\ast \mid n\geq 0 \}$
        
        Sea la gramática $G=\left(V,T,P,S\right)$ dada por:
        \begin{align*}
            V &= \{S, X, Y\} \\
            T &= \{a,b\} \\
            S &= S \\
            P &= \left\{
                \begin{array}{rcl}
                    S &\rightarrow & X \mid Y \mid \veps \\
                    X &\rightarrow & aX \mid \veps \\
                    Y &\rightarrow & aYb \mid \veps
                \end{array}
            \right\}
        \end{align*}
        \item $\{ a^nb^{2n}c^m \in \{a,b,c\}^\ast \mid n,m>0 \}$
        
        Sea la gramática $G=\left(V,T,P,S\right)$ dada por:
        \begin{align*}
            V &= \{S, X, Y, Z\} \\
            T &= \{a,b,c\} \\
            S &= S \\
            P &= \left\{
                \begin{array}{rcl}
                    S &\rightarrow & aXbbcY \\
                    X &\rightarrow & aXbb \mid \veps \\
                    Y &\rightarrow & cY \mid \veps
                \end{array}
            \right\}
        \end{align*}
        \item $\{ a^nb^ma^n \in \{a,b\}^\ast \mid m,n\geq 0 \}$
        
        Sea la gramática $G=\left(V,T,P,S\right)$ dada por:
        \begin{align*}
            V &= \{S, X\} \\
            T &= \{a,b\} \\
            S &= S \\
            P &= \left\{
                \begin{array}{rcl}
                    S &\rightarrow & aSa \mid bX \mid \veps \\
                    X &\rightarrow & bX \mid \veps \\
                \end{array}
            \right\}
        \end{align*}

        \item Palabras con 0's y 1's que contengan la subcadena 00 y 11.
        
        Sea la gramática $G=\left(V,T,P,S\right)$ dada por:
        \begin{align*}
            V &= \{S, X\} \\
            T &= \{0,1\} \\
            S &= S \\
            P &= \left\{
                \begin{array}{rcl}
                    S &\rightarrow & X00X11X \mid X11X00X \\
                    X &\rightarrow & 0X \mid 1X \mid \veps
                \end{array}
            \right\}
        \end{align*}

        Notemos que esta gramática es de tipo $2$. Busquemos una de tipo $3$.
        Sea la gramática $G'=\left(V',T',P',S'\right)$ dada por:
        \begin{align*}
            V' &= \{S, X, A, B, F\} \\
            T' &= \{0,1\} \\
            S' &= S \\
            P' &= \left\{
                \begin{array}{rcl}
                    S &\rightarrow & 0S \mid 1S \mid X\\
                    X &\rightarrow & 00A \mid 11B \\
                    A &\rightarrow & 0A \mid 1A \mid 11F \\
                    B &\rightarrow & 0B \mid 1B \mid 00F \\
                    F &\rightarrow & 0F \mid 1F \mid \veps
                \end{array}
            \right\}
        \end{align*}

        Notemos que:
        \begin{itemize}
            \item $S$: No hemos encontrado ninguna subcadena.
            \item $X$: Hemos encontrado una subcadena, y ahora buscamos la otra.
            \item $A$: Hemos encontrado la subcadena $00$, y ahora buscamos la subcadena $11$.
            \item $B$: Hemos encontrado la subcadena $11$, y ahora buscamos la subcadena $00$.
            \item $F$: Hemos encontrado ambas subcadenas.
        \end{itemize}
        
        \item Palíndromos formados con las letras $a$ y $b$.
        
        Sea la gramática $G=\left(V,T,P,S\right)$ dada por:
        \begin{align*}
            V &= \{S, X, Y\} \\
            T &= \{a,b\} \\
            S &= S \\
            P &= \left\{
                \begin{array}{rcl}
                    S &\rightarrow & aSa \mid bSb \mid \veps \mid a \mid b
                \end{array}
            \right\}
        \end{align*}
        Notemos que las reglas $S\rightarrow a\mid b$ se han añadido para añadir los palíndromos de longitud impar.
    \end{enumerate}
\end{ejercicio}

\begin{ejercicio}[Complejidad: Media]
    Calcula, de forma razonada, gramáticas que generen cada uno de los siguientes lenguajes:
    \begin{enumerate}
        \item $\{uv \in \{0,1\}^\ast \mid u^{-1} \text{ es un prefijo de } v\}$
        
        Sea la gramática $G=\left(V,T,P,S\right)$ dada por:
        \begin{align*}
            V &= \{S, X, Y\} \\
            T &= \{0,1\} \\
            S &= S \\
            P &= \left\{
                \begin{array}{rcl}
                    S &\rightarrow & XY \\
                    X &\rightarrow & 0X0 \mid 1X1 \mid \veps \\
                    Y &\rightarrow & 0Y \mid 1Y \mid \veps
                \end{array}
            \right\}
        \end{align*}
        Notemos que $X$ deriva en el palíndromo, $uu^{-1}$, y $Y$ en el resto de la palabra de $v$.
        \item $\{ucv \in \{a,b,c\}^\ast \mid |u| = |v|\}$
        
        Sea la gramática $G=\left(V,T,P,S\right)$ dada por:
        \begin{align*}
            V &= \{S, X\} \\
            T &= \{a,b,c\} \\
            S &= S \\
            P &= \left\{
                \begin{array}{rcl}
                    S &\rightarrow & XSX \mid c \\
                    X &\rightarrow & a \mid b \mid c
                \end{array}
            \right\}
        \end{align*}

        \item $\{u1^n \in \{0,1\}^\ast \mid |u| = n\}$
        
        Sea la gramática $G=\left(V,T,P,S\right)$ dada por:
        \begin{align*}
            V &= \{S, X\} \\
            T &= \{0,1\} \\
            S &= S \\
            P &= \left\{
                \begin{array}{rcl}
                    S &\rightarrow & XS1 \mid \veps \\
                    X &\rightarrow & 0 \mid 1
                \end{array}
            \right\}
        \end{align*}

        \item $\{a^nb^na^{n+1} \in \{a,b\}^\ast \mid n\geq 0\}$ (observar transparencias de teoría)
        
        Sea la gramática $G=\left(V,T,P,S\right)$ dada por:
        \begin{align*}
            V &= \{S, X, Y\} \\
            T &= \{a,b\} \\
            S &= S \\
            P &= \left\{
                \begin{array}{rcl}
                    S &\rightarrow & a\mid abaa\mid aXbaa\\
                    Xb & \rightarrow & bX\\
                    Xa & \rightarrow & Ybaa\\
                    bY & \rightarrow & Yb\\
                    aY & \rightarrow & aa\mid aaX
                \end{array}
            \right\}
        \end{align*}
    \end{enumerate}
\end{ejercicio}


\begin{ejercicio}[Complejidad: Difícil] 
    Calcula, de forma razonada, gramáticas que generen cada uno de los siguientes lenguajes:
    \begin{enumerate}
        \item $\{a^nb^mc^k \in \{a,b,c\}^\ast \mid k = m + n\}$
        
        Sea la gramática $G=\left(V,T,P,S\right)$ dada por:
        \begin{align*}
            V &= \{S, X\} \\
            T &= \{a,b,c\} \\
            S &= S \\
            P &= \left\{
                \begin{array}{rcl}
                    S &\rightarrow & aSc \mid X \\
                    X &\rightarrow & bXc \mid \veps
                \end{array}
            \right\}
        \end{align*}
        
        \item Palabras que son múltiplos de 7 en binario.
            \begin{description}
                \item [Opción 1.] 
                    Hacer un autómata que acepte el lenguaje. Aunque un concepto del Tema 2, lo añadimos por ser más simple que la segunda opción.
                    Cada estado, donde $N$ es el número que llevamos leído, viene notado por:
                    \begin{equation*}
                        q_i: N \text{\ mod\ } 7 = i \quad \forall i\in \{0,\ldots,6\}
                    \end{equation*}
                    Usamos que:
                    \begin{itemize}
                        \item Añadirle un $0$ al final a un número en binario es multiplicarlo por $2$.
                        \item Añadirle un $1$ al final a un número en binario es multiplicarlo por $2$ y sumarle $1$.
                    \end{itemize}

                    Por tanto, el AFD sería:
                    \begin{figure}[H]
                        \centering
                        \begin{tikzpicture}
                            \node[state,initial,accepting] (q0) {$q_0$};
                            \node[state] (q1) [right of=q0] {$q_1$};
                            \node[state] (q2) [above right of=q1] {$q_2$};
                            \node[state] (q3) [below right of=q1] {$q_3$};
                            \node[state] (q4) [right of=q2] {$q_4$};
                            \node[state] (q5) [right of=q3] {$q_5$};
                            \node[state] (q6) [right of=q5] {$q_6$};

                            \draw   (q0) edge[loop above] node {0} (q0)
                                    (q0) edge[above] node {1} (q1)
                                    (q1) edge[above] node {0} (q2)
                                    (q1) edge[above] node {1} (q3)
                                    (q2) edge[above] node {0} (q4)
                                    (q2) edge[left, bend left] node[pos=0.8] {1} (q5)
                                    (q3) edge[below, bend right] node {0} (q6)
                                    (q3) edge[above] node {1} (q0)
                                    (q4) edge[above, bend left] node {0} (q1)
                                    (q4) edge[above, bend right] node {1} (q2)
                                    (q5) edge[above] node {0} (q3)
                                    (q5) edge[right] node {1} (q4)
                                    (q6) edge[above] node {0} (q5)
                                    (q6) edge[loop above] node {1} (q6);
                        \end{tikzpicture}
                        \caption{AFD que acepta los múltiplos de 7 en binario.}
                    \end{figure}

                    La gramática, por tanto, que genera este lenguaje es $G=\left(V,T,P,S\right)$ dada por:
                    \begin{align*}
                        V &= Q=\{q_0,\dots,q_6\} \\
                        T &= A=\{0,1\} \\
                        S &= q_0 \\
                        P &= \left\{
                            \begin{array}{rcl}
                                q_0 &\rightarrow & 0q_0 \mid 1q_1 \mid \veps \\
                                q_1 &\rightarrow & 0q_2 \mid 1q_3 \\
                                q_2 &\rightarrow & 0q_4 \mid 1q_5 \\
                                q_3 &\rightarrow & 0q_6 \mid 1q_0 \\
                                q_4 &\rightarrow & 0q_1 \mid 1q_2 \\
                                q_5 &\rightarrow & 0q_3 \mid 1q_4 \\
                                q_6 &\rightarrow & 0q_5 \mid 1q_6
                            \end{array}
                        \right.
                    \end{align*}

                \item [Opción 2.] 
                    Un tanto más complicada, introduce cálculos con números binarios. La idea principal es que si $x$ es un número natural, entonces:
                    \begin{equation*}
                        7x = (8-1)x = 8x - x
                    \end{equation*}
                    \begin{itemize}
                        \item Multiplicar un número en binario por 8 es añadirle tres 0s al final.
                        \item Restar un número binario menos otro es realizarle el complemento a dos al segundo, sumar los números y descartar el primer 1.
                    \end{itemize}
                    Realizar estas dos operaciones es más sencillo que multiplicar un número cualquiera en binario por 7. Procedemos por tanto, a:
                    \begin{enumerate}
                        \item Generar un número cualquiera en binario.
                        \item Multiplicarlo por 8.
                        \item Generar su complemento a 2 en binario.
                        \item Sumar ambos números.
                        \item Descarar el bit de acarreo (el más significativo).
                    \end{enumerate}
                    Para esta opción, construiremos la gramática $G=(V,T,P,S)$ dada por:
                    \begin{align*}
                        V &= \{S, N, \alpha, \beta, \delta,\gamma, Z, Z', A, D, E, E_0, E_1, E_2, E_0', E_1', \overline{E_0}, \overline{E_1}, \overline{E_2}, L_0, L_1, X\} \\
                        T &= \{0,1\} \\
                        S &= S
                    \end{align*}
                    Y $P$ es un conjunto que contiene todas las reglas de producción que se mostrarán a continuación.

                    La idea es:
                    \begin{itemize}
                        \item Generar entre $\alpha$ y $\beta$ cualquier número en binario, mientras generamos entre $\beta$ y $\gamma$ su complemento a 1 en espejo (es decir, el número invertido). Finalmente, multiplicaremos el de la izquierda por 8 y se verá reflejado en la izquierda con 1s.
                        \item Posteriormente, usaremos la variable $Z$ para sumarle 1 al complemento a 1 del número generado. 
                        \item Como no podemos modificar símbolos terminales una vez ya generados, trabajaremos todo el rato hasta el final con variables, de forma que $A$ será un 0 y $B$ un 1.
                        \item Una vez generado el número $8x$ y $x$ en complemento a dos en espejo, pasaremos a sumar ambos números usando para ello las variables $E$ y $L$. Los símbolos del número en complemento a 2 los iremos eliminando y en la izquierda controlaremos los bits del número que ya hemos usado con la variable $\delta$.
                        \item Cuando las variables $\beta$ y $\gamma$ ``se toquen'', habremos terminado de sumar y ya sólo quedará eliminar las variables delimitadoras (las letras griegas) y sustituir $A$ y $B$ por 0 y 1, respectivamente.
                    \end{itemize}
                    Comenzamos ya describiendo las reglas de producción:
                    \begin{itemize}
                        \item En primer lugar, creamos el entorno en el que trabajaremos, aceptando ``0'' como número en binario múltiplo de 7:
                            \begin{equation*}
                                S \rightarrow \alpha BNABBB\gamma\ |\ 0
                            \end{equation*}
                            Iremos usando $N$ para generar nuestro número a su izquierda y el complemento a 1 en espejo a la derecha. 
                            \begin{itemize}
                                \item Las tres $B$s ya introducidas a la derecha son para luego compensar la multiplicación por 8.
                                \item Así mismo, hemos generado ya $B$ a la izquierda y $A$ a la derecha para aceptar sólo números binarios que comiencen por 1.
                            \end{itemize}
                            Usamos ahora la variable $N$ para generar cualquier número en binario a la izquierda, con su complemento a 1 en espejo a la derecha:
                            \begin{equation*}
                                N \rightarrow ANB\ |\ BNA\ |\ AAA\gamma\beta Z
                            \end{equation*}
                            Una vez generado el número, terminaremos añadiendo 3 $A$s a la izquierda (multiplicar por 8), incluyendo los separadores $\gamma$ y $\beta$ y la variable $Z$, que se encargará de sumar 1 al número en complemento a 1 para pasarlo a complemento a 2.
                        \item Usamos ahora $Z$ para pasar el número de la derecha a complemento a 2:
                            \begin{equation*}
                                ZB \rightarrow BZ
                            \end{equation*}
                            Buscamos el primer 0, por lo que saltamos los 1s.
                            \begin{equation*}
                                ZA \rightarrow Z'B
                            \end{equation*}
                            Hemos encontrado el primer 0, lo cambiamos por 1 y volvemos con la variable $Z'$.
                            \begin{equation*}
                                BZ' \rightarrow Z'A
                            \end{equation*}
                            Volvemos a la izquierda, cambiando todos los 1s que saltamos anteriormente por 0s.
                            \begin{equation*}
                                \beta Z'\rightarrow\beta L_0
                            \end{equation*}
                            Una vez llegamos a $\beta$, tenemos el número en complemento a 1 y comenzamos con la aritmética ($L_0$ representa que no hemos cogido ningún número y que no nos llevamos nada de la suma anterior).
                        \item Comenzamos ahora con la aritmética, la parte más complicada de la gramática. Distinguimos dos casos:
                            \begin{enumerate}
                                \item No nos llevamos nada de la operación anterior ($L_0$):
                                    \begin{equation*}
                                        L_0A \rightarrow E_0
                                    \end{equation*}
                                    Cogemos un 0 de la derecha y la variable $E_0$ lo transportará a la izquierda.
                                    \begin{align*}
                                        A E_0 &\rightarrow E_0 A \\
                                        B E_0 &\rightarrow E_0 B \\
                                        \beta E_0 &\rightarrow E_0 \beta
                                    \end{align*}
                                    Nos movemos hacia la izquierda, buscando $\delta$ (que indica por dónde nos quedamos sumando).
                                    \begin{equation*}
                                        \delta E_0 \rightarrow \overline{E_0}
                                    \end{equation*}
                                    Donde la barra indica que hemos ``cogido'' $\delta$, la cual tendremos que soltar en el siguiente dígito.
                                    \begin{align*}
                                        A\overline{E_0} &\rightarrow \delta A E_0' \\
                                        B\overline{E_0} &\rightarrow \delta B E_0' 
                                    \end{align*}
                                    Como estamos sumando 0, dejamos el dígito invariante, sólo movemos $\delta$ hacia la izquierda. Usamos la variable $E_0'$ para volver, que indica que no nos llevamos nada de la suma:
                                    \begin{align*}
                                        E_0' A &\rightarrow AE_0' \\
                                        E_0' B &\rightarrow BE_0' \\
                                        E_0' \beta &\rightarrow \beta L_0
                                    \end{align*}
                                    Nos desplazamos hacia la derecha, hasta encontrar $\beta$, ya que después encontraremos el siguiente dígito con el que operar. Como no nos llevábamos nada, volvemos a $L_0$.

                                    Si ahora no nos llevamos nada y en vez de un 0 (una $A$) hay un 1 (una $B$), repetimos el proceso pero usando para ello $E_1$:
                                    \begin{align*}
                                        L_0 B &\rightarrow E_1 \\
                                        AE_1 &\rightarrow E_1 A \\
                                        BE_1 &\rightarrow E_1 B \\
                                        \beta E_1 &\rightarrow E_1 \beta \\
                                        \delta E_1 &\rightarrow \overline{E_1}
                                    \end{align*}
                                    A continuación, $\overline{E_1}$ se encontrará con el dígito con el que operar:
                                    \begin{align*}
                                        A\overline{E_1} &\rightarrow \delta B E_0' \\
                                        B\overline{E_1} &\rightarrow \delta A E_1'
                                    \end{align*}
                                    \begin{itemize}
                                        \item Si era un 0 (una $A$), lo cambiamos por un 1.
                                        \item Si era un 1 (una $B$), lo cambiamos por un 0 y nos llevamos 1 (que es lo que indica $E_1'$).
                                    \end{itemize}
                                    El comportamiento de $E_1'$ es similar a $E_0'$ pero ahora pasando a $L_1$:
                                    \begin{align*}
                                        E_1' A &\rightarrow A E_1' \\
                                        E_1' B &\rightarrow B E_1' \\
                                        E_1' \beta &\rightarrow \beta L_1 
                                    \end{align*}
                                \item Ahora, estamos en el caso en el que nos llevamos un 1 de la operación anterior ($L_1$), que hemos visto que puede suceder:
                                    \begin{equation*}
                                        L_1A \rightarrow E_1
                                    \end{equation*}
                                    Si nos encontramos un 0 llevando 1, es como si nos hubiéramos encontrado un 1 llevando 0, por lo que no hay nada nuevo que hacer. Sin embargo, si nos encontramos un 1:
                                    \begin{equation*}
                                        L_1B \rightarrow E_2
                                    \end{equation*}
                                    Tenemos que tener en mente que el siguiente dígito con el que realizar la suma lo sumaremos con 2 ($E_2$ es análogo a $E_0$ y $E_1$ pero ahora ``transportando'' un 2):
                                    \begin{align*}
                                        A E_2 &\rightarrow E_2 A \\
                                        B E_2 &\rightarrow E_2 B \\
                                        \beta E_2 &\rightarrow E_2 \beta \\
                                        \delta E_2 &\rightarrow \overline{E_2}
                                    \end{align*}
                                    A continuación, $\overline{E_2}$ se encontrará con el dígito con el que operar:
                                    \begin{align*}
                                        A\overline{E_2} &\rightarrow \delta A E_1' \\
                                        B\overline{E_2} &\rightarrow \delta B E_1'
                                    \end{align*}
                                    Similar al caso de $\overline{E_0}$, dejamos el dígito invariante pero ahora tenemos que llevarnos 1 para la siguiente opereación.
                            \end{enumerate}
                        \item A poco que se piense, como $8x$ y su complemento a 2 tienen la misma cantidad de bits, terminaremos de realizar la operación cuando nos llevemos 1 y no queden bits del número en complemento a 2, dando lugar a:
                            \begin{equation*}
                                \beta L_1 \gamma \rightarrow X
                            \end{equation*}
                            donde $X$ es una variable finalizadora, que usamos para cambiar las $A$s por 0s, las $B$s por 1s y eliminar las variables auxiliares que nos quedan (ya hemos eliminado $\beta$ y $\gamma$ directamente al crear $X$, por lo que nos quedan $\delta$ y $\alpha$):
                            \begin{align*}
                                AX &\rightarrow X0 \\
                                BX &\rightarrow X1 \\
                                \delta X &\rightarrow X \\
                                \alpha X &\rightarrow \veps
                            \end{align*}
                            Cuando lleguemos a $\alpha X$, habremos ``limpiado'' la palaba, por lo que ya podemos quitar todas las variables, generando una palabra de la gramática, que forzosamente tiene que ser un múltiplo de 7 en binario (acabamos de multiplicar cualquier número por 7). Además, como con esta gramática podemos multiplicar cualquier número por 7, esta genera todos los números que son múltiplos de 7 en binario.
                    \end{itemize}
                Mostramos finalmente un ejemplo de producción de una palabra mediante esta gramática. Trataremos de generar ``14'' en binario (la 3ª palabra que usa menos reglas de producción para ser creada, tras 0 y 7):
                \begin{align*}
                    S &\rightarrow \alpha BNABBB \gamma \rightarrow \alpha BANBABBB\gamma \rightarrow \alpha BAAAA \delta \beta Z BABBB\gamma \rightarrow \\
                      &\rightarrow \alpha BAAAA\delta \beta BZABBB\gamma \rightarrow \alpha BAAAA \delta \beta BZ' BBBB \gamma \rightarrow \\
                      &\rightarrow \alpha BAAAA \delta \beta Z' ABBBB \gamma \rightarrow \alpha BAAAA\delta \beta L_0 ABBBB\gamma \rightarrow \\
                      &\rightarrow \alpha BAAAA\delta \beta E_0 BBBB \gamma \rightarrow \alpha BAAAA\delta E_0 \beta BBBB\gamma \rightarrow \\
                      &\rightarrow \alpha BAAAA \overline{E_0}\beta BBBB\gamma \rightarrow \alpha BAAA \delta A E_0' \beta BBBB \gamma \rightarrow \\
                      &\rightarrow \alpha BAAA \delta A\beta L_0 BBBB \gamma \rightarrow \alpha BAAA \delta A\beta E_1 BBB \gamma \rightarrow \\
                      &\rightarrow \alpha BAAA \delta A E_1\beta BBB \gamma \rightarrow \alpha BAAA\delta E_1 A \beta BBB \gamma \rightarrow \\
                      &\rightarrow \alpha B AAA \overline{E_1} A \beta BBB \gamma \rightarrow \alpha BAA \delta B E_0' A \beta BBB \gamma \rightarrow \\ 
                      &\rightarrow \alpha BAA \delta BA E_0' \beta BBB \gamma \rightarrow \alpha BAA \delta BA \beta L_0 BBB \gamma \rightarrow \\
                      &\rightarrow \alpha BAA \delta BA \beta E_1 BB \gamma \rightarrow \alpha BAA \delta BA E_1 \beta BB \gamma \rightarrow \\
                      &\rightarrow \alpha BAA \delta BE_1 A \beta BB \gamma \rightarrow \alpha BAA \delta E_1 BA \beta BB \gamma \rightarrow \\
                      &\rightarrow \alpha BAA \overline{E_1} BA \beta BB \gamma \rightarrow \alpha BA \delta B E_0' BA \beta BB \gamma \rightarrow \\
                      &\rightarrow \alpha BA \delta BBE_0' A \beta BB \gamma \rightarrow \alpha BA \delta BBA E_0' \beta BB \gamma \rightarrow \\
                      &\rightarrow \alpha BA \delta BBA \beta L_0 BB \gamma \rightarrow \alpha BA \delta BBA \beta E_1 B \gamma \rightarrow \alpha BA \delta BBA E_1 \beta B\gamma \rightarrow \\
                      &\rightarrow \alpha BA \delta BBE_1 A \beta B \gamma \rightarrow \alpha BA \delta B E_1 BA \beta B \gamma \rightarrow \alpha BA \delta E_1 BBA \beta B \gamma \rightarrow \\
                      &\rightarrow \alpha BA \overline{E_1} BBA \beta B \gamma \rightarrow \alpha B \delta B E_0' BBA \beta B \gamma \rightarrow \alpha B\delta BB E_0' BA \beta B \gamma \rightarrow \\
                      &\rightarrow \alpha B\delta BBB E_0' A \beta B \gamma \rightarrow \alpha B\delta BBBA E_0' \beta B \gamma \rightarrow \alpha B \delta BBBA \beta L_0 B \gamma \rightarrow \\
                      &\rightarrow \alpha B \delta BBBA \beta E_1 \gamma \rightarrow \alpha B \delta BBBA E_1 \beta \gamma \rightarrow \alpha B \delta BBB E_1 A \beta \gamma \rightarrow \\
                      &\rightarrow \alpha B \delta BB E_1 BA \beta \gamma \rightarrow \alpha B \delta B E_1 BBA \beta \gamma \rightarrow \alpha B \delta E_1 BBBA \delta \gamma \rightarrow \\
                      &\rightarrow \alpha B \overline{E_1} BBBA \beta \gamma \rightarrow \alpha \delta A E_1' BBBA \beta \gamma \rightarrow \alpha \delta AB E_1' BBA \beta \gamma \rightarrow \\
                      &\rightarrow \alpha \delta ABB E_1' BA \beta \gamma \rightarrow \alpha \delta ABBB E_1' A \beta \gamma \rightarrow \alpha \delta ABBBA E_1' \beta \gamma \rightarrow \\
                      &\rightarrow \alpha \delta ABBBA \beta L_1 \gamma \rightarrow \alpha \delta ABBBA X \rightarrow \alpha \delta ABBBX0 \rightarrow \alpha \delta ABBX10 \rightarrow \\
                      &\rightarrow \alpha \delta ABX110 \rightarrow \alpha \delta AX1110 \rightarrow \alpha \delta X01110 \rightarrow \alpha X 01110 \rightarrow 01110
                \end{align*}
            \end{description}
    \end{enumerate}
\end{ejercicio}


\begin{ejercicio}[Complejidad: Extrema (no son libres de contexto)]
    Calcula, de forma razonada, gramáticas que generen cada uno de los siguientes lenguajes:
    \begin{enumerate}
        \item $\{ww \mid w \in \{0,1\}^\ast\}$\\
            Para este lenguaje, hemos construido la gramática $G=(V,T,P,S)$ dada por:
            \begin{align*}
                V &= \{S, \alpha, \beta, \gamma, X, E, E_1, E_0, E', B\} \\
                T &= \{0,1\} \\
                S &= S
            \end{align*}
            $P$ que contiene las reglas de producción que se mostrarán a continuación. 

            La idea principal en la gramática es generar entre las variables $\alpha$ y $\beta$ cualqueir palabra del lenguaje ${\{0,1\}}^{\ast}$. Posteriormente, iremos copiando dicha palabra a la derecha de $\beta$ usando para ello las variables $E$ y $\gamma$, de forma que con $\gamma$ controlaremos la parte de la palabra de la izquierda que ya hayamos copiado a la derecha de $\beta$.

            Finalmente, usaremos $B$ para eliminar cualquier rastro de las variable auxiliares. De esta forma, las reglas de $P$ son:
            \begin{itemize}
                \item Para generar cualquier palabra entre $\alpha$ y $\beta$:
                    \begin{align*}
                        S &\rightarrow \alpha X \beta \\
                        X &\rightarrow 0X\ |\ 1X\ |\ E\gamma
                    \end{align*}
                \item Para coger un 1 y copiarlo a la derecha:

                    Hemos de estar al final de la parte de la palabra no copiada (luego ha de ser $xE\gamma$ siendo $x$ 0 o 1). Posteriormente, avanzamos $\gamma$ a la izquierda para indicar que dicho 1 ya está copiado y cambiamos a la variable que transporta el 1 a la derecha:
                    \begin{equation*}
                        1E\gamma \rightarrow \gamma 1 E_1 
                    \end{equation*}
                    Posteriormente, movemos dicha variable a la derecha:
                    \begin{align*}
                        E_1 1 &\rightarrow 1E_1 \\
                        E_1 0 &\rightarrow 0E_1
                    \end{align*}
                    Cuando lleguemos al final de la palabra de la izquierda, soltamos el 1 al inicio de la palabra de la derecha:
                    \begin{equation*}
                        E_1 \beta \rightarrow E' \beta 1
                    \end{equation*}
                \item Para coger un 0 y copiarlo a la derecha, es una situación análoga pero usamos otra variable:
                    \begin{align*}
                        0E\gamma &\rightarrow \gamma 0 E_0 \\
                        \\
                        E_0 1 &\rightarrow 1E_0 \\
                        E_0 0 &\rightarrow 0E_0 \\
                        \\
                        E_0 \beta &\rightarrow E'\beta 0
                    \end{align*}
                \item Ahora, explicamos $E'$, cuya única funcionalidad es volver al final de la parte no copiada de la palabra de la izquierda:
                    \begin{align*}
                        1E' &\rightarrow E'1 \\
                        0E' &\rightarrow E'0 \\
                        \gamma E' &\rightarrow E\gamma
                    \end{align*}
                \item La copia de la palabra terminará cuando se de $\alpha E\gamma$ (ya que estará toda la palabra copiada a la derecha). En dicho caso, eliminamos todas las variables auxiliares restantes:
                    \begin{align*}
                        \alpha E\gamma &\rightarrow B \\
                        B1 &\rightarrow 1B \\
                        B0 &\rightarrow 0B \\
                        B\beta &\rightarrow \veps 
                    \end{align*}
            \end{itemize}
            Puede demostrarse que el lenguaje generado por esta gramática es el solicitado. Por la complejidad de la gramática, nos limitamos a mostrar un ejemplo para ver de forma intuitiva el buen funcionamiento de la misma.

            Trataremos de generar la cadena: $10111011$ (es decir, ${(1011)}^{2}$):
            \begin{align*}
                S &\rightarrow \alpha X \beta \rightarrow \alpha 1X\beta \rightarrow \alpha 10X\beta \rightarrow \alpha101X \beta \rightarrow \alpha1011X\beta \rightarrow \alpha1011E\gamma\beta \rightarrow \\
                  &\rightarrow \alpha101\gamma1E_1\beta \rightarrow \alpha101\gamma1E'\beta1 \rightarrow \alpha101\gamma E'1\beta1 \rightarrow \alpha101E\gamma1\beta1 \rightarrow \\
                  &\rightarrow \alpha10\gamma1E_11\beta1 \rightarrow \alpha 10\gamma11E_1\beta1 \rightarrow \alpha 10\gamma11E'\beta11 \rightarrow \alpha10\gamma1E'1\beta11 \rightarrow \\
                  &\rightarrow \alpha10\gamma E'11\beta11 \rightarrow \alpha10E\gamma11\beta11 \rightarrow \alpha 1\gamma0E_011\beta11 \rightarrow \alpha1\gamma01E_01\beta11 \rightarrow\\
                  &\rightarrow\alpha1\gamma011E_0\beta11 \rightarrow \alpha1\gamma011E'\beta011 \rightarrow \alpha1\gamma01E'1\beta011 \rightarrow \alpha1\gamma0E'11\beta011 \rightarrow \\
                  &\rightarrow \alpha1\gamma E'011\beta011 \rightarrow \alpha1E\gamma011\beta011 \rightarrow \alpha\gamma1E_1011\beta011 \rightarrow \alpha\gamma10E_111\beta011 \rightarrow \\
                  &\rightarrow \alpha\gamma101E_11\beta011 \rightarrow \alpha\gamma1011E_1\beta011 \rightarrow \alpha\gamma1011E'\beta1011 \rightarrow \alpha\gamma101E'1\beta1011 \rightarrow \\
                  &\rightarrow \alpha\gamma10E'11\beta1011 \rightarrow \alpha\gamma1E'011\beta1011 \rightarrow \alpha\gamma E'1011\beta1011 \rightarrow \alpha E\gamma1011\beta1011 \rightarrow \\
                  &\rightarrow B1011\beta1011 \rightarrow 1B011\beta1011 \rightarrow10B11\beta1011 \rightarrow 101B1\beta1011\rightarrow \\
                  &\rightarrow 1011B\beta1011 \rightarrow10111011
            \end{align*}
        \item $\{a^{n^2} \in \{a\}^{\ast} \mid n\geq 0\}$\\
            La idea que hemos tenido para hacer una gramática que acepte el lenguaje es la siguiente. Si representamos las 5 primeras palabras del lenguaje (ordenándolas por su longitud):
            \begin{gather*}
                \veps \\
                a \\
                aaaa \\
                aaaaaaaaa \\
                aaaaaaaaaaaaaaaa
            \end{gather*}
            Notemos que podemos ordenar las letras de la siguiente forma (olvidándonos de $\veps$, que no será relevante):
            \begin{gather*}
                a \\
                aa\ aa \\
                aaa\ aaa\ aaa\\
                aaaa\ aaaa\ aaaa\ aaaa 
            \end{gather*}
            De forma que tenemos 1 grupo de 1 ``a'', dos grupos de 2 ``a'', 3 grupos de 3 ``a'', \ldots Notemos que dados $n$ grupos de $n$ ``a'', será sencillo construir $n+1$ grupos de $n+1$ ``a'', ya que nos bastará con añadir una ``a'' a cada grupo y con duplicar el último grupo de ``a''.

            Hemos construido una gramática $G = (V, T, S, P)$ que simula este comportamiento inductivo del lenguaje, con lo que el lenguaje generado por la misma es el solicitado. Tenemos:
            \begin{align*}
                V &= \{\alpha, \beta, \delta, \gamma, \sigma, X, A, E, E_{\sigma}, \overline{E}, E', I, R, L, Z \} \\
                T &= \{0,1\} \\
                S &= S
            \end{align*}
            Donde $P$ es el conjunto de reglas de producción que contiene todas las reglas que explicaremos a continuación.

            La idea es que si queremos generar la palabra $a^{n^2}$, que generemos $n-1$ $A$s entre $\alpha$ y $\beta$. Tendremos ya creada una letra $a$ y lo que haremos será que por cada $A$ que hayamos generado, repitamos el proceso inductivo descrito anteriormente. Además, separaremos los ``grupos'' de ``a'' con variables $I$.
            Finalmente, para duplicar un grupo de ``a'', usaremos las variables $\delta$ y $\sigma$.\\

                Empezamos generando nuestro entorno en el que trabajaremos (o la palabra vacía):
                \begin{equation*}
                    S \rightarrow \alpha X \beta Ia\delta \gamma \ |\ \veps
                \end{equation*}
                A continuación, usamos $X$ para generar las $A$s:
                \begin{equation*}
                    X \rightarrow AX\ |\ E
                \end{equation*}
                Una vez terminadas de leer las $A$s, generaremos $E$, que se encargará de ir eliminando una $A$, de realizar el proceso inductivo y de volver al estado inicial, hasta terminar con todas las $A$s generadas.

             Ahora, hacemos que $E$ coja una $A$, con lo que le dejamos salir de la región comprendida por $\alpha$ y $\beta$:
                \begin{equation*}
                    AE\beta \rightarrow \beta E
                \end{equation*}
                Ahora desplazamos la variable $E$ a la derecha, haciendo que cada vez que entre en un grupo de ``a'' (cuando pase una variable $I$) añada una nueva:
                \begin{align*}
                    Ea &\rightarrow aE \\
                    EI &\rightarrow IaE
                \end{align*}
                Cuando la variable $E$ se encuentre con $\delta$, habremos terminado de incrementar las ``a'', con lo que tendremos que duplicar ahora el último grupo de ``a''. Para ello, prepararemos un entorno, de forma que entre $I$ y $\sigma$ vayamos generando el nuevo grupo de ``a'', entre $\sigma$ y $\delta$ se encuentre las ``a'' por copiar; y que entre $\delta$ y $\gamma$ se encuentren las ``a'' que ya hayan sido duplicadas.

                De esta forma, cuando $E$ se encuentre con $\delta$, pasaremos a una variable que busque $I$ para colocar delante suya $\sigma$:
                \begin{align*}
                    E\delta &\rightarrow E_{\sigma} \delta \\
                    aE_{\sigma} &\rightarrow E_{\sigma} a \\
                    IE_{\sigma} &\rightarrow I\sigma R
                \end{align*}
                Ahora, usaremos $R$ para movernos a la derecha tras copiar una letra y $L$ para movernos a la izquierda con el fin de pegar una letra.
                \begin{equation*}
                    Ra \rightarrow aR 
                \end{equation*}
                Nos movemos a la derecha
                \begin{equation*}
                    aR\delta \rightarrow L\delta a
                \end{equation*}
                Cuando lleguemos a $\delta$, guardamos una ``a'' más como copiada.
                \begin{equation*}
                    aL \rightarrow La
                \end{equation*}
                Nos moveremos hacia la izquierda buscando $\sigma$ para crear una $a$:
                \begin{equation*}
                    \sigma L \rightarrow a \sigma R
                \end{equation*}
                Pegaremos una $a$ tras $\sigma$ y repetiremos el proceso.

                El proceso terminará cuando no haya más ``a'' entre $\sigma$ y $\delta$:
                \begin{equation*}
                    \sigma R\delta \rightarrow I\overline{E}
                \end{equation*}
                Cuando hayamos terminado, colocamos una $I$ para hacer efectivo el nuevo grupo. Finalmente, debemos colocar nuevamente $\delta$ a la izquierda de $\gamma$ para la siguiente vez que copiemos. Usamos para ello $\overline{E}$:
                \begin{equation*}
                    \overline{E}a \rightarrow a\overline{E}
                \end{equation*}
                Nos movemos a la izquierda buscando $\gamma$ y cuando la encontremos, colocamos $\delta$:
                \begin{equation*}
                    \overline{E}\gamma \rightarrow E' \delta \gamma
                \end{equation*}
                Usaremos finalmente $E'$ para desplazarnos a la izquierda, tras $\beta$, donde volveremos a la variable $E$, que reiniciará el proceso descrito para realizarlo nuevamente:
                \begin{align*}
                    aE' &\rightarrow E' a \\
                    IE' &\rightarrow E' I \\
                    \beta E' &\rightarrow E\beta
                \end{align*}
                Este proceso terminará cuando no queden $A$s por copiar. En dicho caso, pasaremos a una variable $Z$ que eliminará todas las variables auxiliares:
                \begin{equation*}
                    \alpha E \rightarrow Z
                \end{equation*}
                De esta forma, $Z$ se mueve a la derecha, eliminando todas las variables y pasando a través de las letras:
                \begin{align*}
                    Z\beta &\rightarrow Z \\
                    ZI &\rightarrow Z \\
                    Za &\rightarrow aZ \\
                    Z\delta &\rightarrow Z \\
                    Z\gamma &\rightarrow \veps
                \end{align*}
            Como ejemplo y para comprobar que el lenguaje generado por dicha gramática funciona es el deseado mostramos el siguiente ejemplo, en el que generamos $a^9$:
            \begin{align*}
                S &\rightarrow \alpha X \beta I a \delta\gamma \rightarrow \alpha AX\beta Ia\delta\gamma \rightarrow \alpha AAX\beta Ia\delta\gamma \rightarrow \alpha AAE \beta Ia \delta\gamma \rightarrow \alpha A \beta E Ia\delta\gamma \rightarrow \\
                  &\rightarrow \alpha A\beta IaEa\delta\gamma \rightarrow \alpha A\beta IaaE\delta\gamma \rightarrow \alpha A\beta IaaE_{\sigma}\delta\gamma \rightarrow \alpha A\beta IaE_{\sigma}a\delta\gamma \rightarrow \\ 
                  &\rightarrow\alpha A\beta IE_{\sigma}aa\delta\gamma \rightarrow \alpha A\beta I\sigma Raa\delta\gamma \rightarrow \alpha A\beta I \sigma aRa \delta\gamma \rightarrow \alpha A \beta I \sigma aaR \delta\gamma\rightarrow \\
                  &\rightarrow \alpha A\beta I\sigma aL\delta a\gamma \rightarrow \alpha A \beta I\sigma L a\delta a\gamma \rightarrow \alpha A \beta I a\sigma R a \delta a\gamma  \rightarrow \alpha A \beta Ia\sigma aR\delta a\gamma \rightarrow \\
                  &\rightarrow \alpha A \beta I a\sigma L\delta aa \gamma \rightarrow \alpha A \beta Iaa \sigma R \delta aa \gamma \rightarrow \alpha A\beta IaaI\overline{E}aa\gamma \rightarrow \alpha A \beta IaaIa\overline{E}a\gamma \rightarrow \\
                  &\rightarrow \alpha A\beta IaaIaa\overline{E} \gamma \rightarrow \alpha A \beta IaaIaaE'\delta \gamma \rightarrow \alpha A\beta IaaIaE'a\delta\gamma \rightarrow \alpha A\beta IaaIE'aa\delta\gamma \rightarrow \\
                  &\rightarrow \alpha A\beta IaaE'Iaa\delta\gamma \rightarrow \alpha A \beta IaE'aIaa\delta\gamma \rightarrow \alpha A\beta IE'aaIaa\delta\gamma \rightarrow \\
                  &\rightarrow \alpha A \beta E'IaaIaa\delta\gamma \rightarrow \alpha AE\beta IaaIaa\delta\gamma \rightarrow \alpha \beta E IaaIaa\delta \gamma \rightarrow \\
                  &\rightarrow \alpha\beta IaEaaIaa\delta\gamma\rightarrow \alpha\beta IaaEaIaa\delta\gamma\rightarrow\alpha\beta IaaaEIaa\delta\gamma\rightarrow \\
                  &\rightarrow \alpha\beta IaaaIaEaa\delta\gamma\rightarrow\alpha\beta IAAAIaaEa\delta\gamma\rightarrow \alpha\beta IaaaIaaaE\delta\gamma\rightarrow \\
                  &\rightarrow \alpha\beta IaaaIaaaE_{\sigma}\delta\gamma \rightarrow \alpha\beta IaaaIaaE_{\sigma}a\delta\gamma\rightarrow \alpha\beta IaaaIaE_{\sigma}aa\delta\gamma \rightarrow \\
                  &\rightarrow \alpha\beta IaaaIE_{\sigma}aaa\delta\gamma\rightarrow \alpha\beta IaaaI\sigma R aaa\delta\gamma\rightarrow\alpha\beta IaaaI\sigma aRaa\delta\gamma \rightarrow \\
                  &\rightarrow\alpha\beta IaaaI\sigma aaRa\delta\gamma\rightarrow\alpha\beta IaaaI\sigma aaaR\delta\gamma\rightarrow \alpha\beta IaaaI\sigma aaL\delta a\gamma\rightarrow \\
                  &\rightarrow \alpha\beta IaaaI\sigma aLa\delta a\gamma \rightarrow \alpha\beta IaaaI\sigma Laa\delta a\gamma\rightarrow \alpha\beta IaaaIa\sigma R aa\delta a \gamma \rightarrow \\
                  &\rightarrow \alpha\beta IaaaIa\sigma aRa \delta a\gamma\rightarrow\alpha\beta IaaaIa\sigma aaR\delta a \gamma \rightarrow \alpha \beta IaaaIa\sigma aL\delta aa \gamma \rightarrow \\
                  &\rightarrow\alpha\beta IaaaIa\sigma La\delta aa\gamma\rightarrow\alpha\beta IaaaIaa\sigma Ra\delta aa\gamma\rightarrow \alpha\beta IaaaIaa\sigma aR\delta aa\gamma\rightarrow \\
                  &\rightarrow\alpha\beta IaaaIaa\sigma L\delta aaa\gamma \rightarrow \alpha\beta IaaaIaaa\sigma R\delta aaa\gamma\rightarrow \alpha\beta IaaaIaaaI\overline{E}aaa\gamma \rightarrow \\
                  &\rightarrow \alpha\beta IaaaIaaaIa\overline{E}aa\gamma\rightarrow\alpha\beta IaaaIaaaIaa\overline{E}a\gamma\rightarrow\alpha\beta IaaaIaaaIaaa\overline{E}\gamma\rightarrow \\
                  &\rightarrow \alpha\beta IaaaIaaaIaaaE'\delta\gamma \stackrel{(\ast)}{\rightarrow} \alpha\beta E' IaaaIaaaIaaa\delta\gamma \rightarrow \alpha E\beta IaaaIaaaIaaa\delta\gamma \rightarrow \\
                  &\rightarrow Z\beta IaaaIaaaIaaa\delta\gamma\rightarrow ZIaaaIaaaIaaa\delta\gamma \rightarrow ZaaaIaaaIaaa\delta\gamma\rightarrow \\
                  &\rightarrow aZaaIaaaIaaa\delta\gamma\rightarrow aaZaIaaaIaaa\delta\gamma\rightarrow aaaZIaaaIaaa\delta\gamma\rightarrow \\
                  &\rightarrow aaaZaaaIaaa\delta\gamma \stackrel{(\ast\ast)}{\rightarrow} aaaaaaaaaZ\delta\gamma \rightarrow aaaaaaaaa Z\gamma\rightarrow aaaaaaaaa
            \end{align*}
            \begin{itemize}
                \item Donde en $(\ast)$ hemos aplicado reiteradas veces que $aE'\rightarrow E'a$ y que $IE'\rightarrow E'I$.
                \item Donde en $(\ast\ast)$ hemos aplicado varias veces que $ZI\rightarrow Z$ y que $Za\rightarrow aZ$.
            \end{itemize}
        \item $\{a^p \in \{a\}^{\ast} \mid p \text{ es primo}\}$\\

            Se subirá próximamente una gramática.
            % // TODO: primos
        \item $\{a^nb^m \in \{a,b\}^{\ast} \mid n\leq m^2\}$\\

            Una vez conocida una gramática para el lenguaje $\{a^n^2 \mid n \in \mathbb{N}\}$, dar una gramática para este lenguaje es sencillo. Lo que haremos será generar primero un número arbitrarios de $A$s (variables) y de $b$s. Seguidamente, generaremos tantas $B$s como número de $b$ al cuadrado haya (usando para ello la gramática del lenguaje de los cuadrados perfectos), y finalmente, por cada $B$ que tengamos sustituiremos una $A$ por una $a$, de forma que si se nos gastan las $B$s y no hemos sustituido todas las $A$s, entonces era porque teníamos más $a$s de las permitidas en este lenguaje ($n>m$), con lo que no podremos generar ninguna palabra. Si por el contrario sustituimos todas las $A$s y nos siguen quedando $B$s, eliminaremos todas las $B$s para poder dar una palabra formada solo por símbolos terminales.

            Antes de consultar la gramática, recomendamos encarecidamente entender primero la gramática de los cuadrados perfectos, ya que es más sencilla y la usaremos con soltura para dar esta gramática.

            Los pasos que hace esta gramática son:
            \begin{enumerate}
                \item Primero, genera un número indeterminado de $A$s, tras la variable $\lm$. 
                \item Posteriormente, plantea el ``entorno'' de variables usado en la gramática de los cuadrados perfectos, que parte del caso base de tener una $b$.
                \item Entre $\alpha$ y $\beta$ se generan todas las $b$ que tendrá la posibel palabra a generar (menos una, que se generó en el caso base).
                \item Entre $\beta$ y $\gamma$, iremos colocando tantas $B$s como número de $b$s al cuadrado haya.
                \item Para controlar que una $b$ ya le hemos usado para generar el cuadrado, en vez de borrarla como hacíamos en la gramática de cuadrados perfectos, la metemos entre el espacio comprendido entre $\varphi$ y $\beta$. De esta forma, el entorno de trabajo será similar a:
                    \begin{equation*}
                        \lm A\ldots A b\alpha b\ldots b E\varphi b\ldots b \beta I B\ldots B I B \ldots BI \ldots I B \ldots B \delta\gamma
                    \end{equation*}
                \item Una vez generadas todas las $B$s, usamos la variable $Z$ para borrar todas las variables auxiliares para generar las $B$s, dejando la palabra de la forma
                    \begin{equation*}
                        \lm A \ldots A b\ldots b \beta B\ldots BH\gamma
                    \end{equation*}
                    de forma que el número de $B$s coincide con el cuadrado del número de $b$s.
                \item Posteriormente, usaremos la variable $H$ para borrar una $B$ y posteriormente sustituir una $A$ por una $a$, hasta que se nos acaben las $A$s (en cuyo caso la palabra es válida y eliminamos todas las variables dejando sólo los símbolos terminales) o las $B$s (en cuyo caso, había más $A$s, con lo que no generamos la palabra).
            \end{enumerate}

            De esta forma, damos la gramática $G=(V,T,P,S)$ dada por:
            \begin{align*}
                V &= \{S, \lm, \alpha,\beta,\varphi, \delta,\gamma, \sigma, \psi, G, A, B, X, I, E, E_{\sigma}, E', \overline{E}, L, R, Z, H, H', \overline{H}\} \\
                T &= \{a,b\}
            \end{align*}
            Y $P$ el conjunto de todas las producciones explicadas a continuación.

            En primer lugar, aceptamos que $\veps$ es una palabra del lenguaje. Habiendo considerado dicho caso, comenzamos pues generando todas las $A$s que queramos al inicio de la palabra:
            \begin{align*}
                S &\rightarrow \lm G\ |\ \veps \\
                G &\rightarrow AG
            \end{align*}
            Cuando hayamos generado todas las $A$s deseadas, situaremos nuestro entorno de trabajo para generar $b$s y posteriormente crear tantas $B$s como el cuadrado del número de $b$s:
            \begin{equation*}
                G \rightarrow b\alpha X\varphi\beta IB\delta\gamma
            \end{equation*}
            Y generaremos tantas $b$s como queramos con la variable $X$ (notemos que ya hemos generado una $b$ y una $B$, el caso base cuando $n = 1$).
            \begin{equation*}
                X \rightarrow bX\ |\ E
            \end{equation*}
            Cuando hayamos ya generado todas las $b$s, pasamos a generar las $B$s, usando para ello la variable $E$, que realizará un comportamiento iterativo, de forma que coja una $b$ (la sitúe detrás de $\varphi$), añada una $B$ en cada subgrupo de $B$s (separados por $I$), que duplique este último subgrupo (utilizando para ello $\sigma$, $L$ y $R$), y que vuelva con $E'$ a donde empezó, delante de $\varphi$:

            \begin{enumerate}
                \item En primer lugar, la variable $E$ coge una $b$ (la sitúa tras $\varphi$):
                    \begin{equation*}
                        bE\varphi \rightarrow \varphi bE
                    \end{equation*}
                    Y se desplaza a la derecha de $\varphi$.
                \item Posteriormente, se desplaza a la derecha, añadiendo una $B$ en cada subgrupo de $B$s, proceso que terminará cuando se encuentre con $\delta$:
                    \begin{align*}
                        Eb &\rightarrow bE \\
                        E\beta &\rightarrow \beta E \\
                        EI &\rightarrow IBE \qquad \text{(Añade una\ } B \text{)} \\
                        EB &\rightarrow BE \\
                        E\delta &\rightarrow E_{\sigma}\delta
                    \end{align*}
                \item Una vez topada con $\delta$, pasaremos a realizar la copia del último grupo de $B$s, con lo que tendremos que fijar el $\sigma$ que usábamos en la gramática de los cuadrados perfectos. Para ello:
                    \begin{align*}
                        BE_{\sigma} &\rightarrow E_{\sigma}B \\
                        IE_{\sigma} &\rightarrow I\sigma R
                    \end{align*}
                \item Una vez colocado el $\sigma$, comienza la copia de las $B$s, usando para ello las variables $L$, $R$ y el delimitador $\delta$, que marca las $B$s que ya han sido copiadas:
                    \begin{align*}
                        RB &\rightarrow BR \\
                        BR\delta &\rightarrow L\delta B \qquad \text{(Coge una $B$)} \\
                        BL &\rightarrow LB \\
                        \sigma L &\rightarrow B\sigma R \qquad \text{(Deja la $B$)}
                    \end{align*}
                \item El proceso terinará cuando no haya más $B$s entre $\sigma$ y $\delta$ una vez copiada la última $B$ (con lo que tendremos la variable $R$). En dicho caso, colocamos la nueva $I$ que delimita la separación con el grupo de $B$s recién creado y usamos $\overline{E}$ para devolver $\delta$ al final del último grupo de $B$s:
                    \begin{align*}
                        \sigma R\delta &\rightarrow I\overline{E} \\
                        \overline{E}B &\rightarrow B\overline{E} \\
                        \overline{E}\gamma &\rightarrow E'\delta\gamma
                    \end{align*}
                \item Una vez colocada $\delta$, volveremos con $E'$ hacia la izquierda hasta la situación inicial de $E$, que es tras $\varphi$:
                    \begin{align*}
                        BE' &\rightarrow E'B \\
                        IE' &\rightarrow E'I \\
                        \beta E' &\rightarrow E'\beta \\
                        bE' &\rightarrow E'b \\
                        \varphi E' &\rightarrow E\varphi \\
                    \end{align*}
            \end{enumerate}
            Como hemos mencionado anteriormente, la variable $E$ repetirá este proceso, hasta quedarse sin $b$s por realizar su cuadrado. Hasta este punto, la palabra generada será similar a:
            \begin{equation*}
                \lm A\ldots Ab \alpha E\varphi b\ldots b\beta IB\ldots BIB\ldots BI \ldots I B \ldots B \delta\gamma
            \end{equation*}
            donde tenemos tantas $B$s como número de $b$s al cuadrado (gracias al funcionamiento de la gramáica de los cuadrados perfectos). A continuación, usaremos la variable $Z$ para borrar las variables que ya no nos hacen falta, como $\alpha$, $\varphi$, $\beta$, $I$ y $\delta$:
            \begin{align*}
                \alpha E\varphi &\rightarrow Z \qquad \text{(No quedan $b$s por copiar)} \\
                Zb &\rightarrow bZ \\
                Z\beta &\rightarrow \beta Z \\
                ZI &\rightarrow Z \\
                ZB &\rightarrow BZ \\
                Z\delta &\rightarrow H
            \end{align*}
            Ahora, usaremos $H$ para cambiar una $A$ por una $a$ por cada $B$ que borremos:
            \begin{enumerate}
                \item En primer lugar, borramos una $B$:
                    \begin{equation*}
                        BBH \rightarrow H'B \qquad \text{(No es la última)}
                    \end{equation*}
                \item A continuación, nos desplazamos hacia la izquierda, buscando la primera $A$:
                    \begin{align*}
                        BH' &\rightarrow H'B \\
                        \beta H' &\rightarrow H'\beta \\
                        bH' &\rightarrow H'b \\
                        aH' &\rightarrow H'a \qquad \text{(Puede que ya hayamos sustituido alguna)} \\
                    \end{align*}
                \item Cuando encontremos la primera $A$, la cambiamos por una $a$ y volvemos hacia atrás buscando $\gamma$. Para ello, reutilizaremos $Z$, añadiendo una nueva regla a $Z$:
                    \begin{align*}
                        AH' &\rightarrow aZ \\
                        Z\gamma &\rightarrow H\gamma
                    \end{align*}
            \end{enumerate}
            $H$ realizará este procedimiento de forma iterativa, hasta llegar a la última $B$, operación sensible, por lo que para realizarla usaremos una nueva variable $\overline{H}$:
            \begin{align*}
                \beta BH &\rightarrow \overline{H}\beta \qquad \text{(Cojo la última)} \\
                b\overline{H} &\rightarrow \overline{H}b \\
                a\overline{H} &\rightarrow \overline{H}a \\
            \end{align*}
            Como se trata de la última $B$, sólo vamos a sustituir una $A$ en caso de que sea la última, con lo que:
            \begin{equation*}
                \lm A\overline{H} \rightarrow a\psi
            \end{equation*}
            Donde $\psi$ es la última variable, la cual se encarga de limpiar toda la palabra limpiando las variables.

            Hemos tenido en cuenta el caso en el que $n=m^2$, pero faltan dos casos por considerar:
            \begin{itemize}
                \item $n<m^2$, es decir, hay más $B$s que $A$s. En dicho caso, la última $A$ a sustituir no será consecuencia de quitar la última $B$, sino una anterior, con lo que quedará una $B$ por eliminar (que podrá ser o no la última) que no tenga una $A$ para sustituir. Como $n<m^2$, la palabra es válida. Añadimos por tanto las reglas:
                    \begin{align*}
                        \lm H' &\rightarrow \psi \\
                        \lm \overline{H} &\rightarrow \psi \\
                    \end{align*}
                    Y la variable $\psi$ deberá ser la encargada de eliminar las $B$s sobrantes.
                \item $n>m^2$, es decir, hay más $A$s que $B$s. En dicho caso, cuando borremos la última $B$, no podremos sustituir su $A$ asociada, ya que la regla para realizar esto es $\lm A\overline{H}\rightarrow a\psi$, con lo que es necesario que sólo quede una $A$, cosa que no sucede. En dicho caso, nos quedaremos con una palabra de la forma:
                    \begin{equation*}
                        \lm A\ldots A\overline{H} b\ldots b\beta\gamma
                    \end{equation*}
                    que será imposible sustituir por un símbolo terminal (no tenemos ninguan regla que lo permita). De esta forma, la gramática impide generar palabras que no se encuentren en el lenguaje.
            \end{itemize}
            Finalmente, mostramos el funcionamiento de la variable $\psi$, cuya única funcionalidad es eliminar las variables $\beta$ y $\gamma$ una vez sustituidas todas las $A$s por $a$s (en caso de que la palabra sea válida), y eliminando también las posibles $B$s sobrantes:
            \begin{align*}
                \psi a &\rightarrow a \psi \\
                \psi b &\rightarrow b \psi \\
                \psi \beta &\rightarrow  \psi \\
                \psi B &\rightarrow  \psi \\
                \psi \gamma &\rightarrow  \veps \qquad \text{(Hemos terminado)}
            \end{align*}
    Para esta gramática no mostraremos un ejemplo de su funcionamiento. Si algún lector está dispuesto a realizar un ejemplo de generación de una palabra perteneciente al lenguaje o el intento de una palabra que no pertenezca al lenguaje (con la finalidad de ver que dicha palabra no podrá ser generada), será de nuestro agrado subir el ejemplo a este documento.
    \end{enumerate}
\end{ejercicio}

\subsection{Preguntas Tipo Test}
Se pide discutir la veracidad o falsedad de las siguientes afirmaciones:
\begin{enumerate}
    \item Si un lenguaje es generado por una gramática dependiente del contexto, entonces dicho lenguaje no es independiente del contexto.\\

        Falso, los lenguajes generados por gramáticas independientes del contexto están contenidos en los generados por las gramáticas dependientes del contexto, por lo que muchas veces si tenemos un lenguaje generado por una gramática independiente del contexto, podremos encontrar una dependiente del contexto que nos genere el mismo lenguaje. 

        Por ejemplo, el lenguaje $L=\{a^ib \mid i \in \mathbb{N}\}$ puede generarse por una gramática dependiente del contexto como lo es: $G=(V, T, P, S)$ con
        \begin{align*}
            V &= \{S,B\} \\
            T &= \{a,b\} \\
            P &= \left\{\begin{array}{rl}
                    S &\rightarrow aBb \\
                    aBb &\rightarrow aaBb\ |\ \veps
            \end{array}\right\}
        \end{align*}
        Pero sin embargo, podemos dar una gramática regular (y por tanto, independiente de contexto) para este lenguaje, como por ejemplo $G'=(V, T, P', S)$ con:
        \begin{equation*}
            P' = \left\{\begin{array}{rl}
                    S &\rightarrow aS\ |\ B \\
                    B &\rightarrow b
            \end{array}\right\}
        \end{equation*}
    \item Los alfabetos tienen siempre un número finito de elementos, pero los lenguajes, incluso si el alfabeto tiene sólo un símbolo, tienen infinitas palabras.\\

        Falso, dado un alfabeto $A$, el conjunto relacionado con él y que siempre tiene infinitas palabras es el conjunto de todas las palabras de $A$, $A^\ast$; salvo cuando $A$ es vacío.

        Sin embargo, podemos dar un ejemplo de alfabeto con un lenguaje finito:
        \begin{equation*}
            A = \{a\} \qquad L = \{aa\} \subseteq A^\ast
        \end{equation*}
    \item Si $L$ es un lenguaje no vacío, entonces $L^\ast$ es infinito.\\

        Verdadero, ya que si $L$ es no vacío, entonces existirá una palabra $u\in L$, luego $u^i\in L^\ast$ $\forall i \in \mathbb{N}$. Podemos por tanto construir una aplicación inyectiva $f:\mathbb{N}\rightarrow L^\ast$ que asigna $n\in \mathbb{N}$ en $u^n$. Concluimos por tanto que $L^\ast$ es infinito.
    \item Todo lenguaje con un número finito de palabras es regular e independiente del contexto.\\

        Verdadero, como todo lenguaje regular es a su vez independiente del contexto, será suficiente con probar que todo lenguaje finito es regular. Para ello, dado un lenguaje finito cualquiera $L=\{u_1, u_2, \ldots, u_n\}$, podemos construir una gramática generativa de tipo 3 que nos genere dicho lenguaje (suponiendo que el lenguaje contiene palabras de un cierto alfabeto $A$):
        \begin{align*}
            G &= (\{S\}, A, P, S) \\
            P &= \left\{\begin{array}{rl}
                S \rightarrow u_1\ |\ u_2\ |\ \ldots\ |\ u_n
            \end{array}\right\}
        \end{align*}
        Con lo que $L$ será regular.
    \item Si $L$ es un lenguaje, entonces siempre $L^\ast$ es distinto de $L^+$.\\

        Falso, si $\veps \in L$, como por ejemplo ocurre con el lenguaje $L = \{\veps\}$, entonces tendremos que $L^\ast = L^+$.
    \item $L\emptyset =L$.\\

        Falso (salvo si $L=\emptyset $, en cuyo caso es trivial):
        \begin{equation*}
            L\emptyset = \{uv \mid u\in L, v\in \emptyset \}
        \end{equation*}
        En el caso de ser $L\neq \emptyset$, si $u\in L$ estaríamos diciendo que toda palabra de $L$ puede descomponerse en dos, una de $L$ y otra del vacío, lo que nos lleva a una contradicción. Concluimos que $L\emptyset  = \emptyset $.
    \item \label{item:preg7}
    Si $A$ es un afabeto, la aplicación que transforma cada palabra $u\in A^\ast$ en su inversa es un homomorfismo de $A^\ast$ en $A^\ast$.\\

        Falso, como contraejemplo, trabajaremos con el alfabeto $A=\{a,b\}$ y supongamos que dicha aplicación es $f$, por ser homomorfismo, esta debe cumplir que:
        \begin{equation*}
            f(uv) = f(u)f(v) \qquad \forall u,v\in A^\ast
        \end{equation*}
        Sin embargo, si tomamos $u=ab$ y $v=ba$, tenemos:
        \begin{equation*}
            f(uv) = f(abba) = abba \neq baab = f(ab)f(ba) = f(u)f(v)
        \end{equation*}
    \item \label{item:preg8}
    Si $\veps \in L$, entonces $L^+ = L^\ast$.\\

        Verdadero:
        \begin{equation*}
            L^\ast = \bigcup_{i\geq 0} L^i = L^0 \cup L \cup \left(\bigcup_{i\geq 2} L^i\right) \AstIg  L \cup \left(\bigcup_{i\geq 2} L^i\right) = \bigcup_{i\geq 1}L^i = L^+
        \end{equation*}
        Donde en $(\ast)$ hemos aplicado que $L^0 = \{\veps\} \subseteq L = L^1$.
    \item La transformación que a cada palabra sobre $\{0,1\}$ le añade $00$ al principio y $11$ al final es un homomorfismo.\\

        Falso, como contraejemplo, trabajaremos con el alfabeto $A=\{0,1\}$ y supongamos que dicha aplicación es $f$, por ser homomorfismo, esta debería cumplir que:
        \begin{equation*}
            f(uv) = f(u)f(v) \qquad \forall u,v\in A^\ast
        \end{equation*}
        Sin embargo, si tomamos $u = v = 1$:
        \begin{equation*}
            f(uv) = f(11) = 001111 \neq 0011100111 = f(1)f(1) = f(u)f(v)
        \end{equation*}
    \item Se puede construir un programa que tenga como entrada un programa y unos datos y que siempre nos diga si el programa leido termina para esos datos.\\

        Falso, este problema es conocido como el problema de la parada, y es conocido que no es posible construir dicho programa.
    \item La cabecera del lenguaje $L$ siempre incluye a $L$.\\

        Falso, $CAB(L)$  es el conjunto de prefijos de palabras de $L$ y puede suceder que los prefijos de palabras de $L$ no sean palabras de $L$. Como contraejemplo, consideramos $L=\{ab\}$, luego $CAB(L) = \{\veps, a, ab\}\neq L$.

        Notemos que si $\veps \notin L$, entonces $CAB(L)\nsubseteq L$, ya que $\veps \in CAB(L)$ para cualquier lenguaje no vacío $L$.
    \item Un lenguaje nunca puede ser igual a su inverso.\\

        Falso, todos los lenguajes unitarios (que contienen solo una palabra) son iguales a su inverso. Como contraejemplo más elaborado, cualquier lenguaje $L$ que cumpla que
        \begin{equation*}
            L \subseteq \{u\in A^\ast \mid u^{-1} = u\}
        \end{equation*}
        sirve de contraejemplo.
    \item La aplicación que transforma cada palabra $u$ sobre el alfabeto $\{0,1\}$ en $u^3$ es un homomorfismo.\\

        Falso, como contraejemplo, suponemos que la aplicación $f$ transforma cada palabra $u$ en $u^3$. De ser un homomorfismo, se debería cumplir que:
        \begin{equation*}
            f(uv) = f(u)f(v) \qquad \forall u,v\in A^\ast
        \end{equation*}
        Sin embargo, si tomamos $u = 0$ y $v = 1$:
        \begin{equation*}
            f(uv) = f(01) = 010101 \neq 000111 = f(0)f(1) = f(u)f(v)
        \end{equation*}
        De forma general, podemos decir que cualquier aplicación que transforme cada palabra $u$ en $u^n$ para $n \geq 2$ (la identidad sí que es un homomorfismo) no es un homomorfismo, por un razonamiento análogo al anterior.
    \item El lenguaje que contiene sólo la palabra vacía es el elemento neutro para la concatenación de lenguajes.\\

        Verdadero, sea $L_e = \{\veps\}$, recordamos que:
        \begin{equation*}
            L_1L_2 = \{uv \mid u\in L_1, v\in L_2\}
        \end{equation*}
        Por tanto:
        \begin{gather*}
            L_e L = \{\veps v \mid v\in L\} = L \\
            L L_e = \{v\veps \mid v\in L\} = L
        \end{gather*}
        Para cualquier lenguaje $L$.
    \item Si $L$ es un lenguaje, en algunas ocasiones se tiene que $L^\ast = L^+$.\\

        Verdadero, como hemos visto anteriormente, es condición suficiente (se deduce trivialmente que es necesaria, ya que $\veps \in L^\ast$ siempre) que $\veps \in L$.
    \item Hay lenguajes con un número infinito de palabras que no son regulares.\\

        Verdadero, sabemos que los lenguajes con un número finito de palabras son regulares, luego cualquier lenguaje que no sea regular deberá tener un número infinito de palabras. Como sabemos la existencia de lenguajes no regulares, como por ejemplo $L = \{a^ib^ja^i b^j \mid i,j\in \mathbb{N}\}$, entonces estos han de tener un número no finito de palabras.
    \item Si un lenguaje tiene un conjunto infinito de palabras sabemos que no es regular.\\

        Falso, el lenguaje $L=\{a^i \mid i \in \mathbb{N}\setminus \{0\}\}$ tiene un conjunto infinito de palabras y es regular, ya que podemos dar una gramática generativa de tipo 3:\newline $G=(\{S\}, \{a\}, \{S\rightarrow aS, S\rightarrow a\}, S)$ que genera dicho lenguaje.
    \item Si $L$ es un lenguaje finito, entonces su cabecera ($CAB(L)$) también será finita.\\

        Verdadero, supongamos que tenemos un lenguaje finito $L$ formado por $n\in \mathbb{N}$ palabras y recordamos que $CAB(L)$ es el conjunto formado por todos los prefijos de palabras de $L$. Dada una palabra $u\in L$ con $|u| =m$, esta estará formada por $m$ letras del alfabeto $A$:
        \begin{equation*}
            u = a_1a_2\ldots a_{m} \qquad a_i \in A \quad \forall i \in \{1, \ldots, m\}
        \end{equation*}
        Por lo que aceptará $m+1$ prefijos distintos:
        \begin{equation*}
            \veps \qquad a_1 \qquad a_1a_2 \qquad \ldots \qquad a_1a_2\ldots a_{m-1} \qquad a_1a_2\ldots a_m
        \end{equation*}
        No obstante, hemos de tener en cuenta que un mismo prefijo puede ser prefijo de dos palabras distintas de $L$. En particular, como $\veps$ es prefijo de todas las palabras, tendremos que, sin contar $\veps$, cada palabra de $L$ aportará $m$ prefijos distintos. Por tanto, el conjunto $CAB(L)$ tendrá como máximo:
        \begin{equation*}
            |CAB(L)| \leq \left(\sum_{u\in L}|u|\right) +1
        \end{equation*}
        donde el $+1$ se debe a la presencia de $\veps$ en $CAB(L)$. Sea ahora $w\in L$ la palabra de $L$ con mayor longitud, tenemos que
        \begin{equation*}
            |CAB(L)| \leq \left(\sum_{u\in L}|u|\right) + 1 \leq \left(\sum_{i=1}^{n} |w|\right)+1 = n\cdot |w| + 1
        \end{equation*}
    \item El conjunto de palabras sobre un alfabeto dado con la operación de concatenación tiene una estructura de monoide.\\

        Verdadero, recordamos que un par conjunto-operación tiene estructura de monoide si:
        \begin{itemize}
            \item Se trata de una operación interna al conjunto, lo cual es cierto, ya que la concatenación de dos palabras cualquiera es una palabra.
            \item La operación es asociativa, algo que también cumple la concatenación.
            \item Existe un elemento neutro del conjunto para dicha operación, lo cual es cierto, debido a la existencia de $\veps$ en cualquier conjunto de palabras dado por un alfabeto.
        \end{itemize}
    \item La transformación entre el conjunto de palabras del alfabeto $\{0,1\}$ que duplica cada símbolo (la palabra $011$ se transforma en $001111$) es un homomorfismo.\\

        Verdadero, sea el alfabeto $A=\{0,1\}$ y $f:A^\ast\rightarrow A^\ast$ la aplicación enunciada, definida por:
        \begin{equation*}
            f(u) = a_1a_1a_2a_2\ldots a_na_n \qquad \forall u = a_1a_2\ldots a_n \quad a_i \in A \quad \forall i \in \{1,\ldots, n\}
        \end{equation*}
        Para ver que sea un homomorfismo, debemos comprobar que se cumpla
        \begin{equation*}
            f(uv) = f(u)f(v) \qquad \forall u,v\in A^\ast
        \end{equation*}
        Para ello, sean $u,v\in A^\ast$ palabras de longitud $n,m\in \mathbb{N}$ respectiavmente, entonces tendremos que
        \begin{gather*}
            u = a_1a_2 \ldots a_n \qquad a_i \in A \quad \forall i \in \{1,\ldots, n\} \\
            v = b_1b_2 \ldots b_m \qquad b_i \in A \quad \forall i \in \{1,\ldots, m\} 
        \end{gather*}
        De esta forma:
        \begin{multline*}
            f(uv) = f(a_1a_2 \ldots a_nb_1b_2\ldots b_m) = a_1a_2a_2a_2\ldots a_na_nb_1b_1b_2b_2\ldots b_mb_m = \\ = f(a_1a_2\ldots a_n)f(b_1b_2\ldots b_m) = f(u)f(v)
        \end{multline*}
        Con lo que $f$ es un homomorfismo.
    \item Si $f$ es un homomorfismo entre palabras del alfabeto $A_1$ en palabras del alfabeto de $A_2$, entonces si conocemos $f(a)$ para cada $a\in A_1$ se puede calcular $f(u)$ para cada palabra $u\in A_1^\ast$.\\

        Verdadero, sea $f:A_1^\ast\rightarrow A_2^\ast$ un homomorfismo, este cumplirá por tanto que:
        \begin{equation*}
            f(uv) = f(u)f(v) \qquad \forall u,v\in A_1^\ast
        \end{equation*}
        Puede demostrarse fácilmente por inducción que si tenemos una palabra ${u\in A_1^\ast}$ formada por $n$ letras del alfabeto:
        \begin{equation*}
            u = a_1a_2\ldots a_n \qquad a_i \in A \quad \forall i \in \{1,\ldots n\}
        \end{equation*}
        entonces, se tiene que
        \begin{equation*}
            f(u) = f(a_1)f(a_2)\ldots f(a_n)
        \end{equation*}
        Por tanto, el enunciado es cierto.
    \item Si $A$ es un afabeto, la aplicación que transforma cada palabra $u\in A^\ast$ en su inversa es un homomorfismo de $A^\ast$ en $A^\ast$.\\

        Falso, la pregunta es idéntica a la pregunta~\ref{item:preg7}.
    \item Si $\veps \in L$, entonces $L^+ = L^\ast$.\\

        Verdadero, la pregunta es idéntica a la pregunta~\ref{item:preg8}.
    \item Si $f$ es un homomorfismo, entonces necesariamente se verifica $f(\veps) = \veps$.\\
        
        Verdadero, sea $f:A^\ast\rightarrow A^\ast$ un homomorfismo y consideramos $u\in A^\ast$. Por ser $f$ un homomorfismo, tenemos que:
        \begin{equation*}
            f(u) = f(\veps u) = f(\veps)f(u)
        \end{equation*}
        Con lo que $f(\veps) = \veps$.
    \item Si $A$ es un alfabeto, entonces $A^+$ no incluye nunca la palabra vacía.\\
        \begin{description}
            \item[Considerando $A$ solamente como alfabeto] Verdadero. Por definición del operador $^+$ aplicado a alfabetos, tenemos que:
            \begin{equation*}
                A^+ = A^*\setminus \{\veps\}
            \end{equation*}

            \item[Considerando $A$ como lenguaje] Verdadero. Por definición del operador $^+$ aplicado a lenguajes, tenemos que:
            \begin{equation*}
                A^+ = \bigcup_{i\geq 1} A^i
            \end{equation*}

            Por tanto, si $u\in A^+$, entonces $\exists n\in \bb{N}$ tal que $u\in A^n$, por lo que $u=a_1a_2\ldots a_n$ con $a_i\in A$ $\forall i\in \{1,\ldots,n\}$. Como $a_i\in A$, entonces $a_i\neq \veps$ $\forall i\in \{1,\ldots,n\}$, con lo que $u\neq \veps$.
        \end{description}
    \item Es posible diseñar un algoritmo que lea un lenguaje cualquiera sobre el alfabeto $\{0,1\}$ y nos diga si es regular o no.
    
    Esto es falso, y se verá en la asignatura de Modelos Avanzados de Computación.
\end{enumerate}

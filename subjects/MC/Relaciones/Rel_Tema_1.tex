\section{Introducción a la Computación}


\begin{ejercicio}
    Sea la gramática $G=\left(V,T,P,S\right)$ dada por:
    \begin{align*}
        V &= \{S, X, Y\} \\
        T &= \{a,b\} \\
        S &= S
    \end{align*}
    \begin{enumerate}
        \item Describe el lenguaje generado por la gramática teniendo en cuenta que $P$ viene descrito por:
        \begin{align*}
            S &\rightarrow XYX \\
            X &\rightarrow aX \mid bX \mid \veps \\
            Y &\rightarrow bbb
        \end{align*}

        Sea $L=\{ubbbv\mid u,v\in\{a,b\}^\ast\}$. Demostraremos mediante doble inclusión que $L=\cc{L}(G)$.
        \begin{description}
            \item[$\subset)$] Sea $w\in L$. Entonces, $w=ubbbv$ con $u,v\in\{a,b\}^\ast$. Veamos que
            $S \stackrel{\ast}{\Longrightarrow} w$:
            \begin{equation*}
                S \Longrightarrow XYX \Longrightarrow XbbbX
            \end{equation*}

            Además, es fácil ver que la regla de producción $X\rightarrow aX \mid bX \mid \veps$ nos permite generar cualquier palabra $u\in\{a,b\}^\ast$. Por tanto, tenemos que $X \stackrel{\ast}{\Longrightarrow} u$ y $X \stackrel{\ast}{\Longrightarrow} v$; teniendo así que $S \stackrel{\ast}{\Longrightarrow} ubbbv$.

            \item[$\supset)$] Sea $w\in\cc{L}(G)$. Veamos la forma de $w$:
            \begin{equation*}
                S \Longrightarrow XYX \Longrightarrow XbbbX \Longrightarrow ubbbv \mid u,v\in\{a,b\}^\ast
            \end{equation*}
            donde en el último paso hemos empleado lo visto en el apartado anterior de la regla de producción $X\rightarrow aX \mid bX \mid \veps$. Por tanto, $w\in L$.
        \end{description}


        \item Describe el lenguaje generado por la gramática teniendo en cuenta que $P$ viene descrito por:
        \begin{align*}
            S &\rightarrow aX \\
            X &\rightarrow aX \mid bX \mid \veps
        \end{align*}

        Sea $L=\{au \mid u\in\{a,b\}^\ast\}$. Demostraremos mediante doble inclusión que $L=\cc{L}(G)$.
        \begin{description}
            \item[$\subset)$] Sea $w\in L$. Entonces, $w=au$ con $u\in\{a,b\}^\ast$. Veamos que
            $S \stackrel{\ast}{\Longrightarrow} w$:
            \begin{equation*}
                S \Longrightarrow aX \Longrightarrow au
            \end{equation*}
            donde en el último paso hemos empleado lo visto respecto a la regla de producción $X\rightarrow aX \mid bX \mid \veps$. Por tanto, $w\in \cc{L}(G)$.

            \item[$\supset)$] Sea $w\in\cc{L}(G)$. Veamos la forma de $w$:
            \begin{equation*}
                S \Longrightarrow aX \Longrightarrow au \mid u\in\{a,b\}^\ast
            \end{equation*}
            donde en el último paso hemos empleado lo visto respecto a la regla de producción $X\rightarrow aX \mid bX \mid \veps$. Por tanto, $w\in L$.
        \end{description}

        \item Describe el lenguaje generado por la gramática teniendo en cuenta que $P$ viene descrito por:
        \begin{align*}
            S &\rightarrow XaXaX \\
            X &\rightarrow aX \mid bX \mid \veps
        \end{align*}

        Sea $L=\{uavawa \mid u,v,w\in\{a,b\}^\ast\}$. Demostraremos mediante doble inclusión que $L=\cc{L}(G)$.
        \begin{description}
            \item[$\subset)$] Sea $z\in L$. Entonces, $z=uavawa$ con $u,v,w\in\{a,b\}^\ast$. Veamos que
            $S \stackrel{\ast}{\Longrightarrow} z$:
            \begin{equation*}
                S \Longrightarrow XaXaX \Longrightarrow uavawa
            \end{equation*}
            donde en el último paso hemos empleado lo visto respecto a la regla de producción $X\rightarrow aX \mid bX \mid \veps$. Por tanto, $z\in \cc{L}(G)$.

            \item[$\supset)$] Sea $z\in\cc{L}(G)$. Veamos la forma de $z$:
            \begin{equation*}
                S \Longrightarrow XaXaX \Longrightarrow uavawa \mid u,v,w\in\{a,b\}^\ast
            \end{equation*}
            donde en el último paso hemos empleado lo visto respecto a la regla de producción $X\rightarrow aX \mid bX \mid \veps$. Por tanto, $z\in L$.
        \end{description}

        \item Describe el lenguaje generado por la gramática teniendo en cuenta que $P$ viene descrito por:
        \begin{align*}
            S &\rightarrow SS \mid XaXaX \mid \veps \\
            X &\rightarrow bX \mid \veps
        \end{align*}

        Sea el siguiente lenguaje:
        % // TODO: Sé que es tantas b seguidas de una a, todas las veces que quiera. Cómo lo noto?
        \begin{comment}
        $$L=\{b^{i_1} a b^{i_2} a b^{i_3} \mid i_k\in \bb{N}\cup \{0\},~k\in \{0,1,2\}\} \cup
        \{b^{j_1} a b^{j_2} a b^{j_3} a a b^{j_4} a b^{j_5} \mid i_k\in \bb{N}\cup \{0\},~k\in \{0,1,2,3\}\}$$
        Demostraremos mediante doble inclusión que $L=\cc{L}(G)$.
        \begin{description}
            \item[$\subset)$] Sea $z\in L$. Entonces, $z=(b^{i_1} a b^{i_2} a b^{i_3})^n$ con $i_1,i_2,i_3\in \bb{N}\cup \{0\}$ y, además, $n\in \{0,1,2\}$. Veamos que
            $S \stackrel{\ast}{\Longrightarrow} z$ distinguiendo casos:
            \begin{itemize}
                \item \ul{Caso $n=0$}: $z=\veps$. Entonces, $S \Longrightarrow \veps$, y tenemos que $z\in \cc{L}(G)$.
                \item \ul{Caso $n=1$}: $z=b^{i_1} a b^{i_2} a b^{i_3}$. Veamos que $S \stackrel{\ast}{\Longrightarrow} z$. En primer lugar, tenemos que:
                \begin{equation*}
                    S \Longrightarrow XaXaX 
                \end{equation*}
                Respecto a cada uno de los $i_k$ ($k=1,2,3$), hacemos lo siguiente (el ejemplo es para $i_1$, ya que para $i_2$ e $i_3$ es análogo):
                \begin{equation*}
                    S \Longrightarrow XaXaX \stackrel{i_1\text{ veces}}{\Longrightarrow} b^{i_1} XaXaX
                    \Longrightarrow b^{i_1} aXaX
                \end{equation*}
                Como hemos indicado, repetiríamos para $i_2$ e $i_3$, llegando a que $S \stackrel{\ast}{\Longrightarrow} b^{i_1} a b^{i_2} a b^{i_3}$. Por tanto, $z\in \cc{L}(G)$.
            \end{itemize}

            \item[$\supset)$] Sea $z\in\cc{L}(G)$. Veamos la forma de $z$:
            \begin{equation*}
                S \Longrightarrow SS \Longrightarrow XaXaX \Longrightarrow b^{i_1} a b^{i_2} a b^{i_3} \mid i_1,i_2,i_3\in \bb{N}\cup 0
            \end{equation*}
            donde en el último paso hemos empleado lo visto respecto a la regla de producción $X\rightarrow bX \mid \veps$. Por tanto, $z\in L$.
        \end{description}
        \end{comment}

    \end{enumerate}
\end{ejercicio}



\begin{ejercicio}
    Sea la gramática $G=\left(V,T,P,S\right)$. Determinar en cada caso el lenguaje generado por la gramática.
    \begin{enumerate}
        \item Tenga en cuenta que:
        \begin{align*}
            V &= \{S,A\}\\
            T &= \{a,b\}\\
            S &= S \\
            P &= \left\{
                \begin{array}{rcl}
                    S &\rightarrow & abAS \mid a \\
                    abA &\rightarrow & baab \\
                    A &\rightarrow & b
                \end{array}
            \right\}
        \end{align*}

        \item Tenga en cuenta que:
        \begin{align*}
            V &= \{\langle \text{número} \rangle, \langle \text{dígito} \rangle\} \\
            T &= \{0,1,2,3,4,5,6,7,8,9\} \\
            S &= \langle \text{número} \rangle \\
            P &= \left\{
                \begin{array}{rcl}
                    \langle \text{número} \rangle &\rightarrow & \langle \text{número} \rangle \langle \text{dígito} \rangle \\
                    \langle \text{número} \rangle &\rightarrow & \langle \text{dígito} \rangle \\
                    \langle \text{dígito} \rangle &\rightarrow & 0 \mid 1 \mid 2 \mid 3 \mid 4 \mid 5 \mid 6 \mid 7 \mid 8 \mid 9
                \end{array}
            \right\}
        \end{align*}

        Tenemos que $\cc{L}(G)$ es el conjunto de los números naturales, permitiendo
        tantos ceros a la izquierda como se quiera. Es decir (usando la notación de potencia y concatenación vista para lenguajes):
        \begin{equation*}
            L = \{0^i n \mid i\in \bb{N}\cup\{0\},~n\in \bb{N}\cup \{0\}\}
        \end{equation*}
        Demostrémoslo mediante doble inclusión que $L=\cc{L}(G)$.
        \begin{description}
            \item[$\subset)$] Sea $w\in L$. Entonces, $w=0^i n$ con $i\in \bb{N}\cup\{0\}$ y $n\in \bb{N}\cup \{0\}$. Veamos que
            $\langle \text{número} \rangle \stackrel{\ast}{\Longrightarrow} w$:
            \begin{itemize}
                \item En primer lugar, aplicamos $|w|-1$ veces la regla de producción $\langle \text{número} \rangle \rightarrow \langle \text{número} \rangle \langle \text{dígito} \rangle$ y la regla
                que lleva de $\langle \text{dígito} \rangle$ a uno de los símbolos terminales, consiguiendo así en cada etapa reemplazar
                la última variable presente en la cadena por un dígito.
                \item Finalmente, aplicamos la regla de producción $\langle \text{número} \rangle \rightarrow \langle \text{dígito} \rangle$ para reemplazar la última variable por un dígito, que será el primero del número formado.
            \end{itemize}
            Por tanto, $\langle \text{número} \rangle \stackrel{\ast}{\Longrightarrow} w$, teniendo que $w\in \cc{L}(G)$.

            \item[$\supset)$] Sea $w\in\cc{L}(G)$. Como la única regla que
            aumenta la longitud es la regla de producción $\langle \text{número} \rangle \rightarrow \langle \text{número} \rangle \langle \text{dígito} \rangle$, tenemos que $w$ tiene la forma:
            \begin{align*}
                \langle \text{número} \rangle &\Longrightarrow \langle \text{número} \rangle \langle \text{dígito} \rangle \stackrel{|w|-1\text{ veces}}{\Longrightarrow} \\
                &\Longrightarrow
                \langle \text{número} \rangle \langle \text{dígito} \rangle \langle \text{dígito} \rangle \stackrel{|w|-1\text{ veces}}{\cdots} \langle \text{dígito} \rangle
                \Longrightarrow \\& \Longrightarrow
                \langle \text{dígito} \rangle \stackrel{|w|\text{ veces}}{\cdots} \langle \text{dígito} \rangle
            \end{align*}
            Por tanto, tenemos que se trata una sucesión de $|w|$ dígitos, lo que nos lleva a que $w\in L$.
        \end{description}

        \item Tenga en cuenta que:
        \begin{align*}
            V &= \{A,S\} \\
            T &= \{a,b\} \\
            S &= S \\
            P &= \left\{
                \begin{array}{rcl}
                    S &\rightarrow & aS \mid aA \\
                    A &\rightarrow & bA \mid b
                \end{array}
            \right\}
        \end{align*}

        Sea $L=\{a^nb^m \in \{a,b\}^\ast \mid n, m \in \bb{N}\}$. Demostraremos mediante doble inclusión que $L=\cc{L}(G)$.
        \begin{description}
            \item[$\subset)$] Sea $w\in L$. Entonces, $w=a^nb^m$ con $n,m\in \bb{N}$. Veamos que
            $S \stackrel{\ast}{\Longrightarrow} w$:
            \begin{itemize}
                \item En primer lugar, aplicamos $n-1$ veces la regla de producción $S \rightarrow aS$ para obtener $a^{n-1}S$,
                \begin{equation*}
                    S \stackrel{\ast}{\Longrightarrow} a^{n-1}S
                \end{equation*}

                \item Para cambiar a la etapa de añadir $b$'s, aplicamos la regla de producción $S \rightarrow aA$, obteniendo así $a^{n}A$,
                \item Después, aplicamos $m-1$ veces la regla de producción $A \rightarrow bA$ para obtener $a^nb^{m-1}A$.
                \item Para finalizar, aplicamos la regla de producción $A \rightarrow b$ para obtener $a^nb^m$.
            \end{itemize}
            Por tanto, $S \stackrel{\ast}{\Longrightarrow} w$, teniendo que $w\in \cc{L}(G)$.

            \item[$\supset)$] Sea $w\in\cc{L}(G)$. Vemos que en la palabra siempre
            va a haber tan solo una variable (ya sea $S$ o $A$). Se empezará con la $S$, y en cierto momento se cambiará a la $A$,
            sin poder entonces volver a la $S$.
            \begin{itemize}
                \item Cuando se está en la etapa en la que hay $S$, tan solo se pueden añadir $a$'s,
                o bien cambiar a la $A$.
                \item Cuando se está en la etapa en la que hay $A$, tan solo se pueden añadir $b$'s.
            \end{itemize}
            Por tanto, tenemos que $w$ estará formada por una sucesión de
            $a$'s seguida de una sucesión de $b$'s, lo que nos lleva a que $w\in L$.
        \end{description}
    \end{enumerate}
\end{ejercicio}

\begin{ejercicio}
    Encontrar gramáticas de tipo 2 para los siguientes lenguajes sobre el alfabeto $\{a, b\}$. En cada caso determinar si los lenguajes generados son de tipo 3, estudiando si existe una gramática de tipo 3 que los genera.
    \begin{enumerate}
        \item Palabras en las que el número de $b$ no es tres.
        \item Palabras que tienen 2 ó 3 $b$.
    \end{enumerate}
\end{ejercicio}

\begin{ejercicio}
    Encontrar gramáticas de tipo 2 para los siguientes lenguajes sobre el alfabeto $\{a, b\}$. En cada caso determinar si los lenguajes generados son de tipo 3, estudiando si existe una gramática de tipo 3 que los genera.
    \begin{enumerate}
        \item Palabras que no contienen la subcadena $ab$.
        \item Palabras que no contienen la subcadena $baa$.
    \end{enumerate}
\end{ejercicio}

\begin{ejercicio}
    Encontrar una gramática libre de contexto que genere el lenguaje sobre el alfabeto $\{a, b\}$ de las palabras que tienen más $a$ que $b$ (al menos una más).
\end{ejercicio}

\begin{ejercicio}
    Encontrar, si es posible, una gramática regular (o, si no es posible, una gramática libre del contexto) que genere el lenguaje $L$ supuesto que $L \subset \{a, b\}^\ast$ y verifica:
    \begin{enumerate}
        \item $u \in L$ si, y solamente si, verifica que $u$ no contiene dos símbolos $b$ consecutivos.
        \item $u \in L$ si, y solamente si, verifica que $u$ contiene dos símbolos $b$ consecutivos.
    \end{enumerate}
\end{ejercicio}

\begin{ejercicio}
    Encontrar, si es posible, una gramática regular (o, si no es posible, una gramática libre del contexto) que genere el lenguaje $L$ supuesto que $L \subset \{a, b\}^\ast$ y verifica:
    \begin{enumerate}
        \item $u \in L$ si, y solamente si, verifica que contiene un número impar de símbolos $a$.
        \item $u \in L$ si, y solamente si, verifica que no contiene el mismo número de símbolos $a$ que de símbolos $b$.
    \end{enumerate}
\end{ejercicio}


\begin{ejercicio}
    Dado el alfabeto $A = \{a, b\}$ determinar si es posible encontrar una gramática libre de contexto que:
    \begin{enumerate}
        \item Genere las palabras de longitud impar, y mayor o igual que 3, tales que la primera letra coincida con la letra central de la palabra.
        \item Genere las palabras de longitud par, y mayor o igual que 2, tales que las dos letras centrales coincidan.
    \end{enumerate}
\end{ejercicio}

\begin{ejercicio}
    Sea la gramática $G=\left(V,T,P,S\right)$ dada por:
    \begin{align*}
        V &= \{S, X\} \\
        T &= \{a,b\} \\
        S &= S\\
        P &= \left\{
            \begin{array}{rcl}
                S &\rightarrow & SS \\
                S &\rightarrow & XXX \\
                X &\rightarrow & aX \mid Xa \mid b
            \end{array}
        \right\}
    \end{align*}
    Determinar si el lenguaje generado por la gramática es regular. Justificar la respuesta.
\end{ejercicio}

\begin{ejercicio}
    Dado un lenguaje $L$ sobre un alfabeto $A$, ¿es $L^{\ast}$ siempre numerable? ¿nunca lo es? ¿o puede serlo unas veces sí y otras, no? Pon ejemplos en este último caso.
\end{ejercicio}

\begin{ejercicio}
    Dado un lenguaje $L$ sobre un alfabeto $A$, caracterizar cuando $L^{\ast} = L$. Esto es, dar un conjunto de propiedades sobre $L$ de manera que $L$ cumpla esas propiedades si y sólo si $L^{\ast} = L$.
\end{ejercicio}

\begin{ejercicio}
    Dados dos homomorfismos $f : A^{\ast} \rightarrow B^{\ast}$, $g : A^{\ast} \rightarrow B^{\ast}$, se dice que son iguales si $f(x) = g(x)$, $\forall x \in A^{\ast}$. ¿Existe un procedimiento algorítmico para comprobar si dos homomorfismos son iguales?
\end{ejercicio}

\begin{ejercicio}
    Sea $L \subseteq A^{\ast}$ un lenguaje arbitrario. Sea $C_0 = L$ y definamos los lenguajes $S_i$ y $C_i$, para todo $i \geq 1$, por $S_i = C_{i-1}^+$ y $C_i = \ol{S_i}$. 
    \begin{enumerate}
        \item ¿Es $S_1$ siempre, nunca o a veces igual a $C_2$? Justifica la respuesta.
        \item Demostrar que $S_2 = C_3$, cualquiera que sea $L$.
        \begin{observacion}
            Demuestra que $C_2$ es cerrado para la concatenación.
        \end{observacion}
    \end{enumerate}
\end{ejercicio}

\begin{ejercicio}
    Demuestra que, para todo alfabeto $A$, el conjunto de los lenguajes finitos sobre dicho alfabeto es numerable.
\end{ejercicio}




\subsection{Cálculo de gramáticas}

\begin{ejercicio}[Complejidad: Sencilla]
    Calcula, de forma razonada, gramáticas que generen cada uno de los siguientes lenguajes:
    \begin{enumerate}
        \item $\{ u\in \{0,1\}^\ast \mid |u|\leq 4 \}$
        \item Palabras con 0's y 1's que no contengan dos 1's consecutivos y que empiecen por un 1 y que terminen por dos 0's.
        \item El conjunto vacío.
        \item El lenguaje formado por los números naturales.
        \item $\{ a^n \in \{a,b\}^\ast \mid n\geq 0 \} \cup \{ a^nb^n \in \{a,b\}^\ast \mid n\geq 0 \}$
        \item $\{ a^nb^{2n}c^m \in \{a,b,c\}^\ast \mid n,m>0 \}$
        \item $\{ a^nb^ma^n \in \{a,b\}^\ast \mid m,n\geq 0 \}$
        \item Palabras con 0's y 1's que contengan la subcadena 00 y 11.
        \item Palíndromos formados con las letras $a$ y $b$.
    \end{enumerate}
\end{ejercicio}

\begin{ejercicio}[Complejidad: Media]
    Calcula, de forma razonada, gramáticas que generen cada uno de los siguientes lenguajes:
    \begin{enumerate}
        \item $\{uv \in \{0,1\}^\ast \mid u^{-1} \text{ es un prefijo de } v\}$
        \item $\{ucv \in \{a,b,c\}^\ast \mid |u| = |v|\}$
        \item $\{u1^n \in \{0,1\}^\ast \mid |u| = n\}$
        \item $\{a^nb^{n+1} \in \{a,b\}^\ast \mid n\geq 0\}$ (observar transparencias de teoría)
    \end{enumerate}
\end{ejercicio}


\begin{ejercicio}[Complejidad: Difícil]
    Calcula, de forma razonada, gramáticas que generen cada uno de los siguientes lenguajes:
    \begin{enumerate}
        \item $\{a^nb^mc^k \in \{a,b,c\}^\ast \mid k = m + n\}$
        \item Palabras que son múltiplos de 7 en binario.
    \end{enumerate}
\end{ejercicio}


\begin{ejercicio}[Complejidad: Extrema (no son libres de contexto)]
    Calcula, de forma razonada, gramáticas que generen cada uno de los siguientes lenguajes:
    \begin{enumerate}
        \item $\{ww \mid w \in \{0,1\}^\ast\}$
        \item $\{a^{n^2} \in \{a\}^{\ast} \mid n\geq 0\}$
        \item $\{a^p \in \{a\}^{\ast} \mid p \text{ es primo}\}$
        \item $\{a^nb^m \in \{a,b\}^{\ast} \mid n\leq m^2\}$
    \end{enumerate}

\end{ejercicio}


\section{Introducción a la Computación}


\begin{ejercicio}
    Sea la gramática $G=\left(V,T,P,S\right)$ dada por:
    \begin{align*}
        V &= \{S, X, Y\} \\
        T &= \{a,b\} \\
        S &= S
    \end{align*}
    \begin{enumerate}
        \item Describe el lenguaje generado por la gramática teniendo en cuenta que $P$ viene descrito por:
        \begin{align*}
            S &\rightarrow XYX \\
            X &\rightarrow aX \mid bX \mid \veps \\
            Y &\rightarrow bbb
        \end{align*}

        Sea $L=\{ubbbv\mid u,v\in\{a,b\}^\ast\}$. Demostraremos mediante doble inclusión que $L=\cc{L}(G)$.
        \begin{description}
            \item[$\subset)$] Sea $w\in L$. Entonces, $w=ubbbv$ con $u,v\in\{a,b\}^\ast$. Veamos que
            $S \stackrel{\ast}{\Longrightarrow} w$:
            \begin{equation*}
                S \Longrightarrow XYX \Longrightarrow XbbbX
            \end{equation*}

            Además, es fácil ver que la regla de producción $X\rightarrow aX \mid bX \mid \veps$ nos permite generar cualquier palabra $u\in\{a,b\}^\ast$. Por tanto, tenemos que $X \stackrel{\ast}{\Longrightarrow} u$ y $X \stackrel{\ast}{\Longrightarrow} v$; teniendo así que $S \stackrel{\ast}{\Longrightarrow} ubbbv$.

            \item[$\supset)$] Sea $w\in\cc{L}(G)$. Veamos la forma de $w$:
            \begin{equation*}
                S \Longrightarrow XYX \Longrightarrow XbbbX \Longrightarrow ubbbv \mid u,v\in\{a,b\}^\ast
            \end{equation*}
            donde en el último paso hemos empleado lo visto en el apartado anterior de la regla de producción $X\rightarrow aX \mid bX \mid \veps$. Por tanto, $w\in L$.
        \end{description}


        \item Describe el lenguaje generado por la gramática teniendo en cuenta que $P$ viene descrito por:
        \begin{align*}
            S &\rightarrow aX \\
            X &\rightarrow aX \mid bX \mid \veps
        \end{align*}

        Sea $L=\{au \mid u\in\{a,b\}^\ast\}$. Demostraremos mediante doble inclusión que $L=\cc{L}(G)$.
        \begin{description}
            \item[$\subset)$] Sea $w\in L$. Entonces, $w=au$ con $u\in\{a,b\}^\ast$. Veamos que
            $S \stackrel{\ast}{\Longrightarrow} w$:
            \begin{equation*}
                S \Longrightarrow aX \Longrightarrow au
            \end{equation*}
            donde en el último paso hemos empleado lo visto respecto a la regla de producción $X\rightarrow aX \mid bX \mid \veps$. Por tanto, $w\in \cc{L}(G)$.

            \item[$\supset)$] Sea $w\in\cc{L}(G)$. Veamos la forma de $w$:
            \begin{equation*}
                S \Longrightarrow aX \Longrightarrow au \mid u\in\{a,b\}^\ast
            \end{equation*}
            donde en el último paso hemos empleado lo visto respecto a la regla de producción $X\rightarrow aX \mid bX \mid \veps$. Por tanto, $w\in L$.
        \end{description}

        \item Describe el lenguaje generado por la gramática teniendo en cuenta que $P$ viene descrito por:
        \begin{align*}
            S &\rightarrow XaXaX \\
            X &\rightarrow aX \mid bX \mid \veps
        \end{align*}

        Sea $L=\{uavawa \mid u,v,w\in\{a,b\}^\ast\}$. Demostraremos mediante doble inclusión que $L=\cc{L}(G)$.
        \begin{description}
            \item[$\subset)$] Sea $z\in L$. Entonces, $z=uavawa$ con $u,v,w\in\{a,b\}^\ast$. Veamos que
            $S \stackrel{\ast}{\Longrightarrow} z$:
            \begin{equation*}
                S \Longrightarrow XaXaX \Longrightarrow uavawa
            \end{equation*}
            donde en el último paso hemos empleado lo visto respecto a la regla de producción $X\rightarrow aX \mid bX \mid \veps$. Por tanto, $z\in \cc{L}(G)$.

            \item[$\supset)$] Sea $z\in\cc{L}(G)$. Veamos la forma de $z$:
            \begin{equation*}
                S \Longrightarrow XaXaX \Longrightarrow uavawa \mid u,v,w\in\{a,b\}^\ast
            \end{equation*}
            donde en el último paso hemos empleado lo visto respecto a la regla de producción $X\rightarrow aX \mid bX \mid \veps$. Por tanto, $z\in L$.
        \end{description}

        \item Describe el lenguaje generado por la gramática teniendo en cuenta que $P$ viene descrito por:
        \begin{align*}
            S &\rightarrow SS \mid XaXaX \mid \veps \\
            X &\rightarrow bX \mid \veps
        \end{align*}

        Sea el lenguaje $L=\{b^i a b^j a b^k \mid i,j,k\in \bb{N}\cup \{0\}\}$. Demostraremos mediante doble inclusión que $L^\ast=\cc{L}(G)$.
        \begin{description}
            \item[$\subset)$] Sea $z\in L^\ast=\bigcup\limits_{i\in \bb{N}} L^i$.
            Sea $n$ el menor número natural tal que $z\in L^n$.
            Notando por $n_a(z)$ al número de $a$'s en $z$, tenemos que $n_a(z)=2n$.
            Entonces, $z\in L\cdot \ldots \cdot L$ ($n$ veces), por lo que existen
            $i_1,j_1,k_1,\ldots,i_n,j_n,k_n\in \bb{N}\cup \{0\}$ tales que $z=b^{i_1} a b^{j_1} a b^{k_1} \cdot \ldots \cdot b^{i_n} a b^{j_n} a b^{k_n}$. Veamos que
            $S \stackrel{\ast}{\Longrightarrow} z$:
            \begin{itemize}
                \item Para conseguir el número de $a$'s deseado, empleamos la regla de producción $S \rightarrow SS$ y reemplazamos una de las $S$ por $XaXaX$. Esto lo hacemos $n$ veces.
                \item Posteriormente, cada $X$ la sustituiremos tantas veces como sea necesario por $bX$ para conseguir el número de $b$'s deseado en cada posición, y finalizaremos con $X\rightarrow \veps$.
            \end{itemize}

            \item[$\supset)$] Sea $z\in\cc{L}(G)$, y sea $n_a(z)$ el número de $a$'s en $z$. Entonces, como el número de $a$ siempre aumenta de dos en dos, tenemos que $n_a(z)=2n$ para algún $n\in \bb{N}\cup \{0\}$.
            Veamos la forma de $z$:
            \begin{itemize}
                \item Para llegar a $z$, hemos tenido que emplear la regla de producción $S \rightarrow SS\rightarrow SXaXaX$ $n$ veces. Una vez llegados aquí, para eliminar la $S$ (ya que habremos llegado a $n_a(z)$ $a$'s), empleamos la regla de producción $S\rightarrow \veps$.
                \item Posteriormente, para cada $X$, tan solo podemos emplear la regla de producción $X\rightarrow bX \mid \veps$ para conseguir el número de $b$'s deseado en cada posición.
            \end{itemize}
            Por tanto, es directo ver que $z\in L^n\subseteq L^\ast$.
        \end{description}
    \end{enumerate}
\end{ejercicio}



\begin{ejercicio} \label{ej:1.2}
    Sea la gramática $G=\left(V,T,P,S\right)$. Determinar en cada caso el lenguaje generado por la gramática.
    \begin{enumerate}
        \item Tenga en cuenta que:
        \begin{align*}
            V &= \{S,A\}\\
            T &= \{a,b\}\\
            S &= S \\
            P &= \left\{
                \begin{array}{rcl}
                    S &\rightarrow & abAS \mid a \\
                    abA &\rightarrow & baab \\
                    A &\rightarrow & b
                \end{array}
            \right\}
        \end{align*}

        Sea $L=\{ua \mid u\in \{abb, baab\}^\ast\}$. Demostraremos mediante doble inclusión que $L=\cc{L}(G)$.
        \begin{description}
            \item[$\subset)$] Sea $w\in L$. Entonces, $w=ua$ con $u\in \{abb, baab\}^\ast$. Veamos que
            $S \stackrel{\ast}{\Longrightarrow} w$. Para ello, sabemos que $u\in \{abb, baab\}^\ast=\bigcup\limits_{i\in \bb{N}} \{abb, baab\}^i$.
            Sea $n$ el menor número natural tal que $u\in \{abb, baab\}^n$, es decir, es una concatenación de $n$ subcadenas, cada una de las cuales es o bien $abb$ o bien $baab$. Veamos que $S$ produce ambas subcadenas:
            \begin{itemize}
                \item Para producir $abb$, tenemos que $S\rightarrow abAS \rightarrow abbS$.
                \item Para producir $baab$, tenemos que $S\rightarrow abAS \rightarrow baabS$.
            \end{itemize}
            Como vemos, en cada caso podemos concatenar la subcadena necesaria, pero siempre nos quedará una $S$ al final. Usamos la regla de producción $S\rightarrow a$ para eliminarla, llegando así a $w$, por lo que $S \stackrel{\ast}{\Longrightarrow} w$ y $w\in \cc{L}(G)$.

            \item[$\supset)$] Sea $w\in\cc{L}(G)$. Veamos la forma de $w$, para lo cual hay dos opciones:
            \begin{itemize}
                \item $S\rightarrow a$: En este caso, habremos finalizado la palabra con $a$, por lo que habremos añadido la subcadena $a$ a la palabra al final.
                \item $S \rightarrow abAS$: En este caso, también hay dos opciones:
                \begin{itemize}
                    \item $S \rightarrow abAS \rightarrow baabS$: En este caso, habremos concatenado $baab$ con $S$, por lo que habremos añadido la subcadena $baab$ a la palabra.
                    \item $S \rightarrow abAS \rightarrow abbS$: En este caso, habremos concatenado $abb$ con $S$, por lo que habremos añadido la subcadena $abb$ a la palabra.
                \end{itemize}
            \end{itemize}
            Por tanto, $w$ es de la forma $ua$ con $u$ una concatenación de $abb$'s y $baab$'s, es decir, $u\in\{abb, baab\}^\ast$.
            Por tanto, $w\in L$.
        \end{description}

        \item \label{ej:1.2.b} Tenga en cuenta que:
        \begin{align*}
            V &= \{\langle \text{número} \rangle, \langle \text{dígito} \rangle\} \\
            T &= \{0,1,2,3,4,5,6,7,8,9\} \\
            S &= \langle \text{número} \rangle \\
            P &= \left\{
                \begin{array}{rcl}
                    \langle \text{número} \rangle &\rightarrow & \langle \text{número} \rangle \langle \text{dígito} \rangle \\
                    \langle \text{número} \rangle &\rightarrow & \langle \text{dígito} \rangle \\
                    \langle \text{dígito} \rangle &\rightarrow & 0 \mid 1 \mid 2 \mid 3 \mid 4 \mid 5 \mid 6 \mid 7 \mid 8 \mid 9
                \end{array}
            \right\}
        \end{align*}

        Tenemos que $\cc{L}(G)$ es el conjunto de los números naturales, permitiendo
        tantos ceros a la izquierda como se quiera. Es decir (usando la notación de potencia y concatenación vista para lenguajes):
        \begin{equation*}
            L = \{0^i n \mid i\in \bb{N}\cup\{0\},~n\in \bb{N}\cup \{0\}\}
        \end{equation*}
        Demostrémoslo mediante doble inclusión que $L=\cc{L}(G)$.
        \begin{description}
            \item[$\subset)$] Sea $w\in L$. Entonces, $w=0^i n$ con $i\in \bb{N}\cup\{0\}$ y $n\in \bb{N}\cup \{0\}$. Veamos que
            $\langle \text{número} \rangle \stackrel{\ast}{\Longrightarrow} w$:
            \begin{itemize}
                \item En primer lugar, aplicamos $|w|-1$ veces la regla de producción $\langle \text{número} \rangle \rightarrow \langle \text{número} \rangle \langle \text{dígito} \rangle$ y la regla
                que lleva de $\langle \text{dígito} \rangle$ a uno de los símbolos terminales, consiguiendo así en cada etapa reemplazar
                la última variable presente en la cadena por un dígito.
                \item Finalmente, aplicamos la regla de producción $\langle \text{número} \rangle \rightarrow \langle \text{dígito} \rangle$ para reemplazar la última variable por un dígito, que será el primero del número formado.
            \end{itemize}
            Por tanto, $\langle \text{número} \rangle \stackrel{\ast}{\Longrightarrow} w$, teniendo que $w\in \cc{L}(G)$.

            \item[$\supset)$] Sea $w\in\cc{L}(G)$. Como la única regla que
            aumenta la longitud es la regla de producción $\langle \text{número} \rangle \rightarrow \langle \text{número} \rangle \langle \text{dígito} \rangle$, tenemos que $w$ tiene la forma:
            \begin{align*}
                \langle \text{número} \rangle &\Longrightarrow \langle \text{número} \rangle \langle \text{dígito} \rangle \stackrel{|w|-1\text{\ veces}}{\Longrightarrow} \\
                &\Longrightarrow
                \langle \text{número} \rangle \langle \text{dígito} \rangle \langle \text{dígito} \rangle \stackrel{|w|-1\text{\ veces}}{\cdots} \langle \text{dígito} \rangle
                \Longrightarrow \\& \Longrightarrow
                \langle \text{dígito} \rangle \stackrel{|w|\text{\ veces}}{\cdots} \langle \text{dígito} \rangle
            \end{align*}
            Por tanto, tenemos que se trata una sucesión de $|w|$ dígitos, lo que nos lleva a que $w\in L$.
        \end{description}

        \item Tenga en cuenta que:
        \begin{align*}
            V &= \{A,S\} \\
            T &= \{a,b\} \\
            S &= S \\
            P &= \left\{
                \begin{array}{rcl}
                    S &\rightarrow & aS \mid aA \\
                    A &\rightarrow & bA \mid b
                \end{array}
            \right\}
        \end{align*}

        Sea $L=\{a^nb^m \in \{a,b\}^\ast \mid n, m \in \bb{N}\}$. Demostraremos mediante doble inclusión que $L=\cc{L}(G)$.
        \begin{description}
            \item[$\subset)$] Sea $w\in L$. Entonces, $w=a^nb^m$ con $n,m\in \bb{N}$. Veamos que
            $S \stackrel{\ast}{\Longrightarrow} w$:
            \begin{itemize}
                \item En primer lugar, aplicamos $n-1$ veces la regla de producción $S \rightarrow aS$ para obtener $a^{n-1}S$,
                \begin{equation*}
                    S \stackrel{\ast}{\Longrightarrow} a^{n-1}S
                \end{equation*}

                \item Para cambiar a la etapa de añadir $b$'s, aplicamos la regla de producción $S \rightarrow aA$, obteniendo así $a^{n}A$,
                \item Después, aplicamos $m-1$ veces la regla de producción $A \rightarrow bA$ para obtener $a^nb^{m-1}A$.
                \item Para finalizar, aplicamos la regla de producción $A \rightarrow b$ para obtener $a^nb^m$.
            \end{itemize}
            Por tanto, $S \stackrel{\ast}{\Longrightarrow} w$, teniendo que $w\in \cc{L}(G)$.

            \item[$\supset)$] Sea $w\in\cc{L}(G)$. Vemos que en la palabra siempre
            va a haber tan solo una variable (ya sea $S$ o $A$). Se empezará con la $S$, y en cierto momento se cambiará a la $A$,
            sin poder entonces volver a la $S$.
            \begin{itemize}
                \item Cuando se está en la etapa en la que hay $S$, tan solo se pueden añadir $a$'s,
                o bien cambiar a la $A$.
                \item Cuando se está en la etapa en la que hay $A$, tan solo se pueden añadir $b$'s.
            \end{itemize}
            Por tanto, tenemos que $w$ estará formada por una sucesión de
            $a$'s seguida de una sucesión de $b$'s, lo que nos lleva a que $w\in L$.
        \end{description}
    \end{enumerate}
\end{ejercicio}

\begin{ejercicio}
    Encontrar gramáticas de tipo 2 para los siguientes lenguajes sobre el alfabeto $\{a, b\}$. En cada caso determinar si los lenguajes generados son de tipo 3, estudiando si existe una gramática de tipo 3 que los genera.
    \begin{enumerate}
        \item Palabras en las que el número de $b$ no es tres.
        
        Tenemos varias opciones:
        \begin{itemize}
            \item Que no tenga $b$'s.
            \item Que tenga una $b$.
            \item Que tenga dos $b$'s.
            \item Que tenga $4$ o más $b$'s.
        \end{itemize}

        Sea la gramática $G=\left(V,T,P,S\right)$ dada por:
        \begin{align*}
            V &= \{S, A, X\} \\
            T &= \{a,b\} \\
            S &= S \\
            P &= \left\{
                \begin{array}{rcl}
                    S &\rightarrow & A \mid AbA \mid AbAbA \mid XbXbXbXbX \\
                    A &\rightarrow & aA \mid \veps \\
                    X &\rightarrow & aX \mid bX \mid \veps
                \end{array}
            \right\}
        \end{align*}

        Esta gramática no obstante es de tipo $2$. Busquemos otra que sea de tipo 3.
        Sea la gramática $G'=\left(V',T',P',S'\right)$ dada por:
        \begin{align*}
            V' &= \{S, X,Y,Z, W\} \\
            T' &= \{a,b\} \\
            S' &= S \\
            P' &= \left\{
                \begin{array}{rcl}
                    S &\rightarrow & \veps \mid aS \mid bX \\
                    X &\rightarrow & \veps \mid aX \mid bY \\
                    Y &\rightarrow & \veps \mid aY \mid bZ \\
                    Z &\rightarrow & aZ \mid bW \\
                    W &\rightarrow & \veps \mid aW \mid bW
                \end{array}
            \right\}
        \end{align*}

        Esta sí es de tipo $3$, y genera el lenguaje deseado.



        \item Palabras que tienen 2 ó 3 $b$.
        
        Sea la gramática $G=\left(V,T,P,S\right)$ dada por:
        \begin{align*}
            V &= \{S, A, B\} \\
            T &= \{a,b\} \\
            S &= S \\
            P &= \left\{
                \begin{array}{rcl}
                    S &\rightarrow & AbAbABA \\
                    A &\rightarrow & aA \mid \veps \\
                    B &\rightarrow & b \mid \veps
                \end{array}
            \right\}
        \end{align*}

        Esta gramática no obstante es de tipo $2$. Busquemos otra que sea de tipo 3.
        Sea la gramática $G'=\left(V',T',P',S'\right)$ dada por:
        \begin{align*}
            V' &= \{S, X,Y,Z,W,V,T\} \\
            T' &= \{a,b\} \\
            S' &= S \\
            P' &= \left\{
                \begin{array}{rcl}
                    S &\rightarrow & aS \mid X \\
                    X &\rightarrow & bY \\
                    Y &\rightarrow & aY \mid Z \\
                    Z &\rightarrow & bW \\
                    W &\rightarrow & aW \mid \veps \mid V \\
                    V &\rightarrow & bT \\
                    T &\rightarrow & aT \mid \veps
                \end{array}
            \right\}
        \end{align*}

        Esta gramática ya es de tipo $3$, pero contiene un número elevado de variables. Veamos si podemos reducirlo:
        Sea la gramática $G''=\left(V'',T'',P'',S''\right)$ dada por:
        \begin{align*}
            V'' &= \{S, X,Y,Z\} \\
            T'' &= \{a,b\} \\
            S'' &= S \\
            P'' &= \left\{
                \begin{array}{rcl}
                    S &\rightarrow & aS \mid bX \\
                    X &\rightarrow & aX \mid bY \\
                    Y &\rightarrow & aY \mid \veps \mid bZ \\
                    Z &\rightarrow & aZ \mid \veps
                \end{array}
            \right\}
        \end{align*}

        Notemos que, en esta gramática de tipo $3$, ya hemos conseguido el menor número de variables posibles, que representan las $4$ etapas. Como la última es opcional, está la regla $Y\rightarrow \veps$, para así no agregar la tercera $b$.

    \end{enumerate}
\end{ejercicio}

\begin{ejercicio}
    Encontrar gramáticas de tipo 2 para los siguientes lenguajes sobre el alfabeto $\{a, b\}$. En cada caso determinar si los lenguajes generados son de tipo 3, estudiando si existe una gramática de tipo 3 que los genera.
    \begin{enumerate}
        \item Palabras que no contienen la subcadena $ab$.
        
        Sea la gramática $G=\left(V,T,P,S\right)$ dada por:
        \begin{align*}
            V &= \{S, A\} \\
            T &= \{a,b\} \\
            S &= S \\
            P &= \left\{
                \begin{array}{rcl}
                    S &\rightarrow & aA \mid bS \mid \veps \\
                    A &\rightarrow & aA \mid \veps \\
                \end{array}
            \right\}
        \end{align*}

        Notemos además que esta gramática es de tipo $3$, y se tiene que:
        \begin{equation*}
            \cc{L}(G) = \{b^i a^j \mid i,j\in \bb{N}\cup \{0\}\}
        \end{equation*}


        \item Palabras que no contienen la subcadena $baa$.
        
        Sea la gramática $G=\left(V,T,P,S\right)$ dada por:
        \begin{align*}
            V &= \{S, B\} \\
            T &= \{a,b\} \\
            S &= S \\
            P &= \left\{
                \begin{array}{rcl}
                    S &\rightarrow & aS \mid bB \mid \veps \\
                    B &\rightarrow & bB \mid abB \mid a \mid \veps
                \end{array}
            \right\}
        \end{align*}
        Notemos además que esta gramática es de tipo $3$.
    \end{enumerate}
\end{ejercicio}

\begin{ejercicio}
    Encontrar una gramática libre de contexto que genere el lenguaje sobre el alfabeto $\{a, b\}$ de las palabras que tienen más $a$ que $b$ (al menos una más).

    Sea la gramática $G=\left(V,T,P,S\right)$ dada por:
    \begin{align*}
        V &= \{S, S'\} \\
        T &= \{a,b\} \\
        S &= S \\
        P &= \left\{
            \begin{array}{rcl}
                S &\rightarrow & S'aS'\\
                S' &\rightarrow & S'aS' \mid aS'bS' \mid bS'aS' \mid \veps
            \end{array}
        \right\}
    \end{align*}
\end{ejercicio}

\begin{ejercicio}
    Encontrar, si es posible, una gramática regular (o, si no es posible, una gramática libre del contexto) que genere el lenguaje $L$ supuesto que $L \subset \{a, b\}^\ast$ y verifica:
    \begin{enumerate}
        \item $u \in L$ si, y solamente si, verifica que $u$ no contiene dos símbolos $b$ consecutivos.
        
        Sea la gramática $G=\left(V,T,P,S\right)$ dada por:
        \begin{align*}
            V &= \{S\} \\
            T &= \{a,b\} \\
            S &= S \\
            P &= \left\{
                \begin{array}{rcl}
                    S &\rightarrow & aS \mid baS \mid b\mid \veps
                \end{array}
            \right\}
        \end{align*}
        \item $u \in L$ si, y solamente si, verifica que $u$ contiene dos símbolos $b$ consecutivos.
        
        Sea la gramática $G=\left(V,T,P,S\right)$ dada por:
        \begin{align*}
            V &= \{S, B, F\} \\
            T &= \{a,b\} \\
            S &= S \\
            P &= \left\{
                \begin{array}{rcl}
                    S &\rightarrow & aS \mid bB \\
                    B &\rightarrow & bF \mid aS \\
                    F &\rightarrow & aF \mid bF \mid \veps
                \end{array}
            \right\}
        \end{align*}
        Notemos que, en este caso, tenemos tres estados:
        \begin{itemize}
            \item $S$: No hemos encontrado dos $b$'s consecutivas.
            \item $B$: Hemos encontrado una $b$, y puede ser que nos encontremos la segunda $b$.
            \item $F$: Hemos encontrado dos $b$'s consecutivas; ya hay libertad.
        \end{itemize}

        Sí es cierto que usamos tres variables. Para usar solo dos variables,
        podemos hacer lo siguiente.
        Sea la gramática $G'=\left(V',T',P',S'\right)$ dada por:
        \begin{align*}
            V' &= \{S, X\} \\
            T' &= \{a,b\} \\
            S' &= S \\
            P' &= \left\{
                \begin{array}{rcl}
                    S &\rightarrow & aS \mid bS \mid bbX \\
                    X &\rightarrow & aX \mid bX \mid \veps
                \end{array}
            \right\}
        \end{align*}
    \end{enumerate}
\end{ejercicio}

\begin{ejercicio}
    Encontrar, si es posible, una gramática regular (o, si no es posible, una gramática libre del contexto) que genere el lenguaje $L$ supuesto que $L \subset \{a, b\}^\ast$ y verifica:
    \begin{enumerate}
        \item $u \in L$ si, y solamente si, verifica que contiene un número impar de símbolos $a$.
        
        Sea la gramática $G=\left(V,T,P,S\right)$ dada por:
        \begin{align*}
            V &= \{S, X\} \\
            T &= \{a,b\} \\
            S &= S \\
            P &= \left\{
                \begin{array}{rcl}
                    S &\rightarrow & aX \mid bS\\
                    X &\rightarrow & aS \mid bX \mid \veps
                \end{array}
            \right\}
        \end{align*}
        \item $u \in L$ si, y solamente si, verifica que no contiene el mismo número de símbolos $a$ que de símbolos $b$.
        
        Sea la gramática $G=\left(V,T,P,S\right)$ dada por:
        \begin{align*}
            V &= \{S, A, B, X\} \\
            T &= \{a,b\} \\
            S &= S \\
            P &= \left\{
                \begin{array}{rcl}
                    S &\rightarrow & AaA \mid BbB \\
                    A &\rightarrow & AaA\mid X \\
                    B &\rightarrow & BbB \mid X \\
                    X &\rightarrow & aXbX \mid bXaX \mid \veps
                \end{array}
            \right\}
        \end{align*}
    \end{enumerate}
\end{ejercicio}


\begin{ejercicio}
    Dado el alfabeto $A = \{a, b\}$ determinar si es posible encontrar una gramática libre de contexto que:
    \begin{enumerate}
        \item Genere las palabras de longitud impar, y mayor o igual que 3, tales que la primera letra coincida con la letra central de la palabra.
        
        Sea la gramática $G=\left(V,T,P,S\right)$ dada por:
        \begin{align*}
            V &= \{S, X, A, B, C, D\} \\
            T &= \{a,b\} \\
            S &= S \\
            P &= \left\{
                \begin{array}{rcl}
                    S &A\mid B \\
                    A &\rightarrow & aCX \\
                    C & \rightarrow & a \mid XCX \\
                    B &\rightarrow & bDX \\
                    D & \rightarrow & b \mid XDX \\
                    X &\rightarrow & a\mid b
                \end{array}
            \right\}
        \end{align*}
        \item Genere las palabras de longitud par, y mayor o igual que 2, tales que las dos letras centrales coincidan.
        
        Sea la gramática $G=\left(V,T,P,S\right)$ dada por:
        \begin{align*}
            V &= \{S, X\} \\
            T &= \{a,b\} \\
            S &= S \\
            P &= \left\{
                \begin{array}{rcl}
                    S &\rightarrow & XSX\mid C \\
                    C &\rightarrow & aa \mid bb \\
                    X &\rightarrow & a \mid b
                \end{array}
            \right\}
        \end{align*}
    \end{enumerate}
\end{ejercicio}

\begin{ejercicio}
    Sea la gramática $G=\left(V,T,P,S\right)$ dada por:
    \begin{align*}
        V &= \{S, X\} \\
        T &= \{a,b\} \\
        S &= S\\
        P &= \left\{
            \begin{array}{rcl}
                S &\rightarrow & SS \\
                S &\rightarrow & XXX \\
                X &\rightarrow & aX \mid Xa \mid b
            \end{array}
        \right\}
    \end{align*}
    Determinar si el lenguaje generado por la gramática es regular. Justificar la respuesta.\\

    Sea la siguiente gramática regular $G'=\left(V',T',P',S'\right)$ dada por:
    \begin{align*}
        V' &= \{S, X\} \\
        T' &= \{a,b\} \\
        S' &= S \\
        P' &= \left\{
            \begin{array}{rcl}
                S &\rightarrow & aS \mid bX \\
                X &\rightarrow & aX \mid bY \\
                Y &\rightarrow & aY \mid bZ \\
                Z &\rightarrow & aZ \mid bW \mid \veps \\
                W &\rightarrow & aW \mid bU \\
                U &\rightarrow & aU \mid bV \\
                V &\rightarrow & aV \mid \veps
            \end{array}
        \right\}
    \end{align*}

    Tenemos que $\cc{L}(G) = \cc{L}(G')$, y como $G'$ es una gramática regular, tenemos que $\cc{L}(G)$ es regular.
    Sí es cierto que en el tema $2$ aprendemos otras maneras de demostrarlo más sencillas, como buscar un autómata finito que lo genere.
\end{ejercicio}

\begin{ejercicio}
    Dado un lenguaje $L$ sobre un alfabeto $A$, ¿es $L^{\ast}$ siempre numerable? ¿nunca lo es? ¿o puede serlo unas veces sí y otras, no? Pon ejemplos en este último caso.\\

    $L^{\ast}$ es siempre numerable, veámos por qué. $L^{\ast}$ es un lenguaje sobre el alfabeto $A$, por lo que $L^{\ast}\subseteq A^{\ast}$ y $A^{\ast}$ es numerable (visto en teoría), luego $L^{\ast}$ también lo es.
\end{ejercicio}

\begin{ejercicio}
    Dado un lenguaje $L$ sobre un alfabeto $A$, caracterizar cuando $L^{\ast} = L$. Esto es, dar un conjunto de propiedades sobre $L$ de manera que $L$ cumpla esas propiedades si y sólo si $L^{\ast} = L$.

    \begin{equation*}
        L = L^{\ast} \Longleftrightarrow \left\{
            \begin{array}{cl}
                \veps \in L \\ \land \\ u,v \in L & \Longrightarrow uv\in L
            \end{array}
        \right.
    \end{equation*}
    Es decir, $L=L^{\ast}$ si y solo si la cadena vacía está en $L$ y además es cerrado para concatenaciones.

    \begin{proof} Demostramos mediante doble implicación.
        \begin{description}
            \item [$\Longleftarrow)$] La inclusión $L\subseteq L^{\ast}$ es obvia, por lo que solo falta demostrar la otra inclusión.\\

                Sea $v\in L^{\ast}$:
                \begin{enumerate}
                    \item Si $v = \veps \Longrightarrow v\in L$ por hipótesis.
                    \item Si $v\neq \veps$, $\exists n\in \mathbb{N}$ tal que 
                        \begin{equation*}
                            v = a_1 a_2 \ldots a_n
                        \end{equation*}
                        con $a_i \in L$ $\forall i \in \{1, \ldots, n\}$, de donde tenemos que $v\in L$, por ser cerrado para concatenaciones. Luego $L^{\ast}\subseteq L$.
                \end{enumerate}
            \item [$\Longrightarrow)$] Hemos de probar dos cosas:
                \begin{enumerate}
                    \item $\veps \in L^{\ast}=L$.
                    \item Sean $u,v\in L=L^{\ast} \Longrightarrow uv\in L^{\ast}=L$.
                \end{enumerate}
        \end{description}
    \end{proof}
\end{ejercicio}

\begin{ejercicio}
    Dados dos homomorfismos $f : A^{\ast} \rightarrow B^{\ast}$, $g : A^{\ast} \rightarrow B^{\ast}$, se dice que son iguales si $f(x) = g(x)$, $\forall x \in A^{\ast}$. ¿Existe un procedimiento algorítmico para comprobar si dos homomorfismos son iguales?\\

    Sí, basta probar que su imagen coincide sobre un conjunto finito de elementos, los de $A$:
    \begin{equation*}
        f(x) = g(x) \quad \forall x\in A^{\ast} \Longleftrightarrow f(a)=g(a) \quad \forall a\in A
    \end{equation*}
    \begin{proof}\ 
        \begin{description}
            \item [$\Longleftarrow)$] Sea $v\in A^{\ast}$, $\exists n\in \mathbb{N}$ tal que $v=a_1a_2\ldots a_n$ con $a_i \in A$ $\forall i \in \{1,\ldots, n\}$
                \begin{equation*}
                    f(v) = f(a_1)f(a_2)\ldots f(a_n) = g(a_1)g(a_2)\ldots g(a_n) = g(v)
                \end{equation*}
            \item [$\Longrightarrow)$] Sea $a\in A \Longrightarrow a\in A^{\ast}\Longrightarrow f(a)=g(a)$.
        \end{description}
    \end{proof}
\end{ejercicio}

\begin{ejercicio}
    Sea $L \subseteq A^{\ast}$ un lenguaje arbitrario. Sea $C_0 = L$ y definamos los lenguajes $S_i$ y $C_i$, para todo $i \geq 1$, por $S_i = C_{i-1}^+$ y $C_i = \ol{S_i}$. 
    \begin{enumerate}
        \item ¿Es $S_1$ siempre, nunca o a veces igual a $C_2$? Justifica la respuesta.
        \item Demostrar que $S_2 = C_3$, cualquiera que sea $L$.
        \begin{observacion}
            Demuestra que $C_2$ es cerrado para la concatenación.
        \end{observacion}
    \end{enumerate}
    % // TODO: Hacer JJ
\end{ejercicio}

\begin{ejercicio}
    Demuestra que, para todo alfabeto $A$, el conjunto de los lenguajes finitos sobre dicho alfabeto es numerable.

    Sea $A=\{a_1, a_2, \ldots, a_n\}$, con $n\in \mathbb{N}$. Definimos el siguiente conjunto:
    \begin{equation*}
        \Gamma = \{L\subseteq A^{\ast} \mid L \text{\ es finito}\}
    \end{equation*}

    Dado un símbolo $z\notin A$, definimos el conjunto $B=\{z\}\cup A$. Sea $B^{\ast}$ numerable, y buscamos una inyección de $\Gamma$ en $B^{\ast}$.
    Dado un lenguaje $L\in \Gamma$, sea $L=\{l_1, l_2, \ldots, l_m\}$, con $m\in \mathbb{N}$ y $l_i\in A^{\ast}$ $\forall i\in \{1, \ldots, m\}$. Definimos la siguiente función:
    \Func{f}{\Gamma}{B^\ast}{L}{zl_1zl_2\ldots zl_mz}

    Veamos que $f$ es inyectiva. Sean $L_1, L_2\in \Gamma$ tales que $f(L_1)=f(L_2)$. Entonces,
    \begin{equation*}
        zl_1zl_2\ldots zl_kz = zl'_1zl'_2\ldots zl'_{k'}z
    \end{equation*}
    Por ser ambas palabras iguales, tenemos que $k=k'$ y $l_i=l'_i$ $\forall i\in \{1, \ldots, k\}$, de donde $L_1=L_2$. Por tanto, $f$ es inyectiva, por lo que $\Gamma$ es inyectivo con un subconjunto de $B^{\ast}$, que es numerable. Por tanto, $\Gamma$ es numerable.
\end{ejercicio}




\subsection{Cálculo de gramáticas}

\begin{ejercicio}[Complejidad: Sencilla]
    Calcula, de forma razonada, gramáticas que generen cada uno de los siguientes lenguajes:
    \begin{enumerate}
        \item $\{ u\in \{0,1\}^\ast \mid |u|\leq 4 \}$
        
        Sea la gramática $G=\left(V,T,P,S\right)$ dada por:
        \begin{align*}
            V &= \{S, X\} \\
            T &= \{0,1\} \\
            S &= S \\
            P &= \left\{
                \begin{array}{rcl}
                    S &\rightarrow & XXXX \\
                    X &\rightarrow & 0 \mid 1 \mid \veps
                \end{array}
            \right\}
        \end{align*}

        No obstante, esta gramática es de tipo $2$. Busquemos una de tipo $3$.
        Sea la gramática $G'=\left(V',T',P',S'\right)$ dada por:
        \begin{align*}
            V' &= \{S, X, Y, Z\} \\
            T' &= \{0,1\} \\
            S' &= S \\
            P' &= \left\{
                \begin{array}{rcl}
                    S &\rightarrow & 0X \mid 1X \mid \veps \\
                    X &\rightarrow & 0Y \mid 1Y \mid \veps \\
                    Y &\rightarrow & 0Z \mid 1Z \mid \veps \\
                    Z &\rightarrow & 0 \mid 1
                \end{array}
            \right\}
        \end{align*}
        Tenemos que $\cc{L}(G) = \cc{L}(G')$, y es igual al lenguaje deseado. Tenemos por tanto que es un lenguaje regular.


        \item Palabras con 0's y 1's que no contengan dos 1's consecutivos y que empiecen por un 1 y que terminen por dos 0's.
        
        Sea la gramática $G=\left(V,T,P,S\right)$ dada por:
        \begin{align*}
            V &= \{S, X, Y\} \\
            T &= \{0,1\} \\
            S &= S \\
            P &= \left\{
                \begin{array}{rcl}
                    S &\rightarrow & 1X00 \\
                    X &\rightarrow & 0Y \mid \veps \\
                    Y &\rightarrow & 0Y \mid 1X \mid \veps \\
                \end{array}
            \right\}
        \end{align*}

        Notemos que esta gramática es de tipo 2 debido a la primera regla de producción. Busquemos una de tipo 3. 
        Sea la gramática $G'=\left(V',T',P',S'\right)$ dada por:
        \begin{align*}
            V' &= \{S, X, Y\} \\
            T' &= \{0,1\} \\
            S' &= S \\
            P' &= \left\{
                \begin{array}{rcl}
                    S &\rightarrow & 1X \\
                    X &\rightarrow & 0Y \mid F \\
                    Y &\rightarrow & 0Y \mid 1X \mid F \\
                    F &\rightarrow & 00
                \end{array}
            \right\}
        \end{align*}

        Tenemos que $\cc{L}(G) = \cc{L}(G')$, y es igual al lenguaje deseado. Tenemos por tanto que es un lenguaje regular. En esta última gramática, tenemos los siguientes estados:
        \begin{itemize}
            \item $S$: Es el estado inicial, empezamos con un $1$.
            \item $X$: Acabamos de escribir un $1$, por lo que ahora tan solo podemos escribir $0$'s.
            \item $Y$: Acabamos de escribir un $0$, por lo que ahora podemos escribir tanto $0$'s como $1$'s.
            \item $F$: Ya hemos terminado, y escribimos los dos $0$'s finales por la restricción impuesta.
        \end{itemize}
        

        \item El conjunto vacío.
        
        Sea la gramática $G=\left(V,T,P,S\right)$ dada por:
        \begin{align*}
            V &= \{S\} \\
            T &= \emptyset \\
            S &= S \\
            P &= \left\{
                \begin{array}{rcl}
                    S &\rightarrow & S
                \end{array}
            \right\}
        \end{align*}

        \item El lenguaje formado por los números naturales.
        
        Sea la gramática $G=\left(V,T,P,S\right)$ dada por:
        \begin{align*}
            V &= \{\langle \text{número no iniciado} \rangle, \langle \text{dígito no cero} \rangle, \langle \text{dígito} \rangle, \langle \text{número iniciado} \rangle\} \\
            T &= \{0,1,2,3,4,5,6,7,8,9\} \\
            S &= \langle \text{número no iniciado} \rangle \\
            P &= \left\{
                \begin{array}{rcl}
                    \langle \text{número no iniciado} \rangle &\rightarrow & \langle \text{dígito no cero} \rangle \mid \langle \text{dígito no cero} \rangle \langle \text{número iniciado} \rangle \\
                    \langle \text{número iniciado} \rangle &\rightarrow & \langle \text{dígito} \rangle \mid \langle \text{dígito} \rangle \langle \text{número iniciado} \rangle \\
                    \langle \text{dígito no cero} \rangle &\rightarrow & 1 \mid 2 \mid 3 \mid 4 \mid 5 \mid 6 \mid 7 \mid 8 \mid 9 \\
                    \langle \text{dígito} \rangle &\rightarrow & 0 \mid \langle \text{dígito no cero} \rangle
                \end{array}
            \right\}
        \end{align*}

        Notemos que esta gramática es similar a la descrita en el Ejercicio~\ref{ej:1.2}.\ref{ej:1.2.b}, pero adaptada para que los números naturales no puedan empezar por $0$.
        No obstante, esta gramática es de tipo $2$. Busquemos una de tipo $3$.
        Sea la gramática $G'=\left(V',T',P',S'\right)$ dada por:
        \begin{align*}
            V' &= \{S, X, Y, Z\} \\
            T' &= \{0,1,2,3,4,5,6,7,8,9\} \\
            S' &= S \\
            P' &= \left\{
                \begin{array}{rcl}
                    S &\rightarrow & 0 \mid 1N \mid 2N \mid 3N \mid 4N \mid 5N \mid 6N \mid 7N \mid 8N \mid 9N\\
                    N &\rightarrow & 0N\mid 1N \mid 2N \mid 3N \mid 4N \mid 5N \mid 6N \mid 7N \mid 8N \mid 9N \mid \veps
                \end{array}
            \right\}
        \end{align*}
        \item $\{ a^n \in \{a,b\}^\ast \mid n\geq 0 \} \cup \{ a^nb^n \in \{a,b\}^\ast \mid n\geq 0 \}$
        
        Sea la gramática $G=\left(V,T,P,S\right)$ dada por:
        \begin{align*}
            V &= \{S, X, Y\} \\
            T &= \{a,b\} \\
            S &= S \\
            P &= \left\{
                \begin{array}{rcl}
                    S &\rightarrow & X \mid Y \mid \veps \\
                    X &\rightarrow & aX \mid \veps \\
                    Y &\rightarrow & aYb \mid \veps
                \end{array}
            \right\}
        \end{align*}
        \item $\{ a^nb^{2n}c^m \in \{a,b,c\}^\ast \mid n,m>0 \}$
        
        Sea la gramática $G=\left(V,T,P,S\right)$ dada por:
        \begin{align*}
            V &= \{S, X, Y, Z\} \\
            T &= \{a,b,c\} \\
            S &= S \\
            P &= \left\{
                \begin{array}{rcl}
                    S &\rightarrow & aXbbcY \\
                    X &\rightarrow & aXbb \mid \veps \\
                    Y &\rightarrow & cY \mid \veps
                \end{array}
            \right\}
        \end{align*}
        \item $\{ a^nb^ma^n \in \{a,b\}^\ast \mid m,n\geq 0 \}$
        
        Sea la gramática $G=\left(V,T,P,S\right)$ dada por:
        \begin{align*}
            V &= \{S, X\} \\
            T &= \{a,b\} \\
            S &= S \\
            P &= \left\{
                \begin{array}{rcl}
                    S &\rightarrow & aSa \mid bX \mid \veps \\
                    X &\rightarrow & bX \mid \veps \\
                \end{array}
            \right\}
        \end{align*}

        \item Palabras con 0's y 1's que contengan la subcadena 00 y 11.
        
        Sea la gramática $G=\left(V,T,P,S\right)$ dada por:
        \begin{align*}
            V &= \{S, X\} \\
            T &= \{0,1\} \\
            S &= S \\
            P &= \left\{
                \begin{array}{rcl}
                    S &\rightarrow & X00X11X \mid X11X00X \\
                    X &\rightarrow & 0X \mid 1X \mid \veps
                \end{array}
            \right\}
        \end{align*}

        Notemos que esta gramática es de tipo $2$. Busquemos una de tipo $3$.
        Sea la gramática $G'=\left(V',T',P',S'\right)$ dada por:
        \begin{align*}
            V' &= \{S, X, A, B, F\} \\
            T' &= \{0,1\} \\
            S' &= S \\
            P' &= \left\{
                \begin{array}{rcl}
                    S &\rightarrow & 0S \mid 1S \mid X\\
                    X &\rightarrow & 00A \mid 11B \\
                    A &\rightarrow & 0A \mid 1A \mid 11F \\
                    B &\rightarrow & 0B \mid 1B \mid 00F \\
                    F &\rightarrow & 0F \mid 1F \mid \veps
                \end{array}
            \right\}
        \end{align*}

        Notemos que:
        \begin{itemize}
            \item $S$: No hemos encontrado ninguna subcadena.
            \item $X$: Hemos encontrado una subcadena, y ahora buscamos la otra.
            \item $A$: Hemos encontrado la subcadena $00$, y ahora buscamos la subcadena $11$.
            \item $B$: Hemos encontrado la subcadena $11$, y ahora buscamos la subcadena $00$.
            \item $F$: Hemos encontrado ambas subcadenas.
        \end{itemize}
        
        \item Palíndromos formados con las letras $a$ y $b$.
        
        Sea la gramática $G=\left(V,T,P,S\right)$ dada por:
        \begin{align*}
            V &= \{S, X, Y\} \\
            T &= \{a,b\} \\
            S &= S \\
            P &= \left\{
                \begin{array}{rcl}
                    S &\rightarrow & aSa \mid bSb \mid \veps \mid a \mid b
                \end{array}
            \right\}
        \end{align*}
        Notemos que las reglas $S\rightarrow a\mid b$ se han añadido para añadir los palíndromos de longitud impar.
    \end{enumerate}
\end{ejercicio}

\begin{ejercicio}[Complejidad: Media]
    Calcula, de forma razonada, gramáticas que generen cada uno de los siguientes lenguajes:
    \begin{enumerate}
        \item $\{uv \in \{0,1\}^\ast \mid u^{-1} \text{ es un prefijo de } v\}$
        
        Sea la gramática $G=\left(V,T,P,S\right)$ dada por:
        \begin{align*}
            V &= \{S, X, Y\} \\
            T &= \{0,1\} \\
            S &= S \\
            P &= \left\{
                \begin{array}{rcl}
                    S &\rightarrow & XY \\
                    X &\rightarrow & 0X0 \mid 1X1 \mid \veps \\
                    Y &\rightarrow & 0Y \mid 1Y \mid \veps
                \end{array}
            \right\}
        \end{align*}
        Notemos que $X$ deriva en el palíndromo, $uu^{-1}$, y $Y$ en el resto de la palabra de $v$.
        \item $\{ucv \in \{a,b,c\}^\ast \mid |u| = |v|\}$
        
        Sea la gramática $G=\left(V,T,P,S\right)$ dada por:
        \begin{align*}
            V &= \{S, X\} \\
            T &= \{a,b,c\} \\
            S &= S \\
            P &= \left\{
                \begin{array}{rcl}
                    S &\rightarrow & XSX \mid c \\
                    X &\rightarrow & a \mid b \mid c
                \end{array}
            \right\}
        \end{align*}

        \item $\{u1^n \in \{0,1\}^\ast \mid |u| = n\}$
        
        Sea la gramática $G=\left(V,T,P,S\right)$ dada por:
        \begin{align*}
            V &= \{S, X\} \\
            T &= \{0,1\} \\
            S &= S \\
            P &= \left\{
                \begin{array}{rcl}
                    S &\rightarrow & XS1 \mid \veps \\
                    X &\rightarrow & 0 \mid 1
                \end{array}
            \right\}
        \end{align*}

        \item $\{a^nb^na^{n+1} \in \{a,b\}^\ast \mid n\geq 0\}$ (observar transparencias de teoría)
        
        Sea la gramática $G=\left(V,T,P,S\right)$ dada por:
        \begin{align*}
            V &= \{S, X, Y\} \\
            T &= \{a,b\} \\
            S &= S \\
            P &= \left\{
                \begin{array}{rcl}
                    S &\rightarrow & a\mid abaa\mid aXbaa\\
                    Xb & \rightarrow & bX\\
                    Xa & \rightarrow & Ybaa\\
                    bY & \rightarrow & Yb\\
                    aY & \rightarrow aa\mid aaX
                \end{array}
            \right\}
        \end{align*}
    \end{enumerate}
\end{ejercicio}


\begin{ejercicio}[Complejidad: Difícil]
    Calcula, de forma razonada, gramáticas que generen cada uno de los siguientes lenguajes:
    \begin{enumerate}
        \item $\{a^nb^mc^k \in \{a,b,c\}^\ast \mid k = m + n\}$
        
        Sea la gramática $G=\left(V,T,P,S\right)$ dada por:
        \begin{align*}
            V &= \{S, X\} \\
            T &= \{a,b,c\} \\
            S &= S \\
            P &= \left\{
                \begin{array}{rcl}
                    S &\rightarrow & aSc \mid X \\
                    X &\rightarrow & bXc \mid \veps
                \end{array}
            \right\}
        \end{align*}
        
        \item Palabras que son múltiplos de 7 en binario.
            \begin{description}
                \item [Opción 1.] 
                    Hacer un autómata que acepte el lenguaje.
                \item [Opción 2.] 
                    Un tanto más complicada, introduce cálculos con números binarios. La idea principal es que si $x$ es un número natural, entonces:
                    \begin{equation*}
                        7x = (8-1)x = 8x - x
                    \end{equation*}
                    \begin{itemize}
                        \item Multiplicar un número en binario por 8 es añadirle tres 0s al final.
                        \item Restar un número binario menos otro es realizarle el complemento a dos al segundo, sumar los números y descartar el primer 1.
                    \end{itemize}
                    Realizar estas dos operaciones es más sencillo que multiplicar un número cualquiera en binario por 7. Procedemos por tanto, a:
                    \begin{enumerate}
                        \item Generar un número cualquiera en binario.
                        \item Multiplicarlo por 8.
                        \item Generar su complemento a 2 en binario.
                        \item Sumar ambos números.
                        \item Descarar el bit de acarreo (el más significativo).
                    \end{enumerate}
                    Para esta opción, construiremos la gramática $G=(V,T,P,S)$ dada por:
                    \begin{align*}
                        V &= \{S, N, \alpha, \beta, \delta,\gamma, Z, Z', A, D, E, E_0, E_1, E_2, E_0', E_1', \overline{E_0}, \overline{E_1}, \overline{E_2}, L_0, L_1, X\} \\
                        T &= \{0,1\} \\
                        S &= S
                    \end{align*}
                    Y $P$ es un conjunto que contiene todas las reglas de producción que se mostrarán a continuación.

                    La idea es:
                    \begin{itemize}
                        \item Generar entre $\alpha$ y $\beta$ cualquier número en binario, mientras generamos entre $\beta$ y $\gamma$ su complemento a 1 en espejo (es decir, el número invertido). Finalmente, multiplicaremos el de la izquierda por 8 y se verá reflejado en la izquierda con 1s.
                        \item Posteriormente, usaremos la variable $Z$ para sumarle 1 al complemento a 1 del número generado. 
                        \item Como no podemos modificar símbolos terminales una vez ya generados, trabajaremos todo el rato hasta el final con variables, de forma que $A$ será un 0 y $B$ un 1.
                        \item Una vez generado el número $8x$ y $x$ en complemento a dos en espejo, pasaremos a sumar ambos números usando para ello las variables $E$ y $L$. Los símbolos del número en complemento a 2 los iremos eliminando y en la izquierda controlaremos los bits del número que ya hemos usado con la variable $\delta$.
                        \item Cuando las variables $\beta$ y $\gamma$ ``se toquen'', habremos terminado de sumar y ya sólo quedará eliminar las variables delimitadoras (las letras griegas) y sustituir $A$ y $B$ por 0 y 1, respectivamente.
                    \end{itemize}
                    Comenzamos ya describiendo las reglas de producción:
                    \begin{itemize}
                        \item En primer lugar, creamos el entorno en el que trabajaremos, aceptando ``0'' como número en binario múltiplo de 7:
                            \begin{equation*}
                                S \rightarrow \alpha BNABBB\gamma\ |\ 0
                            \end{equation*}
                            Iremos usando $N$ para generar nuestro número a su izquierda y el complemento a 1 en espejo a la derecha. 
                            \begin{itemize}
                                \item Las tres $B$s ya introducidas a la derecha son para luego compensar la multiplicación por 8.
                                \item Así mismo, hemos generado ya $B$ a la izquierda y $A$ a la derecha para aceptar sólo números binarios que comiencen por 1.
                            \end{itemize}
                            Usamos ahora la variable $N$ para generar cualquier número en binario a la izquierda, con su complemento a 1 en espejo a la derecha:
                            \begin{equation*}
                                N \rightarrow ANB\ |\ BNA\ |\ AAA\gamma\beta Z
                            \end{equation*}
                            Una vez generado el número, terminaremos añadiendo 3 $A$s a la izquierda (multiplicar por 8), incluyendo los separadores $\gamma$ y $\beta$ y la variable $Z$, que se encargará de sumar 1 al número en complemento a 1 para pasarlo a complemento a 2.
                        \item Usamos ahora $Z$ para pasar el número de la derecha a complemento a 2:
                            \begin{equation*}
                                ZB \rightarrow BZ
                            \end{equation*}
                            Buscamos el primer 0, por lo que saltamos los 1s.
                            \begin{equation*}
                                ZA \rightarrow Z'B
                            \end{equation*}
                            Hemos encontrado el primer 0, lo cambiamos por 1 y volvemos con la variable $Z'$.
                            \begin{equation*}
                                BZ' \rightarrow Z'A
                            \end{equation*}
                            Volvemos a la izquierda, cambiando todos los 1s que saltamos anteriormente por 0s.
                            \begin{equation*}
                                \beta Z'\rightarrow\beta L_0
                            \end{equation*}
                            Una vez llegamos a $\beta$, tenemos el número en complemento a 1 y comenzamos con la aritmética ($L_0$ representa que no hemos cogido ningún número y que no nos llevamos nada de la suma anterior).
                        \item Comenzamos ahora con la aritmética, la parte más complicada de la gramática. Distinguimos dos casos:
                            \begin{enumerate}
                                \item No nos llevamos nada de la operación anterior ($L_0$):
                                    \begin{equation*}
                                        L_0A \rightarrow E_0
                                    \end{equation*}
                                    Cogemos un 0 de la derecha y la variable $E_0$ lo transportará a la izquierda.
                                    \begin{align*}
                                        A E_0 &\rightarrow E_0 A \\
                                        B E_0 &\rightarrow E_0 B \\
                                        \beta E_0 &\rightarrow E_0 \beta
                                    \end{align*}
                                    Nos movemos hacia la izquierda, buscando $\delta$ (que indica por dónde nos quedamos sumando).
                                    \begin{equation*}
                                        \delta E_0 \rightarrow \overline{E_0}
                                    \end{equation*}
                                    Donde la barra indica que hemos ``cogido'' $\delta$, la cual tendremos que soltar en el siguiente dígito.
                                    \begin{align*}
                                        A\overline{E_0} &\rightarrow \delta A E_0' \\
                                        B\overline{E_0} &\rightarrow \delta B E_0' 
                                    \end{align*}
                                    Como estamos sumando 0, dejamos el dígito invariante, sólo movemos $\delta$ hacia la izquierda. Usamos la variable $E_0'$ para volver, que indica que no nos llevamos nada de la suma:
                                    \begin{align*}
                                        E_0' A &\rightarrow AE_0' \\
                                        E_0' B &\rightarrow BE_0' \\
                                        E_0' \beta &\rightarrow \beta L_0
                                    \end{align*}
                                    Nos desplazamos hacia la derecha, hasta encontrar $\beta$, ya que después encontraremos el siguiente dígito con el que operar. Como no nos llevábamos nada, volvemos a $L_0$.

                                    Si ahora no nos llevamos nada y en vez de un 0 (una $A$) hay un 1 (una $B$), repetimos el proceso pero usando para ello $E_1$:
                                    \begin{align*}
                                        L_0 B &\rightarrow E_1 \\
                                        AE_1 &\rightarrow E_1 A \\
                                        BE_1 &\rightarrow E_1 B \\
                                        \beta E_1 &\rightarrow E_1 \beta \\
                                        \delta E_1 &\rightarrow \overline{E_1}
                                    \end{align*}
                                    A continuación, $\overline{E_1}$ se encontrará con el dígito con el que operar:
                                    \begin{align*}
                                        A\overline{E_1} &\rightarrow \delta B E_0' \\
                                        B\overline{E_1} &\rightarrow \delta A E_1'
                                    \end{align*}
                                    \begin{itemize}
                                        \item Si era un 0 (una $A$), lo cambiamos por un 1.
                                        \item Si era un 1 (una $B$), lo cambiamos por un 0 y nos llevamos 1 (que es lo que indica $E_1'$).
                                    \end{itemize}
                                    El comportamiento de $E_1'$ es similar a $E_0'$ pero ahora pasando a $L_1$:
                                    \begin{align*}
                                        E_1' A &\rightarrow A E_1' \\
                                        E_1' B &\rightarrow B E_1' \\
                                        E_1' \beta &\rightarrow \beta L_1 
                                    \end{align*}
                                \item Ahora, estamos en el caso en el que nos llevamos un 1 de la operación anterior ($L_1$), que hemos visto que puede suceder:
                                    \begin{equation*}
                                        L_1A \rightarrow E_1
                                    \end{equation*}
                                    Si nos encontramos un 0 llevando 1, es como si nos hubiéramos encontrado un 1 llevando 0, por lo que no hay nada nuevo que hacer. Sin embargo, si nos encontramos un 1:
                                    \begin{equation*}
                                        L_1B \rightarrow E_2
                                    \end{equation*}
                                    Tenemos que tener en mente que el siguiente dígito con el que realizar la suma lo sumaremos con 2 ($E_2$ es análogo a $E_0$ y $E_1$ pero ahora ``transportando'' un 2):
                                    \begin{align*}
                                        A E_2 &\rightarrow E_2 A \\
                                        B E_2 &\rightarrow E_2 B \\
                                        \beta E_2 &\rightarrow E_2 \beta \\
                                        \delta E_2 &\rightarrow \overline{E_2}
                                    \end{align*}
                                    A continuación, $\overline{E_2}$ se encontrará con el dígito con el que operar:
                                    \begin{align*}
                                        A\overline{E_2} &\rightarrow \delta A E_1' \\
                                        B\overline{E_2} &\rightarrow \delta B E_1'
                                    \end{align*}
                                    Similar al caso de $\overline{E_0}$, dejamos el dígito invariante pero ahora tenemos que llevarnos 1 para la siguiente opereación.
                            \end{enumerate}
                        \item A poco que se piense, como $8x$ y su complemento a 2 tienen la misma cantidad de bits, terminaremos de realizar la operación cuando nos llevemos 1 y no queden bits del número en complemento a 2, dando lugar a:
                            \begin{equation*}
                                \beta L_1 \gamma \rightarrow X
                            \end{equation*}
                            donde $X$ es una variable finalizadora, que usamos para cambiar las $A$s por 0s, las $B$s por 1s y eliminar las variables auxiliares que nos quedan (ya hemos eliminado $\beta$ y $\gamma$ directamente al crear $X$, por lo que nos quedan $\delta$ y $\alpha$):
                            \begin{align*}
                                AX &\rightarrow X0 \\
                                BX &\rightarrow X1 \\
                                \delta X &\rightarrow X \\
                                \alpha X &\rightarrow \veps
                            \end{align*}
                            Cuando lleguemos a $\alpha X$, habremos ``limpiado'' la palaba, por lo que ya podemos quitar todas las variables, generando una palabra de la gramática, que forzosamente tiene que ser un múltiplo de 7 en binario (acabamos de multiplicar cualquier número por 7). Además, como con esta gramática podemos multiplicar cualquier número por 7, esta genera todos los números que son múltiplos de 7 en binario.
                    \end{itemize}
                Mostramos finalmente un ejemplo de producción de una palabra mediante esta gramática. Trataremos de generar ``14'' en binario (la 3ª palabra que usa menos reglas de producción para ser creada, tras 0 y 7):
                \begin{align*}
                    S &\rightarrow \alpha BNABBB \gamma \rightarrow \alpha BANBABBB\gamma \rightarrow \alpha BAAAA \delta \beta Z BABBB\gamma \rightarrow \\
                      &\rightarrow \alpha BAAAA\delta \beta BZABBB\gamma \rightarrow \alpha BAAAA \delta \beta BZ' BBBB \gamma \rightarrow \\
                      &\rightarrow \alpha BAAAA \delta \beta Z' ABBBB \gamma \rightarrow \alpha BAAAA\delta \beta L_0 ABBBB\gamma \rightarrow \\
                      &\rightarrow \alpha BAAAA\delta \beta E_0 BBBB \gamma \rightarrow \alpha BAAAA\delta E_0 \beta BBBB\gamma \rightarrow \\
                      &\rightarrow \alpha BAAAA \overline{E_0}\beta BBBB\gamma \rightarrow \alpha BAAA \delta A E_0' \beta BBBB \gamma \rightarrow \\
                      &\rightarrow \alpha BAAA \delta A\beta L_0 BBBB \gamma \rightarrow \alpha BAAA \delta A\beta E_1 BBB \gamma \rightarrow \\
                      &\rightarrow \alpha BAAA \delta A E_1\beta BBB \gamma \rightarrow \alpha BAAA\delta E_1 A \beta BBB \gamma \rightarrow \\
                      &\rightarrow \alpha B AAA \overline{E_1} A \beta BBB \gamma \rightarrow \alpha BAA \delta B E_0' A \beta BBB \gamma \rightarrow \\ 
                      &\rightarrow \alpha BAA \delta BA E_0' \beta BBB \gamma \rightarrow \alpha BAA \delta BA \beta L_0 BBB \gamma \rightarrow \\
                      &\rightarrow \alpha BAA \delta BA \beta E_1 BB \gamma \rightarrow \alpha BAA \delta BA E_1 \beta BB \gamma \rightarrow \\
                      &\rightarrow \alpha BAA \delta BE_1 A \beta BB \gamma \rightarrow \alpha BAA \delta E_1 BA \beta BB \gamma \rightarrow \\
                      &\rightarrow \alpha BAA \overline{E_1} BA \beta BB \gamma \rightarrow \alpha BA \delta B E_0' BA \beta BB \gamma \rightarrow \\
                      &\rightarrow \alpha BA \delta BBE_0' A \beta BB \gamma \rightarrow \alpha BA \delta BBA E_0' \beta BB \gamma \rightarrow \\
                      &\rightarrow \alpha BA \delta BBA \beta L_0 BB \gamma \rightarrow \alpha BA \delta BBA \beta E_1 B \gamma \rightarrow \alpha BA \delta BBA E_1 \beta B\gamma \rightarrow \\
                      &\rightarrow \alpha BA \delta BBE_1 A \beta B \gamma \rightarrow \alpha BA \delta B E_1 BA \beta B \gamma \rightarrow \alpha BA \delta E_1 BBA \beta B \gamma \rightarrow \\
                      &\rightarrow \alpha BA \overline{E_1} BBA \beta B \gamma \rightarrow \alpha B \delta B E_0' BBA \beta B \gamma \rightarrow \alpha B\delta BB E_0' BA \beta B \gamma \rightarrow \\
                      &\rightarrow \alpha B\delta BBB E_0' A \beta B \gamma \rightarrow \alpha B\delta BBBA E_0' \beta B \gamma \rightarrow \alpha B \delta BBBA \beta L_0 B \gamma \rightarrow \\
                      &\rightarrow \alpha B \delta BBBA \beta E_1 \gamma \rightarrow \alpha B \delta BBBA E_1 \beta \gamma \rightarrow \alpha B \delta BBB E_1 A \beta \gamma \rightarrow \\
                      &\rightarrow \alpha B \delta BB E_1 BA \beta \gamma \rightarrow \alpha B \delta B E_1 BBA \beta \gamma \rightarrow \alpha B \delta E_1 BBBA \delta \gamma \rightarrow \\
                      &\rightarrow \alpha B \overline{E_1} BBBA \beta \gamma \rightarrow \alpha \delta A E_1' BBBA \beta \gamma \rightarrow \alpha \delta AB E_1' BBA \beta \gamma \rightarrow \\
                      &\rightarrow \alpha \delta ABB E_1' BA \beta \gamma \rightarrow \alpha \delta ABBB E_1' A \beta \gamma \rightarrow \alpha \delta ABBBA E_1' \beta \gamma \rightarrow \\
                      &\rightarrow \alpha \delta ABBBA \beta L_1 \gamma \rightarrow \alpha \delta ABBBA X \rightarrow \alpha \delta ABBBX0 \rightarrow \alpha \delta ABBX10 \rightarrow \\
                      &\rightarrow \alpha \delta ABX110 \rightarrow \alpha \delta AX1110 \rightarrow \alpha \delta X01110 \rightarrow \alpha X 01110 \rightarrow 01110
                \end{align*}
            \end{description}
    \end{enumerate}
\end{ejercicio}


\begin{ejercicio}[Complejidad: Extrema (no son libres de contexto)]
    Calcula, de forma razonada, gramáticas que generen cada uno de los siguientes lenguajes:
    \begin{enumerate}
        \item $\{ww \mid w \in \{0,1\}^\ast\}$\\
            Para este lenguaje, hemos construido la gramática $G=(V,T,P,S)$ dada por:
            \begin{align*}
                V &= \{S, \alpha, \beta, \gamma, X, E, E_1, E_0, E', B\} \\
                T &= \{0,1\} \\
                S &= S
            \end{align*}
            $P$ que contiene las reglas de producción que se mostrarán a continuación. 

            La idea principal en la gramática es generar entre las variables $\alpha$ y $\beta$ cualqueir palabra del lenguaje ${\{0,1\}}^{\ast}$. Posteriormente, iremos copiando dicha palabra a la derecha de $\beta$ usando para ello las variables $E$ y $\gamma$, de forma que con $\gamma$ controlaremos la parte de la palabra de la izquierda que ya hayamos copiado a la derecha de $\beta$.

            Finalmente, usaremos $B$ para eliminar cualquier rastro de las variable auxiliares. De esta forma, las reglas de $P$ son:
            \begin{itemize}
                \item Para generar cualquier palabra entre $\alpha$ y $\beta$:
                    \begin{align*}
                        S &\rightarrow \alpha X \beta \\
                        X &\rightarrow 0X\ |\ 1X\ |\ E\gamma
                    \end{align*}
                \item Para coger un 1 y copiarlo a la derecha:

                    Hemos de estar al final de la parte de la palabra no copiada (luego ha de ser $xE\gamma$ siendo $x$ 0 o 1). Posteriormente, avanzamos $\gamma$ a la izquierda para indicar que dicho 1 ya está copiado y cambiamos a la variable que transporta el 1 a la derecha:
                    \begin{equation*}
                        1E\gamma \rightarrow \gamma 1 E_1 
                    \end{equation*}
                    Posteriormente, movemos dicha variable a la derecha:
                    \begin{align*}
                        E_1 1 &\rightarrow 1E_1 \\
                        E_1 0 &\rightarrow 0E_1
                    \end{align*}
                    Cuando lleguemos al final de la palabra de la izquierda, soltamos el 1 al inicio de la palabra de la derecha:
                    \begin{equation*}
                        E_1 \beta \rightarrow E' \beta 1
                    \end{equation*}
                \item Para coger un 0 y copiarlo a la derecha, es una situación análoga pero usamos otra variable:
                    \begin{align*}
                        0E\gamma &\rightarrow \gamma 0 E_0 \\
                        \\
                        E_0 1 &\rightarrow 1E_0 \\
                        E_0 0 &\rightarrow 0E_0 \\
                        \\
                        E_0 \beta &\rightarrow E'\beta 0
                    \end{align*}
                \item Ahora, explicamos $E'$, cuya única funcionalidad es volver al final de la parte no copiada de la palabra de la izquierda:
                    \begin{align*}
                        1E' &\rightarrow E'1 \\
                        0E' &\rightarrow E'0 \\
                        \gamma E' &\rightarrow E\gamma
                    \end{align*}
                \item La copia de la palabra terminará cuando se de $\alpha E\gamma$ (ya que estará toda la palabra copiada a la derecha). En dicho caso, eliminamos todas las variables auxiliares restantes:
                    \begin{align*}
                        \alpha E\gamma &\rightarrow B \\
                        B1 &\rightarrow 1B \\
                        B0 &\rightarrow 0B \\
                        B\beta &\rightarrow \veps 
                    \end{align*}
            \end{itemize}
            Puede demostrarse que el lenguaje generado por esta gramática es el solicitado. Por la complejidad de la gramática, nos limitamos a mostrar un ejemplo para ver de forma intuitiva el buen funcionamiento de la misma.

            Trataremos de generar la cadena: $10111011$ (es decir, ${(1011)}^{2}$):
            \begin{align*}
                S &\rightarrow \alpha X \beta \rightarrow \alpha 1X\beta \rightarrow \alpha 10X\beta \rightarrow \alpha101X \beta \rightarrow \alpha1011X\beta \rightarrow \alpha1011E\gamma\beta \rightarrow \\
                  &\rightarrow \alpha101\gamma1E_1\beta \rightarrow \alpha101\gamma1E'\beta1 \rightarrow \alpha101\gamma E'1\beta1 \rightarrow \alpha101E\gamma1\beta1 \rightarrow \\
                  &\rightarrow \alpha10\gamma1E_11\beta1 \rightarrow \alpha 10\gamma11E_1\beta1 \rightarrow \alpha 10\gamma11E'\beta11 \rightarrow \alpha10\gamma1E'1\beta11 \rightarrow \\
                  &\rightarrow \alpha10\gamma E'11\beta11 \rightarrow \alpha10E\gamma11\beta11 \rightarrow \alpha 1\gamma0E_011\beta11 \rightarrow \alpha1\gamma01E_01\beta11 \rightarrow\\
                  &\rightarrow\alpha1\gamma011E_0\beta11 \rightarrow \alpha1\gamma011E'\beta011 \rightarrow \alpha1\gamma01E'1\beta011 \rightarrow \alpha1\gamma0E'11\beta011 \rightarrow \\
                  &\rightarrow \alpha1\gamma E'011\beta011 \rightarrow \alpha1E\gamma011\beta011 \rightarrow \alpha\gamma1E_1011\beta011 \rightarrow \alpha\gamma10E_111\beta011 \rightarrow \\
                  &\rightarrow \alpha\gamma101E_11\beta011 \rightarrow \alpha\gamma1011E_1\beta011 \rightarrow \alpha\gamma1011E'\beta1011 \rightarrow \alpha\gamma101E'1\beta1011 \rightarrow \\
                  &\rightarrow \alpha\gamma10E'11\beta1011 \rightarrow \alpha\gamma1E'011\beta1011 \rightarrow \alpha\gamma E'1011\beta1011 \rightarrow \alpha E\gamma1011\beta1011 \rightarrow \\
                  &\rightarrow B1011\beta1011 \rightarrow 1B011\beta1011 \rightarrow10B11\beta1011 \rightarrow 101B1\beta1011\rightarrow \\
                  &\rightarrow 1011B\beta1011 \rightarrow10111011
            \end{align*}
        \item $\{a^{n^2} \in \{a\}^{\ast} \mid n\geq 0\}$\\
            La idea que hemos tenido para hacer una gramática que acepte el lenguaje es la siguiente. Si representamos las 5 primeras palabras del lenguaje (ordenándolas por su longitud):
            \begin{gather*}
                \veps \\
                a \\
                aaaa \\
                aaaaaaaaa \\
                aaaaaaaaaaaaaaaa
            \end{gather*}
            Notemos que podemos ordenar las letras de la siguiente forma (olvidándonos de $\veps$, que no será relevante):
            \begin{gather*}
                a \\
                aa\ aa \\
                aaa\ aaa\ aaa\\
                aaaa\ aaaa\ aaaa\ aaaa 
            \end{gather*}
            De forma que tenemos 1 grupo de 1 ``a'', dos grupos de 2 ``a'', 3 grupos de 3 ``a'', \ldots Notemos que dados $n$ grupos de $n$ ``a'', será sencillo construir $n+1$ grupos de $n+1$ ``a'', ya que nos bastará con añadir una ``a'' a cada grupo y con duplicar el último grupo de ``a''.

            Hemos construido una gramática $G = (V, T, S, P)$ que simula este comportamiento inductivo del lenguaje, con lo que el lenguaje generado por la misma es el solicitado. Tenemos:
            \begin{align*}
                V &= \{\alpha, \beta, \delta, \gamma, \sigma, X, A, E, E_{\sigma}, \overline{E}, E', I, R, L, Z \} \\
                T &= \{0,1\} \\
                S &= S
            \end{align*}
            Donde $P$ es el conjunto de reglas de producción que contiene todas las reglas que explicaremos a continuación.

            La idea es que si queremos generar la palabra $a^{n^2}$, que generemos $n-1$ $A$s entre $\alpha$ y $\beta$. Tendremos ya creada una letra $a$ y lo que haremos será que por cada $A$ que hayamos generado, repitamos el proceso inductivo descrito anteriormente. Además, separaremos los ``grupos'' de ``a'' con variables $I$.
            Finalmente, para duplicar un grupo de ``a'', usaremos las variables $\delta$ y $\sigma$.\\

                Empezamos generando nuestro entorno en el que trabajaremos (o la palabra vacía):
                \begin{equation*}
                    S \rightarrow \alpha X \beta Ia\delta \gamma \ |\ \veps
                \end{equation*}
                A continuación, usamos $X$ para generar las $A$s:
                \begin{equation*}
                    X \rightarrow AX\ |\ E
                \end{equation*}
                Una vez terminadas de leer las $A$s, generaremos $E$, que se encargará de ir eliminando una $A$, de realizar el proceso inductivo y de volver al estado inicial, hasta terminar con todas las $A$s generadas.

             Ahora, hacemos que $E$ coja una $A$, con lo que le dejamos salir de la región comprendida por $\alpha$ y $\beta$:
                \begin{equation*}
                    AE\beta \rightarrow \beta E
                \end{equation*}
                Ahora desplazamos la variable $E$ a la derecha, haciendo que cada vez que entre en un grupo de ``a'' (cuando pase una variable $I$) añada una nueva:
                \begin{align*}
                    Ea &\rightarrow aE \\
                    EI &\rightarrow IaE
                \end{align*}
                Cuando la variable $E$ se encuentre con $\delta$, habremos terminado de incrementar las ``a'', con lo que tendremos que duplicar ahora el último grupo de ``a''. Para ello, prepararemos un entorno, de forma que entre $I$ y $\sigma$ vayamos generando el nuevo grupo de ``a'', entre $\sigma$ y $\delta$ se encuentre las ``a'' por copiar; y que entre $\delta$ y $\gamma$ se encuentren las ``a'' que ya hayan sido duplicadas.

                De esta forma, cuando $E$ se encuentre con $\delta$, pasaremos a una variable que busque $I$ para colocar delante suya $\sigma$:
                \begin{align*}
                    E\delta &\rightarrow E_{\sigma} \delta \\
                    aE_{\sigma} &\rightarrow E_{\sigma} a \\
                    IE_{\sigma} &\rightarrow I\sigma R
                \end{align*}
                Ahora, usaremos $R$ para movernos a la derecha tras copiar una letra y $L$ para movernos a la izquierda con el fin de pegar una letra.
                \begin{equation*}
                    Ra \rightarrow aR 
                \end{equation*}
                Nos movemos a la derecha
                \begin{equation*}
                    aR\delta \rightarrow L\delta a
                \end{equation*}
                Cuando lleguemos a $\delta$, guardamos una ``a'' más como copiada.
                \begin{equation*}
                    aL \rightarrow La
                \end{equation*}
                Nos moveremos hacia la izquierda buscando $\sigma$ para crear una $a$:
                \begin{equation*}
                    \sigma L \rightarrow a \sigma R
                \end{equation*}
                Pegaremos una $a$ tras $\sigma$ y repetiremos el proceso.

                El proceso terminará cuando no haya más ``a'' entre $\sigma$ y $\delta$:
                \begin{equation*}
                    \sigma R\delta \rightarrow I\overline{E}
                \end{equation*}
                Cuando hayamos terminado, colocamos una $I$ para hacer efectivo el nuevo grupo. Finalmente, debemos colocar nuevamente $\delta$ a la izquierda de $\gamma$ para la siguiente vez que copiemos. Usamos para ello $\overline{E}$:
                \begin{equation*}
                    \overline{E}a \rightarrow a\overline{E}
                \end{equation*}
                Nos movemos a la izquierda buscando $\gamma$ y cuando la encontremos, colocamos $\delta$:
                \begin{equation*}
                    \overline{E}\gamma \rightarrow E' \delta \gamma
                \end{equation*}
                Usaremos finalmente $E'$ para desplazarnos a la izquierda, tras $\beta$, donde volveremos a la variable $E$, que reiniciará el proceso descrito para realizarlo nuevamente:
                \begin{align*}
                    aE' &\rightarrow E' a \\
                    IE' &\rightarrow E' I \\
                    \beta E' &\rightarrow E\beta
                \end{align*}
                Este proceso terminará cuando no queden $A$s por copiar. En dicho caso, pasaremos a una variable $Z$ que eliminará todas las variables auxiliares:
                \begin{equation*}
                    \alpha E \rightarrow Z
                \end{equation*}
                De esta forma, $Z$ se mueve a la derecha, eliminando todas las variables y pasando a través de las letras:
                \begin{align*}
                    Z\beta &\rightarrow Z \\
                    ZI &\rightarrow Z \\
                    Za &\rightarrow aZ \\
                    Z\delta &\rightarrow Z \\
                    Z\gamma &\rightarrow \veps
                \end{align*}
            Como ejemplo y para comprobar que el lenguaje generado por dicha gramática funciona es el deseado mostramos el siguiente ejemplo, en el que generamos $a^9$:
            \begin{align*}
                S &\rightarrow \alpha X \beta I a \delta\gamma \rightarrow \alpha AX\beta Ia\delta\gamma \rightarrow \alpha AAX\beta Ia\delta\gamma \rightarrow \alpha AAE \beta Ia \delta\gamma \rightarrow \alpha A \beta E Ia\delta\gamma \rightarrow \\
                  &\rightarrow \alpha A\beta IaEa\delta\gamma \rightarrow \alpha A\beta IaaE\delta\gamma \rightarrow \alpha A\beta IaaE_{\sigma}\delta\gamma \rightarrow \alpha A\beta IaE_{\sigma}a\delta\gamma \rightarrow \\ 
                  &\rightarrow\alpha A\beta IE_{\sigma}aa\delta\gamma \rightarrow \alpha A\beta I\sigma Raa\delta\gamma \rightarrow \alpha A\beta I \sigma aRa \delta\gamma \rightarrow \alpha A \beta I \sigma aaR \delta\gamma\rightarrow \\
                  &\rightarrow \alpha A\beta I\sigma aL\delta a\gamma \rightarrow \alpha A \beta I\sigma L a\delta a\gamma \rightarrow \alpha A \beta I a\sigma R a \delta a\gamma  \rightarrow \alpha A \beta Ia\sigma aR\delta a\gamma \rightarrow \\
                  &\rightarrow \alpha A \beta I a\sigma L\delta aa \gamma \rightarrow \alpha A \beta Iaa \sigma R \delta aa \gamma \rightarrow \alpha A\beta IaaI\overline{E}aa\gamma \rightarrow \alpha A \beta IaaIa\overline{E}a\gamma \rightarrow \\
                  &\rightarrow \alpha A\beta IaaIaa\overline{E} \gamma \rightarrow \alpha A \beta IaaIaaE'\delta \gamma \rightarrow \alpha A\beta IaaIaE'a\delta\gamma \rightarrow \alpha A\beta IaaIE'aa\delta\gamma \rightarrow \\
                  &\rightarrow \alpha A\beta IaaE'Iaa\delta\gamma \rightarrow \alpha A \beta IaE'aIaa\delta\gamma \rightarrow \alpha A\beta IE'aaIaa\delta\gamma \rightarrow \\
                  &\rightarrow \alpha A \beta E'IaaIaa\delta\gamma \rightarrow \alpha AE\beta IaaIaa\delta\gamma \rightarrow \ldots
            \end{align*}
            % // TODO: Seguir
        \item $\{a^p \in \{a\}^{\ast} \mid p \text{ es primo}\}$
        \item $\{a^nb^m \in \{a,b\}^{\ast} \mid n\leq m^2\}$
    \end{enumerate}

\end{ejercicio}


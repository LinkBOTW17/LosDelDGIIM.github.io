\subsection{Relación I}
En los siguientes enunciados, $A, B, C, \ldots$ refieren a subconjuntos arbirarios de un conjunto dado $X$, y se pide demostrar la veracidad de las equivalencias o igualdades propuestas.

\begin{ejercicio}
    \begin{equation*}
        A \subseteq B \Longleftrightarrow \mathcal{P}(A) \subseteq \mathcal{P}(B)
    \end{equation*}
    \textbf{Solución:}
    \begin{description}
        \item [$\Longrightarrow)$] Sea $E\in \mathcal{P}(A) \Longrightarrow E\subseteq A\subseteq B \Longrightarrow E\in \mathcal{P}(B) \Longrightarrow \mathcal{P}(A)\subseteq \mathcal{P}(B)$
        \item [$\Longleftarrow)$] $A\in \mathcal{P}(A)\subseteq \mathcal{P}(B)\Longrightarrow A\in \mathcal{P}(B) \Longrightarrow A\subseteq B$
    \end{description}
\end{ejercicio}

\begin{ejercicio}
    \begin{equation*}
        A \subseteq B \Longleftrightarrow A \cap B = A \Longleftrightarrow A \cup B = B
    \end{equation*}    
    \textbf{Solución:}
    Dividimos la demostración en dos partes, demostrando dos equivalencias:
    \begin{enumerate}
        \item $A\subseteq B\Longleftrightarrow A\cap B=A$
            \begin{description}
                \item [$\Longrightarrow)$] $A\cap B = \{x\in X\mid x\in A\land x\in B\} \stackrel{\text{?}}{=}A$
                    \begin{description}
                        \item [$\subseteq)$] Sea $x\in A\cap B \Longrightarrow x\in A \land x\in B \Longrightarrow x\in A$, y por la arbitrariedad de $x\in A\cap B$, deducimos que $A\cap B\subseteq A$.
                        \item [$\supseteq)$] Sea $x\in A\subseteq B\Longrightarrow x\in B\Longrightarrow x\in A\land x\in B \Longrightarrow x\in A\cap B$, y por la arbitrariedad de $x\in A$, tenemos que $x\in A\cap B$.
                    \end{description}
                \item [$\Longleftarrow)$] Sea $x\in A=A\cap B\Longrightarrow x\in A\land x\in B \Longrightarrow x\in B$, y por la arbitrariedad de $x\in A$, $A\subseteq B$.
            \end{description}
        \item $A\subseteq B\Longleftrightarrow A\cup B = B$
            \begin{description}
                \item [$\Longrightarrow)$] $A\cup B = \{x\in X\mid x\in A\lor x\in B\} \stackrel{\text{?}}{=}B$
                    \begin{description}
                        \item [$\subseteq)$] 
                            Sea $x\in A\cup B \Longrightarrow x\in A\lor x\in B$.
                            \begin{itemize}
                                \item Si $x\in A \subseteq B\Longrightarrow x\in B$.
                                \item Si $x\in B$, está claro.
                            \end{itemize}
                            Por la arbitrariedad de $x\in A\cup B$, deducimos que $A\cup B\subseteq B$.
                        \item [$\supseteq)$] 
                            Sea $x\in B\Longrightarrow x\in A\lor x\in B$, y por la arbitrariedad de $x\in B$, tenemos que $B\subseteq A\cup B$.
                    \end{description}
                \item [$\Longleftarrow)$] Sea $x\in A\Longrightarrow x\in A\cup B =B$, y por la arbitrariedad de $x\in A$, tenemos que $A\subseteq B$.
            \end{description}
    \end{enumerate}
\end{ejercicio}

\begin{ejercicio}
    \ 
    \begin{description}
        \item [(a)] $A\cap B = \emptyset \Longleftrightarrow A \subseteq c(B) \Longleftrightarrow B\subseteq c(A)$.
        \item [(b)] $A\cup B = X \Longleftrightarrow c(A) \subseteq B \Longleftrightarrow c(B)\subseteq A$.
    \end{description}
    \textbf{Solución:}
    \begin{description}
        \item [(a)] Lo dividimos en dos equivalencias:
            \begin{description}
                \item [$(i)$] $A\cap B=\emptyset \Longleftrightarrow A\subseteq c(B)$
                    \begin{description}
                        \item [$\Longrightarrow)$] Sea $x\in A\Longrightarrow x\notin B$ (ya que si no $x\in A\cap B=\emptyset $) $\Longrightarrow x\in c(B)$, y por la arbitrariedad de $x\in A$, deducimos que $A\subseteq c(B)$.
                        \item [$\Longleftarrow)$] Supongamos que $A\cap B\neq \emptyset \Longrightarrow \exists x\in A\cap B\Longrightarrow {x\in A\subseteq c(B)} \land {x\in B} \Longrightarrow x\notin B \land x\in B$, contradicción. Por lo que $A\cap B = \emptyset $.
                    \end{description}
                \item [$(ii)$] $A\cap B=\emptyset \Longleftrightarrow B\subseteq c(A)$

                    Por la conmutatividad de la intersección, tenemos que $A\cap B = B\cap A$ y, aplicando el apartado $(i)$, obtenemos que $B\cap A=\emptyset \Longleftrightarrow B\subseteq c(A)$.
            \end{description}
        \item [(b)] Procedemos también mediante dos equivalencias:
            \begin{description}
                \item [$(i)$] $A\cup B=X \Longleftrightarrow c(A)\subseteq B$
                    \begin{description}
                        \item [$\Longrightarrow)$] Sea $x\in c(A) \Longrightarrow x\in X=A\cup B\land x\notin A \Longrightarrow (x\in A\lor x\in B)\land x\notin A \Longrightarrow x\in B$, de donde tenemos que $c(A)\subseteq B$.
                        \item [$\Longleftarrow)$] \ 
                            \begin{description}
                                \item [$\subseteq)$] $A\cup B = \{x\in X\mid x\in A\lor x\in B\} \subseteq X$
                                \item [$\supseteq)$] Sea $x\in X$:
                                    \begin{itemize}
                                        \item Si $x\in A \Longrightarrow x\in A\cup B$.
                                        \item Si $x\notin A\Longrightarrow x\in c(A)\subseteq B\Longrightarrow x\in B\Longrightarrow x\in A\cup B$
                                    \end{itemize}
                                    En conclusión, tenemos que $X\subseteq A\cup B$
                            \end{description}
                    \end{description}
                \item [$(ii)$] $A\cup B=X \Longleftrightarrow c(B) \subseteq A$

                    Por la conmutatividad de la unión, tenemos que $A\cup B = B\cup A$ y, aplicando el apartado $(i)$, obtenemos que $B\cup A=X \Longleftrightarrow c(B)\subseteq A$.
            \end{description}
    \end{description}
\end{ejercicio}

\begin{ejercicio}
    \begin{equation*}
        (A-B)\cap(A-C) = A-(B\cup C)
    \end{equation*}
\end{ejercicio}

\begin{ejercicio}
    \ 
    \begin{description}
        \item [(a)] $A-B=A\Longleftrightarrow A\cap B = \emptyset $.
        \item [(b)] $A\cap(B-C) = (A\cap B) - (A \cap C)$.
    \end{description}
\end{ejercicio}

\begin{ejercicio}
    Siendo la ``diferencia simétrica'' $A\Delta B$ de $A$ y $B$ el subconjunto
    \begin{equation*}
        A\Delta B = (A-B)\cup (B-A)
    \end{equation*}
    demostrad:
    \begin{description}
        \item [(a)] $A\Delta B = (A\cup B) - (A \cap B)$.
        \item [(b)] $A\Delta B = B\Delta A$.
        \item [(c)] $A\Delta \emptyset = A$.
        \item [(d)] $(A\Delta B)\Delta C = A\Delta(B\Delta C)$.
        \item [(e)] $A\cap (B\Delta C) = (A\cap B)\Delta (A\cap C)$.
    \end{description}
\end{ejercicio}

\begin{ejercicio}
    Si $A$ y $B$ son finitos, $|A\cup B| + |A\cap B| = |A| + |B|$.
\end{ejercicio}

\begin{ejercicio}
    Si $A$, $B$ y $C$ son finitos,
    \begin{equation*}
        |A\cup B \cup C| = |A| + |B| + |C| - |A\cap B| - |A\cap C| - |B\cap C| + |A\cap B\cap C|
    \end{equation*}
\end{ejercicio}

\noindent
En los siguientes ejercicios, $P$, $Q$, $R$, \ldots refieren a las propiedades que pueden ser satisfechas, o no, por los elementos de un conjunto $X$.

\begin{ejercicio}
    Argumentar que las siguientes proposiciones son equivalentes.
    \begin{description}
        \item [(a)] $P \Longrightarrow Q$.
        \item [(b)] $P\lor Q \Longleftrightarrow Q$.
        \item [(c)] $P\land Q \Longleftrightarrow P$.
    \end{description}
\end{ejercicio}

\begin{ejercicio}
    Argumentar la veracidad de las siguientes equivalencias.
    \begin{description}
        \item [(a)] $P\lor(Q\land R) \Longleftrightarrow (P\lor Q)\land (P\lor R)$.
        \item [(b)] $P\land(Q\lor R) \Longleftrightarrow (P\land Q)\lor (P\land R)$.
        \item [(c)] $\neg(P\lor Q)\Longleftrightarrow \neg P \land \neg Q$.
        \item [(d)] $\neg(P\land Q)\Longleftrightarrow \neg P \lor \neg Q$.
        \item [(e)] $(P\lor \neg Q) \lor (P\lor \neg R) \Longleftrightarrow P\lor \neg (Q\land R)$.
        \item [(f)] $P\lor Q\lor \neg R \Longleftrightarrow P\lor Q\lor \neg (P\lor R)$.
        \item [(g)] $(P\lor \neg Q)\land(Q\lor \neg P)\Longleftrightarrow (P\land Q)\lor\neg(P\lor Q)$.
    \end{description}
\end{ejercicio}

\begin{ejercicio}
    Sean $S$ y $T$ conjuntos, $A\subseteq S$ y $B\subseteq T$.
    \begin{description}
        \item [(a)] Probar $A\times B$ es un subconjunto de $S\times T$.
        \item [(b)] Probar, con el siguiente ejemlo, que no todo subconjunto $X$ de $S\times T$ es de la forma $X=A\times B$:
            \begin{equation*}
                S = T = \{ 0,1 \} \qquad X = \{(0,0), (1,1)\} \subseteq S\times T
            \end{equation*}
    \end{description}
\end{ejercicio}

\begin{ejercicio}
    Sean $f:S\to T$ y $g:T\to U$ aplicaciones.
    \begin{description}
        \item [(a)] Probar que si ambas son inyectivas, entonces su composición $g\circ f:S\to U$ es también inyectiva.
        \item [(b)] Probar que si ambas son sobreyectivas, entonces su composición $g\circ f:S\to U$ es también sobreyectiva.
        \item [(c)] Si su compuesta $g\circ f:S\to U$ es inyectiva o sobreyectiva, ¿qué podemos decir sobre $f$ y $g$?
    \end{description}
\end{ejercicio}

\begin{ejercicio}
    Sea $f:S\to T$ una aplicación.
    \begin{description}
        \item [(a)] Probar que $f$ es inyectiva si y solo si tiene una \emph{inversa por la izquierda}, es decir, existe una aplicación $g:T\to S$ tal que $g\circ f = id_S$.
        \item [(b)] Dar un ejemplo de una aplicación inyectiva con dos diferentes inversas por la izquierda.
        \item [(c)] Probar que $f$ es sobreyectiva si y solo si tiene una \emph{inversa por la derecha}, es decir, existe una aplicación $g:T\to S$ tal que $f\circ g = id_T$.
        \item [(b)] Dar un ejemplo de una aplicación sobreyectiva con dos diferentes inversas por la derecha.
    \end{description}
\end{ejercicio}

\begin{ejercicio}
    Denotemos por $2=\{0,1\}$ al conjunto con dos elementos. Sea $X$ un conjunto no vacío y sea $2^X$ el conjunto de todas las aplicaciones $f:X\to 2$. Si $A\in \mathcal{P}(X)$, se define su \textbf{aplicación característica} $\chi_A:X\to 2$ por:
    \begin{equation*}
        \chi_A(x) = \left\{
            \begin{array}{ccl}
                1 & \text{si} & x\in A \\
                0 & \text{si} & x\notin A
            \end{array}\right.
    \end{equation*}
    Probar que la correspondencia $A\longmapsto \chi_A$ define una aplicación biyectiva
    \begin{equation*}
        \chi:\mathcal{P}(X) \mathop{\longrightarrow}^{\cong}2^X
    \end{equation*}
\end{ejercicio}

\ \\
\noindent
Toda aplicación $f:S\to T$ determina otras
\begin{equation*}
    f_{*}:\mathcal{P}(S)\to\mathcal{P}(T)\qquad f^{*}:\mathcal{P}(T)\to \mathcal{P}(S)
\end{equation*}
llamadas las aplicaciones \textbf{imagen} e \textbf{imagen inversa} por $f$, respectivamente, que están definidas, para cada $A\subseteq S$ y $X\subseteq T$, por
\begin{equation*}
    f_{*}(A) = \{f(a) \mid a\in A\}\qquad f^{*}(X) = \{a\in S \mid f(a) \in X\}
\end{equation*}
En los ejercicios siguientes, $f:S\to T$ refiere a una aploicación dada, $A$, $B\subseteq S$ son subconjuntos de $S$ y $X, Y \subseteq T$ son subconjunos de $T$.
\begin{observacion}
    Es común en matemáticas usar las aplicaciones imagen e imagen inversa, aunque la notación usual para estas es $f$ y $f^{-1}$, respectivamente. Estas no deben confundirse con la aplicación y con la inversa de la aplicación, dada una aplicación $f:A\to B$, tenemos:
    \begin{equation*}
        f:\mathcal{P}(A)\to \mathcal{P}(B)\qquad f^{-1}:\mathcal{P}(B)\to \mathcal{P}(A)
    \end{equation*}
    que son las aplicaciones imagen e imagen inversa, mientras que la inversa (en caso de existir), es una aplicación $f^{-1}:B\to A$.
\end{observacion}

\begin{ejercicio}
    Probar que $f^{*}(X\cup Y)=f^*(X) \cup f^*(Y)$ y $f_*(A\cup B)=f_*(A)\cup f_*(B)$.
\end{ejercicio}

\begin{ejercicio}
    Probar que $f^*(X\cap Y) = f^*(X)\cap f^*(Y)$ y $f_*(A\cap B)\subseteq f_*(A)\cap f_*(B)$.
\end{ejercicio}

\begin{ejercicio}
    Demostrar que si $f$ es inyectiva, entonces $f_*(A\cap B)=f_*(A)\cap f_*(B)$.
\end{ejercicio}

\begin{ejercicio}
    Demostrar con el siguiente ejemplo que, en general, $f_*(A\cap B)\neq f_*(A)\cap f_*(B)$:\newline
    Sea $f=||:\mathbb{R}\to\mathbb{R}$ la aplicación ``valor absoluto'', $A=(0,1)$ y $B=(-1,0)$.
\end{ejercicio}

\begin{ejercicio}
    $f_*(f^*(X))\subseteq X$, y se da la igualdad si $f$ es sobreyectiva.
\end{ejercicio}

\begin{ejercicio}
    $A\subseteq f^*(f_*(A))$, y se da la igualdad si $f$ es inyectiva.
\end{ejercicio}

\begin{ejercicio}
    Probar que, si $f$ es una biyección, entonces las aplicaciones $f_*:\mathcal{P}(S)\to \mathcal{P}(T)$ y $f^*:\mathcal{P}(T)\to \mathcal{P}(S)$ son biyectivas e inversas una de la otra.
\end{ejercicio}

\resetearcontador
\newpage

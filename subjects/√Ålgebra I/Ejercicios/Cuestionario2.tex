\subsection{Cuestionario II}
\begin{ejercicio}
    Sean $X$ e $Y$ dos conjuntos finitos con $|X| = |Y|$ y $f:X \rightarrow Y$ una aplicación. La afirmación ``Si $f$ es inyectiva o sobreyectiva, entonces $f$ es biyectiva'' es:
    \begin{itemize}
        \item Verdadera o falsa, depende de $f$.
        \item Siempre verdadera.
        \item Siempre falsa.
    \end{itemize}
\end{ejercicio}

\begin{ejercicio}
    Sea $f:X \rightarrow Y$ una aplicación inyectiva y sean $A, B \subseteq X$. Selecciona la afirmación verdadera:
    \begin{itemize}
        \item $f_{*}(A) - f_{*}(B)$ es un subconjunto propio de $f_{*}(A-B)$.
        \item $f_{*}(A-B)$ es un subconjunto propio de $f_{*}(A) - f_{*}(B)$.
        \item $f_{*}(A-B) = f_{*}(A) - f_{*}(B)$.
    \end{itemize}
\end{ejercicio}

\begin{ejercicio}
    Sea $f:X \rightarrow X$ una aplicación tal que $f_{*}(c(A)) = c(f_{*}(A))$, para todo $A \in \mathcal{P}(X)$. Entonces:
    \begin{itemize}
        \item $f$ es inyectiva, pero no necesariamente sobreyectiva.
        \item $f$ es sobreyectiva, pero no necesariamente inyectiva.
        \item $f$ es biyectiva.
    \end{itemize}
\end{ejercicio}

\begin{ejercicio}
Sea $X$ un conjunto con $|X|\geq 2$. La afirmación ``Todo subconjunto de $X \times X$ es de la forma $A \times B$ para ciertos subconjuntos $A, B \subseteq X$'' es:
    \begin{itemize}
        \item Verdadera o falsa, depende de $X$.
        \item Siempre verdadera.
        \item Siempre falsa.
    \end{itemize}
\end{ejercicio}

\begin{ejercicio}
    Sea $R$ una relación simétrica y transitiva en un conjunto $X \neq \emptyset$ ¿Prueba el siguiente razonamiento que $R$ es reflexiva?:\newline
    ``Por simetría, $aRb$ implica $bRa$ y entonces, por transitividad, concluimos que $aRa$.''
    \begin{itemize}
        \item Sí.
        \item No.
    \end{itemize}
\end{ejercicio}

\newpage
\ % --------------------------------------------------------------------------------
\resetearcontador

\begin{ejercicio}
    Sean $X$ e $Y$ dos conjuntos finitos con $|X| = |Y|$ y $f:X \rightarrow Y$ una aplicación. La afirmación ``Si $f$ es inyectiva o sobreyectiva, entonces $f$ es biyectiva'' es:
    \begin{itemize}
        \item Verdadera o falsa, depende de $f$.
        \item \underline{Siempre verdadera.}
        \item Siempre falsa.
    \end{itemize}

    \noindent
    \textbf{Justificación}:
    Si $f$ es inyectiva, entonces $|X| = |Img(f)|$, luego $|Img(f)| = |Y|$ y por tanto, $Img(f) = Y$ y $f$ es sobreyectiva luego biyectiva.\newline
    Si $f$ es sobreyectiva, entonces $|Y|=|Img(f)|$, luego $|Img(f)| = |X|$ y por tanto, $f$ es necesariamente inyectiva luego biyectiva.
\end{ejercicio}

\begin{ejercicio}
    Sea $f:X \rightarrow Y$ una aplicación inyectiva y sean $A, B \subseteq X$. Selecciona la afirmación verdadera:
    \begin{itemize}
        \item $f_{*}(A) - f_{*}(B)$ es un subconjunto propio de $f_{*}(A-B)$.
        \item $f_{*}(A-B)$ es un subconjunto propio de $f_{*}(A) - f_{*}(B)$.
        \item \underline{$f_{*}(A-B) = f_{*}(A) - f_{*}(B)$.}
    \end{itemize}

    \noindent
    \textbf{Justificación}:
    Empezamos recordando la definición de $f_{*}(A)$ para $A \subseteq X$:
    $$f_{*}(A) = \{y \in X \mid \exists x \in X \mbox{ con } f(x) = y \}$$
    \begin{description}
        \item [$\subseteq)$]
            Sea $y \in f_{*}(A-B) \Rightarrow \exists x \in A -B \mid y = f(x)$.\newline
            Esto es, $\exists x \in A \land x \notin B \mid y = f(x)$.\newline
            Como $x \in A \Rightarrow y = f(x) \in f_{*}(A)$. Además, por ser $f$ inyectiva, se tiene que $y \notin f_{*}(B)$, ya que si suponemos que $y \in f_{*}(B)$:

            $y \in f_{*}(B) \Rightarrow \exists b \in B \mid y = f(b) \Rightarrow f(x) = f(b)$ con lo que $x = b \in B$, en contradicción con que $x \notin B$.

            \noindent
            Así, $y \in f_{*}(A) - f_{*}(B)$ para todo $y \in f_{*}(A-B)$. Luego:
            $$f_{*}(A-B) \subseteq f_{*}(A) - f_{*}(B)$$

        \item [$\supseteq)$]
            Sea $y \in f_{*}(A) - f_{*}(B) \Rightarrow y \in f_{*}(A) \land y \notin f_{*}(B)$.\newline
            Como $y \in f_{*}(A) \Rightarrow \exists x \in A \mid y = f(x)$.\newline
            Como $y \notin f_{*}(B) \Rightarrow x \notin B$.\newline
            Luego $x \in A -B \Rightarrow y = f(x) \in f_{*}(A-B)$ para todo $y \in f_{*}(A) - f_{*}(B)$. Luego:
            $$f_{*}(A)-f_{*}(B) \subseteq f_{*}(A-B)$$
    \end{description}
\end{ejercicio}

\begin{ejercicio}
    Sea $f:X \rightarrow X$ una aplicación tal que $f_{*}(c(A)) = c(f_{*}(A))$, para todo $A \in \mathcal{P}(X)$. Entonces:
    \begin{itemize}
        \item $f$ es inyectiva, pero no necesariamente sobreyectiva.
        \item $f$ es sobreyectiva, pero no necesariamente inyectiva.
        \item \underline{$f$ es biyectiva.}
    \end{itemize}

    \noindent
    \textbf{Justificación}:
    Procedemos a demostrar la inyectividad y sobreyectividad de la aplicación.\newline
    Para la sobreyectividad, consideramos $\emptyset \in \mathcal{P}(X)$:
    $$f_{*}(c(\emptyset)) = f_{*}(X) = Img(f) = c(f_{*}(\emptyset)) = c(\emptyset) = X$$
    Para la inyectividad, podemos suponer sin perder generalidad que $|X| \geq 2$ (si no lo fuera, la aplicación sería automáticamente inyectiva).\newline
    Sean $x, x' \in X \mid x \neq x'$. Entonces, $x' \in c(\{x\})$ luego:
    $$f(x') \in f_{*}(c(\{x\})) = c(\{f(x)\})$$
    Luego $f(x') \neq f(x)$.
\end{ejercicio}

\begin{ejercicio}
Sea $X$ un conjunto con $|X|\geq 2$. La afirmación ``Todo subconjunto de $X \times X$ es de la forma $A \times B$ para ciertos subconjuntos $A, B \subseteq X$'' es:
    \begin{itemize}
        \item Verdadera o falsa, depende de $X$.
        \item Siempre verdadera.
        \item \underline{Siempre falsa.}
    \end{itemize}

    \noindent
    \textbf{Justificación}: Supongamos que sí y consideremos el siguiente conjunto:\newline
    Sea $D = \{(x,x) \mid x \in X\} \subseteq X \times X$.\newline
    Si $D = A \times B$ para ciertos $A, B \subseteq X$, entonces para todo $x \in X$, $(x,x) \in A \times B$ y, por tanto, $x \in A$ y $x \in B$.\newline
    Así que $A = X = B$ y, necesariamente, $D = X \times X$. Pero $|X| \geq 2$, luego existen $a,b \in X$ con $a \neq b$, esto es, $(a,b) \notin D$ y $D \neq X \times X$.\newline
    Lo que nos lleva a contradicción.
\end{ejercicio}

\begin{ejercicio}
    Sea $R$ una relación simétrica y transitiva en un conjunto $X \neq \emptyset$ ¿Prueba el siguiente razonamiento que $R$ es reflexiva?:\newline
    ``Por simetría, $aRb$ implica $bRa$ y entonces, por transitividad, concluimos que $aRa$.''
    \begin{itemize}
        \item Sí.
        \item \underline{No.}
    \end{itemize}

    \noindent
    \textbf{Justificación}:
    Dado un $a \in X$, no tiene por qué existir a priori un elemento $b \in X$ tal que $aRb$. Por tanto, buscamos un contraejemplo para desmentir la afirmación:\\

    \noindent
    Dado $X = \{ a,b,c \} \neq \emptyset$ y la relación $R = \{ (a,b), (b,a), (b,b),(a,a) \} \subseteq X \times X$. \newline Observemos que $R$ es simétrica y transitiva pero no reflexiva:

    \noindent
    Es simétrica ya que para todos $\alpha, \beta \in X \mid \alpha R \beta \Rightarrow \beta R \alpha$:
    \begin{center}
        Ya que $aRb$, ¿se cumple que $bRa$?. Sí.\\
        Ya que $bRa$, ¿se cumple que $aRb$?. Sí.\\
        Ya que $bRb$, ¿se cumple que $bRb$?. Sí.\\
        Ya que $aRa$, ¿se cumple que $aRa$?. Sí.
    \end{center}
    Es transitiva ya que para todos $\alpha, \beta, \gamma \in X \mid \alpha R \beta \land \beta R \gamma \Rightarrow \alpha R \gamma$:
    \begin{center}
        Ya que $aRb$ y $bRa$, ¿se cumple que $aRa$?. Sí.\\
        Ya que $bRa$ y $aRb$, ¿se cumple que $bRb$?. Sí.\\
        Ya que $bRb$ y $bRb$, ¿se cumple que $bRb$?. Sí.\\
        Ya que $aRa$ y $aRa$, ¿se cumple que $aRa$?. Sí.
    \end{center}
    No es reflexiva, ya que $\exists c \in X \mid c\cancel{R}c$.
\end{ejercicio}

\newpage
\resetearcontador


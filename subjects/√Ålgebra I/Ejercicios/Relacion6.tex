\subsection{Relación VI}

\begin{ejercicio}
    Estudiar la irreducibilidad de los siguientes polinomios de $\mathbb{Z}[x]$ (factorizando en irreducibles en su caso):
    \begin{equation*}
        \begin{array}{lll}
            \scriptstyle 2x^5-6x^3+9x^2-15 & \scriptstyle x^4+15x^3+7 & \scriptstyle 50x^5+30x^4+100x^2+100x+27 \\
            \scriptstyle 12x^7+6x^4+9x+8 & \scriptstyle x^3+17x+36 & \scriptstyle x^5-x^2+1 \\
            \scriptstyle x^4+10x^3+5x^2-2x-3 & \scriptstyle x^4+6x^3+4x^2-15x+1 & \scriptstyle 6x^4+9x^3-3x^2+1 \\
            \scriptstyle x^5+5x^4+7x^3+x^2-3x-11 & \scriptstyle x^5-10x^4+36x^3-53x^2+26x+1 & \scriptstyle x^4+6x^3+4x^2-15x+1 \\
            \scriptstyle x^6+3x^5-x^4+3x^3+3x^2+3x-1 & \scriptstyle x^4+4x^3-x^2+4x+1 & \scriptstyle x^5-6x^4+3x^3+2x-1 \\
            \scriptstyle 2x^4+2x^3+6x^2+4 & \scriptstyle 2x^4+3x^3+3x^2+3x+1 & \scriptstyle x^4-x^3+9x^2-4x-1 \\
            \scriptstyle x^7+5x^6+x^2+6x+5 & \scriptstyle 3x^5+42x^3-147x^2+21 & \scriptstyle x^5+3x^4+10x^2-2 \\
            \scriptstyle x^4+3x^2-2x+5 & \scriptstyle 3x^6+x^5+3x^2+4x+1 & \scriptstyle 2x^4+x^3+5x+3 \\
            \scriptstyle 2x^5-2x^2-4x-2 & \scriptstyle x^6-2x^5-x^4-2x^3-2x^2-2x-1 & \scriptstyle 3x^4+3x^3+9x^2+6 \\
            \scriptstyle x^4+3x^3+5x^2+1 & \scriptstyle 2x^4+8x^3+10x^2+2 & \scriptstyle x^4+4x^3+6x^2+2x+1
        \end{array}
    \end{equation*}
\end{ejercicio}

\begin{ejercicio}
    En el anillo $\mathbb{Z}[i]$, factorizar $-300$ como producto de una unidad por irreducibles no asociados entre sí.
\end{ejercicio}

\begin{ejercicio}
    En el anillo $\mathbb{Z}[i]$, factorizar $66+12i$ como producto de una unidad por irreducibles no asociados entre sí.
\end{ejercicio}

\begin{ejercicio}
    En el anillo $\mathbb{Z}\left[\sqrt{2}\right]$, factorizar $8+14\sqrt{2}$ como producto de una unidad por irreducibles no asociados entre sí (Indicación: Observar primero que $\sqrt{2}$ es un irreducible en este anillo).
\end{ejercicio}

\begin{ejercicio}
    En $\mathbb{Z}\left[\sqrt{3}\right]$, factoriza $6+2\sqrt{3}$ como producto de una unidad por irreducibles no asociados entre sí (Indicación: Observar primero que $1+\sqrt{3}$, $1-\sqrt{3}$ y $\sqrt{3}$, son irreducibles en este anillo. Observar también que los dos primeros son asociados).
\end{ejercicio}

%\resetearcontador
%\newpage

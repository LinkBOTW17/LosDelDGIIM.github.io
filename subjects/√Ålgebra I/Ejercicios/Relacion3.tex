\subsection{Relación III}

\begin{ejercicio}
    \ 
    \begin{description}
        \item [(a)] 
            Si $A$ y $B$ son anillos conmutativos, probar que el conjunto producto cartesiano $A\times B$ con las operaciones
            \begin{equation*}
                (a,a')+(b,b')=(a+b,a'+b')\qquad (a,a')(b,b')=(ab,a'b')
            \end{equation*}
            es efectivamente un nillos conmutativo. Se llama el ``\emph{anillo producto cartesiano}'' de $A$ y $B$ o ``\emph{anillo producto directo}'' de $A$ y $B$.
        \item [(b)] Escribir las tablas de sumar y multiplicar del anillo producto $\mathbb{Z}_2\times \mathbb{Z}_2$.
    \end{description}
\end{ejercicio}

\begin{ejercicio}
    En el conjunto $\mathbb{Z}$ definimos las operaciones de suma $\oplus$ y producto $\otimes$ por
    \begin{gather*}
        a \oplus b = a + b - 1 \\
        a \otimes b = a + b - ab
    \end{gather*}
    Así, por ejemplo, $2\oplus 3 = 4$ y $2\otimes 3 = -1$. ¿Es $\mathbb{Z}$ un anillo conmutativo con estas operaciones?
\end{ejercicio}

\begin{ejercicio}
    \begin{gather*}
        (a,a') + (b,b') = (a+b, a'+b') \\
        (a,a')\cdot (b,b') = (ab, ab'+a'b)
    \end{gather*}
    ¿Es $\mathbb{Z}\times \mathbb{Z}$ un anillo conmutativo con estas operaciones?
\end{ejercicio}

\begin{ejercicio}
    Calcula el cociente y el resto de dividir.
    \begin{description}
        \item [(a)] 17544 entre 123.
        \item [(b)] -17544 entre -123.
        \item [(c)] 17544 entre -123.
        \item [(d)] -17544 entre 123.
    \end{description}
\end{ejercicio}

\begin{ejercicio}
    Escribir las tablas de sumar y multiplicar de los anillos $\mathbb{Z}_5$ y $\mathbb{Z}_6$.
\end{ejercicio}

\begin{ejercicio}
    Sea $R:\mathbb{Z}\to \mathbb{Z}_n$ la aplicación que asigna a cada entero su resto al dividirlo por $n$ ($n\geq 2$). Probar que, para cualquier natural $m\geq 1$ y $r\in \mathbb{Z}_n$, se verifica que
    \begin{equation*}
        mr = R(mr) = R(m)r
    \end{equation*}
    donde el término $mr=\sum_1^m r$ es el resultado de sumar, en $\mathbb{Z}_n$, $r$ consigo mismo $m$ veces, el término $R(mr)$ es el resto de dividir por $n$ el número natural producto de $m$ y $r$; y el término $R(m)r$ es el producto en $\mathbb{Z}_n$, del resto de dividir $m$ entre $n$ por $r$.

    Utilizando lo anterior, ¿es verdad que si, en $\mathbb{Z}_8$, sumas 7 consigo mismo 23 veces obtienes 1, o que si sumas 6 consigo mismo 125 veces obtienes 6?
\end{ejercicio}

\begin{ejercicio}
    Efectuar los siguientes cálculos en el anillo $\mathbb{Z}\left[\sqrt{3}\right]$:
    \begin{equation*}
        (3+2\sqrt{3})+(4-5\sqrt{3})\qquad (3+2\sqrt{3})(4-5\sqrt{3})\qquad {(2-\sqrt{3})}^{3}
    \end{equation*}
\end{ejercicio}

\begin{ejercicio}
    ¿Cuáles de los siguientes son subanillos de los anillos indicados?
    \begin{description}
        \item [(i)] $\{a\in \mathbb{Q}\mid 3a\in \mathbb{Z}\} \subseteq \mathbb{Q}$.
        \item [(ii)] $\{m + 2n\sqrt{3}\mid m,n \in \mathbb{Z}\} \subseteq \mathbb{R}$.
    \end{description}
\end{ejercicio}

\begin{ejercicio}
    Determinar las unidades del anillo definido por el conjunto $\mathbb{Z}\times \mathbb{Z}$, con las operaciones (ver Ejercicio 3)
    \begin{equation*}
        (a,a') + (b,b') = (a+b, a'+b')\qquad (a,a')\cdot (b,b') = (ab, ab'+a'b)
    \end{equation*}
\end{ejercicio}

\begin{ejercicio}
    Encontrar todas las unidades de los anillos $\mathbb{Z}_6, \mathbb{Z}_7$ y $\mathbb{Z}_8$.
\end{ejercicio}

\begin{ejercicio}
    ¿Cuáles de los siguientes son subanillos de los anillos indicados?
    \begin{description}
        \item [(i)] $\sum a_i x^i \in \mathbb{Z}[x] \mid a_1 \text{\ es\ par}\} \subseteq \mathbb{Z}[x]$.
        \item [(ii)] $\sum a_i x^i \in \mathbb{Z}[x] \mid a_2 \text{\ es\ par}\} \subseteq \mathbb{Z}[x]$.
    \end{description}
\end{ejercicio}

\begin{ejercicio}
    Efectuar las siguientes operaciones en el anillo $\mathbb{Z}_5[x]$:
    \begin{gather*}
        (3+4x+x^2+2x^3) + (3+4x+4x^4 + 3x^3) \\
        (3+4x+x^2+2x^3) + (3+4x+4x^4 + 3x^3) \\
        (2-4x+x^2-2x^3) + (3-4x+4x^2-3x^3) \\
        (2-4x+x^2-2x^3) (3-4x+4x^2-3x^3) 
    \end{gather*}
\end{ejercicio}

\begin{ejercicio}
    Si $p(X)\in \mathbb{Z}_5[x]$ es cualquiera de los cuatro polinomio obtenidos al realizar el ejercicio anterior, calcular $p(1)$ y $p(-1)$ en cada caso.
\end{ejercicio}

\begin{ejercicio}
    Ej conjunto $\mathbb{R}^2$ es un anillo con las operaciones:
    \begin{equation*}
        (a,a')+(b,b') = (a+b,a'+b') \qquad (a,a')\cdot (b,b') = (ab-a'b', ab'+a'b)
    \end{equation*}
    Probar que hay un isomorfismo $\mathbb{R}^2\cong \mathbb{C}$.
\end{ejercicio}

\begin{ejercicio}
    Sea $A$ un anillo conmutativo y $u\in A$ una unidad del anillo. Demostrar que la aplicación $f_u:A\to A$ dada por $f_u(x) = uxu^{-1}$ es un automorfismo de $A$.
\end{ejercicio}

\begin{ejercicio}
    Dado un anillo $A$, demostrar que existe un único homomorfismo de anillos de $\mathbb{Z}$ en $A$.
\end{ejercicio}

\begin{ejercicio}
    Para un anillo $A$, se define la \textbf{característica} de $A$ como el menor entero positivo $n$ tal que $n\cdot 1 = \overbrace{1+\cdots+1}^{n} = 0$, siendo 1 el uno del anillo $A$. Si no existe tal $n$, diremos que la característica de $A$ es 0.

    Demostrar que si $A$ es un anillo de característica $n\geq 2$, entonces existe un único homomorfismo de anillos de $\mathbb{Z}_n$ en $A$.
\end{ejercicio}

\begin{ejercicio}
    Dados dos números naturales, $n, m\geq 2$, dar condiciones par que exista un homomorfismo de anillos de $\mathbb{Z}_n$ en $\mathbb{Z}_m$.
\end{ejercicio}

\resetearcontador
\newpage

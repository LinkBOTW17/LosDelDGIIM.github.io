\subsection{Relación IV}

\begin{ejercicio}
    Argumenta si los siguientes anillos son, o no, Dominios de Integridad:
    \begin{equation*}
        \mathbb{Z}_8 \qquad \mathbb{Z}\left[\sqrt{2}\right]\qquad \mathbb{Z}_3 \qquad \mathbb{Z}\times \mathbb{Z}\qquad \mathbb{Z}_6[x] \qquad \mathbb{Z}[i] \qquad \mathbb{Z}_5[x]
    \end{equation*}
\end{ejercicio}

\begin{ejercicio}
    ¿Es el anillo definido por el conjunto $\mathbb{Z}\times \mathbb{Z}$ con las operaciones:
    \begin{equation*}
        (a,a')+(b,b') = (a+b, a'+b')\qquad (a,a')(b,b') = (ab,ab'+a'b)
    \end{equation*}
    un Dominio de Integridad? (ver Ejercicio 3 de la Relación III).
\end{ejercicio}

\begin{ejercicio}
    ¿Es el anillo definido por el conjunto $\mathbb{Z}$ de los números enteros con las operaciones:
    \begin{equation*}
        a\oplus b = a+b-1 \qquad a\otimes b = a+b-ab
    \end{equation*}
    un Dominio de Integridad? (ver Ejercicio 2 de la Relación III).
\end{ejercicio}

\begin{ejercicio}
    Demuestra que un Dominio de Integridad finito es un cuerpo.
\end{ejercicio}

\begin{ejercicio}
    Sea $n\in \mathbb{Z}$ un entero no cuadrado en $\mathbb{Z}$. Demuestra que el cuerpo de fracciones de $\mathbb{Z}\left[\sqrt{n}\right]$ es $\mathbb{Q}\left[\sqrt{n}\right]$.
\end{ejercicio}

\begin{ejercicio}
    Se define el cuerpo $\mathbb{Q}(x)$ como el cuerpo de fracciones del anillo $\mathbb{Z}[x]$, esto es, $\mathbb{Q}(x) := \mathbb{Q}(\mathbb{Z}[x])$. Demuestra que $\mathbb{Z}[x]$ y $\mathbb{Q}[x]$ tienen el mismo cuerpo de fracciones. Esto es:
    \begin{equation*}
        \mathbb{Q}(\mathbb{Q}[x]) = \mathbb{Q}(x)
    \end{equation*}
\end{ejercicio}

\begin{ejercicio}
    Sea $A=\{ \frac{m}{2^k} \in \mathbb{Q} \mid m \in \mathbb{Z} \land k \geq 0\}$. Argumentar que:
    \begin{description}
        \item [(a)] $A$ es subanillo de $\mathbb{Q}$.
        \item [(b)] $\mathbb{Z}\subsetneq A$.
        \item [(c)] El cuerpo de fracciones de $A$ es el mismo que el de $\mathbb{Z}$, o sea, $\mathbb{Q}$.
    \end{description}
\end{ejercicio}

\begin{ejercicio}
    Sea $A$ un DI y consideremos en $A$ la relación binaria $\sim$ de ser asociados. Esto es, $a\sim b$ si $a$ es asociado con $b$.
    \begin{description}
        \item [(a)] Probar que $\sim$ es una relación de equivalencia en $A$.
        \item [(b)] Sea $A/\sim = \{[a] \mid a\in A\}$, el correspondiente conjunto cociente. Establecemos entre sus elementos ls relación por la cual $[a]\leq [b]$ si $a$ es un divisor de $b$ en el anillo $A$. ¿Está bien definida esa relación en $A/\sim$? ¿Es una relación de orden?
    \end{description}
\end{ejercicio}

\begin{ejercicio}
    Para $n$ un número natural, calcular $\text{mcd}(n,n^2)$, $\text{mcd}(n,n+1)$ y $\text{mcd}(n,n+2)$.
\end{ejercicio}

\begin{ejercicio}
    ¿Podremos rellenar con precisión un depósito de 5388033 litros usando un recipiente de 371? En caso afirmativo, ¿cuántas veces usaremos el recipiente?
\end{ejercicio}

\begin{ejercicio}
    Determinar, si existe, un polinomio $p(x)\in \mathbb{Q}[x]$ tal que:
    \begin{equation*}
        \left(\dfrac{3}{5}x^3+\dfrac{1}{2}x+\dfrac{2}{3}\right)p(x) = \dfrac{9}{20}x^5 + \dfrac{147}{40}x^3+\dfrac{1}{2}x^2+\dfrac{11}{4}x+\dfrac{11}{3}
    \end{equation*}
\end{ejercicio}

\begin{ejercicio}
    Calcular el cociente y el resto de dividir, en el anillo $\mathbb{Q}[x]$, el polinomio $p$ entre el polinomio $q$:
    \begin{gather*}
        p = \dfrac{9}{20}x^5 + \dfrac{147}{40}x^3+\dfrac{1}{2}x^2+\dfrac{17}{4}x+\dfrac{17}{3} \\
        q = \dfrac{3}{5}x^3+\dfrac{1}{2}x+\dfrac{2}{3}
    \end{gather*}
\end{ejercicio}

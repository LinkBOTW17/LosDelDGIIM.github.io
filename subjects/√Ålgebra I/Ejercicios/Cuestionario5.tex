\subsection{Cuestionario V}

\begin{ejercicio}
    En relación con los anillos $\bb{Z}_6$ y $\bb{Z} \times \bb{Z}$, selecciona la afirmación correcta:
    \begin{itemize}
        \item Ambos son DI.
        \item Uno de ellos es DI, pero el otro no.
        \item Ninguno es DI.
    \end{itemize}
\end{ejercicio}

\begin{ejercicio}
    En relación a las siguientes proposiciones, referidas a los elementos de un Dominio de Integridad:
    \begin{enumerate}
        \item [(a)] $a\mid b \land a \nmid c \Rightarrow b \nmid b+c$.
        \item [(b)] $a\mid b \land a \nmid c \Rightarrow a \nmid b+c$.
    \end{enumerate}
    Selecciona la afirmación correcta:
    \begin{itemize}
        \item Ambas son verdad.
        \item Una es verdad y la otra es falsa.
        \item Ambas son falsas.
    \end{itemize}
\end{ejercicio}

\begin{ejercicio}
    Polinomios de grado uno que son unidades en el anillo de polinomios $\bb{Z}_4[x]$:
    \begin{itemize}
        \item No hay.
        \item Hay dos.
        \item Hay infinitos.
    \end{itemize}
\end{ejercicio}

\begin{ejercicio}
    En el anillo $\bb{Z}[i]$:
    \begin{itemize}
        \item $3$ es unidad.
        \item $3$ es irreducible.
        \item $3$ no es irreducible.
    \end{itemize}
\end{ejercicio}

\begin{ejercicio}
    En el anillo $\bb{Z}[i]$:
    \begin{itemize}
        \item $2$ es unidad.
        \item $2$ es irreducible.
        \item $2$ no es irreducible.
    \end{itemize}
\end{ejercicio}

\newpage
\ % --------------------------------------------------------------------------------
\resetearcontador

\begin{ejercicio}
    En relación con los anillos $\bb{Z}_6$ y $\bb{Z} \times \bb{Z}$, selecciona la afirmación correcta:
    \begin{itemize}
        \item Ambos son DI.
        \item Uno de ellos es DI, pero el otro no.
        \item \underline{Ninguno es DI.}
    \end{itemize}

    \noindent
    \textbf{Justificación}:
    \begin{itemize}
        \item En $\mathbb{Z}_6$, $2\cdot 3=0$.
        \item En $\mathbb{Z}\times \mathbb{Z}$, $(1,0)\cdot (0,1)=(0,0)$.
    \end{itemize}
\end{ejercicio}

\begin{ejercicio}
    En relación a las siguientes proposiciones, referidas a los elementos de un Dominio de Integridad:
    \begin{enumerate}
        \item [(a)] $a\mid b \land a \nmid c \Rightarrow b \nmid b+c$.
        \item [(b)] $a\mid b \land a \nmid c \Rightarrow a \nmid b+c$.
    \end{enumerate}
    Selecciona la afirmación correcta:
    \begin{itemize}
        \item Ambas son verdad.
        \item \underline{Una es verdad y la otra es falsa.}
        \item Ambas son falsas.
    \end{itemize}

    \noindent
    \textbf{Justificación}:
    \begin{itemize}
        \item La primera es cierta: si $b=ax$ y fuese $b+c=ay$, tendríamos que $c=ay-ax=a(x-y)$, así que $a\mid c$, lo que es contradictorio.
        \item La segunda es falsa: por ejemplo, en $\mathbb{Z}$, $2\nmid 1$ y $2\nmid 3$, pero $2\mid 1+3=4$.
    \end{itemize}
\end{ejercicio}

\begin{ejercicio}
    Polinomios de grado uno que son unidades en el anillo de polinomios $\bb{Z}_4[x]$:
    \begin{itemize}
        \item No hay.
        \item \underline{Hay dos.}
        \item Hay infinitos.
    \end{itemize}

    \noindent
    \textbf{Justificación}:
    La tabla de multiplicar en $\mathbb{Z}_4$ es:
    \begin{equation*}
       \begin{array}{c|cccc}
           (\mathbb{Z}_4, \cdot) & 0 & 1 & 2 & 3 \\
           \hline
           0 & 0 & 0 & 0 & 0 \\
           1 & 0 & 1 & 2 & 3 \\
           2 & 0 & 2 & 0 & 2 \\
           3 & 0 & 3 & 2 & 1
       \end{array} 
    \end{equation*}
    Buscamos estudiar el cardinal del conjunto:
    \begin{equation*}
        \left\{p \in U(\mathbb{Z}_4[x]) \mid \deg(p) =1\right\}
    \end{equation*}
    Sea $ax+b \in U(\mathbb{Z}_4[x])$ con $a\neq 0$:
    \begin{align*}
        (ax+b)(ax+b) &= 1 \Longrightarrow {(ax+b)}^{2}=1 \Longrightarrow a^2x + 2abx + b^2 = 1 \\
                     &\Longrightarrow a^2 = 0 \quad\land\quad 2ab = 0 \quad\land\quad b^2 = 1
    \end{align*}
    \begin{equation*}
        \left\{\begin{array}{lll}
                a^2 = 0 & \Longrightarrow & a = 2 \\
                2ab = 0 & \Longrightarrow & 4b = 0 \Longrightarrow 0b = 0 \Longrightarrow 0=0 \\
                b^2 = 1 & \Longrightarrow & b = 1 \quad\lor\quad b = 3
        \end{array}\right.
    \end{equation*}
    Luego:
    \begin{gather*}
        2x+1 \in U(\mathbb{Z}_4[x]) \\
        2x+3 \in U(\mathbb{Z}_4[x])
    \end{gather*}
    Tenemos dos polinomios que verifican la segunda opción. Además, la última no puede ser por ser $\mathbb{Z}_4[x]$ finito.
\end{ejercicio}

\begin{ejercicio}
    En el anillo $\bb{Z}[i]$:
    \begin{itemize}
        \item $3$ es unidad.
        \item \underline{$3$ es irreducible.}
        \item $3$ no es irreducible.
    \end{itemize}

    \noindent
    \textbf{Justificación}:
    \begin{equation*}
        N(3) = 9 \neq \pm 1 \Longrightarrow 3 \notin U(\mathbb{Z}[i])
    \end{equation*}
    Para probar que $3$ es irreducible, supongamos una factorización $3=\alpha \cdot \beta$ con $\alpha, \beta \in \mathbb{Z}[i]\setminus U(\mathbb{Z}[i])$. Entonces:
    \begin{equation*}
        N(3) = N(\alpha)N(\beta) \Longrightarrow 9 = N(\alpha)N(\beta) \quad N(\alpha), N(\beta) \in \mathbb{Z}
    \end{equation*}
    Como $\alpha, \beta \notin U(\mathbb{Z}[i]) \Longrightarrow N(\alpha), N(\beta)\neq \pm 1$
    Como $\alpha, \beta \in \mathbb{Z}[i]$, se tiene que:
    \begin{align*}
        N(\alpha) &= a^2 + b^2 \geq 1 \\
        N(\beta) &= {(a')}^{2} + {(b')}^{2} \geq 1
    \end{align*}
    Por tanto, $N(\alpha), N(\beta) \in \N$. Además, $9=N(\alpha)N(\beta)\Longrightarrow N(\alpha)=N(\beta)=3$.
    \begin{equation*}
        N(\alpha) = 3 \Longrightarrow a^2 + b^2 = 3
    \end{equation*}
    Pero $\nexists a,b \in \mathbb{Z} \mid a^2 + b^2 = 3$, por lo que 3 es irreducible.
\end{ejercicio}

\begin{ejercicio}
    En el anillo $\bb{Z}[i]$:
    \begin{itemize}
        \item $2$ es unidad.
        \item $2$ es irreducible.
        \item \underline{$2$ no es irreducible.}
    \end{itemize}

    \noindent
    \textbf{Justificación}:
    \begin{equation*}
        N(2) = 4 \neq 1 \Longrightarrow 2 \notin U(\mathbb{Z}[i])
    \end{equation*}
    Para ver que 2 no es irreducible, supongamos una factorización: $2=\alpha \cdot \beta \mid \alpha, \beta \in \mathbb{Z}[i]\setminus U(\mathbb{Z}[i])$.
    \begin{equation*}
        N(2) = N(\alpha \beta) \Longrightarrow 4=N(\alpha)N(\beta) \Longrightarrow N(\alpha) = N(\beta) = 2
    \end{equation*}
    Por ejemplo, $\alpha = \beta = 1+i$
    \begin{equation*}
        -i{(1+i)}^{2} = (1+i^2 + 2i)(-i) = (-i)(1-1+2i) = (-i)2i = -2i^2 = 2
    \end{equation*}
    Luego $2 = -i{(1+i)}^{2}$ es la factorización esencialmente única de 2 $\Longrightarrow$ es reducible.
\end{ejercicio}
\newpage
\resetearcontador

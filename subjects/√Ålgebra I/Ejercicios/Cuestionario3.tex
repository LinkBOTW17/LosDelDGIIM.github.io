\subsection{Cuestionario III}
\begin{ejercicio}
    Sea $X$ un conjunto no vacío. Definimos en $\mathcal{P}(X)$ operaciones de suma y producto por $A+B = A \cup B$ y $A \cdot B = A \cap B$. Entonces (selecciona la respuesta correcta).
    \begin{itemize}
        \item $\mathcal{P}(X)$ es un anillo conmutativo.
        \item $\mathcal{P}(X)$ no es un anillo conmutativo, falla un axioma.
        \item $\mathcal{P}(X)$ no es un anillo conmutativo, fallan dos axiomas.
    \end{itemize}
\end{ejercicio}

\begin{ejercicio}
    Para enteros $m$ y $n$ tales que $2 \leq m < n$, la afirmación ``$\bb{Z}_m$ es un subanillo de $\bb{Z}_n$'' es:
    \begin{itemize}
        \item Verdadera o falsa, dependiendo de $m$ y de $n$.
        \item Siempre verdadera.
        \item Siempre falsa.
    \end{itemize}
\end{ejercicio}

\begin{ejercicio}
    En el anillo $\bb{Z}_8$ (seleccion la afirmación verdadera).
    \begin{itemize}
        \item $3$ es una unidad y $4 \cdot 3^{-1} = 4$.
        \item $3$ es una unidad, pero $4 \cdot 3^{-1} \neq 4$.
        \item $3$ no es una unidad.
    \end{itemize}
\end{ejercicio}

\begin{ejercicio}
    En el anillo $\bb{Z}[\sqrt{3}]$, la afirmación ``${(7+4\sqrt{3})}^n$ es una unidad para todo natural $n \geq 1$'' es:
    \begin{itemize}
        \item Verdadera o falsa, dependiendo de $n$.
        \item Siempre verdadera.
        \item Siempre falsa.
    \end{itemize}
\end{ejercicio}

\begin{ejercicio}
    Sea $A \subseteq \bb{R}$ un subanillo. La afirmación ``\bb{Z} es un subanillo de A'' es:
    \begin{itemize}
        \item Siempre verdadera.
        \item Siempre falsa.
        \item Verdadera o falsa, dependiendo de $A$.
    \end{itemize}
\end{ejercicio}

\newpage
\ % --------------------------------------------------------------------------------
\resetearcontador

\begin{ejercicio}
    Sea $X$ un conjunto no vacío. Definimos en $\mathcal{P}(X)$ operaciones de suma y producto por $A+B = A \cup B$ y $A \cdot B = A \cap B$. Entonces (selecciona la respuesta correcta).
    \begin{itemize}
        \item $\mathcal{P}(X)$ es un anillo conmutativo.
        \item \underline{$\mathcal{P}(X)$ no es un anillo conmutativo, falla un axioma.}
        \item $\mathcal{P}(X)$ no es un anillo conmutativo, fallan dos axiomas.
    \end{itemize}

    \noindent
    \textbf{Justificación}:
    En este caso, $0 = \emptyset$, ya que:
    \begin{equation*}
        \emptyset + A = \emptyset \cup A = A\quad\forall A \in \mathcal{P}(X)
    \end{equation*}
    Y no hay opuestos, sea $A\neq \emptyset \in \mathcal{P}(X)$:
    \begin{equation*}
        A + B = A \cup B \supseteq A \neq \emptyset\quad\forall B \in \mathcal{P}(X)
    \end{equation*}
    Podemos ver que el resto de axiomas se cumplen:
    
    \begin{itemize}
        \item Conmutativa de la suma:
        \begin{equation*}
            A + B = A \cup B = B \cup A = B + A\quad\forall A,B \in \mathcal{P}(X)
        \end{equation*}
        \item Asociativa de la suma:
        \begin{equation*}
            A + (B + C) = A \cup (B \cup C) = (A \cup B) \cup C = (A+B)+C\quad\forall A,B,C \in \mathcal{P}(X)
        \end{equation*}
        \item Elemento neutro de la suma (ya demostrado).
        \item Existencia de opuestos (ya se ha visto que no se cumple).
        \item Conmutativa del producto:
        \begin{equation*}
            A \cdot B = A \cap B = B \cap A = B \cdot A\quad\forall A,B \in \mathcal{P}(X)
        \end{equation*}
        \item Asociativa del producto:
        \begin{equation*}
            A \cdot (B \cdot C) = A \cap (B \cap C) = (A \cap B) \cap C = (A\cdot B)\cdot C\quad\forall A,B,C \in \mathcal{P}(X)
        \end{equation*}
        \item Elemento neutro del producto:
            \begin{equation*}
                A \cdot X = A\quad\forall A \in \mathcal{P}(X)
            \end{equation*}
        \item Distributiva del producto respecto de la suma:
            \begin{equation*}
                A \cdot (B + C) = A \cap (B \cup C) = (A \cap B) \cup (A \cap C) = (A \cdot B) +(A\cdot C)\quad\forall A,B,C \in \mathcal{P}(X)
            \end{equation*}
    \end{itemize}
\end{ejercicio}

\begin{ejercicio}
    Para enteros $m$ y $n$ tales que $2 \leq m < n$, la afirmación ``$\bb{Z}_m$ es un subanillo de $\bb{Z}_n$'' es:
    \begin{itemize}
        \item Verdadera o falsa, dependiendo de $m$ y de $n$.
        \item Siempre verdadera.
        \item \underline{Siempre falsa.}
    \end{itemize}

    \noindent
    \textbf{Justificación}:
    En $\bb{Z}_m$, se tiene que $m = 0$.\newline
    Sin embargo, por ser $2 \leq m < n$, tenemos que $m \neq 0$ en $\bb{Z}_n$.
\end{ejercicio}

\begin{ejercicio}
    En el anillo $\bb{Z}_8$ (seleccion la afirmación verdadera).
    \begin{itemize}
        \item \underline{$3$ es una unidad y $4 \cdot 3^{-1} = 4$.}
        \item $3$ es una unidad, pero $4 \cdot 3^{-1} \neq 4$.
        \item $3$ no es una unidad.
    \end{itemize}

    \noindent
    \textbf{Justificación}:
    $3$ es una unidad ya que $3 \cdot 3 = 9 = 1$, luego $3^{-1} = 3$.\newline
    Entonces, $4 \cdot 3^{-1} = 4 \cdot 3 = 12 = 4$.
\end{ejercicio}

\begin{ejercicio}
    En el anillo $\bb{Z}[\sqrt{3}]$, la afirmación ``${(7+4\sqrt{3})}^n$ es una unidad para todo natural $n \geq 1$'' es:
    \begin{itemize}
        \item Verdadera o falsa, dependiendo de $n$.
        \item Siempre falsa.
        \item \underline{Siempre verdadera.}
    \end{itemize}

    \noindent
    \textbf{Justificación}:
    Tenemos que $7 + 4\sqrt{3}$ es invertible, puesto que:
    \begin{equation*}
        N(7+4\sqrt{3}) = 7^2 - 3 \cdot 16 = 49 - 48 = 1
    \end{equation*}
    Como el producto de unidades es una unidad, cualquier potencia de una unidad también lo es.
\end{ejercicio}

\begin{ejercicio}
    Sea $A \subseteq \bb{R}$ un subanillo. La afirmación ``\bb{Z} es un subanillo de A'' es:
    \begin{itemize}
        \item \underline{Siempre verdadera.}
        \item Siempre falsa.
        \item Verdadera o falsa, dependiendo de $A$.
    \end{itemize}

    \noindent
    \textbf{Justificación}:
    Por inducción, veamos primero que $\bb{N} = \bb{Z}^{+} \subseteq A$.\newline
    Esto es, que $n \in A\quad\forall n \in \bb{N}$.
    \begin{enumerate}
        \item [$n=0$:]
            Por ser $A$ subanillo de $\bb{R}$, se tiene que $0 \in A$.
        \item [$n=1$:]
            Por ser $A$ subanillo de $\bb{R}$, se tiene que $1 \in A$.
        \item [$n>1$:]
            Como hipótesis de inducción, supongamos que $n \in A$ y veamos que $n+1 \in A$.\newline
            Por ser $A$ cerrado para la suma, tenemos que $1 \in A$ y que $n \in A$ por hipótesis de inducción, luego $n+1 \in A$.
    \end{enumerate}
    Por tanto, $\bb{N} = \bb{Z}^{+} \subseteq A$.\newline
    Ahora, para $n \in \bb{Z}$ con $n \geq 0$, $A$ es cerrado para opuestos, luego $-n \in A$.\newline
    Por tanto, $\bb{Z} \subseteq A$.\\

    \noindent
    Por ser $\bb{Z}$ cerrado para la suma, producto, opuestos y contiene al $0$ y al $1$, $\bb{Z}$ es subanillo de $A$. Por tanto, $\bb{Z}$ es el menor subanillo de $\bb{R}$.
\end{ejercicio}

\newpage
\resetearcontador


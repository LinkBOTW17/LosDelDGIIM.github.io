\subsection{Cuestionario VIII}

\begin{ejercicio}
    En el anillo $\mathbb{Z}[i]$, selecciona las afirmaciones verdaderas:
    \begin{itemize}
        \item $2+ i$ y $2-i$ son unidades.
        \item $2+i$ y $2-i$ son asociados.
        \item $2+i$ y $2-i$ son irreducibles.
    \end{itemize}
\end{ejercicio}

\begin{ejercicio}
    Entre las siguientes afirmaciones, selecciona las afirmaciones verdaderas:
    \begin{itemize}
        \item En el anillo $\mathbb{Z}\left[\sqrt{2}\right]$, los número $2+\sqrt{2}$ y $2-\sqrt{2}$ son asociados.
        \item En el anillo $\mathbb{Z}\left[\sqrt{2}\right]$, los número $2+\sqrt{2}$ y $2-\sqrt{2}$ son primos.
        \item En el anillo $\mathbb{Z}\left[\sqrt{2}\right]$, el número 2 no es primo.
    \end{itemize}
\end{ejercicio}

\begin{ejercicio}
    Entre las siguientes afirmaciones, selecciona las correctas.
    \begin{itemize}
        \item En $\mathbb{Z}[x]$, todo polinomio de grado 1 es irreducible.
        \item En $\mathbb{Z}[x]$, todo polinomio mónico de grado menor o igual que 3 y sin raíces en $\mathbb{Z}$ es irreducible.
        \item Todo polinomio de grado mayor o igual que 1 en $\mathbb{Q}[x]$ es asociado a un primitivo de $\mathbb{Z}[x]$.
    \end{itemize}
\end{ejercicio}

\begin{ejercicio}
    Entre las siguientes afirmaciones relativas a un polinomio $f\in \mathbb{Z}[x]$, selecciona las que son verdad:
    \begin{itemize}
        \item Si el reducido $R_p(f)$ es irreducible en $\mathbb{Z}_p[x]$, entonces $f$ es irreducible.
        \item Si $f$ es mónico y el reducido $R_p(f)$ es irreducible en $\mathbb{Z}_p[x]$, entonces $f$ es irreducible.
        \item Si $f$ es primitivo y el reducido $R_p(f)$ es irreducible en $\mathbb{Z}_p[x]$, entonces $f$ es irreducible.
    \end{itemize}
\end{ejercicio}

\begin{ejercicio}
    Entre las siguientes afirmaciones relativas a un polinomo mónimo $f\in \mathbb{Z}[x]$, selecciona las que son verdad:
    \begin{itemize}
        \item Si $f$ no tiene raíces en $\mathbb{Z}$ y para un primo entero $p\geq 2$, el reducido $R_p(f)$ factoriza en irreducibles $\mathbb{Z}_p[x]$ en la forma $R_p(f) = f_1 \cdot f_2$ con $\deg(f_1)=1$, entonces $f$ es irreducible en $\mathbb{Z}[x]$.
        \item Si para un entero primo $p\geq 2$, el reducido $R_p(f)$ factoriza en irreducibles $\mathbb{Z}_p[x]$ en la forma $R_p(f) = f_1^2$ con $\deg(f_1)=3$ y para un entero primo $q\geq 2$, el reducido $R_q(f)$ factoriza en irredcuibles $\mathbb{Z}_q[x]$ en la forma $R_q(f)=g_1g_2g_3$ con $\deg(g_1)=1=\deg(g_2)$ y $\deg(g_3)=4$, entonces $f$ es irreducible.
        \item Si para un entero primo $p\geq 2$, el reducido $R_p(f)$ factoriza en irreducibles $\mathbb{Z}_p[x]$ en la forma $R_p(f)=f_1^2$ con $\deg(f_1)=2$ y para un entero primo $q\geq 2$, el reducido $R_q(f)$ factoriza en irreducibles $\mathbb{Z}_q[x]$ en la forma $R_q(f)=g_1g_2g_3g_4$ con $\deg(g_1)=1$, entonces $f$ es irreducible.
    \end{itemize}
\end{ejercicio}

\newpage
\ % --------------------------------------------------------------------------------
\resetearcontador

\begin{ejercicio}
    En el anillo $\mathbb{Z}[i]$, selecciona las afirmaciones verdaderas:
    \begin{itemize}
        \item $2+ i$ y $2-i$ son unidades.
        \item $2+i$ y $2-i$ son asociados.
        \item \underline{$2+i$ y $2-i$ son irreducibles.}
    \end{itemize}

    \noindent
    \textbf{Justificación}:
    \begin{itemize}
        \item La primera es falsa:
            \begin{equation*}
                N(2+ì) = N(2-i) = 5 \neq \pm 1
            \end{equation*}
        \item La segunda también:
            \begin{equation*}
                \dfrac{2+i}{2-i} = \dfrac{(2+i)(2+i)}{5} = \dfrac{3+4i}{5} = \dfrac{3}{5}+\dfrac{4}{5}i
            \end{equation*}
            Luego tenemos $q=i+1$ y $r=(2+i)-(2-i)(1+i)=-1\neq 0$, así que $2-i\nmid 2+i$, luego no son asociados.
        \item La tercera es verdad:
            \begin{equation*}
                N(2+i) = N(2-i) = 5 \text{\ que\ es\ un\ primo\ de\ } \mathbb{Z}
            \end{equation*}
    \end{itemize}
\end{ejercicio}

\begin{ejercicio}
    Entre las siguientes afirmaciones, selecciona las afirmaciones verdaderas:
    \begin{itemize}
        \item \underline{En el anillo $\mathbb{Z}\left[\sqrt{2}\right]$, los número $2+\sqrt{2}$ y $2-\sqrt{2}$ son asociados.}
        \item \underline{En el anillo $\mathbb{Z}\left[\sqrt{2}\right]$, los número $2+\sqrt{2}$ y $2-\sqrt{2}$ son primos.}
        \item \underline{En el anillo $\mathbb{Z}\left[\sqrt{2}\right]$, el número 2 no es primo.}
    \end{itemize}

    \noindent
    \textbf{Justificación}:
    Vemos que $2+\sqrt{2} y 2-\sqrt{2}$ son asociados, ya que:
    \begin{gather*}
        \dfrac{2+\sqrt{2}}{2-\sqrt{2}}=\dfrac{{(2+\sqrt{2})}^{2}}{2}=\dfrac{6+4\sqrt{2}}{2}=3+2\sqrt{2} \\
        \dfrac{2-\sqrt{2}}{2+\sqrt{2}}=\dfrac{{(2-\sqrt{2})}^{2}}{2}=3-2\sqrt{2} \\
    \end{gather*}
    Luego $2+\sqrt{2}=(2-\sqrt{2})(3+2\sqrt{2})$ y $2-\sqrt{2}=(2+\sqrt{2})(3-2\sqrt{2})$, así que $2+\sqrt{2}$ y $2-\sqrt{2}$ se dividen mutuamente, luego son asociados (la primera es verdad).

    Puesto que $\mathbb{Z}\left[\sqrt{2}\right]$ es un DE, es un DFU y ser primo es equivalente a ser irreducible. Como:
    \begin{equation*}
        N\left(2+\sqrt{2}\right) = (2+\sqrt{2})(2-\sqrt{2}) = 4-2 = 2
    \end{equation*}
    Es un primo de $\mathbb{Z}$, vemos que tanto $2+\sqrt{2}$ como $2-\sqrt{2}$ son primos (la segunda es verdad):. Como:
    \begin{equation*}
        2=(2+\sqrt{2})(2-\sqrt{2})
    \end{equation*}
    Deducimos que 2 no es irreducible y, por tanto, no es primo (se verifica la tercera).
\end{ejercicio}

\begin{ejercicio}
    Entre las siguientes afirmaciones, selecciona las correctas.
    \begin{itemize}
        \item En $\mathbb{Z}[x]$, todo polinomio de grado 1 es irreducible.
        \item En $\mathbb{Z}[x]$, todo polinomio mónico de grado menor o igual que 3 y sin raíces en $\mathbb{Z}$ es irreducible.
    \item \underline{Todo polinomio de grado mayor o igual que 1 en $\mathbb{Q}[x]$ es asociado a un}\newline
        \underline{ primitivo de $\mathbb{Z}[x]$.}
    \end{itemize}

    \noindent
    \textbf{Justificación}:
    \begin{itemize}
        \item Falsa, sea $f=6x-2$, $\deg(f)=1$ y no es irreducible: $f=2\cdot (3x-1)$.
        \item Sea $f = x^3 + a_2 x^2 + a_1x + a_0$. Por ser mónico, es primitivo. Sus posibles raíces en $\mathbb{Q}$ son de la forma $\nicefrac{a}{b}$ donde $a\mid a_0$ y $b\mid 1 \Longrightarrow b = \pm 1$. 

            Luego sus posibles raíces en $\mathbb{Q}$ son de la forma $\pm a$, donde $a\mid a_0$, luego sus raíces son enteras. Como $f$ no tiene raíces en $\mathbb{Z} \Longrightarrow $ no tiene raíces en $\mathbb{Q}$.

            \begin{description}
                \item [Supuesto $\deg(f)=2 \lor \deg(f)=3$] Entonces, es irreducible en $\mathbb{Q}$ y, por el criterio de al raíz, es irreducible en $\mathbb{Z}$.
                \item [Supuesto $\deg(f)=1$] Entonces, $f=x+a_0 \Longrightarrow x=-a_0$ es raíz de $f$ en $\mathbb{Z}$, contradicción, luego no puede ser $\deg(f)=1$.
                \item [Supuesto $\deg(f)=0$] Entonces, $f\in \mathbb{Z}$ y como es mónico, $f=1\in \mathbb{Z}$. Pero $f=1\in U(\mathbb{Z}[x]) \Longrightarrow f$ no es irreducible.
            \end{description}
            Por lo que es falsa, sólo es cierto si $f\neq 1$.
            
        \item Se ha demostrado que todo $\phi \in \mathbb{Q}[x] \mid \deg(\phi)\geq1$  se puede expresar como $\phi = \nicefrac{a}{b}f$ con $\nicefrac{a}{b}\in \mathbb{Q}$ y $f\in \mathbb{Z}[x]$ primitivo.

            Como $\nicefrac{a}{b}\in \mathbb{Q}$ y $\mathbb{Q}$ es un cuerpo $\Longrightarrow \nicefrac{a}{b}\in U(\mathbb{Q})\Longrightarrow \phi \sim f$, cierto.
    \end{itemize}
\end{ejercicio}

\begin{ejercicio}
    Entre las siguientes afirmaciones relativas a un polinomio $f\in \mathbb{Z}[x]$, selecciona las que son verdad:
    \begin{itemize}
        \item Si el reducido $R_p(f)$ es irreducible en $\mathbb{Z}_p[x]$, entonces $f$ es irreducible.
        \item \underline{Si $f$ es mónico y el reducido $R_p(f)$ es irreducible en $\mathbb{Z}_p[x]$, entonces $f$ es irreducible.}
        \item Si $f$ es primitivo y el reducido $R_p(f)$ es irreducible en $\mathbb{Z}_p[x]$, entonces $f$ es irreducible.
    \end{itemize}

    \noindent
    \textbf{Justificación}:
    \begin{itemize}
        \item Falso, peude ser que $\deg(R_p(f))\neq \deg(f)$:
            \begin{equation*}
                \text{Sea\ } f=2x^2-3x+1=(2x-1)(x-1)\in \mathbb{Z}[x]
            \end{equation*}
            $R_2(f) = x+1\in \mathbb{Z}_2[x]$ es irreducible, pero $f$ es reducible.
        \item 
            \begin{equation*}
                f \text{\ mónico\ } \Longrightarrow \left\{\begin{array}{l}
                    f \text{primitivo} \\
                    \deg(R_p(f)) = \deg(f)
                \end{array}\right.
            \end{equation*}
            Por tanto, aplicando el criterio de reducción, $R_p(f)$ es irreducible en $\mathbb{Z}_p[x] \Longrightarrow f$ es irreducible, cierto.
        \item Falso, puede ser que $\deg(R_p(f)) = \deg(f)$ y tenemos el mismo contraejemplo que para el primer punto.
    \end{itemize}
\end{ejercicio}

\begin{ejercicio}
    Entre las siguientes afirmaciones relativas a un polinomo mónimo $f\in \mathbb{Z}[x]$, selecciona las que son verdad:
    \begin{itemize}
    \item \underline{Si $f$ no tiene raíces en $\mathbb{Z}$ y para un primo entero $p\geq 2$, el reducido $R_p(f)$}\newline
        \underline{ factoriza en irreducibles $\mathbb{Z}_p[x]$ en la forma $R_p(f) = f_1 \cdot f_2$ con $\deg(f_1)=1$,}\newline
        \underline{entonces $f$ es irreducible en $\mathbb{Z}[x]$.}
\item \underline{Si para un entero primo $p\geq 2$, el reducido $R_p(f)$ factoriza en irreducibles}\newline
\underline{$\mathbb{Z}_p[x]$ en la forma $R_p(f) = f_1^2$ con $\deg(f_1)=3$ y para un entero primo $q\geq 2$,}\newline
\underline{el reducido $R_q(f)$ factoriza en irredcuibles $\mathbb{Z}_q[x]$ en la forma $R_q(f)=g_1g_2g_3$}\newline
\underline{ con $\deg(g_1)=1=\deg(g_2)$ y $\deg(g_3)=4$, entonces $f$ es irreducible.}
        \item Si para un entero primo $p\geq 2$, el reducido $R_p(f)$ factoriza en irreducibles $\mathbb{Z}_p[x]$ en la forma $R_p(f)=f_1^2$ con $\deg(f_1)=2$ y para un entero primo $q\geq 2$, el reducido $R_q(f)$ factoriza en irreducibles $\mathbb{Z}_q[x]$ en la forma $R_q(f)=g_1g_2g_3g_4$ con $\deg(g_1)=1$, entonces $f$ es irreducible.
    \end{itemize}

    \noindent
    \textbf{Justificación}:
    \begin{itemize}
        \item Como $f$ es mónico, $f$ y $R_p(f)$ tienen el mismo grado, $n$, y como $f$ no tiene raíces en $\mathbb{Z}$, no tiene divisores de grado 1 ni de grado $n-1$. Además, como $R_p(f)$ no tiene divisores de grado $r$ para cualquier $1<r<n-1$ (sus únicos divisores propios son, salvo asociados, $f_1$ y $f_2$) $f$ tampoco los puede tener. Como es mónico, es primitivo y no tiene divisores propios de grado 0. Luego $f$ es irreducible.
        \item Como es mónico, $f$, $R_p(f)$ y $R_q(f)$ tienen el mismo grado, 6. Como $R_p(f)$ no tiene divisores de grados 1, 2, 4 o 5 $f$ tampoco los puede tener. Como $R_q(f)$ no tiene divisores de grado 3, $f$ tampoco los puede tener. Como es mónico, es primitivo y no tiene divisores propios de grado 0. Luego $f$ es irreducible.
        \item La tercera es falsa: la información sobre $R_p(f)$ nos garantiza que $f$ no tiene divisores de grado 1 o 3, pero puede tenerlos de grado 2, y la segunda información sobre $R_q(f)$ no nos garantiza que $f$ no puede tenerlos. Un contraejemplo sería $f=x^4+2x+1={(x^2+1)}^{2}$. La factorización en irreducibles de $R_3(f)$ en $\mathbb{Z}_3[x]$ es ${(x^2+1)}^{2}$ y la de $R_2(f)$ en $\mathbb{Z}_2[x]$ es ${(x+1)}^{4}$.
    \end{itemize}
\end{ejercicio}

\newpage
\resetearcontador

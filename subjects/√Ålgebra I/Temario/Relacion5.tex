\section{Relación V}
\begin{ejercicio}
    Demostrad
    \begin{enumerate}
        \item $3^{2n}-2^n$ es divisible por 7 para todo entero $n\geq 1$.
        \item $3^{2n+1} + 2^{n+2}$ es divisible por 7 cualquiera que sea el entero $n\geq 1$.
        \item $3^{2n+2}+2^{6n+1}$ es divisible por 11 para todo entero $n\geq 1$.
        \item $3\cdot 5^{2n+1}+2^{3n+1}$ es divisible por 17 cualquiera que sea el entero $n\geq 1$.
        \item Un número es divisible por 4 si y solo si el número formado por sus dos últimas cifras es múltiplo de 4.
    \end{enumerate}
\end{ejercicio}

\begin{ejercicio}
    Sean $a,b\in \mathbb{Z}$. Demostrad que si $3\mid (a^2 + b^2)$ entonces $3\mid a$ y $3\mid b$.
\end{ejercicio}

\begin{ejercicio}
    Demostrad las reglas del 2, 3, 5 y 11 para la división.
\end{ejercicio}

\begin{ejercicio}
    Discutir y resolver el sistema de congruencias
    \begin{equation*}
        \left\{\begin{array}{cccc}
                5x & \equiv & 1 & \mod (14) \\
                11x & \equiv & 10 & \mod(16)
        \end{array}\right.
    \end{equation*}
\end{ejercicio}

\begin{ejercicio}
    Calculad la menor solución positiva del sistema de congruencias
    \begin{equation*}
        \left\{\begin{array}{cccc}
                3x & \equiv & 1 & \mod (4) \\
                2x & \equiv & 2 & \mod(5) \\
                x & \equiv & -1 & \mod(3)
        \end{array}\right.
    \end{equation*}
\end{ejercicio}

\begin{ejercicio}
    Una banda de 13 piratas se repartir $N$ monedas de oro, pero le sobran 8. Dos mueren, las vuelven a repartir y sobran 3. Luego 3 se ahogan y sobran 5. ¿Cuál es la mínima cantidad posible $N$ de monedas?
\end{ejercicio}

\begin{ejercicio}
    En el anillo $\mathbb{Z}\left[\sqrt{3}\right]$, resolved la congruencia
    \begin{equation*}
        (1+\sqrt{3})x\equiv 9-4\sqrt{3}\mod(2\sqrt{3})
    \end{equation*}
\end{ejercicio}

\begin{ejercicio}
    En el anillo $\mathbb{Z}[i]$, resolved el siguiente sistema de congruencias
    \begin{equation*}
        \left\{\begin{array}{cccl}
                x & \equiv & i & \mod(3) \\
                x & \equiv & 1+i & \mod(3+2i) \\
                x & \equiv & 3+2i & \mod(4+i)
        \end{array}\right.
    \end{equation*}
\end{ejercicio}

\begin{ejercicio}
    Determinad todos los polinomios $f(x)\in \mathbb{Z}_5[x]$ tales que
    \begin{equation*}
        (x^4+3x^3+2x^2+3x+1)f(x) \equiv x^4-2x^3-x+2\mod(x^3+3x^2+4x+2)
    \end{equation*}
\end{ejercicio}

\begin{ejercicio}
    Determinad los polinomios $f(x)\in \mathbb{Q}[x]$ de grado menor o igual que tres que satisfacen el sistema de congruencias
    \begin{equation*}
        \begin{array}{cccl}
            f(x) & \equiv & x-1 & \mod(x^2+1) \\
            f(x) & \equiv & x+1 & \mod(x^2+x+1)
        \end{array}
    \end{equation*}
\end{ejercicio}

\begin{ejercicio}
    Probad el Teorema de Ruffini:\newline
    Si $f(x)\in A[x]$, para cualquier $a\in A$, $f(a)$ es igual al resto de dividir $f(x)$ entre $(x-a)$.
\end{ejercicio}

\begin{ejercicio}
    Encontrad un polinomio $f(x)\in \mathbb{Q}[x]$ de grado 3 tal que:
    \begin{equation*}
        f(0) = 6\qquad f(1)=12\qquad f(x)\equiv(3x+3)\mod(x^2+x+1)
    \end{equation*}
\end{ejercicio}

\begin{ejercicio}
    Determinad todos los polinomios $f(x)\in \mathbb{Z}_2[x]$ de grado menor o igual que 4, tales que:
    \begin{enumerate}
        \item El resto de dividir $f(x)$ entre $x^2+1$ es $x$.
        \item El resto de dividir $xf(x)$ entre $x^2+x+1$ es $x+1$.
        \item $f(1)=1$.
    \end{enumerate}
\end{ejercicio}

\begin{ejercicio}
    Calculad el resto de dividir $279^{323}$ entre 17.
\end{ejercicio}

\begin{ejercicio}
    Calculad las dos últimas cifras de $3^{3^{100}}$.
\end{ejercicio}

\begin{ejercicio}
    Resolved, si es posible, la congruencia $43^{51}x\equiv 2\mod(36)$
\end{ejercicio}

\begin{ejercicio}
    Estudiad si $5^{10077}$ es una unidad en $\mathbb{Z}_{38808}$. Calculad su inverso en caso de que lo tenga.
\end{ejercicio}

\begin{ejercicio}
    Resuelve las ecuaciones siguientes.
    \begin{enumerate}
        \item $12x=8$ en el anillo $\mathbb{Z}_{20}$.
        \item $19x=42$ en $\mathbb{Z}_{50}$.
        \item $9x=4$ en $\mathbb{Z}_{1453}$.
        \item $5^{30}x=2$ en $\mathbb{Z}_{7}$.
        \item $20x=984$ en $\mathbb{Z}_{1984}$.
    \end{enumerate}
\end{ejercicio}

\begin{ejercicio}
    Determina los inversos (si existen) de
    \begin{enumerate}
        \item 15 en $\mathbb{Z}_{16}$.
        \item 9 en $\mathbb{Z}_{20}$.
        \item 12 en $\mathbb{Z}_{21}$.
        \item 22 en $\mathbb{Z}_{31}$.
    \end{enumerate}
\end{ejercicio}

\begin{ejercicio}
    Determina cuántas unidades tienen los anillos
    \begin{enumerate}
        \item $\mathbb{Z}_{125}$.
        \item $\mathbb{Z}_{72}$.
        \item $\mathbb{Z}_{88}$.
        \item $\mathbb{Z}_{1000}$.
    \end{enumerate}
\end{ejercicio}

\begin{ejercicio}
    Determina si la igualdad $a=b$ es cierta en los siguientes casos:
    \begin{enumerate}
        \item $a=9^{55^{9}}$ y $b=7^{70^{55}}$, en el anillo $\mathbb{Z}_{21}$.
        \item $a=2^{5^{70}}$ y $b=5^{70^{2}}$, en el anillo $\mathbb{Z}_{21}$.
        \item $a=12^{55^{70}}$ y $b=10^{70^{55}}$, en el anillo $\mathbb{Z}_{22}$.
        \item $a=5^{5^{70}}$ y $b=10^{70^{22}}$, en el anillo $\mathbb{Z}_{22}$.
    \end{enumerate}
\end{ejercicio}

\begin{ejercicio}
    Sea $K$ un cuerpo. Dado un polinomio $f\in K[x]$ cuyo grado es 2 ó 3, demostrar que $f$ es irreducible si, y sólo si, $f$ no tiene raíces en $K$.
\end{ejercicio}

\begin{ejercicio}
    Sea $I$ el ideal principal de $\mathbb{Z}_3[x]$ generado por $x^2+2x+2$. Demostrar que el anillo cociente $\mathbb{Z}_3[x]/I$ es un cuerpo y hallar el inverso de $(ax+b)+I$.
\end{ejercicio}

\begin{ejercicio}
    Determinar los inversos (si existen) de
    \begin{enumerate}
        \item La clase de $x^2+x+1$ en el anillo $\mathbb{Z}_3[x]/\langle x^3+2x+1 \rangle$.
        \item La clase de $x+1$ en el anillo $\mathbb{R}[x]/\langle x^3-2x-3\rangle$.
        \item La clase de $x^2+x$ en el anillo $\mathbb{Z}_2[x]/\langle x^2+1\rangle$.
        \item La clase de $x^3+x+1$ en el anillo $\mathbb{Z}_2[x]/\langle x^2+x+1\rangle$.
        \item La clase del polinomio $2x+1$ en el anillo $\mathbb{Q}[x]/\langle x^3+2x^2+4x-2\rangle$.
        \item La clase de $x$ en el anillo $\mathbb{Q}[x]/\langle x^4 + x + 1\rangle$.
    \end{enumerate}
\end{ejercicio}

\begin{ejercicio}
    Determinar los inversos (si existen) de
    \begin{enumerate}
        \item La clase de $1+i$ en el anillo $\mathbb{Z}[i]/\langle 3+2i\rangle$.
        \item La clase de $2-\sqrt{2}$ en el anillo $\mathbb{Z}\left[\sqrt{2}\right]/\langle 3\rangle$.
        \item La clase de $3+3i\sqrt{2}$ en el anillo $\mathbb{Z}[i\sqrt{2}]/\langle 4-2i\sqrt{2} \rangle $.
        \item La clase de $1+\sqrt{3}$ en el anillo $\mathbb{Z}\left[\sqrt{3}\right]/\langle \sqrt{3} \rangle $.
    \end{enumerate}
\end{ejercicio}

\begin{ejercicio}
    Construir cuerpos con 4 y 8 elementos.
\end{ejercicio}

\begin{ejercicio}
    Calcular las unidades de los anillos cocientes $\mathbb{Z}_5[x]/\langle x^2+x+1 \rangle $, $\mathbb{Z}_5[x]/\langle x^2+1 \rangle $ y $\mathbb{Z}_2[x]/\langle x^2+2 \rangle $.
\end{ejercicio}

\begin{ejercicio}
    Demostrar que el anillo cociente $\mathbb{Z}_2[x]/\langle x^4+x+1 \rangle $ es un cuerpo y calcular el inverso de la clase de $x^2+1$.
\end{ejercicio}

\resetearcontador
\newpage

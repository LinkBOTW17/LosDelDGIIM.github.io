\section{Cuestionario IV}
\begin{ejercicio}
    En el anillo $\bb{Z}_{10}$, la afirmación ``$3^{4k+3} = -3$, para cualquier $k \in \bb{Z}$'' es:
    \begin{itemize}
        \item Siempre falsa.
        \item Siempre cierta.
        \item A veces cierta y a veces falsa, depende de $k$.
    \end{itemize}
\end{ejercicio}

\begin{ejercicio}
    En el anillo $\bb{Z}_n[x]$, la afirmación ``la suma reiterada $n$ veces de cualquier polinomio es $0$'', es:
    \begin{itemize}
        \item Verdera o falsa, depende de $n$.
        \item Siempre falsa.
        \item Siempre verdadera.
    \end{itemize}
\end{ejercicio}

\begin{ejercicio}
    Un subanillo $A$ de un anillo $B$ se dice propio si $A \subsetneq B$. Seleccion el enunciado correcto:
    \begin{itemize}
        \item En anillo $\bb{Z}$ no tiene subanillos propios.
        \item El conjunto $A = \{ 5k \mid k \in \bb{Z} \}$ es un subanillo propio de $\bb{Z}$.
        \item El cuerpo $\bb{Q}$ no tiene subanillos propios.
    \end{itemize}
\end{ejercicio}

\begin{ejercicio}
    Homomorifismos $\phi : \bb{Z}_2 \rightarrow \bb{Z}$,
    \begin{itemize}
        \item Hay exactamente uno.
        \item Hay al menos dos.
        \item No hay ninguno.
    \end{itemize}
\end{ejercicio}

\begin{ejercicio}
    Sea $A$ un anillo comutativo, la afirmación ``Para cualesquiera indeterminadas $x$ e $y$, los anillos de polinomios $A[x]$ y $A[y]$ son isomorifos''. Es:
    \begin{itemize}
        \item Verdadera o falsa, depende de $A$.
        \item Siempre verdadera.
        \item Siempre falsa.
    \end{itemize}
\end{ejercicio}

\newpage
\ % --------------------------------------------------------------------------------
\resetearcontador

\begin{ejercicio}
    En el anillo $\bb{Z}_{10}$, la afirmación ``$3^{4k+3} = -3$, para cualquier $k \in \bb{Z}$'' es:
    \begin{itemize}
        \item Siempre falsa.
        \item \underline{Siempre cierta.}
        \item A veces cierta y a veces falsa, depende de $k$.
    \end{itemize}

    \noindent
    \textbf{Justificación}:
    \begin{equation*}
    3^{4k+3}={(3^4)}^k \cdot 3^3 = {(9 \cdot  9)}^k \cdot 9 \cdot 3 = 1^k \cdot 7 = 7\quad \forall k \in \mathbb{Z}
    \end{equation*}
\end{ejercicio}

\begin{ejercicio}
    En el anillo $\bb{Z}_n[x]$, la afirmación ``la suma reiterada $n$ veces de cualquier polinomio es $0$'', es:
    \begin{itemize}
        \item Verdera o falsa, depende de $n$.
        \item Siempre falsa.
        \item \underline{Siempre verdadera.}
    \end{itemize}

    \noindent
    \textbf{Justificación}:
    Sea $R_n:\mathbb{Z}[x]\to \mathbb{Z}_n[x]$ el homomorfismo de reducción módulo $n$. Para cualquier $f \in \mathbb{Z}_n[x]$:
    \begin{equation*}
        nf = nR_n(f) = R_n(nf) = R_n(n)R_n(f) = 0 \cdot f = 0
    \end{equation*}
\end{ejercicio}

\begin{ejercicio}
    Un subanillo $A$ de un anillo $B$ se dice propio si $A \subsetneq B$. Seleccion el enunciado correcto:
    \begin{itemize}
        \item \underline{En anillo $\bb{Z}$ no tiene subanillos propios.}
        \item El conjunto $A = \{ 5k \mid k \in \bb{Z} \}$ es un subanillo propio de $\bb{Z}$.
        \item El cuerpo $\bb{Q}$ no tiene subanillos propios.
    \end{itemize}

    \noindent
    \textbf{Justificación}:
    Si $A$ es un subanillo de $\mathbb{Z}$, entonces $1 \in A$ con lo que para todo $n \geq 0$, $\overbrace{1+\cdots+1}^{n \text{\ veces}}=n \in A$ y, como $A$ contiene a sus opuestos, entonces $\mathbb{Z}\subseteq A$. Por lo que $A = \mathbb{Z}$.
\end{ejercicio}

\begin{ejercicio}
    Homomorifismos $\phi : \bb{Z}_2 \rightarrow \bb{Z}$,
    \begin{itemize}
        \item Hay exactamente uno.
        \item Hay al menos dos.
        \item \underline{No hay ninguno.}
    \end{itemize}

    \noindent
    \textbf{Justificación}:
    Si $\phi:\mathbb{Z}_2\to \mathbb{Z}$ fuese un homomorfismo, tendríamos que:
    \begin{equation*}
        \phi(1+1) = \phi(1) + \phi(1) = 1+1 = 2
    \end{equation*}
    Pero en $\mathbb{Z}_2$, $1+1=0$ y por tanto, $\phi(1+1)=\phi(0)=0$, así que sería $0 = 2$ en $\mathbb{Z}$, lo que es una contradicción.
\end{ejercicio}

\begin{ejercicio}
    Sea $A$ un anillo comutativo, la afirmación ``Para cualesquiera indeterminadas $x$ e $y$, los anillos de polinomios $A[x]$ y $A[y]$ son isomorifos''. Es:
    \begin{itemize}
        \item Verdadera o falsa, depende de $A$.
        \item \underline{Siempre verdadera.}
        \item Siempre falsa.
    \end{itemize}

    \noindent
    \textbf{Justificación}:
    El automorfismo identidad $id_A:A \cong A$ extiende a un único homomorfismo $\phi:A[x]\to A[y]$ tal que $\phi(x)=y$. Explícitamente:
    \begin{equation*}
        \phi\left(\sum_{i=0}^{n} a_i x^i\right) = \sum_{i=0}^{n} a_i y^i
    \end{equation*}
    Claramente $\phi$ es biyectiva.
\end{ejercicio}

\newpage
\resetearcontador


\section{Cuestionario VI}

\begin{ejercicio}
    En relación a las siguientes proposiciones, referidas a elementos cualesquiera de un DI, selecciona las verdaderas:
    \begin{itemize}
        \item $c\mid ab \Longrightarrow c\mid a \lor c\mid b$.
        \item $a\mid c \land b\mid c \Longrightarrow ab\mid c$. 
        \item $c\mid a \lor c\mid b \Longrightarrow c\mid ab$.
    \end{itemize}
\end{ejercicio}

\begin{ejercicio}
    Entre los siguientes DE, selecciona aquellos en los que el máximo común divisor y el mínimo común múltiplo son únicos salvo signo:
    \begin{itemize}
        \item $\mathbb{Z}\left[\sqrt{-2}\right]$.
        \item $\mathbb{Z}\left[\sqrt{3}\right]$. 
        \item $\mathbb{Z}_3[x]$.
    \end{itemize}
\end{ejercicio}

\begin{ejercicio}
    En un DE, tenemos la ecuación diofántica $px+by=1$, donde $p$ es irreducible. Entre las siguientes afirmaciones, selecciona la que es verdad.
    \begin{itemize}
        \item Nunca tiene solución.
        \item Puede tener solución o no, depende de $b$. 
        \item Siempre tiene solución.
    \end{itemize}
\end{ejercicio}

\begin{ejercicio}
    En un DE, tenemos la ecuación diofántica $px+qy=c$, donde $p$ y $q$ son irreducibles no asociados entre sí. Entre las siguientes afirmaciones, selecciona la que es verdad.
    \begin{itemize}
        \item Nunca tiene solución.
        \item Puede tener solución o no, depende de $p$ y de $q$. 
        \item Siempre tiene solución.
    \end{itemize}
\end{ejercicio}

\begin{ejercicio}
    Entre las siguientes proposiciones, referidas a un DE, selecciona las verdaderas.
    \begin{itemize}
        \item Si la ecuación $ax+by=1$ tiene solución, entonces la ecuación $ax+by=c$ tiene solución para todo $c$.
        \item Si la ecuacióin $ax+bb'y=1$ tiene solución, entonces las ecuaciones $ax+by=1$ y $ax+b'y=1$ tienen solución. 
        \item Si las ecuaciones $ax+by=1$ y $ax+b'y=1$ tienen solución, entonces la ecuación $ax+bb'y=1$ tiene solución.
    \end{itemize}
\end{ejercicio}

\newpage
\ % --------------------------------------------------------------------------------
\resetearcontador

\begin{ejercicio}
    En relación a las siguientes proposiciones, referidas a elementos cualesquiera de un DI, selecciona las verdaderas:
    \begin{itemize}
        \item $c\mid ab \Longrightarrow c\mid a \lor c\mid b$.
        \item $a\mid c \land b\mid c \Longrightarrow ab\mid c$. 
        \item \underline{$c\mid a \lor c\mid b \Longrightarrow c\mid ab$.}
    \end{itemize}

    \noindent
    \textbf{Justificación}:
    \begin{itemize}
        \item La primera es falsa, en $\mathbb{Z}$, $6\mid 12 = 4 \cdot 3$ pero $6\nmid 4$.
        \item La segunda es falsa, en $\mathbb{Z}$, $2\mid 6$ pero $2 \cdot 2 \nmid 6$.
        \item La tercera es verdadera. De hecho, basta con que $c$ divida a uno de ellos para que divida al producto:
            \begin{equation*}
                a = ca' \Longrightarrow ab=c(a'b)
            \end{equation*}
    \end{itemize}
\end{ejercicio}

\begin{ejercicio}
    Entre los siguientes DE, selecciona aquellos en los que el máximo común divisor y el mínimo común múltiplo son únicos salvo signo:
    \begin{itemize}
        \item \underline{$\mathbb{Z}\left[\sqrt{-2}\right]$.}
        \item $\mathbb{Z}\left[\sqrt{3}\right]$. 
        \item \underline{$\mathbb{Z}_3[x]$.}
    \end{itemize}

    \noindent
    \textbf{Justificación}:
    Serán aquellos cuyas unidades sean $\pm 1$:
    \begin{itemize}
        \item En $\mathbb{Z}\left[\sqrt{-2}\right]$, $a+b\sqrt{-2}$ es unidad si y sólo si $a^2 + 2b^2 =1$, lo que sólo se verifica si $a=1$ y $b=0$.
        \item En $\mathbb{Z}\left[\sqrt{3}\right]$, $a+b\sqrt{3}$ es unidad si y sólo si $a^2 - 3b^2 =\pm 1$, lo que verifica por ejemplo $2+\sqrt{3}\neq \pm 1$, luego aquí el mcd y el mcm no son únicos salvo signo.
        \item En $\mathbb{Z}_3[x]$:
            \begin{equation*}
                U\left(\mathbb{Z}_3[x]\right) = U\left(\mathbb{Z}_3\right) = \{1,2\} = \{ 1, -1 \} = \{ \pm 1\}
            \end{equation*}
    \end{itemize}
\end{ejercicio}

\begin{ejercicio}
    En un DE, tenemos la ecuación diofántica $px+by=1$, donde $p$ es irreducible. Entre las siguientes afirmaciones, selecciona la que es verdad.
    \begin{itemize}
        \item Nunca tiene solución.
        \item \underline{Puede tener solución o no, depende de $b$.} 
        \item Siempre tiene solución.
    \end{itemize}

    \noindent
    \textbf{Justificación}:
    La ecuación tendrá solución $\Longleftrightarrow \text{mcd}(p,b)\mid 1 \Longleftrightarrow \text{mcd}(p,b)=1$. Como $p$ es irreducible, equivale a que $p \nmid b$, luego puede tener solución o no, dependiendo de $b$:
    \begin{itemize}
        \item Para $b=1$ sí tiene solución.
        \item Pero para $b=2p \Longrightarrow \text{mcd}(p,2p)=p\neq 1$ no tiene solución.
    \end{itemize}
\end{ejercicio}

\begin{ejercicio}
    En un DE, tenemos la ecuación diofántica $px+qy=c$, donde $p$ y $q$ son irreducibles no asociados entre sí. Entre las siguientes afirmaciones, selecciona la que es verdad.
    \begin{itemize}
        \item Nunca tiene solución.
        \item Puede tener solución o no, depende de $p$ y de $q$. 
        \item \underline{Siempre tiene solución.}
    \end{itemize}

    \noindent
    \textbf{Justificación}:
    La ecuación tendrá solución $\Longleftrightarrow \text{mcd}(p,q)\mid c$. Como $p$ y $q$ son irreducibles no asociados, tenemos que $\text{mcd}(p,q)=1$ y como $1\mid c$ $\forall c \in A$, la ecuación siempre tendrá solución.
\end{ejercicio}

\begin{ejercicio}
    Entre las siguientes proposiciones, referidas a un DE, selecciona las verdaderas.
    \begin{itemize}
    \item \underline{Si la ecuación $ax+by=1$ tiene solución, entonces la ecuación $ax+by=c$} \newline
        \underline{tiene solución para todo $c$.}
\item \underline{Si la ecuacióin $ax+bb'y=1$ tiene solución, entonces las ecuaciones $ax+by=1$}
    \underline{y $ax+b'y=1$ tienen solución.} 
\item \underline{Si las ecuaciones $ax+by=1$ y $ax+b'y=1$ tienen solución, entonces la }\newline
    \underline{ecuación $ax+bb'y=1$ tiene solución.}
    \end{itemize}

    \noindent
    \textbf{Justificación}:
    \begin{itemize}
        \item Sea $(x_0,y_0)$ solución de $ax+by=1 \Longrightarrow (cx_0, cy_0)$ es solución de $ax+by=c$.
        \item Sea $(x_0,y_0)$ solución de $ax+bb'y=1 \Longrightarrow (x_0, y_0b')$ es solución de $ax+by=1$ y $(x_0, y_0b)$ es solución de $ax+b'y=1$.
        \item 
            \begin{equation*}
                \left.
                    \begin{array}{lcr}
                        ax+by=1 \text{\ tiene\ solución} & \Longrightarrow & \text{mcd}(a,b)=1 \\
                        ax+b'y=1 \text{\ tiene\ solución} & \Longrightarrow & \text{mcd}(a,b')=1
                    \end{array}
                \right\} \Longrightarrow \text{mcd}(a,bb')=1
            \end{equation*}
            Luego $ax+bb'y=1$ tiene solución. 
    \end{itemize}
\end{ejercicio}

\newpage
\resetearcontador

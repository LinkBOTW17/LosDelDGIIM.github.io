% Iniciamos el contador a 0
\setcounter{section}{-1}
\section{Preeliminares}

\begin{ejercicio}
    Se estudian las plantas de una determinada zona donde ha atacado un virus. La probabilidad de
    que cada planta esté contaminada es $0.35$.
    \begin{enumerate}
        \item ¿Cuál es el número esperado de plantas contaminadas en 5 analizadas?\\
        Sea $X$ la variable aleatoria que representa el número de plantas contaminadas en $5$ análisis.
        Como la probabilidad de que una planta esté contaminada es $0.35$, tenemos que sigue una distribución de probabilidad binomial con $n=5$ y $p=0.35$. Es decir:
        \begin{equation*}
            X \sim B(5,0.35)
        \end{equation*}

        En este caso, como nos piden el número esperado de plantas contaminadas, tenemos que calcular la esperanza:
        \begin{equation*}
            E[X] = n \cdot p = 5 \cdot 0.35 = 1.75
        \end{equation*}

        \item Calcular la probabilidad de encontrar entre 2 y 5 plantas contaminadas en 9 exámenes.
        
        Sea $Y$ la variable aleatoria que representa el número de plantas contaminadas en $9$ análisis.
        De igual forma que en el apartado anterior, sigue una distribución de probabilidad binomial con $n=9$ y $p=0.35$. Es decir:
        \begin{equation*}
            Y \sim B(9,0.35)
        \end{equation*}

        En este caso, nos piden calcular la probabilidad de encontrar entre $2$ y $5$ plantas contaminadas. Es decir:
        \begin{equation*}
            P[2 \leq Y \leq 5] = F_Y(5) - F_Y(1) = P[Y\leq 5] - P[Y\leq 1] = \AstIg 0.9464 - 0.1211 = 0.8253
        \end{equation*}
        donde en $(\ast)$ hemos utilizado la tabla de la distribución binomial.

        \item Hallar la probabilidad de encontrar 4 plantas no contaminadas en 6 análisis.
        
        Sea $Z$ la variable aleatoria que representa el número de plantas contaminadas en $6$ análisis.
        De igual forma que en los apartados anteriores, sigue una distribución de probabilidad binomial con $n=6$ y $p=0.35$. Es decir:
        \begin{equation*}
            Z \sim B(6,0.35)
        \end{equation*}

        En este caso, nos piden calcular la probabilidad de encontrar $4$ plantas no contaminadas. Es decir, la probabilidad de que $2$ plantas estén contaminadas. Es decir:
        \begin{equation*}
            P[Z=2] = \binom{6}{2} \cdot 0.35^2 \cdot 0.65^4 = 0.328
        \end{equation*}
    \end{enumerate}
    
\end{ejercicio}

\begin{ejercicio}
    Cada vez que una máquina dedicada a la fabricación de comprimidos produce uno, la probabilidad
    de que sea defectuoso es $0.01$.
    \begin{enumerate}
        \item Si los comprimidos se colocan en tubos de $25$, ¿cuál es la probabilidad de que en un tubo todos
        los comprimidos sean buenos?

        \item Si los tubos se colocan en cajas de $10$, ¿cuál es la probabilidad de que en una determinada caja
        haya exactamente $5$ tubos con un comprimido defectuoso?
    \end{enumerate}
\end{ejercicio}

\begin{ejercicio}
    Un pescador desea capturar un ejemplar de sardina que se encuentra siempre en una determinada
    zona del mar con probabilidad $0.15$. Hallar la probabilidad de que tenga que pescar $10$ peces de
    especies distintas de la deseada antes de:
    \begin{enumerate}
        \item Pescar la sardina buscada.
        
        Sea $X$ la variable aleatoria que representa el número de peces de especies distintas de la deseada que el pescador tiene que pescar antes de pescar la sardina buscada. Como la probabilidad de que la sardina buscada se encuentre en la zona es $0.15$, tenemos que sigue una distribución de probabilidad geométrica con $p=0.15$. Es decir:
        \begin{equation*}
            X \sim G(0.15)
        \end{equation*}

        En este caso, nos piden calcular la probabilidad de que tenga que pescar $10$ peces de especies distintas de la deseada antes de pescar la sardina buscada. Es decir:
        \begin{align*}
            P[X=10] &= P[X\leq 10] - P[X\leq 9] = (1-(1-0.15)^{11}) - (1-(1-0.15)^{10}) =\\
            &= (1-0.15)^{10} - (1-0.15)^{11} = 0.85^{10}\cdot (1-0.85) \approx 0.029
        \end{align*}
        
        \item Pescar tres ejemplares de la sardina buscada.
        
        Sea $Y$ la variable aleatoria que representa el número de peces de especies distintas de la deseada que el pescador tiene que pescar antes de pescar tres ejemplares de la sardina buscada. En este caso, tenemos que sigue una distribución de probabilidad negativa binomial con $k=3$ y $p=0.15$. Es decir:
        \begin{equation*}
            Y \sim BN(3,0.15)
        \end{equation*}

        En este caso, nos piden calcular la probabilidad de que tenga que pescar $10$ peces de especies distintas de la deseada antes de pescar tres ejemplares de la sardina buscada. Es decir:
        \begin{align*}
            P[Y=10] &= \dfrac{(10+3-1)!}{10!(3-1)!} \cdot (1-0.15)^{10} \cdot 0.15^3 = \dfrac{12!}{10!2!} \cdot 0.85^{10} \cdot 0.15^3 =\\
            &= \dfrac{12\cdot 11}{2} \cdot 0.85^{10} \cdot 0.15^3 \approx 0.0438
        \end{align*}
    \end{enumerate}
\end{ejercicio}

\begin{ejercicio}
    Un científico necesita $5$ monos afectados por cierta enfermedad para realizar un experimento. La
    incidencia de la enfermedad en la población de monos es siempre del $30\%$. El científico examinará
    uno a uno los monos de un gran colectivo, hasta encontrar $5$ afectados por la enfermedad.
    \begin{enumerate}
        \item Calcular el número medio de exámenes requeridos.
        \item Calcular la probabilidad de que encuentre $10$ monos sanos antes de encontrar los $5$ afectados.
        \item Calcular la probabilidad de que tenga que examinar por lo menos $20$ monos.
    \end{enumerate}
\end{ejercicio}

\begin{ejercicio}
    Se capturan $100$ peces de un estanque que contiene $10000$. Se les marca con una anilla y se
    devuelven al agua. Transcurridos unos días se capturan de nuevo $100$ peces y se cuentan los
    anillados.
    \begin{enumerate}
        \item Calcular la probabilidad de que en la segunda captura se encuentre al menos un pez anillado.
        
        Sea $X$ la variable aleatoria que representa el número de peces anillados en la segunda captura. En este caso, tenemos que sigue una distribución de probabilidad hipergeométrica con $N=10000$, $N_1=100$ y $n=100$. Es decir:
        \begin{equation*}
            X \sim H(10000,100,100)
        \end{equation*}
        
        La probabilidad de que en la segunda captura se encuentre al menos un pez anillado es:
        \begin{align*}
            P[X\geq 1] &= 1 - P[X=0] = 1 - \dfrac{\binom{100}{0} \cdot \binom{9900}{100}}{\binom{10000}{100}} =\\
            &= 1- \dfrac{1 \cdot \dfrac{9900!}{100!9800!}}{\dfrac{10000!}{100!9900!}} = 1 - \dfrac{9900! \cdot 100! \cdot 9900!}{10000! \cdot 100! \cdot 9800!}
        \end{align*}

        Como podemos ver, calcular dicha probabilidad de esta forma es complicado debido a la cantidad de factoriales que hay que calcular. Por ello, aproximaremos la distribución hipergeométrica a una binomial, tomando $n=100$ y $p=\dfrac{N_1}{N} = \dfrac{100}{10000} = 0.01$. Es decir:
        \begin{equation*}
            X \sim B(100,0.01)
        \end{equation*}

        Por lo que la probabilidad de que en la segunda captura se encuentre al menos un pez anillado es:
        \begin{equation*}
            P[X\geq 1] = 1 - P[X=0] = 1 - \binom{100}{0} \cdot 0.01^0 \cdot 0.99^{100} \approx 1-0.366 = 0.634
        \end{equation*}

        \item Calcular el número esperado de peces anillados en la segunda captura.
        
        El número esperado de peces anillados en la segunda captura es:
        \begin{equation*}
            E[X] = n \cdot \dfrac{N_1}{N} = 100 \cdot \dfrac{100}{10000} = 1
        \end{equation*}

        Usando la aproximaxión a la binomial, tenemos que:
        \begin{equation*}
            E[X] = n \cdot p = 100 \cdot 0.01 = 1
        \end{equation*}

        Efectivamente vemos que el resultado en ambos casos coincide.
    \end{enumerate}
\end{ejercicio}

\begin{ejercicio}
    Cada página impresa de un libro tiene $40$ líneas, y cada línea tiene $75$ posiciones de impresión. Se
    supone que la probabilidad de que en cada posición haya error es $1/6000$.
    \begin{enumerate}
        \item ¿Cuál es la distribución del número de errores por página?
        \item Calcular la probabilidad de que una página no contenga errores y de que contenga como mínimo
        $5$ errores.
        \item ¿Cuál es la probabilidad de que un capítulo de $20$ páginas no contenga errores?
    \end{enumerate}
\end{ejercicio}

\begin{ejercicio}
    En un departamento de control de calidad se inspeccionan las unidades terminadas que provienen
    de una línea de ensamble. La probabilidad de que cada unidad sea defectuosa es $0.05$.
    \begin{enumerate}
        \item ¿Cuál es la probabilidad de que la vigésima unidad inspeccionada sea la segunda que se encuentra defectuosa?
        
        Sea $X$ la variable aleatoria que representa el número de unidades inspeccionadas hasta encontrar la segunda defectuosa. En este caso, tenemos que sigue una distribución de probabilidad binomial negativa con $k=2$ y $p=0.05$. Es decir:
        \begin{equation*}
            X \sim BN(2,0.05)
        \end{equation*}

        Como buscamos la probabilidad de que la vigésima unidad inspeccionada sea la segunda que se encuentra defectuosa, hemos de calcular la probabilidad de
        encontrar $18$ unidades no defectuosas antes de encontrar la segunda defectuosa. Es decir:
        \begin{align*}
            P[X=18] &= \dfrac{(18+2-1)!}{18!(2-1)!} \cdot 0.05^2 \cdot 0.95^{18} = \dfrac{19!}{18!1!} \cdot 0.05^2 \cdot 0.95^{18} =\\
            &= 19 \cdot 0.05^2 \cdot 0.95^{18} \approx 0.0188
        \end{align*}
        \item ¿Cuántas unidades deben inspeccionarse por término medio hasta encontrar cuatro defectuosas?
        
        Sea $Y$ la variable aleatoria que representa el número de unidades inspeccionadas hasta encontrar cuatro defectuosas. En este caso, tenemos que sigue una distribución de probabilidad binomial negativa con $k=4$ y $p=0.05$. Es decir:
        \begin{equation*}
            Y \sim BN(4,0.05)
        \end{equation*}

        En este caso, nos piden calcular el número medio de unidades que deben inspeccionarse hasta encontrar cuatro defectuosas. Es decir:
        \begin{equation*}
            E[Y] = \dfrac{k(1-p)}{p} = \dfrac{4\cdot 0.95}{0.05} = 76
        \end{equation*}
        \item Calcular la desviación típica del número de unidades inspeccionadas hasta encontrar cuatro
        defectuosas.

        Tenemos que:
        \begin{equation*}
            \sigma_Y^2 = \dfrac{k(1-p)}{p^2} = \dfrac{4\cdot 0.95}{0.05^2} = 1520
        \end{equation*}

        Por lo que:
        \begin{equation*}
            \sigma_Y = \sqrt{1520} \approx 38.98
        \end{equation*}
    \end{enumerate}
\end{ejercicio}


\begin{ejercicio}
    Los números $1,2,3,...,10$ se escriben en diez tarjetas y se colocan en una urna. Las tarjetas se
    extraen una a una y sin devolución. Calcular las probabilidades de los siguientes sucesos:
    \begin{enumerate}
        \item Hay exactamente tres números pares en cinco extracciones.
        
        Sea $X$ la variable aleatoria que representa el número de números pares en cinco extracciones. En este caso, tenemos que sigue una distribución de probabilidad hipergeométrica con $N=10$, $N_1=5$ y $n=5$. Es decir:
        \begin{equation*}
            X \sim H(10,5,5)
        \end{equation*}

        La probabilidad de que haya exactamente tres números pares en cinco extracciones es:
        \begin{align*}
            P[X=3] &= \dfrac{\binom{5}{3} \cdot \binom{5}{2}}{\binom{10}{5}} = \dfrac{\dfrac{5!}{3!2!} \cdot \dfrac{5!}{2!3!}}{\dfrac{10!}{5!5!}} = \dfrac{(5!)^4}{10!(3!)^2(2!)^2} = \dfrac{5^4\cdot 4^4 \cdot 3^4 \cdot 2^4}{10!\cdot 3^2\cdot 2^4} =\\
            &= \dfrac{5^4\cdot 3^4 \cdot 2^{12}}{5^2\cdot 3^6\cdot 2^{12}\cdot 7} = \dfrac{5^2}{7\cdot 3^2} = \dfrac{25}{63} \approx 0.3968
        \end{align*}
        \item Se necesitan cinco extracciones para obtener tres números pares.
        
        \item Obtener el número $7$ en la cuarta extracción.
    \end{enumerate}



    % // TODO: Este
\end{ejercicio}

\begin{ejercicio}
    Supongamos que el número de televisores vendidos en un comercio durante un mes se distribuye
    según una Poisson de parámetro $10$, y que el beneficio neto por unidad es $30$ euros.
    \begin{enumerate}
        \item ¿Cuál es la probabilidad de que el beneficio neto obtenido por un comerciante durante un mes
        sea al menos de $360$ euros?

        Sea $X$ la variable aleatoria que representa el número de televisores vendidos en un mes. En este caso, tenemos que sigue una distribución de probabilidad de Poisson con $\lambda=10$. Es decir:
        \begin{equation*}
            X \sim \cc{P}(10)
        \end{equation*}

        Sabemos que el beneficio neto por unidad es de $30$ euros, por lo que para obtener un beneficio neto de al menos $360$ euros, el comerciante debe vender al menos $12$ televisores. Por lo que la probabilidad de que el beneficio neto obtenido por un comerciante durante un mes sea al menos de $360$ euros es:
        \begin{align*}
            P[X\geq 12] &= 1 - P[X\leq 11] \AstIg 1-0.6968 = 0.3032
        \end{align*}
        donde en $(\ast)$ hemos utilizado la tabla de la distribución de Poisson.

        \item ¿Cuántos televisores debe tener el comerciante a principio de mes para tener al menos probabilidad $0.95$ de satisfacer toda la demanda?
        
        Se pide el menor valor de $\wh{x}\in \bb{N}$ tal que:
        \begin{equation*}
            P[X\leq \wh{x}] \geq 0.95
        \end{equation*}

        Para resolverlo, buscamos el valor en la tabla de la distribución de Poisson que cumpla la condición. En este caso, el valor que cumple la condición es $15$. Por tanto, el comerciante debe tener al menos $15$ televisores a principio de mes para tener al menos probabilidad $0.95$ de satisfacer toda la demanda.
    \end{enumerate}

    % // TODO: Este
\end{ejercicio}

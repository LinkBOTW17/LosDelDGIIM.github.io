\section{Distribuciones de Probabilidad Continua}

% // TODO: 5,6,9,15
% // El 16 es la Beta

\begin{ejercicio}
    La llegada de viajeros a una estación de tren se distribuye uniformemente en el tiempo. Cada 20 minutos se produce la salida del tren. Hallar:
    \begin{enumerate}
        \item La función de distribución de la variable aleatoria tiempo de espera, su media y su varianza.
        \item La probabilidad de que un viajero espere al tren menos de 7 minutos.
    \end{enumerate}
\end{ejercicio}

\begin{ejercicio}
    La temperatura media diaria en una región se distribuye según una normal con media 25 grados centígrados y desviación típica $10$ grados centígrados.
    \begin{enumerate}
        \item Calcular la probabilidad de que en un día elegido al azar la temperatura media esté comprendida entre $20$ y $32$ grados centígrados.
        \item Calcular la probabilidad de que en un día elegido al azar la temperatura media difiera de la media de las temperaturas medias diarias más de 5 grados centígrados.
    \end{enumerate}
\end{ejercicio}

\begin{ejercicio}
    De una variable aleatoria uniformemente distribuida se conoce su esperanza, $\mu$, y su desviación típica, $\sigma$. Hallar el rango de valores de la variable, en función de $\mu$ y $\sigma$.
\end{ejercicio}

\begin{ejercicio}
    Los precios de venta de un artículo se distribuyen según una ley normal. Se sabe que el $20\%$ son superiores a $1000$ euros y que el $30\%$ no superan los $800$ euros. Hallar la ganancia media y su desviación típica, si las ganancias ($Y$) están relacionadas con los precios ($X$) según la expresión $Y = 350+0.15X$.
\end{ejercicio}

\begin{ejercicio}
    Se clasifican los cráneos en dolicocéfalos (si el índice cefálico, anchura/longitud, es menor que $75$), mesocéfalos (si el índice está entre $75$ y $80$), y braquicéfalos (si el índice es superior a $80$). Suponiendo que la distribución de los índices es normal, hallar la media y la desviación típica en una población en la que el $65\%$ de los individuos son dolicocéfalos, el $30\%$ mesocéfalos y el $5\%$ braquicéfalos.
\end{ejercicio}

\begin{ejercicio}
    La probabilidad de contagio por unidad de tiempo viene dada por:
    \begin{equation*}
        P[T \leq 1] = 1-\exp(-\lambda), \qquad \lambda = 5
    \end{equation*}
    Calcular:
    \begin{enumerate}
        \item El número medio de nuevas infecciones, sobre la población de susceptibles, cuyo tamaño observado es de $50$ individuos, e indicar la distribución aleatoria de la variable que contabiliza las nuevas infecciones en dicha población.
        \item Calcular la probabilidad de que se produzcan $10$ contagios en un intervalo de tiempo de longitud $10$ unidades temporales. Determinar la distribución de probabilidad de dicha variable aleatoria, así como el número medio de contagios en dicho intervalo temporal.
        \item Calcular la probabilidad de que no se produzcan contagios en un intervalo de longitud $20$ unidades temporales, así como
        el tiempo medio transcurrido entre contagios.
    \end{enumerate}
\end{ejercicio}

\begin{ejercicio}
    La probabilidad de que un individuo sufra reacción al inyectarle un determinado suero es $0.1$. Usando la aproximación normal adecuada, calcular la probabilidad de que al inyectar el suero a una muestra de $400$ personas, sufran reacción entre $33$ y $50$.\\

    Sea $X$ la variable aleatoria que representa el número de personas que sufren reacción al inyectarles el suero. Entonces, $X\sim B(400,0.1)$.
    Como $n=400>30$ y $p\geq 0.1$, podemos aproximar la distribución binomial a una normal de media $np = 400\cdot 0.1 = 40$ y varianza $np(1-p) = 400\cdot 0.1\cdot 0.9 = 36$. Por tanto, $X\sim N(40,36)$.
    Tenemos que:
    \begin{align*}
        P[33\leq X\leq 50] &= P\left[X\leq 50\right] - P\left[X\leq 32\right]
    \end{align*}

    Usando la corrección por continuidad de la aproximación normal a la binomial, tenemos que:
    \begin{align*}
        P\left[X\leq 50\right] &= P\left[X\leq 50.5\right]
        P\left[X\leq 32\right] &= P\left[X\leq 32.5\right]
    \end{align*}

    Sea $Z$ la variable aleatoria $X$ tipificada, es decir, $Z = \dfrac{X-40}{6}$. Entonces, $Z\sim N(0,1)$, y tenemos que:
    \begin{align*}
        P\left[X\leq 50.5\right] &= P\left[Z\leq \dfrac{50.5-40}{6}\right] = P\left[Z\leq 1.75\right] \approx 0.95994\\
        P\left[X\leq 32.5\right] &= P\left[Z\leq \dfrac{32.5-40}{6}\right] = P\left[Z\leq -1.25\right]
        = 1-P\left[Z\leq 1.25\right] \approx \\&\qquad \approx 1-0.89435 = 0.10565
    \end{align*}

    Por tanto, $P[33\leq X\leq 50] \approx 0.95994-0.10565 = 0.85429$.
\end{ejercicio}

\begin{ejercicio}
    El tiempo de duración de una pieza de un cierto equipo, medido en horas, se distribuye según una ley exponencial de parámetro $0.2$. 
    Si el equipo deja de funcionar cuando fallan $3$ piezas, determinar:
    \begin{enumerate}
        \item Probabilidad de que el equipo funcione más de $10$ horas.
        
        Sea $X$ la variable aleatoria que representa el tiempo de duración de una pieza del equipo. Entonces, $X\sim \cc{E}(3,0.2)$.
        Queremos calcular entonces $P[X>10]$. Tenemos que:
        \begin{align*}
            P[X>10] &= 1-P[X\leq 10] = 1-\int_{0}^{10} \dfrac{0.2^3}{\Gamma(3)} x^{3-1} e^{-0.2x} \, dx =&\\&= 1-\dfrac{0.2^3}{\Gamma(3)} \int_{0}^{10}x^2 e^{-0.2x} \, dx
            = ... = 1-\frac{0.2^3}{2}\cdot 80.8262 = 0.6767
        \end{align*}
        % // TODO: Completar cálculos
        \item Probabilidad de que el equipo funcione entre $10$ y $15$ horas.
        
        Queremos calcular entonces $P[10\leq X\leq 15]$. Tenemos que:
        \begin{align*}
            P[10\leq X\leq 15] &= P[X\leq 15] - P[X\leq 10] =\\&= ... = 0.2228
        \end{align*}
    \end{enumerate}
\end{ejercicio}

\begin{ejercicio}
    El número de piezas defectuosas diarias en un proceso de fabricación se distribuye según una Poisson. Sabiendo que el número medio de piezas defectuosas diarias es $25$, calcular mediante la aproximación normal:
    \begin{enumerate}
        \item Probabilidad de que el número de defectuosas durante un día oscile entre $24$ y $28$.
        \item Número máximo de defectuosas que con probabilidad $0.97725$ se fabrican al día.
        \item Número mínimo de defectuosas que con probabilidad $0.15866$ se fabrican al día.
    \end{enumerate}
\end{ejercicio}

\begin{ejercicio}
    Un grupo de investigadores ha determinado que el $3\%$ de los individuos afectados por cierto virus fallece. Determinar:
    \begin{enumerate}
        \item La probabilidad de que en una población de $10000$ afectados fallezcan más de $100$.
        \item El número esperado de fallecidos en dicha población.
    \end{enumerate}
\end{ejercicio}

\begin{ejercicio}
    La experiencia ha demostrado que las calificaciones obtenidas en un test de aptitud por los alumnos de un determinado centro siguen una distribución normal de media $400$ y desviación típica $100$. Si se realiza el test a un determinado grupo de alumnos, calcular:
    \begin{enumerate}
        \item El porcentaje de alumnos que obtendrán calificaciones comprendidas entre $300$ y $500$.
        \item La probabilidad de que, elegido un alumno al azar, su calificación difiera de la media en $150$ puntos como máximo.
    \end{enumerate}
\end{ejercicio}

\begin{ejercicio}
    En un parking público se ha observado que los coches llegan, aleatoria e independientemente, a razón de $360$ coches por hora.
    \begin{enumerate}
        \item Utilizando la distribución exponencial, encontrar la probabilidad de que una vez que llega un coche, el próximo no llegue antes de medio minuto.
        
        Sea $X$ la variable aleatoria que representa el tiempo que transcurre entre la llegada de dos coches consecutivos (en minutos). Entonces, como los coches llegan a razón de $360$ coches por hora, tenemos que $\lambda = \nicefrac{360}{6} = 6$ coches por minuto. Por tanto, $X\sim \exp(6)$.
        Queremos calcular entonces $P[X>0.5]$. Tenemos que:
        \begin{equation*}
            P[X>0.5] = 1-P[X\leq 0.5] = 1-(1-e^{-6\cdot 0.5}) = e^{-3} \approx 0.04978
        \end{equation*}

        \item Utilizando la distribución de Poisson, obtener la misma probabilidad anterior.
        
        Sea $Y$ la variable aleatoria que representa el número de coches que llegan en un intervalo de tiempo de longitud $0.5$ minutos. Entonces, $Y\sim \cc{P}\left(\lm\cdot 0.5\right)$, donde $\lm = 6$ coches por minuto. Por tanto, $Y\sim \cc{P}(3)$.
        Queremos calcular entonces $P[Y=0]$. Tenemos que:
        \begin{equation*}
            P[Y=0] = \dfrac{e^{-3}\cdot 3^0}{0!} = e^{-3}
        \end{equation*}

        Por tanto, la probabilidad de que una vez que llega un coche, el próximo no llegue antes de medio minuto es $e^{-3}$, que coincide con el resultado obtenido en el apartado anterior.
    \end{enumerate}
\end{ejercicio}

\begin{ejercicio}
    Cierta enfermedad puede ser producida por tres tipos de virus: $A$, $B$ y $C$. En un laboratorio se tienen tres tubos con el virus $A$, dos tubos con el virus $B$ y cinco con el virus $C$. La probabilidad de que el virus $A$ produzca la enfermedad es $P(|X| < 4)$, siendo $X \sim N(3,25)$. La probabilidad de que el virus $B$ produzca la enfermedad es $P(Y \geq 3)$, siendo $Y \sim B(5,0.7)$. Por último, la probabilidad de que el virus $C$ produzca la enfermedad es $P(Z \leq 5)$, siendo $Z \sim P(4)$. Se elige un tubo al azar y al inocular el virus a un animal, contrae la enfermedad. Hallar la probabilidad de que el virus inoculado sea del tipo $C$.
\end{ejercicio}

\begin{ejercicio}
    Una máquina fabrica tornillos cuyas longitudes se distribuyen según una ley normal con media $20$ mm y desviación típica $0.25$ mm. Un tornillo se considera defectuoso si su longitud no está comprendida entre $19.5$ y $20.5$ mm. Los tornillos se fabrican de forma independiente.
    \begin{enumerate}
        \item Cuál es la probabilidad de fabricar un tornillo defectuoso?
        \item Calcular la probabilidad de que en $10$ tornillos fabricados no haya más de dos defectuosos.
        \item Cuántos tornillos se fabricarán por término medio hasta obtener el primero defectuoso?
    \end{enumerate}
\end{ejercicio}

\begin{ejercicio}
    Si la proporción de personas que consumen una determinada marca de aceite de oliva sigue una distribución beta de parámetros $2$ y $3$, determinar la probabilidad de que dicha proporción esté comprendida entre el $0.1$ y $0.5$.
\end{ejercicio}

\begin{ejercicio}
    La proporción diaria de piezas defectuosas en determinada fábrica tiene distribución beta, y el segundo parámetro es $4$. Sabiendo que la proporción media diaria es $0.2$, calcular la probabilidad de que un día resulte una proporción de defectuosas superior a la media.
\end{ejercicio}

\begin{ejercicio}
    El Instituto de Estadística de una determinada comunidad autónoma convoca unas pruebas selectivas para cubrir vacantes. La puntuación obtenida por cada candidato se calcula mediante el promedio de las calificaciones obtenidas en las pruebas realizadas, y se sabe, de experiencias previas, que dichas puntuaciones tienen media $100$, se distribuyen de forma normal y que el $44.04\%$ de los aspirantes que realizan la prueba supera la puntuación $100.6$.
    \begin{enumerate}
        \item La convocatoria de las pruebas establece una nota mínima de $105$ puntos para superar la oposición. ¿Qué porcentaje de opositores consiguen una plaza?
        
        Sea $X$ la variable aleatoria que representa la puntuación obtenida por un candidato. Entonces, $X\sim \cc{N}(100,\sigma^2)$. Sabemos que $P[X>100.6] = 0.4404$. Por tanto, $P[X\leq 100.6] = 1-0.4404 = 0.5596$.
        Sea $Z$ la variable aleatoria $X$ tipificada, es decir, $Z = \dfrac{X-100}{\sigma}$. Entonces, $Z\sim \cc{N}(0,1)$, y tenemos que:
        \begin{equation*}
            P\left[Z\leq \dfrac{100.6-100}{\sigma}\right] = 0.5596 = P\left[Z\leq \dfrac{0.6}{\sigma}\right]
        \end{equation*}

        Usando la tabla de la distribución normal estándar, buscamos el valor de $z$ tal que $P[Z\leq z] = 0.5596$. Dicho valor, tras consultar la tabla, es $z = 0.15$. Por tanto,
        \begin{equation*}
            \dfrac{0.6}{\sigma} = z = 0.15 \Longrightarrow \sigma = 4 \Longrightarrow \sigma^2 = 16
        \end{equation*}

        Tenemos por tanto que $X\sim \cc{N}(100,16)$. Ahora, queremos calcular el valor de $P[X>105]$. Tenemos que:
        \begin{align*}
            P[X>105] &= 1-P[X\leq 105] = 1-P\left[Z\leq \dfrac{105-100}{4}\right] = 1-P\left[Z\leq 1.25\right] \approx\\&\approx 1-0.89435 = 0.10565
        \end{align*}



        \item No obstante, se sabe que en ocasiones el tribunal decide, dependiendo de las necesidades de personal, rebajar las condiciones para que un candidato sea admitido. ¿Cuál sería la nota mínima necesaria para que la probabilidad de superar la prueba de selección sea $0.33$?
        
        Sea $a$ la nota mínima necesaria para que la probabilidad de superar la prueba de selección sea $0.33$; es decir, el valor buscado. Buscamos $a$ tal que $P[X>a] = 0.33$. Tenemos que:
        \begin{multline*}
            P[X>a] = 1-P[X\leq a] = 1-P\left[Z\leq \dfrac{a-100}{4}\right] = 0.33 \Longrightarrow \\ \Longrightarrow
            P\left[Z\leq \dfrac{a-100}{4}\right] = 0.67
        \end{multline*}

        Usando la tabla de la distribución normal estándar, buscamos el valor de $z$ tal que $P[Z\leq z] = 0.67$. Dicho valor, tras consultar la tabla, es $z = 0.44$. Por tanto,
        \begin{equation*}
            \dfrac{a-100}{4} = z = 0.44 \Longrightarrow a = 100+4\cdot 0.44 = 101.76
        \end{equation*}

        Por tanto, la nota mínima necesaria para que la probabilidad de superar la prueba de selección sea $0.33$ es $101.76$.

        \item El instituto decide crear una bolsa de interinos para cubrir temporalmente posibles eventualidades. A esa bolsa pertenecerán todos los candidatos cuyas puntuaciones estén entre la media de las puntuaciones y la nota establecida en el apartado anterior. ¿Qué porcentaje de candidatos estarán en dicha situación?
        
        En este caso, nos piden que calculemos $P[100\leq X\leq 101.76]$. Tenemos que:
        \begin{align*}
            P[100\leq X\leq 101.76] &= P\left[0\leq Z \leq \dfrac{101.76-100}{4}\right] = P\left[0\leq Z \leq 0.44\right] =\\&= P\left[Z\leq 0.44\right] - P\left[Z\leq 0\right] \AstIg 0.67-0.5 = 0.17
        \end{align*}
        donde en $(\ast)$ hemos usado que $P[Z\leq 0] = 0.5$ por ser esta la distribución normal estándar y $P[Z\leq 0.44] = 0.67$ tras consultar la tabla de la distribución normal estándar.
    \end{enumerate}

\end{ejercicio}
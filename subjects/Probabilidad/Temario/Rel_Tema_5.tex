\section{Algunos Modelos Multivariantes}

\begin{ejercicio}
    El 60\% de los clientes de un almacén paga con dinero, el 30\% con tarjeta y el 10\% con cheques. Calcular la probabilidad de que de 10 clientes, 5 paguen con dinero, 2 con tarjeta y 3 con cheques.\\

    Dado $i\in \{\text{dinero},\text{tarjeta},\text{cheques}\}$, definimos $X_i$ como la variable aleatoria que cuenta el número de clientes que pagan con el método $i$. Entonces, $(X_{\text{dinero}},X_{\text{tarjeta}},X_{\text{cheques}})$ sigue una distribución multinomial con parámetros $n = 10$ y probabilidades $(p_{\text{dinero}},p_{\text{tarjeta}},p_{\text{cheques}})$, donde:
    \begin{gather*}
        p_{\text{dinero}} = 0.6,\qquad 
        p_{\text{tarjeta}} = 0.3,\qquad
        p_{\text{cheques}} = 0.1.
    \end{gather*}

    Tenemos entonces que:
    \begin{align*}
        P[X_{\text{dinero}} &= 5,X_{\text{tarjeta}} = 2,X_{\text{cheques}} = 3] = \dfrac{10!}{5!2!3!}\cdot p_{\text{dinero}}^5 \cdot p_{\text{tarjeta}}^2 \cdot p_{\text{cheques}}^3
        =\\&= \dfrac{10!}{5!2!3!} \cdot 0.6^5 \cdot 0.3^2 \cdot 0.1^3
        \approx 0.0176
    \end{align*}
\end{ejercicio}

\begin{ejercicio}
    En un hotel hay tres salas de televisión. En un determinado instante, cada televisor puede sintonizar uno entre 6 canales distintos, A, B, C, D, E y F. Cada canal tiene probabilidad $\nicefrac{1}{36}, \nicefrac{3}{36}, \nicefrac{5}{36}, \nicefrac{7}{36}, \nicefrac{9}{36}$ y $\nicefrac{11}{36}$, respectivamente, de ser sintonizado, con independencia unas televisiones de otras. Calcular:
    \begin{enumerate}
        \item Probabilidad de que en un instante dado se sintonicen los canales B, D y E.
        
        Dado $i\in \{A,B,C,D,E,F\}$, definimos $X_i$ como la variable aleatoria que cuenta el número de televisores sintonizando el canal $i$. Entonces, $(X_A,X_B,X_C,X_D,X_E)$ sigue una distribución multinomial con parámetros $n = 3$ y probabilidades $(p_A,p_B,p_C,p_D,p_E)$, donde
        \begin{gather*}
            p_A = \dfrac{1}{36},\qquad 
            p_B = \dfrac{3}{36},\\
            p_C = \dfrac{5}{36},\qquad 
            p_D = \dfrac{7}{36},\qquad 
            p_E = \dfrac{9}{36}.
        \end{gather*}

        Consideramos ahora el vector aleatorio $(X_B,X_D,X_E)$, que sabemos que:
        \begin{equation*}
            (X_B,X_D,X_E) \sim M_3\left(3,p_B,p_D,p_E\right).
        \end{equation*}

        Tenemos que:
        \begin{equation*}
            P[X_B = 1,X_D = 1,X_E = 1] = \dfrac{3!}{1!1!1!0!}\cdot p_B \cdot p_D \cdot p_E = 6\cdot \dfrac{3}{36}\cdot \dfrac{7}{36}\cdot \dfrac{9}{36} = \dfrac{7}{288} \approx 0.0243
        \end{equation*}

        \item Probabilidad de que en un instante dado haya un televisor sintonizando el canal B y otro el E.
        
        Consideramos ahora el vector aleatorio $(X_B,X_E)$, que sabemos que:
        \begin{equation*}
            (X_B,X_E) \sim M_2\left(3,p_B,p_E\right).
        \end{equation*}

        Tenemos que:
        \begin{equation*}
            P[X_B = 1,X_E = 1] = \dfrac{3!}{1!1!1!}\cdot p_B \cdot p_E \cdot \left(1-p_B-p_E\right) = 6\cdot \dfrac{3}{36}\cdot \dfrac{9}{36}\cdot \dfrac{24}{36} = \dfrac{1}{12} \approx 0.0833
        \end{equation*}
        \item Probabilidad de que en un instante dado los tres televisores sintonicen el canal F.
        
        Consideramos ahora el vector aleatorio $(X_F)$, que sabemos que:
        \begin{equation*}
            (X_F) \sim M_1\left(3,p_F\right).
        \end{equation*}

        Tenemos que:
        \begin{equation*}
            P[X_F = 3] = \dfrac{3!}{3!0!}\cdot p_F^3 = \left(\dfrac{11}{36}\right)^3 = \dfrac{1331}{46656} \approx 0.0285
        \end{equation*}

        \item Probabilidad de que en un instante dado no estén sintonizados A, B, C, D.
        
        Consideramos ahora el vector aleatorio $(X_A,X_B,X_C,X_D)$, que sabemos que:
        \begin{equation*}
            (X_A,X_B,X_C,X_D) \sim M_4\left(3,p_A,p_B,p_C,p_D\right).
        \end{equation*}

        Tenemos que:
        \begin{align*}
            P[X_A = 0,&X_B = 0,X_C = 0,X_D = 0] = \dfrac{3!}{0!0!0!0!3!}\cdot \left(1-p_A-p_B-p_C-p_D\right)^3 =\\&= 1\cdot \left(1-\dfrac{1}{36}-\dfrac{3}{36}-\dfrac{5}{36}-\dfrac{7}{36}\right)^3 = \left(\dfrac{5}{9}\right)^3 = \dfrac{125}{729} \approx 0.1715
        \end{align*}
    \end{enumerate}
\end{ejercicio}

\begin{ejercicio}
    Un ordenador genera números aleatorios del 0 al 9 con independencia e igual probabilidad para cada dígito. Si se generan 12 números aleatorios, calcular:
    \begin{enumerate}
        \item La probabilidad de que aparezca 6 veces el dígito 0, 4 veces el 1 y 2 veces el 2.
        
        Dado $i\in \{0,1,2,\ldots,8\}$, definimos $X_i$ como la variable aleatoria que cuenta el número de veces que aparece el dígito $i$. Entonces, $(X_0,X_1,X_2,\ldots,X_8)$ sigue una distribución multinomial con parámetros $n = 12$ y probabilidades $(p_0,p_1,p_2,\ldots,p_8)$, donde:
        \begin{gather*}
            p := p_0 = p_1 = p_2 = \ldots = p_8 = \dfrac{1}{10}.
        \end{gather*}

        Consideramos ahora el vector aleatorio $(X_0,X_1,X_2)$, que sabemos que:
        \begin{equation*}
            (X_0,X_1,X_2) \sim M_3\left(12,p_0,p_1,p_2\right).
        \end{equation*}

        Tenemos que:
        \begin{equation*}
            P[X_0 = 6,X_1 = 4,X_2 = 2] = \dfrac{12!}{6!4!2!0!}\cdot p_0^6 \cdot p_1^4 \cdot p_2^2
            = \dfrac{12!}{6!4!2!0!} p^{12}
            \approx 13.86\cdot 10^{-9}
        \end{equation*}
        \item El número esperado de veces que aparece el 0.
        
        Tenemos que $X_0 \sim M_1\left(12,p_0\right)$, por lo que:
        \begin{equation*}
            E[X_0] = n\cdot p_0 = 12\cdot \dfrac{1}{10} = \frac{6}{5} = 1.2
        \end{equation*}
        \item La probabilidad de que aparezca 4 veces el 1 y 3 veces el 6.
        
        Consideramos ahora el vector aleatorio $(X_1,X_6)$, que sabemos que:
        \begin{equation*}
            (X_1,X_6) \sim M_2\left(12,p_1,p_6\right).
        \end{equation*}

        Tenemos que:
        \begin{equation*}
            P[X_1 = 4,X_6 = 3] = \dfrac{12!}{4!3!5!}\cdot p_1^4 \cdot p_6^3 \cdot (1-p_1-p_6)^5
            = \dfrac{12!}{4!3!5!} p^7 (1-2p)^5
            \approx 9.08\cdot 10^{-4}
        \end{equation*}
    \end{enumerate}
\end{ejercicio}

\begin{ejercicio}
    Sea $F : \mathbb{R} \to \mathbb{R}$ una función de distribución continua, y sean $\alpha_1$ y $\alpha_2$ números reales tales que $F(\alpha_1) = 0.3$ y $F(\alpha_2) = 0.8$. Si se seleccionan al azar 25 observaciones independientes de la distribución cuya función de distribución es $F$. Calcular la probabilidad de que seis de los valores
    observados sean menores que $\alpha_1$, diez de los valores observados estén entre $\alpha_1$ y $\alpha_2$ y 9 sean mayores que $\alpha_2$.\\

    Como $F$ es una función de distribución, es creciente. Como $F(\alpha_1) = 0.3$ y $F(\alpha_2) = 0.8$, tenemos que $\alpha_1\leq \alpha_2$. Definimos las siguientes variables aleatorias:
    \begin{align*}
        X_1 &:= \text{Número de observaciones menores que }\alpha_1,\\
        X_2 &:= \text{Número de observaciones entre }\alpha_1\text{ y }\alpha_2,\\
        X_3 &:= \text{Número de observaciones mayores que }\alpha_2.
    \end{align*}
    
    Tenemos las siguientes probabilidades:
    \begin{align*}
        p_1 &= \text{Prob. de que una observación sea menor que }\alpha_1 = F(\alpha_1) = 0.3,\\
        p_2 &= \text{Prob. de que una observación esté entre }\alpha_1\text{ y }\alpha_2 = F(\alpha_2) - F(\alpha_1) = 0.8 - 0.3 = 0.5,\\
        p_3 &= \text{Prob. de que una observación sea mayor que }\alpha_2 = 1 - F(\alpha_2) = 0.2 = 1 - p_1 - p_2.
    \end{align*}

    Tenemos que:
    \begin{equation*}
        (X_1,X_2) \sim M_2\left(25,p_1,p_2\right).
    \end{equation*}

    Por lo que la probabilidad pedida es:
    \begin{equation*}
        P[X_1 = 6,X_2 = 10] = \dfrac{25!}{6!10!9!}\cdot p_1^6 \cdot p_2^{10} \cdot p_3^9
        = \dfrac{25!}{6!10!9!} \cdot 0.3^6 \cdot 0.5^{10} \cdot 0.2^9
        \approx 0.0059
    \end{equation*}
\end{ejercicio}

\begin{ejercicio}
    Si se lanzan 5 dados equilibrados de forma independiente, calcular la probabilidad de que los números 1 y 4 aparezcan el mismo número de veces.\\

    Dado $i\in \{1,2,3,4,5,6\}$, definimos $X_i$ como la variable aleatoria que cuenta el número de veces que aparece el número $i$. Entonces, $(X_1,X_2,X_3,X_4,X_5)$ sigue una distribución multinomial con parámetros $n = 5$ y probabilidades $(p_1,p_2,p_3,p_4,p_5)$, donde:
    \begin{gather*}
        p := p_1 = p_2 = p_3 = p_4 = p_5 = \dfrac{1}{6}.
    \end{gather*}

    Consideramos ahora el vector aleatorio $(X_1,X_4)$, que sabemos que:
    \begin{equation*}
        (X_1,X_4) \sim M_2\left(5,p_1,p_4\right).
    \end{equation*}

    Tenemos que la probabilidad pedida es:
    \begin{equation*}
        P[X_1=X_2]=\sum_{i=0}^2 P[X_1 = i,X_4 = i]
    \end{equation*}

    Calculamos cada uno de los sumandos:
    \begin{align*}
        P[X_1 = 0,X_4 = 0] &= \dfrac{5!}{0!0!5!}\cdot p_1^0 \cdot p_4^0 \cdot (1-p_1-p_4)^5 = \dfrac{5!}{5!} \cdot \left(\dfrac{4}{6}\right)^5 = \dfrac{32}{243},\\
        P[X_1 = 1,X_4 = 1] &= \dfrac{5!}{1!1!3!}\cdot p_1^1 \cdot p_4^1 \cdot (1-p_1-p_4)^3 = 5\cdot 4\cdot \left(\dfrac{1}{6}\right)\cdot \left(\dfrac{1}{6}\right)\cdot \left(\dfrac{4}{6}\right)^3 = \dfrac{40}{243},\\
        P[X_1 = 2,X_4 = 2] &= \dfrac{5!}{2!2!1!}\cdot p_1^2 \cdot p_4^2 \cdot (1-p_1-p_4)^1 = \frac{5!}{2!2!} \cdot \left(\dfrac{1}{6}\right)^2 \cdot \left(\dfrac{1}{6}\right)^2 \cdot \dfrac{4}{6} = \dfrac{5}{324}.
    \end{align*}

    Por lo que la probabilidad pedida es:
    \begin{equation*}
        P[X_1=X_2]=\dfrac{32}{243} + \dfrac{40}{243} + \dfrac{5}{324} = \dfrac{101}{324} \approx 0.312
    \end{equation*}
\end{ejercicio}

\begin{ejercicio}
    En un determinado juego de azar existen tres posibles resultados A, B y C, que se dan con probabilidad $0.8$,$0.15$ y $0.05$, respectivamente. Una persona realiza 5 veces el juego de forma independiente, calcular la probabilidad de que no obtenga ninguna vez el resultado C ni más de una vez el resultado B.\\

    Dado $i\in \{A,B,C\}$, sea $p_i$ la probabilidad de obtener el resultado $i$. Definimos la variable aleatoria $X_i$ como el número de veces que se obtiene el resultado $i$. Se pide:
    \begin{equation*}
        P[X_B \leq 1,X_C = 0] = P[X_B = 0,X_C = 0] + P[X_B = 1,X_C = 0]
    \end{equation*}

    Para calcular cada una de las probabilidades, consideramos el vector aleatorio $(X_B,X_C)$, que sabemos que:
    \begin{equation*}
        (X_B,X_C) \sim M_2\left(5,p_B,p_C\right).
    \end{equation*}

    Tenemos que:
    \begin{align*}
        P[X_B = 0,X_C = 0] &= \dfrac{5!}{0!0!5!}\cdot p_B^0 \cdot p_C^0 \cdot (1-p_B-p_C)^5 = \dfrac{5!}{5!} \cdot 0.8^5 = \frac{1024}{3125},\\
        P[X_B = 1,X_C = 0] &= \dfrac{5!}{1!0!4!}\cdot p_B^1 \cdot p_C^0 \cdot (1-p_B-p_C)^4 = 5 \cdot 0.15 \cdot 0.8^4= \frac{192}{625}.
    \end{align*}

    Por lo que la probabilidad pedida es:
    \begin{equation*}
        P[X_B \leq 1,X_C = 0] = \frac{1024}{3125} + \frac{192}{625} = \frac{1984}{3125} \approx 0.63488
    \end{equation*}
\end{ejercicio}

\begin{ejercicio}
    Sea $(X,Y)$ un vector aleatorio con distribución normal con parámetros $\mu_1 = 5$, $\mu_2 = 8$, $\sigma_1^2 = 16$, $\sigma_2^2 = 9$, $\rho = 0.6$. Calcular
    \[
        P[5 < Y < 11 \mid X = 2].
    \]

    Calculemos en primer lugar la distribución condicional de $Y$ dado $X = 2$. Sabemos que:
    \begin{equation*}
        Y \mid X = 2 \sim \cc{N}\left(\mu_2 + \rho\dfrac{\sigma_2}{\sigma_1}(2-\mu_1),\sigma_2^2(1-\rho^2)\right)
        = \cc{N}\left(6.65, 5.76\right).
    \end{equation*}

    Por tanto, y notando por $Z$ a la tipificada de $Y\mid X=2$, tenemos que:
    \begin{align*}
        P[5 < Y < 11 \mid X = 2] &= P[-0.6875 < Z < 1.8125] = P[Z < 1.8125] - P[Z < -0.6875] =\\
        &= P[Z < 1.8125] - 1+P[Z > -0.6875]
        =\\&= P[Z < 1.8125] - 1+P[Z < 0.6875] \approx 0.96485-1+0.75490 = 0.71975.
    \end{align*}
\end{ejercicio}

\begin{ejercicio}
    Calcular la distribución de probabilidad de la suma $X + Y$ para $(X,Y) \sim \cc{N}_2((\mu_1,\mu_2),\Sigma)$, siendo
    \[
        \Sigma = \begin{pmatrix}
            \sigma_1^2 & \sigma_1\sigma_2\rho \\
            \sigma_1\sigma_2\rho & \sigma_2^2
        \end{pmatrix}.
    \]
    
    Sea la matriz $A$ dada por:
    \begin{equation*}
        A=\begin{pmatrix}
            1\\
            1
        \end{pmatrix}
    \end{equation*}

    Sea $Z=X+Y=(X,Y)A$. Entonces, usando la normalidad para las combinaciones lineales de normales, tenemos:
    \begin{equation*}
        X+Y\sim \cc{N}(\mu A,A^t\Sigma A)
    \end{equation*}

    Tenemos que:
    \begin{align*}
        \mu A &= \mu_1+\mu_2\\
        A^t\Sigma A &= \begin{pmatrix}
            \sigma_1^2 + \sigma_1\sigma_2\rho & \sigma_2^2 + \sigma_1\sigma_2\rho\\
        \end{pmatrix}
        \begin{pmatrix}
            1\\
            1
        \end{pmatrix}
        = \sigma_1^2 + \sigma_2^2 + 2\sigma_1\sigma_2\rho
    \end{align*}

    Por tanto, $X+Y\sim \cc{N}(\mu_1+\mu_2,\sigma_1^2 + \sigma_2^2 + 2\sigma_1\sigma_2\rho)$.
\end{ejercicio}

\begin{ejercicio}
    Si $(X,Y)$ es un vector aleatorio con distribución normal, encontrar una condición necesaria y suficiente para que $X + Y$ y $X - Y$ sean independientes.\\

    Tenemos que $(X,Y)\sim \cc{N}_2(\mu,\Sigma)$ para cierto vector de esperanzas $\mu$ y matriz de covarianzas $\Sigma$.
    Sea la siguiente matriz:
    \begin{equation*}
        A=\begin{pmatrix}
            1 & 1\\
            1 & -1
        \end{pmatrix}
    \end{equation*}

    De esta forma, $Z=(X,Y)A=(X+Y,X-Y)$, por lo que usando la normalidad para las combinaciones lineales de normales, tenemos que:
    \begin{equation*}
        Z\sim \cc{N}_2\left(\mu A, A^t\Sigma A\right).
    \end{equation*}

    Calculamos la matriz de covarianzas de $Z$:
    \begin{align*}
        A^t\Sigma A &= \begin{pmatrix}
            1 & 1\\
            1 & -1
        \end{pmatrix}
        \begin{pmatrix}
            \sigma_1^2 & \sigma_1\sigma_2\rho \\
            \sigma_1\sigma_2\rho & \sigma_2^2
        \end{pmatrix}
        \begin{pmatrix}
            1 & 1\\
            1 & -1
        \end{pmatrix}
        =\\&= \begin{pmatrix}
            \sigma_1^2 + \sigma_1\sigma_2\rho & \sigma_1\sigma_2\rho + \sigma_2^2\\
            \sigma_1^2 - \sigma_1\sigma_2\rho & \sigma_1\sigma_2\rho - \sigma_2^2
        \end{pmatrix}
        \begin{pmatrix}
            1 & 1\\
            1 & -1
        \end{pmatrix}
        =\\&= \begin{pmatrix}
            \sigma_1^2 + 2\sigma_1\sigma_2\rho + \sigma_2^2 & \sigma_1^2 - \sigma_2^2\\
            \sigma_1^2 - \sigma_2^2 & \sigma_1^2 - 2\sigma_1\sigma_2\rho + \sigma_2^2
        \end{pmatrix}
    \end{align*}

    Por tanto, tenemos que:
    \begin{equation*}
        \Cov[X+Y,X-Y] = \sigma_1^2 - \sigma_2^2.
    \end{equation*}

    Al ser $Z=(X+Y,X-Y)$ una normal bidimensional, por la caracterización de la independencia de las componentes de normales bidimensionales, tenemos que:
    \begin{equation*}
        X+Y\text{ y }X-Y\text{ son independientes} \Leftrightarrow \Cov[X+Y,X-Y] = 0 \Leftrightarrow \sigma_1^2 = \sigma_2^2.
    \end{equation*}
\end{ejercicio}

\begin{ejercicio}
    Para cada una de las siguientes densidades normales bidimensionales, hallar $\mu_1,\mu_2,\sigma_1^2,\sigma_2^2$ y $\rho$:
    \begin{enumerate}
        \item $f_{(X,Y)}(x, y) = \dfrac{1}{2\pi}\exp\left(-\dfrac{1}{2}((x-1)^2 + (y-2)^2)\right)$
        
        Buscamos expresar la función de densidad de la forma:
        \begin{equation*}
            f_{(X,Y)}(x, y) = \dfrac{1}{2\pi\sqrt{|\Sigma|}}\exp\left(-\dfrac{(x-\mu_1,y-\mu_2)\Sigma^{-1}(x-\mu_1,y-\mu_2)^t}{2}\right)
        \end{equation*}

        Tenemos que:
        \begin{align*}
            f_{(X,Y)}(x, y) &= \dfrac{1}{2\pi}\exp\left(-\dfrac{1}{2}((x-1)^2 + (y-2)^2)\right)\\
            &= \frac{1}{2\pi}\exp\left(-\dfrac{1}{2}
            \begin{pmatrix}
                x-1&
                y-2
            \end{pmatrix}
            \begin{pmatrix}
                1 & 0\\
                0 & 1
            \end{pmatrix}
            \begin{pmatrix}
                x-1\\
                y-2
            \end{pmatrix}
            \right)
        \end{align*}

        Por tanto, tomando $\Sigma=(Id_2)^{-1}=Id_2$, tenemos que $|\Sigma|=1$ y cuadra. Por tanto,
        \begin{equation*}
            (X,Y)\sim \cc{N}_2((1,2), Id_2)
        \end{equation*}

        Por tanto:
        \begin{equation*}
            \mu_1 = 1,\quad \mu_2 = 2,\quad \sigma_1^2 = 1,\quad \sigma_2^2 = 1,\quad \rho = 0.
        \end{equation*}

        \item $f_{(X,Y)}(x, y) = \dfrac{1}{2\pi}\exp\left(-\dfrac{1}{2}(x^2 +y^2 +4x-6y+13)\right)$
        
        Tenemos que:
        \begin{align*}
            f_{(X,Y)}(x, y) &= \dfrac{1}{2\pi}\exp\left(-\dfrac{1}{2}(x^2 +y^2 +4x-6y+13)\right)\\
            &= \dfrac{1}{2\pi}\exp\left(-\dfrac{1}{2}((x+2)^2 + (y-3)^2)\right)\\
            &= \frac{1}{2\pi}\exp\left(-\dfrac{1}{2}
            \begin{pmatrix}
                x+2&
                y-3
            \end{pmatrix}
            \begin{pmatrix}
                1 & 0\\
                0 & 1
            \end{pmatrix}
            \begin{pmatrix}
                x+2\\
                y-3
            \end{pmatrix}
            \right)
        \end{align*}

        Por tanto, tomando $\Sigma=(Id_2)^{-1}=Id_2$, tenemos que $|\Sigma|=1$ y cuadra. Por tanto,
        \begin{equation*}
            (X,Y)\sim \cc{N}_2((-2,3), Id_2)
        \end{equation*}

        Por tanto:
        \begin{equation*}
            \mu_1 = -2,\quad \mu_2 = 3,\quad \sigma_1^2 = 1,\quad \sigma_2^2 = 1,\quad \rho = 0.
        \end{equation*}
        \item $f_{(X,Y)}(x, y) = \dfrac{1}{2.4\pi}\exp\left(-\dfrac{1}{0.72}\left(\dfrac{x^2}{4} -0.8xy+y^2\right)\right)$
        
        Este caso es más complejo, porque $\Sigma\neq Id_2$. Buscamos ahora expresar la función de densidad de la forma:
        \begin{equation*}
            \hspace{-3cm}
            f_X(x,y)=\dfrac{1}{2\pi \sigma_1\sigma_2\sqrt{1-\rho^2}}\exp{\left(-\frac{1}{2(1-\rho^2)}\left[\left(\frac{x-\mu_1}{\sigma_1}\right)^2-2\rho\left(\frac{x-\mu_1}{\sigma_1}\right)\left(\frac{y-\mu_2}{\sigma_2}\right)+\left(\frac{y-\mu_2}{\sigma_2}\right)^2\right]\right)}
        \end{equation*}

        Identificando términos, en primer lugar tenemos que:
        \begin{equation*}
            \begin{cases}
                2.4\pi=2\pi\sigma_1\sigma_2\sqrt{1-\rho^2}\\
                0.72=2(1-\rho^2)
            \end{cases}
        \end{equation*}

        Por tanto, $1-\rho^2=0.36$, de lo que obtenemos que:
        \begin{align*}
            \sqrt{1-\rho^2} &= 0.6 \Longrightarrow \sigma_1\sigma_2 = \dfrac{2.4\pi}{2\pi\cdot 0.6} = 2,\\
            1-\rho^2 &= 0.36 \Longrightarrow \rho^2 = 0.64 \Longrightarrow |\rho|= 0.8
        \end{align*}
        El signo de $\rho$ no podemos determinarlo (ya que tampoco se puede determinar el signo de $\sigma_1$ y $\sigma_2$).


        Por tanto, considerando los siguientes valores, tenemos que cuadraría:
        \begin{equation*}
            \mu_1=\mu_2=0,\quad \sigma_1^2=4,\quad \sigma_2^2=1
        \end{equation*}

    \end{enumerate}
\end{ejercicio}

\begin{ejercicio}
    Sea $(X,Y)$ un vector aleatorio con función de densidad en $\mathbb{R}^2$ dada por:
    \begin{align*}
        f_{(X,Y)}(x, y) &= f_1(x, y) + f_2(x, y),\\
        f_1(x, y) &= \dfrac{1}{4\pi\sqrt{1-\rho^2}}\exp\left(-\dfrac{1}{2(1-\rho^2)}(x^2 +y^2 -2\rho xy)\right),\quad |\rho| < 1,\\
        f_2(x, y) &= \dfrac{1}{4\pi\sqrt{1-\tau^2}}\exp\left(-\dfrac{1}{2(1-\tau^2)}(x^2 +y^2 -2\tau xy)\right),\quad |\tau| < 1.
    \end{align*}
    
    \begin{enumerate}
        \item Demostrar que las distribuciones marginales son normales de media cero y varianza uno.
        
        Tenemos que:
        \begin{equation*}
            f_1(x,y)=\frac{1}{2\cdot }g_1(x,y),\qquad
            f_2(x,y)=\frac{1}{2}\cdot g_2(x,y)
        \end{equation*}
        donde:
        \begin{itemize}
            \item $g_1(x,y)$ es la función de densidad de $(X_1,Y_1)\sim \cc{N}_2(0,0,1,1,\rho)$.
            \item $g_2(x,y)$ es la función de densidad de $(X_2,Y_2)\sim \cc{N}_2(0,0,1,1,\tau)$.
        \end{itemize}

        Calculemos la función de densidad marginal de $X$:
        \begin{align*}
            f_X(x) &= \int_{-\infty}^\infty f_{(X,Y)}(x,y)\,dy = \int_{-\infty}^\infty f_1(x,y) + f_2(x,y)\,dy =\\
            &= \int_{-\infty}^\infty f_1(x,y)\,dy + \int_{-\infty}^\infty f_2(x,y)\,dy
            =\\&= \frac{1}{2}\left(\int_{-\infty}^\infty g_1(x,y)\,dy + \int_{-\infty}^\infty g_2(x,y)\,dy\right)
            =\\&= \frac{1}{2}\left(f_{X_1}(x) + f_{X_2}(x)\right)
        \end{align*}
        donde, en la útlima igualdad, hemos usado que esas integrales son la función de densidad de las marginales de $X_1$ y $X_2$ respectivamente. Como sabemos que $X_1,X_2\sim \cc{N}(0,1)$, tenemos que $f_{X_1}=f_{X_2}$, por lo que:
        \begin{equation*}
            f_X(x) = \frac{1}{2}\left(2f_{X_1}(x)\right) = f_{X_1}(x) = \frac{1}{\sqrt{2\pi}}\exp\left(-\frac{x^2}{2}\right)
        \end{equation*}

        Por tanto, tenemos que:
        \begin{equation*}
            X\sim \cc{N}(0,1)
        \end{equation*}

        Análogamnete, demostramos que $Y\sim \cc{N}(0,1)$.

        \item Obtener la covarianza de $X$ e $Y$.
        
        Sabemos que $E[X]=E[Y]=0$. Calculemos $E[XY]$:
        \begin{align*}
            E[XY] &= \int_{-\infty}^\infty\int_{-\infty}^\infty xyf_{(X,Y)}(x,y)\,dx\,dy = \int_{-\infty}^\infty\int_{-\infty}^\infty xy(f_1(x,y)+f_2(x,y))\,dx\,dy =\\
            &= \int_{-\infty}^\infty\int_{-\infty}^\infty xyf_1(x,y)\,dx\,dy + \int_{-\infty}^\infty\int_{-\infty}^\infty xyf_2(x,y)\,dx\,dy
            =\\&= \frac{1}{2}\left(\int_{-\infty}^\infty\int_{-\infty}^\infty xyg_1(x,y)\,dx\,dy + \int_{-\infty}^\infty\int_{-\infty}^\infty xyg_2(x,y)\,dx\,dy\right)
            =\\&= \frac{1}{2}\left(E[X_1Y_1] + E[X_2Y_2]\right)
        \end{align*}

        Como $E[X_1]=E[Y_1]=E[X_2]=E[Y_2]=0$, tenemos que:
        \begin{align*}
            E[XY]
            &= \frac{1}{2}\left(E[X_1Y_1] - E[X_1]E[Y_1] + E[X_2Y_2] - E[X_2]E[Y_2]\right)
            =\\&= \frac{1}{2}\left(\Cov[X_1,Y_1] + \Cov[X_2,Y_2]\right)
        \end{align*}

        Usando que $(X_1,Y_1)\sim \cc{N}_2(0,0,1,1,\rho)$ y $(X_2,Y_2)\sim \cc{N}_2(0,0,1,1,\tau)$, tenemos que:
        \begin{align*}
            \Cov[X_1,Y_1] &= \rho,\\
            \Cov[X_2,Y_2] &= \tau
        \end{align*}

        Por tanto, tenemos que:
        \begin{equation*}
            \Cov[X,Y] = \frac{1}{2}(\rho + \tau)
        \end{equation*}
        \item ¿Son $X$ e $Y$ independientes? ¿Son $X$ e $Y$ incorreladas?
        
        Veamos en primer lugar si son independientes, ya que en el caso de que lo sean sabemos que serán incorreladas.
        Tenemos que $X,Y\sim \cc{N}(0,1)$, por lo que:
        \begin{align*}
            f_X(x)f_Y(y) &= \frac{1}{\sqrt{2\pi}}\exp\left(-\frac{x^2}{2}\right)\cdot \frac{1}{\sqrt{2\pi}}\exp\left(-\frac{y^2}{2}\right) = \frac{1}{2\pi}\exp\left(-\frac{x^2+y^2}{2}\right)\\
            f_{(X,Y)}(x,y) &= \frac{1}{2}\left(g_1(x,y)+g_2(x,y)\right)
            =\\&= \frac{1}{2}\left(\frac{1}{2\pi\sqrt{1-\rho^2}}\exp\left(-\dfrac{1}{2(1-\rho^2)}(x^2 +y^2 -2\rho xy)\right)+\right.\\&\hspace{1cm}\left.+\frac{1}{2\pi\sqrt{1-\tau^2}}\exp\left(-\dfrac{1}{2(1-\tau^2)}(x^2 +y^2 -2\tau xy)\right)\right)
        \end{align*}

        Evaluando por ejemplo en el $(0,0)$, tenemos que:
        \begin{align*}
            f_X(0)f_Y(0) &= \frac{1}{2\pi}\\
            f_{(X,Y)}(0,0) &= \frac{1}{2}\left(\frac{1}{2\pi\sqrt{1-\rho^2}}+\frac{1}{2\pi\sqrt{1-\tau^2}}\right)
            = \frac{1}{2\pi}\left(\frac{1}{2\sqrt{1-\rho^2}}+\frac{1}{2\sqrt{1-\tau^2}}\right)
        \end{align*}

        Por tanto, tenemos que:
        \begin{align*}
            f_X(0)f_Y(0) = f_{(X,Y)}(0,0) &\Longleftrightarrow
            1 = \frac{1}{2\sqrt{1-\rho^2}}+\frac{1}{2\sqrt{1-\tau^2}}\\
            & \Longleftrightarrow 2 = \frac{1}{\sqrt{1-\rho^2}}+\frac{1}{\sqrt{1-\tau^2}}
        \end{align*}

        En este caso, como $\rho^2,\tau^2>0$, tenemos que ambos radicandos son menores o iguales que uno, por lo que ambos denomininadores son menores o iguales que uno. Por tanto, ambos sumandos son mayores o iguales que uno.
        Para que su suma sea $2$, es necesario que ambos sumandos sean iguales a uno, y esto solo se tiene si $\rho=\tau=0$. Por tanto, tenemos que:
        \begin{equation*}
            f_X(0)f_Y(0) = f_{(X,Y)}(0,0) \Longleftrightarrow \rho=\tau=0
        \end{equation*}

        Por tanto, en caso contrario, tenemos que $X$ e $Y$ no son independientes. En el caso de que $\rho=\tau=0$, tenemos que:
        \begin{align*}
            f_{(X,Y)}(x,y) &= \frac{1}{2}\left(\frac{1}{2\pi}\exp\left(-\dfrac{1}{2}(x^2 +y^2)\right)+\frac{1}{2\pi}\exp\left(-\dfrac{1}{2}(x^2 +y^2)\right)\right)
            =\\&= \frac{1}{2\pi}\exp\left(-\dfrac{x^2 +y^2}{2}\right)
            = f_X(x)f_Y(y)\qquad \forall (x,y)\in \mathbb{R}^2
        \end{align*}

        Por tanto, concluimos que:
        \begin{equation*}
            X\text{ e }Y\text{ son independientes} \Longleftrightarrow \rho=\tau=0
        \end{equation*}

        Respecto a su correlación, como $\Cov[X,Y]=\frac{1}{2}(\rho+\tau)$, tenemos que:
        \begin{equation*}
            X\text{ e }Y\text{ son incorreladas} \Longleftrightarrow \Cov[X,Y]=0 \Longleftrightarrow \rho=-\tau
        \end{equation*}

        De nuevo, notemos que si son independientes, entonces son incorreladas, pero el recíproco no es cierto.
    \end{enumerate}
\end{ejercicio}

\begin{ejercicio}
    Sea $(X,Y)$ un vector aleatorio con distribución normal con parámetros $\mu_1,\mu_2,\sigma_1^2,\sigma_2^2$ y $\rho$. Demostrar que $X$ e $Y$ son independientes $\Leftrightarrow \rho = 0$.\\

    Demostrado en la Proposición~\ref{prop:independencia_normal_bidimensional}.
\end{ejercicio}



\begin{ejercicio}
    Sea $(X,Y)$ un vector aleatorio con distribución normal bidimensional con las siguientes características:
    \begin{itemize}
        \item La mediana de $X$ es 1.
        \item $\Var(X) = \Var(Y)$.
        \item $\rho_{X,Y} = 0.5$.
        \item $\Cov(X,Y) = \nicefrac{2}{3}$.
        \item $P[X\leq -1\mid Y = 1] = 0.06681$.
    \end{itemize}
    Determinar el vector de medias y la matriz de covarianzas de $(X,Y)$.\\

    Tenemos que:
    \begin{equation*}
        0.5 = \rho_{X,Y} = \dfrac{\Cov(X,Y)}{\sqrt{\Var(X)\Var(Y)}} \AstIg \dfrac{\nicefrac{2}{3}}{\Var(X)}\Longrightarrow \Var(X) = \Var(Y)=\frac{2}{3}\cdot 2 = \frac{4}{3}
    \end{equation*}
    donde en $(\ast)$ he empleado que $\Var(X) = \Var(Y)$. Por tanto, la matriz de covarianzas es:
    \begin{equation*}
        \Sigma = \begin{pmatrix}
            \Var(X) & \Cov(X,Y)\\
            \Cov(X,Y) & \Var(Y)
        \end{pmatrix} = \frac{1}{3}\begin{pmatrix}
            4 & 2\\
            2 & 4
        \end{pmatrix}
    \end{equation*}

    Como $(X,Y)$ es una normal bidimensional, $X$ es una normal, luego su mediana es igual a su media. Por tanto, $E[X]=1$.
    Para calcular $E[Y]$, usamos el único resultado que no hemos empleado, la distribución $X\mid Y=1$. Como tenemos que $(X,Y)\sim \cc{N}_2(1,E[Y],\nicefrac{4}{3},\nicefrac{4}{3}, \nicefrac{1}{2})$, entonces:
    \begin{align*}
        (X\mid Y=1)&\sim \cc{N}\left(\mu_1+\rho\dfrac{\sigma_1}{\sigma_2}(1-\mu_2),\sigma_1^2(1-\rho^2)\right)
        = \cc{N}\left(1+\dfrac{1}{2}\cdot 1(1-E[Y]),\dfrac{4}{3}\cdot \dfrac{3}{4}\right)
        =\\&= \cc{N}\left(1.5-\dfrac{E[Y]}{2},1\right)
    \end{align*}

    Tipificando, y usando lo anterior, tenemos que:
    \begin{multline*}
        0.06681 = P[X\leq -1\mid Y=1] = P\left[Z\leq \dfrac{-1-1.5+\nicefrac{E[Y]}{2}}{1}\right] = P\left[Z\leq -2.5+\frac{E[Y]}{2}\right]
        \\\Longrightarrow
        1-0.06681 = 0.93319 = P\left[Z>-2.5+\frac{E[Y]}{2}\right] = P\left[Z\leq 2.5-\frac{E[Y]}{2}\right]
    \end{multline*}

    Consultando la tabla de la normal estándar, como $P[Z\leq 1.5]=0.93319$, tenemos que:
    \begin{equation*}
        2.5-\frac{E[Y]}{2} = 1.5 \Longrightarrow E[Y] = 2
    \end{equation*}

    Por tanto, el vector de medias es:
    \begin{equation*}
        \mu = \begin{pmatrix}
            1&
            2
        \end{pmatrix}
    \end{equation*}
\end{ejercicio}
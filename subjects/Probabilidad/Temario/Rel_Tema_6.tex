\section{Teorema Central del Límite}

\begin{ejercicio}
    Se realizan 2500 lanzamientos independientes de una moneda correcta. Calcular la probabilidad aproximada de obtener $\nicefrac{1}{2}$ como frecuencia relativa de cara con un error máximo de $0.02$.\\

    De aquí en adelante, fijemos $n=2500$, aunque lo haremos dependiente de $n$ para una mayor generalidad.
    Para cada $i=1, \ldots, n$, sea $X_i$ una variable aleatoria que toma el valor $1$ si en el $i$-ésimo lanzamiento se obtiene cara, y $0$ en caso contrario, de forma que $X_i \sim B(1, \nicefrac{1}{2})$. Consideramos la variable aleatoria:
    \[
        S_{n} = \sum_{i=1}^{n} X_i
    \]

    De esta forma, por la Reproductividad de la Binomial, $S_{n} \sim B(n, \nicefrac{1}{2})$. Tiene sentido, puesto que $S_{n}$ es el número de caras obtenidas en $n$ lanzamientos de una moneda correcta.
    Nos piden:
    \begin{align*}
        P\left[\left|\frac{S_{n}}{n} - \frac{1}{2}\right| \leq 0.02\right] &= 
        P\left[-0.02 \leq \frac{S_{n}}{n} - \frac{1}{2} \leq 0.02\right]
        =\\&= P\left[0.48 \leq \frac{S_{n}}{n} \leq 0.52\right]
    \end{align*}

    Para calcular esta probabilidad, hay diversas opciones:
    \begin{description}
        \item[Usar la distribución de $S_n$] Dado que $S_{n} \sim B(n, \nicefrac{1}{2})$, podemos calcular la probabilidad directamente. Por el Teorema de Cambio de Variable, tenemos que:
        \begin{align*}
            P\left[\left|\frac{S_{n}}{n} - \frac{1}{2}\right| \leq 0.02\right] &= P\left[0.48 \leq \frac{S_{n}}{n} \leq 0.52\right]
            =\\&= P\left[n \cdot 0.48 \leq S_{n} \leq n \cdot 0.52\right]
        \end{align*}

        Esta probabilidad teóricamente sabemos calcularla de forma exacta, aunque en la práctica es complicado para valores grandes de $n$, ya que obtenemos factoriales de números grandes. Probemos para $n=2500$:
        \begin{align*}
            P\left[\left|\frac{S_{2500}}{2500} - \frac{1}{2}\right| \leq 0.02\right] &= P\left[0.48 \leq \frac{S_{2500}}{2500} \leq 0.52\right]
            =\\&= P\left[1200 \leq S_{2500} \leq 1300\right]
            =\\&= \sum_{k=1200}^{1300} P[S_{2500} = k]
            = \sum_{k=1200}^{1300} \binom{2500}{k} \left(\frac{1}{2}\right)^{2500}
        \end{align*}

        Mediante cálculo numérico con herramientas disponibles en internet, tenemos que:
        \begin{equation*}
            P\left[\left|\frac{S_{2500}}{2500} - \frac{1}{2}\right| \leq 0.02\right] \approx 0.95664
        \end{equation*}

        No obstante, usaremos lo desarrollado en teoría para obtener otras aproximaciones.

        \item [Aproximación de la Binomial a la Normal] Dado que $n\geq 30$ y $0.1 \leq p \leq 0.9$, podemos aproximar la distribución de $S_{n}$ por una normal de media $E[S_{n}] = n \cdot p$ y varianza $\Var[S_{n}] = n \cdot p \cdot (1-p)$. Además, debemos realizar la corrección por continuidad, ya que estamos aproximando una variable discreta por una continua. Por tanto, la probabilidad que nos piden la podemos calcular como:
        \begin{align*}
            P\left[\left|\frac{S_{n}}{n} - \frac{1}{2}\right| \leq 0.02\right] &= P\left[n \cdot 0.48 \leq S_{n} \leq n \cdot 0.52\right]
            \approx\\&\approx P\left[n\cdot 0.48 - 0.5 \leq Y \leq n\cdot 0.52 + 0.5\right]
        \end{align*}
        donde $Y \sim N(n \cdot 0.5, n \cdot 0.5 \cdot 0.5)$. Notando por $Z$ a la tipificada de $Y$, tenemos que:
        \begin{align*}
            P\left[\left|\frac{S_{n}}{n} - \frac{1}{2}\right| \leq 0.02\right] &\approx P\left[\frac{n\cdot 0.48 - 0.5-n\cdot 0.5}{\sqrt{n \cdot 0.5 \cdot 0.5}} \leq Z \leq \frac{n\cdot 0.52 + 0.5-0.5\cdot n}{\sqrt{n \cdot 0.5 \cdot 0.5}}\right]
        \end{align*}

        En concreto, para $n=2500$, tenemos que:
        \begin{align*}
            P\left[\left|\frac{S_{2500}}{2500} - \frac{1}{2}\right| \leq 0.02\right] &\approx P\left[-2.02 \leq Z \leq 2.02\right]=P[Z \leq 2.02] - P[Z < -2.02]
            =\\&= P[Z \leq 2.02] -(1-P[Z \geq -2.02])
            =\\&= P[Z \leq 2.02] - (1-P[Z \leq 2.02])
            = 2 \cdot P[Z \leq 2.02] - 1
            \approx \\& \approx 2 \cdot 0.97831 - 1= 0.95662
        \end{align*}

        \item [Teorema de Lévy. Distribución Degenerada.] Dado que $n$ es grande, podemos aplicar el Teorema de Lévy. Tenemos que:
        \begin{align*}
            P\left[\left|\frac{S_{n}}{n} - \frac{1}{2}\right| \leq 0.02\right] &= 
            P\left[0.48 \leq \frac{S_{n}}{n} \leq 0.52\right]
            =\\&= P\left[0.48 - \frac{E[S_{n}]}{n} \leq \frac{S_{n}-E[S_n]}{n} \leq 0.52 - \frac{E[S_{n}]}{n}\right]
            = \\&= P\left[\frac{S_{n}-E[S_n]}{n} \leq 0.52 - \frac{E[S_{n}]}{n}\right] - P\left[\frac{S_{n}-E[S_n]}{n} < 0.48 - \frac{E[S_{n}]}{n}\right]
        \end{align*}

        Por la Convergencia en Ley a la Distribución Degenerada, tenemos que:
        \begin{equation*}
            \left\{\frac{S_{n}-E[S_n]}{n}\right\}\xrightarrow{L} 0
        \end{equation*}
        
        Por tanto, tenemos que:
        \begin{align*}
            P\left[\left|\frac{S_{n}}{n} - \frac{1}{2}\right| \leq 0.02\right] &= 
            P\left[\frac{S_{n}-E[S_n]}{n} \leq 0.52 - \frac{E[S_{n}]}{n}\right] - P\left[\frac{S_{n}-E[S_n]}{n} < 0.48 - \frac{E[S_{n}]}{n}\right]
            \approx \\&\approx P\left[0 \leq 0.52 - \frac{E[S_{n}]}{n}\right] - P\left[0 < 0.48 - \frac{E[S_{n}]}{n}\right]
            = \\&= P\left[0 \leq 0.52 - p\right] - P\left[0 < 0.48 - p\right]
            = 1 - 0 = 1
        \end{align*}

        \item [Teorema de Lévy. Distribución Normal.] Dado que $n$ es grande, podemos aplicar el Teorema de Lévy. Tenemos que:
        \begin{align*}
            P\left[\left|\frac{S_{n}}{n} - \frac{1}{2}\right| \leq 0.02\right] &= 
            P\left[n\cdot 0.48 \leq S_{n} \leq n\cdot 0.52\right]
            =\\&= P\left[\frac{n\cdot 0.48 - E[S_{n}]}{\sqrt{\Var[S_{n}]}} \leq \frac{S_{n}-E[S_n]}{\sqrt{\Var[S_{n}]}} \leq \frac{n\cdot 0.52 - E[S_{n}]}{\sqrt{\Var[S_{n}]}}\right]
        \end{align*}

        Por la Convergencia en Ley a la Normal, tenemos que:
        \begin{equation*}
            \left\{\frac{S_{n}-E[S_n]}{\sqrt{\Var[S_{n}]}}\right\}\xrightarrow{L} \cc{N}(0,1)
        \end{equation*}

        Por tanto, notando por $Z$ a la tipificada de la normal, tenemos que:
        \begin{align*}
            P\left[\left|\frac{S_{n}}{n} - \frac{1}{2}\right| \leq 0.02\right] &= 
            P\left[\frac{n\cdot 0.48 - E[S_{n}]}{\sqrt{\Var[S_{n}]}} \leq \frac{S_{n}-E[S_n]}{\sqrt{\Var[S_{n}]}} \leq \frac{n\cdot 0.52 - E[S_{n}]}{\sqrt{\Var[S_{n}]}}\right]
            \approx \\&\approx P\left[\frac{n\cdot 0.48 - E[S_{n}]}{\sqrt{\Var[S_{n}]}} \leq Z \leq \frac{n\cdot 0.52 - E[S_{n}]}{\sqrt{\Var[S_{n}]}}\right]
        \end{align*}

        Usando ahora que $S_{n} \sim B(n, \nicefrac{1}{2})$ y que $n=2500$, tenemos que:
        \begin{align*}
            P\left[\left|\frac{S_{2500}}{2500} - \frac{1}{2}\right| \leq 0.02\right] &\approx P\left[\frac{2500\cdot 0.48 - 2500\cdot 0.5}{\sqrt{2500 \cdot 0.5 \cdot 0.5}} \leq Z \leq \frac{2500\cdot 0.52 - 2500\cdot 0.5}{\sqrt{2500 \cdot 0.5 \cdot 0.5}}\right]
            =\\&= P\left[-2\leq Z \leq 2\right]
            =\\&= 2\cdot P\left[Z \leq 2\right]-1
            \approx 2\cdot 0.97725 - 1 = 0.9545
        \end{align*}
    \end{description}
    
    Por tanto, hemos obtenido tres aproximaciones distintas. Notemos que la primera y la tercera coinciden a excepción de la corrección por continuidad, mientras que la segunda es ligeramente distinta.

    % // TODO: Utilidad de la segunda?
\end{ejercicio}

\begin{ejercicio}
    Calcular, aproximadamente, la probabilidad de que al lanzar 100 veces un dado, la media de los puntos obtenidos sea mayor que $3.7$.\\

    De aquí en adelante, fijemos $n=100$, aunque lo haremos dependiente de $n$ para una mayor generalidad.

    Para cada $i=1, \ldots, n$, sea $X_i$ una variable aleatoria que toma el valor $1, 2, 3, 4, 5$ o $6$ según el resultado del $i$-ésimo lanzamiento de un dado, de forma que se tiene $X_i \sim \cc{U}(\{1, 2, 3, 4, 5, 6\})$. Consideramos la variable aleatoria:
    \[
        S_{n} = \sum_{i=1}^{n} X_i
    \]

    Por tanto, nos piden:
    \begin{align*}
        P\left[\frac{S_{n}}{n} > 3.7\right] &= 1 - P\left[\frac{S_{n}}{n} \leq 3.7\right]
    \end{align*}

    Calculemos dicha probabilidad. Tenemos que, para $n=100$:
    \begin{align*}
        P\left[\frac{S_{n}}{n} \leq 3.7\right] &= P\left[S_{n} \leq 3.7 \cdot n\right]
        = P\left[S_{100} \leq 370\right]
        = \sum_{k=100}^{370} P[S_{100} = k]
    \end{align*}

    Calcular de forma directa dicha probabilidad no es sencillo, por lo que haremos aproximaciones. Tenemos que:
    \begin{align*}
        E[X_1] &= \frac{1+2+3+4+5+6}{6} = \frac{7}{2} = 3.5\\
        E[X_1^2] &= \frac{1^2+2^2+3^2+4^2+5^2+6^2}{6} = \frac{91}{6}\\
        \Var[X_1] &= E[X_1^2] - E[X_1]^2 = \frac{91}{6} - \frac{49}{4} = \frac{35}{12}\\
        E[S_{n}] &= E\left[\sum_{i=1}^{n} X_i\right] = \sum_{i=1}^{n} E[X_i] = n \cdot E[X_1] = 3.5 \cdot n\\
        \Var[S_{n}] &= \Var\left[\sum_{i=1}^{n} X_i\right] = \sum_{i=1}^{n} \Var[X_i] = n \cdot \Var[X_1] = \frac{35}{12} \cdot n
    \end{align*}

    Por el Teorema de Lévy, tenemos que:
    \begin{align*}
        \left\{\frac{S_{n}-E[S_n]}{\sqrt{\Var[S_{n}]}}\right\} &\xrightarrow{L} \cc{N}(0,1)
    \end{align*}

    Por tanto, sea $Z$ la tipificada de la normal, tenemos que:
    \begin{align*}
        P\left[\frac{S_{n}}{n} > 3.7\right] &= 1 - P\left[\frac{S_{n}}{n} \leq 3.7\right]
        = 1 - P\left[\frac{S_{n}-E[S_n]}{\sqrt{\Var[S_{n}]}} \leq \frac{3.7 \cdot n - E[S_n]}{\sqrt{\Var[S_{n}]}}\right]
        \approx\\&\approx 1 - P\left[Z \leq \frac{3.7 \cdot n - 3.5\cdot n}{\sqrt{\frac{35}{12}\cdot n}}\right]
    \end{align*}

    En concreto, para $n=100$, tenemos que:
    \begin{align*}
        P\left[\frac{S_{100}}{100} > 3.7\right] &\approx 1 - P\left[Z \leq 1.17\right]
        \approx 1 - 0.87900 = 0.121
    \end{align*}
\end{ejercicio}

\begin{ejercicio}
    El número de piezas correctas elaboradas en una fábrica cuadruplica el de piezas defectuosas, y la probabilidad de producir una pieza defectuosa se mantiene constante durante todo el proceso de fabricación. Se eligen al azar $200$ piezas, calcular aproximadamente la probabilidad de que el número de defectuosas oscile entre $40$ y $50$.

    De aquí en adelante, fijemos $n=200$, aunque lo haremos dependiente de $n$ para una mayor generalidad.

    Obtengamos en primer lugar la probabilidad de que una pieza sea defectuosa. Sea $k$ el número de piezas defectuosas, de forma que el número de piezas correctas es $4k$. Por tanto, tenemos que:
    \begin{align*}
        P[\text{defectuosa}] &= \frac{k}{4k+k} = \frac{k}{5k} = \frac{1}{5}
    \end{align*}

    Para cada $i=1, \ldots, n$, sea $X_i$ una variable aleatoria que toma el valor $1$ si la $i$-ésima pieza es defectuosa, y $0$ en caso contrario, de forma que se tiene $X_i \sim B(1, \nicefrac{1}{5})$. Consideramos la variable aleatoria:
    \[
        S_{n} = \sum_{i=1}^{n} X_i
    \]
    Notemos que $S_{n} \sim B(n, \nicefrac{1}{5})$ por la propiedad de reproductividad de la binomial. Además, $S_{n}$ es el número de piezas defectuosas en $n$ piezas elegidas al azar. Por tanto, nos piden:
    \begin{align*}
        P[40 \leq S_{n} \leq 50]
    \end{align*}

    Calcular de forma directa dicha probabilidad no es sencillo, por lo que hemos de aproximarlo, y lo haremos por la Convergencia en Ley a la Normal demostrada en el Teorema de Lévy. Tenemos que:
    \begin{equation*}
        \left\{\frac{S_{n}-E[S_n]}{\sqrt{\Var[S_{n}]}}\right\} \xrightarrow{L} \cc{N}(0,1)
    \end{equation*}

    Por tanto, sea $Z$ la tipificada de la normal, tenemos que:
    \begin{align*}
        P[40 \leq S_{n} \leq 50] &= P\left[\frac{40 - E[S_n]}{\sqrt{\Var[S_{n}]}} \leq \frac{S_{n}-E[S_n]}{\sqrt{\Var[S_{n}]}} \leq \frac{50 - E[S_n]}{\sqrt{\Var[S_{n}]}}\right]
        \approx\\&\approx P\left[\frac{40 - E[S_n]}{\sqrt{\Var[S_{n}]}} \leq Z \leq \frac{50 - E[S_n]}{\sqrt{\Var[S_{n}]}}\right]
    \end{align*}

    Usando que $S_n\sim B(n, \nicefrac{1}{5})$, tenemos que:
    \begin{align*}
        E[S_n] &= n \cdot \frac{1}{5} = \frac{n}{5}\\
        \Var[S_n] &= n \cdot \frac{1}{5} \cdot \frac{4}{5} = \frac{4n}{25}
    \end{align*}

    Por tanto, para $n=200$, tenemos que:
    \begin{align*}
        P[40 \leq S_{200} \leq 50] &\approx P\left[\frac{40 - \frac{200}{5}}{\sqrt{\frac{4\cdot 200}{25}}} \leq Z \leq \frac{50 - \frac{200}{5}}{\sqrt{\frac{4\cdot 200}{25}}}\right]
        =\\&= P\left[0 \leq Z \leq 1.7678\right]
        =\\&= P\left[Z \leq 1.7678\right]-P\left[Z < 0\right]
        =\\&= 0.96164 - 0.5 = 0.46164
    \end{align*}
\end{ejercicio}

\begin{ejercicio}
    Se elige un punto aleatorio $(X_1, \ldots, X_{100})$ en el espacio $\mathbb{R}^{100}$. Suponiendo que las variables aleatorias son independientes e idénticamente distribuidas según una $\cc{U}([-1, 1])$, calcular, aproximadamente, la probabilidad de que el cuadrado de la distancia del punto al origen sea menor que $40$.\\

    De aquí en adelante, fijemos $n=100$, aunque lo haremos dependiente de $n$ para una mayor generalidad.

    Para cada $i=1, \ldots, n$, sea $X_i$ una variable aleatoria que toma valores en $[-1, 1]$, de forma que se tiene $X_i \sim \cc{U}([-1, 1])$. Además, para cada $i=1, \ldots, n$, sea $Y_i = X_i^2$. Consideramos la variable aleatoria que mide el cuadrado de la distancia al origen:
    \[
        S_{n} = \sum_{i=1}^{n} X_i^2 = \sum_{i=1}^{n} Y_i
    \]

    Por tanto, nos piden:
    \begin{align*}
        P[S_{n} < 40]
    \end{align*}

    Calcular de forma directa dicha probabilidad no es sencillo, por lo que hemos de aproximarlo, y lo haremos por la Convergencia en Ley a la Normal demostrada en el Teorema de Lévy. Comprobemos las hipótesis. Como $X_1, \ldots, X_{100}$ son independientes e idénticamente distribuidas, tenemos que $Y_1, \ldots, Y_{100}$ también lo son. La función de densidad de $X_1$ es:
    \[
        f_{X_1}(x) = \begin{cases}
            \frac{1}{1-(-1)} = \nicefrac{1}{2}& x \in \left[-1, 1\right]\\
            0 & \text{en otro caso}
        \end{cases}
    \]
    
    Por tanto, tenemos que:
    \begin{align*}
        E[Y_1] &= E[X_1^2] = \int_{-1}^{1} x^2 \cdot \frac{1}{2} \ dx= \frac{1}{2} \int_{-1}^{1} x^2 \ dx = \frac{1}{2} \left[\frac{x^3}{3}\right]_{-1}^{1} = \frac{1}{2} \left(\frac{1}{3} - \frac{-1}{3}\right) = \frac{1}{3}\\
        E[Y_1^2] &= E[X_1^4] = \int_{-1}^{1} x^4 \cdot \frac{1}{2} \ dx= \frac{1}{2} \int_{-1}^{1} x^4 \ dx = \frac{1}{2} \left[\frac{x^5}{5}\right]_{-1}^{1} = \frac{1}{2} \left(\frac{1}{5} - \frac{-1}{5}\right) = \frac{1}{5}\\
        \Var[Y_1] &= E[Y_1^2] - E[Y_1]^2 = \frac{1}{5} - \frac{1}{9} = \frac{4}{45}\\
        E[S_{n}] &= E\left[\sum_{i=1}^{n} Y_i\right] = \sum_{i=1}^{n} E[Y_i] = n \cdot E[Y_1] = \frac{n}{3}\\
        \Var[S_{n}] &= \Var\left[\sum_{i=1}^{n} Y_i\right] = \sum_{i=1}^{n} \Var[Y_i] = n \cdot \Var[Y_1] = \frac{4n}{45}
    \end{align*}

    Por tanto, por el Teorema de Lévy, tenemos que:
    \begin{align*}
        \left\{\frac{S_{n}-E[S_n]}{\sqrt{\Var[S_{n}]}}\right\} &\xrightarrow{L} \cc{N}(0,1)
    \end{align*}

    Por tanto, sea $Z$ la tipificada de la normal, tenemos que:
    \begin{align*}
        P[S_{n} < 40] &= P\left[\frac{S_{n}}{n} < \frac{40}{n}\right]
        = P\left[\frac{S_{n}-E[S_n]}{\sqrt{\Var[S_{n}]}} < \frac{40 - E[S_n]}{\sqrt{\Var[S_{n}]}}\right]
        \approx P\left[Z < \frac{40 - \frac{n}{3}}{\sqrt{\frac{4n}{45}}}\right]
    \end{align*}

    En concreto, para $n=100$, tenemos que:
    \begin{align*}
        P[S_{100} < 40] &\approx P\left[Z < \frac{20\cdot 3}{3\cdot 4\sqrt{5}}\right]
        = P\left[Z < \sqrt{5}\right]
        \approx P\left[Z < 2.2361\right]
        \approx 0.98745
    \end{align*}
\end{ejercicio}

\begin{ejercicio}
    Cierta enfermedad afecta al $0.5\%$ de una población. Existe una prueba para la detección de la enfermedad, que da positiva en los individuos enfermos con probabilidad $0.99$ y da negativa en los individuos sanos con probabilidad $0.99$.
    \begin{enumerate}
        \item Calcular la probabilidad de que un individuo elegido al azar esté realmente enfermo si la prueba da resultado positivo.
        
        Notemos de la siguiente forma los siguientes sucesos:
        \begin{itemize}
            \item $E$: El individuo está enfermo.
            \item $S$: El individuo está sano.
            \item $+$: La prueba da positiva.
            \item $-$: La prueba da negativa.
        \end{itemize}

        Sabemos que:
        \begin{equation*}
            P[E] = 0.005\quad
            P[S] = 0.995\quad
            P[+\mid E] = 0.99\quad
            P[-\mid S] = 0.99
        \end{equation*}

        Nos piden calcular $P[E\mid +]$. Por la regla de Bayes, tenemos que:
        \begin{align*}
            P[E\mid +] &= \frac{P[+\mid E] \cdot P[E]}{P[+]}
            = \frac{P[+\mid E] \cdot P[E]}{P[+\mid E] \cdot P[E] + P[+\mid S] \cdot P[S]}
            =\\&= \frac{0.99 \cdot 0.005}{0.99 \cdot 0.005 + 0.01 \cdot 0.995}
            = \frac{99}{298}
            \approx 0.3322
        \end{align*}
        \item Calcular, aproximadamente, el número mínimo de personas con resultado positivo en la prueba que deben ser elegidas, de forma aleatoria e independiente, para asegurar una proporción de personas realmente enfermas en la muestra inferior a un $\nicefrac{1}{2}$, con probabilidad mayor o igual que $0.95$.\\
        
        Sea $n$ el número de personas elegidas. Para cada $i=1, \ldots, n$, sea $X_i$ una variable aleatoria que toma el valor $1$ si el $i$-ésimo individuo está enfermo, y $0$ en caso contrario, de forma que se tiene $X_i \sim B(1, P[E\mid +])$. Consideramos la variable aleatoria:
        \[
            S_{n} = \sum_{i=1}^{n} X_i
        \]

        Por la reproducción de la binomial, tenemos que $S_{n} \sim B(n, P[E\mid +])$. Además, $S_{n}$ es el número de personas realmente enfermas en $n$ elegidas al azar con resultado positivo en la prueba. Nos piden:
        \begin{align*}
            P\left[\frac{S_{n}}{n} < \frac{1}{2}\right] &\geq 0.95
        \end{align*}

        Como se pide un resultado aproximado, usaremos el Teorema de Lévy. Tenemos que:
        \begin{align*}
            E[S_{n}] &= n \cdot P[E\mid +]\\
            \Var[S_{n}] &= n \cdot P[E\mid +] \cdot P[S\mid +]
        \end{align*}

        Por tanto, por el Teorema de Lévy, tenemos que:
        \begin{align*}
            \left\{\frac{S_{n}-E[S_n]}{\sqrt{\Var[S_{n}]}}\right\} &\xrightarrow{L} \cc{N}(0,1)
        \end{align*}

        Por tanto, sea $Z$ la tipificada de la normal, tenemos que:
        \begin{align*}
            P\left[\frac{S_{n}}{n} < \frac{1}{2}\right] \geq 0.95
            &\Longleftrightarrow P\left[\frac{S_{n}-E[S_n]}{\sqrt{\Var[S_{n}]}} < \frac{\nicefrac{n}{2} - E[S_n]}{\sqrt{\Var[S_{n}]}}\right] \geq 0.95
            \Longleftrightarrow \\ & \Longleftrightarrow P\left[Z < \frac{\nicefrac{n}{2} - n \cdot P[E\mid +]}{\sqrt{n \cdot P[E\mid +] \cdot P[S\mid +]}}\right] \geq 0.95
        \end{align*}

        Por tanto, consultando en la tabla de la normal, tenemos que:
        \begin{align*}
            \frac{\nicefrac{n}{2} - n \cdot P[E\mid +]}{\sqrt{n \cdot P[E\mid +] \cdot P[S\mid +]}} \geq 1.65
            &\Longleftrightarrow \frac{\nicefrac{n}{2} - n \cdot \nicefrac{99}{298}}{\sqrt{n \cdot \nicefrac{99}{298} \cdot \nicefrac{199}{298}}} \geq 1.65
            \Longleftrightarrow \\&\Longleftrightarrow \sqrt{n}\cdot \frac{\nicefrac{1}{2} - \nicefrac{99}{298}}{\sqrt{\nicefrac{99}{298} \cdot \nicefrac{199}{298}}} \geq 1.65
            \Longleftrightarrow \\&\Longleftrightarrow \sqrt{n} \geq 4.6319
            \Longleftrightarrow n \geq 21.45
        \end{align*}

        Por tanto, como $n$ ha de ser un número entero, el número mínimo de personas con resultado positivo en la prueba que deben ser elegidas es $22$.
    \end{enumerate}
\end{ejercicio}

\begin{ejercicio}
    Una empresa necesita adquirir al menos $100$ vehículos. Para ello realiza una prueba a una población de coches compuesta de dos tipos distintos (un $40\%$ de tipo A y un $60\%$ de tipo B) y un coche es adquirido si supera la prueba. Un coche de tipo A supera la prueba con probabilidad $\nicefrac{1}{3}$, mientras que para uno de tipo B, dicha probabilidad es $\nicefrac{2}{3}$.
    \begin{enumerate}
        \item Calcular la probabilidad de que un coche elegido al azar supere la prueba.
        
        Notemos de la siguiente forma los siguientes sucesos:
        \begin{itemize}
            \item $A$: El coche es de tipo A.
            \item $B$: El coche es de tipo B.
            \item $S$: El coche supera la prueba.
            \item $F$: El coche no supera la prueba.
        \end{itemize}

        Sabemos que:
        \begin{equation*}
            P[A] = 0.4\quad
            P[B] = 0.6\quad
            P[S\mid A] = \nicefrac{1}{3}\quad
            P[S\mid B] = \nicefrac{2}{3}
        \end{equation*}

        Nos piden calcular $P[S]$. Por la regla de la probabilidad total, tenemos que:
        \begin{align*}
            P[S] &= P[S\mid A] \cdot P[A] + P[S\mid B] \cdot P[B]
            = \frac{1}{3} \cdot 0.4 + \frac{2}{3} \cdot 0.6
            = \frac{8}{15} = 0.5333
        \end{align*}
        \item Calcula el número de coches que deben examinarse para cubrir las necesidades de la empresa con probabilidad mayor o igual que $0.90147$.
        
        Sea $n$ el número de coches examinados. Para cada $i=1, \ldots, n$, sea $X_i$ una variable aleatoria que toma el valor $1$ si el $i$-ésimo coche supera la prueba, y $0$ en caso contrario, de forma que se tiene $X_i \sim B(1, P[S])$. Consideramos la variable aleatoria:
        \[
            S_{n} = \sum_{i=1}^{n} X_i
        \]

        Por la reproducción de la binomial, tenemos que $S_{n} \sim B(n, P[S])$. Además, $S_{n}$ es el número de coches que superan la prueba en $n$ examinados. Nos piden:
        \begin{align*}
            P[S_{n} \geq 100] &\geq 0.90147
        \end{align*}

        Como se pide un resultado aproximado, usaremos el Teorema de Lévy. Tenemos que:
        \begin{align*}
            E[S_{n}] &= n \cdot P[S]\\
            \Var[S_{n}] &= n \cdot P[S] \cdot P[F]
        \end{align*}

        Por tanto, por el Teorema de Lévy, tenemos que:
        \begin{align*}
            \left\{\frac{S_{n}-E[S_n]}{\sqrt{\Var[S_{n}]}}\right\} &\xrightarrow{L} \cc{N}(0,1)
        \end{align*}

        Por tanto, sea $Z$ la tipificada de la normal, tenemos que:
        \begin{align*}
            P[S_{n} \geq 100] \geq 0.90147
            &\Longleftrightarrow P\left[\frac{S_{n}-E[S_n]}{\sqrt{\Var[S_{n}]}} \geq \frac{100 - E[S_n]}{\sqrt{\Var[S_{n}]}}\right] \geq 0.90147
            \Longleftrightarrow \\ & \Longleftrightarrow P\left[Z \geq \frac{100 - n \cdot P[S]}{\sqrt{n \cdot P[S] \cdot P[F]}}\right] \geq 0.90147
            \Longleftrightarrow \\ & \Longleftrightarrow P\left[Z < -\frac{100-n \cdot P[S]}{\sqrt{n \cdot P[S] \cdot P[F]}}\right] \geq 0.90147
        \end{align*}

        Por tanto, consultando en la tabla de la normal, tenemos que:
        \begin{align*}
            -\frac{100-n \cdot P[S]}{\sqrt{n \cdot P[S] \cdot P[F]}} \geq -1.29
            &\Longleftrightarrow \frac{100-n \cdot P[S]}{\sqrt{n \cdot P[S] \cdot P[F]}} \leq 1.29
            \Longleftrightarrow \\&\Longleftrightarrow \frac{100-n \cdot \nicefrac{8}{15}}{\sqrt{n \cdot \nicefrac{8}{15} \cdot \nicefrac{7}{15}}} \leq 1.29
            \Longleftrightarrow \\&\Longleftrightarrow 100-n \cdot \nicefrac{8}{15} \leq \sqrt{n}\cdot 0.6435
            \Longleftrightarrow \\&\Longleftrightarrow \frac{375}{2}-n-\sqrt{n}\cdot 1.2067 \leq 0
        \end{align*}

        Resolvemos para el caso de la igualdad. Tenemos que:
        \begin{equation*}
            \sqrt{n} = \frac{1.2067\pm \sqrt{1.2067^2+4\cdot \nicefrac{375}{2}}}{-2}
            = \frac{1.2067\pm 27.4127}{-2}
            \Longrightarrow
            \sqrt{n}=13.103 \Longrightarrow n=171.69
        \end{equation*}

        Por tanto, evaluando vemos que los valores que cumplen la desigualdad han de ser mayores o iguales que $171.69$. Como $n$ ha de ser un número entero, el número de coches que deben examinarse para cubrir las necesidades de la empresa con probabilidad mayor o igual que $0.90147$ es $172$.
    \end{enumerate}
\end{ejercicio}

\begin{ejercicio}
    Para realizar una compra de un determinado material eléctrico del que se sabe que es defectuoso con probabilidad $0.25$, se le somete a una determinada prueba que da los resultados A, B y C, con probabilidades $0.8$, $0.15$ y $0.05$, si el material es válido, y probabilidades $0.2$, $0.3$ y $0.5$, si el material es defectuoso. Si cada lote se somete a seis pruebas de forma independiente, y se acepta para su compra si no aparece nunca el resultado C ni más de dos veces el resultado B, calcular:
    \begin{enumerate}
        \item La probabilidad de que un lote elegido al azar sea aceptado para su compra.
        
        Notemos de la siguiente forma los siguientes sucesos:
        \begin{itemize}
            \item $V$: El material es válido.
            \item $D$: El material es defectuoso.
            \item $A$: Aparece el resultado A.
            \item $B$: Aparece el resultado B.
            \item $C$: Aparece el resultado C.
        \end{itemize}

        Sabemos que:
        \begin{align*}
            P[V] &= 0.75\quad
            P[D] = 0.25\\
            P[A\mid V] &= 0.8\quad
            P[B\mid V] = 0.15\quad
            P[C\mid V] = 0.05\\
            P[A\mid D] &= 0.2\quad
            P[B\mid D] = 0.3\quad
            P[C\mid D] = 0.5
        \end{align*}

        Por la regla de la probabilidad total, tenemos que:
        \begin{align*}
            P[A] &= P[A\mid V] \cdot P[V] + P[A\mid D] \cdot P[D] = 0.8 \cdot 0.75 + 0.2 \cdot 0.25 = \frac{13}{20} = 0.65\\
            P[B] &= P[B\mid V] \cdot P[V] + P[B\mid D] \cdot P[D] = 0.15 \cdot 0.75 + 0.3 \cdot 0.25 = \frac{3}{16} = 0.1875\\
            P[C] &= P[C\mid V] \cdot P[V] + P[C\mid D] \cdot P[D] = 0.05 \cdot 0.75 + 0.5 \cdot 0.25 = \frac{13}{80} = 0.1625
        \end{align*}

        Dado $i\in \{A,B,C\}$, sea $X_i$ la variable aleatoria que modela el número de veces que aparece el resultado $i$ en las seis pruebas de un lote. Tenemos que:
        \begin{equation*}
            (X_B, X_C) \sim \cc{M}_2(6,P[B],P[C])
        \end{equation*}

        De esta forma, la probabilidad de que un lote elegido al azar sea aceptado para su compra es:
        \begin{align*}
            P[\text{aceptado}] &= P[X_B \leq 2, X_C = 0]
            =\\&= P[X_B=2, X_C=0] + P[X_B=1, X_C=0] + P[X_B=0, X_C=0]
        \end{align*}

        Calculamos cada una de las probabilidades:
        \begin{align*}
            P[X_B=2, X_C=0] &= \binom{6!}{2!4!} \cdot 0.1875^2 \cdot 0.65^4 \approx 0.0941\\
            P[X_B=1, X_C=0] &= \binom{6!}{1!5!} \cdot 0.1875 \cdot 0.65^5 \approx 0.1305\\
            P[X_B=0, X_C=0] &= \binom{6!}{0!6!} \cdot 0.65^6 \approx 0.07542
        \end{align*}

        Por tanto, la probabilidad de que un lote elegido al azar sea aceptado para su compra es:
        \begin{align*}
            P[\text{aceptado}] &\approx 0.0941 + 0.1305 + 0.07542 = 0.3
        \end{align*}

        \item El número de lotes que hay que probar para comprar al menos $20$ con probabilidad mayor o igual que $0.9452$.
        
        Sea $n$ el número de lotes probados. Para cada $i=1, \ldots, n$, sea $X_i$ una variable aleatoria que toma el valor $1$ si el $i$-ésimo lote es aceptado, y $0$ en caso contrario, de forma que se tiene $X_i \sim B(1, P[\text{aceptado}])$. Consideramos la variable aleatoria:
        \[
            S_{n} = \sum_{i=1}^{n} X_i
        \]

        Por la reproducción de la binomial, tenemos que $S_{n} \sim B(n, P[\text{aceptado}])$. Además, $S_{n}$ es el número de lotes aceptados en $n$ probados. Nos piden:
        \begin{align*}
            P[S_{n} \geq 20] &\geq 0.9452
        \end{align*}

        Como se pide un resultado aproximado, usaremos el Teorema de Lévy. Tenemos que:
        \begin{align*}
            E[S_{n}] &= n \cdot P[\text{aceptado}]\\
            \Var[S_{n}] &= n \cdot P[\text{aceptado}] \cdot P[\text{no aceptado}]
        \end{align*}

        Por tanto, por el Teorema de Lévy, tenemos que:
        \begin{align*}
            \left\{\frac{S_{n}-E[S_n]}{\sqrt{\Var[S_{n}]}}\right\} &\xrightarrow{L} \cc{N}(0,1)
        \end{align*}

        Por tanto, sea $Z$ la tipificada de la normal, tenemos que:
        \begin{align*}
            P[S_{n} \geq 20] \geq 0.9452
            &\Longleftrightarrow P\left[\frac{S_{n}-E[S_n]}{\sqrt{\Var[S_{n}]}} \geq \frac{20 - E[S_n]}{\sqrt{\Var[S_{n}]}}\right] \geq 0.9452
            \Longleftrightarrow \\ & \Longleftrightarrow P\left[Z \geq \frac{20 - n \cdot P[\text{aceptado}]}{\sqrt{n \cdot P[\text{aceptado}] \cdot P[\text{no aceptado}]}}\right] \geq 0.9452
            \Longleftrightarrow \\ & \Longleftrightarrow P\left[Z < -\frac{20 - n \cdot P[\text{aceptado}]}{\sqrt{n \cdot P[\text{aceptado}] \cdot P[\text{no aceptado}]}}\right] \geq 0.9452
            \Longleftrightarrow \\ & \Longleftrightarrow P\left[Z < -\frac{20 - 0.3n}{\sqrt{0.21n}}\right] \geq 0.9452
        \end{align*}

        Por tanto, consultando en la tabla de la normal, tenemos que:
        \begin{align*}
            -\frac{20 - 0.3n}{\sqrt{0.21n}} \geq 1.6
            &\Longleftrightarrow \frac{0.3n-20}{\sqrt{0.21n}} \geq 1.6
            \Longleftrightarrow \\&\Longleftrightarrow 0.3n-20 \geq 0.733\sqrt{n}
            \Longleftrightarrow \\&\Longleftrightarrow 0.3n-0.733\sqrt{n}-20 \geq 0
        \end{align*}

        Resolvemos para el caso de la igualdad. Tenemos que:
        \begin{equation*}
            \sqrt{n} = \frac{0.733\pm \sqrt{0.733^2+4\cdot 0.3\cdot 20}}{0.6}
            = \frac{0.733\pm 4.9535}{0.6}
            \Longrightarrow
            \sqrt{n}=9.477 \Longrightarrow n=89.823
        \end{equation*}

        Por tanto, evaluando vemos que los valores que cumplen la desigualdad han de ser mayores o iguales que $89.823$. Como $n$ ha de ser un número entero, el número de lotes que hay que probar para comprar al menos $20$ con probabilidad mayor o igual que $0.9452$ es $90$.
    \end{enumerate}
    
\end{ejercicio}

\begin{ejercicio}
    La longitud de vida (en horas) de una determinada pieza de cierta máquina es una variable aleatoria que se distribuye de acuerdo a la función de densidad
    \[
        f (x) = \exp(1 - x), \quad x > 1.
    \]
    Cuando falla la pieza, se sustituye por otra de las mismas características, y se supone que las longitudes de vida de distintas piezas son independientes. Calcular aproximadamente el número de piezas de recambio necesario para asegurar el funcionamiento de la máquina al menos durante $1000$ horas, con probabilidad mayor que $0.95$.

    Sea $n$ el número de piezas de recambio necesarias. Para cada $i=1, \ldots, n$, sea $X_i$ una variable aleatoria que modela la longitud de vida de la $i$-ésima pieza de recambio. Consideramos la variable aleatoria:
    \[
        S_{n} = \sum_{i=1}^{n} X_i
    \]
    Vemos que $S_{n}$ es la longitud de vida total de las $n$ piezas de recambio. Nos piden:
    \begin{align*}
        P[S_{n} \geq 1000] &\geq 0.95
    \end{align*}

    Como se pide un resultado aproximado, usaremos el Teorema de Lévy. Tenemos que:
    \begin{align*}
        E[X_1] &= \int_{1}^{\infty} x \cdot \exp(1-x) \ dx = \int_{1}^{\infty} x \cdot \exp(1) \cdot \exp(-x) \ dx = e\cdot \int_{1}^{\infty} x \cdot e^{-x} \ dx
        =\\&= e\cdot \left[-(x+1)e^{-x}\right]_1^\infty
        = e\cdot 2e^{-1} = 2\\
        E[X_1^2] &= \int_{1}^{\infty} x^2 \cdot \exp(1-x) \ dx = \int_{1}^{\infty} x^2 \cdot \exp(1) \cdot \exp(-x) \ dx = e\cdot \int_{1}^{\infty} x^2 \cdot e^{-x} \ dx
        =\\&= e\cdot \left[-(x^2+2x+2)e^{-x}\right]_1^\infty
        = e\cdot 5e^{-1} = 5\\
        \Var[X_1] &= E[X_1^2] - E[X_1]^2 = 5 - 4 = 1\\
        E[S_{n}] &= n \cdot E[X_1] = 2n\\
        \Var[S_{n}] &= n \cdot \Var[X_1] = n
    \end{align*}

    Por tanto, por el Teorema de Lévy, tenemos que:
    \begin{align*}
        \left\{\frac{S_{n}-E[S_n]}{\sqrt{\Var[S_{n}]}}\right\} &\xrightarrow{L} \cc{N}(0,1)
    \end{align*}

    Por tanto, sea $Z$ la tipificada de la normal, tenemos que:
    \begin{align*}
        P[S_{n} \geq 1000] \geq 0.95
        &\Longleftrightarrow P\left[\frac{S_{n}-E[S_n]}{\sqrt{\Var[S_{n}]}} \geq \frac{1000 - E[S_n]}{\sqrt{\Var[S_{n}]}}\right] \geq 0.95
        \Longleftrightarrow \\ & \Longleftrightarrow P\left[Z \geq \frac{1000 - 2n}{\sqrt{n}}\right] \geq 0.95
        \Longleftrightarrow \\ & \Longleftrightarrow P\left[Z < -\frac{1000 - 2n}{\sqrt{n}}\right] \geq 0.95
    \end{align*}

    Por tanto, consultando en la tabla de la normal, tenemos que:
    \begin{align*}
        -\frac{1000 - 2n}{\sqrt{n}} \geq 1.645
        &\Longleftrightarrow \frac{2n-1000}{\sqrt{n}} \geq 1.645
        \Longleftrightarrow \\&\Longleftrightarrow 2n-1000 \geq 1.645\sqrt{n}
        \Longleftrightarrow \\&\Longleftrightarrow 2n-1.645\sqrt{n}-1000 \geq 0
    \end{align*}

    Resolvemos para el caso de la igualdad. Tenemos que:
    \begin{equation*}
        \sqrt{n} = \frac{1.645\pm \sqrt{1.645^2+4\cdot 2\cdot 1000}}{4}
        = \frac{1.645\pm 89.4578}{4}
        \Longrightarrow
        \sqrt{n}=22.7757 \Longrightarrow n=518.73
    \end{equation*}

    Por tanto, evaluando vemos que los valores que cumplen la desigualdad han de ser mayores o iguales que $518.73$. Como $n$ ha de ser un número entero, el número de piezas de recambio necesario para asegurar el funcionamiento de la máquina al menos durante $1000$ horas, con probabilidad mayor que $0.95$ es $519$.
\end{ejercicio}
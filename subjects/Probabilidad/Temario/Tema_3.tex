\chapter{Independencia de variables Aleatorias}

Estudiemos ahora la independencia de variables aleatorias, que puede recordarnos a indepencia de sucesos en estadística descriptiva.
\begin{definicion}[Independencia de variables aleatorias]
    Sean $X_1, \ldots, X_n$ variables aleatorias definidas sobre el mismo espacio de probabilidad con funciones de distribución $F_{X_1}, \ldots, F_{X_n}$. Consideramos el vector aleatorio $X=(X_1,\dots,X_n)$ con función de distribución conjunta $F_X$. Se dice que dichas variables son mutuamente independientes (o, simplemente, independientes) si:
    \[
        F_X(x_1, \ldots, x_n) = F_{X_1}(x_1) \cdot \ldots \cdot F_{X_n}(x_n), \quad \forall x_1, \ldots, x_n \in \bb{R}
    \]
\end{definicion}
\begin{observacion}
    Notemos que esta definición considera el vector aleatorio $X=(X_1,\dots,X_n)$, y en esta asignatura ya mencionamos que no consideraremos vectores aleatorios mixtos. Por tanto, dadas $X$ y $Y$ dos variables aleatorias, una de ellas discreta y la otra continua, no estudiaremos su independencia.
\end{observacion}


\section{Caracterizaciones de independencia para variables discretas}

A continuación, veremos dos caracterizaciones de independencia para variables aleatorias discretas.
\begin{prop}[Caracterización mediante funciones de masa de probabilidad]
    Sean $X_1, \ldots, X_n$ variables aleatorias discretas definidas sobre el mismo espacio de probabilidad. Consideramos el vector aleatorio $X=(X_1,\dots,X_n)$. Entonces, $X_1, \ldots, X_n$ son independientes si y solo si:
    \[
        P[X_1 = x_1, \ldots, X_n = x_n] = P[X_1 = x_1] \cdot \ldots \cdot P[X_n = x_n], \quad \forall x_1, \ldots, x_n \in \bb{R}
    \]
\end{prop}
% // TODO: Caracterización mediante funciones de masa de probabilidad
% https://www.ugr.es/~cdpye/CursoProbabilidad/pdf/P_T07_CaracterizacionDiscreta.pdf

\begin{ejemplo}
    Consideramos las variables aleatorias $X_1$ y $X_2$ con la siguiente función de masa de probabilidad conjunta:
    \begin{equation*}
        \begin{array}{c|cc|c}
            X_1\backslash X2 & -2 & 1 \\
            \hline
            -1 & \nicefrac{1}{12} & \nicefrac{1}{6}  & \nicefrac{3}{12}= \nicefrac{1}{4} \\
            0 & \nicefrac{1}{6} & \nicefrac{1}{3} & \nicefrac{3}{6} = \nicefrac{1}{2} \\
            1 & \nicefrac{1}{12} & \nicefrac{1}{6} & \nicefrac{3}{12} = \nicefrac{1}{4} \\ \hline
            &  \nicefrac{4}{12}= \nicefrac{1}{3} & \nicefrac{4}{6} = \nicefrac{2}{3} & 1
        \end{array}
    \end{equation*}
    Comprobemos que $X_1$ y $X_2$ son independientes.\\

    Notemos que hemos añadido la última fila y columna para facilitar los cálculos. Para comprobar la independencia, debemos comprobar que la función de masa de probabilidad conjunta es igual al producto de las funciones de masa de probabilidad marginales. Directamente con la tabla, vemos se tiene para todos los casos, aunque vamos a escribirlo explícitamente:
    \begin{align*}
        P[X_1 = -1, X_2 = -2] &= \nicefrac{1}{12} = P[X_1 = -1] \cdot P[X_2 = -2] = \nicefrac{1}{3} \cdot \nicefrac{1}{4} = \nicefrac{1}{12} \\
        P[X_1 = -1, X_2 = 1] &= \nicefrac{1}{6} = P[X_1 = -1] \cdot P[X_2 = 1] = \nicefrac{2}{3} \cdot \nicefrac{1}{4} = \nicefrac{2}{12}=\nicefrac{1}{6} \\
        P[X_1 = 0, X_2 = -2] &= \nicefrac{1}{6} = P[X_1 = 0] \cdot P[X_2 = -2] = \nicefrac{1}{2} \cdot \nicefrac{1}{3} = \nicefrac{1}{6} \\
        P[X_1 = 0, X_2 = 1] &= \nicefrac{1}{3} = P[X_1 = 0] \cdot P[X_2 = 1] = \nicefrac{1}{2} \cdot \nicefrac{2}{3} = \nicefrac{1}{3} \\
        P[X_1 = 1, X_2 = -2] &= \nicefrac{1}{12} = P[X_1 = 1] \cdot P[X_2 = -2] = \nicefrac{1}{4} \cdot \nicefrac{1}{3} = \nicefrac{1}{12} \\
        P[X_1 = 1, X_2 = 1] &= \nicefrac{1}{6} = P[X_1 = 1] \cdot P[X_2 = 1] = \nicefrac{1}{4} \cdot \nicefrac{2}{3} = \nicefrac{1}{6}
    \end{align*}

    Por tanto, $X_1$ y $X_2$ son independientes.
\end{ejemplo}


\begin{prop}[Caracterización mediante factorización de la función masa de probabilidad conjunta]
    Sean $X_1, \ldots, X_n$ variables aleatorias discretas definidas sobre el mismo espacio de probabilidad. Entonces, $X_1, \ldots, X_n$ son independientes si y solo si:
    \[
        P[X_1 = x_1, \ldots, X_n = x_n] = h_1(x_1) \cdot \ldots \cdot h_n(x_n), \quad \forall x_1, \ldots, x_n \in \bb{R}
    \]
    siendo $h_i: \bb{R} \to \bb{R}$ funciones arbitrarias para $i = 1, \ldots, n$.
\end{prop}
\begin{observacion}
    Notemos que la segunda caracterización de independencia para variables discretas es más general que la primera, ya que las funciones $h_i$ pueden ser arbitrarias. Por tanto, no es necesario el cálculo de las funciones masa de probabilidad marginales para comprobar la independencia.
\end{observacion}
% // TODO: Caracterización mediante factorización de la función masa de probabilidad conjunta
% https://www.ugr.es/~cdpye/CursoProbabilidad/pdf/P_T07_FactorizacionDiscreta.pdf

Veamos un ejemplo de esta caracterización, donde la anterior observación entra en juego.
\begin{ejemplo}
    Sea $X=(X_1,X_2)$ un vector aleatorio con función de masa de probabilidad conjunta:
    \begin{equation*}
        P[X_1=x_1,X_2=x_2]=\dfrac{1}{2^{x_1+1}}
        \qquad \forall x_1\in \bb{N},~x_2\in \{0,1\}
    \end{equation*}
    Comprobemos que $X_1$ y $X_2$ son independientes.\\

    \begin{description}
        \item[Cálculo de las funciones masa de probabilidad marginales]
        
        Calculamos ambas marginales:
        \begin{align*}
            P[X_1=x_1] &= \sum_{x_2\in \{0,1\}} P[X_1=x_1,X_2=x_2] = \sum_{x_2\in \{0,1\}} \dfrac{1}{2^{x_1+1}} = \dfrac{1}{2^{x_1+1}}\cdot 2 = \dfrac{1}{2^{x_1}} \\
            P[X_2=x_2] &= \sum_{x_1\in \bb{N}} P[X_1=x_1,X_2=x_2] = \sum_{x_1=1}^{\infty} \dfrac{1}{2^{x_1+1}} 
            =\dfrac{1}{2}\sum_{x_1=1}^{\infty} \dfrac{1}{2^{x_1}}
            =\dfrac{1}{2}\left(-1+\sum_{x_1=0}^{\infty} \dfrac{1}{2^{x_1}}\right)
            =\\&=\dfrac{1}{2}\left(-1+\dfrac{1}{1-\nicefrac{1}{2}}\right)
            = \dfrac{1}{2}\left(-1+2\right) = \dfrac{1}{2}
        \end{align*}

        Por tanto, tenemos que:
        \begin{align*}
            P[X_1=x_1]\cdot P[X_2=x_2] &= \dfrac{1}{2^{x_1}}\cdot \dfrac{1}{2} = \dfrac{1}{2^{x_1+1}} = P[X_1=x_1,X_2=x_2]
        \end{align*}
        Por tanto, $X_1$ y $X_2$ son independientes.

        \item[Factorización de la función masa de probabilidad conjunta]
        
        Consideramos las siguientes funciones auxiliares:
        \Func{h_1}{\bb{R}}{\bb{R}}{x_1}{\begin{cases}
            \dfrac{1}{2^{x_1+1}} & \text{si } x_1\in \bb{N} \\
            0 & \text{en otro caso}
        \end{cases}}
        \Func{h_2}{\bb{R}}{\bb{R}}{x_2}{\begin{cases}
            1 & \text{si } x_2\in \{0,1\} \\
            0 & \text{en otro caso}
        \end{cases}}
    
        Dado $x_1\in \bb{N}$ y $x_2\in \{0,1\}$, se tiene:
        \begin{equation*}
            P[X_1=x_1,X_2=x_2]=\dfrac{1}{2^{x_1+1}}=h_1(x_1)\cdot h_2(x_2)
        \end{equation*}
        Por tanto, $X_1$ y $X_2$ son independientes.
    \end{description}

    Como vemos, la segunda caracterización nos ha permitido comprobar la independencia de $X_1$ y $X_2$ sin necesidad de calcular las funciones masa de probabilidad marginales. 
\end{ejemplo}


\section{Caracterizaciones de independencia para variables continuas}

A continuación, veremos dos caracterizaciones de independencia para variables aleatorias continuas; las cuales serán análogas a las de variables discretas.
\begin{prop}[Caracterización mediante funciones de densidad de probabilidad]
    Sean $X_1, \ldots, X_n$ variables aleatorias continuas definidas sobre el mismo espacio de probabilidad. Entonces, $X_1, \ldots, X_n$ son independientes si y solo si:
    \[
        \left\{
            \begin{array}{l}
                X=(X_1, \ldots, X_n) \text{ es un vector aleatorio continuo} \\
                \quad \land \\
                f_X(x_1, \ldots, x_n) = f_{X_1}(x_1) \cdot \ldots \cdot f_{X_n}(x_n), \quad \forall x_1, \ldots, x_n \in \bb{R}
            \end{array}
        \right.
    \]
\end{prop}
\begin{proof}
    Demostraremos mediante doble implicación.
    \begin{description}
        \item[$\Longrightarrow)$] Supongamos que $X_1, \ldots, X_n$ son independientes. Entonces, la función de distribución conjunta es:
        \begin{align*}
            F_X(x_1, \ldots, x_n) &= F_{X_1}(x_1) \cdot \ldots \cdot F_{X_n}(x_n) 
            =\\&= \left(\int_{-\infty}^{x_1} f_{X_1}(t_1) \ dt_1\right) \cdot \dots \cdot \left(\int_{-\infty}^{x_n} f_{X_n}(t_n) \ dt_n\right)
            =\\&= \int_{-\infty}^{x_1} \int_{-\infty}^{x_2} \dots \int_{-\infty}^{x_n} f_{X_1}(t_1) \cdot \ldots \cdot f_{X_n}(t_n) \ dt_1 \dots dt_n
        \end{align*}

        Por tanto, tenemos que $X$ es de tipo continuo, con función de densidad:
        \begin{equation*}
            f_X(x_1, \ldots, x_n) = f_{X_1}(x_1) \cdot \ldots \cdot f_{X_n}(x_n)
            \qquad \forall x_1, \ldots, x_n \in \bb{R}
        \end{equation*}
        Vemos, de hecho, que esta es integrable y no negativa.

        \item[$\Longleftarrow)$] Supongamos que $X_1, \ldots, X_n$ son tales que $X$ es un vector aleatorio continuo y:
        \begin{equation*}
            f_X(x_1, \ldots, x_n) = f_{X_1}(x_1) \cdot \ldots \cdot f_{X_n}(x_n), \quad \forall x_1, \ldots, x_n \in \bb{R}
        \end{equation*}

        Comprobemos ahora que son independientes. Tenemos que:
        \begin{align*}
            F_X(x_1, \ldots, x_n) &= \int_{-\infty}^{x_1} \int_{-\infty}^{x_2} \dots \int_{-\infty}^{x_n} f_{X_1}(t_1) \cdot \ldots \cdot f_{X_n}(t_n) \ dt_1 \dots dt_n
            =\\&= \left(\int_{-\infty}^{x_1} f_{X_1}(t_1) \ dt_1\right) \cdot \dots \cdot \left(\int_{-\infty}^{x_n} f_{X_n}(t_n) \ dt_n\right)
            =\\&= F_{X_1}(x_1) \cdot \ldots \cdot F_{X_n}(x_n)
        \end{align*}

        Por tanto, $X_1, \ldots, X_n$ son independientes.
    \end{description}
\end{proof}

\begin{prop}[Caracterización mediante factorización de la función densidad de probabilidad conjunta]
    Sean $X_1, \ldots, X_n$ variables aleatorias continuas definidas sobre el mismo espacio de probabilidad. Entonces, $X_1, \ldots, X_n$ son independientes si y solo si:
    \[
        \left\{
            \begin{array}{l}
                X=(X_1, \ldots, X_n) \text{ es un vector aleatorio continuo} \\
                \quad \land \\
                f_X(x_1, \ldots, x_n) = h_1(x_1) \cdot \ldots \cdot h_n(x_n), \quad \forall x_1, \ldots, x_n \in \bb{R}
            \end{array}
        \right.
    \]
    siendo $h_i: \bb{R} \to \bb{R}$ funciones arbitrarias para $i = 1, \ldots, n$.
\end{prop}
% // TODO: Caracterización mediante factorización de la función densidad de probabilidad conjunta
% https://www.ugr.es/~cdpye/CursoProbabilidad/pdf/P_T07_FactorizacionContinua.pdf

\begin{observacion}
    De nuevo, consideramos la misma observación que en el caso de variables discretas: la segunda caracterización de independencia para variables continuas es más general que la primera, ya que las funciones $h_i$ pueden ser arbitrarias. Por tanto, no es necesario el cálculo de las funciones densidad de probabilidad marginales para comprobar la independencia.
\end{observacion}


\section{Caracterización mediante conjuntos de Borel}

Esta es la última caracterización de independencia que veremos, la cual es más general que las anteriores. Aunque esta no se usará en la práctica, es interesante verla puesto que nos relaciona directamente con la independencia de sucesos, explicada en Estadística Descriptiva.
\begin{prop}[Caracterización mediante conjuntos de Borel]
    Sean $X_1, \ldots, X_n$ variables aleatorias definidas sobre el mismo espacio de probabilidad. Entonces, $X_1, \ldots, X_n$ son independientes si y solo si:
    \[
        P[X_1 \in B_1, \ldots, X_n \in B_n] = P[X_1 \in B_1] \cdot \ldots \cdot P[X_n \in B_n], \quad \forall B_1, \ldots, B_n \in \mathcal{B}
    \]
\end{prop}
\begin{proof}
    Demostramos mediante doble implicación:
    \begin{description}
        \item[$\Longrightarrow)$] Supongamos que $X_1, \ldots, X_n$ son independientes. Entonces, distinguimos en función de si son discretas o continuas:
        \begin{itemize}
            \item Si son discretas, para todo $B_1, \ldots, B_n \in \mathcal{B}$, tenemos que:
            \begin{align*}
                P[X_1 \in B_1, \ldots, X_n \in B_n] &= \sum_{\substack{x_1 \in B_1\\ \vdots \\ x_n \in B_n}} P[X_1 = x_1, \ldots, X_n = x_n]
                \AstIg \\ &\AstIg
                \sum_{\substack{x_1 \in B_1\\ \vdots \\ x_n \in B_n}} P[X_1 = x_1] \cdot \ldots \cdot P[X_n = x_n]
                =\\&= \sum_{x_1 \in B_1} P[X_1 = x_1] \cdot \ldots \cdot \sum_{x_n \in B_n} P[X_n = x_n]
                =\\&= P[X_1 \in B_1] \cdot \ldots \cdot P[X_n \in B_n]
            \end{align*}
            donde en $(\ast)$ hemos usado la independencia de las variables aleatorias discretas.
            \item Si son continuas, tenemos que:
            \begin{align*}
                P[X_1 \in B_1, \ldots, X_n \in B_n] &= \int_{B_1} \dots \int_{B_n} f_X(x_1, \ldots, x_n) \ dx_1 \dots dx_n
                \AstIg\\& \AstIg \int_{B_1} \dots \int_{B_n} f_{X_1}(x_1) \cdot \ldots \cdot f_{X_n}(x_n) \ dx_1 \dots dx_n
                =\\&= \int_{B_1} f_{X_1}(x_1) \ dx_1 \cdot \ldots \cdot \int_{B_n} f_{X_n}(x_n) \ dx_n
                =\\&= P[X_1 \in B_1] \cdot \ldots \cdot P[X_n \in B_n]
            \end{align*}
            donde en $(\ast)$ hemos usado la independencia de las variables aleatorias continuas.
        \end{itemize}

        \item[$\Longleftarrow)$] Supongamos que $X_1, \ldots, X_n$ son tales que:
        \begin{equation*}
            P[X_1 \in B_1, \ldots, X_n \in B_n] = P[X_1 \in B_1] \cdot \ldots \cdot P[X_n \in B_n], \quad \forall B_1, \ldots, B_n \in \mathcal{B}
        \end{equation*}

        Tomando $B_i=\left]-\infty, x_i\right]$ para $i=1,\dots,n$, siendo $x_i\in \bb{R}$,
        tenemos que:
        \begin{align*}
            P[X_1 \leq x_1, \ldots, X_n \leq x_n] &= P[X_1 \in B_1, \ldots, X_n \in B_n]
            =\\&= P[X_1 \in B_1] \cdot \ldots \cdot P[X_n \in B_n]
            = P[X_1 \leq x_1] \cdot \ldots \cdot P[X_n \leq x_n]
        \end{align*}

        Por tanto, $X_1, \ldots, X_n$ son independientes.
    \end{description}
\end{proof}

\section{Propiedades de la independencia}

\begin{prop}
    Sea $X=c$ una variable aleatoria degenerada. Entonces $X$ es independiente de cualquier otra variable aleatoria $Y$.
\end{prop}
\begin{proof}
    Sea $X=c$ una variable aleatoria degenerada, y sea $Y$ otra variable aleatoria cualquiera. Entonces, $X$ y $Y$ son independientes, ya que:
    \begin{equation*}
        P[X\leq x] = \begin{cases}
            0 & \text{si } x< c \\
            1 & \text{si } x\geq c
        \end{cases}
        \qquad
        P[X\leq x, Y\leq y] = \begin{cases}
            0 & \text{si } x< c \\
            P[Y\leq y] & \text{si } x\geq c
        \end{cases}
    \end{equation*}

    Por tanto, $P[X\leq x, Y\leq y] = P[X\leq x]P[Y\leq y]$.
\end{proof}

\begin{prop}
    Las variables de cualquier subconjunto de variables independientes son independientes. Es decir, si $X_1, \ldots, X_n$ son independientes, entonces $X_{i_1}, \ldots, X_{i_k}$ son independientes para cualquier subconjunto $\{i_1, \ldots, i_k\}\subset \{1, \ldots, n\}$.
\end{prop}
\begin{proof}
    En las distribuciones marginales, vimos que:
    \begin{equation*}
        F_{X_{i_1},\dots,X_{i_k}}(x_{i_1},\dots,x_{i_k}) = \lim_{\substack{t_j\to +\infty,\\j\neq i_1,\dots,i_k}} F_X(t_1, \ldots, x_{i_1}, \ldots, x_{i_k}, \ldots, t_n)
    \end{equation*}
    \begin{observacion}
        Notemos que esta notación no es del todo precisa, pero se emplea para no complicar en exceso la notación. En cada componente de la función de distribución conjunta evaluamos en $t_j$ en las variables que no pertenecen al subconjunto, y en $x_{i_j}$ en las que sí pertenecen.
    \end{observacion}

    Como $X_1, \ldots, X_n$ son independientes, tenemos que:
    \begin{equation*}
        F_{X_{i_1},\dots,X_{i_k}}(x_{i_1},\dots,x_{i_k}) = \lim_{\substack{t_j\to +\infty,\\j\neq i_1,\dots,i_k}} F_{X_1}(t_1) \cdot \ldots \cdot F_{X_{i_1}}(x_{i_1}) \cdot \ldots \cdot F_{X_{i_k}}(x_{i_k}) \cdot \ldots \cdot F_{X_n}(t_n)
    \end{equation*}

    Usamos ahora que, dada una función de distribución $F$ de una variable aleatoria cualquiera se tiene que:
    \begin{equation*}
        \lim_{t\to +\infty} F(t) = 1
    \end{equation*}

    Por tanto:
    \begin{equation*}
        F_{X_{i_1},\dots,X_{i_k}}(x_{i_1},\dots,x_{i_k}) = F_{X_{i_1}}(x_{i_1}) \cdot \ldots \cdot F_{X_{i_k}}(x_{i_k})
    \end{equation*}

    Por tanto, $X_{i_1}, \ldots, X_{i_k}$ son independientes.
\end{proof}
% https://www.ugr.es/~cdpye/CursoProbabilidad/pdf/P_T07_DefinicionIndependencia.pdf


\begin{coro}
    Sean $X_1, \ldots, X_n$ variables aleatorias sobre el mismo espacio de probabilidad. Estas son independientes si y solo si las distribuciones condicionadas de cualquier subvector a cualquier otro coinciden con la marginal del primero.
\end{coro}
\begin{proof}
    Supongamos que las variables aleatorias $X_1, \ldots, X_n$ son discretas. Demostramos por doble implicación:
    \begin{description}
        \item[$\Longrightarrow)$] Supongamos que $X_1, \ldots, X_n$ son independientes. Entonces, para cualquier subvector $X_{i_1}, \ldots, X_{i_k}$ y cualquier otro subvector $X_{j_1}, \ldots, X_{j_l}$, tenemos que:
        \begin{align*}
            P[X_{i_1} = x_{i_1}, &\ldots, X_{i_k} = x_{i_k} | X_{j_1} = x_{j_1}, \ldots, X_{j_l} = x_{j_l}] =\\&= \dfrac{P[X_{i_1} = x_{i_1}, \ldots, X_{i_k} = x_{i_k}, X_{j_1} = x_{j_1}, \ldots, X_{j_l} = x_{j_l}]}{P[X_{j_1} = x_{j_1}, \ldots, X_{j_l} = x_{j_l}]}
        \end{align*}

        Dado que $X_1, \ldots, X_n$ son independientes, por la propiedad anterior, tenemos que $X_{i_1}, \ldots, X_{i_k}$ y $X_{j_1}, \ldots, X_{j_l}$ son dos conjuntos de variables independientes. Por tanto, tenemos que:
        \begin{align*}
            P[X_{i_1} = x_{i_1}, &\ldots, X_{i_k} = x_{i_k} \mid X_{j_1} = x_{j_1}, \ldots, X_{j_l} = x_{j_l}]
            =\\&=  \dfrac{P[X_{i_1} = x_{i_1}] \cdot \ldots \cdot P[X_{i_k} = x_{i_k}] \cdot \cancel{P[X_{j_1} = x_{j_1}] \cdot \ldots \cdot P[X_{j_l} = x_{j_l}]} }{\cancel{P[X_{j_1} = x_{j_1}] \cdot \ldots \cdot P[X_{j_l} = x_{j_l}]} }
            =\\&= P[X_{i_1} = x_{i_1}] \cdot \ldots \cdot P[X_{i_k} = x_{i_k}]
            =\\&= P[X_{i_1} = x_{i_1}, \ldots, X_{i_k} = x_{i_k}]
        \end{align*}
        donde hemos empleado la factorización de la función masa de probabilidad conjunta en funciones masa de probabilidad marginales. Por tanto, las distribuciones condicionadas coinciden con las marginales.

        \item[$\Longleftarrow)$] Supongamos que las distribuciones condicionadas de cualquier subvector a cualquier otro coinciden con la marginal del primero. Entonces, para cualquier subvector $X_{i_1}, \ldots, X_{i_k}$ y cualquier otro subvector $X_{j_1}, \ldots, X_{j_l}$, tenemos que:
        \begin{align*}
            P[X_{i_1} = x_{i_1}, &\ldots, X_{i_k} = x_{i_k}, X_{j_1} = x_{j_1}, \ldots, X_{j_l} = x_{j_l}]
            =\\&= P[X_{i_1} = x_{i_1}, \ldots, X_{i_k} = x_{i_k}\mid X_{j_1} = x_{j_1}, \ldots, X_{j_l} = x_{j_l}] \cdot\\&\qquad\qquad \cdot  P[X_{j_1} = x_{j_1}, \ldots, X_{j_l} = x_{j_l}]
        \end{align*}
        Usando que las distribuciones condicionadas coinciden con las marginales, tenemos que:
        \begin{align*}
            P[X_{i_1} = x_{i_1}, &\ldots, X_{i_k} = x_{i_k}, X_{j_1} = x_{j_1}, \ldots, X_{j_l} = x_{j_l}]
            =\\&= P[X_{i_1} = x_{i_1}, \ldots, X_{i_k} = x_{i_k}] \cdot P[X_{j_1} = x_{j_1}, \ldots, X_{j_l} = x_{j_l}]
        \end{align*}

        Usando la proposición anterior, tenemos que $X_{i_1}, \ldots, X_{i_k}$ y $X_{j_1}, \ldots, X_{j_l}$ son dos conjuntos de variables independientes. Usando la factorización de su función masa de probabilidad conjunta, tenemos que:
        \begin{align*}
            P[X_{i_1} = x_{i_1}, &\ldots, X_{i_k} = x_{i_k}, X_{j_1} = x_{j_1}, \ldots, X_{j_l} = x_{j_l}]
            =\\&= P[X_{i_1} = x_{i_1}] \cdot \ldots \cdot P[X_{i_k} = x_{i_k}] \cdot P[X_{j_1} = x_{j_1}] \cdot \ldots \cdot P[X_{j_l} = x_{j_l}]
        \end{align*}

        Por tanto, usando la factorización de la función masa de probabilidad conjunta de $X_1, \ldots, X_n$, tenemos que estas son independientes.
    \end{description}

    El caso continuo es análogo, usando la factorización de la función densidad de probabilidad conjunta.
\end{proof}
% https://www.ugr.es/~cdpye/CursoProbabilidad/pdf/P_T07_PropiedadesIndependencia.pdf

\begin{prop}
    Sean $X_1, \ldots, X_n$ variables aleatorias sobre el mismo espacio de probabilidad, y consideramos funciones medibles $g_i: \bb{R} \to \bb{R}$ para $i = 1, \ldots, n$. Entonces:
    \begin{equation*}
        X_1, \ldots, X_n \text{ son independientes} \Longrightarrow g_1(X_1), \ldots, g_n(X_n) \text{ son independientes}
    \end{equation*}
\end{prop}
\begin{proof}
    Calculamos la función de distribución conjunta de $g_1(X_1), \ldots, g_n(X_n)$ usando el Teorema de Cambio de Variable:
    \begin{align*}
        F_{g_1(X_1), \ldots, g_n(X_n)}(y_1, \ldots, y_n)
        &= F_{X_1, \ldots, X_n}(g_1^{-1}(y_1), \ldots, g_n^{-1}(y_n))
        =\\&= F_{X_1}(g_1^{-1}(y_1)) \cdot \ldots \cdot F_{X_n}(g_n^{-1}(y_n))
        =\\&= F_{g_1(X_1)}(y_1) \cdot \ldots \cdot F_{g_n(X_n)}(y_n)
    \end{align*}
    Por tanto, $g_1(X_1), \ldots, g_n(X_n)$ son independientes.
\end{proof}


\subsection{Teorema de la multiplicación de las esperanzas}
Incluimos el siguiente teorema. Este es de tal importancia que se le ha dado una sección propia, ya que tiene gran variedad de resultados.
\begin{teo}[Teorema de la Multiplicación de las Esperanzas]
    Sean $X_1, \ldots, X_n$ variables aleatorias sobre el mismo espacio de probabilidad, donde suponemos que $\exists E[X_i]$ para todo $i = 1, \ldots, n$. Entonces:
    \begin{equation*}
        X_1, \ldots, X_n \text{ son independientes} \Longrightarrow \left\{
            \begin{array}{l}
                \exists E[X_1 \cdot \ldots \cdot X_n] \\
                \quad \land \\
                E[X_1 \cdot \ldots \cdot X_n] = E[X_1] \cdot \ldots \cdot E[X_n]
            \end{array}
        \right.
    \end{equation*}
\end{teo}
% // TODO: Demostración
% https://www.ugr.es/~cdpye/CursoProbabilidad/pdf/P_T07_PropiedadesIndependencia.pdf

\begin{coro}
    Sean $X,Y$ dos variables aleatorias sobre el mismo espacio de probabilidad independientes. Entonces, si $\exists E[X^2],E[Y^2]$, se tiene:
    \[
        \Cov[X,Y] = 0
    \]
\end{coro}
\begin{proof}
    Dado que $X$ e $Y$ son independientes, se tiene:
    \[
        \Cov[X,Y] = E[XY] - E[X]E[Y] = E[X]E[Y] - E[X]E[Y] = 0
    \]
\end{proof}

\begin{coro}
    Sean $X_1,\dots,X_n$ variables aleatorias independientes. Entonces, si $\exists E[X_i]$ para todo $i = 1, \ldots, n$, se tiene:
    \begin{equation*}
        \exists  \Var\left[\sum_{i=1}^n X_i\right] = \sum_{i=1}^n \Var[X_i]
    \end{equation*}
\end{coro}
\begin{proof}
    Por la Proposición~\ref{prop:VarSuma}, se tiene la existencia y:
    \begin{equation*}
        \Var\left[\sum_{i=1}^n X_i\right] = \sum_{i=1}^n \Var[X_i] + \sum_{\substack{i,j=1\\i\neq j}}^n \Cov(X_i,X_j)
    \end{equation*}

    Por el resultado anterior, todas las covarianzas son nulas, por lo que se tiene lo deseado.
\end{proof}

Notemos que el resultado anterior se tiene que demostrar directamente para $n$, no se puede demostrar en primer lugar para $2$ y luego demostrar por inducción, puesto que si $X_1, X_2,X_3$ son tres variables aleatorias independientes, no tenemos de forma directa que $X_1+X_2$ y $X_3$ sean independientes, luego en el paso inductivo no podríamos aplicar la hipótesis de inducción.


\begin{coro}[Caracterización por Funciones Generatrices de Momentos]\label{coro:generatriz_momentos_independientes}
    Sean $X_1,X_2,\ldots,X_n$ variables aleatorias independientes tales que existe su función generatriz de momentos $M_{X_i}$ en un entorno $I_i$ del origen para todo $i=1,\dots,n$. Entonces, $X_1,X_2,\ldots,X_n$ son independientes si y solo si:
    \[
            \left\{
                \begin{array}{l}
                    \exists M_X(t_1, \ldots, t_n) \quad \forall (t_1, \ldots, t_n) \in I_1 \times \ldots \times I_n \\
                    M_X(t_1,\dots,t_n)=\prod\limits_{i=1}^n M_{X_i}(t_i)\qquad \forall t_i\in I_i
                \end{array}
            \right.
        \]

    % https://www.ugr.es/~cdpye/CursoProbabilidad/pdf/P_T07_CaracterizacionFGM.pdf
\end{coro}
\begin{proof}
    Demostramos por doble implicación.
    \begin{description}
        \item[$\Longrightarrow)$] Supongamos que $X_1,X_2,\ldots,X_n$ son independientes. Entonces, para todo $(t_1, \ldots, t_n) \in I_1 \times \ldots \times I_n$, tenemos que:
        \begin{align*}
            M_X(t_1, \ldots, t_n) &= E\left[\exp\left(\sum_{i=1}^n t_iX_i\right)\right]
            = E\left[\prod_{i=1}^n \exp(t_iX_i)\right]
        \end{align*}

        Como $X_1,X_2,\ldots,X_n$ son independientes, $\exp(t_1X_1),\exp(t_2X_2),\ldots,\exp(t_nX_n)$ también lo son por ser estas transformaciones medibles de las variables aleatorias originales. Por tanto, aplicando el Teorema de la multiplicación de las esperanzas, tenemos que:
        \begin{align*}
            M_X(t_1, \ldots, t_n) &= E\left[\prod_{i=1}^n \exp(t_iX_i)\right]
            = \prod_{i=1}^n E[\exp(t_iX_i)]
            = \prod_{i=1}^n M_{X_i}(t_i)
        \end{align*}

        % // TODO: Caract fgm independientes
        % https://www.ugr.es/~cdpye/CursoProbabilidad/pdf/P_T07_CaracterizacionFGM.pdf
    \end{description}
\end{proof}


\section{Distribuciones Reproductivas}

\begin{definicion}[Reproductividad de distribuciones]
    Una distribución de variables aleatorias se dice reproductiva si, dadas $X_1$, $X_2$, \ldots, $X_n$ variables aleatorias que sigan una distribución de dicho tipo (aunque la distribución de cada variable tenga distintos parámetros), entonces la variable aleatoria dada por:
    \begin{equation*}
        X = \sum_{i=1}^{n} X_i
    \end{equation*}
    sigue una distribución del mismo tipo.
\end{definicion}

\begin{observacion}
    Notemos que, para ver que una distribución es reproductiva, usaremos el Corolario~\ref{coro:generatriz_momentos_independientes} para, dadas $n$ variables aleatorias $\{X_i\}_{i \in \{1,\ldots,n\}}$ que sigan una distribución de dicho tipo, calcular la función generatriz de momentos de la variable aleatoria $X=\sum\limits_{i=1}^n X_i$ y, como $M_X(t)$ caracteriza la distribución de $X$, calculando $M_X(t)$ sabremos ya si $X$ sigue una distribución del mismo tipo que las variables aleatorias $X_i$.
\end{observacion}

Veamos en primer lugar algunas distribuciones reproductivas para variables discretas.

\begin{prop}[Distribución Reproductiva - Binomial]
    Sean $X_1, \dots, X_n$ variables aleatorias independientes y $X_i\sim B(k_i, p)$ para $i=1,\dots,n$. Entonces:
    \begin{equation*}
        \sum_{i=1}^{n}X_i \sim B\left(\sum_{i=1}^{n}k_i, p\right)
    \end{equation*}

    \begin{proof}
        Sabemos que la función generatriz de momentos de cada una de las variables aleatorias $X_i$ es:
        \begin{equation*}
            M_{X_i}(t) = (1-p+pe^t)^{k_i} \qquad i=1,\dots,n
        \end{equation*}

        Usando el Corolario \ref{coro:generatriz_momentos_independientes}, tenemos que:
        \begin{align*}
            M_{\sum\limits_{i=1}^{n}X_i}(t) &= \prod_{i=1}^{n}M_{X_i}(t) = \prod_{i=1}^{n}(1-p+pe^t)^{k_i} = (1-p+pe^t)^{\sum\limits_{i=1}^{n}k_i}
        \end{align*}

        Esta es la función generatriz de momentos de una variable aleatoria que sigue una distribución binomial con parámetros $\left(\sum\limits_{i=1}^{n}k_i, p\right)$. Por tanto, hemos demostrado que:
        \begin{equation*}
            \sum_{i=1}^{n}X_i \sim B\left(\sum_{i=1}^{n}k_i, p\right)
        \end{equation*}
    \end{proof}
\end{prop}

\begin{prop}[Distribución Reproductiva - Poisson]
    Sean $X_1, \dots, X_n$ variables aleatorias independientes y $X_i\sim \cc{P}(\lambda_i)$ para $i=1,\dots,n$. Entonces:
    \begin{equation*}
        \sum_{i=1}^{n}X_i \sim \cc{P}\left(\sum_{i=1}^{n}\lambda_i\right)
    \end{equation*}

    \begin{proof}
        Sabemos que la función generatriz de momentos de cada una de las variables aleatorias $X_i$ es:
        \begin{equation*}
            M_{X_i}(t) = e^{\lambda_i(e^t-1)} \qquad i=1,\dots,n
        \end{equation*}

        Usando el Corolario \ref{coro:generatriz_momentos_independientes}, tenemos que:
        \begin{align*}
            M_{\sum\limits_{i=1}^{n}X_i}(t) &= \prod_{i=1}^{n}M_{X_i}(t) = \prod_{i=1}^{n}e^{\lambda_i(e^t-1)} = e^{\left(\sum\limits_{i=1}^{n}\lambda_i\right)(e^t-1)}
        \end{align*}

        Esta es la función generatriz de momentos de una variable aleatoria que sigue una distribución de Poisson con parámetro $\left(\sum\limits_{i=1}^{n}\lambda_i\right)$. Por tanto, hemos demostrado que:
        \begin{equation*}
            \sum_{i=1}^{n}X_i \sim \cc{P}\left(\sum_{i=1}^{n}\lambda_i\right)
        \end{equation*}
    \end{proof}
\end{prop}

\begin{prop}[Distribución Reproductiva - Binomial Negativa]
    Sean $X_1, \dots, X_n$ variables aleatorias independientes y $X_i\sim BN(k_i, p)$ para $i=1,\dots,n$. Entonces:
    \begin{equation*}
        \sum_{i=1}^{n}X_i \sim BN\left(\sum_{i=1}^{n}k_i, p\right)
    \end{equation*}

    \begin{proof}
        Sabemos que la función generatriz de momentos de cada una de las variables aleatorias $X_i$ es:
        \begin{equation*}
            M_{X_i}(t) = \left(\dfrac{p}{1-(1-p)e^t}\right)^{k_i} \qquad i=1,\dots,n
        \end{equation*}

        Usando el Corolario \ref{coro:generatriz_momentos_independientes}, tenemos que:
        \begin{align*}
            M_{\sum\limits_{i=1}^{n}X_i}(t) &= \prod_{i=1}^{n}M_{X_i}(t) = \prod_{i=1}^{n}\left(\dfrac{p}{1-(1-p)e^t}\right)^{k_i} = \left(\dfrac{p}{1-(1-p)e^t}\right)^{\sum\limits_{i=1}^{n}k_i}
        \end{align*}

        Esta es la función generatriz de momentos de una variable aleatoria que sigue una distribución binomial negativa con parámetros $\left(\sum\limits_{i=1}^{n}k_i, p\right)$. Por tanto, hemos demostrado que:
        \begin{equation*}
            \sum_{i=1}^{n}X_i \sim BN\left(\sum_{i=1}^{n}k_i, p\right)
        \end{equation*}
    \end{proof}
\end{prop}


Veamos el siguiente caso, último para distribuciones discretas. Aunque no es una familia de variables reproductivas, cumple la siguiente propiedad.
\begin{prop}
    Sean $X_1, \dots, X_n$ variables aleatorias independientes y $X_i\sim G(p)$ para $i=1,\dots,n$. Entonces:
    \begin{equation*}
        \sum_{i=1}^{n}X_i \sim BN(n,p)
    \end{equation*}
    \begin{proof}
        Sabemos que la función generatriz de momentos de cada una de las variables aleatorias $X_i$ es:
        \begin{equation*}
            M_{X_i}(t) = \dfrac{p}{1-(1-p)e^t} \qquad i=1,\dots,n
        \end{equation*}

        Usando el Corolario \ref{coro:generatriz_momentos_independientes}, tenemos que:
        \begin{align*}
            M_{\sum\limits_{i=1}^{n}X_i}(t) &= \prod_{i=1}^{n}M_{X_i}(t) = \prod_{i=1}^{n}\dfrac{p}{1-(1-p)e^t} = \dfrac{p^n}{(1-(1-p)e^t)^n}
        \end{align*}

        Esta es la función generatriz de momentos de una variable aleatoria que sigue una distribución binomial negativa con parámetros $BN(n,p)$. Por tanto, hemos demostrado que:
        \begin{equation*}
            \sum_{i=1}^{n}X_i \sim BN(n,p)
        \end{equation*}
    \end{proof}
\end{prop}

Veamos ahora algunas distribuciones reproductivas para variables continuas.

\begin{prop}[Distribución Reproductiva - Normal]
    Sean $X_1, \dots, X_n$ variables aleatorias independientes y $X_i\sim N(\mu_i, \sigma_i^2)$ para $i=1,\dots,n$. Entonces:
    \begin{equation*}
        \sum_{i=1}^{n}X_i \sim N\left(\sum_{i=1}^{n}\mu_i, \sum_{i=1}^{n}\sigma_i^2\right)
    \end{equation*}

    \begin{proof}
        Sabemos que la función generatriz de momentos de cada una de las variables aleatorias $X_i$ es:
        \begin{equation*}
            M_{X_i}(t) = e^{\mu_i t + \frac{\sigma_i^2 t^2}{2}} \qquad i=1,\dots,n
        \end{equation*}

        Usando el Corolario \ref{coro:generatriz_momentos_independientes}, tenemos que:
        \begin{align*}
            M_{\sum\limits_{i=1}^{n}X_i}(t) &= \prod_{i=1}^{n}M_{X_i}(t) = \prod_{i=1}^{n}e^{\mu_i t + \frac{\sigma_i^2 t^2}{2}} = e^{\left(\sum\limits_{i=1}^{n}\mu_i\right)t + \frac{\left(\sum\limits_{i=1}^{n}\sigma_i^2\right)t^2}{2}}
        \end{align*}

        Esta es la función generatriz de momentos de una variable aleatoria que sigue una distribución normal con parámetros $\left(\sum\limits_{i=1}^{n}\mu_i, \sum\limits_{i=1}^{n}\sigma_i^2\right)$. Por tanto, hemos demostrado que:
        \begin{equation*}
            \sum_{i=1}^{n}X_i \sim N\left(\sum_{i=1}^{n}\mu_i, \sum_{i=1}^{n}\sigma_i
            ^2\right)
        \end{equation*}
    \end{proof}
\end{prop}

Al igual que ocurría en el caso de la distribución geométrica, la distribución exponencial no es reproductiva. Sin embargo, cumple la siguiente propiedad.
\begin{prop}
    Sean $X_1, \dots, X_n$ variables aleatorias independientes tal que $X_i\sim \exp(\lambda)$ para $i=1,\dots,n$. Entonces:
    \begin{equation*}
        \sum_{i=1}^{n}X_i \sim \cc{E}\left(n,\lm\right)
    \end{equation*}

    \begin{proof}
        Sabemos que la función generatriz de momentos de cada una de las variables aleatorias $X_i$ es:
        \begin{equation*}
            M_{X_i}(t) = \dfrac{\lambda}{\lambda-t} \qquad i=1,\dots,n
        \end{equation*}
    
        Usando el Corolario \ref{coro:generatriz_momentos_independientes}, tenemos que:
        \begin{align*}
            M_{\sum\limits_{i=1}^{n}X_i}(t) &= \prod_{i=1}^{n}M_{X_i}(t) = \prod_{i=1}^{n}\dfrac{\lambda}{\lambda-t} = \dfrac{\lambda^n}{(\lambda-t)^n}
        \end{align*}

        Esta es la función generatriz de momentos de una variable aleatoria que sigue una distribución de Erlang con parámetros $n,\lm$. Por tanto, hemos demostrado que:
        \begin{equation*}
            \sum_{i=1}^{n}X_i \sim \cc{E}\left(n,\lm\right)
        \end{equation*}
    \end{proof}
\end{prop}

\begin{prop}[Distribución Reproductiva - Erlang]
    Sean $X_1, \dots, X_n$ variables aleatorias independientes tal que $X_i\sim \cc{E}(k_i,\lambda)$ para $i=1,\dots,n$. Entonces:
    \begin{equation*}
        \sum_{i=1}^{n}X_i \sim \cc{E}\left(\sum_{i=1}^{n}k_i,\lambda\right)
    \end{equation*}

    \begin{proof}
        Sabemos que la función generatriz de momentos de cada una de las variables aleatorias $X_i$ es:
        \begin{equation*}
            M_{X_i}(t) = \dfrac{\lambda^{k_i}}{(\lambda-t)^{k_i}} \qquad i=1,\dots,n
        \end{equation*}
    
        Usando el Corolario \ref{coro:generatriz_momentos_independientes}, tenemos que:
        \begin{align*}
            M_{\sum\limits_{i=1}^{n}X_i}(t) &= \prod_{i=1}^{n}M_{X_i}(t) = \prod_{i=1}^{n}\dfrac{\lambda^{k_i}}{(\lambda-t)^{k_i}} = \dfrac{\lambda^{\sum\limits_{i=1}^{n}k_i}}{(\lambda-t)^{\sum\limits_{i=1}^{n}k_i}}
        \end{align*}

        Esta es la función generatriz de momentos de una variable aleatoria que sigue una distribución de Erlang con parámetros $\left(\sum\limits_{i=1}^{n}k_i,\lambda\right)$. Por tanto, hemos demostrado que:
        \begin{equation*}
            \sum_{i=1}^{n}X_i \sim \cc{E}\left(\sum_{i=1}^{n}k_i,\lambda\right)
        \end{equation*}
    \end{proof}
\end{prop}

\begin{prop}[Distribución Reproductiva - Gamma]
    Sean $X_1, \dots, X_n$ variables aleatorias independientes tal que $X_i\sim \Gamma(u_i,\lambda)$ para $i=1,\dots,n$. Entonces:
    \begin{equation*}
        \sum_{i=1}^{n}X_i \sim \Gamma\left(\sum_{i=1}^{n}u_i,\lambda\right)
    \end{equation*}

    \begin{proof}
        Sabemos que la función generatriz de momentos de cada una de las variables aleatorias $X_i$ es:
        \begin{equation*}
            M_{X_i}(t) = \dfrac{\lambda^{u_i}}{(\lambda-t)^{u_i}} \qquad i=1,\dots,n
        \end{equation*}
    
        Usando el Corolario \ref{coro:generatriz_momentos_independientes}, tenemos que:
        \begin{align*}
            M_{\sum\limits_{i=1}^{n}X_i}(t) &= \prod_{i=1}^{n}M_{X_i}(t) = \prod_{i=1}^{n}\dfrac{\lambda^{u_i}}{(\lambda-t)^{u_i}} = \dfrac{\lambda^{\sum\limits_{i=1}^{n}u_i}}{(\lambda-t)^{\sum\limits_{i=1}^{n}u_i}}
        \end{align*}

        Esta es la función generatriz de momentos de una variable aleatoria que sigue una distribución gamma con parámetros $\left(\sum\limits_{i=1}^{n}u_i,\lambda\right)$. Por tanto, hemos demostrado que:
        \begin{equation*}
            \sum_{i=1}^{n}X_i \sim \Gamma\left(\sum_{i=1}^{n}u_i,\lambda\right)
        \end{equation*}
    \end{proof}
\end{prop}


\section{Independencia para familias de variables aleatorias}

Notemos que las definiciones de independencia que hemos dado hasta ahora son para un número finito de variables aleatorias. Sin embargo, podemos extender estas definiciones a familias de variables aleatorias infinitas numerables.
\begin{definicion}[Independencia mutua]
    Sea $\{X_i\}_{i\in T}$ una familia de variables aleatorias, con $T$ infinito numerable. Diremos que la familia $\{X_i\}_{i\in T}$, es independiente si, para todo subconjunto finito $T_0\subset T$, las variables aleatorias $\{X_i\}_{i\in T_0}$ son independientes.
\end{definicion}

\begin{definicion}[Independencia dos a dos]
    Sea $\{X_i\}_{i\in T}$ una familia de variables aleatorias, con $T$ infinito numerable. Diremos que la familia $\{X_i\}_{i\in T}$, es independiente dos a dos si, para todo par de índices $i,j\in T$, $i\neq j$, las variables aleatorias $X_i$ y $X_j$ son independientes.
\end{definicion}

Es directo ver que la independencia mutua implica la independencia dos a dos. Sin embargo, la independencia dos a dos no implica la independencia mutua. Por tanto, la independencia mutua es una propiedad más fuerte que la independencia dos a dos.

\section{Independencia para vectores aleatorios}

Hasta ahora, hemos visto la independencia para variables aleatorias unidimensionales. Sin embargo, podemos extender estas definiciones a vectores aleatorios multidimensionales.
\begin{definicion}[Independencia para vectores aleatorios]
    Sean $X^{(1)},\dots,X^{(n)}$ vectores aleatorios de dimensión $d_1,\dots,d_n$ respectivamente. Sea $F_X^{(i)}$ la función de distribución de $X^{(i)}$ para $i=1,\dots,n$, y $F_X$ la función de distribución conjunta de $X=(X^{(1)},\dots,X^{(n)})$. Diremos que los vectores aleatorios $X^{(1)},\dots,X^{(n)}$ son independientes si:
    \begin{equation*}
        F_X(x^{(1)},\dots,x^{(n)}) = F_X^{(1)}(x^{(1)})\cdot\dots\cdot F_X^{(n)}(x^{(n)}) \quad \forall x^{(1)}\in\bb{R}^{d_1},\dots,x^{(n)}\in\bb{R}^{d_n}
    \end{equation*}
\end{definicion}

\section{Sistemas de ecuaciones en diferencias lineales}

\begin{ejercicio}
Se consideran las siguientes matrices:
\begin{equation*}
    A_1 =
    \begin{pmatrix}
    6 & 2\\
    -3 & 2
    \end{pmatrix}
    , \hspace{1cm}
    A_2=
    \begin{pmatrix}
    -1 & 1 & 0\\
    1 & 1 & 2\\
    1 & -2 & -1
    \end{pmatrix}
    , \hspace{1cm}
    A_3=\begin{pmatrix}
    3 & 0 & 0 & 0\\
    2 & 7 & -2 & 2\\
    -8 & 1 & -2 & 1\\
    2 & 0 & 0 & 4
    \end{pmatrix}.
\end{equation*}

Determina en cada caso su radio espectral. Para cada matriz, ¿qué ecuación en diferencias de orden superior verifica cada componente del vector solución del sistema lineal homogéneo asociado?\\

En primer lugar, hallaremos el radio espectral de cada matriz. Para ello, calcularemos los valores propios de cada matriz y tomaremos el mayor de sus módulos.
\begin{enumerate}
    \item $A_1$.
    
    Tenemos que su polinomio característico es:
    \begin{equation*}
        p_{A_1}(\lambda) = \lambda^2 - 8\lambda + 18
    \end{equation*}

    Por tanto, sus valores propios son:
    \begin{equation*}
        \lm = \frac{8\pm \sqrt{2^6-2^3\cdot 3^2}}{2}
        = 4\pm \sqrt{2^4-2\cdot 3^2}
        = 4\pm \sqrt{2}\cdot i
    \end{equation*}

    Por tanto, tenemos que:
    \begin{equation*}
        \rho(A_1) = |\lm| = \sqrt{4^2 + 2} = 3\sqrt{2}
    \end{equation*}


    \item $A_2$.
    
    Tenemos que su polinomio característico es:
    \begin{align*}
        p_{A_2}(\lambda) &= |A_2 - \lambda I| =
        \begin{vmatrix}
            -1-\lambda & 1 & 0\\
            1 & 1-\lambda & 2\\
            1 & -2 & -1-\lambda
        \end{vmatrix}
        =\\&=
        (1+\lm)^2(1-\lm) +2 -4(1+\lm) + (1+\lm) =\\&=
        (1+\lm^2+2\lm)(1-\lm) +2 -4(1+\lm) + (1+\lm) =\\&=
        1-\lm + \lm^2 - \lm^3 + 2\lm - 2\lm^2 + 2 - 4 - 4\lm + 1 + \lm =\\&=
        -\lm^3 - \lm^2 - 2\lm = -\lm(\lm^2 + \lm + 2)
    \end{align*}

    Por tanto, sus valores propios, además de $\lm_1 = 0$, son:
    \begin{equation*}
        \lm_2 = \frac{-1\pm \sqrt{1-4\cdot 2}}{2}
        = \frac{-1\pm \sqrt{7}i}{2}
    \end{equation*}

    Por tanto, tenemos que:
    \begin{equation*}
        \rho(A_2) = |\lm_2| = \frac{1}{2}\sqrt{1+7} = \frac{\sqrt{8}}{2} = \sqrt{2}
    \end{equation*}

    \item $A_3$.
    
    Tenemos que su polinomio característico es:
    \begin{align*}
        p_{A_3}(\lambda) &= |A_3 - \lambda I| =
        \begin{vmatrix}
            3-\lambda & 0 & 0 & 0\\
            2 & 7-\lambda & -2 & 2\\
            -8 & 1 & -2-\lambda & 1\\
            2 & 0 & 0 & 4-\lambda
        \end{vmatrix}
        =\\&=
        (3-\lm)\begin{vmatrix}
            7-\lm & -2 & 2\\
            1 & -2-\lm & 1\\
            0 & 0 & 4-\lm
        \end{vmatrix}
        =\\&=
        (3-\lm)(4-\lm)\begin{vmatrix}
            7-\lm & -2\\
            1 & -2-\lm
        \end{vmatrix}
        =\\&= (3-\lm)(4-\lm)\left[(7-\lm)(-2-\lm) + 2\right]
        = (3-\lm)(4-\lm)(\lm^2 -5\lm - 12)
        =\\&= (3-\lm)(4-\lm)\left(\lm-\frac{5-\sqrt{73}}{2}\right)\left(\lm-\frac{5+\sqrt{73}}{2}\right)
    \end{align*}

    Aproximando los valores con la calculadora, vemos que:
    \begin{equation*}
        \rho(A_3) = \frac{5+\sqrt{73}}{2}
    \end{equation*}
\end{enumerate}

Una vez hallado el radio espectral de cada matriz, pasamos a estudiar el sistema lineal homogéneo asociado a cada matriz. Para ello, consideramos el sistema $X_{n+1} = A_iX_n$, y aplicamos el Teorema de Cayley-Hamilton para obtener una ecuación en diferencias de orden superior que verifique cada componente del vector solución del sistema.
\begin{enumerate}
    \item $A_1$.
    
    Por el Teorema de Cayley-Hamilton, tenemos que:
    \begin{equation*}
        A_1^2 - 8A_1 + 18I = 0
    \end{equation*}
    
    Tenemos que:
    \begin{align*}
        X_{n} &= A_1^nX_0
        = A_1^{n-2}A_1^2X_0
        \AstIg\\&\AstIg
        A_1^{n-2}(8A_1 - 18I)X_0
        = 8A_1^{n-1}X_0 - 18A_1^{n-2}X_0
        = 8X_{n-1} - 18X_{n-2}
    \end{align*}
    donde en $(\ast)$ hemos empleado el Teorema de Cayley-Hamilton.

    \item $A_2$.
    
    Por el Teorema de Cayley-Hamilton, tenemos que:
    \begin{equation*}
        -A_2^3 - A_2^2 - 2A_2 = 0
    \end{equation*}

    Tenemos que:
    \begin{align*}
        X_{n} &= A_2^nX_0
        = A_2^{n-3}A_2^3X_0
        \AstIg\\&\AstIg
        A_2^{n-3}(-A_2^2 - 2A_2)X_0
        = -A_2^{n-1}X_0 - 2A_2^{n-2}X_0
        = -X_{n-1} - 2X_{n-2}
    \end{align*}

    \item $A_3$.
    
    Desarrollamos su polinomio característico:
    \begin{align*}
        p_{A_3}(\lm) &= (3-\lm)(4-\lm)\left(\lm^2-5\lm-12\right)
        =\\&= \left(\lm^2-7\lm +12\right)\left(\lm^2-5\lm-12\right)
        =\\&= \lm^4 -12\lm^3+35\lm^2+24\lm-144
    \end{align*}

    Por tanto, tenemos que:
    \begin{equation*}
        A_3^4 - 12A_3^3 + 35A_3^2 + 24A_3 - 144I = 0
    \end{equation*}

    Tenemos que:
    \begin{align*}
        X_{n} &= A_3^nX_0
        = A_3^{n-4}A_3^4X_0
        \AstIg\\&\AstIg
        A_3^{n-4}(12A_3^3 - 35A_3^2 - 24A_3 + 144I)X_0
        =\\&= 12A_3^{n-1}X_0 - 35A_3^{n-2}X_0 - 24A_3^{n-3}X_0 + 144A_3^{n-4}X_0
        =\\&= 12X_{n-1} - 35X_{n-2} - 24X_{n-3} + 144X_{n-4}
    \end{align*}
\end{enumerate}
\end{ejercicio}

\begin{ejercicio}
    Sea $A$ una matriz real tal que $|A| > 1$. Demuestra que el siguiente sistema no es convergente.
    \begin{equation*}
    X_{n+1} = AX_n
    \end{equation*}
    
    Proporciona un ejemplo de matriz $A$ tal que dicho sistema no sea convergente y $0 < |A| < 1$.\\

    Supongamos que $A$ es diagonalizable, de modo que $A = PDP^{-1}$, con $D$ diagonal y $P$ matriz de cambio de base.
    Tenemos que:
    \begin{equation*}
        |A| = |PDP^{-1}| = |P||D||P^{-1}| = |D| = \prod_{i=1}^n d_i > 1
    \end{equation*}

    Por tanto, tenemos que $d_i > 1$ para algún $i\in \{1,\ldots,n\}$, por lo que
    $\rho(A) = \max_{1\leq i\leq n} |d_i| > 1$, y por tanto el sistema no es convergente.

    Para el caso de que $A$ no sea diagonalizable, es necesario emplear
    la descomposición de Jordan, concepto que no se ha visto en clase.\\

    Un ejemplo de matriz $A$ tal que el sistema no sea convergente y $0 < |A| < 1$ es:
    \begin{equation*}
        A = \begin{pmatrix}
            2 & 0\\
            0 & \nicefrac{1}{4}
        \end{pmatrix}
    \end{equation*}

    Tenemos que $|A| = \nicefrac{1}{2} < 1$, pero el sistema no es convergente debido a
    que $\rho(A) = 2 > 1$.
\end{ejercicio}

\begin{ejercicio}
Sea $A$ una matriz cuadrada con radio espectral menor que 1. Demuestra las siguientes propiedades:

\begin{enumerate}
    \item La matriz $I - A$ es invertible.
    
    Supongamos que $I-A$ no es invertible, por lo que $|I-A|=0$.
    Entonces, si $n\in \bb{N}$ es la fimensión de la matriz, tenemos que $|A-I|=(-1)^n|I-A| = 0$,
    por lo que $\lm=1\in \sigma(A)$, y por tanto $\rho(A)\geq 1$, por lo que llegamos a una contradicción.
    Por el contrarrecíproco, tenemos lo pedido.
    \item El sistema $X = AX + B$ es compatible determinado para cualquier $B \in \mathbb{R}^k$.
    
    Tenemos que:
    \begin{align*}
        X &= AX + B \Longleftrightarrow
        X - AX = B \Longleftrightarrow
        (I - A)X = B \Longleftrightarrow
        X = (I - A)^{-1}B
    \end{align*}
    donde hemos usado que $I - A$ es invertible, como se ha demostrado en el apartado anterior.
    Por tanto, tenemos que $X$ es único, por lo que el sistema es compatible determinado.

    \item Dado $B \in \mathbb{R}^k$, la solución de $X_{n+1} = AX_n + B$ tiene límite para cualquier condición inicial.
    
    La solución de la parte homogénea sabemos que es:
    \begin{equation*}
        X_n^{(h)} = A^nX_0
    \end{equation*}

    Además, sabemos que la solución del sistema del apartado es una solución constante $X^\ast$,
    por lo que la solución general es:
    \begin{equation*}
        X_n = A^nX_0 + X^\ast
    \end{equation*}

    Tomando límite, tenemos que:
    \begin{equation*}
        \lim_{n\to\infty}X_n = \lim_{n\to\infty}A^nX_0 + X^\ast
        \AstIg X^\ast
    \end{equation*}
    donde en $(\ast)$ hemos usado que $A^nX_0$ tiende a 0 por ser $\rho(A) < 1$.
\end{enumerate}
\end{ejercicio}

\begin{ejercicio}
Demuestra que las tres componentes de la solución del sistema:
\begin{equation*}
\left\{
    \begin{array}{rcrrrrr}
        a_{n+1} &=&  0.1a_n &+ 0.2b_n &+ 0.3c_n &+ 1\\
        b_{n+1} &=&  0.1a_n &+ 0.1b_n &- 0.2c_n &+ 1\\
        c_{n+1} &=& -0.2a_n &- 0.3b_n &+ 0.2c_n &+ 1
    \end{array}
\right.
\end{equation*}
con valor inicial $a_0 = 1.53$, $b_0 = 1.43$ y $c_0 = 2.2$, tienen límite y calcúlalo.\\

Tenemos que el sistema viene dado por $X_{n+1} = MX_n+B$, donde:
\begin{equation*}
    X_0 = \begin{pmatrix}
        1.53\\
        1.43\\
        2.2
    \end{pmatrix},\hspace{1cm}
    M = \begin{pmatrix}
        0.1 & 0.2 & 0.3\\
        0.1 & 0.1 & -0.2\\
        -0.2 & -0.3 & 0.2
    \end{pmatrix},
    \hspace{1cm}
    B = \begin{pmatrix}
        1\\
        1\\
        1
    \end{pmatrix}
\end{equation*}

Calculemos una solución constante del sistema, $X^\ast$, tal que $X^\ast = MX^\ast + B$;
es decir, $(I-M)X^\ast = B$:
\begin{equation*}
    \begin{pmatrix}
        0.9 & -0.2 & -0.3\\
        -0.1 & 0.9 & 0.2\\
        0.2 & 0.3 & 0.8
    \end{pmatrix}\begin{pmatrix}
        a^\ast\\
        b^\ast\\
        c^\ast
    \end{pmatrix} = \begin{pmatrix}
        1\\
        1\\
        1
    \end{pmatrix}
    \Longrightarrow
    \begin{pmatrix}
        a^\ast\\
        b^\ast\\
        c^\ast
    \end{pmatrix} = X^\ast =  \begin{pmatrix}
        \dfrac{320}{211} \\ \\
        \dfrac{250}{211} \\ \\
        \dfrac{90}{211}
    \end{pmatrix}
\end{equation*}

Calculemos ahora el límite de la solución de la parte homogénea, $X_n^{(h)} = M^nX_0$.
\begin{description}
    \item[Opción 1.] Calculando los valores propios.
    \begin{align*}
        p_M(\lm) &= |M - \lm I| = \begin{vmatrix}
            0.1-\lm & 0.2 & 0.3\\
            0.1 & 0.1-\lm & -0.2\\
            -0.2 & -0.3 & 0.2-\lm
        \end{vmatrix} =\\
        &= (0.1-\lm)^2(0.2-\lm) + 0.2^3-0.3^2\cdot 0.1 +\cancel{0.2\cdot 0.3\cdot (0.1-\lm)}-\\&\qquad -\cancel{0.2\cdot 0.3\cdot (0.1-\lm)}-0.2\cdot 0.1\cdot (0.2-\lm) =\\
        &= (0.01+\lm^2-0.2\lm)(0.2-\lm) -0.001 -0.004 +0.02\lm =\\
        &= 0.002 -0.01\lm +0.2\lm^2 -\lm^3-0.04\lm+0.2\lm^2-0.001 -0.004 +0.02\lm=\\
        &= -\lm^3+0.4\lm^2-0.03\lm -0.003
    \end{align*}

    Haciendo uso de la calculadora, vemos que las soluciones aproximadas son:
    \begin{equation*}
        \lm_1 \approx -0.05,\qquad
        \lm_2 \approx 0.22 + 0.056i,\qquad
        \lm_3 = \ol{\lm_2}
    \end{equation*}

    Por tanto, deducimos de forma directa que $\rho(M)<1$.

    \item[Opción 2.] Usando la norma de la matriz.
    
    Empleando la norma$-1$, tenemos que:
    \begin{equation*}
        \|M\|_1 = \max_{1\leq j\leq 3}\sum_{i=1}^3|m_{ij}| = 0.7 < 1
    \end{equation*}

    Por tanto, tenemos que $\rho(M)\leq \|M\|_1 < 1$.
\end{description}

En cualquier caso, $\{X_n^{h}\}\to 0$ y, por tanto,
\begin{equation*}
    \lim_{n\to \infty} X_n = X^\ast
\end{equation*}
\end{ejercicio}


\begin{ejercicio}
Se supone que el precio del desayuno en los bares sigue el modelo de la oferta y la demanda en función del precio de la leche en el mercado. Concretamente, si denotamos $p^l$ el precio de la leche y $p^d$ el precio del desayuno, y consideramos:
\begin{itemize}
    \item La demanda de la leche $D_l(p^l) = 2 - 2p^l$,
    \item La oferta de la leche $O_l(p^l) = 1 + p^l$,
    \item La demanda del desayuno $D_d(p^d) = 1 - 3p^d$,
    \item La oferta del desayuno $O_d(p^d) = 1 + 2p^d - \alpha p^l$, con $\alpha\in \bb{R}^+$,
\end{itemize}

Entonces, el modelo viene dado por:
\begin{align*}
D_l(p_n) &= O_l(p_{n-1}),\\
D_d(p_n) &= O_d(p_{n-1}),
\end{align*}
donde hemos denotado $p^d_n$ y $p^l_n$ el precio del desayuno y de la leche en el año $n$, respectivamente.
\begin{enumerate}
    \item Escribe el sistema de ecuaciones en diferencias del modelo.
    \begin{equation*}
        \left\{
            \begin{array}{l}
                2-2p_n^l = 1+p_{n-1}^l\\
                1-3p_n^d = 1+2p_{n-1}^d - \alpha p_{n-1}^l
            \end{array}
        \right\}
        \Longrightarrow
        \left\{
            \begin{array}{l}
                p_n^l = -\dfrac{1+p_{n-1}^l}{2}+1 = -\dfrac{p_{n-1}^l}{2} + \dfrac{1}{2}
                \\ \\
                p_n^d = -\dfrac{\cancel{1}+2p_{n-1}^d - \alpha p_{n-1}^l -\cancel{1}}{3}
            \end{array}
        \right\}
    \end{equation*}

    Por tanto, notamos el sistema de la forma $X_n=MX_{n-1}+B$ con:
    \begin{equation*}
        X_n = \begin{pmatrix}
            p_n^l\\
            p_n^d
        \end{pmatrix},\hspace{1cm}
        M = \begin{pmatrix}
            \nicefrac{-1}{2} & 0\\
            \nicefrac{\alpha}{3} & \nicefrac{-2}{3}
        \end{pmatrix},\hspace{1cm}
        B = \begin{pmatrix}
            \nicefrac{1}{2}\\
            0
        \end{pmatrix}
    \end{equation*}

    \item Resuelve el sistema en función del dato inicial y del parámetro $\alpha$.
    
    Como $M$ es diagonal, tenemos que sus valores propios son $\lm_1 = \nicefrac{-1}{2}$ y $\lm_2 = \nicefrac{-2}{3}$.
    Calculemos los vectores propios asociados:
    \begin{align*}
        V_{\lm_1} &= \left\{
            \begin{pmatrix}
                x \\ y
            \end{pmatrix} \in \bb{R}^2 \left|
            \begin{pmatrix}
                0 & 0\\
                \nicefrac{\alpha}{3} & \nicefrac{-2}{3}+\nicefrac{1}{2}
            \end{pmatrix}\begin{pmatrix}
                x \\ y
            \end{pmatrix} = \begin{pmatrix}
                0 \\ 0
            \end{pmatrix}
            \right.
        \right\}=\\
        &= \left\{
            \begin{pmatrix}
                x \\ y
            \end{pmatrix} \in \bb{R}^2 \left|
            \begin{pmatrix}
                0 & 0\\
                \nicefrac{\alpha}{3} & \nicefrac{-1}{6}
            \end{pmatrix}\begin{pmatrix}
                x \\ y
            \end{pmatrix} = \begin{pmatrix}
                0 \\ 0
            \end{pmatrix}
            \right.
        \right\}=\\
        &= \cc{L}\left(\left\{\begin{pmatrix}
                1 \\ 2\alpha
            \end{pmatrix}\right\}\right)
        \\ \\
        V_{\lm_2} &= \left\{
            \begin{pmatrix}
                x \\ y
            \end{pmatrix} \in \bb{R}^2 \left|
            \begin{pmatrix}
                \nicefrac{-1}{2}+\nicefrac{2}{3} & 0\\
                \nicefrac{\alpha}{3} & 0
            \end{pmatrix}\begin{pmatrix}
                x \\ y
            \end{pmatrix} = \begin{pmatrix}
                0 \\ 0
            \end{pmatrix}
            \right.
        \right\}=\\
        &= \left\{
            \begin{pmatrix}
                x \\ y
            \end{pmatrix} \in \bb{R}^2 \left|
            \begin{pmatrix}
                \nicefrac{1}{6} & 0\\
                \nicefrac{\alpha}{3} & 0
            \end{pmatrix}\begin{pmatrix}
                x \\ y
            \end{pmatrix} = \begin{pmatrix}
                0 \\ 0
            \end{pmatrix}
            \right.
        \right\}=\\
        &= \cc{L}\left(\left\{\begin{pmatrix}
                0 \\ 1
            \end{pmatrix}\right\}\right)
    \end{align*}

    Por tanto, tenemos que la solución de la parte homogénea es:
    \begin{align*}
        X_n^{(h)} &= M^nX_0
        = \begin{pmatrix}
            1 & 0\\
            2\alpha & 1
        \end{pmatrix}\begin{pmatrix}
            \left(\nicefrac{-1}{2}\right)^n & 0\\
            0 & \left(\nicefrac{-2}{3}\right)^n
        \end{pmatrix}
        \begin{pmatrix}
            1 & 0\\
            2\alpha & 1
        \end{pmatrix}^{-1}X_0 =\\
        &= \begin{pmatrix}
            \left(\nicefrac{-1}{2}\right)^n & 0\\
            2\alpha\left[\left(\nicefrac{-1}{2}\right)^n - \left(\nicefrac{-2}{3}\right)^n\right] & \left(\nicefrac{-2}{3}\right)^n
        \end{pmatrix}X_0 =\\
        &= \begin{pmatrix}
            \left(\nicefrac{-1}{2}\right)^n & 0\\
            2\alpha\cdot \frac{(-1)^n -(-3)^n}{2^n} & \left(\nicefrac{-2}{3}\right)^n
        \end{pmatrix}X_0
    \end{align*}

    Buscamos ahora una solución constante del sistema, $X^\ast$:
    \begin{equation*}
        X^\ast = MX^\ast + B
        \Longrightarrow
        (I-M)X^\ast = B
        \Longrightarrow
        X^\ast = (I-M)^{-1}B = \frac{1}{3}\begin{pmatrix}
            1 \\ \nicefrac{\alpha}{5}
        \end{pmatrix}
    \end{equation*}

    Por tanto, como la solución del sistema es una solución de la parte homogénea más
    una solución de la parte particular, tenemos que:
    \begin{equation*}
        X_n = X_n^{(h)} + X^\ast
        = \begin{pmatrix}
            \left(\nicefrac{-1}{2}\right)^n & 0\\
            2\alpha\cdot \frac{(-1)^n -(-3)^n}{2^n} & \left(\nicefrac{-2}{3}\right)^n
        \end{pmatrix}X_0 + \frac{1}{3}\begin{pmatrix}
            1 \\ \nicefrac{\alpha}{5}
        \end{pmatrix}
    \end{equation*}
\end{enumerate}
\end{ejercicio}

\begin{ejercicio}
Los siguientes modelos representan una población compuesta por dos especies en competición. Estudia en cada caso la estabilidad de los puntos fijos.
\begin{enumerate}
    \item El modelo es:
    \begin{equation*}
        \begin{cases}
            x_{n+1} = x_n(1.7 - 0.02x_n - 0.08y_n)\\
            y_{n+1} = y_n(1.5 - 0.03x_n - 0.04y_n)
        \end{cases}
    \end{equation*}

    Sean las funciones $f_1, f_2:\bb{R}^2\to\bb{R}$ dadas por:
    \begin{align*}
        f_1(x,y) &= x(1.7 - 0.02x - 0.08y)\\
        f_2(x,y) &= y(1.5 - 0.03x - 0.04y)
    \end{align*}

    Notando $F=(f_1,f_2)$, tenemos que el sistema viene dado por $X_{n+1} = F(X_n)$.
    Buscamos los puntos fijos del sistema, es decir, los puntos $(x,y)$ tales que $F(x,y) = (x,y)$:
    \begin{equation*}
        \left\{
            \begin{array}{lr}
                x(1.7 - 0.02x - 0.08y) &= x\\
                y(1.5 - 0.03x - 0.04y) &= y
            \end{array}
        \right\}
        \Longrightarrow
        \left\{
            \begin{array}{lr}
                x(0.7 - 0.02x - 0.08y) &=0\\
                y(0.5 - 0.03x - 0.04y) &=0
            \end{array}
        \right.
    \end{equation*}

    Hay distintas soluciones constantes del modelo, veámoslas:
    \begin{itemize}
        \item $x_1=0$, $y_1=0$. En este caso $X_1=(0,0)$.
        \item $x_2=0$, $y_2\neq 0$. Tenemos $0.5-0.04y_2=0\Longrightarrow y_2=12.5$.
        En este caso, $X_2=(0,12.5)$.
        \item $x_3\neq 0$, $y_3=0$. Tenemos $0.7-0.02x_3=0\Longrightarrow x_3=35$.
        En este caso, $X_3=(35,0)$.
        \item $x_4\neq 0$, $y_4\neq 0$. En este caso, tenemos el siguiente sistema:
        \begin{equation*}
            \left\{
                \begin{array}{lr}
                    0.02x + 0.08y &=0.7\\
                    0.03x + 0.04y &=0.5
                \end{array}
            \right\}
            \Longrightarrow
            \left\{
                \begin{array}{l}
                    x=7.5\\
                    y=6.875
                \end{array}
            \right.
        \end{equation*}
        Por tanto, $X_4=(7.5,6.875)$.
    \end{itemize}

    Estudiemos ahora la estabilidad de los puntos fijos. Para ello,
    calculamos la matriz jacobiana de $F$ en cada punto fijo y estudiamos
    su radio espectral. Sea $P=(x,y)\in \bb{R}^2$ un punto cualquiera. Tenemos que:
    \begin{equation*}
        JF(x,y) = \begin{pmatrix}
            \deld{f_1}{x}(P) & \deld{f_1}{y}(P)\\ \\
            \deld{f_2}{x}(P) & \deld{f_2}{y}(P)
        \end{pmatrix}
        = \begin{pmatrix}
            1.7-0.04x-0.08y & -0.08x\\
            -0.03y & 1.5-0.03x-0.08y
        \end{pmatrix}
    \end{equation*}

    Evaluando en cada punto, tenemos:
    \begin{align*}
        &JF(X_1) = \begin{pmatrix}
            1.7 & 0\\
            0 & 1.5
        \end{pmatrix},\hspace{1cm}
        JF(X_2) = \begin{pmatrix}
            0.7 & 0\\
            -0.375 & 0.5
        \end{pmatrix},\\
        &JF(X_3) = \begin{pmatrix}
            0.3 & -2.8\\
            0 & 0.45
        \end{pmatrix},\hspace{1cm}
        JF(X_4) = \begin{pmatrix}
            0.85 & -0.6\\
            -0.20625 & 0.725
        \end{pmatrix},
    \end{align*}

    Estudiamos ahora la estabilidad de cada punto fijo:
    \begin{itemize}
        \item $X_1=(0,0)$.
        
        Como $\rho(JF(X_1)) = 1.7>1$, tenemos que el punto fijo es inestable.

        \item $X_2=(0,12.5)$.
        
        Como $\rho(JF(X_2)) = 0.7<1$, tenemos que el punto fijo es asintóticamente estable localmente.

        \item $X_3=(35,0)$.
        
        Como $\rho(JF(X_3)) = 0.45<1$, tenemos que el punto fijo es asintóticamente estable localmente.

        \item $X_4=(7.5,6.875)$.
        
        Calculamos sus valores propios:
        \begin{equation*}
            p(\lm) = \lm^2-1.575\lm +0.4925
        \end{equation*}

        Tenemos que los valores propios son $\lm_1\approx 1.14$ y $\lm_2\approx 0.43$,
        por lo que $\rho(JF(X_4)) \approx 1.14>1$. Por tanto, el punto fijo es inestable.
    \end{itemize}

    \item El modelo es:
    \begin{equation*}
        \begin{cases}
            x_{n+1} = x_n(1.8 - 0.06x_n - 0.03y_n)\\
            y_{n+1} = y_n(1.9 - 0.02x_n - 0.04y_n)
        \end{cases}
    \end{equation*} 

    Sean las funciones $f_1, f_2:\bb{R}^2\to\bb{R}$ dadas por:
    \begin{align*}
        f_1(x,y) &= x(1.8 - 0.06x - 0.03y)\\
        f_2(x,y) &= y(1.9 - 0.02x - 0.04y)
    \end{align*}

    Notando $F=(f_1,f_2)$, tenemos que el sistema viene dado por $X_{n+1} = F(X_n)$.
    Buscamos los puntos fijos del sistema, es decir, los puntos $(x,y)$ tales que $F(x,y) = (x,y)$:
    \begin{equation*}
        \left\{
            \begin{array}{lr}
                x(1.8 - 0.06x - 0.03y) &= x\\
                y(1.9 - 0.02x - 0.04y) &= y
            \end{array}
        \right\}
        \Longrightarrow
        \left\{
            \begin{array}{lr}
                x(0.8 - 0.06x - 0.03y) &=0\\
                y(0.9 - 0.02x - 0.04y) &=0
            \end{array}
        \right.
    \end{equation*}

    Hay distintas soluciones constantes del modelo, veámoslas:
    \begin{itemize}
        \item $x_1=0$, $y_1=0$. En este caso $X_1=(0,0)$.
        \item $x_2=0$, $y_2\neq 0$. Tenemos $0.9-0.04y_2=0\Longrightarrow y_2=22.5$.
        En este caso, $X_2=(0,22.5)$.
        \item $x_3\neq 0$, $y_3=0$. Tenemos $0.8-0.06x_3=0\Longrightarrow x_3=\nicefrac{40}{3}$.
        En este caso, $X_3=(\nicefrac{40}{3},0)$.
        \item $x_4\neq 0$, $y_4\neq 0$. En este caso, tenemos el siguiente sistema:
        \begin{equation*}
            \left\{
                \begin{array}{lr}
                    0.06x + 0.03y &=0.8\\
                    0.02x + 0.04y &=0.9
                \end{array}
            \right\}
            \Longrightarrow
            \left\{
                \begin{array}{l}
                    x=\nicefrac{25}{9}\\
                    y=\nicefrac{190}{9}
                \end{array}
            \right.
        \end{equation*}
        Por tanto, $X_4=(\nicefrac{25}{9},\nicefrac{190}{9})$.
    \end{itemize}

    Estudiemos ahora la estabilidad de los puntos fijos. Para ello,
    calculamos la matriz jacobiana de $F$ en cada punto fijo y estudiamos
    su radio espectral. Sea $P=(x,y)\in \bb{R}^2$ un punto cualquiera. Tenemos que:
    \begin{equation*}
        JF(x,y) = \begin{pmatrix}
            \deld{f_1}{x}(P) & \deld{f_1}{y}(P)\\ \\
            \deld{f_2}{x}(P) & \deld{f_2}{y}(P)
        \end{pmatrix}
        = \begin{pmatrix}
            1.8-0.12x-0.03y & -0.03x\\
            -0.02y & 1.9-0.02x-0.08y
        \end{pmatrix}
    \end{equation*}

    Evaluando en cada punto, tenemos:
    \begin{align*}
        &JF(X_1) = \begin{pmatrix}
            1.8 & 0\\
            0 & 1.9
        \end{pmatrix},\hspace{1cm}
        JF(X_2) = \begin{pmatrix}
            1.125 & 0\\
            -0.45 & 0.1
        \end{pmatrix},\\
        &JF(X_3) = \begin{pmatrix}
            0.2 & -0.4\\
            0 & \nicefrac{49}{30}
        \end{pmatrix},\hspace{1cm}
        JF(X_4) = \begin{pmatrix}
            \nicefrac{5}{6} & \nicefrac{-1}{12} \\
            \nicefrac{-19}{45} & \nicefrac{7}{45}
        \end{pmatrix},
    \end{align*}

    Estudiamos ahora la estabilidad de cada punto fijo:
    \begin{itemize}
        \item $X_1=(0,0)$.
        
        Como $\rho(JF(X_1)) = 1.9>1$, tenemos que el punto fijo es inestable.

        \item $X_2=(0,22.5)$.
        
        Como $\rho(JF(X_2)) = 1.125>1$, tenemos que el punto fijo es inestable.

        \item $X_3=(\nicefrac{40}{3},0)$.
        
        Como $\rho(JF(X_3)) = \nicefrac{49}{30}>1$, tenemos que el punto fijo es inestable.

        \item $X_4=(\nicefrac{25}{9},\nicefrac{190}{9})$.
        
        Calculamos sus valores propios:
        \begin{equation*}
            p(\lm) = \lm^2-\frac{89}{90}\lm +\frac{137}{1350}
        \end{equation*}

        Tenemos que los valores propios son $\lm_1\approx 0.87$ y $\lm_2\approx 0.11$,
        por lo que $\rho(JF(X_4)) \approx 0.87<1$. Por tanto, el punto fijo es asintóticamente estable localmente.
    \end{itemize}
\end{enumerate}

\end{ejercicio}

\begin{ejercicio}
    Si en el modelo de crecimiento de una población estructurada por sexos no se asume una distribución equitativa entre hembras y machos, el modelo resultante es

    \begin{equation*}
        \begin{cases}
        x_{n+1} = x_n + \alpha_x x_n y_n - \mu_x x_n\\
        y_{n+1} = y_n + \alpha_y x_n y_n - \mu_y y_n
        \end{cases}
    \end{equation*}
    donde $x_n$ denota el número de hembras e $y_n$ el número de machos en el $n-$ésimo año. Los parámetros $\alpha_x$ y $\alpha_y$ representan la tasa de natalidad por pareja de hembras y machos, respectivamente, y $0 < \mu_x < 1$ y $0 < \mu_y < 1$ son las respectivas mortalidades.
    Dados $\alpha_x = 0.05$, $\alpha_y = 0.02$ y $\mu_x = \mu_y = 0.3$, estudia la estabilidad de los puntos de equilibrio.\\

    Sean las funciones $f_1, f_2:\bb{R}^2\to\bb{R}$ dadas por:
    \begin{align*}
        f_1(x,y) &= x + \alpha_x xy - \mu_x x\\
        f_2(x,y) &= y + \alpha_y xy - \mu_y y
    \end{align*}

    Notando $F=(f_1,f_2)$, tenemos que el sistema viene dado por $X_{n+1} = F(X_n)$.
    Buscamos los puntos fijos del sistema, es decir, los puntos $(x,y)$ tales que se tiene $F(x,y) = (x,y)$:
    \begin{equation*}
        \left\{
            \begin{array}{lr}
                x + \alpha_x xy - \mu_x x &= x\\
                y + \alpha_y xy - \mu_y y &= y
            \end{array}
        \right\}
        \Longrightarrow
        \left\{
            \begin{array}{lr}
                \alpha_x xy - \mu_x x &=0\\
                \alpha_y xy - \mu_y y &=0
            \end{array}
        \right.
    \end{equation*}

    Hay distintas soluciones constantes del modelo, veámoslas:
    \begin{itemize}
        \item $x_1=0$, $y_1=0$. En este caso $X_1=(0,0)$.
        \item $x_2=0$, $y_2\neq 0$. Se da un absurdo en la segunda ecuación, por lo que no se puede dar.
        \item $x_3\neq 0$, $y_3=0$. Se da un absurdo en la primera ecuación, por lo que no se puede dar.
        \item $x_4\neq 0$, $y_4\neq 0$. En este caso, tenemos el siguiente sistema:
        \begin{equation*}
            \left\{
                \begin{array}{lr}
                    \alpha_x xy - \mu_x x &=0\\
                    \alpha_y xy - \mu_y y &=0
                \end{array}
            \right\}
            \Longrightarrow
            \left\{
                \begin{array}{lr}
                    \alpha_x y - \mu_x &=0\\
                    \alpha_y x - \mu_y &=0
                \end{array}
            \right\}
            \Longrightarrow
            \left\{
                \begin{array}{l}
                    y=\nicefrac{\mu_x}{\alpha_x}\\
                    x=\nicefrac{\mu_y}{\alpha_y}
                \end{array}
            \right.
        \end{equation*}
        Por tanto, $X_4=\left(\dfrac{\mu_y}{\alpha_y},\dfrac{\mu_x}{\alpha_x}\right)$.
    \end{itemize}

    Estudiemos ahora la estabilidad de los puntos fijos. Para ello,
    calculamos la matriz jacobiana de $F$ en cada punto fijo y estudiamos
    su radio espectral. Sea $P=(x,y)\in \bb{R}^2$ un punto cualquiera. Tenemos que:
    \begin{equation*}
        JF(x,y) = \begin{pmatrix}
            \deld{f_1}{x}(P) & \deld{f_1}{y}(P)\\ \\
            \deld{f_2}{x}(P) & \deld{f_2}{y}(P)
        \end{pmatrix}
        = \begin{pmatrix}
            1 + \alpha_x y - \mu_x & \alpha_x x\\
            \alpha_y y & 1 + \alpha_y x - \mu_y
        \end{pmatrix}
    \end{equation*}

    Evaluando en cada punto, tenemos:
    \begin{align*}
        &JF(X_1) = \begin{pmatrix}
            1-\mu_x & 0\\
            0 & 1-\mu_y
        \end{pmatrix},\hspace{1cm}
        JF(X_4) = \begin{pmatrix}
            1 & \alpha_x\left(\dfrac{\mu_y}{\alpha_y}\right)\\
            \alpha_y\left(\dfrac{\mu_x}{\alpha_x}\right) & 1
        \end{pmatrix},
    \end{align*}

    Concretando para los valores dados en el enunciado, tenemos que:
    \begin{align*}
        &JF(X_1) = \begin{pmatrix}
            0.7 & 0\\
            0 & 0.7
        \end{pmatrix},\hspace{1cm}
        JF(X_4) = \begin{pmatrix}
            1 & 0.75\\
            0.12 & 1
        \end{pmatrix},
    \end{align*}

    Estudiamos ahora la estabilidad de cada punto fijo:
    \begin{itemize}
        \item $X_1=(0,0)$.
        
        Como $\rho(JF(X_1)) = 0.7<1$, tenemos que el punto fijo es asintóticamente estable localmente.

        \item $X_4=\left(\dfrac{\mu_y}{\alpha_y},\dfrac{\mu_x}{\alpha_x}\right)$.
        
        Calculamos sus valores propios:
        \begin{equation*}
            p(\lm) = \lm^2 - 2\lm +0.91
        \end{equation*}

        Tenemos que los valores propios son $\lm_1=1.3$ y $\lm_2=0.7$,
        por lo que $\rho(JF(X_4)) =1.3>1$. Por tanto, el punto fijo es inestable.
    \end{itemize}
\end{ejercicio}

\begin{ejercicio}
    Estudia la estabilidad de los puntos de equilibrio del siguiente modelo de presa-depredador:
    \begin{equation*}
        \begin{cases}
            x_{n+1} = 2x_n - x_n y_n\\
            y_{n+1} = 1.5y_n - 2y_n^2 + x_n y_n
        \end{cases}
    \end{equation*}

    Sean las funciones $f_1, f_2:\bb{R}^2\to\bb{R}$ dadas por:
    \begin{align*}
        f_1(x,y) &= 2x - xy\\
        f_2(x,y) &= 1.5y - 2y^2 + xy
    \end{align*}

    Notando $F=(f_1,f_2)$, tenemos que el sistema viene dado por $X_{n+1} = F(X_n)$.
    Buscamos los puntos fijos del sistema, es decir, los puntos $(x,y)$ tales que se tiene $F(x,y) = (x,y)$:
    \begin{equation*}
        \left\{
            \begin{array}{lr}
                2x - xy &= x\\
                1.5y - 2y^2 + xy &= y
            \end{array}
        \right\}
        \Longrightarrow
        \left\{
            \begin{array}{lr}
                x(1 - y) &=0\\
                y(0.5 - 2y + x) &=0
            \end{array}
        \right.
    \end{equation*}

    Hay distintas soluciones constantes del modelo, veámoslas:
    \begin{itemize}
        \item $x_1=0$, $y_1=0$. En este caso $X_1=(0,0)$.
        \item $x_2=0$, $y_2\neq 0$. Tenemos $0.5-2y_2=0\Longrightarrow y_2=0.25$. En este caso, se tiene $X_2=(0,0.25)$.
        \item $x_3\neq 0$, $y_3=0$. Se da un absurdo en la primera ecuación, por lo que no se puede dar.
        \item $x_4\neq 0$, $y_4\neq 0$. En este caso, tenemos el siguiente sistema:
        \begin{equation*}
            \left\{
                \begin{array}{lr}
                    1 - y &=0\\
                    0.5 - 2y + x &=0
                \end{array}
            \right\}
            \Longrightarrow
            \left\{
                \begin{array}{lr}
                    x &=1.5\\
                    y &=1
                \end{array}
            \right.
        \end{equation*}
        Por tanto, $X_4=(1.5,1)$.
    \end{itemize}

    Estudiemos ahora la estabilidad de los puntos fijos. Para ello,
    calculamos la matriz jacobiana de $F$ en cada punto fijo y estudiamos
    su radio espectral. Sea $P=(x,y)\in \bb{R}^2$ un punto cualquiera. Tenemos que:
    \begin{equation*}
        JF(x,y) = \begin{pmatrix}
            \deld{f_1}{x}(P) & \deld{f_1}{y}(P)\\ \\
            \deld{f_2}{x}(P) & \deld{f_2}{y}(P)
        \end{pmatrix}
        = \begin{pmatrix}
            2 - y & -x\\
            y & 1.5 - 4y + x
        \end{pmatrix}
    \end{equation*}

    Evaluando en cada punto, tenemos:
    \begin{gather*}
        F(X_1) = \begin{pmatrix}
            2 & 0\\
            0 & 1.5
        \end{pmatrix},\hspace{1cm}
        JF(X_2) = \begin{pmatrix}
            1.75 & 0\\
            0.25 & 0.5
        \end{pmatrix},\\
        JF(X_4) = \begin{pmatrix}
            1 & -1.5\\
            1 & -1
        \end{pmatrix}
    \end{gather*}

    Estudiamos ahora la estabilidad de cada punto fijo:
    \begin{itemize}
        \item $X_1=(0,0)$.
        
        Como $\rho(JF(X_1)) = 2>1$, tenemos que el punto fijo es inestable.

        \item $X_2=(0,0.25)$.
        
        Como $\rho(JF(X_2)) = 1.75>1$, tenemos que el punto fijo es inestable.

        \item $X_4=(1.5,1)$.
        
        Calculamos sus valores propios:
        \begin{equation*}
            p(\lm) = \lm^2 + 0.5
        \end{equation*}

        Tenemos que los valores propios son $\lm=\pm \dfrac{i}{\sqrt{2}}$, por lo que el radio espectral es $\rho(JF(X_4)) = \dfrac{1}{\sqrt{2}} < 1$. Por tanto, el punto fijo es asintóticamente estable localmente.
    \end{itemize}

    Notemos que tan solo sobrevive en el caso de que las dos especies convevivan.
\end{ejercicio}

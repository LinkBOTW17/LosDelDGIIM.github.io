\section{Ecuaciones en diferencias lineales de orden superior}

\begin{ejercicio}
    Calcula la solución de la ecuación $x_{n+2} - 3x_{n+1} + 2x_n = 0$, con condiciones iniciales $x_0 = 1$, $x_1 = 0$. Determina
    $\lim\limits_{n\to\infty} x_n$.\\

    El polinomio característico asociado a dicha recurrencia es:
    \begin{equation*}
        p(\lm) = \lm^2-3\lm+2 = (\lm-1)(\lm-2)
    \end{equation*}

    Por tanto, tenemos que la solución general de la recurrencia es:
    \begin{equation*}
        x_n = c_1\cdot 1^n + c_2\cdot 2^n = c_1+c_2\cdot 2^n
    \end{equation*}

    Usando las condiciones iniciales, tenemos que:
    \begin{equation*}
        \left\{
        \begin{array}{l}
            x_0=1=c_1+c_2\\
            x_1=0=c_1+2c_2
        \end{array}
        \right.
    \end{equation*}

    Restándole a la 2ª ecuación la 1ª, tenemos que $-1=c_2$, por lo que $c_1=2$. Entonces, la solución a la recurrencia dada es:
    \begin{equation*}
        x_n = 2- 2^n
    \end{equation*}

    Por tanto, tenemos que:
    \begin{equation*}
        \lim_{n\to \infty} x_n
        = \lim_{n\to \infty} 2- 2^n = -\infty
    \end{equation*}
\end{ejercicio}

\begin{ejercicio}
    Determina el espacio de soluciones de la ecuación $4x_{n+2} - x_n = 0$ que verifica $x_8 = 0$. ¿Qué dimensión tiene?\\

    El polinomio característico asociado a dicha recurrencia es:
    \begin{equation*}
        p(\lm) = 4\lm^2-1 = (2\lm-1)(2\lm-2)
        = 4\left(\lm-\frac{1}{2}\right)\left(\lm+\frac{1}{2}\right)
    \end{equation*}

    Por tanto, tenemos que la solución general de la recurrencia es:
    \begin{equation*}
        x_n = c_1\cdot \left(\frac{1}{2}\right)^n + c_2\cdot \left(-\frac{1}{2}\right)^n
        = \frac{c_1}{2^n} + c_2\cdot \left(-\frac{1}{2}\right)^n
    \end{equation*}

    Usando la condición inicial, tenemos que:
    \begin{equation*}
        x_8 = 0 = \frac{c_1}{2^8} + \frac{c_2}{2^8}
        = \frac{c_1+c_2}{2^8}\Longrightarrow c_2=-c_1
    \end{equation*}

    Por tanto, tenemos que las soluciones de la recurrencia son:
    \begin{align*}
        x_n &= \frac{c_1}{2^n} -c_1\cdot \left(-\frac{1}{2}\right)^n =\\
        &= c_1\left(\frac{1-(-1)^n}{2^n}\right) =\\
        &= c_1\left(\frac{1+(-1)^{n-1}}{2^n}\right)
    \end{align*}

    Tenemos por tanto que el espacio de soluciones de la ecuación $4x_{n+2} - x_n = 0$ que verifica $x_8 = 0$, al que notaremos por $V_8$, es:
    \begin{equation*}
        V_8 = \cc{L}\left\{\left\{\frac{1+(-1)^{n-1}}{2^n}\right\}\right\}
    \end{equation*}

    Como la base del espacio vectorial tiene un elemento, tenemos que su dimensión es $1$, $\dim V_8=1$.
\end{ejercicio}

\begin{ejercicio}
    ¿Pueden ser $x_n = 2^n$, $y_n = 3^n$ y $z_n = 4^n$ soluciones de una misma ecuación de la forma $a_2 x_{n+2} + a_1 x_{n+1} + a_0 x_n = 0$?
    Razona tu respuesta.\\

    Sabemos que $\{x_n,y_n,z_n\}$ son linealmente independientes. Notando por $V$ al espacio de las soluciones de la recurrencia dada, suponiendo que $\{x_n,y_n,z_n\}\in V$, tenemos que $\dim V\geq 3$. No obstante, sabemos que $\dim V=2$, por lo que llegamos a una contradicción.
    Tenemos por tanto que no es posible.
\end{ejercicio}

\begin{ejercicio}
    ¿Pueden ser $x_n = 5^n$ e $y_n = 1 + 2^n + 5^n$ soluciones de una misma ecuación de la forma $x_{n+2} + a_1 x_{n+1} + a_0 x_n = 0$?
    ¿Y de una ecuación de la forma $x_{n+2} + a_1 x_{n+1} + a_0 x_n = b_n$? Razona tus respuestas.\\

    Supongamos que $x_n$ e $y_n$ sean soluciones de una ecuación de la forma:
    \begin{equation}\label{eq:forma_ej_4_rel3}
        x_{n+2} + a_1 x_{n+1} + a_0 x_n = 0
    \end{equation}

    Entonces, como $y_n$ es solución, tendríamos que:
    \begin{align*}
        y_{n+2} + a_1 y_{n+1} + a_0 y_n &= (1+2^{n+2}+5^{n+2}) + a_1(1+2^{n+1}+5^{n+1}) + a_0 (1+2^{n}+5^{n}) \\
        &= (1+4\cdot 2^n + 25\cdot 5^n) + a_1(1+2\cdot 2^n + 5\cdot 5^n) + a_0 (1 + 2^n + 5^n) \\
        &= 0\qquad \forall n \in \bb{N}
    \end{align*}
    
    Análogamente, como $x_n$ es solución, tenemos:
    \begin{align*}
        x_{n+2} + a_1 x_{n+1} + a_0 x_n &= 5^{n+2} + a_15^{n+1} + a_05^n \\
        &= 25\cdot 5^n + 5a_15^n + a_0 5^n \\
        &= 0 \qquad \forall n \in \bb{N}
    \end{align*}
    
    Igualando los coeficientes de $2^n,~5^n$ y los términos independientes a $0$, tenemos:
    \begin{equation*}
        \left\{\begin{array}{l}
            a_1 + a_0 + 1 = 0 \\
            2a_1 + a_0 + 4 = 0 \\
            5a_1 + a_0 + 25 = 0 
        \end{array}\right.
    \end{equation*}

    Este sistema es incompatible, luego dichos $a_0$, $a_1$ no existen para que $x_n$ e $y_n$ sean solución de una ecuación de la forma~\ref{eq:forma_ej_4_rel3}.\\

    Supongamos ahora que $x_n$ e $y_n$ son soluciones de una ecuación de la forma:
    \begin{equation}\label{eq:forma_ej_4_rel3_2}
        x_{n+2} + a_1 x_{n+1} + a_0 x_n = b_n
    \end{equation}

    Entonces, como $y_n$ es solución, tenemos:
    \begin{align*}
        y_{n+2} + a_1 y_{n+1} + a_0 y_n &= (1+4\cdot 2^n + 25\cdot 5^n) + a_1(1+2\cdot 2^n + 5\cdot 5^n) + a_0 (1 + 2^n + 5^n) \\
        &= b_n \qquad \forall n \in \bb{N}
    \end{align*}
    
    Como $x_n$ es solución, tenemos:
    \begin{equation*}
        x_{n+2} + a_1 x_{n+1} + a_0 x_n = 25\cdot 5^n + 5a_15^n + a_0 5^n = b_n \qquad \forall n \in \bb{N}
    \end{equation*}

    De ambos resultados, partiendo de la trivialidad $b_n=b_n$, para todo $n\in \bb{N}$ se tiene:
    de donde tenemos que:
    \begin{align*}
        b_n &= b_n \\
        y_{n+2} + a_1y_{n+1} + a_0 y_n &= x_{n+2} + a_1x_{n+1} + a_0 x_n   \\
        (1+4\cdot 2^n + 25\cdot 5^n) + a_1(1+2\cdot 2^n + 5\cdot 5^n) +
         a_0 (1 + 2^n + 5^n) &= 25\cdot 5^n + 5a_15^n + a_0 5^n \\
        \cancel{5^n(25+5a_1 + a_0)} + 2^n (4+2a_1 + a_0)+
         a_1 + a_0 + 1 &= \cancel{5^n(25 + 5a_1 + a_0)} \\
        2^n (4+2a_1 + a_0) + a_1 + a_0 + 1 &= 0 \qquad \forall n \in \bb{N}
    \end{align*}

    La ecuación anterior nos da el siguiente sistema:
    \begin{equation*}
        \left\{\begin{array}{l}
            a_1 + a_0 + 1 = 0 \\
            2a_1 + a_0 + 4 = 0
        \end{array}\right\}
        \Longrightarrow
        \left\{\begin{array}{l}
            a_1 = -3\\
            a_0 = 2
        \end{array}\right.
    \end{equation*}

    Entonces, tenemos que el valor de $b_n$ de forma que $x_n$ y $y_n$ sean soluciones es:
    \begin{align*}
        b_n &= 25\cdot 5^n + 5(-3)5^n + 2\cdot 5^n \\
        &= 5^n ( 25 -15 + 2) \\
        &= 12\cdot 5^n \qquad \forall n \in \bb{N}
    \end{align*}

    Tenemos así que $x_n$ e $y_n$ son solución de una ecuación de la forma~\ref{eq:forma_ej_4_rel3_2}:
    \begin{equation*}
        x_{n+2} + -3x_{n+1} + 2 x_n = 12\cdot 5^n
    \end{equation*}
\end{ejercicio}

\begin{ejercicio}
    Una empresa fija el precio de su producto haciendo la media de los precios de los dos años anteriores. Si los precios
    de los dos primeros años son $p_0$ y $p_1$, proporciona la expresión del precio en función del año y calcula su valor a
    largo plazo. ¿Te parece razonable este modelo para fijar el precio del producto?\\

    Sea $p_n$ el precio en el año $n\in \bb{N}_0$. Tenemos que:
    \begin{equation*}
        p_{n+2} = \frac{1}{2}\left(p_{n+1} + p_{n}\right)
    \end{equation*}

    Su polinomio característico asociado es:
    \begin{equation*}
        p(\lm) = \lm^2-\frac{\lm}{2} - \frac{1}{2}
        = \left(x-1\right)\left(x+\frac{1}{2}\right)
    \end{equation*}

    Tenemos entonces que la solución general a la recurrencia es:
    \begin{equation*}
        p_n = c_1 + c_2\cdot \left(-\frac{1}{2}\right)^n
    \end{equation*}

    Calculemos ahora los valores de $c_1,~c_2$ en función de $p_0,~p_1$:
    \begin{align*}
        p_0 &= c_1+c_2\\
        p_1 &= c_1-\frac{c_2}{2}
    \end{align*}

    Restando ambas ecuaciones, obtenemos:
    \begin{multline*}
        p_0-p_1 = \frac{3}{2}c_2 \Longrightarrow c_2 = \frac{2}{3}\left(p_0-p_1\right)
        \Longrightarrow\\\Longrightarrow
        c_1 = p_0-c_2 = p_0-\frac{2}{3}\left(p_0-p_1\right)
        = \frac{3p_0-2p_0+2p_1}{3} = \frac{p_0+2p_1}{3}
    \end{multline*}

    Por tanto, tenemos que la solución a la recurrencia es:
    \begin{equation*}
        p_n = \frac{p_0+2p_1}{3} + \frac{2}{3}\left(p_0-p_1\right)\cdot \left(-\frac{1}{2}\right)^n
    \end{equation*}

    A largo, plazo, tenemos entonces que:
    \begin{equation*}
        \lim_{n\to \infty} p_n = \frac{p_0+2p_1}{3}
    \end{equation*}

    Notemos que, económicamente hablando, esto podría no tener sentido; ya que el precio a largo plazo tan solo depende de los valores iniciales de $p_0,~p_1$. Por ejemplo, en el caso de que la reputación de la empresa aumentase mucho, o que por ejemplo aumentase la inflación, los precios deberían aumentar.
\end{ejercicio}

\begin{ejercicio}
    Dado $\alpha\in \bb{R}$, encuentra una sucesión $\{x_n\}_{n\geq 0}$ que satisfaga:
    \begin{equation*}
        \left\{
            \begin{array}{rl}
                x_0&=1,\\
                x_1&=\alpha,\\
                x_{n+2}&= x_{n+1} + x_n,\qquad n\geq 0
            \end{array}
        \right.
    \end{equation*}

    Determina $\lim\limits_{n\to\infty} \dfrac{x_{n+1}}{x_n}$.\\

    El polinomio característico asociado a dicha recurrencia es:
    \begin{equation*}
        p(\lm) = \lm^2-\lm -1 = \left(\lm -\frac{1+\sqrt{5}}{2}\right)\left(\lm -\frac{1-\sqrt{5}}{2}\right)
    \end{equation*}

    Por tanto, tenemos que la solución general de la recurrencia es:
    \begin{equation*}
        x_n = c_1\cdot \left(\frac{1+\sqrt{5}}{2}\right)^n + c_2\cdot \left(\frac{1-\sqrt{5}}{2}\right)^n
    \end{equation*}

    Usando las condiciones iniciales, tenemos que:
    \begin{equation*}
        \left\{
        \begin{array}{l}
            x_0=1=c_1+c_2\\
            x_1=\alpha=c_1\cdot \left(\frac{1+\sqrt{5}}{2}\right) + c_2\cdot \left(\frac{1-\sqrt{5}}{2}\right)
        \end{array}
        \right.
    \end{equation*}

    Como $c_1=1-c_2$, tenemos:
    \begin{align*}
        \alpha&=(1-c_2)\cdot \left(\frac{1+\sqrt{5}}{2}\right) + c_2\cdot \left(\frac{1-\sqrt{5}}{2}\right)
        =\\&= \left(\frac{1+\sqrt{5}}{2}\right) + c_2\left(-\frac{1+\sqrt{5}}{2}+\frac{1-\sqrt{5}}{2}\right)
        =\\&= \left(\frac{1+\sqrt{5}}{2}\right) + c_2\left(\frac{1-\sqrt{5}-1-\sqrt{5}}{2}\right)
        =\\&= \left(\frac{1+\sqrt{5}}{2}\right) - \sqrt{5}c_2
    \end{align*}

    Por tanto, tenemos que:
    \begin{align*}
        c_2 &= -\frac{\alpha}{\sqrt{5}} + \frac{1+\sqrt{5}}{2\sqrt{5}}=\\
        &= -\frac{\sqrt{5}\alpha}{5} + \frac{\sqrt{5}+5}{10}
        = \frac{-(2\sqrt{5})\cdot \alpha + 5+\sqrt{5}}{10}\\
        c_1 &= 1-c_2 = \frac{(2\sqrt{5})\cdot \alpha +5-\sqrt{5}}{10}
    \end{align*}
    
    Entonces, la solución a la recurrencia dada es:
    \begin{equation*}
        x_n = \frac{(2\sqrt{5})\cdot \alpha +5-\sqrt{5}}{10}\cdot \left(\frac{1+\sqrt{5}}{2}\right)^n + \frac{-(2\sqrt{5})\cdot \alpha + 5+\sqrt{5}}{10}\cdot \left(\frac{1-\sqrt{5}}{2}\right)^n
    \end{equation*}

    Para calcular el límite pedido, usaremos la solución general, en función de $c_1,~c_2$:
    \begin{align*}
        \lim_{n\to \infty} \frac{x_{n+1}}{x_n}
        &= \lim_{n\to \infty} \dfrac{c_1\cdot \left(\frac{1+\sqrt{5}}{2}\right)^{n+1} + c_2\cdot \left(\frac{1-\sqrt{5}}{2}\right)^{n+1}}{c_1\cdot \left(\frac{1+\sqrt{5}}{2}\right)^n + c_2\cdot \left(\frac{1-\sqrt{5}}{2}\right)^n}
        =\\&= \lim_{n\to \infty} \dfrac{c_1\cdot \left(\frac{1+\sqrt{5}}{2}\right)^{n}\left(\frac{1+\sqrt{5}}{2}\right) + c_2\cdot \left(\frac{1-\sqrt{5}}{2}\right)^{n}\left(\frac{1-\sqrt{5}}{2}\right)}{c_1\cdot \left(\frac{1+\sqrt{5}}{2}\right)^n + c_2\cdot \left(\frac{1-\sqrt{5}}{2}\right)^n}
    \end{align*}

    Dividiendo numerador y denominador por $\left(\frac{1+\sqrt{5}}{2}\right)^n$, obtenemos:
    \begin{align*}
        \lim_{n\to \infty} \frac{x_{n+1}}{x_n}
        &= \lim_{n\to \infty} \dfrac{c_1\cdot \left(\frac{1+\sqrt{5}}{2}\right) + c_2\cdot \left(\frac{1-\sqrt{5}}{1+\sqrt{5}}\right)^{n}\left(\frac{1-\sqrt{5}}{2}\right)}{c_1 + c_2\cdot \left(\frac{1-\sqrt{5}}{1+\sqrt{5}}\right)^n}
        =\\&= \dfrac{c_1\cdot \left(\frac{1+\sqrt{5}}{2}\right)}{c_1}
        = \frac{1+\sqrt{5}}{2}
    \end{align*}
    donde he usado que $1+\sqrt{5} > 1-\sqrt{5}$, por lo que el límite de ese sumando es $0$.
    Notemos que este límite es bien conocido, y es el número áureo, $\varphi = \frac{1+\sqrt{5}}{2}$.
\end{ejercicio}

\begin{ejercicio}
    Determina, en función de $a\in \bb{R}$, el valor del determinante tridiagonal siguiente:
    \begin{equation*}
        D_n = \left|
            \begin{array}{ccccc}
                a & 1 & 0 & \cdots & 0\\
                1 & a & 1 & \cdots & 0\\
                0 & 1 & a & \cdots & 0\\
                \vdots & \vdots & \vdots & \ddots & \vdots\\
                0 & 0 & 0 & \cdots & a
            \end{array}
        \right| = |M_n|
    \end{equation*}

    Para ello, desarrolla por elementos de una línea con el fin de obtener una ecuación de orden $2$, y tenga en cuenta que el determinante indicado es el asociado a una matriz $M_n\in \cc{M}_n(\bb{R})$.\\

    Tenemos que:
    \begin{align*}
        D_n &= \left|
            \begin{array}{ccccc}
                a & 1 & 0 & \cdots & 0\\
                1 & a & 1 & \cdots & 0\\
                0 & 1 & a & \cdots & 0\\
                \vdots & \vdots & \vdots & \ddots & \vdots\\
                0 & 0 & 0 & \cdots & a
            \end{array}
        \right|
        = a\left|
            \begin{array}{cccc}
                a & 1 & \cdots & 0\\
                1 & a & \cdots & 0\\
                \vdots & \vdots & \ddots & \vdots\\
                0 & 0 & \cdots & a
            \end{array}
        \right|-\left|
            \begin{array}{cccc}
                1 & 0 & \cdots & 0\\
                1 & a & \cdots & 0\\
                \vdots & \vdots & \ddots & \vdots\\
                0 & 0 & \cdots & a
            \end{array}
        \right| =\\&
        = a\left|
            \begin{array}{cccc}
                a & 1 & \cdots & 0\\
                1 & a & \cdots & 0\\
                \vdots & \vdots & \ddots & \vdots\\
                0 & 0 & \cdots & a
            \end{array}
        \right|-\left|
            \begin{array}{ccc}
                a & \cdots & 0\\
                \vdots & \ddots & \vdots\\
                0 & \cdots & a
            \end{array}
        \right| =\\&
        = aD_{n-1} -D_{n-2}
    \end{align*}

    Calculemos además algunos valores iniciales, para obtener solución única. Tenemos que:
    \begin{align*}
        D_1 &= |a| = a\\
        D_2 &= a^2-1
    \end{align*}

    Por tanto, hemos de resolver el siguiente PVI:
    \begin{equation*}
        \text{(PVI)}\equiv \left\{ \begin{array}{l}
            D_{n+2}=aD{n+1}-D_n\\
            D_1=a\in \bb{R}\\
            D_2=a^2-1
        \end{array}\right.
    \end{equation*}
    
    Su polinomio característico asociado es:
    \begin{equation*}
        p(\lm) = \lm^2-a\lm +1
    \end{equation*}

    Las raíces del polinomio característico son:
    \begin{equation*}
        \lm = \frac{a\pm \sqrt{a^2-4}}{2}
    \end{equation*}

    Distinguimos en función del valor de $a$:
    \begin{itemize}
        \item \ul{Si $|a|\neq 2$}:

        Entonces, $a^2!=4$, por lo que tenemos dos raíces, por lo que:
        \begin{equation*}
            D_n = c_1\left(\frac{a +\sqrt{a^2-4}}{2}\right)^n
            + c_2\left(\frac{a -\sqrt{a^2-4}}{2}\right)^n
        \end{equation*}

        Usando las condiciones iniciales, tenemos que:
        \begin{align*}
            D_1 &= a = c_1\left(\frac{a +\sqrt{a^2-4}}{2}\right)
            + c_2\left(\frac{a -\sqrt{a^2-4}}{2}\right) =\\
            &=\frac{a(c_1+c_2) +\sqrt{a^2-4}(c_1-c_2)}{2} \\
            D_2 &= a^2-1 = c_1\left(\frac{a +\sqrt{a^2-4}}{2}\right)^2
            + c_2\left(\frac{a -\sqrt{a^2-4}}{2}\right)^2 =\\
            &= \frac{c_1(a^2+a^2-4+2a\sqrt{a^2-4}) +c_2(a^2+a^2-4-2a\sqrt{a^2-4})}{4}=\\
            &= \frac{c_1(2a^2-4+2a\sqrt{a^2-4}) +c_2(2a^2-4-2a\sqrt{a^2-4})}{4}=\\
            &= \frac{c_1(a^2-2+a\sqrt{a^2-4}) +c_2(a^2-2-a\sqrt{a^2-4})}{2}=\\
            &= \frac{(a^2-2)(c_1+c_2)+a\sqrt{a^2-4}(c_1-c_2)}{2}
        \end{align*}

        Para facilitar la resolución del sistema, realizamos los cambios de variable $x=c_1+c_2$ e $y=c_1-c_2$, obteniendo el siguiente sistema:
        \begin{equation*}
            \left\{
                \begin{array}{rl}
                    2a &= ax +\sqrt{a^2-4}y\\
                    2a^2-2 &= (a^2-2)x + a\sqrt{a^2-4}y
                \end{array}
            \right.
        \end{equation*}

        Tenemos entonces que $\sqrt{a^2-4}y=2a-ax = a(2-x)$, por lo que:
        \begin{gather*}
            2a^2 -2 = (a^2-2)x + a^2(2-x) \Longrightarrow
            -2 = (a^2-2-a^2)x = -2x \Longrightarrow x=1\\ \\
            \sqrt{a^2-4}y = a\Longrightarrow y=\frac{a\sqrt{a^2-4}}{a^2-4}
        \end{gather*}

        Para deshacer el cambio de variable, nos queda que:
        \begin{equation*}
            \left\{
                \begin{array}{rl}
                    c_1+c_2=1\\
                    c_1-c_2=\frac{a\sqrt{a^2-4}}{a^2-4}
                \end{array}
            \right.
        \end{equation*}

        Sumando y restando respectivamente, obtenemos:
        \begin{align*}
            c_1 &= \dfrac{1+\frac{a\sqrt{a^2-4}}{a^2-4}}{2}
            = \frac{1}{2} + \frac{a\sqrt{a^2-4}}{2(a^2-4)}\\
            c_2 &= \dfrac{1-\frac{a\sqrt{a^2-4}}{a^2-4}}{2}
            = \frac{1}{2} - \frac{a\sqrt{a^2-4}}{2(a^2-4)}
        \end{align*}
        

        \item \ul{Si $a=2$}:

        Entonces, $a^2=4$, por lo que tenemos una raíz real con multiplicidad doble, $\lm = 1$.
        Tenemos que:
        \begin{equation*}
            D_n = c_1+nc_2
        \end{equation*}

        Usando las condiciones iniciales, tenemos que:
        \begin{align*}
            D_1 &= 2 = c_1+c_2\\
            D_2 &= 3 = c_1+2c_2\\
        \end{align*}

        Restándole la 1ª a la 2ª, tenemos que:
        \begin{equation*}
            1 = c_2 \Longrightarrow c_1=1
        \end{equation*}

        Por tanto, tenemos que:
        \begin{equation*}
            D_n = 1+n
        \end{equation*}

        \item \ul{Si $a=-2$}:

        Entonces, $a^2=4$, por lo que tenemos una raíz real con multiplicidad doble, $\lm = -1$.
        Tenemos que:
        \begin{equation*}
            D_n = (c_1+nc_2)(-1)^n
        \end{equation*}

        Usando las condiciones iniciales, tenemos que:
        \begin{align*}
            D_1 &= 2 = c_1+c_2\\
            D_2 &= 3 = c_1+2c_2\\
        \end{align*}

        Restándole la 1ª a la 2ª, tenemos que:
        \begin{equation*}
            1 = c_2 \Longrightarrow c_1=1
        \end{equation*}

        Por tanto, tenemos que:
        \begin{equation*}
            D_n = (1+n)(-1)^n
        \end{equation*}

    \end{itemize}
\end{ejercicio}

\begin{ejercicio}
    Encuentra las ecuaciones en diferencias homogéneas y reales de orden mínimo que tienen por solución las siguientes
    expresiones:
    \begin{enumerate}
        \item $2^{n-1} - 5^{n+1}$.

        Tenemos que:
        \begin{equation*}
            2^{n-1} - 5^{n+1} = \frac{1}{2}\cdot 2^n -5\cdot 5^n
        \end{equation*}

        Por tanto, el polinomio característico asociado a dicha recurrencia es:
        \begin{equation*}
            p(\lm) = (\lm-2)(\lm-5) = \lm^2-7\lm +10
        \end{equation*}

        Por tanto, la recurrencia buscada es $x_{n+2}-7x_{n+1}+10x_n=0$. Calculemos ahora los valores de $x_0,x_1$ que hacen única la solución:
        \begin{align*}
            x_0 &= \frac{1}{2} -5 = -\frac{9}{2}\\
            x_1 &= 1-5^2 = 1-25=-24
        \end{align*}

        Por tanto, el PVI dado es:
        \begin{equation*}
            \text{(PVI)}\equiv \left\{ \begin{array}{l}
                x_{n+2}-7x_{n+1}+10x_n=0,\\
                x_0=\nicefrac{-9}{5},\\
                x_1=-24
            \end{array}\right.
        \end{equation*}

        
        \item $3\cos\left(\frac{n\pi}{2}\right) - \sen\left(\frac{n\pi}{2}\right)$.

        Tenemos que $e^{i\cdot \frac{\pi}{2}}=i$ es una solución del polinomio característico, por lo que este es:
        \begin{equation*}
            p(\lm) = (\lm+i)(\lm-i) = \lm^2-i^2 = \lm^2+1
        \end{equation*}

        Por tanto, la recurrencia buscada es $x_{n+2}+x_n=0$. Calculemos ahora los valores de $x_0,x_1$ que hacen única la solución:
        \begin{align*}
            x_0 &= 3\cos 0 = 3\\
            x_1 &= -\sen\nicefrac{\pi}{2} = -1
        \end{align*}

        Por tanto, el PVI dado es:
        \begin{equation*}
            \text{(PVI)}\equiv \left\{ \begin{array}{l}
                x_{n+2}+x_n=0,\\
                x_0=3,\\
                x_1=-1
            \end{array}\right.
        \end{equation*}
        
        \item $(n+2)5^n\sen\left(\frac{n\pi}{4}\right)$.

        Tenemos que $\lm_1=5e^{i\cdot \frac{\pi}{4}}$ es una solución del polinomio característico con multiplicidad doble, por lo que este es:
        \begin{align*}
            p(\lm) &= [(\lm-\lm_1)(\lm-\ol{\lm_1})]^2 = (\lm^2-(\lm_1+\ol{\lm_1})\lm + \lm_1\cdot \ol{\lm_1})^2 =\\
            &= (\lm^2-2\Re(\lm_i)\lm + |\lm_i|^2)^2
            = (\lm^2-2\cdot 5\cdot \cos \nicefrac{\pi}{4}\cdot \lm + 5^2)^2 =\\
            &= (\lm^2-5\sqrt{2}\lm + 25)^2 =\\
            &=\lm^4 + 25\cdot 2\cdot \lm^2 + 25^2 -10\sqrt{2}\lm^3 +50\lm^2 -5^3\cdot 2\sqrt{2}\lm =\\
            &= \lm^4 -10\sqrt{2}\lm^3 +100\lm^2-250\sqrt{2} \lm +625
        \end{align*}

        Por tanto, la recurrencia buscada es:
        \begin{equation*}
            x_{n+4} -10\sqrt{2}x_{n+3} +100x_{n+2}-250\sqrt{2}x_{n+1} +625x_n=0
        \end{equation*}
        
        Calculemos ahora los valores de $x_0,x_1,x_2,x_3$ que hacen única la solución:
        \begin{align*}
            x_0 &= 2\cdot \sen 0 = 0 \\
            x_1 &= 15\cdot \sen \nicefrac{\pi}{4} = \frac{15\sqrt{2}}{2}\\
            x_2 &= 4\cdot 25 \sen \nicefrac{\pi}{2} = 100\\
            x_3 &= 5^4\sen \nicefrac{3\pi}{4} = \frac{625\sqrt{2}}{2}
        \end{align*}

        Por tanto, el PVI dado es:
        \begin{equation*}
            \text{(PVI)}\equiv \left\{ \begin{array}{l}
                x_{n+4} -10\sqrt{2}x_{n+3} +100x_{n+2}-250\sqrt{2}x_{n+1} +625x_n=0,\\
                x_0=0,\\
                x_1=\frac{15\sqrt{2}}{2},\\
                x_2=100,\\
                x_3=\frac{625\sqrt{2}}{2}
            \end{array}\right.
        \end{equation*}
        
        \item $(1+\sqrt{2}n+n^2)7^n$.

        Tenemos que $\lm_1=7$ es una solución del polinomio característico con multiplicidad triple, por lo que este es:
        \begin{align*}
            p(\lm) &= (\lm-7)^3 =\\
            &= \lm^3 -3\cdot 7\lm^2 + 3\cdot 7^2\lm -7^3=\\
            &= \lm^3 -21\lm^2 + 147 \lm -343
        \end{align*}

        Por tanto, la recurrencia buscada es:
        \begin{equation*}
            x_{n+3} -21x_{n+2}+147x_{n+1} -343x_n=0
        \end{equation*}
        
        Calculemos ahora los valores de $x_0,x_1,x_2$ que hacen única la solución:
        \begin{align*}
            x_0 &= 1 \\
            x_1 &= 7(2+\sqrt{2})\\
            x_2 &= 49(5+2\sqrt{2})
        \end{align*}

        Por tanto, el PVI dado es:
        \begin{equation*}
            \text{(PVI)}\equiv \left\{ \begin{array}{l}
                x_{n+3} -21x_{n+2}+147x_{n+1} -343x_n,\\
                x_0=1,\\
                x_1=7(2+\sqrt{2}),\\
                x_2=49(5+2\sqrt{2})
            \end{array}\right.
        \end{equation*}
        
        \item $1+3n-5n^2+6n^3$.

        Tenemos que $\lm_1=1$ es una solución del polinomio característico con multiplicidad cuádruple, por lo que este es:
        \begin{align*}
            p(\lm) &= (\lm-1)^4 =\\
            &= \lm^4-4\lm^3 +6\lm^2-4\lm +1
        \end{align*}

        Por tanto, la recurrencia buscada es:
        \begin{equation*}
            x_{n+4} -4x_{n+3} +6x_{n+2}-4x_{n+1} +x_n=0
        \end{equation*}
        
        Calculemos ahora los valores de $x_0,x_1,x_2,x_3$ que hacen única la solución:
        \begin{align*}
            x_0 &= 1 \\
            x_1 &= 1+3-5+6=5\\
            x_2 &= 1+6-20+48 = 35\\
            x_3 &= 1+9-45+27\cdot 6 = 127
        \end{align*}

        Por tanto, el PVI dado es:
        \begin{equation*}
            \text{(PVI)}\equiv \left\{ \begin{array}{l}
                x_{n+4} -4x_{n+3} +6x_{n+2}-4x_{n+1} +x_n,\\
                x_0=1,\\
                x_1=5,\\
                x_2=35,\\
                x_3=127
            \end{array}\right.
        \end{equation*}
    \end{enumerate}
\end{ejercicio}

\begin{ejercicio}
    Determina el valor de $\sum\limits_{n=0}^{\infty} x_n$, donde
    \begin{equation*}
        2x_{n+2} - x_{n+1} + x_n = 0, \qquad x_0 = x_1 = 1.
    \end{equation*}

    Tenemos que:
    \begin{equation*}
        \sum\limits_{n=0}^{\infty} x_n = x_0+x_1 + \sum\limits_{n=2}^{\infty}x_n
        = x_0+x_1 + \sum\limits_{n=0}^{\infty}x_{n+2}
        = x_0+x_1 + \frac{1}{2}\sum\limits_{n=0}^{\infty}\left(x_{n+1} - x_n\right)
    \end{equation*}

    Demostremos en primer lugar por inducción que, fijado $n\in \bb{N}$, se tiene que:
    \begin{equation*}
        \sum_{k=0}^n \left(x_{k+1} - x_k\right) = x_{n+1} - x_0
    \end{equation*}
    \begin{itemize}
        \item \ul{Para $n=0$}:
        \begin{equation*}
            \sum_{k=0}^0 \left(x_{k+1} - x_k\right) = x_1 - x_0
        \end{equation*}

        \item \ul{Supuesto cierto para $n$, demostramos para $n+1$}:
        \begin{align*}
            \sum_{k=0}^{n+1} \left(x_{k+1} - x_k\right) &= 
            \sum_{k=0}^{n} \left(x_{k+1} - x_k\right) + x_{n+2} - x_{n+1}
            \AstIg x_{n+1} - x_0+ x_{n+2} - x_{n+1} =\\&= x_{n+2} - x_0
        \end{align*}
        donde en $(\ast)$ hemos empleado la hipótesis de inducción.
    \end{itemize}

    Por tanto, tenemos que:
    \begin{align*}
        \sum\limits_{n=0}^{\infty} x_n
        &= x_0+x_1 + \frac{1}{2}\sum\limits_{n=0}^{\infty}\left(x_{n+1} - x_n\right) =\\
        &=  x_0+x_1 + \frac{1}{2} \cdot \lim_{n\to \infty}\sum_{k=0}^n \left(x_{k+1} - x_k\right) =\\
        &=  x_0+x_1 + \frac{1}{2} \cdot \lim_{n\to \infty}\left(x_{n+1} - x_0\right) =\\
        &=  x_0+x_1 + \frac{1}{2} \cdot \left(\lim_{n\to \infty}x_{n} - x_0\right)
    \end{align*}

    Por tanto, tan solo necesitamos calcular $\lim_{n\to \infty} x_n$. Resolvamos la recurrencia dada, cuyo polinomio característico es:
    \begin{equation*}
        p(\lm) = \lm^2 - \frac{1}{2}\lm +\frac{1}{2} = \left(\lm -\frac{1-\sqrt{7}i}{4}\right)\left(\lm -\frac{1+\sqrt{7}i}{4}\right)
    \end{equation*}

    Vemos que tiene dos soluciones complejas, que son:
    \begin{equation*}
        \lm_1 = \frac{1+\sqrt{7}i}{4},\hspace{2cm} \lm_2=\ol{\lm_1}
    \end{equation*}

    Calculemos su módulo y su fase:
    \begin{align*}
        |\lm_1| &= \sqrt{\frac{1+7}{16}} = \frac{1}{\sqrt{2}} = \frac{\sqrt{2}}{2}\\
        \theta_1 &= \arctan \sqrt{7} \approx 1.209
    \end{align*}

    Por tanto, la solución de la recurrencia es:
    \begin{equation*}
        x_n = \left(\frac{\sqrt{2}}{2}\right)^n\left(k_1\cos(n\theta_1) + k_2\sen(n\theta_2)\right)
    \end{equation*}

    Como $0\leq \frac{\sqrt{2}}{2}<1$, tenemos que $\{x_n\}\to 0$, por lo que:
    \begin{align*}
        \sum\limits_{n=0}^{\infty} x_n
        &=  x_0+x_1 + \frac{1}{2} \cdot \left(\lim_{n\to \infty}x_{n} - x_0\right)=\\
        &= x_0+x_1-\frac{x_0}{2} = x_1+\frac{x_0}{2} = \frac{3}{2}
    \end{align*}
\end{ejercicio}

\begin{ejercicio}
    Sea una ecuación del tipo:
    \begin{equation*}
        a_2 x_{n+2} + a_1 x_{n+1} + a_0 x_n = b_0,
    \end{equation*}
    donde $a_2 + a_1 + a_0 \neq 0$.
    \begin{enumerate}
        \item Demuestra que existe una única solución constante y calcúlala.

        Sea $x_c=\{x_c\}$ una solución constante. Entonces, tenemos que:
        \begin{equation*}
            a_2x_c+a_1x_c+a_0x_c=b_0 \Longleftrightarrow
            x_c = \dfrac{b_0}{a_2+a_1+a_0}
        \end{equation*}

        Por tanto, tenemos que la solución constante es única.
        
        \item Demuestra que si las raíces de la ecuación característica tienen módulo menor que uno, entonces todas las soluciones tienden a la solución constante.

        Se ha demostrado en la Proposición~\ref{prop:converger_0} que toda solución de la parte homogénea tenderá a $0$. Notando por $x_n^{h}$ a una solución de la parte homogénea y por $x_n^{p}$ a una solución particular de la ecuación recurrencia, tenemos que:
        \begin{equation*}
            x_n = x_n^{h} + x_n^{p}
            \Longrightarrow
            \lim_{n\to \infty}x_n = 
            \lim_{n\to \infty}x_n^{h} + \lim_{n\to \infty}x_n^{p}
            = \lim_{n\to \infty}x_n^{p}
        \end{equation*}

        Tomando como solución particular la constante, $x_n^p=x_c$, tenemos que:
        \begin{equation*}
            \lim_{n\to \infty}x_n = \lim_{n\to \infty}x_n^{p}
            = \lim_{n\to \infty}x_c = x_c
        \end{equation*}
    \end{enumerate}
\end{ejercicio}

\begin{ejercicio}
    Se considera la ecuación en diferencias:
    \begin{equation*}
        x_{n+2} - \beta x_{n+1} + \beta x_n = 1, \qquad \beta\in \bb{R}^+
    \end{equation*}
    \begin{enumerate}
        \item Calcula la solución constante.

        Sea $x_c=\{x_c\}$ una solución constante. Entonces, tenemos que:
        \begin{equation*}
            x_c - \beta x_c + \beta x_c = 1 \Longrightarrow x_c=1
        \end{equation*}
        
        \item Proporciona condiciones sobre $\beta$ para que las soluciones de la ecuación converjan a la solución de equilibrio.\\

        Sea el polinomio característico asociado a la recurrencia el siguiente:
        \begin{equation*}
            p(\lm) = \lm^2-\beta\lm + \beta
        \end{equation*}

        Por lo visto en el ejercicio anterior, que las soluciones de la ecuación converjan a la solución de equilibrio equivale a que las soluciones de la parte homogénea converjan a $0$.
        Por el Corolario del Lema~\ref{lema:raices_orden2}, tenemos que las condiciones a imponer son:
        \begin{align*}
            p(0)<1 &\Longleftrightarrow  \beta < 1\\
            p(1)>0 &\Longleftrightarrow 1-\beta + \beta > 0\\
            p(-1)>0 &\Longleftrightarrow 1+2\beta > 0 \Longleftrightarrow \beta > -\frac{1}{2}
        \end{align*}

        Por tanto, tan solo es necesario que $\beta\in ]\nicefrac{-1}{2},1[$. Como $\beta\in \bb{R}^+$, tan solo hay que imponer que:
        \begin{equation*}
            \beta \in~]0,1[
        \end{equation*}
    \end{enumerate}
\end{ejercicio}

\begin{ejercicio}
    Se consideran las funciones de oferta y demanda:
    \begin{equation*}
        O(p) = a + b p, \qquad D(p) = c - d p.
    \end{equation*}
    Se modifica el modelo de la telaraña de acuerdo a la ley (Goodwin, 1941):
    \begin{equation*}
        O(p_n^e) = D(p_n),
    \end{equation*}
    donde $p_n^e$ es el precio esperado para el año $n$:
    \begin{equation*}
        p_n^e = p_{n-1} + \rho(p_{n-1} - p_{n-2}),
    \end{equation*}
    y $\rho\in \bb{R}^+$ un parámetro (si $\rho = 0$ se vuelve al modelo de la telaraña).
    \begin{enumerate}
        \item Demuestra que $p_n$ cumple una ecuación del tipo:
        \begin{equation*}
            p_{n+2} + a_1 p_{n+1} + a_0 p_n = k
        \end{equation*}
        que tiene como solución constante el precio de equilibrio.\\

        Tenemos que:
        \begin{align*}
            O(p_n^e) &= a+bp_n^e = a+b\left(p_{n-1} + \rho(p_{n-1} - p_{n-2})\right)
            =a+b(1+\rho)p_{n-1} -b\rho p_{n-2} \\
            D(p_n) &= c-dp_n
        \end{align*}

        Igualando de acuerdo con la Ley de Goodwin, tenemos:
        \begin{align*}
            O(p_n^e)=D(p_n)\Longleftrightarrow a+b(1+\rho)p_{n-1} -b\rho\cdot p_{n-2} = c-dp_n
        \end{align*}

        Ajustando los índices y operando, llegamos a que:
        \begin{equation*}
            p_{n+2} +\frac{b}{d}(1+\rho)p_{n+1}-\frac{b}{d}\cdot \rho\cdot p_{n} = \frac{c-a}{d}
        \end{equation*}

        Identificando términos, es directo ver que cumple una ecuación del tipo del enunciado. Busquemos la solución constante $p_e$:
        \begin{equation*}
            p_{e} +\frac{b}{d}(1+\rho)p_{e}-\frac{b}{d}\cdot \rho\cdot p_{e} = \frac{c-a}{d}
            \Longleftrightarrow
            p_e = \dfrac{c-a}{d\left[1+\frac{b}{d}(1+\rho)-\frac{b}{d}\cdot \rho\right]}
            = \dfrac{c-a}{d+b}
        \end{equation*}

        Efectivamente, tenemos que se trata del precio de equilibrio visto en el Modelo de la Telaraña.
        
        \item Se supone $b = d = 1$. Calcula las soluciones y describe el comportamiento de los precios a largo plazo. ¿Son
        las predicciones idénticas a las que produciría el modelo simple de la telaraña?\\

        Tenemos que la ecuación de recurrencia queda:
        \begin{equation*}
            p_{n+2} +(1+\rho)p_{n+1}-\rho\cdot p_{n} = c-a
        \end{equation*}

        El polinomio característico queda:
        \begin{equation*}
            p(\lm) = \lm^2 +(1+\rho)\lm-\rho
        \end{equation*}

        Las raíces del polinomio característico son:
        \begin{equation*}
            \lm = \frac{-1-\rho \pm \sqrt{(1+\rho)^2 +4\rho}}{2}
            = \frac{-1-\rho \pm \sqrt{\rho^2+6\rho+1}}{2}
        \end{equation*}

        Veamos en primer lugar el signo de las raíces:
        \begin{align*}
             \lm_1 = \frac{-1-\rho + \sqrt{(1+\rho)^2 +4\rho}}{2} > 0
             &\Longleftrightarrow
             (1+\rho)^2 +4\rho > 1+\rho \\
             \lm_2 = \frac{-1-\rho - \sqrt{(1+\rho)^2 +4\rho}}{2} < 0
             &\Longleftrightarrow -\sqrt{(1+\rho)^2 +4\rho} < 1+\rho
        \end{align*}

        Veamos ahora el intervalo en el que está cada raíz:
        \begin{align*}
             \lm_1 = \frac{-1-\rho + \sqrt{(1+\rho)^2 +4\rho}}{2} < 1
             &\Longleftrightarrow
             -1-\rho + \sqrt{(1+\rho)^2 +4\rho} < 2
             \Longleftrightarrow \\ &\Longleftrightarrow
             (1+\rho)^2 +4\rho < (3+\rho)^2
             \Longleftrightarrow \\ &\Longleftrightarrow
             1 + \bcancel{\rho^2} +\cancel{6\rho}  < 9 + \bcancel{\rho^2} +\cancel{6\rho} \\
             \lm_2 = \frac{-1-\rho - \sqrt{(1+\rho)^2 +4\rho}}{2} < -1
             &\Longleftrightarrow
             -1-\rho - \sqrt{(1+\rho)^2 +4\rho} < -2
             \Longleftrightarrow \\ &\Longleftrightarrow
             (1-\rho)^2 < (1+\rho)^2 +4\rho
             \Longleftrightarrow \\ &\Longleftrightarrow
             \cancel{1} + \bcancel{\rho^2} -2\rho  < \cancel{1} + \bcancel{\rho^2} +{6\rho}
             \Longleftrightarrow \\ &\Longleftrightarrow
             -2<6
        \end{align*}

        Por tanto, ya podemos estudiar el comportamiento a largo plazo. Tenemos que:
        \begin{equation*}
            p_n = c_1 \lm_1^n + c_2\lm_2^n + c-a
        \end{equation*}
        donde he empleado que $p_e=c-a$ es una solución particular. Como $\lm_1\in~]0,1[$, tenemos que $\{\lm_1^n\}\to 0$. No obstante, como $\lm_2<-1$, tenemos que $\{|\lm_2^n|\}\to \infty$.
        Por tanto, y suponiendo que $c_2\neq 0$ (algo que dependerá de $p_0,p_1$) tenemos que $\{|p_n|\}\to \infty$.\\

        Además, tenemos que las predicciones no son las mismas que las que se producían en el modelo simple de la telaraña, ya que en dicho caso si $d=b$, como es el caso, se trata de un $2-$ciclo.
    \end{enumerate}
\end{ejercicio}

\begin{ejercicio}
    Se considera el siguiente modelo de Samuelson modificado:
    \begin{equation*}
        \begin{array}{rcl}
            Y_n & = & C_n + I_n\\
            C_n & = & b I_{n-1}\\
            I_n & = & C_n - k C_{n-1} + G
        \end{array}
    \end{equation*}
    donde $Y_n$, $C_n$, $I_n$ son la renta, consumo e inversión anual, respectivamente, $G$ es el gasto público (que se supone
    constante) y $0 < b < 1$, $k > 0$. Escribe la ley de recurrencia que cumplen las inversiones anuales $I_n$. Haz un análisis
    del plano de parámetros $k$, $b$ donde se reflejen la estabilidad y las oscilaciones de la renta en torno al equilibrio
    económico.\\

    La ley de recurrencia que cumplen las inversiones anuales $I_n$ es:
    \begin{align*}
        I_n &= C_n - kC_{n-1} + G =\\
        &= bI_{n-1} - kbI_{n-2} + G
    \end{align*}

    Por tanto, las inversiones del equilibrio económico, $I^\ast$, cumplen:
    \begin{align*}
        I^\ast = bI^\ast - kbI^\ast + G
        \Longrightarrow I^\ast = \frac{G}{1-b+kb}
    \end{align*}

    Además, el coste y la renta del equilibrio económico, $C^\ast,Y^\ast$, cumplen:
    \begin{align*}
        I^\ast &= \frac{G}{1-b+kb} \\
        C^\ast &= bI^\ast = \frac{bG}{1-b+kb}\\
        Y^\ast &= C^\ast + I^\ast = \frac{G+bG}{1-b+kb}
    \end{align*}

    Resolvamos ahora la recurrencia dada. El polinomio característico asociado a la recurrencia es:
    \begin{equation*}
        p(\lm) = \lm^2 - b\lm + kb = 0
        \Longleftrightarrow
        \lm = \frac{b\pm \sqrt{b^2-4kb}}{2}
        = \frac{b\pm \sqrt{b(b-4k)}}{2}
    \end{equation*}

    Realizamos por tanto la siguiente distinción de casos:
    \begin{itemize}
        \item \ul{$b>4k$}:
        
        En este caso, las raíces son reales y distintas, y son:
        \begin{equation*}
            \lm_1 = \frac{b+ \sqrt{b(b-4k)}}{2},\hspace{2cm} \lm_2 = \frac{b- \sqrt{b(b-4k)}}{2}
        \end{equation*}

        Veamos que $0<\lm_2<\lm_1<1$. Para ello, vemos que:
        \begin{align*}
            0< \lm_2 &\Longleftrightarrow
            b> \sqrt{b(b-4k)} \Longleftrightarrow
            b^2> b^2-4bk \Longleftrightarrow
            0<4bk
        \end{align*}
        Cierto por ser $b,k>0$. Por otro lado, tenemos que:
        \begin{align*}
            \lm_1 = \frac{b+ \sqrt{b(b-4k)}}{2}
            = \frac{b+ \sqrt{b^2-4kb}}{2}
            \leq \frac{b+ \sqrt{b^2}}{2} = \frac{2b}{2} = b < 1
        \end{align*}

        Como trivialmente $\lm_2<1$, tenemos lo buscado. Por tanto,
        $|\lm_1|,|\lm_2|<1$, por lo que $\{I_n^{(h)}\}\to 0$,
        y por tanto $\{I_n\}\to I^\ast$. De aquí se deduce que se tiende al
        equilibrio económico.

        \item \ul{$b=4k$}:
        
        En este caso, la raíz es real y con multiplicidad doble, y es:
        \begin{equation*}
            \lm = \frac{b}{2} = 2k
        \end{equation*}

        En este caso, la solución de la parte homogénea es:
        \begin{equation*}
            I_n^{(h)} = (c_1+c_2n)(2k)^n
            = (c_1+c_2n)\left(\frac{b}{2}\right)^n
        \end{equation*}

        Como $0<\nicefrac{b}{2}<b<1$, tenemos que $\{I_n^{(h)}\}\to 0$,
        y por tanto $\{I_n\}\to I^\ast$. De aquí se deduce que se tiende al
        equilibrio económico.

        \item \ul{$b<4k$}:
        
        En este caso, las raíces son complejas, y son:
        \begin{equation*}
            \lm = \frac{b\pm \sqrt{b(b-4k)}}{2}
            = \frac{b\pm \sqrt{b^2-4kb}}{2}
            = \frac{b\pm i\sqrt{4kb-b^2}}{2}
        \end{equation*}

        Su módulo es:
        \begin{equation*}
            |\lm| = \sqrt{\frac{b^2}{4}+\frac{4kb-b^2}{4}}
            = \sqrt{\frac{b^2+4kb-b^2}{4}}
            = \sqrt{kb}
        \end{equation*}

        Por tanto, tenemos que $|\lm|<1$ si y solo si $kb<1$. Por tanto, distingumos:
        \begin{itemize}
            \item \ul{$kb\geq 1$}:
            
            En este caso, existe una solución $\{I_n^{(h)}\}\to \infty$, por lo que no se tiende al equilibrio económico.

            \item \ul{$kb<1$}:
            
            En este caso, $\{I_n^{(h)}\}\to 0$, por lo que se tiende al equilibrio económico.
        \end{itemize}
    \end{itemize}    
\end{ejercicio}

\begin{ejercicio}
    Se considera el siguiente modelo de Samuelson modificado:
    \begin{equation*}
        \begin{array}{rcl}
            Y_n & = & \alpha C_n + \beta I_n\\
            C_n & = & Y_{n-1}\\
            I_n & = & k(C_n - C_{n-1}) + G
        \end{array}
    \end{equation*}
    donde $Y_n$, $C_n$, $I_n$ son la renta, consumo e inversión anual, respectivamente, $G$ es el gasto público (que se supone
    constante) y $k > 0$. Los parámetros $0 < \alpha < 1$ y $0 < \beta < 1$ están ligados a impuestos al consumo y la inversión
    que se destinan a ayuda exterior (los números $1 - \alpha$ y $1 - \beta$ representan las partes proporcionales de consumo e
    inversión destinadas a dicha ayuda). Efectúa un análisis similar al del ejercicio anterior para el caso en que $\alpha = \beta$.\\

    La ley de recurrencia que cumple la renta anual $Y_n$ es:
    \begin{align*}
        Y_n &= \alpha C_n + \beta I_n =\\
        &= \alpha Y_{n-1} + \beta\left(k(Y_{n-1} - Y_{n-2}) + G\right)
        =\\&= (\alpha+\beta k)Y_{n-1} - \beta k Y_{n-2} + \beta G
    \end{align*}

    Por tanto, las rentas del equilibrio económico, $Y^\ast$, cumplen:
    \begin{align*}
        Y^\ast &= (\alpha+\beta k)Y^\ast - \beta k Y^\ast + \beta G
        \Longrightarrow Y^\ast = \frac{\beta G}{1-\alpha-\beta k + \beta k}
        = \frac{\beta G}{1-\alpha}
    \end{align*}

    Además, el coste y la inversión del equilibrio económico, $C^\ast,I^\ast$, cumplen:
    \begin{align*}
        C^\ast &= Y^\ast = \frac{\beta G}{1-\alpha}\\
        I^\ast &= k(C^\ast - C^\ast) + G = G
    \end{align*}

    Resolvamos ahora la recurrencia dada. El polinomio característico asociado a la recurrencia es:
    \begin{equation*}
        p(\lm) = \lm^2 - (\alpha+\beta k)\lm + \beta k = 0
    \end{equation*}

    Tenemos que:
    \begin{align*}
        p(0) &= \beta k < 1 \Longleftrightarrow \beta k < 1\\
        p(1) &= 1-\alpha-\beta k + \beta k = 1-\alpha > 0 \Longleftrightarrow \alpha < 1 \\
        p(-1) &= 1+\alpha+\beta k + \beta k > 0
    \end{align*}

    Como $\alpha<1$, tenemos que $\{Y_n^{(h)}\}\to 0$
    si y solo si $\beta k < 1$. Por tanto, $\{Y_n\}\to Y^\ast$. De aquí se deduce que se tiende al equilibrio económico.
\end{ejercicio}

\begin{ejercicio}
    Determina las soluciones $x_n$ de la ecuación siguiente, siendo $i$ la unidad imaginaria:
    \begin{equation*}
        x_{n+2} - (i + 1)x_{n+1} + ix_n = 0
    \end{equation*}
    
    ¿Son $\Re(x_n)$ e $\Im(x_n)$ soluciones de la ecuación en diferencias? (Denotamos por $\Re(z)$ e $\Im(z)$ las partes reales e imaginarias, respectivamente, del número complejo~$z$).\\

    Tenemos que el polinomio característico asociado a la recurrencia es:
    \begin{equation*}
        p(\lm)=\lm^2-(i+1)\lm + i
    \end{equation*}

    Las raíces del polinomio característico son:
    \begin{equation*}
        \lm = \frac{i+1 \pm \sqrt{(i+1)^2-4i}}{2}
        = \frac{i+1 \pm \sqrt{i^2+2i+1-4i}}{2}
        = \frac{i+1 \pm \sqrt{-2i}}{2}
        = \frac{i+1 \pm (1-i)}{2}
    \end{equation*}

    Tenemos entonces que $\lm_1=1,~\lm_2=i$. Entonces:
    \begin{equation*}
        x_n = a + b\cdot i^n,\qquad a,b\in \bb{C}
    \end{equation*}

    Usando las potencias de $i$, tenemos el siguiente $4-$ciclo:
    \begin{equation*}
        \{x_n\} = \{a+b,~a+bi,~a-b,~a-bi,\dots\}
    \end{equation*}

    Veamos ahora que $\Re(x_n),~\Im(x_n)$ no tienen por qué ser soluciones, algo que sí ocurría en el caso de que los coeficientes fuesen reales. Encontremos valores de $a,b$ tal que muestren esto. \\
    
    Supongamos que $a,b\in \bb{R}^\ast\subset \bb{C}$, y veamos que $\Re(x_n)$ no es solución. Tenemos que:
    \begin{align*}
        0 &= (a-b) - (i+1)(a) + i(a+b)\\
        &= a-b-ia-a+i(a+b)\\
        &= -b-ia+ia+ib = -b +ib
    \end{align*}
    Igualando las partes reales, llegamos a que $b=0$, pero hemos supuesto $b\in \bb{R}^\ast$, por lo que $\Re(x_n)$ no es solución.
    
    Por tanto, $\Re(x_n)$ no tiene por qué ser solución. De igual forma, veamos que $\Im(x_n)$ no es solución:
    \begin{align*}
        0 &= 0 - (i+1)(b) + 0\\
        &= -b-bi
    \end{align*}
    Igualando las partes reales, llegamos a que $b=0$, pero hemos supuesto $b\in \bb{R}^\ast$, por lo que de igual forma $\Im(x_n)$ no es solución.
\end{ejercicio}
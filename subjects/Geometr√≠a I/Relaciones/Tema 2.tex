\section{Espacios Vectoriales}

\begin{ejercicio} En $\mathbb{R}^3$ se considera la suma de elementos coordenada a coordenada y el producto por escalares reales dado por:
	\[ \alpha \star (x,y,z) = (\alpha \cdot x, \alpha^2 \cdot y, \alpha^3 \cdot z), \]
	para todo $\alpha \in \mathbb{R}$ y $(x,y,z) \in \mathbb{R}^3$. Determine si $\mathbb{R}^3$ con estas operaciones satisface las propiedades de un espacio vectorial real.

	Para comprobar si $\R^3$ con estas operaciones satisface las propiedades de un espacio vectorial real, sean $u = (x,y,x), v = (x',y',z'), w = (x'',y'', z'')$ con $u,v,w \in \R^3, \quad a,b \in \R$

	\begin{itemize}
		\item Asociativa de $+$:
		      \begin{align*}
			      (u+v) + w & = (x + x', y + y', z + z') + (x'',y'',z'') \\&= ((x+x') + x'', (y + y') + y'', (z + z') + z'') = u + (v + w)
		      \end{align*}
		\item Conmutativa de $+$:
		      \begin{equation*}
			      u + v = (x + x', y + y', z + z') = (x' + x, y' + y, z' + z) = v + u
		      \end{equation*}
		\item Existencia elemento neutro:
		      \begin{equation*}
			      \text{Sea } 0 = (0,0,0), \text{ entonces} u + 0 = (x + 0, y + 0, z + 0) = (x,y,z) = 0
		      \end{equation*}
		\item Existencia de opuestos:
		      \begin{align*}
			      \text{Sea } -u & = (-x,-y,-z), \text{ entonces } u + (-u) = (x + (-x), y + (-y), z + (-z)) \\&= (0,0,0) = 0
		      \end{align*}
	\end{itemize}
	Por tanto $\R^3$ con estas operaciones es un grupo abeliano. Ahora veamos las propiedades acerca del producto por escalares:
	\begin{itemize}
		\item Distributividad (1):
		      \begin{align*}
			      a(u+v) & = a(x + x', y + y', z + z') \\&= (a(x + x'), a^2(y + y'), a^3(z + z')) = (ax + ax', a^2y + a^2y', a^3z + a^3z')
			      \\&= (ax, a^2y, a^3z) + (ax', a^2y', a^3z') = au + av
		      \end{align*}
		\item Distributividad (2):
		      \begin{align*}
			      (a+b)u & = ((a+b)x, (a+b)^2y, (a+b)^3z)                     \\&= (ax + bx, a^2y + b^2y + 2aby, a^3z + b^3z 3a^2bz + 3ab^2z) \\
			             & \neq (ax + bx, a^2y + b^2y, a^3z + b^3z) = au + bu
		      \end{align*}
	\end{itemize}
	Por lo que $\R^3$ con estas operaciones no satisface las propiedades de un espacio vectorial real.
\end{ejercicio}
\begin{ejercicio}Sea $X$ un conjunto no vacío y $V$ un espacio vectorial sobre un cuerpo $K$. Denotamos por $F(X,V)$ al conjunto de las aplicaciones $f : X \to V$. En $F(X,V)$ se define la suma y el producto por elementos de $K$ siguientes:
	\begin{align*}
		(f + g)(x)          & = f(x) + g(x),       & \forall x \in X, \forall g,f \in F(X,V),                     \\
		(\alpha \cdot f)(x) & = \alpha \cdot f(x), & \forall x \in X, \forall \alpha \in K, \forall f \in F(X,V).
	\end{align*}
	Demostrar que, con estas operaciones, $F(X,V)$ es un espacio vectorial sobre $K$.
	Para comprobar si es un espacio vectorial, sean $f,g,h \in F(X, V), \quad x \in X, \quad a,b, \in \mathbb{K}$
	\begin{itemize}
		\item Asociativa para la $+$:
		      \begin{equation*}
			      (f+(g+h))(x) = f(x) + (g+h)(x) = f(x) + (f(x) + g(x)) + h(x) = ((f+g) + h)(x)
		      \end{equation*}
		\item Conmutativa para la $+$:
		      \begin{equation*}
			      (f+g)(x) = f(x) + g(x) = g(x) + f(x) = (g+f)(x)
		      \end{equation*}
		\item Existencia de elemento neutro:
		      \begin{equation*}
			      \text{Sea } 0 = f_0 : \mathbb{X} \mapsto V \text{dada por} f_0(x) = 0_{v(k)}
		      \end{equation*}
		      \text{ entonces }
		      \begin{equation*}
			      (f+0)(x) = (f + f_0)(x) = f(x) + f_0(x) = f(x) + 0_{v(k)} = f(x)
		      \end{equation*}
		\item Existencia de opuestos:
		      Sea $f' : X \to V$ dada por $f'(x) = -f(x)$, entonces:
		      \begin{equation*}
			      (f+f')(x) = f(x) + f'(x) = f(x) - f(x) = 0_{v(k)} \Rightarrow f' = -f
		      \end{equation*}
	\end{itemize}
	Por tanto $F(X,V)$ con estas operaciones es un grupo abeliano, veamos las propiedades del producto por escalares:
	\begin{itemize}
		\item Distributividad (1):
		      \begin{align*}
			      a(f+g)(x) & = a(f+g)(x) = a(f(x) + g(x)) = a \cdot f(x) + a \cdot g(x) = (af + ag)(x)
		      \end{align*}
		\item Distributividad (2):
		      \begin{align*}
			      (a+b)f(x) & = (a+b)f(x) = (a+b) \cdot f(x) = a \cdot f(x) + b \cdot f(x) = (af + bf)(x)
		      \end{align*}
		\item Pseudoasociatividad:
		      \begin{equation*}
			      (   \alpha \cdot \beta) \cdot f(x) = \alpha \cdot (\beta \cdot f(x))
		      \end{equation*}
		\item Propiedad modular:
		      \begin{equation*}
			      1 \cdot f(x) = f(x)
		      \end{equation*}
	\end{itemize}

	Por tanto   $F(X,V)$ con estas operaciones es un espacio vectorial sobre $K$.
\end{ejercicio}


\begin{ejercicio} Sean $V_1$ y $V_2$ los espacios vectoriales sobre un mismo cuerpo $K$.
	\begin{enumerate}
		\item \textit{Demostrar que el conjunto $V_1 \times V_2 = \{ (v_1,v_2) | v_1 \in V_1, v_2 \in V_2 \}$ es un espacio vectorial sobre $K$ cuando definimos la suma y el producto por elementos de $K$ como:}
		      \begin{align*}
			      (u_1,u_2) + (v_1,v_2)   & = (u_1 + v_1, u_2 + v_2),               \\
			      \alpha \cdot (v_1, v_2) & = (\alpha \cdot v_1, \alpha \cdot v_2).
		      \end{align*}
		      Sean $(u_1,u_2),(v_1,v_2),(w_1,w_2) \in V_1 \times V_2, \quad \alpha,\beta \in K$
		      \begin{itemize}
			      \item Asociatividad para $+$:
			            \begin{align*}
				            (u_1,u_2) + ((v_1,v_2) + (w_1,w_2)) & = (u_1,u_2) + (v_1 + w_1,v_2 + w_2)   \\ = (u_1 + v_1 + w_1,u_2 + v_2 + w_2)
				                                                & = (u_1 + v_1,u_2 + v_2) + (w_1,w_2)   \\
				                                                & = ((u_1,u_2) + (v_1,v_2)) + (w_1,w_2)
			            \end{align*}
			      \item Conmutativa para $+$:
			            \begin{align*}
				            (u_1,u_2) + (v_1,v_2) & = (u_1 + v_1,u_2 + v_2) = (v_1 + u_1,v_2 + u_2) & \\= (v_1,v_2) + (u_1,u_2)
			            \end{align*}
			      \item Existencia de elemento neutro:
			            \begin{align*}
				            0 = (0_{V_1},0_{V_2}) \text{ entonces } (u_1,u_2) + 0 & = (u_1 + 0_{V_1},u_2 + 0_{V_2}) = (u_1,u_2)
			            \end{align*}
			      \item Existencia de opuestos:
			            \begin{align*}
				            -(u_1,u_2) & = (-u_1,-u_2) \text{ entonces } (u_1,u_2) + -(u_1,u_2) \\&= (u_1 + -u_1,u_2 + -u_2) = (0_{V_1},0_{V_2})
			            \end{align*}
		      \end{itemize}
		      Por tanto $V_1 \times V_2$ con estas operaciones es un grupo abeliano. Ahora veamos las propiedades del producto por escalares:
		      \begin{itemize}
			      \item Distributividad (1):
			            \begin{align*}
				            \alpha \cdot ((u_1,u_2) + (v_1,v_2)) & = \alpha \cdot (u_1 + v_1,u_2 + v_2)                                        \\
				                                                 & = (\alpha \cdot (u_1 + v_1),\alpha \cdot (u_2 + v_2))                       \\
				                                                 & = (\alpha \cdot u_1 + \alpha \cdot v_1,\alpha \cdot u_2 + \alpha \cdot v_2) \\
				                                                 & = (\alpha \cdot u_1,\alpha \cdot u_2) + (\alpha \cdot v_1,\alpha \cdot v_2) \\
				                                                 & = \alpha \cdot (u_1,u_2) + \alpha \cdot (v_1,v_2)
			            \end{align*}
			      \item Distributividad (2):
			            \begin{align*}
				            (\alpha + \beta) \cdot (u_1,u_2) & = ((\alpha + \beta) \cdot u_1,(\alpha + \beta) \cdot u_2)                 \\
				                                             & = (\alpha \cdot u_1 + \beta \cdot u_1,\alpha \cdot u_2 + \beta \cdot u_2) \\
				                                             & = (\alpha \cdot u_1,\alpha \cdot u_2) + (\beta \cdot u_1,\beta \cdot u_2) \\ & = \alpha \cdot (u_1,u_2) + \beta \cdot (u_1,u_2)
			            \end{align*}
			      \item Pseudoasociatividad:
			            \begin{align*}
				            (\alpha \cdot \beta) \cdot (u_1,u_2) & = ((\alpha \cdot \beta) \cdot u_1,(\alpha \cdot \beta) \cdot u_2) = (\alpha \cdot (\beta \cdot u_1),\alpha \cdot (\beta \cdot u_2)) \\
				                                                 & = \alpha \cdot (\beta \cdot u_1,\beta \cdot u_2) = \alpha \cdot (\beta \cdot (u_1,u_2))
			            \end{align*}
			      \item Propiedad modular:
			            \begin{align*}
				            1 \cdot (u_1,u_2) & = (1 \cdot u_1,1 \cdot u_2) = (u_1,u_2)
			            \end{align*}
		      \end{itemize}
		      Por tanto $V_1 \times V_2$ con estas operaciones es un espacio vectorial sobre $K$.
		\item \textit{Supongamos que $U_i$ es un subespacio vectorial de $V_i$ para cada $i = 1, 2$. Demostrar que $U_1 \times U_2$ es un subespacio vectorial de $V_1 \times V_2$. ¿Es todo subespacio vectorial de $V_1 \times V_2$ de la forma $U_1 \times U_2$ donde cada $U_i$ es un subespacio vectorial de $V_i$?}
		      $U_1 \times U_2$ es un subespacio vectorial de $V_1 \times V_2$ si y solo si:
		      \begin{equation*}
			      \begin{cases}
				      (u_1,u_2) + (v_1,v_2) \in U_1 \times U_2 \\
				      \alpha \cdot (u_1,u_2) \in U_1 \times U_2
			      \end{cases}
		      \end{equation*}
		      Sean $(u_1,u_2),(v_1,v_2) \in U_1 \times U_2, \quad \alpha \in K$
		      \begin{itemize}
			      \item $(u_1,u_2) + (v_1,v_2) = (u_1 + v_1,u_2 + v_2) \in U_1 \times U_2$
			      \item $\alpha \cdot (u_1,u_2) = (\alpha \cdot u_1,\alpha \cdot u_2) \in U_1 \times U_2$
		      \end{itemize}
		      Por tanto $U_1 \times U_2$ es un subespacio vectorial de $V_1 \times V_2$.\\
	\end{enumerate}

\end{ejercicio}


\begin{ejercicio}En cada uno de los siguientes casos estudiar si $U$ es o no un subespacio vectorial de $V$:
	% Enumerado usando a), b), c).. en lugar de 1, 2, 3..
	\begin{enumerate}
		\item $V = \mathbb{R}^2$, $U = \{ (x,y) \in \mathbb{R}^2 | x = y^2 \}$
		      \\ \\ No lo es, para verlo, sean $(4,2), (9,3) \in \R^2$ entonces:
		      \begin{equation*}
			      (4,2), (9,3) \in U \text{ sin embargo } (4,2)+(9,3)=(13,5) \notin U
		      \end{equation*}
		\item $V = \mathbb{R}^2$, $U = \{ (1,0), (0,0) \}$ \\ \\ No lo es, para verlo, sea $(1,0) \in \mathbb{R}^2$ entonces:
		      \begin{equation*}
			      (1,0) \in U \text{ sin embargo } (1,0)+(1,0)=(2,0) \notin U
		      \end{equation*}
		      De otra manera, podríamos haber dicho simplemente que todo espacio vectorial con mas de un vector debe
		      tener infinitos vectores, sin embargo $U$ solo tiene dos.
		\item $V = M_2 (\R) $, $U = \left\{
			      \begin{pmatrix}
				      a  & b \\
				      -b & c
			      \end{pmatrix}
			      \in M_2 (\R) \ | \ a,b,c \in \R \right\}$ \\ \\ Si lo es, para verlo, sean $x,y \in \R $ y sean $\begin{pmatrix}
				      a  & b \\
				      -b & c
			      \end{pmatrix},
			      \begin{pmatrix}
				      a'  & b' \\
				      -b' & c'
			      \end{pmatrix} \in U$ entonces:
		      \begin{equation*}
			      x \begin{pmatrix}
				      a  & b \\
				      -b & c
			      \end{pmatrix} + y \begin{pmatrix}
				      a'  & b' \\
				      -b' & c'
			      \end{pmatrix} = \begin{pmatrix}
				      xa + ya'  & xb + yb' \\
				      -xb - yb' & xc + yc'
			      \end{pmatrix} =
			      \begin{pmatrix}
				      xa + ya'   & xb + yb' \\
				      -(xb +yb') & xc + yc'
			      \end{pmatrix}\in U
		      \end{equation*}
		\item $V = \mathbb{K} [x]$, $U_n = \{ p(x) \in K[x] | \text{ grado(p(x)) = n } \}$ \\  \\ No lo es, para verlo
		      sean $p(x),q(x) \in K[x]$ tales que $ p(x)= 2x^n +1, \ q(x)=-2x^n  $ entonces:
		      \begin{equation*}
			      p(x),q(x) \in U_n \text{ sin embargo } p(x)+q(x)=1_K \notin U_n
		      \end{equation*}
		\item $V = \mathbb{R}^4$, $U = \{(x,y,z,t) \in \mathbb{R}^4 | 2x - y = 3\}$.
		      \\ \\No lo es, para verlo, sean $(1,1,1,1), (2,1,1,1) \in \R^4$ entonces:
		      \begin{equation*}
			      (1,-1,1,1), (2,1,1,1) \in U \text{ sin embargo } (1,-1,1,1)+(2,1,1,1)=(3,0,2,2) \notin U
		      \end{equation*}
		\item $V = \mathbb{R}^n$, $U = \mathbb{Q}^n$.
		      \\ \\No lo es, para verlo, sean $a = \sqrt{2}$ y $v = 1_{\R^n}$ entonces:
		      \begin{equation*}
			      a \cdot v = \sqrt{2} \cdot 1_{\R^n} = \sqrt{2} \cdot (1,1,1,...,1) = (\sqrt{2},\sqrt{2},\sqrt{2},...,\sqrt{2}) \notin \mathbb{Q}^n
		      \end{equation*}

		\item $V = \mathbb{R}^5$, $U = \{(x,y,z,t,s) \in \mathbb{R}^5 | -y = 2x + z\}$.
		      \\ \\Si lo es, para verlo, sean $u = (x, -2x-z, z, t, s), v = (x', -2x'-z', z', t', s') \in \R^5$ entonces:
		      \begin{align*}
			      u + v &= (x + x', -2x - 2x' - z - z', z + z', t + t', s + s') \\&= (x + x', -2(x + x') - (z + z'), z + z', t + t', s + s')
		      \end{align*}
		      Que claramente pertenece a $U$.
		      \begin{equation*}
			      \alpha \cdot u = (\alpha x, -2\alpha x - \alpha z, \alpha z, \alpha t, \alpha s) = (\alpha x, -2\alpha x - \alpha z, \alpha z, \alpha t, \alpha s) \in U
		      \end{equation*}
		      Por tanto $U$ es un subespacio vectorial de $\R^5$.
		\item $V = \mathbb{R}^3$, $U = \{(x,y,z) \in \mathbb{R}^3 | x^2 yz = 0\}$.
		      \\ \\No lo es, para verlo, sean $ u = (0,1,1) \in U $ y $ v = (1,0,1) \in U $ entonces:
		      \begin{equation*}
			      u + v = (0,1,1) + (1,0,1) = (1,1,2) \notin U
		      \end{equation*}
		\item $V = \mathbb{R}^3$, $U = \{(x,y,z) \in \mathbb{R}^3 | x^2 + y^2 + z \geq 0\}$.
		      \\ \\No lo es, para verlo sea $a = -1 \in \R$ y $u = (0,0,1) \in U$, entonces
		      \begin{equation*}
			      a \cdot u = -1 \cdot (0,0,1) = (0,0,-1) \notin U
		      \end{equation*}
		\item $V = \R^3$, $U = \{(x,y,z) \in \R^3 | x = y = z\}$.
		      \\ \\Si lo es, para verlo, sean $u = (x,y,z), v = (x',y',z') \in U$ entonces:
		      \begin{equation*}
			      u + v = (x + x', y + y', z + z') = (x + x', x + x', x + x') \in U
		      \end{equation*}
		      \begin{equation*}
			      \alpha \cdot u = (\alpha x, \alpha y, \alpha z) = (\alpha x, \alpha x, \alpha x) \in U
		      \end{equation*}
		      Por tanto $U$ es un subespacio vectorial de $\R^3$.
		\item $V = \R^5$, $U = \{(0,0,1,-1,2), (3,2,\sqrt{5},-8,32)\}$
		      \\ \\No lo es, para verlo, sean $a = 2, u = (0,0,1,-1,2) \in U$ entonces:
		      \begin{equation*}
			      a \cdot u = 2 \cdot (0,0,1,-1,2) = (0,0,2,-2,4) \notin U
		      \end{equation*}
		\item $V = M_2(\R)$, $U = \left\{\begin{pmatrix}
				      a & 1 + a \\
				      0 & 0
			      \end{pmatrix} \arrowvert \ a \in \R \right\}$
		      \\ \\ No lo es, para verlo, sean $u = \begin{pmatrix}
				      0 & 1 \\
				      0 & 0
			      \end{pmatrix} \in U$ y $v = \begin{pmatrix}
				      1 & 2 \\
				      0 & 0
			      \end{pmatrix} \in U$ entonces:
		      \begin{equation*}
			      u + v = \begin{pmatrix}
				      0 & 1 \\
				      0 & 0
			      \end{pmatrix} + \begin{pmatrix}
				      1 & 2 \\
				      0 & 0
			      \end{pmatrix} = \begin{pmatrix}
				      1 & 3 \\
				      0 & 0
			      \end{pmatrix} \notin U
		      \end{equation*}
		\item $V = \mathcal{F(\R,\R)}$, $U = \{f \in \mathcal{F(\R,\R)} | f''(x) + f(x) = 0, \quad \forall x \in \R\}$
		      \\ \\ Si lo es, para verlo sean $a,b \in \R$ y $f,g \in U$ entonces:
		      \begin{equation*}
			      af(x) + bg(x) = (af)(x) + (bg)(x) = (ab + fg)(x) \in U
		      \end{equation*}
		      \begin{align*}
			      (af + bg)''(x) + (af + bg)(x) & = (af)''(x) + (af)(x) + (bg)''(x) + (bg)(x) \\&= a(f''(x) + f(x)) + b(g''(x) + g(x)) = 0
		      \end{align*}
		\item $V = M_n(\mathbb{K})$, $U = \{ A \in M_n(\mathbb{K}) \arrowvert \ A $ es diagonal $\}$
		      Si lo es, para verlo, sean $x,y \in \mathbb{K}$ y $A,B \in U$ entonces:
		      \begin{align*}
			      A &= \begin{pmatrix}
				      a_1    & 0      & \cdots & 0      \\
				      0      & a_2    & \cdots & 0      \\
				      \vdots & \vdots & \ddots & \vdots \\
				      0      & 0      & \cdots & a_n
			      \end{pmatrix} \arrowvert \ a_1,a_2,...,a_n \in \mathbb{K}
			      \\ B &=\begin{pmatrix}
				      b_1    & 0      & \cdots & 0      \\
				      0      & b_2    & \cdots & 0      \\
				      \vdots & \vdots & \ddots & \vdots \\
				      0      & 0      & \cdots & b_n
			      \end{pmatrix} \arrowvert \ b_1,b_2,...,b_n \in \mathbb{K}
		      \end{align*}
		      Entonces
		      \begin{equation*}
			      xA + yB = \begin{pmatrix}
				      xa_1 + yb_1 & 0           & \cdots & 0           \\
				      0           & xa_2 + yb_2 & \cdots & 0           \\
				      \vdots      & \vdots      & \ddots & \vdots      \\
				      0           & 0           & \cdots & xa_n + yb_n
			      \end{pmatrix} \in U
		      \end{equation*}
		\item $V = M_n(\mathbb{K})$, $U = \{ A \in M_n(\mathbb{K}) \arrowvert \ A $ es triangular superior $\}$
		      \\ \\ Si lo es, para verlo, sean $a,b \in \mathbb{K}$ y $A,B \in U$ entonces:
		      \begin{equation*}
			      A = \begin{pmatrix}
				      a_{11} & a_{12} & \cdots & a_{1n} \\
				      0      & a_{22} & \cdots & a_{2n} \\
				      \vdots & \vdots & \ddots & \vdots \\
				      0      & 0      & \cdots & a_{nn}
			      \end{pmatrix} \quad \text{y} \quad B =\begin{pmatrix}
				      b_{11} & b_{12} & \cdots & b_{1n} \\
				      0      & b_{22} & \cdots & b_{2n} \\
				      \vdots & \vdots & \ddots & \vdots \\
				      0      & 0      & \cdots & b_{nn}
			      \end{pmatrix}
		      \end{equation*}
		      Entonces
		      \begin{equation*}
			      aA + bB = \begin{pmatrix}
				      aa_{11} + bb_{11} & aa_{12} + bb_{12} & \cdots & aa_{1n} + bb_{1n} \\
				      0                 & aa_{22} + bb_{22} & \cdots & aa_{2n} + bb_{2n} \\
				      \vdots            & \vdots            & \ddots & \vdots            \\
				      0                 & 0                 & \cdots & aa_{nn} + bb_{nn}
			      \end{pmatrix} \in U
		      \end{equation*}
	\end{enumerate}
\end{ejercicio}


\begin{ejercicio} En cada uno de los siguientes casos decidir si el vector \( v \) del espacio vectorial \( V \) pertenece o no al subespacio \( L(S) \), en caso afirmativo, expresar \( v \) como combinación lineal de \( S \):
	\begin{enumerate}
		\item \( V = \mathbb{R}^3, v = (0,2,-5), S = \{(1,-3,2), (2,-4,-1), (1,-5,7)\} \).
		      \\ \\ Sean $a,b,c \in \R,\ v \in \R^3 $ tales que $v=a(1,-3,2)+b(2,-4,-1)+c(1,-5,7)$ entonces:
		      \begin{equation*}
			      \left.
			      \begin {array}{rcl}
			      a+2b+c & = & 0 \\
			      -3a-4b-5c & = & 2 \\
			      2a-b+7c & = & -5
			      \end {array}
			      \right\} \Rightarrow
                  \begin{cases}
                    \begin{vmatrix}
                        1  & 2  & 1  \\
                        -3 & -4 & -5 \\
                        2  & -1 & 7
                    \end{vmatrix} = 0  \\
                    \begin{vmatrix}
                        1  & 2  & 0  \\
                        -3 & -4 & 2  \\
                        -5 & 7  & -5
                    \end{vmatrix} \neq 0
                  \end{cases}
			        \Rightarrow \text{ SCI} \Rightarrow v \notin \mathcal{L} (S)
		      \end{equation*}
		\item \( V = \mathbb{R}^4, v = (9,-17,10,-5), S = \{(2,-1,0,0), (-1,3,-2,1)\} \).
		      \\ \\ Sean $a,b \in \R, v \in \R^4$ tales que $v = a(2,-1,0,0)+b(-1,3,-2,1)$ entonces:
		      \begin{equation*}
			      \left.
			      \begin {array}{rcl}
			      2a-b & = & 9 \\
			      -a+3b & = & -17 \\
			      -2b & = & 10 \\
			      b & = & -5
			      \end {array}
			      \right\} \Rightarrow \begin{cases}
                  b = -5 \\
                    a = 2
                  \end{cases} \Rightarrow 2(2,-1,0,0)-5(-1,3,-2,1) = v\in \mathcal L(S)
		      \end{equation*}
		\item \( V = \mathbb{R}^4, v = (5,7,a,6), S = \{(1,2,3,0), (1,1,1,2)\} \).
		      \\ \\ Sean $c,d \in \R, v \in \R^4$ tales que $v = c(1,2,3,0)+d(1,1,1,2)$ entonces:
		      \begin{equation*}
			      \left.
			      \begin{array}{rcl}
				      c+d  & = & 5 \\
				      2c+d & = & 7 \\
				      3c+d & = & a \\
				      2d   & = & 6
			      \end{array}
			      \right\} \Rightarrow d=3 \Rightarrow c=2 \Rightarrow 2(1,2,3,0)+3(1,1,1,2) = v\in \mathcal L(S)
		      \end{equation*}
		      Lo que ocurre si, y solo si $3c + d = a$, o lo que es lo mismo, $a = 9$.
		\item \( V = M_2(\mathbb{C}), v = \begin{pmatrix} i & 0 \\ 1 & -i \end{pmatrix}, S = \left\{ \begin{pmatrix} 0 & 1 \\ i & 2i \end{pmatrix}, \begin{pmatrix} i & 0 \\ 0 & 1 \end{pmatrix} \right\} \).
		      Sean $a,b \in \R$ tales que $v = a\begin{pmatrix} 0 & 1 \\ i & 2i \end{pmatrix} + b\begin{pmatrix} i & 0 \\ 0 & 1 \end{pmatrix}$ entonces:
		      \begin{equation*}
			      % matrix
			      \begin{pmatrix}
				      i & 0  \\
				      1 & -i
			      \end{pmatrix} = a\begin{pmatrix}
				      0 & 1  \\
				      i & 2i
			      \end{pmatrix} + b\begin{pmatrix}
				      i & 0 \\
				      0 & 1
			      \end{pmatrix}
			      \Rightarrow
			      \begin{pmatrix}
				      i & 0  \\
				      1 & -i
			      \end{pmatrix} = \begin{pmatrix}
				      bi & a       \\
				      ai & 2ai + b
			      \end{pmatrix}
		      \end{equation*}
		      Que claramente no tiene solución, por lo que \( v \) no pertenece a \( \mathcal{L}(S) \).
		\item \( V = \mathbb{R}[x], v = x^2 + x + 1, S = \{x, x^2 + 1, x^3\} \).
		      Sea $a = -1, b = 1, c = 0$, $a,b,c \in \R$ tales que $v = a(x) + b(x^2 + 1) + c(x^3)$ entonces:
		      \begin{equation*}
			      x^2 + x + 1 = x^2 + x + 1 \Rightarrow v \in \mathcal{L}(S)
		      \end{equation*}
	\end{enumerate}
\end{ejercicio}



\begin{ejercicio} Sea \( V \) un espacio vectorial sobre un cuerpo \( K \). Demostrar los siguientes hechos:
	\begin{enumerate}
		\item \textit{Si \( S \) y \( S' \) son subconjuntos no vacíos de \( V \) con \( S \subseteq S' \) entonces \( L(S) \subseteq L(S') \).}
		      \\ \\ Sea \( v \in L(S) \), entonces \( v = \sum_{i=1}^{n} \alpha_i v_i \), con \( v_i \in S \). Por hipótesis, \( v_i \in S' \), por lo que \( v \in L(S') \). Por tanto, \( L(S) \subseteq L(S') \).
		\item \textit{\( L(S) = S \) si y sólo si \( S \) es un subespacio vectorial de \( V \).}
		      \\ \\ Si \( L(S) = S \), entonces \( S \) es un subespacio vectorial de \( V \) por definición de subespacio vectorial. Si \( S \) es un subespacio vectorial de \( V \), entonces \( S \subseteq L(S) \) y \( L(S) \subseteq S \), por lo que \( L(S) = S \).
		\item Si \( U_i = L(S_i) \) para cada \( i = 1, \ldots, m \), entonces \( \sum_{i=1}^{m} U_i = L(\bigcup_{i=1}^{m} S_i) \).
	\end{enumerate}
	Supongamos que \( U_i = L(S_i) \), para cada \( i = 1, \ldots, m \). ¿Es cierto que \( \bigcap_{i=1}^{m} U_i = L(\bigcap_{i=1}^{m} S_i) \)?
	\\ \\ Sea \( v \in \sum_{i=1}^{m} U_i \), entonces \( v = \sum_{i=1}^{m} \alpha_i v_i \), con \( v_i \in S_i \). Por lo tanto, \( v \in L(\bigcup_{i=1}^{m} S_i) \). Por otro lado, sea \( v \in \mathcal{L}(\bigcap_{i=1}^{m} S_i) \), entonces \( v = \sum_{i=1}^{m} \alpha_i v_i \), con \( v_i \in \bigcap_{i=1}^{m} S_i \). Por lo tanto, \( v \in \sum_{i=1}^{m} U_i \). Por tanto, \( \sum_{i=1}^{m} U_i = \mathcal{L}(\bigcup_{i=1}^{m} S_i) \).

\end{ejercicio}



\begin{ejercicio}¿Qué se puede decir sobre dos subespacios vectoriales de \( \mathbb{R}^n \) cuya suma es \( \{0\} \)?¿ y cuya intersección es \( \mathbb{R}^n \)?
	\begin{itemize}
		\item Si la suma es \( \{0\} \):
		      \begin{equation*}
			      U_1 + U_2 = \{0\} \Rightarrow \forall u_1 \in U_1, \forall u_2 \in U_2 \Rightarrow u_1 + u_2 = 0 \Rightarrow u_1 = -u_2 \Rightarrow U_1 = -U_2
		      \end{equation*}
		      Es decir, $U_1 = U_2 = \{0\}$
		\item Si la intersección es \( \mathbb{R}^n \):
		      \\ \\
		      Por ser subespacios vectoriales, $U_1 \subseteq \mathbb{R}^n$ y $U_2 \subseteq \mathbb{R}^n$ entonces:
		      \begin{equation*}
			      U_1 \cap U_2 = \mathbb{R}^n \Rightarrow \R^n \subseteq U_1 \cap U_2 \Rightarrow \R^n \subseteq U_1  \ \land \ \R^n \subseteq U_2 \Rightarrow U_1 = \R^n \ \land \ U_2 = \R^n
		      \end{equation*}
	\end{itemize}
\end{ejercicio}


\begin{ejercicio}
	En cada uno de los siguientes casos demostrar que \( U_1 \) y \( U_2 \) son subespacios vectoriales de \( V \). Estudiar también si se cumple \( V = U_1 \oplus U_2 \).
	\begin{enumerate}
		\item \( V = \mathbb{R}^3, U_1 = \{(x,y,z) \in \mathbb{R}^3 | x = z\}, U_2 = L\{(3,0,2)\} \).
		      \\ \\
		      Sean $(x,y,x),(x',y',x') \in \R^3$, y sean $a,b \in \R$, entonces:
		      \begin{equation*}
			      a(x,y,x) + b(x',y',x') = (ax+bx',ay+by',ax+bx') \in U_1
		      \end{equation*}
		      Que $U_2$ sea subespacio vectorial es trivial por definición de subespacio generado. Veamos ahora si $V = U_1 \oplus U_2$. Para ello, calculamos
		      una base de $U_1$ y $U_2$ y vemos si el sistema de generadores generado por la suma de ambas bases es una base de $V$:
		      \begin{itemize}
			      \item Para $U_1$:
			            \begin{equation*}
				            B = \{(1,0,1),(0,1,0)\}
			            \end{equation*}
			      \item Para $U_2$:
			            \begin{equation*}
				            B' = \{(3,0,2)\}
			            \end{equation*}
		      \end{itemize}
		      Veamos ahora si la suma de ambas bases es una base de $V$:
		      \begin{equation*}
			      \begin{vmatrix}
				      1 & 0 & 3 \\
				      0 & 1 & 0 \\
				      1 & 0 & 2
			      \end{vmatrix} = -1 \neq 0 \Rightarrow \text{ L.I } \Rightarrow \dim(U_1 \oplus U_2) = 3 = \dim(V) \Rightarrow V = U_1 \oplus U_2
		      \end{equation*}

		\item \( V = F(\mathbb{R},\mathbb{R}), U_1 = \{f \in F(\mathbb{R},\mathbb{R}) | f(-x) = f(x), \forall x \in \mathbb{R}\}, U_2 = \{f \in F(\mathbb{R},\mathbb{R}) | f(-x) = -f(x), \forall x \in \mathbb{R}\} \).
		      \\ \\
		      Sean $f,g \in F(\R,\R)$, y sean $a,b \in \R$, entonces:
		      \begin{equation*}
			      a f(-x) + b g(-x) = a f(x) + b g(x) \in U_1
		      \end{equation*}
		      \begin{equation*}
			      a f(-x) + b g(-x) = a f(x) - b g(x) \in U_2
		      \end{equation*}
		      Veamos ahora si $V = U_1 \oplus U_2$. Para ello, comprobamos si $ U_1 \cap U_2 = \{0\} $:
		      \begin{equation*}
			      U_1 \cap U_2 = \{f \in F(\R,\R) | f(-x) = f(x) \ \land \ f(-x) = -f(x)\} = \{0\}
		      \end{equation*}
		      Veamos ahora si $U_1 \oplus U_2 = V$:
		      \begin{equation*}
			      \forall f \in V \Rightarrow f(x) = \frac{f(x) + f(-x)}{2} + \frac{f(x) - f(-x)}{2} \in U_1 + U_2
		      \end{equation*}
		      Por tanto, \( V = U_1 \oplus U_2 \).
		\item \( V = \mathbb{R}[x], U_1 = \{p(x) \in \mathbb{R}[x] | p(1) + p'(1) = 0\}, U_2 = \{p(x) \in \mathbb{R}[x] | p(0) + p''(0) = 0\} \).
              \\ \\ Demostraremos, en primer lugar, que $U_1$ es un subespacio vectorial de $R_n[X]$. Para ello, 
              sean $f, g \in U_1$ y $a,b \in \R$, entonces:
              \begin{equation*}
                (af +bg)(1) + (af + bg)'(1) = a(f(1) + f'(1)) + b(g(1) + g'(1)) = 0
              \end{equation*}
              Por lo que $af + bg \in U_1$. Ahora, para $U_2$:
              Sea $f,g \in U_2$ y $a,b \in \R$, entonces:
                \begin{equation*}
                    (af + bg)(0) + (af + bg)''(0) = a(f(0) + f''(0)) + b(g(0) + g''(0)) = 0
                \end{equation*}
                Por lo que $af + bg \in U_2$. 
                Nos queda ver si $V = U_1 \oplus U_2$. Para ello, sea $p \in \R_n[x]$, entonces
                \begin{align*}
                    p(x) &= a_0 + a_1 x + a_2 x^2 + \ldots + a_n x^n \\
                    p'(x) &= a_1 + 2a_2 x + \ldots + na_n x^{n-1} \\
                    p''(x) &= 2a_2 + 3 \cdot 2 a_3 x + \ldots + n(n-1)a_n x^{n-2}
                \end{align*}
                Buscamos ahora unas ecuaciones canónicas de $U_1$ y $U_2$:
                Trabajamos con $U_1$:
                \begin{equation*}
                    p(1) + p'(1) = 0 = (a_0 + a_1 + \ldots + a_n) + (a_1 + 2a_2 + \ldots + na_n) = 0
                \end{equation*}
                Trabajamos con $U_2$:
                \begin{equation*}
                    p(0) + p''(0) = 0 = a_0 + 2a_2 + \ldots + n(n-1)a_n = 0
                \end{equation*}
                De donde tenemos que:
                \begin{align*}
                    U_1 &= \{p(x) \in \R_n[x] | p(x) = a_0 + 2a_1  + \ldots + na_n x^{n-1} = 0\}
                    \\ U_2 &= \{p(x) \in \R_n[x] | p(x) = a_0 + 2a_2 = 0\}
                \end{align*}
                Por tanto: 
                \begin{equation*}
                    U_1 \cap U_2 = \{ (a_0, a_1, a_2, \ldots, a_n) \in \R^{n+1} : \begin{cases}
                        a_0 + a_1 + \ldots + a_n = 0 \\
                        a_0 + 2a_2 = 0
                    \end{cases}
                    \} 
                \end{equation*}
                Por lo que $ \dim(U_1 \cap U_2) = n + 1 - 2 = n - 1$. Por tanto, para $n \neq 1$ no se cumple que $V = U_1 \oplus U_2$.
                Si $n = 1$, entonces:
                \begin{equation*}
                    U_1 + U_2 = \mathcal{L}\{(0,1), (2,-1)\}
                \end{equation*}
                Como $\begin{vmatrix}
                    0 & 2 \\
                    1 & -1
                \end{vmatrix} \neq 0$, entonces $V = U_1 \oplus U_2$.


		\item \( V = \mathbb{R}^3, U_1 = L\{(1,0,-1), (0,1,-1)\}, U_2 = \{(x,y,z) \in \mathbb{R}^3 | z = 0\} \).
		\item Por definición, $U_1$ es un subespacio vectorial de $\R_3$. Veamos ahora si lo es $U_2$. Sea 
		$u, v \in U_2$ y $a, b \in \R$, entonces:
        \begin{equation*}
            au + bv = a(u_1, u_2, 0) + b(v_1, v_2, 0) = (au_1 + bv_1, au_2 + bv_2, 0) \in U_2
        \end{equation*}
        Veamos ahora si $U_1 \cap U_2 = \{0\}$. Para ello obtenemos ecuaciones cartesianas de $U_1$:
        \begin{equation*}
            \begin{vmatrix}
                1 & 0 & x \\
                0 & 1 & y \\
                -1 & -1 & z
            \end{vmatrix} = 0 \Rightarrow z + x + y = 0
        \end{equation*}
        Entonces, $U_1 \cap U_2 = \left\{ (x,y,z) \in \R^3 : \begin{cases}
            z = 0 \\
            x + y + z = 0
        \end{cases} \right\} \neq \{0\}$. Por tanto, $V \neq U_1 \oplus U_2$.
	\end{enumerate}
	¿De cuántas formas podemos escribir \( v = u_1 + u_2 \) con \( v \in \mathbb{R}^3, u_1 \in U_1, u_2 \in U_2 \)?
    Sea $v \in \R^3$, entonces: 
    \begin{equation*}
        u_1 + u_2 = a(1,0,-1) + b(0,1,-1) + (x,y,0) = (a, b, -a) + (x, y, 0) = (a + x, b + y, -a-b)
    \end{equation*}
    Por tanto:
    \begin{equation*}
        \begin{cases}
            a + x = v_1 \\
            b + y = v_2 \\
            -a - b = v_3
        \end{cases} \Rightarrow rg(A) = rg\begin{pmatrix}
            1 & 0 & 1 & 0 \\
            0 & 1 & 0 & 1 \\
            -1 & -1 & 0 & 0
        \end{pmatrix} = 3 \Rightarrow \text{ SCI }
    \end{equation*}
    Por lo que hay infinitas formas de escribir $v = u_1 + u_2$.
\end{ejercicio}




\begin{ejercicio}Consideremos en \( \mathbb{R}^4 \) los siguientes subespacios vectoriales:
	\begin{enumerate}
		\item \( U_1 = \{(x,y,z,t) \in \mathbb{R}^4 | x - y = 0, z - t = 0\} \),
		\item \( U_2 = L\{(0,1,1,0)\} \),
		\item \( U_3 = \{(x,y,z,t) \in \mathbb{R}^4 | x - y + z - t = 0\} \).
	\end{enumerate}
	\textit{Probar que \( U_1 + U_2 = U_3 \). ¿Se cumple que \( U_3 = U_1 \oplus U_2 \)?} \\ \\
	Para probarlo, comenzamos calculando una base de cada subespacio, para $U_2$ resulta trivial, para $U_1$, tenemos dos ecuaciones linealmente independientes, por tanto, una base de $U_1$ estará
	formada por $\dim(\R^4)=4-2=2$ vectores, por tanto, una base de $U_1$ estará formada por:
	\begin{align*}
		B_1 = \{(1,1,0,0),(0,0,1,1)\}
	\end{align*}

	Análogamente, tenemos que una base de $U_3$ estará formada por $\dim(\R^4)=4-1=3$ vectores, por tanto, una base de $U_3$ estará formada por:
	\begin{equation*}
		B_3 = \{(1,1,0,0),(0,0,1,1),(0,1,1,0)\}
	\end{equation*}
	Donde los vectores han sido convenientemente elegidos para que se vea directamente que $U_1 \oplus U_2 = U_3$.
\end{ejercicio}



\begin{ejercicio} Sea \( V = F(\mathbb{R},\mathbb{R}) \) el espacio vectorial real de las funciones \( f: \mathbb{R} \to \mathbb{R} \). Analizar si la familia \( S = \{f, g, h\} \) es linealmente independiente, donde \( f(x) = x^2 + 1, g(x) = 2x \) y \( h(x) = e^x \).
	\\ \\Para que $S$ sea linealmente independiente, se debe verificar $af(x)+bg(x)+ch(x)=0 \Rightarrow a=b=c=0$. Veamos si esto se cumple:
	\begin{equation*}
		a(x^2+1)+b(2x)+ce^x=0 \Rightarrow \left.
		\begin {array}{rcl}
		ax^2 & = & 0 \\
		2bx & = & 0 \\
		a+ce^x & = & 0
		\end {array}
		\right\} \Rightarrow a=b=c=0 \Rightarrow S \text{ es L.I}
	\end{equation*}
\end{ejercicio}



\begin{ejercicio} Sea \( V \) un espacio vectorial sobre un cuerpo infinito \( K \). Tomemos \( \beta = \{v_1, \ldots, v_n\} \) una base de \( V \) y \( \{a_1, \ldots, a_n\} \) una familia en \( K \) con \( a_i \neq 0 \) para cada \( i = 1, \ldots, n \). Demostrar que \( \beta' = \{a_1v_1, \ldots, a_nv_n\} \) es una base de \( V \). Concluir que \( V \) tiene infinitas bases.
    \\ \\Como $\mathcal{B}$ es base de $V$, entonces $\dim_\R (V) = n$. Por tanto:
    \begin{equation*}
        \{a_1 v_1, \ldots, a_n v_n\} \text{ base } \Leftrightarrow \{a_1 v_1, \ldots, a_n v_n\} \text{ sistema de generador}
    \end{equation*}
    Demostraremos por tanto que $\{a_1 v_1, \ldots, a_n v_n\}$ es un sistema de generadores de $V$. Como
    $B = \{ v_1, \ldots v_n \}$ es base de $V$, entonces $\forall w \in V(\mathbb{K})$:
    \begin{align*}  
        w &= \alpha_1 v_1 + \ldots + \alpha_n v_n \quad \alpha_i \in \mathbb{K} \quad \forall i = 1, \ldots, n \\
        &= \gamma_1 a_1 v_1 + \ldots + \gamma_n a_n v_n \quad \gamma_i = \alpha_i a_i \quad \forall i = 1, \ldots, n \\
        &= \frac{\alpha_1}{a_1} a_1 v_1 + \ldots + \frac{\alpha_n}{a_n} a_n v_n \\
        &= \alpha_1 v_1 + \ldots + \alpha_n v_n = w
    \end{align*}
    Por tanto $\{a_1 v_1, \ldots, a_n v_n\}$ es un sistema de generadores de $V$, por lo que es una base de $V$. Por tanto, $V$ tiene infinitas bases.
\end{ejercicio}


\begin{ejercicio} Sean \( V_1 \) y \( V_2 \) espacios vectoriales finitamente generados sobre un cuerpo \( K \). Demostrar que el espacio vectorial producto \( V_1 \times V_2 \) definido en el ejercicio 3 es finitamente generado. Construir una base de \( V_1 \times V_2 \) a partir de bases de \( V_1 \) y de \( V_2 \). Calcular \( \dim(V_1 \times V_2) \).
	Sean $\beta_1 = \{ u_1, \ldots, u_n \}$ y $\beta_2 = \{ v_1, \ldots, v_m \}$ bases de $V_1$ y $V_2$ respectivamente.
	Veamos que $\beta$ es una base de $V_1 \times V_2$. Para ello, sea $(u,v) \in V_1 \times V_2$, entonces:
	\begin{align*}
		u \in v_1 & \Rightarrow u = \sum_{i=1}^{n} \alpha_i u_i \\
		v \in v_2 & \Rightarrow v = \sum_{j=1}^{m} \beta_j v_j
	\end{align*}
	Por tanto, $(u,v) = (\sum_{i=1}^{n} \alpha_i u_i, \sum_{j=1}^{m} \beta_j v_j) = \sum_{i=1}^{n} \alpha_i (u_i,0) + \sum_{j=1}^{m} \beta_j (0,v_j)$. De donde,
	\begin{equation*}
		S = \{ (u_1,0), \ldots, (u_n,0), (0,v_1), \ldots, (0,v_m) \}
	\end{equation*}
	es un sistema de generador de $V_1 \times V_2$. Además son linealmente independientes, por lo que $S = \beta$ es una base de $V_1 \times V_2$. Por tanto, $\dim(V_1 \times V_2) = n+m$.
\end{ejercicio}



\begin{ejercicio} Sea \( V \) un espacio vectorial complejo con \( \dim_{\mathbb{C}}(V) = n \). Demostrar que \( V \) es un espacio vectorial real con \( \dim_{\mathbb{R}}(V) = 2n \).\\
	Sea $\beta = \{ v_1, \ldots, v_n \}$ una base de $V$ sobre $\mathbb{C}$. Entonces $\forall x in V, x = (z_1, \ldots, z_n)_{\beta}$
	= $\sum_{j=1}^{n} (a_j + b_j i) v_j$. Con $\{ v_1, \ldots, v_n, i v_1, \ldots, i v_n \}$ es un sistema de generadores de $V(\R)$.
	\\ \\Veamos si son linealmente independientes:
	Sea $d_i, e_i \in \R$, entonces $d_1 v_1 + \ldots + d_n v_n + e_1 i v_1 + \ldots + e_n i v_n = 0 \Rightarrow
		d_1 v_1 + \ldots + d_n v_n + e_1 v_1 + \ldots + e_n v_n = 0 \Rightarrow
	$ Como $\{v_1, \ldots, v_n\}$ es linealmente independiente en $V(\C)$, entonces $d_1 = \ldots = d_n = e_1 = \ldots = e_n = 0$.
	Por tanto, $\{ v_1, \ldots, v_n, i v_1, \ldots, i v_n \}$ es una base de $V(\R)$, por lo que $\dim_{\mathbb{R}}(V) = 2n$.
\end{ejercicio}



\begin{ejercicio} Sea \( V \) un espacio vectorial sobre un cuerpo \( K \). Supongamos que \( S = \{v_1, \ldots, v_n\} \) es una familia de vectores de \( V \). Demostrar que:
	\begin{enumerate}
		\item Si \( S \) es sistema de generadores de \( V \) y cumple la propiedad de que cuando se elimina cualquier vector de \( S \) la familia resultante es un sistema de generadores de \( V \), entonces \( S \) es una base de \( V \).
		\\ \\
        Si $S$ es un sistema de generadores, $V(K) = \mathcal{L}(S)$. Si al eliminar cualquier vector de $V$ la familia 
        resultante no es sistema de generadores de $V$, significa que ese vector era linealmente dependiente de los demás,
        por tanto, como los vectores de $S$ son linealmente independientes, entonces $S$ es una base de $V$.
        \item Si \( S \) es linealmente independiente y cumple la propiedad de que cuando se añade a \( S \) cualquier vector de \( V \) la familia resultante es linealmente independiente, entonces \( S \) es una base de \( V \).
        \\ \\
        Si al añadir cualquier vector de $V$ a $S$ la familia resultante es linealmente dependiente, significa que $v$ es combinación 
        lineal de los vectores de $S$, por tanto, $S$ es un sistema de generadores de $V$, y, como los vectores son linealmente independientes,
         $S$ es una base de $V$.
    \end{enumerate}
\end{ejercicio}




\begin{ejercicio} Describir todos los subespacios vectoriales de \( \mathbb{R}^2 \) y de \( \mathbb{R}^3 \).
	\begin{itemize}
		\item \( \mathbb{R}^2 \): $\dim(\R^2)=2$, por tanto, los subespacios vectoriales de $\R^2$ serán de dimensión 0, 1 y 2. Es decir:
		      \begin{align*}
			      U \subseteq \R^2 / \dim{U} = 0 & \Rightarrow U = \{0\}                                  \\
			      V \subseteq \R^2 / \dim{V} = 1 & \Rightarrow V = \mathcal{L}\{v\} \text{ con } v \neq 0 \\
			      W \subseteq \R^2 / \dim{W} = 2 & \Rightarrow W = \R^2
		      \end{align*}
		\item \( \mathbb{R}^3 \): $\dim(\R^3)=3$, por tanto, los subespacios vectoriales de $\R^3$ serán de dimensión 0, 1, 2 y 3.
		      \begin{align*}
			      U \subseteq \R^3 / \dim{U} = 0 & \Rightarrow U = \{0\}                                                   \\
			      V \subseteq \R^3 / \dim{V} = 1 & \Rightarrow V = \mathcal{L}\{v\} \text{ con } v \neq 0                  \\
			      W \subseteq \R^3 / \dim{W} = 2 & \Rightarrow W = \mathcal{L}\{v_1,v_2\} \text{ con } v_1,v_2 \text{ L.I} \\
			      Z \subseteq \R^3 / \dim{Z} = 3 & \Rightarrow Z = \R^3
		      \end{align*}
	\end{itemize}
\end{ejercicio}



\begin{ejercicio} En \( \mathbb{R}^3 \) se consideran los subespacios vectoriales dados por:
	\begin{align*}
		U_1 & = L\{ (1, 1 - \alpha^2, 2), (1 + \alpha, 1 - \alpha, -2) \}, \\
		U_2 & = \{ (x,y,z) \in \mathbb{R}^3 \mid x+y+z = 0 \}.
	\end{align*}
	Calcular todos los valores de \( \alpha \in \mathbb{R} \) para los que \( U_1 = U_2 \).
	En primer lugar, calculamos la dimensión de $U_1$:
	\begin{equation*}
		\begin{vmatrix}
			1          & 1+\alpha \\
			1-\alpha^2 & 1-\alpha \\
			2          & -2
		\end{vmatrix} = 0 \Leftrightarrow \begin{vmatrix}
			1 & 1+\alpha \\
			2 & -2       \\
		\end{vmatrix} = 0 \Leftrightarrow -2 -2\alpha -2 = 0 \Leftrightarrow \alpha = -2
	\end{equation*}
	Por tanto distinguimos dos casos:
	\begin{itemize}
		\item Si $\alpha = -2$, entonces:
		      \begin{equation*}
			      U_1 = L\{ (1, -3, 2), (-1, 3, -2) \} = \{ (x,y,z) \in \mathbb{R}^3 \mid x+y+z = 0 \} = U_2
		      \end{equation*}
		      \begin{equation*}
			      \dim{U_1} = 1 \Rightarrow B_{U_1} = \{(1,-3,2)\} \Rightarrow U_1 \neq U_2
		      \end{equation*}
		\item Si $\alpha \neq -2$, entonces:
		      \begin{equation*}
			      \dim{U_1} = 2 \Rightarrow B_{U_1} = \{(1, 1 - \alpha^2, 2), (1 + \alpha, 1 - \alpha, -2)\}
		      \end{equation*}
		      Para que $U_1 = U_2$, necesitamos que $B_{U_1} \subseteq U_2$, es decir, que los vectores de $B_{U_1}$ sean solución de la ecuacion que define $U_2$, es decir:
		      \begin{equation*}
			      \left.
			      \begin {array}{rcl}
			      1 + 1 - \alpha^2 + 2 & = & 0 \\
			      1 + \alpha + 1 - \alpha - 2 & = & 0
			      \end {array}
			      \right\} \Rightarrow \left.
			      \begin {array}{rcl}
			      \alpha & = & \pm 2 \\
			      0 & = & 0
			      \end {array}
			      \right\} \text{ Como } \alpha \neq -2 \Rightarrow \alpha = 2
		      \end{equation*}
	\end{itemize}
	Por tanto para $\alpha = 2$ se cumple que $U_1 = U_2$.
\end{ejercicio}



\begin{ejercicio} Calcular una base y la dimensión de los subespacios vectoriales que aparecen en los apartados \( c \), \( g \) y \( n \) del ejercicio 4.
	\begin{itemize}
		\item[c)] $V = M_2(\R), \quad U = \left\{ \begin{pmatrix}
					a  & b \\
					-b & c
				\end{pmatrix} : a,b,c, \in \R \right\}$
			\begin{equation*}
				\forall u \in U, u = \begin{pmatrix}
					a  & b \\
					-b & c
				\end{pmatrix} = a \begin{pmatrix}
					1 & 0 \\
					0 & 0
				\end{pmatrix} + b \begin{pmatrix}
					0  & 1 \\
					-1 & 0
				\end{pmatrix} + c \begin{pmatrix}
					0 & 0 \\
					0 & 1
				\end{pmatrix}
			\end{equation*}
			Además
			\begin{equation*}
				a \begin{pmatrix}
					1 & 0 \\
					0 & 0
				\end{pmatrix}
				+ b \begin{pmatrix}
					0  & 1 \\
					-1 & 0
				\end{pmatrix}
				+ c \begin{pmatrix}
					0 & 0 \\
					0 & 1
				\end{pmatrix} = \begin{pmatrix}
					0 & 0 \\
					0 & 0
				\end{pmatrix} \Rightarrow a = b = c = 0 \Rightarrow \text{ L.I}
			\end{equation*}
			Por tanto una base de $U$ es $\left\{ \begin{pmatrix}
					1 & 0 \\
					0 & 0
				\end{pmatrix}, \begin{pmatrix}
					0  & 1 \\
					-1 & 0
				\end{pmatrix}, \begin{pmatrix}
					0 & 0 \\
					0 & 1
				\end{pmatrix} \right\}$ y $\dim(U) = 3$.
		\item[g)] $V = \mathbb{R}^5, \quad U = \{ (x,y,z,t,s) \in \R^5 / -y = 2x + z \}$
			\begin{equation*}
				\forall u \in U, u = \begin{pmatrix}
					x       \\
					-2x - z \\
					z       \\
					t       \\
					s
				\end{pmatrix} = x \begin{pmatrix}
					1  \\
					-2 \\
					0  \\
					0  \\
					0
				\end{pmatrix} + z \begin{pmatrix}
					0  \\
					-1 \\
					1  \\
					0  \\
					0
				\end{pmatrix} + t \begin{pmatrix}
					0 \\
					0 \\
					0 \\
					1 \\
					0
				\end{pmatrix} + s \begin{pmatrix}
					0 \\
					0 \\
					0 \\
					0 \\
					1
				\end{pmatrix}
			\end{equation*}
			Es un sistema de generadores de $U$, ademas es linealmente independiente, (det $= -2 \neq 0$) por
			tanto, $\beta = \{ (1,-2,0,0,0), (0,-1,1,0,0), (0,0,0,1,0), (0,0,0,0,1) \}$ es una base de $U$ y $\dim(U) = 4$.
		\item[n)] $V = M_n(\mathbb{K})$, $U = \{ A \in M_n(\mathbb{K}) \mid A\text{ es diagonal} \} $
			\begin{equation*}
				\forall u \in U, u = \begin{pmatrix}
					a_{11} & 0      & \cdots & 0      \\
					0      & a_{22} & \cdots & 0      \\
					\vdots & \vdots & \ddots & \vdots \\
					0      & 0      & \cdots & a_{nn}
				\end{pmatrix} = a_{11} \begin{pmatrix}
					1      & 0      & \cdots & 0      \\
					0      & 0      & \cdots & 0      \\
					\vdots & \vdots & \ddots & \vdots \\
					0      & 0      & \cdots & 0
				\end{pmatrix} +
				\cdots + a_{nn} \begin{pmatrix}
					0      & 0      & \cdots & 0      \\
					0      & 0      & \cdots & 0      \\
					\vdots & \vdots & \ddots & \vdots \\
					0      & 0      & \cdots & 1
				\end{pmatrix}
			\end{equation*}
			Es un sistema de generadores de $U$, ademas es claramente linearmente independiente, por tanto, $\beta = \left\{ \begin{pmatrix}
					1      & 0      & \cdots & 0      \\
					0      & 0      & \cdots & 0      \\
					\vdots & \vdots & \ddots & \vdots \\
					0      & 0      & \cdots & 0
				\end{pmatrix}, \cdots, \begin{pmatrix}
					0      & 0      & \cdots & 0      \\
					0      & 0      & \cdots & 0      \\
					\vdots & \vdots & \ddots & \vdots \\
					0      & 0      & \cdots & 1
				\end{pmatrix} \right\}$ es una base de $U$ y $\dim(U) = n$.
	\end{itemize}
\end{ejercicio}


\begin{ejercicio}Para cada uno de los subespacios vectoriales \( U \) del espacio vectorial \( V \) que aparecen a continuación calcular una base, la dimensión y un subespacio complementario:
	\begin{enumerate}
		\item \( U = L\{ (1, -2, 1, 0), (2, 3, -2, 1), (4, -1, 0, 1) \}, \quad V = \mathbb{R}^4 \).

		\item \( U = \{ (x,y,z,t) \in \mathbb{R}^4 \mid 2x - y + z = 0, x + y + z + t = 0 \}, \quad V = \mathbb{R}^4 \).
		\item \( U = \{ p(x) \in \mathbb{R}_3[x] \mid p'(1) = 0 \}, \quad V = \mathbb{R}_3[x] \} \) ( \( p'(1) \) es la derivada de \( p(x) \) en \( x = 1 \)).
		\item \( U = \{ p(x) \in \mathbb{R}_2[x] \mid \int_{0}^{1} p(x) \, dx = 0 \}, \quad V = \mathbb{R}_2[x] \).
	\end{enumerate}
\end{ejercicio}



\begin{ejercicio} Sea \( K \) un cuerpo en el que \( 2 \neq 0 \). Calcular una base y la dimensión de los subespacios de matrices \( S_n(K) \) y \( A_n(K) \) (estudiar primero los casos particulares \( n = 2 \) y \( n = 3 \)).
\end{ejercicio}


\begin{ejercicio} Sea \( K \) un cuerpo. Demostrar que si \( S \) es una familia de \( K[x] \) que no contiene dos polinomios con el mismo grado, entonces \( S \) es linealmente independiente. Deducir que si \( \beta = \{ p_0(x), \ldots, p_n(x) \} \) es una familia de \( K[x] \) de forma que \( \text{grado}(p_i(x)) = i \), para cada \( i = 0, \ldots, n \), entonces \( \beta \) es una base de \( K_n[x] \).
\end{ejercicio}


\begin{ejercicio} Encontrar bases \( \beta \) y \( \beta' \) del espacio vectorial \( \mathbb{R}_2[x] \) en las que el polinomio \( p(x) = x + 1 \) cumpla que \( p(x)_\beta = (1, 0, 0)^T \) y \( p(x)_{\beta'} = (1, 1, 0)^T \).
\end{ejercicio}


\begin{ejercicio} En el espacio vectorial \( V = \mathbb{R}_2[x] \) se consideran las bases \( \beta = \{1, 1 + x, 1 + x + x^2\} \) y \( \beta' = \{1, x, x^2\} \). ¿Qué relación existe entre las coordenadas de un polinomio \( p(x) \) en \( \mathbb{R}_2[x] \) con respecto a \( \beta \) y \( \beta' \)? Encontrar \( p(x) \) en \( \mathbb{R}_2[x] \) tal que \( p(x)_\beta = (1, -2, 4)^T \).
\end{ejercicio}


\begin{ejercicio} En el espacio vectorial \( M_2(\mathbb{C}) \) se consideran las matrices:
	\[
		A = \begin{pmatrix} i & 0 \\ 1 & -i \end{pmatrix}, \quad
		B = \begin{pmatrix} -1 & -i \\ 1 & 2i \end{pmatrix}, \quad
		C = \begin{pmatrix} \alpha & 0 \\ 1 & -i \end{pmatrix}.
	\]
	¿Para qué números \( \alpha \in \mathbb{C} \) el subespacio \( U = L(A, B, C) \) de \( M_2(\mathbb{C}) \) tiene dimensión 2? Para tales valores calcular una base de \( U \) y las coordenadas de la matriz
	\[
		v = \begin{pmatrix} 2i - 1 & -i \\ 3 & 0 \end{pmatrix}.
	\]
\end{ejercicio}

\begin{ejercicio}
    Se consideran los subespacios vectoriales de \( \mathbb{R}^4 \) dados por:
\[
\begin{aligned}
U_1 &= \cc{L}\{(3,6,1,0), (1,0,-1,2), (2,3,0,1)\}, \\
U_2 &= \cc{L}\{(2,0,-1,3), (3,3,-2,4)\}, \\
U_3 &= \{(x,y,z,t) \in \mathbb{R}^4 \mid x - z = 0\}, \\
U_4 &= \{(x,y,z,t) \in \mathbb{R}^4 \mid x - 2y + t = 0, \; 3x + y + 6z= 0\}.
\end{aligned}
\]

\begin{enumerate}
    \item[a)] Calcular una base y la dimensión de \( U_i \), para \( i = 1, 2, 3, 4 \).
    \item[b)] Calcular una base y la dimensión de \( U_1 \cap U_2 \), \( U_2 \cap U_4 \) y \( U_3 \cap U_4 \).
    \item[c)] Calcular una base y la dimensión de \( U_1 + U_2 \), \( U_2 + U_4 \) y \( U_3 + U_4 \).
\end{enumerate}

\begin{enumerate}
    \item[a)] Trabajamos en primer lugar, con $U_1$.
    \begin{equation*}
        \begin{pmatrix}
            3 & 1 & 2 \\
            6 & 0 & 3 \\
            1 & -1 & 0 \\
            0 & 2 & 1
        \end{pmatrix} = A \quad \begin{vmatrix}
            3 & 1 \\
            6 & 0
        \end{vmatrix} \neq 0 \quad \begin{vmatrix}
            3 & 1 & 2 \\
            6 & 0 & 3 \\
            1 & -1 & 0
        \end{vmatrix} = 0 \quad \begin{vmatrix}
            3 & 1 & 2 \\
            6 & 0 & 3 \\
            0 & 2 & 1
        \end{vmatrix} = 0
    \end{equation*} 
    Por tanto, $rg(A) = 2$, con $(2,3,0,1)$ dependiente de $(3,6,1,0)$ y $(1,0,-1,2)$, por lo que una base de $U_1$ es $\{(3,6,1,0), (1,0,-1,2)\}$ y $\dim(U_1) = 2$.
    \\ \\
    Trabajamos ahora con $U_2$:
    \begin{equation*}
        \begin{pmatrix}
            2 & 3 \\
            0 & 3 \\
            -1 & -2 \\
            3 & 4
        \end{pmatrix} = A \quad \begin{vmatrix}
            2 & 3 \\
            0 & 3
        \end{vmatrix} \neq 0 \Rightarrow rg(A) = 2
    \end{equation*}
    Son linealmente independientes, por lo que una base de $U_2$ es $\{(2,0,-1,3), (3,3,-2,4)\}$ y $\dim(U_2) = 2$.
    \\ \\
    Trabajamos ahora con $U_3$, como $\dim(U_3) = 3$, entonces una base de $U_3$ es $\{(1,0,1,0), (0,1,0,0), (0,0,0,1)\}$.
    Como son linealmente independientes y la dimensión es 3, forman una base.
    \\ \\
    Trabajamos ahora con $U_4$, como $\dim(U_4) = 2$, entonces obtenemos unas ecuaciones paramétricas de $U_4$:
    \begin{equation*}
        \begin{cases}
            x = 2y - t \\
            z = -3y + 2t
        \end{cases}
    \end{equation*}
    Por tanto, una base de $U_4$ es $\{(1,3, -1, 5), (0, 6, -1, 12)\}$ y $\dim(U_4) = 2$.
    \item[b)] Calculamos ahora una base de $U_1 \cap U_2$:
    Sea $(x,y,z,t) \in U_1$: 
    \begin{equation*}
        \begin{vmatrix}
            3 & 1 & x \\
            6 & 0 & y \\
            1 & -1 & z \\
        \end{vmatrix} = 0 \Rightarrow -3x -3z + 2y = 0 \quad 
        \begin{vmatrix}
            3 & 1 & x \\
            6 & 0 & y \\
            0 & 2 & t 
        \end{vmatrix} = 0 \Rightarrow 2x - t - y = 0
    \end{equation*}
    Por tanto, $U_1 = \{(x,y,z,t) \in \R^4 : \begin{cases}
        -3x + 2y - 3z = 0 \\
        2x - y - t = 0
    \end{cases}\}$
    Obtenemos ahora unas ecuaciones implícitas de $U_2$. Sea $(x,y,z,t) \in U_2$:
    \begin{equation*}
        \begin{vmatrix}
            2 & 3 & x \\
            0 & 3 & y \\
            -1 & -2 & z
        \end{vmatrix} = 0 \Rightarrow 3x + y + 6z = 0 \quad 
        \begin{vmatrix}
            2 & 3 & x \\
            0 & 3 & y \\
            3 & 4 & t
        \end{vmatrix} = 0 \Rightarrow -9x + y + 6t = 0
    \end{equation*}
    Por tanto, $U_2 = \{(x,y,z,t) \in \R^4 : \begin{cases}
        3x + y + 6z = 0 \\
        -9x + y + 6t = 0
    \end{cases}\}$
    Por tanto, $U_1 \cap U_2 = \{(x,y,z,t) \in \R^4 : \begin{cases}
        -3x + 2y - 3z = 0 \\
        2x - y - t = 0 \\
        3x + y + 6z = 0 \\
        -9x + y + 6t = 0
    \end{cases}\}$
    \begin{equation*}
        \begin{vmatrix}
            -3 & 2 & -3 & 0 \\
            2 & -1 & 0 & -1 \\
            3 & 1 & 6 & 0 \\
            -9 & 1 & 0 & 6
        \end{vmatrix} = 0 
    \end{equation*}
    Las tres primeras filas son linealmente independientes, por tanto, 
    $U_1 \cap U_2 = \{(x,y,z,t) \in \R^4 : \begin{cases}
        -3x + 2y - 3z = 0 \\
        2x - y - t = 0 \\
        3x + y + 6z = 0 
    \end{cases}\}$
    Y $\dim(U_1 \cap U_2) = 1$. Para obtener una base, obtenemos las ecuaciones paramétricas 
    \begin{equation*}
        \begin{cases}
            -3x + 2y - 3z = 0 \\
        2x - y - t = 0 \\
        3x + y + 6z = 0 
        \end{cases} \Rightarrow x = x, y = 3x/5, z = -3x/5, t = 7x/5
    \end{equation*} Por tanto, una base de $U_1 \cap U_2$ es $\{(5, 3, -3, 7)\}$.
    \\ \\
    Calculamos ahora una base de $U_2 \cap U_4$:
    \begin{align*}
        U_2 &= \{(x,y,z,t) \in \R^4 : \begin{cases}
            3x + y + 6z = 0 \\
            -9x + y + 6t = 0
        \end{cases}\} \\
        U_4 &= \{(x,y,z,t) \in \R^4 : \begin{cases}
            x - 2y + t = 0 \\
            3x + y + 6z = 0
        \end{cases}\}
    \end{align*}
    Por tanto, $U_2 \cap U_4 = \{(x,y,z,t) \in \R^4 : \begin{cases}
        3x + y + 6z = 0 \\
        -9x + y + 6t = 0 \\
        x - 2y + t = 0
    \end{cases}\}$
    \begin{equation*}
        \begin{pmatrix}
            3 & 1 & 6 & 0 \\
            -9 & 1 & 0 & 6 \\
            1 & -2 & 0 & 1
        \end{pmatrix} = A \Rightarrow \begin{vmatrix}
            3 & 1 & 6 \\
            -9 & 1 & 0 \\
            1 & -2 & 0
        \end{vmatrix} \neq 0 \Rightarrow rg(A) = 3
    \end{equation*}
    Por tanto, $\dim_\R(U_2 \cap U_4) = 1$. Para obtener una base, obtenemos las ecuaciones paramétricas
    \begin{equation*}
        \begin{cases}
            3x + y + 6z = 0 \\
            -9x + y + 6t = 0 \\
            x - 2y + t = 0
        \end{cases} \Rightarrow x = 13/17t, y = 15/17t, z = -9/17t, t = t
    \end{equation*}
    Por tanto, una base de $U_2 \cap U_4$ es $\{(13, 15, -9, 17)\}$.
    \\ \\
    Calculamos ahora una base de $U_3 \cap U_4 = \{(x,y,z,t) \in \R^4 : \begin{cases}
        x - z = 0 \\
        x - 2y + t = 0 \\
        3x + y + 6z = 0
    \end{cases}\}$
    \begin{equation*}
        \begin{pmatrix}
            1 & 0 & -1 & 0 \\
            1 & -2 & 0 & 1 \\
            3 & 1 & 6 & 0
        \end{pmatrix} = A \Rightarrow \begin{vmatrix}
            0 & -1 & 0 \\
            -2 & 0 & 1 \\
            1 & 6 & 0
        \end{vmatrix} \neq 0 \Rightarrow rg(A) = 3
    \end{equation*}
    Por tanto, $\dim_\R(U_3 \cap U_4) = 1$. Para obtener una base, obtenemos las ecuaciones paramétricas
    \begin{equation*}
        \begin{cases}
            x = x \\
            y = -9x \\
            z = x \\
            t = -19x
        \end{cases}
    \end{equation*}
    Por tanto, una base de $U_3 \cap U_4$ es $\{(1, -9, 1, -19)\}$.
    \item[c)] Calculamos ahora una base de $U_1 + U_2$:
    Como $\dim(U_1) = 2$ y $\dim(U_2) = 2$, entonces $\dim(U_1 + U_2) = 3$. Para obtener una base:
    \begin{equation*}
        U_1 + U_2 = \mathcal{L}\left\{(3,6,1,0), (1,0,-1,2), (2,0,-1,3), (3,3,-2,4)\right\}
    \end{equation*}
    \begin{equation*}
        \begin{vmatrix}
            3 & 1 & 2 & 3 \\
            6 & 0 & 0 & 3 \\
            1 & -1 & -1 & -2 \\
            0 & 2 & 3 & 4
        \end{vmatrix} = 0 \text { y } \begin{vmatrix}
            3 & 1 & 2 \\ 
            6 & 0 & 0 \\
            1 & -1 & -1 
        \end{vmatrix} = -6 \neq 0
    \end{equation*}
    Por tanto el 4º vector es combinación lineal de los 3 primeros, por lo que una base de $U_1 + U_2$ es $\{(3,6,1,0), (1,0,-1,2), (2,0,-1,3)\}$.
    \\ \\
    Calculamos ahora una base de $U_2 + U_4$:
    Como $\dim(U_2) = 2$ y $\dim(U_4) = 2$, entonces $\dim(U_2 + U_4) = 3$. Para obtener una base:
    \begin{equation*}
        U_2 + U_4 = \mathcal{L}\left\{(2,0,-1,3), (3,3,-2,4), (1,3,-1,5), (0,6,-1,12)\right\}
    \end{equation*}
    \begin{equation*}
        \begin{vmatrix}
            2 & 3 & 1 & 0 \\
            0 & 3 & 3 & 6 \\
            -1 & -2 & -1 & -1 \\
            3 & 4 & 5 & 12
        \end{vmatrix} = 0 \text { y } \begin{vmatrix}
            2 & 3 & 1 \\
            0 & 3 & 3 \\
            3 & 4 & 5
        \end{vmatrix} \neq 0 
    \end{equation*}
    Por tanto el 4º vector es combinación lineal de los 3 primeros, por lo que una base de $U_2 + U_4$ es $\{(2,0,-1,3), (3,3,-2,4), (1,3,-1,5)\}$.
    \\ \\
    Calculamos ahora una base de $U_3 + U_4$:
    Como $\dim(U_3) = 3$ y $\dim(U_4) = 4$, entonces $\dim(U_3 + U_4) = 4$. Para obtener una base:
    \begin{equation*}
        U_3 + U_4 = \mathcal{L}\left\{(1,0,1,0), (0,1,0,0), (0,0,0,1), (1,3,-1,5), (0,6,-1,12)\right\}
    \end{equation*}
    Veamos cuales son linealmente independientes:
    \begin{equation*}
        \begin{vmatrix}
            1 & 0 & 0 & 1 & 0 \\
            0 & 1 & 0 & 3 & 6 \\
            1 & 0 & 0 & -1 & -1 \\
            0 & 0 & 1 & 5 & 12
        \end{vmatrix} = 0 \text { y } \begin{vmatrix}
            1 & 0 & 0 & 1 \\
            0 & 1 & 0 & 3 \\
            1 & 0 & 0 & -1 \\
            0 & 0 & 1 & 5
        \end{vmatrix} \neq 0
    \end{equation*}
    Por tanto el 5º vector es combinación lineal de los 4 primeros, por lo que una base de $U_3 + U_4$ es $\{(1,0,1,0), (0,1,0,0), (0,0,0,1), (1,3,-1,5)\}$.
\end{enumerate}

\end{ejercicio}

\begin{ejercicio}
	Sea $A = (a_{ij})$ una matriz en $M_n(\mathbb{R})$. Se define la traza de $A$ como:
	\[
		\text{tr}(A) = \sum_{i=1}^n a_{ii}.
	\]
	Para cada $n \in \mathbb{N}$ sea $U_n = \{A \in M_n(\mathbb{R}) | \text{tr}(A) = 0\}$. Se pide lo siguiente:
	\begin{enumerate}
		\item \textit{Demostrar que $U_n$ es un subespacio vectorial de $M_n(\mathbb{R})$.}
		      Sea $x,y \in \R$, $A,B \in U_n$, entonces:
		      \begin{equation*}
			      xA + yB = x\begin{pmatrix}
				      0      & a_{12} & \cdots & a_{1n} \\
				      a_{21} & 0      & \cdots & a_{2n} \\
				      \vdots & \vdots & \ddots & \vdots \\
				      a_{n1} & a_{n2} & \cdots & 0
			      \end{pmatrix} + y\begin{pmatrix}
				      0      & b_{12} & \cdots & b_{1n} \\
				      b_{21} & 0      & \cdots & b_{2n} \\
				      \vdots & \vdots & \ddots & \vdots \\
				      b_{n1} & b_{n2} & \cdots & 0
			      \end{pmatrix}
		      \end{equation*}
		      \begin{equation*}
			      = \begin{pmatrix}
				      0                 & xa_{12} + yb_{12} & \cdots & xa_{1n} + yb_{1n} \\
				      xa_{21} + yb_{21} & 0                 & \cdots & xa_{2n} + yb_{2n} \\
				      \vdots            & \vdots            & \ddots & \vdots            \\
				      xa_{n1} + yb_{n1} & xa_{n2} + yb_{n2} & \cdots & 0
			      \end{pmatrix}
			      \in U_n
		      \end{equation*}
		\item \textit{Calcular una base y la dimensión de $U_n$ cuando $n = 2, 3$.}
		      Sea $A \in U_2$, entonces:
		      \begin{equation*}
			      A = \begin{pmatrix}
				      a_{11} & a_{12}  \\
				      a_{21} & -a_{11}
			      \end{pmatrix} \Rightarrow \text{tr}(A) = 0 \Rightarrow a_{11} = 0
		      \end{equation*}
		      Por tanto, una base de $U_2$ es $\left\{ \begin{pmatrix}
				      1 & 0  \\
				      0 & -1
			      \end{pmatrix}, \begin{pmatrix}
				      0 & 1 \\
				      0 & 0
			      \end{pmatrix}, \begin{pmatrix}
				      0 & 0 \\
				      1 & 0
			      \end{pmatrix} \right\}$ y $\dim(U_2) = 3$.
		      \\ \\
		      Sea ahora $B \in U_3$, entonces:
		      \begin{equation*}
			      B = \begin{pmatrix}
				      a_{11} & a_{12} & a_{13}           \\
				      a_{21} & a_{22} & a_{23}           \\
				      a_{23} & a_{32} & - a_{11} -a_{22}
			      \end{pmatrix} \Rightarrow \text{tr}(B) = 0 \Rightarrow a_{11} + a_{22} - a_{11} - a_{22} = 0 \Rightarrow 0 = 0
		      \end{equation*}
		      Por tanto, una base de $U_3$ es  
                \begin{align*}
                    \mathcal{U} = \{
                    &\begin{pmatrix}
				      1 & 0 & 0  \\
				      0 & 0 & 0  \\
				      0 & 0 & -1
                    \end{pmatrix}, 
                    \begin{pmatrix}
                        0 & 1 & 0 \\
                        0 & 0 & 0 \\
                        0 & 0 & 0
                    \end{pmatrix}, 
                    \begin{pmatrix}
                        0 & 0 & 1 \\
                        0 & 0 & 0 \\
                        0 & 0 & 0
                    \end{pmatrix}, 
                    \begin{pmatrix}
                        0 & 0 & 0 \\
                        1 & 0 & 0 \\
                        0 & 0 & 0
                    \end{pmatrix}, \\ 
                    &
                    \begin{pmatrix}
                        0 & 0 & 0  \\
                        0 & 1 & 0  \\
                        0 & 0 & -1
                    \end{pmatrix}, 
                    \begin{pmatrix}
                        0 & 0 & 0 \\
                        0 & 0 & 1 \\
                        0 & 0 & 0
                    \end{pmatrix}, 
                    \begin{pmatrix}
                        0 & 0 & 0 \\
                        0 & 0 & 0 \\
                        1 & 0 & 0
                    \end{pmatrix}, 
                    \begin{pmatrix}
                        0 & 0 & 0 \\
                        0 & 0 & 0 \\
                        0 & 1 & 0
                    \end{pmatrix}\}
                   \text{ y }  \dim(U_3) = 8.
                \end{align*} 
		\item \textit{Calcular $U_2 \cap S_2(\mathbb{R})$ y $U_2 + S_2(\mathbb{R})$. ¿Es cierto que $M_2(\mathbb{R}) = U_2 \oplus S_2(\mathbb{R})$?}
	\end{enumerate}
\end{ejercicio}

\begin{ejercicio}
	Decidir de forma razonada si las siguientes afirmaciones son verdaderas o falsas:
	\begin{enumerate}
		\item \textit{Existe en $\mathbb{R}$ una estructura de espacio vectorial complejo.}
		\item \textit{Si $U$ es un subespacio vectorial de $V$ entonces el conjunto $V - U$ es un subespacio vectorial de $V$.}
		\item \textit{Si $K$ es un cuerpo entonces los únicos subespacios vectoriales de $K$ son los impropios.}
		\item \textit{En un espacio vectorial $V$, si dos planos vectoriales no son iguales entonces su intersección es una recta o el vector nulo.}
		\item \textit{En un espacio vectorial $V$, la suma de dos rectas vectoriales es un plano vectorial.}
		\item \textit{Si $V$ es un espacio vectorial y $U$ es un subespacio vectorial suyo entonces $U + U = U$.}
		\item \textit{El espacio vectorial real $F(\mathbb{R}, \mathbb{R})$ es finitamente generado.}
		\item \textit{$\mathbb{R}$ no es finitamente generado como espacio vectorial sobre $\mathbb{Q}$.}
	\end{enumerate}
\end{ejercicio}

\begin{ejercicio}
	Dado $k \in \mathbb{R}$, consideramos en $\mathbb{R}^4$ el subespacio:
	\[
		U_k = \cc{L}\left\{ (0, -1, k, 3), (0, k, -2 - k, 3), (k-2, -1, -2, 3) \right\}.
	\]
	\begin{enumerate}
		\item Calcular $\dim_{\mathbb{R}}(U_k)$ en función de $k$. Determinar una base y unas ecuaciones cartesianas de $U_k$ para cada $k \in \mathbb{R}$.
		\item Para $k$ con $\dim_{\mathbb{R}}(U_k) = 2$, encontrar un subespacio $W$ de $\mathbb{R}^4$ tal que $\mathbb{R}^4 = U_k \oplus W$. Determinar unas ecuaciones cartesianas para $W$.
	\end{enumerate}
\end{ejercicio}
\newpage

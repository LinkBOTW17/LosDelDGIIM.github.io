\section{Espacio dual}
\begin{ejercicio}
	Sea V un espacio vectorial sobre $K$ finitamente generado. Demostrar que si $v \in V$ y $v \neq 0$
	entonces existe $\varphi \in V^{*}$ tal que $\varphi(v) \neq 0$. ¿ Es $\varphi$ única es estas condiciones?
	\\ \\ Sea $B=\{v_1,...,v_n\}$ una base de $V$ y $v \in V$, entonces, si $v \neq 0$ se puede ampliar una base $B=\{v=v_1,...,v_n\}$ y tomando
	$\varphi^1 \in V^{*}$ se tiene que $\varphi^1(v) = 1 \neq 0$. Por tanto, $\varphi^1$ existe. \\ \\
	Además se tiene que $\varphi^1$ no es única ya que tomando por ejemplo, otra forma lineal tal que $\psi(v) = k \neq 0$
	y $\psi$ diferente de $\varphi^1$ en otros vectores de la base $B$, se tiene que $\psi$ también cumple las condiciones y es
	diferente de $\varphi^1$.
\end{ejercicio}

\begin{ejercicio}
	Calcular la base dual de la base $B$ del espacio $V$ en estos casos:
	\begin{itemize}
		\item[\textit{a})] $B = \{(1,0,0),(1,1,0),(1,1,1)\}, \quad V = R^3$
			Para calcular la base dual $B^{*}$, de la base $B$ de $V$, necesitamos encontrar las formas lineales $\varphi^1,\varphi^2,\varphi^3$
			tales que $\varphi^i(v_j) = \delta_{ij}$, es decir, que $\varphi^i(v_j) = 1$ si $i=j$ y $\varphi^i(v_j) = 0$ si $i \neq j$. \\ \\
			Dada la base $B=\{(1,0,0),(1,1,0),(1,1,1)\}$, se tiene que:
			\begin{align*}
				\varphi^1(1,0,0) & = 1, \quad \varphi^1(1,1,0) = 0, \quad \varphi^1(1,1,1) = 0 \\
				\varphi^2(1,0,0) & = 0, \quad \varphi^2(1,1,0) = 1, \quad \varphi^2(1,1,1) = 0 \\
				\varphi^3(1,0,0) & = 0, \quad \varphi^3(1,1,0) = 0, \quad \varphi^3(1,1,1) = 1
			\end{align*}
			Tenemos que resolver el siguiente sistema de ecuaciones:
			\begin{align*}
				\varphi^1(1,0,0)= a_1 * 1 + a_2 * 0 + a_3 * 0 & = 1 \\
				\varphi^1(1,1,0)= a_1 * 1 + a_2 * 1 + a_3 * 0 & = 0 \\
				\varphi^1(1,1,1)= a_1 * 1 + a_2 * 1 + a_3 * 1 & = 0
			\end{align*}
			Resolviendo este mismo sistema para $\varphi^2$ y $\varphi^3$, se obtiene que la base dual $B^{*}$ de la base $B$ está formada por:
			\begin{align*}
				\varphi^1 (x,y,z) & = x-y \\
				\varphi^2 (x,y,z) & = y-z \\
				\varphi^3 (x,y,z) & = z
			\end{align*}
			Si nos fijamos bien, podemos ver que lo que hemos estado haciendo realmente es imponer lo siguiente:
			\begin{equation*}
				\begin{pmatrix}
					a_1 & a_2 & a_3 \\
					b_1 & b_2 & b_3 \\
					c_1 & c_2 & c_3
				\end{pmatrix} \cdot \begin{pmatrix}
					1 & 1 & 1 \\
					0 & 1 & 1 \\
					0 & 0 & 1
				\end{pmatrix} = \begin{pmatrix}
					1 & 0 & 0 \\
					0 & 1 & 0 \\
					0 & 0 & 1
				\end{pmatrix}
			\end{equation*}
			Así que en adelante, simplemente haremos la inversa de la matriz con las coordenadas de la base $B$ expresadas por columnas.
		\item[\textit{b})] $B = \{ (i,0),(0,i)\}, \quad V = \C^2$ \\ \\
			Sea $B_u = \{ (1,0),(0,1)\}$ la base usual de $\C^2$, y sea $B_u^{*} = \{ \varphi^1,\varphi^2\}$ la base dual de $B_u$, entonces:
			\\ \\
			Sea $B^* = \{ \psi ^1,\psi ^2\}$ la base dual de $B$, entonces: \\ \\
			\begin{equation*}
				\begin{pmatrix}
					a_1 & a_2 \\
					b_1 & b_2
				\end{pmatrix} \cdot \begin{pmatrix}
					i & 0 \\
					0 & i
				\end{pmatrix} = \begin{pmatrix}
					1 & 0 \\
					0 & 1
				\end{pmatrix} \Rightarrow \begin{pmatrix}
					a_1 & a_2 \\
					b_1 & b_2
				\end{pmatrix} = \begin{pmatrix}
					i & 0 \\
					0 & i
				\end{pmatrix}^{-1} = \begin{pmatrix}
					-i & 0  \\
					0  & -i
				\end{pmatrix}
			\end{equation*}
			Por tanto, tenemos que:
			\begin{align*}
				\psi^1 & = -i \varphi^1 \Rightarrow \psi^1(x,y) = -i x \\
				\psi^2 & = -i \varphi^2 \Rightarrow \psi^2(x,y) = -i x
			\end{align*}
		\item[\textit{c})] $B = \{ 1, 1+x, 1 +x^2, 1 +x^3\}, \quad V = \R_3[x]$ \\ \\
			Sea $B_u = \{ 1,x,x^2,x^3\}$ la base usual de $\R_3[x]$, y sea $B_u^{*} = \{ \varphi^1,\varphi^2,\varphi^3,\varphi^4\}$ la base dual de $B_u$, entonces:
			\\ \\
			Sea $B^* = \{ \psi ^1,\psi ^2,\psi ^3,\psi ^4\}$ la base dual de $B$, entonces: \\ \\
			\begin{align*}
				\begin{pmatrix}
					a_1 & a_2 & a_3 & a_4 \\
					b_1 & b_2 & b_3 & b_4 \\
					c_1 & c_2 & c_3 & c_4 \\
					d_1 & d_2 & d_3 & d_4
				\end{pmatrix} \cdot \begin{pmatrix}
					1 & 1 & 1 & 1 \\
					0 & 1 & 0 & 0 \\
					0 & 0 & 1 & 0 \\
					0 & 0 & 0 & 1
				\end{pmatrix} &= \begin{pmatrix}
					1 & 0 & 0 & 0 \\
					0 & 1 & 0 & 0 \\
					0 & 0 & 1 & 0 \\
					0 & 0 & 0 & 1
				\end{pmatrix} \\ &\Rightarrow \begin{pmatrix}
					a_1 & a_2 & a_3 & a_4 \\
					b_1 & b_2 & b_3 & b_4 \\
					c_1 & c_2 & c_3 & c_4 \\
					d_1 & d_2 & d_3 & d_4
				\end{pmatrix} = \begin{pmatrix}
					1 & 1 & 1 & 1 \\
					0 & 1 & 0 & 0 \\
					0 & 0 & 1 & 0 \\
					0 & 0 & 0 & 1
				\end{pmatrix}^{-1}
			\end{align*}
			Calculamos la inversa de la matriz:
			\begin{equation*}
				\begin{pmatrix}
					1 & 1 & 1 & 1 \\
					0 & 1 & 0 & 0 \\
					0 & 0 & 1 & 0 \\
					0 & 0 & 0 & 1
				\end{pmatrix}^{-1} = \begin{pmatrix}
					1 & -1 & -1 & -1 \\
					0 & 1  & 0  & 0  \\
					0 & 0  & 1  & 0  \\
					0 & 0  & 0  & 1
				\end{pmatrix}
			\end{equation*}
			Por tanto, tenemos que:
			\begin{align*}
				\psi^1 & = \varphi^1 - \varphi^2 - \varphi^3 - \varphi^4 \Rightarrow \psi^1(x) = 1 - x - x^2 - x^3 \\
				\psi^2 & = \varphi^2 \Rightarrow \psi^2(x) = x                                                     \\
				\psi^3 & = \varphi^3 \Rightarrow \psi^3(x) = x^2                                                   \\
				\psi^4 & = \varphi^4 \Rightarrow \psi^4(x) = x^3
			\end{align*}
	\end{itemize}
\end{ejercicio}

\begin{ejercicio}
	En el espacio $R_2[x]$ de los polinomios con coeficientes reales y grado menor o igual a 2 se considera la aplicación
	$\varphi : R_2[x] \mapsto \R$ dada por:
	\begin{equation*}
		\varphi(p(x)) = \int_{-1}^{1} p(x)  \,dx
	\end{equation*}
	Se pide lo siguiente:
	\begin{enumerate}
		\item[a)] Demostrar que $\varphi \in (R_2[x])^{*}$. Calcular las coordenadas de $\varphi$ en la base dual de $\{ 1,x,x^2\}$.
			\\ \\
			Comenzamos por demostrar la linealidad de $\varphi$:
			\begin{itemize}
				\item Aditiva: Sean $p(x),q(x) \in R_2[x]$:
				      \begin{align*}
					      \varphi(p(x)+q(x)) &= \int_{-1}^{1} (p(x)+q(x))  \,dx = \int_{-1}^{1} p(x)  \,dx + \int_{-1}^{1} q(x)  \,dx \\&= \varphi(p(x)) + \varphi(q(x))
				      \end{align*}
				\item Homogénea: Sean $p(x) \in R_2[x]$ y $\lambda \in \R$:
				      \begin{equation*}
					      \varphi(\lambda p(x)) = \int_{-1}^{1} (\lambda p(x))  \,dx = \lambda \int_{-1}^{1} p(x)  \,dx = \lambda \varphi(p(x))
				      \end{equation*}
			\end{itemize}
			Como $\varphi$ es lineal, y $\varphi : R_2[x] \mapsto \R$, entonces $\varphi \in (R_2[x])^{*}$. \\ \\
			Vamos a calcular ahora las coordenadas de $\varphi$ en la base dual de $\{ 1,x,x^2\}$, para ello, sea $B = \{ 1,x,x^2\}$ la base de $R_2[x]$, y sea $B^{*} = \{ \varphi^1,\varphi^2,\varphi^3\}$ la base dual de $B$, entonces,
			calculamos $\varphi(1),\varphi(x),\varphi(x^2)$:
			\begin{align*}
				\varphi(1)   & = \int_{-1}^{1} 1  \,dx = 2             \\
				\varphi(x)   & = \int_{-1}^{1} x  \,dx = 0             \\
				\varphi(x^2) & = \int_{-1}^{1} x^2  \,dx = \frac{2}{3}
			\end{align*}
			Y obtenemos que las coordenadas de $\varphi$ en la base dual de $\{ 1,x,x^2\}$ son: $\left(2,0,\frac{2}{3}\right)_{B^{*}}$.
		\item[b)] Construr una base $\overline{B}$ de $(R_2[x])^{*}$ a partir de $\varphi$.
			\begin{equation*}
				\text{Rg}
				\begin{pmatrix}
					2           & 0 & 0 \\
					0           & 1 & 0 \\
					\frac{2}{3} & 0 & 1
				\end{pmatrix} = 3 \Rightarrow \overline{B} = \left\{ \left(2,0,\frac{2}{3}\right),(0,1,0),(0,0,1)\right\}
			\end{equation*}
		\item[c)] Obtener una base $B$ de $R_2[x]$ tal que $B^{*} = \overline{B}$.
			\begin{equation*}
				\begin{pmatrix}
					2 & 0 & \frac{2}{3} \\
					0 & 1 & 0           \\
					0 & 0 & 1
				\end{pmatrix} \cdot \begin{pmatrix}
					a_1 & b_1 & c_1 \\
					a_2 & b_2 & c_2 \\
					a_3 & b_3 & c_3
				\end{pmatrix} = \begin{pmatrix}
					1 & 0 & 0 \\
					0 & 1 & 0 \\
					0 & 0 & 1
				\end{pmatrix} \Rightarrow \begin{pmatrix}
					a_1 & b_1 & c_1 \\
					a_2 & b_2 & c_2 \\
					a_3 & b_3 & c_3
				\end{pmatrix} = \begin{pmatrix}
					\frac{1}{2} & 0 & \frac{-1}{3} \\
					0           & 1 & 0            \\
					0           & 0 & 1
				\end{pmatrix}
			\end{equation*}
			Por tanto, $B = \left\{\frac{1}{2},x,\frac{-1}{3}+x^2\right\}$.
	\end{enumerate}
\end{ejercicio}

\begin{ejercicio}
	Sea $V$ un espacio vectorial sobre $K$ finitamente generado. Dados dos subespacios vectoriales $U$ y $W$
	de $V$, demostrar que:
	\begin{equation*}
		\text{an}(U+W) = \text{an}(U) \cap \text{an}(W), \quad \text{an}(U \cap W) = \text{an}(U) + \text{an}(W)
	\end{equation*}
	Deducir que si $V=U \oplus W$ entonces $V^{*} = \text{an}(U) \oplus \text{an}(W)$.
	Para demostrar que $\text{an}(U+W) = \text{an}(U) \cap \text{an}(W)$, vamos a demostrar que $\text{an}(U+W) \subseteq \text{an}(U) \cap \text{an}(W)$
	y que $\text{an}(U) \cap \text{an}(W) \subseteq \text{an}(U+W)$.
	\begin{itemize}
		\item $\text{an}(U+W) \subseteq \text{an}(U) \cap \text{an}(W)$: Sea $\varphi \in \text{an}(U+W), u \in U, w \in W$, entonces:
		      \begin{align*}
			      \varphi(u+w) &= \varphi(u) + \varphi(w) = 0 \Rightarrow \varphi(u) \\&= -\varphi(w) \Rightarrow  \left\{\begin{array}{cc}
				      \varphi(u) = 0 \Rightarrow \varphi \in \text{an}(U) \\
				      \varphi(w) = 0 \Rightarrow \varphi \in \text{an}(W)
			      \end{array} \right\} \varphi \in \text{an}(U) \cap \text{an}(W)
		      \end{align*}
		\item $\text{an}(U) \cap \text{an}(W) \subseteq \text{an}(U+W)$: Sea $\varphi \in \text{an}(U) \cap \text{an}(W)$, entonces:
		      \begin{equation*}\left.
			      \begin{array}{cc}
				      \varphi(u) \in \text{an}(U) \Rightarrow \varphi(u) = 0 \\
				      \varphi(w) \in \text{an}(W) \Rightarrow \varphi(w) = 0
			      \end{array} \right\} \forall x \in U+W, \exists u \in U, w \in W : x = u+w
		      \end{equation*}
		      Ahora aplicando la linealidad de $\varphi$:
		      \begin{equation*}
			      \varphi(x) = \varphi(u+w) = \varphi(u) + \varphi(w) = 0 \Rightarrow \varphi \in \text{an}(U+W)
		      \end{equation*}
	\end{itemize}
\end{ejercicio}

\begin{ejercicio}
	Sea $V$ un espacio vectorial sobre $K$ finitamente generado. Sabemos que si $\varphi \in V^{*}$ y
	$\varphi \neq \varphi_0,$ entonces Nuc$(\varphi)$ es un hiperplano de $V$. Demostrar que, dado un hiperplano $H$ de $V$,
	existe $\varphi \in V^{*}$ con Nuc$(\varphi) = H$. ¿Qué relación hay entre dos formas lineales $\varphi$ y $\psi$ sobre $V$
	tales que Nuc$(\varphi) = $ Nuc$(\psi) = H$?
\end{ejercicio}

\begin{ejercicio}
	En cada uno de los siguientes casos, obtener unas ecuaciones implícitas para el subespacio $U$ del espacio vectorial
	$V$:
	Estos ejercicios se pueden resolver mediante dos metodos, a continuación se muestra un metodo que se aplica en el apartado a),
	en el resto de ejercicios se aplica el metodo que considero más conveniente. \\ \\
	Mediante la propiedad: $ U = \text{an}(\text{an}(U))$. Esta propiedad se demuestra fácilmente de la siguiente manera:
	\\ \\ Sea $n = \dim(V)$ y sea $U$ un subespacio de $V$ con $\dim_K (U) = m$, entonces, como se tiene $\dim_K(\text{an}(U)) = n-m$:
	\begin{equation*}
		\dim_K(\text{an}(\text{an}(U))) = n - \dim_K(\text{an}(U)) = n - (n-m) = m = \dim_K(U)
	\end{equation*}
	Que coincide con la dimensión del isomorfismo dado por el Teorema de Reflexividad. Por tanto con comprobar la inclusión
	$\Phi_u \subseteq \text{an}(\text{an}(U))$  es suficiente, esta inclusión es trivial ya que $\forall u \in U, \phi \in \text{an}(U)$.
	Se tiene $\Phi_u(\phi) = \phi(u)=0$, esto es $\Phi_u \in \text{an}(\text{an}(U))$.
	\begin{itemize}
		\item[\textit{a})] $U=\mathcal{L}\{(1,-1,1),(2,1,0),(5,-2,3)\}, \quad V=\R^3$ \\ \\
			Sean $B_u = \{ e_1,e_2,e_3\}$ la base usual de $\R^3$ y $B_u^* = \{ \varphi^1,\varphi^2,\varphi^3\}$ la base dual de $B_u$, en primer lugar:
			\begin{equation*}
				\begin{vmatrix}
					1  & 2 & 5  \\
					-1 & 1 & -2 \\
					1  & 0 & 3
				\end{vmatrix} = 0 \quad \land \quad \begin{vmatrix}
					1  & 2 \\
					-1 & 1
				\end{vmatrix} = 3 \neq 0 \Rightarrow U = \mathcal{L}\{(1,-1,1),(2,1,0) \}
			\end{equation*}
			\begin{align*}
				\text{an}(U) = \text{an}(\mathcal{L}\left\{ (1,-1,1),(2,1,0)\right\}) & = \left\{ \varphi \in V^{*} : \begin{array}{ll}
					                                                                                                      \varphi(1,-1,1) = 0 \\
					                                                                                                      \varphi(2,1,0) = 0
				                                                                                                      \end{array}\right\}                   \\
				                                                                      & = \left\{ \varphi \in V^{*} : \begin{array}{ll}
					                                                                                                      \varphi^1 - \varphi^2 + \varphi^3 = 0 \\
					                                                                                                      2\varphi^1 + \varphi^2 = 0
				                                                                                                      \end{array}\right\} \\
				                                                                      & = \mathcal{L} \left\{ \varphi^1 -2 \varphi^2 -3 \varphi^3\right\}
			\end{align*}
			Como hemos visto antes, $U = \text{an}(\text{an}(U))$, por tanto:
			\begin{align*}
				U = \text{an}(\text{an}(U)) & = \left\{ (x,y,z) \in \R^3 : \forall \varphi \in \text{an}(U),\varphi(x,y,z) = 0\right\}          \\
				                            & = \left\{ (x,y,z) \in \R^3 : \varphi^1(x,y,z) -2 \varphi^2(x,y,z) -3 \varphi^3(x,y,z) = 0\right\} \\
				                            & = \left\{ (x,y,z) \in \R^3 : x -2 y -3 z = 0\right\}
			\end{align*}
		\item[\textit{b})] $U=\mathcal{L}\{(-1,1,1,1),(1,-1,1,1)\}, \quad V=\R^4$ \\ \\
			En primer lugar, veamos como podemos ampliar la base de $U$ a una base de $\R^4$, para ello, sean dos vectores $(1,0,0,0),(0,0,1,0)$
			de $\R^4$, veamos que $(-1,1,1,1),(1,-1,1,1),(1,0,0,0),(0,0,1,0)$ son linealmente independientes:
			\begin{equation*}
				\begin{vmatrix}
					1 & 0 & -1 & 1  \\
					0 & 0 & 1  & -1 \\
					0 & 1 & 1  & 1  \\
					0 & 0 & 1  & 1
				\end{vmatrix} = -2 \neq 0 \Rightarrow \text{ Son L.I } \Rightarrow \text{ forman una base de } \R^4
			\end{equation*}
			Por tanto, tenemos que $B=\{(1,0,0,0),(0,0,1,0),(-1,1,1,1),(1,-1,1,1)\}$ por tanto, sea ahora $B^* = \{ \varphi^1,\varphi^2,\varphi^3,\varphi^4\}$ la base dual de $B$, entonces, deducimos que
			$\{\varphi^3,\varphi^4\}$ es una base de an($U$), por tanto:
			\begin{equation*}
				\begin{pmatrix}
					a_1 & a_2 & a_3 & a_4 \\
					b_1 & b_2 & b_3 & b_4 \\
					c_1 & c_2 & c_3 & c_4 \\
					d_1 & d_2 & d_3 & d_4
				\end{pmatrix}
				\cdot
				\begin{pmatrix}
					-1 & 1  & 1 & 0 \\
					1  & -1 & 0 & 0 \\
					1  & 1  & 0 & 1 \\
					1  & 1  & 0 & 0
				\end{pmatrix} =
				\begin{pmatrix}
					1 & 0 & 0 & 0 \\
					0 & 1 & 0 & 0 \\
					0 & 0 & 1 & 0 \\
					0 & 0 & 0 & 1
				\end{pmatrix}
			\end{equation*}
			De donde se obtiene que:
			\begin{equation*}
				\varphi^3(x,y,z,t) = x+y \quad \varphi^4(x,y,z,t) = z-t
			\end{equation*}
			Ahora obtenemos que
			\begin{align*}
				U &= \{ (x,y,z,t) \in R^4 / \varphi(x,y,z,t) = 0, \quad  \forall \varphi \in \text{an}(U)\} \\&= \left\{ (x,y,z,t) \in R^4 \arrowvert \begin{array}{cc}
					x+y = 0 \\
					z-t = 0
				\end{array}\right\}
			\end{align*}
		\item[\textit{c})] $U=\mathcal{L}\{(1,-1,0),(0,1,-1)\} \cap \mathcal{L}\{(0,1,0)\}$ \\ \\
			En primer lugar, hacemos uso de la propiedad an($U + W$) = an($U$) $\cap$ an($W$):
			\begin{align*}
				\text{an}(U) & = \text{an}(\mathcal{L}\{(1,-1,0),(0,1,-1)\}) \cap \text{an}(\mathcal{L}\{(0,1,0)\}) \\
				             & = \text{an}(\mathcal{L}\{(1,-1,0),(0,1,-1)\}) + \text{an}(\mathcal{L}\{(0,1,0)\})    \\
				             & = \text{an}(\{(1,-1,0),(0,1,-1)\}) + \text{an}(\{(0,1,0)\})
			\end{align*}
			Por un lado, tenemos que:
			\begin{align*}
				\text{an}(\mathcal{L}\{(1,-1,0),(0,1,-1)\}) & = \left\{ \varphi \in V^{*} : \begin{array}{ll}
					                                                                            \varphi(1,-1,0) = 0 \\
					                                                                            \varphi(0,1,-1) = 0
				                                                                            \end{array}\right\}                                      \\
				                                            & = \left\{ \varphi \in V^{*} : \begin{array}{ll}
					                                                                            \varphi^1 - \varphi^2  = 0 \\
					                                                                            \varphi^2 + \varphi^3 = 0
				                                                                            \end{array}\right\} = \mathcal{L}\left\{ (1,1,1)\right\}
			\end{align*}
			Por otro lado, tenemos que:
			\begin{align*}
				\text{an}(\mathcal{L}\{(0,1,0)\}) & = \left\{ \varphi \in V^{*} : \begin{array}{ll}
					                                                                  \varphi(0,1,0) = 0
				                                                                  \end{array}\right\}                                                \\
				                                  & = \left\{ \varphi \in V^{*} : \varphi^2 = 0\right\} = \mathcal{L}\left\{ (1,0,0),(0,0,1)\right\}
			\end{align*}
			De donde se obtiene que:
			\begin{equation*}
				\text{an}(U) = \mathcal{L}\left\{ (1,1,1),(1,0,0),(0,0,1)\right\} = V^{*} \Rightarrow U = \{0\}
			\end{equation*}
	\end{itemize}
\end{ejercicio}



\begin{ejercicio}
	Se considera la aplicación lineal $f : \R^3 \mapsto \R^2$ dada por:
	\begin{equation*}
		f(1,1,0) = (1,-1), \quad f(1,0,1) = (3,0), \quad f(0,1,1) = (2,-1)
	\end{equation*}
	Calcular la matriz de $f^t$ con respecto a las bases duales de las bases usuales de $R^2$ y de $R^3$, respectivamente.
	Calcular bases de Nuc$(f^t)$ y de Im$(f^t)$.
	En primer lugar, partimos de que tenemos la siguiente propiedad:
	\begin{equation*}
		M(f^t;B^* \leftarrow B^{'*}) = M(f;B'\leftarrow B)^t
	\end{equation*}
	En este caso, nos piden que calculemos $M(f^t;B^{3*}_u \leftarrow B^{2*}_u)$, por tanto, calcularemos en primer lugar la matriz
	$M(f;B^{2}_u \leftarrow B^{3}_u)$, para ello, calculamos $f(1,0,0),f(0,1,0),f(0,0,1)$:
	\begin{align*}
		f(1,1,0) = f(e_1) + f(e_2) & = (1,-1) \\
		f(1,0,1) = f(e_1) + f(e_3) & = (3,0)  \\
		f(0,1,1) = f(e_2) + f(e_3) & = (-2,1)
	\end{align*}
	De donde se obtiene que:
	\begin{align*}
		f(e_1) & = (3,-1) \\
		f(e_2) & = (-2,0) \\
		f(e_3) & = (0,1)
	\end{align*}
	Por tanto, la matriz $M(f;B^{2}_u \leftarrow B^{3}_u)$ es:
	\begin{equation*}
		M\left(f;B^{2}_u \leftarrow B^{3}_u\right) = \begin{pmatrix}
			3  & -2 & 0 \\
			-1 & 0  & 1
		\end{pmatrix} \Rightarrow M\left(f^t;B^{3*}_u \leftarrow B^{2*}_u\right) = \begin{pmatrix}
			3  & -1 \\
			-2 & 0  \\
			0  & 1
		\end{pmatrix}
	\end{equation*}
	Para calcular una base de Ker$(f^t)$ hacemos uso de que an(Im($f$)) = Ker($f^t$), en primer lugar, Im$(f)$
	tiene dimensión $2$, por tanto, Im$(f)$ = $\R^2$, por tanto, Ker($f^t$) = $\{0\}$, y una base de Ker($f^t$) es $\{0\}$. \\ \\
	Para calcular una base de Im$(f^t)$, hacemos uso de que an(Ker($f$)) = Im($f^t$), en primer lugar:
	\begin{align*}
		\text{Ker}(f) &= \left\{ (x,y,z) \in \R^3 : \begin{pmatrix}
			3  & -2 & 0 \\
			-1 & 0  & 1
		\end{pmatrix} \cdot \begin{pmatrix}
			x \\
			y \\
			z
		\end{pmatrix} = \begin{pmatrix}
			0 \\
			0
		\end{pmatrix}\right\} \\&= \left\{ (x,y,z) \in \R^3 : \begin{array}{cc}
			3x -2y = 0 \\
			-x + z = 0
		\end{array}\right\}
	\end{align*}
	De donde se obtiene que $\text{Ker}(f) = \mathcal{L}\left\{ (2,3,2)\right\}$, por tanto, an(Ker($f$)) es:
	\begin{equation*}
		\text{an}(\text{Ker}(f)) = \left\{ \varphi \in \R^3 : \varphi(2,3,2) = 0\right\} = \left\{ \varphi \in \R^3 : 2\varphi^1 + 3\varphi^2 + 2\varphi^3 = 0\right\}
	\end{equation*}
	De donde se obtiene que an(Ker($f$)) = $\mathcal{L}\left\{ (3,-2,0), (0,2,-3)\right\}$, por tanto, una base de Im($f^t$) es precisamente
	$B = \left\{ (3,-2,0), (0,2,-3)\right\}$.

\end{ejercicio}



\begin{ejercicio}
	Decidir razonadamente si las siguientes afirmaciones son verdaderas o falsas:
	\begin{itemize}
		\item[ \textit{a)}] \textit{Toda forma lineal $\varphi \neq \varphi_0$ sobre un espacio vectorial $V$ es un epimorfismo.} \\ \\
			Sea $\varphi$ una forma lineal, para que sea epimorfismo, $Im(\varphi) = \mathbb{K}$, como $\dim(Im(\varphi)) \leq 1$, e $Im(\varphi) \neq \{0\}$, entonces
			$\dim(Im(\varphi)) = 1$, como además Im$(\varphi) \subseteq \mathbb{K} \Rightarrow Im(\varphi) = \mathbb{K} .$ Por tanto, $\varphi$ es un epimorfismo. y la afirmación es verdadera.
		\item[ \textit{b)}] \textit{Existe un subespacio de $\R^{12}$ que está definido por 7 ecuaciones implícitas independientes y
				esta generado por 4 vectores.}
			\\ \\
			Sea $U$ el subespacio de $\R^{12}$ definido por 7 ecuaciones implícitas independientes, entonces, $\dim(\text{an}(U)) = 7$, pero como
			tenemos que $\dim(\text{an}(U))= n - m$ donde $n = \dim(\R^{12}) = 12$ y $m = 7$, entonces, obtenemos $\dim(\text{an}(U)) = 5$, lo cual es una contradicción
			por ser $5 \neq 7$, por tanto, la afirmación es falsa.
		\item[ \textit{c)}] \textit{Para cada $v \in \R^3$ con $v \neq 0$, existe un epimorfismo $f : \text{an}(\{v\}) \mapsto \R^3$.} \\ \\
			Observamos, en primer lugar que:
			\begin{equation*}
				\dim(\text{an}(\{v\})) = \dim(\text{an}(\mathcal{L}\{x\})), \quad \forall x \in \R^3, x \neq 0
			\end{equation*}
			A partir de aquí, vemos que $\dim(\text{an}(\mathcal{L}\{x\})) = \dim(\R^3)-\dim(\mathcal{L}\{x\}) = 3-1=2$
			Como $\forall f : V \mapsto V'$ se verifica $\dim{\text{Im}(f)} \leq \dim{V}$, tenemos que $\dim(\text{Im}(f)) \leq 2$, por tanto,
			no existe ningún epimorfismo $f : \text{an}(\{v\}) \mapsto \R^3$, es decir, la afirmación es falsa.
		\item[ \textit{d)}] \textit{Una aplicación lineal entre dos espacios vectoriales sobre $K$ finitamente generados es un isomorfismo
				si y solo si también lo es su aplicación traspuesta.}
			\\ \\
			En primer lugar, comprobamos que $A$ regular $\Leftrightarrow A^t$ regular, para ello, sea $A$ una matriz regular, entonces:
			\begin{equation*}
				\text{det}(A) \neq 0 \Rightarrow |A^t| = |A| \neq 0 \Rightarrow A^t \text{ regular}
			\end{equation*}
			A partir de esto:
			\begin{equation*}
				f \text{ isomorfismo} \Leftrightarrow M(f;B' \leftarrow B) \text{ regular} \Leftrightarrow M(f^t;B^* \leftarrow B'^{*}) \text{ regular} \Leftrightarrow f^t \text{ isomorfismo}
			\end{equation*}
	\end{itemize}
\end{ejercicio}
\section{Aplicaciones Lineales}
\begin{ejercicio}Estudiar si las siguientes aplicaciones son lineales o no:
	\begin{enumerate}
		\item \( f: \mathbb{R}^2 \rightarrow \mathbb{R}^3, f(x,y) = (x - y, x + 3y, 2y) \).
		      \\ \\
		      Sean $a,b \in \R$ y sean $(x,y),(x',y') \in \R^2$ entonces:
		      \begin{align*}
			      f(ax+bx',ay+by') & = (ax+bx'-ay-by',ax+3ay+bx'+3by',2ay+2by')               \\
			                       & = (a(x-y)+b(x'-y'),a(x+3y)+b(x'+3y'),a(2y)+b(2y'))       \\
			                       & = a(x-y,x+3y,2y)+b(x'-y',x'+3y',2y') = af(x,y)+bf(x',y') \\
		      \end{align*}
		      Por tanto, la aplicación es lineal.
		      \begin{align*}
			      Ker(f) & = \{(x,y) \in \R^2 | f(x,y) = (0,0,0)\} = \left\{
			      \begin {array}{rcl}
			      x-y    & =                                                                                     & 0 \\
			      x+3y   & =                                                                                     & 0 \\
			      2y     & =                                                                                     & 0
			      \end {array}
			      \right\}                                                                                           \\ &\Rightarrow x=y=0 \Rightarrow Ker(f) = \{(0,0)\}                                          \\
			      Im(f)  & = \{(x,y,z) \in \R^3 | \exists (x',y') \in \R^2 \text{ tal que } f(x',y') = (x,y,z)\}
		      \end{align*}
		      Para calcular la imagen, como $(e_1,e_2)$ es una base de $\R^2$, basta con calcular $f(e_1)$ y $f(e_2)$:
		      \begin{align*}
			      f(e_1) & = (1,1,0)  \\
			      f(e_2) & = (-1,3,2)
		      \end{align*}
		      Como $\begin{vmatrix}
				      1 & -1 \\
				      1 & 3
			      \end{vmatrix} = 4 \neq 0$, los vectores $f(e_1)$ y $f(e_2)$ son L.I.
		      Por tanto, $Im(f) = \mathcal{L}\{(1,1,0),(1,3,2)\}$. Veamos ahora si se cumple la fórmula de las dimensiones:
		      \begin{equation*}
			      \dim(Ker(f))+\dim(Im(f)) = 0+2 = 2 = \dim(\R^2) = \Rightarrow
		      \end{equation*}
		      \text{ Se cumple la fórmula de las dimensiones}
		\item \( f: \mathbb{R}^3 \rightarrow \mathbb{R}^3, f(x,y,z) = (2x - 3y, z^2 - x + 2, 3y - z - x) \).
		      \\ \\
		      Como $f(0,0,0) = (0,2,0) \neq (0,0,0)$, la aplicación no es lineal.
		\item \( f: \mathbb{R}^3 \rightarrow \mathbb{R}^2[u], f(x,y,z) = (2x + y)u^2 + (y - z)u + 2y \).
		\item \( f: \mathbb{R}^3 \rightarrow M_2[\mathbb{R}], f(x,y,z) = \begin{pmatrix} x + y - z & x + 2z \\ y - 3z & 2x + y + z \end{pmatrix} \).
	\end{enumerate}
	Para las aplicaciones que sean lineales, calcular su núcleo e imagen, y comprobar la fórmula de las dimensiones.
\end{ejercicio}

\begin{ejercicio} Calcular una aplicación lineal \( f: \mathbb{R}^3 \rightarrow \mathbb{R}^2 \) cuyo núcleo esté generado por \( \{(1,0,-1), (2,0,1)\} \) y cuya imagen esté generada por \( (1, -2) \).
	Sabemos que $Ker(f) = \{(x,y,z) \in \R^3 | f(x,y,z) = (0,0)\}$. Por tanto: \\ \\
	Para que $Ker(f) = \mathcal{L}\{(1,0,-1),(2,0,1)\}$, debe cumplirse que:
	\begin{equation*}
		\left.
		\begin {array}{rcl}
		x-z & = & 0 \\
		2x+z & = & 0
		\end {array}
		\right\} \Rightarrow x=z=0
	\end{equation*}
	Además $Im(f) = \mathcal{L}\{(1,-2)\}$. Por lo que $f(x,y,z) = a(1,-2)$ es decir, $f(x,y,z) = (a,-2a)$. Como hemos visto antes, $x=z=0$, por lo que:
	\begin{equation*}
		y = a \Rightarrow f(x,y,z) = (y,-2y)
	\end{equation*}
\end{ejercicio}

\begin{ejercicio} Encontrar un automorfismo \( f \) de \( \mathbb{R}^3(\mathbb{R}) \) (esto es, \( f \) es un isomorfismo de \( \mathbb{R}^3(\mathbb{R}) \) en sí mismo) de manera que \( f(U) = U' \) donde
	\[ U = \{(a,b,0) \in \mathbb{R}^3 : a,b \in \mathbb{R}\}, \quad U' = \{(c,c + d,d) \in \mathbb{R}^3 : c,d \in \mathbb{R}\}. \]
	Por ser $f$ un automorfismo, $f$ es biyectiva. Por tanto, $f$ es un isomorfismo y una base de $\R^3$ se aplica sobre otra base de $\R^3$.
	Sea pues $B_u = \{e_1,e_2,e_3\}$ una base de $U$, entonces $f(e_1),f(e_2),f(e_3)$ es una base de $U'$.
	\begin{align*}
		e_1 \in U & \Rightarrow f(e_1) = (1,1,0)                                                                                \\
		e_2 \in U & \Rightarrow f(e_2) = (0,1,1)                                                                                \\
		e_3 \in U & \Rightarrow \text{ No hay restricción respecto de } f(e_3) \Rightarrow \text{ Buscamos } f(e_3) \text{ L.I}
	\end{align*}
	Sea $f(e_3) = (0,1,0)$, entonces:
	\begin{equation*}
		\begin{vmatrix}
			1 & 0 & 0 \\
			1 & 1 & 1 \\
			0 & 1 & 0
		\end{vmatrix} = -1 \neq 0 \Rightarrow \text{ Los vectores son L.I } \Rightarrow \text{ Forman una base de } U'
	\end{equation*}
	Por tanto, $f$ es un automorfismo de las carácterísticas pedidas definido de la siguiente manera:
	\begin{equation*}
		f(x,y,z) = (x,x+y+z,y)
	\end{equation*}
\end{ejercicio}
\begin{ejercicio}
	Sea \( f : V \rightarrow V' \) una aplicación entre dos espacios vectoriales sobre el mismo cuerpo \( K \). Demostrar que \( f \) es lineal si y sólo si el grafo de \( f \), es decir, el conjunto:
	\[ G(f) = \{(v,v') \in V \times V' \mid v' = f(v)\} \]
	es un subespacio vectorial de \( V \times V' \). Calcular también la dimensión de este subespacio cuando \( V \) y \( V' \) son espacios finitamente generados.
\end{ejercicio}

\begin{ejercicio}
	Sean \( V_1 \) y \( V_2 \) espacios vectoriales sobre el mismo cuerpo \( K \). Consideremos el espacio vectorial producto \( V_1 \times V_2 \) definido en el ejercicio 3 de la relación de problemas anterior.
	\begin{enumerate}
		\item Demostrar que la \textit{proyección i-ésima} \( \pi_i: V_1 \times V_2 \rightarrow V_i \) dada por \( \pi_i(v_1,v_2) = v_i \) es un epimorfismo para cada \( i = 1, 2 \).
	\end{enumerate}
\end{ejercicio}


\begin{ejercicio}
	Sea \( V \) un espacio vectorial sobre \( K \) y \( f : V \rightarrow V \) un endomorfismo de forma que \( f \circ f = f \). Demostrar que \( V = \text{Nuc}(f) \oplus \text{Im}(f) \).

	Para ello, deberemos demostrar que $V = \text{Nuc}(f) + \text{Im}(f)$ y que $\text{Nuc}(f) \cap \text{Im}(f) = \{0\}$.
	\begin{itemize}
		\item $\text{Nuc}(f) \cap \text{Im}(f) = \{0\}$:
		      \\ \\
		      Sea $v \in \text{Nuc}(f) \cap \text{Im}(f)$, entonces:
		      \begin{equation*}
			      \left.
			      \begin{aligned}
				      v & \in \text{Nuc}(f) \Rightarrow f(v) = 0                                   \\
				      v & \in \text{Im}(f) \Rightarrow \exists v' \in V \text{ tal que } f(v') = v
			      \end{aligned}
			      \right| \Rightarrow 0 = f(v) = f(f(v')) = f(v') = v  = 0
		      \end{equation*}
		\item $V = \text{Nuc}(f) + \text{Im}(f)$:
		      \\ \\
		      Se demuestra mediante doble inclusión:
		      \begin{itemize}
			      \item $\text{Nuc}(f) + \text{Im}(f) \subseteq V$:
			            \\ \\
			            $\forall x \in \text{Nuc}(f) + \text{Im}(f), x = au + bv, u \in \text{Nuc}(f) \subseteq V , v \in \text{Im}(f) \subseteq V , a,b \in K$. Por ser
			            $V$ espacio vectorial, $au+bv \in V$. Por tanto, $\text{Nuc}(f) + \text{Im}(f) \subseteq V$.
			      \item $V \subseteq \text{Nuc}(f) + \text{Im}(f)$:
			            \\ \\
			            Veamos primero que $x-f(x) \in \text{Nuc}(f)$:
			            \begin{equation*}
				            f(x-f(x)) = f(x)-f(f(x)) = f(x)-f(x) = 0 \Rightarrow x-f(x) \in \text{Nuc}(f)
			            \end{equation*}
			            Ahora, $\forall x \in V, x = (x-f(x))+f(x) \in \text{Nuc}(f) + \text{Im}(f)$. Por tanto, $V \subseteq \text{Nuc}(f) + \text{Im}(f)$.
		      \end{itemize}
	\end{itemize}
\end{ejercicio}

\begin{ejercicio}
	Sea \( V \) un espacio vectorial sobre \( K = \mathbb{Q}, \mathbb{R}, \mathbb{C} \) y \( f : V \rightarrow V \) un endomorfismo de forma que \( f \circ f = \text{Id}_V \). Demostrar que \( f \) es un automorfismo y que \( V = U \oplus W \) donde:
	\[ U = \{ v \in V : f(v) = v \}, \quad W = \{ v \in V : f(v) = -v \} \]
	Para demostrar que $f$ es un automorfismo, debemos demostrar que $f$ es biyectiva. Como $f$ es un endomorfismo, $f$ es lineal y además basta con demostrar la
	inyectividad de $f$ para demostrar que $f$ es biyectiva, o lo que es lo mismo, que $Ker(f) = \{0\}$. \\ \\
	Sea por tanto $v \in Ker(f)$, entonces:
	\begin{equation*}
		f(v) = 0 \Rightarrow f(f(v)) = f(0) = 0, \text{ pero } f(f(v)) = v \Rightarrow v = 0 \Rightarrow Ker(f) = \{0\}
	\end{equation*}
	Por tanto, $f$ es biyectiva y por tanto es un automorfismo. Veamos ahora que $V = U \oplus W$:
	\begin{itemize}
		\item \text{En primer lugar, vemos que } $U \cap W = \{0\}$ ya que si $v \in U \cap W$ entonces $v=-v \Rightarrow v=0$. Veamos ahora que $V = U + W$:
		      \begin{equation*}
			      \forall v \in V, v=ux+wy, u \in U, w \in W, x,y \in K \Rightarrow U + W \subseteq V
		      \end{equation*}
		      \begin{equation*}
			      \forall v \in V, v = \frac{v+f(x)}{2}+\frac{v-f(x)}{2}
		      \end{equation*}
		      Veamos que $\frac{v+f(v)}{2} \in U$ y que $\frac{v-f(v)}{2} \in W$:
		      \begin{align*}
			      f\left(\frac{v+f(v)}{2}\right) & = \frac{f(v)+f(f(v))}{2} = \frac{v+f(v)}{2} \Rightarrow \frac{v+f(v)}{2} \in U  \\
			      f\left(\frac{v-f(v)}{2}\right) & = \frac{f(v)-f(f(v))}{2} = \frac{-v+f(v)}{2} \Rightarrow \frac{v-f(v)}{2} \in W
		      \end{align*}
		      Por tanto $V \subseteq U + W \Rightarrow V = U + W$. Por tanto, $V = U \oplus W$.
	\end{itemize}
\end{ejercicio}

\begin{ejercicio}
	En el espacio \( M_2(\mathbb{C}) \) de las matrices cuadradas de orden dos con coeficientes complejos se considera la matriz
	\[ A = \begin{pmatrix} 1 & -1 \\ 2 & i \end{pmatrix} \]
	Definimos la aplicación \( R : M_2(\mathbb{C}) \rightarrow M_2(\mathbb{C}) \) dada por \( R(X) = X \cdot A \). Demostrar que \( R \) es un automorfismo y calcular su expresión matricial con respecto a una base \( B \) de \( M_2(\mathbb{C}) \). ¿Cuál es la matriz \( M(R^{-1})_B \)?
\end{ejercicio}



\begin{ejercicio}
	Sea \( f : \mathbb{R}^4 \rightarrow \mathbb{R}^3 \) la aplicación lineal dada por:
	\[ f(x,y,z,t) = (x+z-t, y+t,x+y+z) \]
	Se pide lo siguiente:
	\begin{enumerate}
		\item \textit{Calcular bases del núcleo y de la imagen de \( f \). ¿Es \( f \) un monomorfismo o un epimorfismo?}
		      \begin{itemize}
			      \item Ker$(f)$: $\{(x,y,z,t) \in \R^4 | f(x,y,z,t) = (0,0,0)\}$. Por tanto:
			            \\ \\
			            Ker$(f)=\left\{\begin{array}{rcl}
					            x+z-t & = & 0 \\
					            y+t   & = & 0 \\
					            x+y+z & = & 0
				            \end{array} \right\}  =
				            \left\{\begin{array}{rcl}
					            x+y+z & = & 0 \\
					            y+t   & = & 0
				            \end{array} \right\} \Rightarrow \dim{\text{Ker}(f)}=2$
			            \begin{equation*}
				            \text{Ker}(f) = \mathcal{L}\{(-1,0,1,0),(1,-1,0,1)\} \quad \land \quad
				            \begin{vmatrix}
					            -1 & 0  \\
					            1  & -1 \\
				            \end{vmatrix} = 1 \neq 0 \Rightarrow \text{  L.I.}
			            \end{equation*}
			            Por tanto, una base de Ker$(f)$ es $B_{\text{Ker}(f)}=\left\{(-1,0,1,0),(1,-1,0,1)\right\}$. \\
			      \item Im$(f)$: $\{(x,y,z) \in \R^3  \ / \ \exists (x',y',z',t') \in \R^4 \text{ tal que } f(x',y',z',t') = (x,y,z)\}$. Por tanto, tomamos
			            $B_{\R^4} = \{e_1 , e_2 , e_3 , e_4 \}$ y calculamos $f(e_1),f(e_2),f(e_3),f(e_4)$:
			            \begin{align*}
				            f(e_1) & = (1,0,1)  \\
				            f(e_2) & = (0,1,1)  \\
				            f(e_3) & = (1,0,1)  \\
				            f(e_4) & = (-1,1,0)
			            \end{align*}
			            Como $\begin{vmatrix}
					            1 & 0 \\
					            0 & 1 \\
				            \end{vmatrix} = 1 \neq 0$, los vectores $f(e_1)$ y $f(e_2)$ son L.I. Como:
			            \begin{equation*}
				            \begin{vmatrix}
					            1 & 0 & 1 \\
					            0 & 1 & 0 \\
					            1 & 1 & 1 \\
				            \end{vmatrix} = 0 \quad \text{ y } \quad
				            \begin{vmatrix}
					            1 & 0 & -1 \\
					            0 & 1 & 1  \\
					            1 & 1 & 0  \\
				            \end{vmatrix} = 0
			            \end{equation*} $\dim{ \text{Im}(f)} = 2$. Por tanto, una base de Im$(f)$ es $B_{\text{Im}(f)} = \{(1,0,1),(0,1,1)\}$. Sabemos entonces que
			            $\dim{ \text{Im}(f)} = 2 \neq 4 = \dim{\R^4} \land \dim{ \text{Ker}(f)} = 2 \neq 0 \Rightarrow \text{ No es monomorfismo ni epimorfismo}$.
		      \end{itemize}
		\item \textit{Sean \( U = L((1,2,1,2), (0,-1,2,3)) \) y \( U' = \{(x,y,z) \in \mathbb{R}^3 | x - y = 0\} \). Calcular \( f(U) \) y \( f^{-1}(U') \).}
		      \begin{equation*}
			      \left.\begin{array}{rcl}
				      f(1,2,1,2) = (0,4,4) \\
				      f(0,-1,2,3) = (-1,2,1)
			      \end{array} \right\} \Rightarrow f(U) = \mathcal{L}\{(0,4,4),(-1,2,1)\}
		      \end{equation*}
		      \begin{align*}
			      f^{-1}(U') & = \{(x,y,z,t) \in \R^4 | f(x,y,z,t) \in U'\}  \\&= \{(x,y,z,t) \in \R^4 |  (x+z-t, y+t,x+y+z) \in U'\} \\
			                 & = \{(x,y,z,t) \in \R^4 |  (x+z-t)-(y+t) = 0\} \\&= \{(x,y,z,t) \in \R^4 |  x-y+z-2t = 0\}             \\
		      \end{align*}
		      Por tanto $f^{-1}(U') = \mathcal{L}\{(1,1,0,0),(1,0,1,1),(0,1,1,0)\}$
		\item \textit{Encontrar bases \( B \) de \( \mathbb{R}^4 \) y \( B' \) de \( \mathbb{R}^3 \) tales que $M(f:B' \leftarrow B)$ solo tenga unos y ceros.}
		      \\ \\
		      Como $Rg(M(f_{B' \leftarrow B})) = 2$, tomamos la matriz más sencilla con $Rg = 2$:
		      \begin{equation*}
			      M(f_{B \leftarrow B'}) = \begin{pmatrix}
				      1 & 0 & 0 & 0 \\
				      0 & 1 & 0 & 0 \\
				      0 & 0 & 0 & 0
			      \end{pmatrix}
		      \end{equation*}
		      Como $B=\{e_1,e_2,e_3,e_4\}$ y $B'=\{e_1,e_2,e_3\}$, entonces:
		      \begin{align*}
			      f(e_1) & = (1,0,0) = e_1 \\
			      f(e_2) & = (0,1,0) = e_2 \\
			      f(e_3) & = (0,0,0) = 0   \\
			      f(e_4) & = (0,0,0) = 0
		      \end{align*}
		      Por tanto, si $B'=\{(1,0,1),(0,1,1),(0,0,1)\}$:
		      \begin{align*}
			      v_1 & = (1,0,0,0) = e_1                                                     \\
			      v_2 & = (0,1,0,0) = e_2                                                     \\
			      v_3 & \in \text{Nuc}(f) \Rightarrow f(v_3) = 0 \Rightarrow v_3 = (-1,0,1,0) \\
			      v_4 & \in \text{Nuc}(f) \Rightarrow f(v_4) = 0 \Rightarrow v_4 = (1,-1,0,1)
		      \end{align*}
		      Por tanto, $B=\{(1,0,0,0),(0,1,0,0),(-1,0,1,0),(1,-1,0,1)\}$ y\\ $B'=\{(1,0,1),(0,1,1),(0,0,1)\}$.
	\end{enumerate}
\end{ejercicio}


\begin{ejercicio}
	Sean \( V \) y \( V' \) dos espacios vectoriales reales con bases \( B = (v_1, v_2, v_3, v_4) \) y \( B' = (v'_1, v'_2, v'_3) \) respectivamente. Si \( f : V \rightarrow V' \) es la aplicación lineal definida por:
	\[ f(v_1) = v'_1 + v'_2 - 4v'_3, \quad f(v_2) = 2v'_1 + v'_2 - 2v'_3, \quad f(v_3) = 3v'_1 + v'_2 , \quad f(v_4) = v'_1 + 2v'_3 \]
	calcular la matriz \( M(f_{B \leftarrow B'}) \). Calcular bases de \( \text{Nuc}(f) \) y de \( \text{Im}(f) \).
	\begin{equation*}
		M = (f:B' \leftarrow B) = \begin{pmatrix}
			1  & 2  & 3 & 1 \\
			1  & 1  & 1 & 0 \\
			-4 & -2 & 0 & 2
		\end{pmatrix}
	\end{equation*}
	Para calcular bases de $Nuc(f)$ e $Im(f)$, comenzamos viendo que
	\begin{equation*}
		\begin{vmatrix}
			1  & 2  & 3 \\
			1  & 1  & 1 \\
			-4 & -2 & 0
		\end{vmatrix} = 0 \quad \text{ y } \quad
		\begin{vmatrix}
			1  & 2  & 1 \\
			1  & 1  & 0 \\
			-4 & -2 & 2
		\end{vmatrix} = 0
	\end{equation*}
	Por tanto, $\dim{Im(f)} = 2$ y por la fórmula de las dimensiones, $\dim{Nuc(f)} = 2$. Para calcular una base de $Im(f)$ vemos que
	\begin{equation*}
		\text{Im}(f) = \mathcal{L}\{(1,1,-4),(2,1,-2)\} \Rightarrow B_{Im(f)} = \{(1,1,-4),(2,1,-2)\}
	\end{equation*}
	Para calcular una base de $Nuc(f)$, resolvemos el sistema homogéneo asociado a la matriz $M$:
	\begin{equation*}
		\left.\begin{array}{rcl}
			x+2y+3z+w  & = & 0 \\
			x+y+z      & = & 0 \\
			-4x-2y +2w & = & 0 \\
		\end{array} \right\} \Rightarrow \text{ Nuc}(f) = \mathcal{L}\{(-1,1,0,-1),(-1,0,1,-2)\}
	\end{equation*}
	Por tanto, $B_{Nuc(f)} = \{(-1,1,0,-1),(-1,0,1,-2)\}$.
\end{ejercicio}


\begin{ejercicio}
	Sea \( f \) un endomorfismo de \( \mathbb{R}^3 \) que verifica las propiedades:
	\[ f(1,0,1) = (-1,2,0), \quad f(1,-1,0) = (1,2,1), \quad \text{Nuc}(f) = L((0,3,7)) \]
	Obtener la expresión matricial de \( f \) con respecto a la base usual de \( \mathbb{R}^3 \). Calcular la matriz de \( f \) con respecto a la base de \( \mathbb{R}^3 \) dada por \( B = ((-1,1,1), (1,-1,1), (1,1,-1)) \).
	Sea $B_u = \{e_1,e_2,e_3\}$ la base usual de $\R^3$. Entonces:
	\begin{align*}
		f(1,0,1)  & = f(e_1+e_3) = f(e_1) = (-1,2,0)+f(e_3)                          \\
		f(1,-1,0) & = f(e_1-e_2) = f(e_1)-f(e_2) = (1,2,1)                           \\
		f(0,3,7)  & = 3f(e_2)+7f(e_3) = 3(2,1,1)+7f(e_3) = (6,3,3)+7f(e_3) = (0,0,0) \\
	\end{align*}
	Podemos ir despejando y obtenemos:
	\begin{align*}
		f(e_2)+f(e_3) & = (-2,0,-1) \Rightarrow f(e_2) = (-2,0,-1)-f(e_3) \\
		f(e_3)        & = \left(\frac{3}{2},0,\frac{3}{4}\right)
	\end{align*}
	De donde obtenemos:
	\begin{align*}
		f(e_1) & = \left(\frac{-5}{2},2,\frac{3}{4}\right)  \\
		f(e_2) & = \left(\frac{-7}{2},0,\frac{-7}{4}\right)
	\end{align*}
	Por tanto:
	\begin{equation*}
		M(f_{B_u \leftarrow B_u}) = \begin{pmatrix}
			\frac{-5}{2} & \frac{-7}{2} & \frac{3}{2} \\
			2            & 0            & 0           \\
			\frac{3}{4}  & \frac{-7}{4} & \frac{3}{4}
		\end{pmatrix}
	\end{equation*}
	Ahora, sea $B = ((-1,1,1), (1,-1,1), (1,1,-1))$ una base de $\R^3$. Entonces:
	\begin{equation*}
		M(f;B \leftarrow B) = M(I;B \leftarrow B_u) \cdot M(f;B_u \leftarrow B_u) \cdot M(I;B_u \leftarrow B)
	\end{equation*}
	Deberemos calcular $M(I;B \leftarrow B_u)$ y $M(I;B_u \leftarrow B)$. Para ello:
	\begin{equation*}
		M(I;B_u \leftarrow B) = \begin{pmatrix}
			-1 & 1  & 1  \\
			1  & -1 & 1  \\
			1  & 1  & -1
		\end{pmatrix} \quad \land \quad M(I;B \leftarrow B_u) = \begin{pmatrix}
			-1 & 1  & 1  \\
			1  & -1 & 1  \\
			1  & 1  & -1
		\end{pmatrix}^{-1} = \begin{pmatrix}
			0   & 0,5 & 0,5 \\
			0,5 & 0   & 0,5 \\
			0,5 & 0,5 & 0
		\end{pmatrix}
	\end{equation*}
	Por tanto:
	\begin{equation*}
		M(f;B \leftarrow B_u) = \begin{pmatrix}
			\frac{-15}{8} & \frac{21}{8} & \frac{1}{8}   \\
			\frac{-5}{8}  & \frac{23}{8} & \frac{-37}{8} \\
			\frac{-3}{4}  & \frac{9}{4}  & \frac{-11}{4}
		\end{pmatrix}
	\end{equation*}
\end{ejercicio}


\begin{ejercicio}
	Determinar un endomorfismo \( f \) de \( \mathbb{R}^3 \) con núcleo \( \text{Nuc}(f) = L((1,1,0)) \) e imagen dada por \( \text{Im}(f) = \{(x,y,z) \in \mathbb{R}^3 | 2x - 3y = 0\} \). ¿Es \( f \) único en estas condiciones? Analizar si es posible encontrar bases \( B \) y \( B' \) de \( \mathbb{R}^3 \) de forma que:
	\[ M(f_{B \leftarrow B'}) = \begin{pmatrix} 1 & 0 & 0 \\ 0 & 1 & 0 \\ 0 & 0 & 0 \end{pmatrix} \]
	¿Es posible de modo que esta última matriz sea \( M(f_B) \)?
	Para resolverlo, en primer lugar, deberemos encontrar una base de $Im(f)$ y una base de $Nuc(f)$:
	\begin{align*}
		\text{Im}(f)  & = \{(x,y,z) \in \mathbb{R}^3 | 2x - 3y = 0\} \Rightarrow B_{Im(f)} = \{(3,2,0),(0,0,1)\} \\
		\text{Nuc}(f) & = L((1,1,0)) \Rightarrow B_{Nuc(f)} = \{(1,1,0)\}
	\end{align*}
	De donde claramente se obtiene:
	\begin{align*}
		f(1,1,0) & = (0,0,0) \Rightarrow f(e_1+e_2) = 0 \Rightarrow f(e_1) = -f(e_2) \\
		f(1,0,0) & = (3,2,0)                                                         \\
		f(0,1,0) & = (-3,-2,0)                                                       \\
		f(0,0,1) & = (0,0,1)
	\end{align*}
	De donde obtenemos:
	\begin{equation*}
		M(f;B_u \leftarrow B_u) = \begin{pmatrix}
			3 & -3 & 0 \\
			2 & -2 & 0 \\
			0 & 0  & 1
		\end{pmatrix} \Rightarrow f(x,y,z) = (3x-3y,2x-2y,z)
	\end{equation*}
	$f$ no es único en estas condiciones ya que podríamos haber tomado $(0,0,2)$ como vector de Im($f$) y habríamos
	obtenido un endomorfismo distinto. Veamos ahora las bases $B$ y $B'$, sea $B = \{v_1,v_2,v_3\}$ y $B' = \{v_1',v_2',v_3'\}$:
	\begin{align*}
		f(v_1) & = (1,0,0) \\
		f(v_2) & = (0,1,0) \\
		f(v_3) & = (0,0,0)
	\end{align*}
	Por tanto, si $B = \{(1,0,0),(0,1,0),(1,1,0)\}$:
	\begin{align*}
		v_1 & = (1,0,0)                                                          \\
		v_2 & = (0,1,0)                                                          \\
		v_3 & \in \text{Ker}(f) \Rightarrow f(v_3) = 0 \Rightarrow v_3 = (1,1,0)
	\end{align*}
	Se tiene que $B' = \{(3,2,0),(0,0,1)(0,1,0)\}$.
	\ejercicio[13]{}{
	Una aplicación lineal \( f : \mathbb{R}^3 \rightarrow \mathbb{R}^2[x] \) tiene por matriz asociada
	\[ A = \begin{pmatrix} -2 & 1 & 1 \\ 1 & -2 & 1 \\ 1 & 1 & -2 \end{pmatrix} \]
	respecto de las bases \( B = ((1,0,1), (1,1,1), (2,1,0)) \) y \( B' = (1, 1 + 2x, -x^2) \) respectivamente. Calcular la matriz que representa a \( f \) respecto de la base usual de \( \mathbb{R}^3 \) y la base \( B_3' = (1, x, x^2) \) de \( \mathbb{R}^2[x] \). Calcular también bases del núcleo y de la imagen de \( f \).}
	Para resolver este ejercicio, deberemos calcular $M(I;B \leftarrow B_u), M(I;B_3' \leftarrow B')$ y $M(f;B' \leftarrow B_u)$:
	\begin{align*}
		M(I;B \leftarrow B_u) = M(I;B_u \leftarrow B)^{-1} = \begin{pmatrix}
			                                                     1 & 1 & 2 \\
			                                                     0 & 1 & 1 \\
			                                                     1 & 1 & 0
		                                                     \end{pmatrix}^{-1} = \begin{pmatrix}
			                                                                          0,5  & -1 & 0,5  \\
			                                                                          -0,5 & 1  & 0,5  \\
			                                                                          0,5  & 0  & -0,5
		                                                                          \end{pmatrix} \\
		M(I;B_3' \leftarrow B') = M(I;B' \leftarrow B_3')^{-1} = \begin{pmatrix}
			                                                         1 & 1 & 0  \\
			                                                         0 & 2 & 0  \\
			                                                         0 & 0 & -1
		                                                         \end{pmatrix}^{-1}
	\end{align*}
\end{ejercicio}

\begin{ejercicio}
	Calcular:
	\begin{enumerate}
		\item Una base \( B \) de \( \mathbb{R}^3 \) tal que \( M(\text{Id}_{\mathbb{R}^3, B_u \leftarrow B}) \) sea la matriz dada por:
		      \[ A = \begin{pmatrix} 0 & 0 & -1 \\ 0 & 1 & 1 \\ 1 & 3 & 2 \end{pmatrix} \]\\ \\
		      Tenemos que:
		      \begin{align*}
			      M & =(f:B_u \leftarrow B) =(I:B_u \leftarrow B_u)\cdot (\text{Id}_{\mathbb{R}^3}:B_u \leftarrow B_u)\cdot (I:B_u \leftarrow B) \\&= M(I : B_u \leftarrow B) =
			      \begin{pmatrix} 0 & 0 & -1 \\ 0 & 1 & 1 \\ 1 & 3 & 2 \end{pmatrix}
		      \end{align*}
		      Por tanto tomamos $B=\{(0,0,1),(0,1,3),(-1,1,2)\}$.
		\item Una base \( B \) de \( \mathbb{R}^3 \) tal que \( M(\text{Id}_{\mathbb{R}^3, B \leftarrow B_u}) \) sea la matriz \( A \) anterior.
		      \\ \\ Sin mas que tomar la inversa de la matriz anterior, tenemos que \\$B=\{(-1,1,-1),(-3,1,0),(1,0,0)\}$.
	\end{enumerate}
\end{ejercicio}

\begin{ejercicio}
	Sean \( f : V \rightarrow V' \) y \( g : V' \rightarrow V'' \) dos aplicaciones lineales. Supongamos que \( B \) es una base de \( V \), \( B'=\{ v_1',v_2',v_3'\} \) es una base de \( V' \), $B'' $es una base de \( V'' \) y que:
	\begin{equation*}
		M(f;B' \leftarrow B) = \begin{pmatrix} 1 & -1 & 0 & 2 \\ 0 & -2 & 1 & 1 \\ 2 & 3 & 0 & -2 \end{pmatrix}, \quad
		M(g;B'' \leftarrow \overline{B'}) = \begin{pmatrix} -1 & 0 & 2 \\ 0 & 2 & -2 \\ 0 & 1 & -2 \end{pmatrix}
	\end{equation*}
	donde \( \overline{B}' = (2v_2' - v_3', v_2' - v_3', 3v_1' + v_2' - v_3') \). Calcular \( M(g \circ f;B'' \leftarrow B) \).
	\\ \\
	En primer lugar observamos:
	\begin{equation*}
		M(g \circ f;B'' \leftarrow B) = M(g;B'' \leftarrow \overline{B}') \cdot M(I;\overline{B}'\leftarrow B')\cdot M(f;B' \leftarrow B)
	\end{equation*}
	Por lo que solo necesitamos calcular $M(I;\overline{B}'\leftarrow B')$, para ello, puede ser mas sencillo calcular $M(I;B' \leftarrow \overline{B}')$ y calcular su inversa ya
	que los vectores de $\overline{B}'$ se expresan fácilmente en función de los vectores de $B'$:
	\begin{equation*}
		M(I;B' \leftarrow \overline{B}')=\begin{pmatrix}
			0  & 0  & 3  \\
			2  & 1  & 1  \\
			-1 & -1 & -1
		\end{pmatrix} \Rightarrow M(I;\overline{B}'\leftarrow B') = \begin{pmatrix}
			0  & 0  & 3  \\
			2  & 1  & 1  \\
			-1 & -1 & -1
		\end{pmatrix}^{-1} = \begin{pmatrix}
			0            & 1  & 1  \\
			\frac{-1}{3} & -1 & -2 \\
			-1           & -1 & -1
		\end{pmatrix}
	\end{equation*}
	Una vez obtenida la matriz $M(I;\overline{B}'\leftarrow B')$, podemos calcular $M(g \circ f;B'' \leftarrow B)$:
	\begin{equation*}
		M(g \circ f;B'' \leftarrow B) = \begin{pmatrix}
			-1 & 0 & 2  \\
			0  & 2 & -2 \\
			0  & 1 & -2
		\end{pmatrix} \cdot \begin{pmatrix}
			0            & 1  & 1  \\
			\frac{-1}{3} & -1 & -2 \\
			-1           & -1 & -1
		\end{pmatrix} \cdot \begin{pmatrix}
			1 & -1 & 0 & 2  \\
			0 & -2 & 1 & 1  \\
			2 & 3  & 0 & -2
		\end{pmatrix} = \begin{pmatrix}
			1 & -1 & 0 & 2  \\
			0 & -2 & 1 & 1  \\
			2 & 3  & 0 & -2
		\end{pmatrix}
	\end{equation*}
\end{ejercicio}


\begin{ejercicio}
	Se consideran dos aplicaciones lineales entre espacios vectoriales sobre el mismo cuerpo $f : V \mapsto V'$ y $g : V' \mapsto V''$. Demostrar que:
	\begin{equation*}
		g \circ f \text{ es la aplicación nula } \Leftrightarrow \text{ Im}(f) \subseteq \text{ Nuc}(g)
	\end{equation*}
	\begin{itemize}
		\item $\Rightarrow$: Supongamos que $g \circ f$ es la aplicación nula, entonces:
		      \begin{equation*}
			      \forall v \in V, g(f(v)) = 0 \Rightarrow f(v) \in \text{Nuc}(g) \Rightarrow \text{Im}(f) \subseteq \text{Nuc}(g)
		      \end{equation*}
		\item $\Leftarrow$: Supongamos que $\text{Im}(f) \subseteq \text{Nuc}(g)$, entonces:
		      \begin{equation*}
			      \forall v \in V, f(v) \in \text{Im}(f) \subseteq \text{Nuc}(g) \Rightarrow g(f(v)) = 0 \Rightarrow g \circ f \text{ es la aplicación nula}
		      \end{equation*}
	\end{itemize}
\end{ejercicio}


\begin{ejercicio}
    Sea $f: V \rightarrow V$ un endomorfismo de un espacio vectorial tal que $f \circ f = 0$. Demuéstrese:
    \begin{itemize}
        \item Si $v_1, ..., v_r \in V$ verifican que $(f(v_1), ..., f(v_r))$ es linealmente independiente, entonces $(v_1, ..., v_r, f(v_1), ..., f(v_r))$ es linealmente independiente.
        \item Si $\dim_K(V) = n \in \mathbb{N}$, existe una base $B$ de $V$ tal que, escribiendo la matriz por cajas:
        $$M(f,B) = \begin{pmatrix} 0 & I_r \\ 0 & 0 \end{pmatrix}$$
        donde $I_r$ es la matriz identidad de orden $r \leq n/2$.
    \end{itemize}
\end{ejercicio}

\begin{ejercicio}
    Sea $f: \mathbb{R}^4 \rightarrow \mathbb{R}^4$ un endomorfismo del que se sabe que:

    $$f(1,1,0,0) = (0,1,0,-1) \text{ y } f(1,0,1,0) = (1,1,1,0)$$

    Calcular la matriz de $f$ respecto de la base usual en cada uno de los siguientes casos:

    \begin{enumerate}
        \item[a)] $\text{Nuc}(f) = \text{Im}(f)$.
        \item[b)] $f \circ f = f$.
        \item[c)] $f \circ f = Id_{\mathbb{R}^4}$.
    \end{enumerate}

    ¿Cuál es, en cada caso, la imagen del vector $(1,3,7,1)$?
\end{ejercicio}

\begin{ejercicio}
    Dadas las matrices 
    \begin{equation*}
        A = \begin{pmatrix} 1 & 5 & -1 & 2 \\ 2 & 1 & 4 & 1 \\ 1 & 2 & 1 & a \end{pmatrix} \quad \text{ y } \quad A' = \begin{pmatrix} 2 & 0 & 1 & 2 \\ 1 & -1 & 2 & 3 \\ 0 & 2 & -3 & -4 \end{pmatrix}
    \end{equation*}
    \begin{enumerate}
        \item[a)] Calcular los valores de $a$ para los que $A$ y $A'$ son equivalentes.
        \item[b)] Para dichos valores de $a$ encontrar matrices $P \in GL(4, \mathbb{R})$ y $Q \in GL(3, \mathbb{R})$ tales que $A' = Q^{-1} \cdot A \cdot P$.
        \item[c)] Para los valores de $a$ calculados en el primer apartado se considera la aplicación lineal $f: \mathbb{R}^4 \rightarrow \mathbb{R}^3$ cuya matriz con respecto a las bases usuales es $A$. Calcular una base $B$ de $\mathbb{R}^4$ y una base $B'$ de $\mathbb{R}^3$ de forma que $M(f, B' \leftarrow B) = A'$.
    \end{enumerate}
\end{ejercicio}
% FALTAN AQUI

\begin{ejercicio}
Decidir razonadamente si las siguientes afirmaciones son verdaderas o falsas:
\begin{itemize}
    \item[a)] Existe un endomorfismo $f$ de $\mathbb{R}^3$ tal que $f(1,0,0) = (2,0,1)$, $f(0,1,0) = (0,0,0)$ y $f(1,1,0) = (2,-1,7)$.
    \item[b)] Existe una aplicación lineal $f: C \rightarrow C$ distinta de la aplicación lineal cero y con núcleo distinto de $\{0\}$.
    \item[c)] Si $A \in M_{m \times n}(\mathbb{R})$ y $B \in M_{m \times n}(\mathbb{R})$ cumplen que $A \cdot B = I_n$ y $B \cdot A = I_m$, entonces $m = n$.
    \item[d)] Si $n \in \mathbb{N}$ y $m \leq n$ entonces existe un epimorfismo $f: \mathbb{R}^n \rightarrow \mathbb{R}^m$.
    \item[e)] Si existe un isomorfismo $f: C^n \rightarrow M_2(\mathbb{C})$, existe un monomorfismo $f: \mathbb{R}^n \rightarrow \mathbb{R}^m$.
    \item[f)] Para cualesquiera $m, n \in \mathbb{N}$ se cumple que $\mathbb{R}^{m \times n}$ es isomorfo a $\mathbb{R}^{m \cdot n}$.
    \item[g)] Si un sistema de ecuaciones lineales tiene menos ecuaciones que incógnitas entonces el sistema no puede ser compatible determinado.
    \item[h)] Existe un automorfismo $f$ de $\mathbb{R}^2$ de forma que: 
    $$f(\{ (x,y) \in \mathbb{R}^2 / x=0 \}) = \{ (x,y) \in \mathbb{R}^2 / x^3=0 \}.$$
    \item[i)] Para cada $r \in \mathbb{R}$ la aplicación $f: \mathbb{R}^2[x] \rightarrow \mathbb{R}^2[x]$ dada por:
    $$f(ax^2 + bx + c) = rax^2 + bx + c$$
    es un automorfismo.
    \item[j)] Para cada $r \in \R$ la aplicación $f_r : \R^2[x] \rightarrow \R^2[x]$ dada por: 
    \begin{equation*}
        f_r(ax^2 + bx + c) = rax^2 + bx + c
    \end{equation*}
    es un automorfismo.
\end{itemize}
\end{ejercicio}

\begin{ejercicio}
    (La traza de un endomorfismo) Sea $n \in \mathbb{N}$ y $A = (a_{ij}) \in M_n(K)$. Se define la traza de $A$ como el escalar de $K$ dado por:
    $$\text{tr}(A) = \sum_{i=1}^n a_{ii}$$
    
    Se pide lo siguiente:
    \begin{itemize}
        \item[a)] Demostrar que la aplicación $\text{tr}: M_n(K) \rightarrow K$ que asocia a cada matriz cuadrada su traza es lineal.\\
        Sea $A,B \in M_n(K)$
        \begin{equation*}
            tr(A+B) = \sum_{i=1}^n (a_{ii}+b_{ii}) = \sum_{i=1}^n a_{ii} + \sum_{i=1}^n b_{ii} = tr(A) + tr(B)
        \end{equation*}	
        Sea $K\in K$
        \begin{equation*}
            tr(KA) = \sum_{i=1}^n (k\cdot a_{ii}) = k\sum_{i=1}^n a_{ii} = k\cdot tr(A)
        \end{equation*}
        Por tanto, $tr: M_n(K) \rightarrow K$ es lineal.
        \item[b)] Probar que $\text{tr}(AB) = \text{tr}(BA)$ para cualesquiera $A, B \in M_n(K)$. Deducir que dos matrices semejantes tienen la misma traza.\\
        Sea $AB=C; BA=C'$\\
        Los elementos de la diagonal principal de $C$ se calculan:
        \begin{equation*}
            C_{ii}=\sum_{j=1}^n a_{ij}b_{ji} \qquad \forall i \in \{1,...,n\}
        \end{equation*}
        \begin{equation*}
            C'{ii}=\sum{j=1}^n b_{ij}a_{ji} \qquad \forall i \in \{1,...,n\}
        \end{equation*}
        Por tanto,
        \begin{equation*}
            tr(AB)=\sum_{i=1}^{n}C_{ii}=\sum_{i=1}^{n}\sum_{j=1}^{n}a_{ij}b_{ji}=\sum_{i,j=1}^{n}a_{ij}b_{ji}
        \end{equation*}
        \begin{equation*}
            tr(BA)=\sum_{i=1}^{n}C'{ii}=\sum{i=1}^{n}\sum_{j=1}^{n}b_{ij}a_{ji}=\sum_{i,j=1}^{n}b_{ij}a_{ji}
        \end{equation*}
        Por tanto, como itera sobre $i,j$ y, por la propiedad conmutativa en $K$,\\
        $tr(AB)=tr(BA)$\\ \\
        $A,C \in M_n(K)$ son semejantes $\Leftrightarrow \exists P\in M_n(K)/C=P^{-1}AP$\\
        Por tanto, supuestos $A$ y $C$ semejantes,\\
        \begin{equation*}
            C=P^{-1}AP
        \end{equation*}
        \begin{equation*}
            A=P\cdot C\cdot P^{-1}
        \end{equation*}
        \begin{equation*}
            tr(A)=tr(P\cdot [C\cdot P^{-1}])=tr([C\cdot P^{-1}]P)=tr(C\cdot P^{-1}P)=tr(C)
        \end{equation*}
        \item[c)] Utilizar el apartado anterior para definir la traza de un endomorfismo de un espacio vectorial $V$ sobre $K$.\\ \\
        $tr(f)=tr(M(f,B))$, con $B$ base de $V$.
        Veamos que no depende de la base escogida.
        Sea $B'$ base de $V$
        $M(f,B')$ es semejante a $M(f,B)$, ya que son matrices asociadas al mismo endomorfismo $f$ respecto de distintas bases.\\
        Por tanto, $tr(M(f,B'))=tr(M(f,B))=tr(f)$
        Por tanto, no depende de la base escogida.\\
        \begin{equation*}
            tr(f)=tr(M(f,B))
        \end{equation*}
        \item[d)] Encontrar dos matrices con el mismo rango y la misma traza que no sean semejantes.\\
        \begin{equation*}
            Sea
            \begin{pmatrix}
            1 & 0 & 0\\
            0 & 1 & 0\\
            0 & 0 & -2
        \end{pmatrix}
        , |A|=-2
        \end{equation*}
        \begin{equation*}
            \begin{pmatrix}
            2 & 0 & 0\\
            0 & 1 & 0\\
            0 & 0 & -3
            \end{pmatrix}
            , |B|=-6
        \end{equation*}
        Vemos que $Rg(A)=Rg(B)=3$ y $tr(A)=tr(B)=0$
        Supongamos que $A$ semejante a $B \Rightarrow \exists P\in M_n(K)/B=P^{-1}AP$
        Por tanto, $|B|=|P^{-1}AP|=|P^{-1}||A||P|=|P^{-1}||P||A|=|A|$
        Es decir, $|A|=|B|$, por tanto, $A$ no es semejante a $B$
        $A$ y $B$ tienen el mismo rango y la misma traza, pero no son semejantes.
    \end{itemize}
    \end{ejercicio}  





\newpage

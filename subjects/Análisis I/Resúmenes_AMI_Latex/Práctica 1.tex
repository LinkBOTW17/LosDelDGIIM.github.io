\chapter{Práctica 1. Continuidad}

\section{Teoremas relacionados}
\begin{teo}[Carácter local de la continuidad]
    Sean $E$, $F$ espacios topológicos, $\emptyset \neq A \subseteq E$, $x \in A$.

    
    Si $f_{\big| A}$ es continua en $x$ con $A \in \mathbb{U}(x) \Rightarrow f$ es continua en $x$.
\end{teo}

\begin{teo}[Cambio de variable]
    Sean $E$, $F$ espacios métricos, $\emptyset \neq A \subseteq E$, $f:A \rightarrow F$
    y $\alpha \in E$:\\

    
    Si $G$ es un espacio métrico con $T \subseteq G$ y $\varphi:T \rightarrow E$, $z \in T'$ que cumple:
    $$\lim_{t \to z}\varphi(t)=\alpha \in E ~~~~ \varphi(t) \in A\setminus\{\alpha\} ~\forall t \in T\setminus\{z\}$$
    Entonces, $\alpha \in A'$ y se verifica que:
    $$\lim_{x \to \alpha} f(x) = L \Rightarrow \lim_{t \to z}f(\varphi(t)) = L$$
\end{teo}

\vspace{2cm}

Procedemos por tanto a estudiar el siguiente problema:\\

\noindent
Dada $f:E \rightarrow \R^n$ con $E \subseteq \R^m$, $n,m \in \N$. Comprobar que $f$ es continua.

\section{Parte rutinaria del problema}
\begin{itemize}
    \item Definimos $U$ y comprobamos que $U$ sea abierto.
    \item Comprobamos que $f_{\big|U}$ sea continua.
    \item Aplicamos el \underline{carácter local de la continuidad} y tenemos que $f$ es continua en $U$.
\end{itemize}


A continuación se nos presentan distintos puntos problemáticos en los que querremos
estudiar el límite. Nos fereriremos a un punto de estos como $\alpha$.
Calculamos:
$$\lim_{x \to \alpha} f(x)$$
Normalmente, se presentará una indeterminación. A continuación, la intuición nos dirá
si debemos intentar probar que el límite no existe o intentar probar la existencia del límite.


Un camino algo más mecánico es comprobar, en este orden, los límites parciales, límites direccionales,
intentar probar la existencia de límite y, por último, intentar probar que el límite no
existe con un cambio de variable.

\section{El límite no existe}

Si creemos que el límite en $\alpha$ no existe, el procedimiento a seguir es el siguiente:

\subsection{Límites parciales}

Si $e_k$ es el $k$-ésimo vector de la base usual. Sea $t \in \R \mid x \rightarrow \alpha$ si
$t \rightarrow 0$ con $x \neq \alpha$ si $t \neq 0$. Entonces:
$$\lim_{x \to \alpha}f(x) = L \Rightarrow \lim_{t \to 0}f(\alpha + te_k)=L$$
En el caso $n=2$, si $\alpha = (a,b)$. Entonces:
$$\lim_{x \to (a,b)}f(x) = L \Rightarrow \lim_{x \rightarrow a}f(x,b) = \lim_{x \rightarrow b}f(a,x)=L$$

\begin{itemize}
    \item Si uno de los límites parciales no existe, podemos afirmar ($\ast$).
    \item Si existen los dos límites parciales y no son iguales, podemos afirmar ($\ast$).
    \item En caso de que existan y sean iguales, el único candidato a límite será $L$, por lo que si por
          otro método nos sale que el límite no es $L$, podemos afirmar ($\ast$).
\end{itemize}
$$(\ast)\hspace{1cm}\nexists \lim_{x \to \alpha}f(x)$$

\subsection{Límites direccionales}
$$S = \{u \in E\setminus\{0\} \mid \|u\| = 1\}$$


Sea $u \in S$. Entonces, si $t \in \R$:
$$\lim_{x \to \alpha}f(x) = L \Rightarrow \lim_{t \to 0}f(\alpha + tu) = L~~~~~~\forall u \in S$$
El cálculo lo haremos con un $u$ genérico que cumpla estas premisas, de forma que:
\begin{itemize}
    \item Si uno de los límites direccionales no existe, podemos afirmar ($\ast$).
    \item Si el límte direccional depende de $u$, podemos afirmar ($\ast$) al saber que si cambiamos
          $u$ obtenemos distintos valores del límite.
    \item En caso de que existan y sean iguales, el único candidato a límite será $L$, por lo que si por
          otro método nos sale que el límite no es $L$, podemos afirmar ($\ast$).
\end{itemize}

Hay que tener en cuenta que según el $E$ a veces no podemos estudiar ciertos límites direccionales.

\subsubsection{Límites radiales}
$$\lim_{x \to \alpha}f(x) = L \Rightarrow \lim_{t \to 0^{+}}f(\alpha + tu) = L~~~~~~\forall u \in S$$
Hay que tener en cuenta que según el $E$ a veces no podemos estudiar ciertos límites radiales.


Caso $n=2$:
\subsubsection{Coordenadas polares}
$$\alpha = (x,y)$$
$$u \in \R^2 \mid \|u\| = 1 \Leftrightarrow u =(\cos\theta,\sin\theta)~~\theta \in \R$$
$$\rho = \sqrt{x^2+y^2} \in \R$$
\ \\
$$\lim_{t \to 0^{+}}f(x) = \lim_{\rho \to 0}f(a+\rho \cos\theta, b+\rho \sin\theta)$$
$$\lim_{x \to \alpha}=L \Rightarrow \lim_{\rho \rightarrow 0}f(a+\rho \cos\theta, b+\rho \sin\theta)=L~~~
    ~~~ \forall \theta \in \R$$

\subsubsection{Coordenadas cartesianas}
$$u=(u_1, u_2) \in \R^2$$
En vez de normalizar con $\|u\|=1$, tomamos:
$$u_1 = 1 \mbox{ y } u_2 = \lambda \in \R$$
$$\lim_{t \to 0}f(\alpha + tu) = \lim_{t \to 0}f(a+t,b+\lambda t)$$
$$\lim_{x \to \alpha}f(x) = L \Rightarrow \lim_{t \to 0}f(a+t,b+\lambda t) = L~~~~~~ \forall \lambda \in \R$$

\section{Existencia del límite}

La \underline{única} forma de probar que $\lim\limits_{x \to \alpha}f(x) = L \in \R$ es acotando $f$:\\


Necesitamos hallar $r \in \R^{+}$ y $g:B(\alpha,r)\rightarrow \R^{+}$ tal que:
$$0 \leq \left| f(x) - L \right| \leq g(x) ~~~~\forall x \in B(\alpha,r)\setminus\{\alpha\}$$

De tal forma que $$\lim_{x \to \alpha}g(x) = 0$$


Entonces, por el lema del Sándwich, tenemos que:
$$\lim_{x \to \alpha}f(x) = L$$


El estudio fracasado de los límites direccionales puede ayudarnos a la hora de determinar de
forma más fácil una acotación:

\subsection{Acotación por límites direccionales}

Si el estudio de los límites direccionales fracasó fue porque:
$$\lim_{t \to 0}f(\alpha + tu)-L = 0~~~~~~\forall u \in S$$
Luego si hallamos $r \in \R^{+}$ y una función $h:]0,r[\rightarrow \R^{+}$ con $\lim\limits_{t\to 0}h(t) = 0$,
de forma que:
$$0 \leq |f(\alpha + tu)-L| \leq h(t)~~~~~~\forall u \in S~~\forall t \in ]0,r[$$
Tendremos que: $$\lim_{x \to \alpha}f(x) = L$$


En el caso $n=2$:
\subsection{Acotación por uso de coordenadas polares}

Si el estudio de los límites direccionales usando coordenadas polares fracasó fue porque:
$$\lim_{\rho \to 0}f(a + \rho \cos\theta, b + \rho \sin\theta)-L = 0~~~~~~\forall \theta \in \R$$
Luego si hallamos $r \in \R^{+}$ y una función $h:]0,r[\rightarrow \R^{+}$ con $\lim\limits_{\rho\to 0}h(\rho) = 0$,
de forma que:
$$0 \leq |f(a+\rho \cos\theta,b+\rho \sin\theta)-L| \leq h(\rho)~~~~~~\forall \theta \in \R~~\forall \rho \in ]0,r[$$
Tendremos que: $$\lim_{x \to \alpha}f(x) = L$$

\subsection{Acotación por uso de coordenadas cartesianas}

Si el estudio de los límites direccionales usando coordenadas cartesianas fracasó fue porque:
$$\lim_{t \to 0}f(a + t, b + \lambda t)-L = 0~~~~~~\forall \lambda \in \R$$
Luego si hallamos $r \in \R^{+}$ y una función $h:]0,r[\rightarrow \R^{+}$ con $\lim\limits_{t\to 0}h(t) = 0$,
de forma que:
$$0 \leq |f(a+t,b+\lambda t)-L| \leq h(t)~~~~~~\forall t \in \R~~\forall t \in ]-r,r[\setminus\{0\}$$
Tendremos que: $$\lim_{x \to \alpha}f(x) = L$$

\section{Último recurso}

Si no pudimos encontrar ninguna acotación de $f$, deberemos intuir que el límite no existe.
Para probar esto, tenemos que idear un cambio de variable nuevo:\\


Si $L \in \R$ es el único posible límite de $f$ en $\alpha$, podemos probar con un cambio de variable
$x = \varphi(t)$ con $0 < t < r$ tal que:
$$\lim_{t \to 0}\varphi(t) = \alpha ~~~~\mbox{ y } ~~~~ \varphi(t) \neq \alpha ~~\forall t \in ]0,r[$$
$$\lim_{x \to \alpha}f(x) = L \Rightarrow \lim_{t \to 0}f(\varphi(t)) = L$$
Luego buscamos $\varphi$ de forma que $f \circ \varphi$ no tenga límite en 0.\\


Por ejemplo, en el caso $n=2$ con $\alpha = (a,b)$, podemos hacer el cambio de variable:
$$\varphi_p(t)=(a+t, b+t^p)~~~~~~~p \in \R^{+}$$
De forma que calculamos el límite con un $p \in \R^{+}$ cualquiera y luego fijamos un valor de $p$
para el cual el límite no exista.\\


Otro recurso que podemos usar es que si $n=2$ y nuestra función $f$ es un cociente entre dos
términos que contienen $x$ e $y$ de forma que el exponente de $y$ es siempre el doble de $x$, podemos
usar el cambio de variable:
$$\varphi_2(t) = (a+t,b+t^2)$$

\section{Límites famosos}
\begin{equation*}
    \begin{array}{ccc}
        \displaystyle \lim_{t \to 0}\dfrac{\sen t}{t} = 1
        &
        \displaystyle  \lim_{t \to 0}\dfrac{\tan t}{t} = 1
        &
        \displaystyle  \lim_{t \to 0}\dfrac{\arcsen t}{t} = 1\\ \\
        \displaystyle  \lim_{t \to 0}\dfrac{\arctan t}{t} = 1
        &
        \displaystyle  \lim_{t \to 0}\dfrac{1-\cos t}{t^2} = \dfrac{1}{2}
        & 
        \displaystyle  \lim_{t \to 0}\dfrac{e^t -e^0}{t} = 1\\ \\
        &\displaystyle  \lim_{t \to 0}\dfrac{\log(1+t)}{t}=1
    \end{array}
\end{equation*}

\subsection{Cambiar forma de la función}

Dada una función a la que le queremos calcular un límite, es recurrente que nos sepamos a qué tiende
parte del límite ya que nos es conocido y querramos descomponer la función en dos partes, una
de la que conocemos su límite y otra que será más sencillo de calcular. La pregunta es cómo podemos
hacer esto formalmente y sin fallos. Para ello, pondremos el ejemplo de:
$$f(x,y) = \dfrac{x^2 \sen y}{x^2+y^2}~~\mbox{ si } x \in \R^2 \setminus\{(0,0)\}$$
De tal forma que queremos calcular el límite en el punto $(0,0)$. Para ello, uno podría pensar
que podemos hacer:
$$f(x,y) = \dfrac{\sen y}{y} \dfrac{x^2y}{x^2+y^2}$$
Pero debemos tener cuidado, ya que nuestro dominio es $\R^2\setminus\{(0,0)\}$ y al cambiar la
expresión de $f$ estamos dividiendo por cero al considerar cualquier punto del estilo $(a,0)$ con
$a \neq 0$ en nuestro dominio.
Para solucionar este problema, resolveremos el ejercicio de la siguiente forma:\\


Sea $\varphi:\R \rightarrow \R$ una función tal que:
$$\varphi(y) = \dfrac{\sen y}{y}~~~~\forall y \neq 0~~~~~~~~\varphi(0)=1$$
Notemos que $\varphi$ es continua en todo $\R$, al ser $\lim\limits_{y \to 0}\varphi(y) = 1 = \varphi(0)$.


De esta forma, hemos conseguido una función que nos permite hacer lo siguiente:
$$\sen y = y\varphi(y)~~~~~~\forall y \in \R$$
$$f(x,y)=\varphi(y)\dfrac{x^2y}{x^2+y^2}~~~~~~\forall (x,y) \in \R^2\setminus\{(0,0)\}$$
De esta forma, podemos estudiar $f$ en dos partes:\\


Por una lado, sabemos que:
$$\lim_{y \to 0} \varphi(y) = \varphi(0) = 1$$


Y por otro, tenemos que podemos acotar fácilmente la función, haciendo que el otro trozo converja a cero:
$$0 \leq \left| \dfrac{x^2y}{x^2+y^2} \right| = \dfrac{x^2|y|}{x^2+y^2} \leq |y|$$
Con $\lim\limits_{y \to 0}y = 0$. Luego:
$$\lim_{(x,y) \to (0,0)} \dfrac{x^2y}{x^2+y^2}=0$$


Por lo que, finalmente:
$$\lim_{(x,y) \to (0,0)} \dfrac{x^2\sen y}{x^2+y^2} = \lim_{(x,y) \to (0,0)} \varphi(y)\dfrac{x^2y}{x^2+y^2}=0$$

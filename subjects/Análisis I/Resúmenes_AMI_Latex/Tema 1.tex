\chapter{Espacios euclídeos, normados y métricos}
\section{Espacio Euclídeo}
    \begin{definicion} \\ 

    \noindent
        $\R^N \\ N \in \N\quad fijo \quad \Delta_N = \{k \in \N / k <= N\}$ \newline
        $\R = \R \times \ldots \times \R = \{ (x_1, \ldots, x_N) / x_1, \ldots , x_N \in \R\} $\newline
        $x = (x_1, \ldots , x_N) \in \R^N \\ \longleftrightarrow \quad X : \Delta_N \longrightarrow \R \quad con \quad x(k) = x_k \quad \forall k \in \Delta_N$
    \end{definicion}
    \begin{definicion}
        Sean $x, y \in \R^N$ y  $\lambda \in \R$
        \begin{itemize}
            \item Suma:
            $ x + y = (x_1+y_1, \ldots, x_N + y_N)$ \\
            $ (x+y)(k) = x(k) + y(k) \\ \forall k \in \Delta_N$
            \item Producto por escalares:
            $\lambda x = (\lambda x_1, \ldots , \lambda x_N)$ \\
            $(\lambda x)(k) =  \lambda x(k)$
        \end{itemize}
    \end{definicion}
    Así, $\rn$ es un espacio vectorial sobre $\R$

    
    \begin{definicion}[Base usual]
        $\phi = \{ e_1, \ldots , e_N\}$ 
        $$e_k(j) = \left\{ \begin{array}{lll}
            1 & si & j=k\\
            0 & si & j\neq k\\
        \end{array} \right. \\ \forall k \in \Delta_N$$
        $$x = \sum_{k=1}^N x(k)e_k \\ \forall x \in \R^N$$
    \end{definicion}
        
    \begin{definicion}[Producto escalar]
        Sean $x$, $y$ $\in \rn$
        $$(x|y) = \sum_{k=1}^N x(k)y(k)$$
        $$(\cdot|\cdot):\R^N\times \rn \longrightarrow\R$$
    \end{definicion}
    
        \begin{enumerate}
            \item $(\lambda u + \mu v|y) = \lambda(u|y) + \mu (v|y)$
            \item $(x|y) = (y|x) $
            \item $(x|x) > 0$
        \end{enumerate}
    Un espacio pre-hilbertiano es un nespacio vectorial dotado de un producto escalar.
        

\section{Espacios normados}
    \begin{definicion}[Norma]
        $$||\cdot||: X \longrightarrow\R$$
        $$x \longmapsto ||x|| = (x|x)^{\frac{1}{2}}$$   
    \end{definicion}
    \begin{propiedades} []\\
        \begin{enumerate}
            \item $||x+y|| \leq||x||+||y||$
            \item $||\lambda x|| = |\lambda | \cdot||x||$
            \item $||x|| > 0;\\ ||x|| = 0 \sii x = 0$
            \item Desigualdad de Cauchy-Schwartz:
            $|(x|y)| \leq ||x||\cdot ||y|| \\ \forall x, y\in X$
            \item $|||x||-||y||| \leq ||x\pm y|| \leq ||x|| + ||y||$
            \item $n\in \N, \\ x_1, \ldots , x_n\in X, \\ \lambda_1, \ldots, \lambda_n\in \R$ \newline
            $$||\sum_{k=1}^n \lambda_k x_k|| \leq \sum_{k=1}^n |\lambda_k|\cdot||x_k||$$
        \end{enumerate}
    \end{propiedades}
    Un espacio normado es un espacio vectorial $X$ dotado de una norma.
    \newline
    
    \begin{ejemplo}[Norma de la suma]
        $||x||_1 = \sum |x(k)|$
    \end{ejemplo}

    \begin{ejemplo}[Norma infinito]
        $||x||_\infty = \max\{|x(k)| / k \in \Delta_N\}$ 
    \end{ejemplo}
        
    

\section{Espacio métrico}
    \begin{definicion}[Distancia]
        $d(x,y) = ||y - x||$; 
        $||x|| = d(0,x )$
    \end{definicion}
        \begin{enumerate}
            \item $d(x,y) \leq d(x,z) + d(z,y) \\ \forall x,y,z \in X$
            \item $d(x,y) = d(y,x)\\ \forall x,y \in X$
            \item $d(x,y) =0 \sii x=y$
            \item $d(x,y) \geq 0\\ \forall x,y \in E$
            \item $|d(x,z) - d(z,y)| \leq d(x,y)$
            \item $d(x_0,x_n) \leq \sum\limits_{k=1}^n d(x_{k-1},x_k)$
        \end{enumerate}

    
        Cuando tenemos definida una distancia $d:E\times E \longrightarrow \R$, cumpliendo D1, D2 y D3, decimos que E es un espacio métrico. E no tiene que ser un espacio vectorial.
        
        Todo espacio normado $X$ se considera un espacio métrico con la distancia asociada a su norma: 
        $$d(x,y) = ||y-x||$$

    \begin{ejemplo}[Distancia discreta]
        $$\delta (x,y) = \left\{ \begin{array}{lll}
            1 & si & x\neq y\\
            0 & si & x= y\\
        \end{array} \right.$$
    \end{ejemplo}
    


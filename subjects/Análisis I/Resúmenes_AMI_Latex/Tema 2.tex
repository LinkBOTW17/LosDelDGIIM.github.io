\chapter{Topología de un espacio métrico}
\section{Topología de un espacio métrico}
    Para lo que sigue sea $E$ un espacio métrico con distancia $d$.\\
    \begin{definicion}
        Bolas abiertas de centro x y radio r:
        $$B(x,r) = \{ y \in E / d(x,y)< r\} \\ x \in E, \\ r \in \R^+ $$
        \begin{enumerate}
            \item $B(x,r) \neq \emptyset$
            \item $0 < s < r \Longrightarrow B(x,s) \subset B(x,r)$
            \item $\forall y \in B(x,r) \\ \exists \varepsilon >0 \\ / \\ B(y,\varepsilon) \subset B(x,r)$
            \item Se dice $U\in E$ abierto si $\forall x \in U \\\exists \varepsilon > 0 \\/ \\B(x, \varepsilon ) \subset U$
        \end{enumerate}
    \end{definicion}
    
    \begin{definicion}[Topología]
        En un conjunto no vacío $\Omega$ es una familia $T \subset P(\Omega)$ que verifica:
        \begin{itemize}
            \item[(A1)] $\emptyset, \Omega \in T$
            \item[(A2)] $S \subset T \Longrightarrow \cup S \in T$
            \item[(A3)] $U, V \in T \Longrightarrow U \cap V \in T$
        \end{itemize}
    Un espacio topológico es un conjunto no vacío provisto de una topología. 
    \end{definicion}

    \begin{definicion}[Normas equivalentes]
        Dos distancias/normas en un mismo conjunto/espacio vectorial son equivalentes cuando generan una misma topología. 
    \end{definicion}
    \begin{prop}[Inclusion entre las topologías de dos normas] 
        Para dos normas $||\cdot ||_1||$, $||\cdot ||_2$ definidas en un mismo espacio vectorial $X$, equivalen: \\
        
        (i)  $\exists \rho >0$ / $||x||_2 \leq \rho ||x||_1$  $\forall x \in X$ \\
        
        (ii)  $T_2 \subset T_1$
    \end{prop}

    
    \begin{coro}[Criterio de equivalencia entre normas]
        $$T_1 = T_2 \\ \sii \\ \exists \lambda, \rho >0 \\ /\\ \lambda ||x||_1 \leq ||x||_2 \leq \rho ||x||_1 \\  \forall x \in X$$
    \end{coro}
    
    \begin{definicion}[Topología inducida]
        Si $T$ es la topología de un espacio métrico $E$ y $T_A$ la de un subconjunto $A \subset E$, entonces: \\
        $$T_A = \{ U\cap A : U \in T\}$$
        En particular si $A\in T$, entonces:
        $$V \in T_A \\ \sii \\ V \in T$$
    \end{definicion}

%-------------------------------------------------------------------------------------------
\section{Primeras nociones topológicas}
    En todo lo que sigue $E$ espacio métrico, $d$ distancia y $T$ topología.

    \begin{definicion}[Interior y entornos]\\
        \begin{enumerate}
            \item \textbf{Interior}:
            $$A^o = \cup \{U\in T: U \subset A\}$$
            Si $x\in A^o$ entonces decimos que $x$ es \textbf{punto interior} de $A$ o que $A$ es entorno de $x$. Denotamos por $U(x)$ a la \textbf{familia de entornos} de $x$. 
            \item $A^o$ es el máximo abierto de $E$ contenido en $A$.
            \item $x\in A^o \sii A\in U(x) \sii \exists \epsilon>0 : B(x, \epsilon ) \subset A$ 
            \item $A\in T \sii A=A^o \sii A\in U(x) \\ \forall x \in A$
            \item $A\in U(x), \\ A \subset C \subset E \Longrightarrow C \in U(x)$
            \item $\{ A_i \}_{i=1...n} \subset U(x) \Longrightarrow \cap_{i=1}^n A_i \in U(x)$
        \end{enumerate}
           
    \end{definicion}
   
    \begin{definicion}[Conjuntos cerrados]
        Decimos que $C$ es un conjunto cerrado, o simplemente \textbf{cerrado} si $E\setminus C \in T$. \\
        $$C_T = \{ E\setminus U : U \in T\}$$
        \begin{enumerate}
            \item $\emptyset, E \in C_T$
            \item $D \subset C_t  \implies \cap D \in C_T$
            \item Sea $ n\in \N$ fijo, $\{ C_i \}_{i=1...n} \subset C_T \implies \cup_{i=1}^n C_i \in C_T$ 
        \end{enumerate}    
    \end{definicion}
    
    \begin{definicion}[Cierre]
        $$\overline{A} = \cap \{ C \in C_T : A \subset C\}$$
        \begin{enumerate}
            \item $\overline{A}$ es el mínimo conjunto cerrado que contiene a $A$.
            \item $A \in C_T \sii A = \overline{A}$
            \item $E\setminus \overline{A} = (E\setminus A)^o \\ \forall A \in E$
            \item $E\setminus A^o = \overline{E\setminus A} \\ \forall A \in E$
        \end{enumerate}
    \end{definicion}

    \begin{definicion}[Punto adherente a un conjunto]
        $Sea \\ x \in A \implies$
        $$X\in \overline{A} \\ \sii\\ U \cap A \neq \emptyset \\ \forall U \in U(x) \\ \sii\\ B(x,\epsilon ) \cap A \neq \emptyset \\ \forall \epsilon >0$$
    \end{definicion}

    \begin{definicion}[Bola cerrada]
        $\overline{B}(x, r) = \{ y \in E : d(x,y) \leq r\}$
    \end{definicion}

    \begin{definicion}[Esfera]
        $S(x,r) = \{ y \in E: d(x,y) = r\}$
    \end{definicion}

    \begin{definicion}[Frontera]
        $Fr(A) = \overline{A} \setminus A^o$
    \end{definicion} 
    
    \begin{enumerate}
        \item $Fr(A) = \overline{A} \cap \overline{E\setminus A}$
        \item $Fr(A) \in C_T$
        \item $Fr(A) = Fr(E\setminus A)$
        \item $\overline{A} = A\cup Fr(A)$ y $A^o = A \setminus Fr(A)$
        \item $A \in T \\ \sii \\ A \cap Fr(A) = \emptyset$
        \item $A \in C_T \\ \sii \\ Fr(A) \subset A$
        \item $E = A^o \cup Fr(A) \cup (E\setminus A)^o$ es una partición de $E$, es decir:
        $$A^o \cap Fr(A) =  Fr(A) \cap (E\setminus A)^o = A^o \cap (E\setminus A)^o = \emptyset$$
    \end{enumerate}

    \begin{definicion}[Puntos de acumulación]
        Se dice de $x\in A$ si es punto adherente de $A\setminus \{x\}$
        $$A' = \overline{A\setminus \{x\}} = \{x \in E : \forall U \in U(x) \\ U \cap A\setminus \{x\} \neq \emptyset \} $$
        $$= \{x \in E : B(x, \epsilon) \cap (A\setminus \{x\} \neq \emptyset \\ \forall \epsilon >0\}$$
    \end{definicion}    

    \begin{definicion}[Puntos aislados]\\
        $$x \in \overline{A} \setminus A' \\ \sii \\ \exists U \in U(x) / U \cap A = \{x\}\\\sii \\ \exists \varepsilon >0 / B(x,\varepsilon) \cap A = \{x\}$$
        $$\overline{A} = A' \cup A \\ \implies \\ A \in C_T \sii A'\subset A$$
    \end{definicion}

    
\section{Convergencia de sucesiones}

    En lo que sigue $\{x_n\}$ es una sucesión de puntos de E y $x \in E$
    $$\xnx \\ \sii \\ [ \forall U \in \cc{U}(x)\\ \exists m \in \N : n\geq m \implies x_n \in U] $$
    \begin{prop}[Caracterización de la convergencia usando distancias] \\
        \begin{itemize}
            \item En cualquier espacio métrico:
            $\{x_n\} \longrightarrow x \sii [ \forall \varepsilon>0\\ \exists m \in \N : n\geq m \implies d(x_n, x) < \varepsilon]$
            \item En un espacio normado:
            $\xnx  \sii [ \forall \varepsilon>0\\ \exists m \in \N : n\geq m \implies ||x_n-x|| < \varepsilon]$
            \item En $\R$:
            $\{x_n\} \longrightarrow x \sii [ \forall \varepsilon>0\\ \exists m \in \N : n\geq m \implies |x_n-x| < \varepsilon]$
            \item $\{x_\} \longrightarrow x \sii \{d(x_n,x)\} \longrightarrow 0$
            \item $\xnx$, $\{x_n\} \longrightarrow y$ $\implies x = y $
        \end{itemize}
    \end{prop}
    
    Por tanto a ese $x$ al que tiende la sucesión, por ser único, lo llamamos límite.\newline

    \begin{prop}[Convergencia de parciales] \\
        \begin{enumerate}
            \item $\{x_n\} \longrightarrow x \implies \{x_{\sigma(n)}\} \longrightarrow x $
            \item $\{x_n\} \longrightarrow x \sii \{x_{k+n}\} \longrightarrow x $
            \item $\{x_n\} \longrightarrow x \sii \{x_{2n}\} \longrightarrow x \wedge \{x_{2n+1}\} \longrightarrow x $
        \end{enumerate} 
    \end{prop}

    \begin{prop}[Caracterización del cierre]
        $$x\in \overline{A} \\ \sii \\ \exists\{x_n\} : \{x_n\} \longrightarrow x$$
    \end{prop}

    \begin{prop}[Caracterización del un conjunto cerrado]
        $$A \in C_T \\ \sii \\ \forall x\in A \\ \exists\{x_n\} : \{x_n\} \longrightarrow x$$
    \end{prop}

    \begin{prop}[Criterio de equivalencia de dos distancias]
        Equivalen:\newline
        (1) La topología generada por $d_1$ está incluida en la generada por $d_2$ \newline
        (2) Toda sucesión convergente para $d_2$ lo es para $d_1$.\newline
    \end{prop}
    
    \begin{prop}[Convergencia en $\rn$]
        $$\{x_n\} \longrightarrow x \\ \sii \\ \{x_n(k)\} \longrightarrow x(k) \\ \forall k \in \Delta_N$$
    \end{prop}
    
    
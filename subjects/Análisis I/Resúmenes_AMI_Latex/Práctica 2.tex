\chapter{Práctica 2. Diferenciabilidad}

\section{Planteamiento del problema}
\noindent
Dado un abierto $\Omega \subset \rn$ y un campo escalar $f:\Omega \flecha \R$, buscamos estudiar la continuidad, diferenciabilidad y la continuidad de las derivadas parciales.

\section{Teoremas relacionados}
\noindent
Gracias a la teoría vista hasta el momento en los temas 6, 7 y 8, podemos recordar teoremas que nos van a ayudar a lo largo de esta práctica.

\begin{teo}[Carácter local de la continuidad]
    Sean $E$, $F$ espacios topológicos, $\emptyset \neq A \subseteq E$, $x \in A$.

    
    Si $f_{\big| A}$ es continua en $x$ con $A \in \mathbb{U}(x) \Rightarrow f$ es continua en $x$.
\end{teo}

\begin{prop}[Carácter local de la diferenciabilidad]
    Si $U\subset X, a\in Uº$. \newline Entonces: 
    $$f \\ diferenciable \\ a \\ \sii \\ f_{|U} \\ diferenciable\\ a$$
    $$Df(a) = D(f_{|U})(a)$$
\end{prop}

\begin{prop}[Relación de la diferenciabilidad con la continuidad]
    \ \newline
    Si f diferenciable en a $\then $ f continua en a 
\end{prop}

\begin{prop}[Condición suficiente de diferenciabilidad]
    \ \newline
    Sea $\Omega = \Omegaº \subseteq \rn$, $f:\Omega \flecha \R$, $a \in \Omega$, $k \in \Delta_N$.\newline
    Suponemos que $f$ es parcialmente derivable en $a$ y que al menos $N-1$ derivadas parciales son continuas en el punto $a$.\newline
    Entonces, $f$ es diferenciable en el punto $a$.
\end{prop}

\section{Parte rutinaria del problema}
\noindent
Será común encontrar un subconjunto $A$ de $\Omega$ (dominio de $f$) en el que $f_{|A}$ esté formada por operaciones de funciones de clase $C^1$. Nuestro interés será que dicho conjunto $A$ sea abierto, para poder aplicar el carácter local de la continuidad y de la diferenciabilidad.
\vspace{.5cm}

\noindent
Por tanto, buscamos un conjunto abierto $U \subset \Omega$ (el mayor que podamos) tal que $f_{|U}$ se obtiene mediante operaciones con funciones de clase $C^1$. Al ser $U$ abierto, por el carácter local de la continuidad y diferenciabilidad, sabremos que $f$ será continua y diferenciable en $U$.
\vspace{.5cm}

\noindent
Resumiendo, seguimos los siguientes pasos:
\begin{itemize}
    \item Definimos un conjunto $U \subset \Omega$ (el máximo que cumpla lo que queremos) y \textbf{comprobamos} que es abierto.
    \item Comprobamos que $f_{|U} \in C^1(U)$, que será fácil por cómo hayamos escogido $U$.
    \item Usamos los carácteres locales de la continuidad y diferenciabilidad, obteniendo que $f$ es continua y diferenciable en $U$.
\end{itemize}

\noindent
A partir de este momento, el problema de estudiar la continuidad y diferenciabilidad quedará reducido a estudiarla en todos los puntos del conjunto $\Omega \setminus U$.

\section{Cálculo de derivadas parciales}
\noindent
El segundo aspecto a tener en cuenta es la existencia de las derivadas parciales:
\vspace{.5cm}

\noindent
En cada punto $a \in \Omega \setminus U$, estudiaremos la existencia de las derivadas parciales de $f$ en $a$ y si estas existen, calcularlas. Se nos presentan dos posibilidades:

\begin{itemize}
    \item Si no existe alguna de las derivadas parciales de $f$ en $a$, entonces sabremos que $f$ no es diferenciable en $a$. Quedará estudiar la continuidad en $a$ y la continuidad de las derivadas parciales que sí existan en $a$ (\nameref{practica1}).
    \item Si $f$ es parcialmente derivable en $a$ (existen todas sus derivadas parciales), entonces podemos considerar el vector gradiente:
    $$\nabla f(a) = \left( \dfrac{\partial f}{\partial x_1}(a), \dfrac{\partial f}{\partial x_2}(a), \ldots, \dfrac{\partial f}{\partial x_N}(a) \right)$$
\end{itemize}

\section{Estrategias a seguir}
\noindent
Llegado a este punto, tenemos que $f$ es parcialmente derivable en $\Omega$. Fijado $a \in \Omega \setminus U$, podemos seguir tres estrategias:

\subsubsection{Opción optimista}
\noindent
Estudiamos primero la continuidad de las derivadas parciales de $f$ en $a$:

\begin{itemize}
    \item Si tenemos al menos $N-1$ (recordamos que trabajamos en $\Omega \subset \rn$) derivadas parciales continuas de $f$ en $a$, sabremos por la condición suficiente de diferenciabilidad que $f$ es diferenciable en $a$, luego también será continua en $a$. Faltará ver la continuidad de la parcial restante.
    \item Si tenemos que dos o más derivadas parciales de $f$ en $a$ no son continuas no podemos deducir nada más y por tanto, nos quedará estudiar la continuidad y diferenciabilidad de $f$ en $a$ y la continuidad de $N-2$ derivadas parciales de $f$ en $a$.
\end{itemize}

\subsubsection{Opción pesimista}
\noindent
Estudiamos primero la continuidad de la función $f$ en $a$:

\begin{itemize}
    \item Si $f$ no es continua en $a$, tampoco podrá ser diferenciable en $a$ y ninguna derivada parcial podrá ser continua en $a$.
    \item En caso de que $f$ sea continua en $a$, habremos finalizado el estudio de la continuidad de $f$ en $a$, pero nos quedará el estudio de la diferenciabilidad de $f$ en $a$ y de la continuidad de las derivadas parciales de $f$ en $a$.
\end{itemize}

\subsubsection{Opción conservadora}
\noindent
Estudiamos primero la diferenciabilidad de $f$ en $a$:

\begin{itemize}
    \item Si $f$ es diferenciable en $a$, también será continua en el punto $a$. Nos quedará estudiar la continuidad de las derivadas parciales.
    \item Si $f$ no es diferenciable en $a$, tendremos al menos 2 derivadas parciales que no podrán ser continuas. Quedará buscar cuáles son, estudiar la continuidad de las $N-2$ derivadas parciales restantes (si estamos en $\R^2$ será innecesario) y la continuidad de la función $f$ en el punto $a$.
\end{itemize}

\section{Estudio de la diferenciabilidad}
\noindent
Cuando nos dispongamos a estudiar la diferenciabilidad de $f$ en un punto $a\in \Omega \setminus U$, necesitamos comprobar previamente que $f$ es parcialmente derivable en $a$, ya que en caso de serlo, podremos usar el vector gradiente y, en caso de no serlo, descartaremos que $f$ sea diferenciable.
\vspace{.5cm}

\noindent
Por tanto, suponemos teóricamente (en la práctica ya se habrá hecho) que $f$ es parcialmente derivable en $a \in \Omega \setminus U$.
\vspace{.5cm}

$f$ será diferenciable en $a$ si y sólo si se tiene que:
$$\lim_{x \to a} \dfrac{f(x) - f(a) - \left(\nabla f(a)|(x-a)\right)}{\|x-a\|}=0$$
Por tanto, si definimos $\varphi(x) = \dfrac{f(x) - f(a) - \left(\nabla f(a)|(x-a)\right)}{\|x-a\|}~~\forall x \in \Omega \setminus {a}$, $f$ será diferenciable en $a$ si y sólo si:
$$\lim_{x \to a} \varphi(x) = 0$$

\begin{observacion}
\ \\
    \begin{itemize}
        \item La norma en la definición de $\varphi$ podemos elegirla a voluntad (todas las normas en $\rn$ son equivalentes). Suele ser más fácil elegir la norma euclídea en un caso general, aunque si la ocasión lo merece (para simplificar con el numerador) puede elegirse otra.
        \item A veces, estudiando el límite de $\varphi$ obtendremos un límite direccional distinto de 0. Esto ya nos vale para afirmar que $f$ no es diferenciable en $a$: pues de existir el límite, su valor debe coincidir con el del límite direccional, distinto de cero, luego $f$ no es diferenciable en $a$.
    \end{itemize}
\end{observacion}

\section{Ejemplos}
\noindent
Se pide estudiar la continuidad, diferenciabilidad y la continuidad de las derivadas parciales de las siguientes funciones:

\subsubsection{a)}
$$f(x,y) = \dfrac{x^2y}{x^2+y^4}~~\forall (x,y) \in \R^2 \setminus \{(0,0)\}~~~~f(0,0) = 0$$
Definimos $U = \R^2 \setminus \{(0,0)\}$, que es abierto por ser el complementario en $\R^2$ de un cerrado, ${(0,0)}$ (cerrado por ser un punto).
\vspace{.5cm}

\noindent
$f_{|U}$ es una función racional, por lo que $f_{|U} \in C^1(U)$. Por el carácter local de la continuidad y de la diferenciabilidad, tenemos que $f$ es continua y diferenciable en $U$.
\vspace{.5cm}

\noindent
A continuación, calculamos las derivadas parciales $\forall (x,y) \in U$:
$$\dfrac{\partial f}{\partial x}(x,y) = \dfrac{2xy^5}{(x^2+y^4)^2}$$
$$\dfrac{\partial f}{\partial y}(x,y) = \dfrac{x^4-3x^3y^4}{(x^2+y^4)^2}$$
Calculamos el valor de las derivadas parciales en $(0,0)$ y con ello, el vector gradiente. Observemos que:
$$f(x,0) = 0 = f(0,y)~~\forall(x,y) \in \R^2$$
Luego:
$$\dfrac{\partial f}{\partial x}(0,0) \mathop{=}^{def} \lim_{x \to 0} \dfrac{f(x,0)-f(0,0)}{x-0} = \lim_{x \to 0} \dfrac{0}{x} = 0$$
$$\dfrac{\partial f}{\partial y}(0,0) \mathop{=}^{def} \lim_{y \to 0} \dfrac{f(0,y)-f(0,0)}{y-0} = \lim_{y \to 0} \dfrac{0}{y} = 0$$
$$\nabla f(0,0) = (0,0)$$
Obtamos por la estrategia conservadora y nos disponemos a estudiar la diferenciabilidad de $f$ en $(0,0)$.
\vspace{.5cm}

Definimos $\varphi:U \flecha \R$ por: 
$$\varphi(x,y) = \dfrac{f(x,y)-f(0,0)-\left(\nabla f(0,0)|(x,y)-(0,0)\right)}{\|(x,y)-(0,0)\|} =\dfrac{f(x,y)}{\|(x,y)\|}=\dfrac{x^2y}{(x^2+y^4)\sqrt{x^2+y^2}}$$
Observemos que:
$$\varphi(x,x) = \dfrac{x^3}{\sqrt{2}(x^3+x^5)} = \dfrac{1}{\sqrt{2}(1+x^2)}$$
$$\lim_{x \to 0}\varphi(x,x) = \dfrac{1}{\sqrt{2}} \neq 0$$
Luego si $\varphi$ tiene límite en $(0,0)$, este sería $\dfrac{1}{\sqrt{2}} \neq 0$, luego $f$ no es diferenciable en $(0,0)$. Por tanto, sabemos que ni $\dfrac{\partial f}{\partial x}$ ni $\dfrac{\partial f}{\partial y}$ son continuas en $(0,0)$. Falta ver la continuidad de $f$ en $(0,0)$:

$$0 \leq |f(x,y)| = |y|\dfrac{x^2}{x^2+y^4} \leq |y|$$
Con $\lim\limits_{y \to 0}y = 0$, luego $\lim\limits_{(x,y) \to (0,0)} f(x,y) = 0 = f(0,0) \Rightarrow f$ es continua en $(0,0)$.

\subsubsection{b)}
$$g(x,y) = \dfrac{x^2y^2}{x^2+y^4}~~\forall (x,y) \in \R^2 \setminus \{(0,0)\}~~~~g(0,0) = 0$$
Definimos $U = \R^2 \setminus \{(0,0)\}$, que es abierto por ser el complementario en $\R^2$ de un cerrado, ${(0,0)}$ (cerrado por ser un punto).
\vspace{.5cm}

\noindent
$g_{|U}$ es una función racional, por lo que $g_{|U} \in C^1(U)$. Por el carácter local de la continuidad y de la diferenciabilidad, tenemos que $g$ es continua y diferenciable en $U$.
\vspace{.5cm}

\noindent
A continuación, calculamos las derivadas parciales $\forall (x,y) \in U$:
$$\dfrac{\partial g}{\partial x}(x,y) = \dfrac{2xy^6}{(x^2+y^4)^2}$$
$$\dfrac{\partial g}{\partial y}(x,y) = \dfrac{2x^2y(x^2-y^4)}{(x^2+y^4)^2}$$
Calculamos el valor de las derivadas parciales en $(0,0)$ y con ello, el vector gradiente. Observemos que:
$$g(x,0) = 0 = g(0,y)~~\forall(x,y) \in \R^2$$
Luego:
$$\dfrac{\partial g}{\partial x}(0,0) \mathop{=}^{def} \lim_{x \to 0} \dfrac{g(x,0)-g(0,0)}{x-0} = \lim_{x \to 0} \dfrac{0}{x} = 0$$
$$\dfrac{\partial g}{\partial y}(0,0) \mathop{=}^{def} \lim_{y \to 0} \dfrac{g(0,y)-g(0,0)}{y-0} = \lim_{y \to 0} \dfrac{0}{y} = 0$$
$$\nabla g(0,0) = (0,0)$$
\vspace{.5cm}

\noindent
Observamos la segunda derivada parcial de $g$ y tenemos que (estrategia optimista):
$$0 \leq \left| \dfrac{\partial g}{\partial y}(x,y) \right| = 2|y| \dfrac{x^2}{(x^2+y^4)}
\dfrac{(x^2-y^4)}{(x^2+y^4)} \leq 2|y|$$
Luego:
$$\lim_{(x,y) \to (0,0)} \dfrac{\partial g}{\partial y}(x,y) = 0 = \dfrac{\partial g}{\partial y}(0,0) \Rightarrow \dfrac{\partial g}{\partial y} \mbox{ es continua en } (0,0)$$
Luego por la condición suficiente de diferenciabilidad, tenemos que $f$ es diferenciable en $(0,0)$, luego también será continua en $(0,0)$. Falta ver la continuidad de la parcial restante:
$$\dfrac{\partial g}{\partial x}(x,y) = \dfrac{2xy^6}{(x^2+y^4)^2}$$
$$\lim_{x \to 0}\dfrac{\partial g}{\partial x}(x,0) = 0$$
$$\lim_{x \to 0}\dfrac{\partial g}{\partial x}(x^2, x) = \dfrac{2x^8}{4x^8} = \dfrac{1}{4} \neq 0$$
Luego $\nexists \lim\limits_{(x,y) \to (0,0)}\dfrac{\partial g}{\partial x}(x,y)$, por lo que $\dfrac{\partial g}{\partial x}(x,y)$ no es continua en $(0,0)$.
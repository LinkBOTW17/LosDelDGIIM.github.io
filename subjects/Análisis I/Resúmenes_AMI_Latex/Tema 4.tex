\chapter{Compacidad y conexión}
\section{Acotación}

\begin{definicion}[Conjunto acotado]
    $E$ espacio métrico. $A \subset E$.
    $A$ está acotado cuando está incluido en una bola.
    $$A\\ acotado \implies \forall x \in E \\ \exists r > 0 : A \subset B(x,r)$$
\end{definicion}

\begin{ejemplo}
    Todo subconjunto de $E$ está acotado.
\end{ejemplo}

\begin{ejemplo}
    Toda sucesión convergente está acotada.
\end{ejemplo}

\noindent
La acotación no es una propiedad topológica. 
$$\rho = \frac{d(x,y)}{1+d(x,y)}$$
La distancia $d$ y la $\rho$ son equivalentes, por lo que dan la misma topología; pero dan conjuntos acotados distintos.

\begin{prop}
    $X$ espacio normado, $A \subset X$:
    $$A \\ acotado \sii \exists M > 0 : ||x|| \leq M \\ \forall x \in A$$
    Dos normas equivalentes dan lugar a la misma topología.
\end{prop}

\begin{prop}
    $X = X_1 \times ... \times X_N$, $A \subset X$
    $$A \\ acotado \\\sii \\\{x(k): x \in A\} \\ acotado \\ \forall k \in \Delta_N$$
\end{prop}

\begin{teo}[Bolzano-Weierstrass]
    Toda sucesión acotada de vectores de $\rn$ admite una sucesión parcial convergente.
\end{teo}

\section{Compacidad}

\begin{definicion}[Espacio métrico compacto]
    $E$ es compacto cuando toda sucesión de puntos de $E$ admite una sucesión parcial convergente.
\end{definicion}

\begin{prop}
    $E$ espacio métrico, $A\subset E$
    $$A \\ compacto \implies A \\ acotado \\ y \\ \overline{A} = A$$
\end{prop}

\begin{prop}
    $$A \subset \rn \\ compacto \sii A \\ acotado \\ y \\ \overline{A} = A$$
\end{prop}

\begin{teo}[Weierstrass]
    $E,F$ espacios métricos, $\fEF$ continua.
    $$E \\ compacto \implies f(E) \\ compacto$$
\end{teo}

\begin{prop}
    $E$ espacio métrico compacto, $f:E \flecha \R$ continua. Entonces:
    $$\exists u,v \in E: f(u) \leq f(x) \leq f(v) \\ \forall x\in E$$
\end{prop}

\begin{teo}[Hausdorff]\\
    \begin{itemize}
        \item[(i)] Todas las normas en $\rn$ son equivalentes.
        \item[(ii)] Todas las normas en un espacio de dimensión finita son equivalentes.
    \end{itemize}
\end{teo}

\section{Conexión}

\begin{definicion}[Espacio métrico conexo]
    $E$ es conexo cuando no se puede expresar como unión de dos abiertos no vacios disjuntos.
    $$E=U\cup V \\ U = U^o \\ V=V^o \\ U\cap V = \emptyset \\\implies\\ U=\emptyset \\\lor \\ V=\emptyset$$
\end{definicion}

\begin{prop}[Caracterización]
    Sea $E$ espacio métrico. Equivalen:
    \begin{itemize}
        \item[(i)] $E$ es conexo.
        \item[(ii)] $f:E \flecha \R$ continua $\implies$ $f(E)$ intervalo
        \item[(iii)] $f:E \flecha \{0,1\}$ continua $\implies$ $f$ constante
    \end{itemize}
\end{prop}

\begin{prop}
    $$A\subset \R \\ conexo \\\sii\\ A \\ intervalo$$
\end{prop}

\begin{teo}[del valor intermedio]
    $E,F$ espacios métricos, $\fEF$ continua.
    $$E \\ conexo \implies f(E) \\ conexo$$
\end{teo}

\begin{coro}
    $E$ espacio métrico compacto y conexo, $f:E\flecha\R$ continua $\implies$ $f(E)$ es un intervalo cerrado y acotado.
\end{coro}

\begin{prop}
    $E$ es conexo $\sii$ $\forall x,y \in E\\ \exists C \subset E$ conexo : $x,y \in C$ 
\end{prop}

\begin{definicion}[Conjuntos convexos]
    Sea $E\subset X$, $X$ espacio vectorial. $E$ es convexo cuando: 
    $$x,y \in E \implies (1-t)x+ty \in E \\ \forall t \in [0,1]$$
\end{definicion}

\begin{ejemplo}
    Todo subconjunto convexo de un espacio normado es conexo.
\end{ejemplo}

\begin{ejemplo}
    Las bolas de un espacio normado son convexas, luego conexas.
\end{ejemplo}

\begin{ejemplo}
    $C,D \subset E$ conexos, $C\cap D \neq \emptyset \implies C\cup D$ conexo
\end{ejemplo}
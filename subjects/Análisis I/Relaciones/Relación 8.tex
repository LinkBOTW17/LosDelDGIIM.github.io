\section{Vector Gradiente}

\begin{ejercicio}
    Calcular todas las derivadas direccionales en el punto $(-1,0,0)$ de la función $f:\bb{R}^3\to \bb{R}$ definida por
    \begin{equation*}
        f(x,y,z) = x^3  -3xy + z^3 \qquad \forall (x,y,z)\in \bb{R}^3
    \end{equation*}

    Sea $u=(u_1,u_2,u_3)\in \bb{R}^3$ la dirección. Tenemos que:
    \begin{equation*}\begin{split}
        f'_u(-1,0,0) &= \lim_{t\to 0} \frac{f[(-1,0,0) + tu] - f(-1,0,0)}{t}
        = \lim_{t\to 0} \frac{f(-1+tu_1, tu_2, tu_3) +1}{t}
        =\\&= \lim_{t\to 0} \frac{(-1+tu_1)^3-3tu_2(-1+tu_1) + t^2u_3^3 +1}{t}
        =\\&= \lim_{t\to 0} \frac{\cancel{-1}+3tu_1 - 3t^2u_1^2 +t^3u_1^3 +3tu_2 -3t^2u_2u_1 + t^2u_3^3 +\cancel{1}}{t}
        =\\&= \lim_{t\to 0} 3u_1 - 3tu_1^2 +t^2u_1^3 +3u_2 -3tu_2u_1 + tu_3^3
        =\\&= 3(u_1 + u_2)
    \end{split}\end{equation*}

    Por tanto, dada cualquier dirección $u$ se tiene su derivada direccional en dicho punto.
\end{ejercicio}

\begin{ejercicio}
    Fijado $p\in \bb{R}^\ast$, se considera la función $f:\bb{R}^N\setminus \{0\}\to \bb{R}$ definida por $f(x) = \|x\|^p$ para todo $x\in \bb{R}^N\setminus \{0\}$, donde $\|\cdot\|$ es la norma euclídea. Probar que $f\in C^1(\bb{R}^n\setminus \{0\})$, con
    \begin{equation*}
        \nabla f(x) = p\|x\|^{p-2}x \qquad \forall x\in \bb{R}^N\setminus \{0\}
    \end{equation*}

    Como consecuencia, encontrar una función $g\in C^1(\bb{R}^n)$ que verifique 
    \begin{equation*}
        \nabla g(x)=x \qquad \forall x\in \bb{R}^n
    \end{equation*}

    Desarrollamos la expresión de la norma euclídea:
    \begin{equation*}
        f(x) = \|x\|^p = \left(\sqrt{\sum_{i=1}^n x_i^2}\right)^p \qquad \forall x\in \bb{R}^N\setminus \{(0,0)\}
    \end{equation*}

    Calculemos las derivadas parciales de $f$:
    \begin{equation*}
        \del{f}{x_i}(x) = p\|x\|^{p-1}\cdot \frac{1}{2\|x\|}\cdot 2x_i = p\|x\|^{p-2}x_i \qquad \forall x\in \bb{R}^N\setminus \{(0,0)\}
    \end{equation*}

    Por tanto, $\qquad \forall x\in \bb{R}^N\setminus \{(0,0)\}$ tenemos que:
    \begin{equation*}
        \nabla f(x) = \left(p\|x\|^{p-2}x_1,\dots, p\|x\|^{p-2}x_i, \dots, p\|x\|^{p-2}x_n\right) = p\|x\|^{p-2} \cdot x
    \end{equation*}

    Además, como $\nabla f$ es una función claramente continua, $f\in C^1(\bb{R}^n\setminus \{0\})$.

    La función $g$ pedida vemos que es la siguiente:
    \begin{equation*}
        g(x) = \frac{\|x\|^2}{2} = \frac{1}{2} \sum_{i=1}^n x_i^2 \qquad \forall x\in \bb{R}^n
    \end{equation*}
\end{ejercicio}

\begin{ejercicio}
    Sea $J$ un intervalo abierto en $\bb{R}$ y $\Omega$ un subconjunto abierto de $\bb{R}^n$. Si $f:J\to \Omega$ es una función derivable en un punto $a\in J$ y $g:\Omega\to \bb{R}$ es diferenciable en el punto $b=f(a)$, probar que la función $h:g\circ f:J\to \bb{R}$ es derivable en punto $a$, con:
    \begin{equation*}
        h'(a) = (\nabla g(b)\mid f'(a))
    \end{equation*}

    Por la regla de la Cadena, sabemos que $Dh(a) = Dg(b)\circ Df(a)$.
    Como $h$ es una función cuyo dominio es $\bb{R}$, tenemos que $h'(a)=Dh(a)(1)$. Análogamente, tenemos que $Df(a)(1) = f'(a)$. Por tanto,
    \begin{equation*}
        h'(a) = Dh(a)(1) = (Dg(b)\circ Df(a))(1) = Dg(b)(Df(a)(1)) = Dg(b)(f'(a))
    \end{equation*}

    Como $g$ es diferenciable en $b$, tenemos que:
    \begin{equation*}
        Dg(b)(f'(a)) = \left(\nabla g(b)\mid f'(a)\right)
    \end{equation*}
    como se pedía demostrar.
\end{ejercicio}

\begin{ejercicio}
    Calcular la ecuación del plano tangente a la superficie explícita de ecuación $z = x + y^3$ con $(x,y)\in\bb{R}^2$, en el punto $(1,1,2)$.

    Sea $f:\bb{R}^2\to \bb{R}$ dada por $f(x,y) = x+y^3$. Tenemos que $f\in C^1(\bb{R}^2)$; y las derivadas parciales de $f$ son:
    \begin{equation*}
        \del{f}{x}(x,y) = 1
        \qquad 
        \del{f}{y}(x,y) = 3y^2
    \end{equation*}
    Por tanto, $\nabla f(1,1)=(1,3)$. Por tanto, la ecuación del plano tangente a dicha superficie en en punto $(1,1,2)$ es:
    \begin{equation*}
        \pi\equiv z-2 = 1(x-1) + 3(y-1) \Longrightarrow \pi\equiv x+3y-z=2
    \end{equation*}
\end{ejercicio}

\begin{ejercicio}
    Sea $\Omega$ un subconjunto abierto y conexo de $\bb{R}^2$ y $f:\Omega\to \bb{R}$ una función diferenciable. Se considera la superficie explícita $S\subset \bb{R}^3$ dada por
    \begin{equation*}
        S=\{(x,f(x,z), z)\mid (x,z)\in \Omega\}
    \end{equation*}
    Calcular la ecuación del plano tangente a $S$ en un punto arbitrario $(x_0,y_0,z_0)\in S$.\\

    Usamos la siguiente notación:
    \begin{equation*}
        y_0=f(x_0,z_0) \qquad \alpha_0 = \del{f}{x}(x_0, z_0) \qquad \beta_0 = \del{f}{z}(x_0, z_0)
    \end{equation*}

    Veamos ahora que dicho plano es el siguiente:
    \begin{equation*}
        \Pi \equiv y-y_0 = \alpha_0(x-x_0) + \beta_0(z-z_0)
    \end{equation*}

    Tenemos que se trata de una superficie explícita, donde $g:\bb{R}^2\to \bb{R}$ es la función definida por:
    \begin{multline*}
        g(x,z) = f(x_0,z_0) + \alpha_0(x-x_0) + \beta_0(z-z_0) = f(x_0,z_0) + (\nabla f(x_0,z_0)\mid [(x,z) - (x_0,z_0)]) \AstIg\\
        \AstIg f(x_0,z_0) + Df(x_0,y_0)[(x,z) - (x_0,z_0)])
    \end{multline*}
    donde en $(\ast)$ he aplicad la relación de la diferencial con su gradiente como producto escalar. Además, por el significado analítico de la diferencial, tenemos que:
    \begin{equation*}
        \lim_{(x,z)\to (x_0,z_0)} \frac{f(x,z) - g(x,z)}{\|(x,z)-(x_0,z_0)\|} = 0
    \end{equation*}
    Es decir, $g$ es una buena aproximación de $f$ cerca del punto buscado. Por tanto, el plano tangente es $\Pi$ mencionado.
\end{ejercicio}


\begin{ejercicio}
    Sea $\Omega$ un subconjunto abierto de $\bb{R}^2$ y $f:\Omega\to \bb{R}^2$ una función parcialmente derivable en todo punto de $\Omega$. Probar que, si la función $\nabla f :\Omega\to \bb{R}^2$ está acotada, entonces $f$ es continua. Usando la función $f:\bb{R}^2\to \bb{R}$ definida por:
    \begin{equation*}
        f(x,y) = \frac{xy}{\sqrt{x^2+y^2}}~~\forall (x,y)\in \bb{R}^2\setminus \{(0,0)\}, \quad f(0,0)=0
    \end{equation*}
    comprobar que, con las mismas hipótesis, no se puede asegurar que $f$ sea diferenciable.\\
\end{ejercicio}
\section{Continuidad y límite funcional}

\begin{ejercicio}
    Sean $E$ y $F$ espacios métricos y $f:E\to F$ una función. Probar que $f$ es continua si, y sólo si, $f\left(\ol{A}\right)\subset \ol{f(A)}$ para todo conjunto $A \subset E$.
\end{ejercicio}

\begin{ejercicio}
    Dado un subconjunto $A$ de un espacio métrico $E$, la función característica de $A$ es la función $\chi_A:E\to \bb{R}$ definida por:
    \begin{equation*}
        \chi_A(x)=1~~\forall x\in A
        \hspace{1cm} \text{y} \hspace{1cm}
        \chi_A(x)=0~~\forall x\in E\setminus A
    \end{equation*}
Probar que $\chi_A$ es continua en un punto $x\in E$ si, y sólo si, $x \in A^\circ \cup (E\setminus A)^\circ$. Deducir que $\chi_A$ es continua si, y sólo si, $A$ es a la vez abierto y cerrado.
\end{ejercicio}

\begin{ejercicio}
    Si $E$ y $F$ son espacios métricos, se dice que una función $f:E\to F$ es \emph{localmente constante} cuando, para cada $x\in E$, existe $U \in \cc{U}(x)$ tal que $f\big|_{U}$ es constante. Probar que entonces $f$ es continua. Dar un ejemplo de un conjunto $A \subset \bb{R}$ y una función localmente
    constante $f : A\to \bb{R}$, cuya imagen $f(A)$ sea un conjunto infinito.
\end{ejercicio}

\begin{ejercicio}
    Sea $E$ un espacio métrico con distancia $d$ y $A$ un subconjunto no vacío de $E$. Probar la continuidad de la función $f : E \to \bb{R}$ dada por
    \begin{equation*}
        f(x)=d(x,A)\stackrel{\text{def}}{=}\inf\{d(x,a)\mid a \in A\} \qquad \forall x\in E
    \end{equation*}
\end{ejercicio}

\begin{ejercicio}
    Sea $E$ un espacio métrico con distancia $d$, y consideremos el espacio producto $E \times E$.
    Probar que, $\forall r \in \bb{R}^+_0$, el conjunto $\{(x,y) \in E\times E \mid d(x,y) < r\}$ es abierto, mientras que $\{(x,y) \in E\times E \mid d(x,y) \leq r\}$ es cerrado.
    En particular se tiene que la diagonal $\Delta(E)=\{(x,x)\mid x\in E\}$ es un conjunto cerrado. Deducir que, si $F$ es otro espacio métrico y $f,g: E \to F$ son funciones continuas, entonces $\{x\in E\mid f(x)=g(x)\}$ es un subconjunto cerrado de $E$ .
\end{ejercicio}

\begin{ejercicio}
    Sean $E$, $F$ espacios métricos y $f:E\to F$ una función continua. Probar que su gráfica, es decir, el conjunto $\operatorname{Gr} f = \{(x,f(x))\mid x\in E\}$ es un subconjunto cerrado del espacio métrico producto $E \times F$.
\end{ejercicio}


\begin{ejercicio}
    Sea $E$ un espacio métrico e $Y$ un espacio pre-hilbertiano. Para $f,g \in \cc{F}(E,Y)$, se define una función $h \in \cc{F}(E)$ por $h(x)=\left(f(x)\mid g(x)\right)$ para todo $x\in E$ . Probar que, si $f,g$ son continuas en un punto $a \in E$ , entonces $h$ también lo es.
\end{ejercicio}


\begin{ejercicio}
    Sea $E$ un espacio métrico y $f,g : E \to \bb{R}$ funciones continuas en un punto $a\in E$ . Probar que la función $h : E \to \bb{R}$ definida por $h(x) = \max \{f(x), g(x)\}$ para todo $x\in E$, también es continua en $a$.\\

    La clave es expresar $h$ de la siguiente manera:
    $$h(x)=\max\{f(x),g(x)\}=\frac{1}{2}\left(f(x)+g(x)+|f(x)-g(x)|\right) \qquad \forall x\in E$$
    Como $f,g$ son continuas en $a$, entonces $f+g$ y $f-g$ también lo son. Además, el valor absoluto es una función continua en $\bb{R}$,
    por lo que $|f-g|$ es continua en $a$. Por tanto, $h$ es continua en $a$.
\end{ejercicio}


\begin{ejercicio}
    Probar que si $Y$ es un espacio normado, $E$ un espacio métrico y $f:E \to Y$ una función continua en un punto $a \in E$ , entonces la función $g: E \to \bb{R}$ definida por $g(x)=\|f(x)\|$ para todo $x\in E$ , también es continua en el punto $a$.
\end{ejercicio}

\begin{ejercicio}
    Probar las siguientes igualdades:
    \begin{enumerate}
        \item $\displaystyle \lim_{(x,y)\to (0,0)} \frac{\sen(x^2 + y^2)}{x^2+y^2} = 1$.
        \item $\displaystyle \lim_{(x,y)\to (0,0)} \frac{\log(1+x^4 + y^4)}{x^4+y^4} = 1$.
    \end{enumerate}
\end{ejercicio}
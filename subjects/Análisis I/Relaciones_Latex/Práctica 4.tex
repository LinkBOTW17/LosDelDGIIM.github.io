\section{Funciones Implícitas}

\begin{ejercicio}
    Probar que el sistema de ecuaciones
    \begin{equation*}
        \left\{
            \begin{array}{ll}
                e^{u+x} \cos(y+v) - x^2+y^2 &= 0 \\
                e^{u+x} \sen(y+v) - 2xy &= 0
            \end{array}
        \right.
    \end{equation*}
    define funciones implícitas $u = u(x, y)$ y $v = v(x, y)$, diferenciables en un entorno
    del punto $(1, 0)$ , con $u(1, 0)=-1$ y $v(1, 0) = 0$. Calcular las derivadas parciales
    de $u$ y $v$ en dicho punto.\\

    Sea $\Omega = \bb{R}^4\in \mathcal{T}_u$ y $F=(F_1,F_2):\Omega \to \bb{R}^2$ definida por:
    \begin{align*}
        F_1(x,y,u,v) &= e^{u+x} \cos(y+v) - x^2+y^2 \\
        F_2(x,y,u,v) &= e^{u+x} \sen(y+v) - 2xy
    \end{align*}

    Tenemos que ambas componentes de $F$ son de clase $C^1(\bb{R}^4)$,
    ya que son suma, producto y composición de funciones de clase $C^1$: exponencial, el seno, el coseno, y funciones polinómicas.
    Por tanto, tenemos que $F\in C^1(\bb{R}^4,\bb{R}^2)$.

    Evidentemente, $(1,0,-1,0)\in \Omega$ y $F(1,0,-1,0)=(0,0)$. Falta por comprobar que:
    \begin{equation*}
        \det \left( \del{(F_1,F_2)}{(u,v)}(1,0,-1,0) \right) \neq 0
    \end{equation*}

    Para ello, buscamos la matriz jacobiana de $F$ en el punto $(1,0,-1,0)$. Para ello, dado un punto
    genérico $P=(x,y,u,v)\in \Omega$, tenemos que:
    \begin{equation*}
        \begin{array}{ll}
            \deld{F_1}{x}(P) = e^{u+x} \cos(y+v) - 2x & \deld{F_1}{y}(P) = -e^{u+x} \sen(y+v) + 2y \\ \\
            \deld{F_1}{u}(P) = e^{u+x} \cos(y+v) & \deld{F_1}{v}(P) = -e^{u+x} \sen(y+v) \\ \\
            \deld{F_2}{x}(P) = e^{u+x} \sen(y+v) - 2y & \deld{F_2}{y}(P) = e^{u+x} \cos(y+v) - 2x \\ \\
            \deld{F_2}{u}(P) = e^{u+x} \sen(y+v) & \deld{F_2}{v}(P) = e^{u+x} \cos(y+v)
        \end{array}
    \end{equation*}

    En particular, en el punto $P_0=(1,0,-1,0)$ tenemos que:
    \begin{equation*}
        \begin{array}{llll}
            \deld{F_1}{x}(P_0) = -1,& \deld{F_1}{y}(P_0) = 0,& \deld{F_1}{u}(P_0) = 1,& \deld{F_1}{v}(P_0) = 0 \\ \\
            \deld{F_2}{x}(P_0) = 0,& \deld{F_2}{y}(P_0) = -1,& \deld{F_2}{u}(P_0) = 0,& \deld{F_2}{v}(P_0) = 1
        \end{array}
    \end{equation*}

    Por tanto, la matriz jacobiana de $F$ en el punto $P_0$ es:
    \begin{equation*}
        JF(P_0) =
        \begin{pmatrix}
            -1 & 0 & 1 & 0 \\
            0 & -1 & 0 & 1
        \end{pmatrix}
    \end{equation*}

    Por tanto, el determinante que nos interesa es:
    \begin{equation*}
        \det \left( \del{(F_1,F_2)}{(u,v)}(P_0) \right) =
        \left|
            \begin{array}{cc}
                1 & 0 \\
                0 & 1
            \end{array}
        \right| = 1 \neq 0
    \end{equation*}

    Por el teorema de la función implícita, tenemos que existe un abierto $U\subset \bb{R}^2$,
    con $(1,0)\in U$, de manera que $F$ define funciones implícitas $u,v:U\to \bb{R}$, diferenciables en $U$,
    con $u(1,0)=-1$ y $v(1,0)=0$, tales que:
    \begin{equation}\label{eq:funciones_implicitas}
        \left\{
            \begin{array}{ll}
                F_1(x,y,u(x,y),v(x,y)) &= 0 \\
                F_2(x,y,u(x,y),v(x,y)) &= 0
            \end{array}
        \right. \hspace{1cm} \forall (x,y)\in U
    \end{equation}

    Nos falta ahora por calcular las derivadas parciales de $u$ y $v$ en el punto $(1,0)$.
    En la Ecuación \ref{eq:funciones_implicitas}, tenemos dos funciones de dos variables idénticamente nulas,
    por lo que sus derivadas parciales han de ser idénticamente nulas en todo punto de $U$.
    Entonces, si abreviamos entendiendo que todas las funciones se evalúan en
    un punto genérico $(x,y)\in U$, tenemos que las derivadas parciales respecto de $x$ de la Ecuación \ref{eq:funciones_implicitas} son:
    \begin{equation*}
        \begin{array}{ll}
            e^{u+x}\left(1+\deld{u}{x}\right)\cos(y+v) - e^{u+x}\sen(y+v)\deld{v}{x} - 2x &= 0 \\ \\
            e^{u+x}\left(1+\deld{u}{x}\right)\sen(y+v) + e^{u+x}\cos(y+v)\deld{v}{x} - 2y &= 0
        \end{array}
    \end{equation*}
    Evaluando en el punto $(1,0)$, como $u(1,0)=-1$ y $v(1,0)=0$, tenemos que:
    \begin{equation*}
        \begin{array}{ll}
            e^{-1+1}\left(1+\deld{u}{x}(1,0)\right)\cos(0) - e^{-1+1}\sen(0)\deld{v}{x}(1,0) - 2 &= 0 \\
            e^{-1+1}\left(1+\deld{u}{x}(1,0)\right)\sen(0) + e^{-1+1}\cos(0)\deld{v}{x}(1,0) - 0 &= 0 \\
        \end{array}
    \end{equation*}
    Por tanto, tenemos que:
    \begin{equation*}
        \deld{u}{x}(1,0) = 1 \hspace{2cm}
        \deld{v}{x}(1,0) = 0
    \end{equation*}

    De manera análoga, las derivadas parciales respecto de $y$ de la Ecuación \ref{eq:funciones_implicitas} son:
    \begin{equation*}
        \begin{array}{ll}
            e^{u+x}\deld{u}{y}\cos(y+v) - e^{u+x}\sen(y+v)\left(1+\deld{v}{y}\right) + 2y &= 0 \\ \\
            e^{u+x}\deld{u}{y}\sen(y+v) + e^{u+x}\cos(y+v)\left(1+\deld{v}{y}\right) - 2x &= 0
        \end{array}
    \end{equation*}
    Evaluando en el punto $(1,0)$, como $u(1,0)=-1$ y $v(1,0)=0$, tenemos que:
    \begin{equation*}
        \begin{array}{ll}
            e^{-1+1}\deld{u}{y}(1,0)\cos(0) - e^{-1+1}\sen(0)\left(1+\deld{v}{y}(1,0)\right) + 0 &= 0 \\ \\
            e^{-1+1}\deld{u}{y}(1,0)\sen(0) + e^{-1+1}\cos(0)\left(1+\deld{v}{y}(1,0)\right) - 2 &= 0 \\
        \end{array}
    \end{equation*}

    Por tanto, tenemos que:
    \begin{equation*}
        \deld{u}{y}(1,0) = 0 \hspace{2cm}
        \deld{v}{y}(1,0) = 1
    \end{equation*}

    En resumen, las derivadas parciales de $u$ y $v$ en el punto $(1,0)$ son:
    \begin{equation*}
        \deld{u}{x}(1,0) = 1 \hspace{1cm}
        \deld{u}{y}(1,0) = 0 \hspace{1cm}
        \deld{v}{x}(1,0) = 0 \hspace{1cm}
        \deld{v}{y}(1,0) = 1
    \end{equation*}
\end{ejercicio}




\begin{ejercicio}
    Probar que la ecuación
    \begin{equation*}
        xyz + \ln(z-5) - 2x - 2y - 2x^2y^2 = 0
    \end{equation*}
    define una función implícita $z = z(x, y)$, diferenciable en un entorno del punto $(1, 1)$,
    con $z(1, 1) = 6$. Calcular $\nabla z(1, 1)$.\\

    Sea $\Omega = \{(x,y,z)\in \bb{R}^3 : z>5\}$. Tenemos que $\Omega$ es la imagen inversa de $]5,+\infty[$ por la proyección en la tercera componente,
    que es continua. Por tanto, $\Omega$ es abierto. Definimos la función $F:\Omega \to \bb{R}$ como:
    \begin{equation*}
        F(x,y,z) = xyz + \ln(z-5) - 2x - 2y - 2x^2y^2
    \end{equation*}

    Claramente, se verifica que $P_0 = (1,1,6)\in \Omega$ y $F(1,1,6)=0$. Tenemos que la función $g:\Omega \to \bb{R}$ definida como $g(x,y,z)=\ln(z-5)$
    es de clase $C^1(\Omega)$, ya que es composición de una función polinómica ($C^1(\bb{R}^3)$)
    que, en $\Omega$, toma valores en $\bb{R}^+$, con la función logaritmo ($C^1(\bb{R}^+)$).
    Por tanto, como $F$ es suma $g$ con funciones polinómicas ($C^1(\Omega)$), tenemos que $F\in C^1(\Omega)$. Nos falta comprobar que
    $\det \left( \deld{F}{z}(P_0) \right) \neq 0$. Tenemos que:
    \begin{equation*}
        \deld{F}{z}(x,y,z) = xy + \frac{1}{z-5} \Longrightarrow
        \deld{F}{z}(P_0) = 1 + \frac{1}{6-5} = 2 \neq 0
    \end{equation*}

    Por tanto, por el teorema de la función implícita,
    tenemos que existe un abierto $U\subset \bb{R}^2$, con $(1,1)\in U$, de manera que $F$
    define una función implícita $z:U\to \bb{R}$, diferenciable en $U$, con $z(1,1)=6$, tal que:
    \begin{equation*}
        F(x,y,z(x,y)) = 0 \hspace{1cm} \forall (x,y)\in U
    \end{equation*}

    Nos falta por calcular $\nabla z(1,1)$. Para ello, en $U$, tenemos que $F(x,y,z(x,y))=0$,
    por lo que sus derivadas parciales respecto de $x$ e $y$ han de ser idénticamente nulas.
    Por tanto, evaluando todas las funciones en un punto genérico $(x,y)\in U$, tenemos que:
    \begin{align*}
        yz + xy\del{z}{x} + \frac{1}{z-5}\del{z}{x} - 2 - 4xy^2 &= 0 \\
        xz + xy\del{z}{y} + \frac{1}{z-5}\del{z}{y} - 2 - 4x^2y &= 0
    \end{align*}

    Evaluando en el punto $(1,1)$, como $z(1,1)=6$, tenemos que:
    \begin{align*}
        6 + \del{z}{x}(1,1) + \frac{1}{6-5}\del{z}{x}(1,1) - 2 - 4 &= 0 \\
        6 + \del{z}{y}(1,1) + \frac{1}{6-5}\del{z}{y}(1,1) - 2 - 4 &= 0
    \end{align*}

    Por tanto, tenemos que:
    \begin{equation*}
        \del{z}{x}(1,1) = 0 = \del{z}{y}(1,1)
    \end{equation*}

    Por tanto, $\nabla z(1,1) = (0,0)$.
\end{ejercicio}


\begin{ejercicio}
    Probar que el sistema de ecuaciones
    \begin{equation*}
        \left\{
            \begin{array}{ll}
                zx^3 + w^2y^3 &= 1 \\
                2zw^3 + xy^2 &= 0
            \end{array}
        \right.
    \end{equation*}
    define funciones implícitas $z = z(x, y)$ y $w = w(x, y)$, diferenciables en un entorno
    del punto $(0,1)$ , con $z(0,1)=0$ y $w(0,1) = 1$. Probar también que
    la función $(x,y) \mapsto (z(x,y), w(x,y))$ es inyectiva en un entorno de $(0,1)$.\\

    Sea $\Omega = \bb{R}^4\in \mathcal{T}_u$ y $F=(F_1,F_2):\Omega \to \bb{R}^2$ definida por:
    \begin{align*}
        F_1(x,y,z,w) &= zx^3 + w^2y^3 - 1 \\
        F_2(x,y,z,w) &= 2zw^3 + xy^2
    \end{align*}

    Tenemos que ambas componentes de $F$ son de clase $C^1(\bb{R}^4)$ por ser polinómicas,
    por lo que $F\in C^1(\bb{R}^4,\bb{R}^2)$. Además, trivialmente se tiene que $P_0=(0,1,0,1)\in \Omega$ y $F(0,1,0,1)=(0,0)$.
    Nos falta por comprobar que:
    \begin{equation*}
        \det \left( \del{(F_1,F_2)}{(z,w)}(0,1,0,1) \right) \neq 0
    \end{equation*}

    Para ello, buscamos la matriz jacobiana de $F$ en el punto $(0,1,0,1)$. Para ello, dado un punto genérico $P=(x,y,z,w)\in \Omega$, tenemos que:
    \begin{equation*}
        \begin{array}{llll}
            \deld{F_1}{x}(P) = 3zx^2 & \deld{F_1}{y}(P) = 3w^2y^2 & \deld{F_1}{z}(P) = x^3 & \deld{F_1}{w}(P) = 2wy^3 \\ \\
            \deld{F_2}{x}(P) = y^2 & \deld{F_2}{y}(P) = 2xy & \deld{F_2}{z}(P) = 2w^3 & \deld{F_2}{w}(P) = 6zw^2
        \end{array}
    \end{equation*}

    Por tanto, en el punto $P_0=(0,1,0,1)$ tenemos que:
    \begin{equation*}
        JF(P_0) = \left(
            \begin{array}{cccc}
                0 & 3 & 0 & 2 \\
                1 & 0 & 2 & 0
            \end{array}
        \right)
    \end{equation*}

    Por tanto, el determinante que nos interesa es:
    \begin{equation*}
        \det \left( \del{(F_1,F_2)}{(z,w)}(P_0) \right) =
        \left|
            \begin{array}{cc}
                0 & 2 \\
                2 & 0
            \end{array}
        \right| = -4 \neq 0
    \end{equation*}

    Por el teorema de la función implícita, tenemos que existe un abierto $U\subset \bb{R}^2$,
    con $(0,1)\in U$, de manera que $F$ define funciones implícitas $z,w:U\to \bb{R}$, diferenciables en $U$,
    con $z(0,1)=0$ y $w(0,1)=1$, tales que:
    \begin{equation}
        \left\{
            \begin{array}{ll}
                F_1(x,y,z(x,y),w(x,y)) &= 0 \\
                F_2(x,y,z(x,y),w(x,y)) &= 0
            \end{array}
        \right. \hspace{1cm} \forall (x,y)\in U
    \end{equation}

    Nos falta ahora por probar que la función $h:U\to \bb{R}^2$ definida como $h(x,y)=(z(x,y),w(x,y))$ es inyectiva
    en un entorno de $(0,1)$. Para ello, buscamos aplicar
    el teorema de la función inversa local en el punto $(1,0)$.

    En primer lugar, calculamos $Jh(0,1)$. Para ello, calculamos las derivadas parciales de $z$ y $w$ en el punto $(0,1)$.
    En $U$, tenemos que $F(x,y,z(x,y),w(x,y))=0$, por lo que sus derivadas parciales respecto de $x$ e $y$ han de ser idénticamente nulas.
    Por tanto, evaluando todas las funciones en un punto genérico $(x,y)\in U$, tenemos que las derivadas parciales respecto de $x$ son:
    \begin{align*}
        3zx^2 + x^3\del{z}{x} + 2wy^3\del{w}{x} &= 0 \\
        2w^3\del{z}{x} + 6zw^2\del{w}{x} +y^2&= 0
    \end{align*}

    Ahora, buscamos las derivadas parciales respecto de $x$ en un punto genérico $(x,y)\in U$:
    \begin{multline*}
        \del{z}{x}(x,y) = -\frac{2wy^3\del{w}{x} + 3zx^2}{x^3}
        = -\frac{6zw^2\del{w}{x} +y^2}{2w^3} \Longrightarrow \\
        \Longrightarrow 4w^4y^3\del{w}{x} + 6zw^3x^2 = 6zx^3w^2\del{w}{x} + y^2x^3 \Longrightarrow \\
        \Longrightarrow \del{w}{x}(x,y) = \frac{y^2x^3 - 6zx^3w^2}{4w^4y^3 - 6zx^3w^2}
    \end{multline*}
    Por tanto,
    \begin{equation*}
        \del{z}{x}(x,y) = -\frac{6zw^2\cdot \frac{y^2x^3 - 6zx^3w^2}{4w^4y^3 - 6zx^3w^2} +y^2}{2w^3}
        = -\frac{3zx^3\cdot \frac{y^2 - 6zw^2}{2w^2y^3 - 3zx^3} +y^2}{2w^3}
    \end{equation*}

    En concreto, en el punto $(0,1)$, como $z(0,1)=0$ y $w(0,1)=1$, tenemos que:
    \begin{equation*}
        \del{z}{x}(0,1) = -\frac{1}{2} \hspace{1cm}
        \del{w}{x}(0,1) = 0
    \end{equation*}

    Análogamente, las derivadas parciales respecto de $y$ son:
    \begin{align*}
        x^3\del{z}{y} + 2wy^3\del{w}{y} + 3w^2y^2 &= 0 \\
        2w^3\del{z}{y} + 6zw^2\del{w}{y} +2xy&= 0
    \end{align*}

    Buscamos las derivadas parciales respecto de $y$ en un punto genérico $(x,y)\in U$:
    \begin{multline*}
        \del{z}{y}(x,y) = -\frac{2wy^3\del{w}{y} + 3w^2y^2}{x^3}
        = -\frac{3zw^2\del{w}{y} +xy}{w^3} \Longrightarrow \\
        \Longrightarrow 2w^4y^3\del{w}{y} + 3w^4y^2 = 3zw^2x^3\del{w}{y} + x^4y \Longrightarrow \\
        \Longrightarrow \del{w}{y}(x,y) = \frac{x^4y - 3w^4y^2}{2w^4y^3 - 3zw^2x^3}
    \end{multline*}

    Por tanto,
    \begin{equation*}
        \del{z}{y}(x,y) = -\frac{3zw^2\cdot \frac{x^4y - 3w^4y^2}{2w^4y^3 - 3zw^2x^3} +xy}{w^3}
        = -\frac{3yz\cdot \frac{x^4 - 3w^4y}{2w^2y^3 - 3zx^3} +xy}{w^3}
    \end{equation*}

    En concreto, en el punto $(0,1)$, como $z(0,1)=0$ y $w(0,1)=1$, tenemos que:
    \begin{equation*}
        \del{z}{y}(0,1) = 0 \hspace{1cm}
        \del{w}{y}(0,1) = -\frac{3}{2}
    \end{equation*}

    Por tanto, la matriz jacobiana de $h$ en el punto $(0,1)$ es:
    \begin{equation*}
        Jh(0,1) = \left(
            \begin{array}{cc}
                \nicefrac{-1}{2} & 0 \\
                0 & \nicefrac{-3}{2}
            \end{array}
        \right)
    \end{equation*}

    En un punto genérico, tenemos que:
    \begin{equation*}
        Jh(x,y)\left(
            \begin{array}{cc}
                -\dfrac{3zx^3\cdot \frac{y^2 - 6zw^2}{2w^2y^3 - 3zx^3} +y^2}{2w^3} & -\dfrac{3yz\cdot \frac{x^4 - 3w^4y}{2w^2y^3 - 3zx^3} +xy}{w^3} \\ \\
                \dfrac{y^2x^3 - 6zx^3w^2}{4w^4y^3 - 6zx^3w^2} & \dfrac{x^4y - 3w^4y^2}{2w^4y^3 - 3zw^2x^3}
            \end{array}
        \right)
    \end{equation*}

    Empezamos ahora a comprobar las hipótesis del teorema de la función inversa local en el punto $(0,1)$ para la función $h$.
    Claramente, $h$ es diferenciable en $U$, ya que ambas componentes son diferenciables en $U$.
    Además, se tiene que $(0,1)\in U$. También se tiene que $\det Jh(0,1) = \nicefrac{3}{4} \neq 0$, por
    lo que solo falta comprobar que $Dh$ es continua en $(0,1)$. Como $h$ es diferenciable en $U$, tenemos que:
    \begin{align*}
        Dh(x,y) &= Jh(x,y)\left(
            \begin{array}{c}
                x\\y
            \end{array}
        \right) =\\&=
        \left[-\dfrac{3zx^4\cdot \frac{y^2 - 6zw^2}{2w^2y^3 - 3zx^3} +xy^2}{2w^3} -\dfrac{3y^2z\cdot \frac{x^4 - 3w^4y}{2w^2y^3 - 3zx^3} +xy^2}{w^3},\right.
        \\ &\hspace{2cm}
        \left.\dfrac{y^2x^4 - 6zx^4w^2}{4w^4y^3 - 6zx^3w^2} + \dfrac{x^4y^2 - 3w^4y^3}{2w^4y^3 - 3zw^2x^3}
        \right] =\\&=
        \left[-\dfrac{3\cdot \frac{zx^4(y^2 - 6zw^2) + 2y^2z(x^4-3w^4y)}{2w^2y^3 - 3zx^3} +3xy^2}{2w^3},
        \hspace{1cm}
        \dfrac{3y^2x^4 - 6zx^4w^2 -6w^4y^3}{4w^4y^3 - 6zx^3w^2}
        \right]
    \end{align*}

    Por tanto, como la función $Dh$ es racional, tenemos que es continua, y en particular, es continua en $(0,1)$.
    Por tanto, se cumplen todas las hipótesis del teorema de la función inversa local en el punto $(0,1)$ para la función $h$, por lo que
    existe un abierto $V\subset \bb{R}^2$ con $(0,1)\in V$ tal que $h$ es inyectiva en $V$.
\end{ejercicio}


\begin{ejercicio}
    Probar que el sistema de ecuaciones
    \begin{equation*}
        \left\{
            \begin{array}{ll}
                t\cos x + x\cos y + y\cos t &= \pi \\
                t^2 + x^2 + y^2 - tx &= \pi^2
            \end{array}
        \right.
    \end{equation*}
    define funciones implícitas $x=x(t)$ e $y=y(t)$, derivables en un entorno
    del origen, con $x(0)=0$ e $y(0)=\pi$. Calcular $x'(0)$ e $y'(0)$.\\

    Sea $\Omega = \bb{R}^3\in \mathcal{T}_u$ y $F=(F_1,F_2):\Omega \to \bb{R}^2$ definida por:
    \begin{align*}
        F_1(t,x,y) &= t\cos x + x\cos y + y\cos t-\pi \\
        F_2(t,x,y) &= t^2 + x^2 + y^2 - tx - \pi^2
    \end{align*}

    Tenemos que ambas componentes de $F$ son de clase $C^1(\bb{R}^3)$ por ser producto de polinómicas con trigonométricas,
    por lo que $F\in C^1(\bb{R}^3,\bb{R}^2)$. Además, trivialmente se tiene que $P_0=(0,0,\pi)\in \Omega$ y $F(0,0,\pi)=(0,0)$.
    Nos falta por comprobar que:
    \begin{equation*}
        \det \left( \del{(F_1,F_2)}{(x,y)}(0,0,\pi) \right) \neq 0
    \end{equation*}

    Para ello, buscamos la matriz jacobiana de $F$ en el punto $(0,0,\pi)$. Para ello, dado un punto genérico $P=(t,x,y)\in \Omega$, tenemos que:
    \begin{equation*}
        \begin{array}{llll}
            \deld{F_1}{t}(P) = \cos x - y\sen t & \deld{F_1}{x}(P) = -t\sen x +\cos y & \deld{F_1}{y}(P) = -x\sen y +\cos t \\ \\
            \deld{F_2}{t}(P) = 2t - x & \deld{F_2}{x}(P) = 2x - t & \deld{F_2}{y}(P) = 2y
        \end{array}
    \end{equation*}

    Por tanto, en el punto $P_0=(0,0,\pi)$ tenemos que:
    \begin{equation*}
        JF(P_0) = \left(
            \begin{array}{ccc}
                1 & -1 & 1 \\
                0 & 0 & 2\pi
            \end{array}
        \right)
    \end{equation*}

    Por tanto, el determinante que nos interesa es:
    \begin{equation*}
        \det \left( \del{(F_1,F_2)}{(x,y)}(P_0) \right) =
        \left|
            \begin{array}{cc}
                -1 & 1 \\
                0 & 2\pi
            \end{array}
        \right| = -2\pi \neq 0
    \end{equation*}

    Por el teorema de la función implícita, tenemos que existe un abierto $U\subset \bb{R}$,
    con $0\in U$, de manera que $F$ define funciones implícitas $x,y:U\to \bb{R}$, diferenciables en $U$,
    con $x(0)=0$ y $y(0)=\pi$, tales que:
    \begin{equation}
        \left\{
            \begin{array}{ll}
                F_1(t,x(t),y(t)) &= 0 \\
                F_2(t,x(t),y(t)) &= 0
            \end{array}
        \right. \hspace{1cm} \forall t\in U
    \end{equation}

    Nos falta ahora por calcular $x'(0)$ e $y'(0)$. Para ello, para todo $t\in U$, se tiene que $F(t,x(t),y(t))=0$,
    por lo que sus derivadas respecto de $t$ han de ser idénticamente nulas.
    Por tanto, evaluando todas las funciones en un punto genérico $t\in U$, tenemos que:
    \begin{align*}
        \cos x -t\sen xx'(t) +\cos yx'(t) - xsen yy'(t) +\cos ty'(t) -y\sen t&= 0 \\ 
        2t + 2x x'(t) +2y y'(t) - x -tx'(t) &= 0
    \end{align*}

    Evaluando en el punto $t=0$, como $x(0)=0$ e $y(0)=\pi$, tenemos que:
    \begin{align*}
        1 -x'(0) +y'(0) &= 0\\
        2\pi y'(0) &= 0
    \end{align*}

    Por tanto, tenemos que:
    \begin{equation*}
        x'(0) = 1 \hspace{1cm}
        y'(0) = 0.
    \end{equation*}
\end{ejercicio}


\begin{ejercicio}
    Probar que el sistema de ecuaciones
    \begin{equation*}
        \left\{
            \begin{array}{ll}
                x^3u - yu^3 + xv^3 - y^3v &= 0 \\
                (x^2+y^2)(u^4+v^4) + 2uv &= 0
            \end{array}
        \right.
    \end{equation*}
    define funciones implícitas $u=u(x,y)$ y $v=v(x,y)$, diferenciables en un entorno
    del punto $(1,0)$, con $u(1,0)=1$ y $v(1,0)=-1$. Calcular las derivadas parciales de $u$ y $v$ en el punto $(1,0)$.\\
\end{ejercicio}




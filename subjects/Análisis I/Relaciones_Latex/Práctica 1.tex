\section{Continuidad}

\begin{ejercicio}
    Estudiar la continuidad de los campos escalares definidos por
    \begin{equation*}
        f(x,y) = \frac{xy}{x^2+y^2}
        \qquad
        g(x,y)=\frac{x^2y}{x^2+y^4}
        \qquad
        h(x,y)=\frac{xy^2}{x^2+y^4}
    \end{equation*}
    para todo $(x, y) \in \bb{R}^2\setminus \{(0, 0)\}$, con $f(0, 0) = g(0, 0) = h(0, 0) = 0$.\\

    Tenemos que $\{(0,0)\}\in C_{\mathcal{T}_u}$, por lo que $U=\bb{R}^2\setminus \{(0,0)\}\in \mathcal{T}_u$. Tenemos que $f_{\big| U}, g_{\big| U}, h_{\big| U}$ son racionales, por lo que son una funciones continuas. Por tanto, por el carácter local de la continuidad tenemos que $f,g,h$ son continuas en todo punto de $U$. Veamos ahora si son continuas en el origen.\\

    En este caso, al ser el límite en el origen son de ayuda los límites direccionales en coordenadas cartesianas. Dado $x\in \bb{R}^\ast$ y $\lambda\in \bb{R}$, tenemos que:
    \begin{equation*}
        \lim_{x\to 0} f(x,\lambda x) = \lim_{x\to 0}\frac{\lambda x^2}{x^2(1+\lambda^2)} = \frac{\lambda}{1+\lambda^2}
    \end{equation*}
    \begin{equation*}
        \lim_{x\to 0} g(x,\lambda x) = \lim_{x\to 0}\frac{\lambda x^3}{x^2(1+\lambda^4x^2)} = \lim_{x\to 0}\frac{\lambda x}{1+\lambda^4x^2}=0
    \end{equation*}
    \begin{equation*}
        \lim_{x\to 0} h(x,\lambda x) = \lim_{x\to 0}\frac{\lambda^2 x^3}{x^2(1+\lambda^4x^2)} = \lim_{x\to 0}\frac{\lambda^2 x}{1+\lambda^4x^2}=0
    \end{equation*}

    Por tanto, tenemos que los límites direccionales en $f$ depende de $\lambda$, por lo que tenemos que no coinciden. Por tanto, $f$ \textbf{no tiene límite en el origen.}

    Para el caso de $g,h$, tenemos todos sus límites direccionales excepto uno: el uso de coordenadas cartesianas no nos sirve para calcular el límite en la dirección de $e_2$. Calculamos por tanto el segundo límite parcial:
    \begin{equation*}
        \lim_{t\to 0}g(0,t) = \frac{0}{0+t^4} =\lim_{t\to 0}h(0,t)
    \end{equation*}

    Por tanto, tenemos que todos los límites direccionales de $g,h$ existen y son iguales a 0. Por tanto, de existir el límite, este será 0. Intentemos realiza una acotación:
    \begin{gather*}
        0 < |g(x,y)|=\left|\frac{x^2y}{x^2+y^4}\right| = \left|\frac{x^2}{x^2+y^4}\right| \cdot |y| < |y| \\
        0 < |h(x,y)|=\left|\frac{xy^2}{x^2+y^4}\right| = \left|\frac{y^2}{x^2+y^4}\right| \cdot |x| < |x|
    \end{gather*}

    Como $(x,y)\to (0,0)$, tenemos que:
    $$\displaystyle \lim_{(x,y)\to (0,0)} g(x,y)=\lim_{(x,y)\to (0,0)} h(x,y)=0.$$
\end{ejercicio}


\begin{ejercicio}
    Estudiar la continuidad de la función $f : \bb{R}^2 \to \bb{R}$ dada, para $(x, y) \in \bb{R}^2$ , por:
    \begin{equation*}
        f(x,y)=
        \left\{\begin{array}{ccc}
            \displaystyle \frac{x^2+y^2}{(x+y)^2}\cdot \sen(x+y) &\text{si}& x+y\neq 0\\
            0 &\text{si}& x+y=0
        \end{array}\right.
    \end{equation*}

    Veamos que $U=\{(x,y)\in \bb{R}^2\mid x+y\neq 0\}\in {\mathcal{T}_u}$. Para ello, vemos que el siguiente conjunto es un cerrado: $\bb{R}^2\setminus U=\{(x,y)\in \bb{R}^2\mid x+y= 0\}\in C_{\mathcal{T}_u}$. Definimos la función polinómica continua $f(x,y)=x+y$. Tenemos que $\bb{R}^2\setminus U=f^{-1}(\{0\})$, que es un cerrado por ser $f$ continua y $\{0\}\in C_{\mathcal{T}_u}$. Por tanto, tenemos que $U\in \mathcal{T}_u$.

    La restricción de $f$ a $U$ es el producto de una función racional por la composición de la función suma con el seno, ambas continuas. Por tanto, $f_{\big| U}$ es continua. Por tanto, por el carácter local de la continuidad, tenemos que $f$ es continua en todos los puntos de $U$.

    Veamos ahora en los puntos que no pertenecen a $U$:
    \begin{equation*}
        \bb{R}^2\setminus U = \{(x,y)\in \bb{R}^2\mid x+y= 0\}=\{(x,y)\in \bb{R}^2\mid x=-y\} = \{(a,-a)\in \bb{R}^2\}
    \end{equation*}

    Realizamos la siguiente distinción:
    \begin{enumerate}
        \item $a\neq 0$. Es decir, excluimos en esta distinción el origen. Estudiemos los límites parciales:
        \begin{equation*}
            \lim_{x\to a}f(x,-a)
            = \lim_{x\to a} \frac{x^2+a^2}{x-a}\cdot \frac{\sen(x-a)}{x-a} = \frac{2a^2}{0}\cdot 1 = \pm \infty
        \end{equation*}
        donde he aplicado que $\lim\limits _{x\to a}\frac{\sen(x-a)}{x-a}=1$. El signo del límite dependerá si es el límite por la derecha o por la izquierda, pero en cualquier caso diverge. Por tanto, como no existe el primer límite parcial, tenemos que $f$ no es continua en ningún punto de $\{(x,y)\in \bb{R}^2\mid x+y=0\}\setminus \{(0,0)\}$.

        \item $a=0$. Es decir, estudiamos en el origen. En este caso, nos puede ser de ayuda calcular los límites direccionales en coordenadas.
        \begin{multline*}
            \lim_{x\to 0}f(x,\lambda x)
            = \lim_{x\to 0} \frac{x^2(1+\lambda^2)}{x^2(1+\lambda)^2}\cdot \sen[x(1+\lambda)]
            = \lim_{x\to 0} \frac{x^2(1+\lambda^2)}{x(1+\lambda)}\cdot \frac{\sen[x(1+\lambda)]}{x(1+\lambda)}
            =\\= \lim_{x\to 0} \frac{x(1+\lambda^2)}{1+\lambda}\cdot \frac{\sen[x(1+\lambda)]}{x(1+\lambda)} = 0
        \end{multline*}
        donde he aplicado que $\lim\limits _{x\to a}\frac{\sen(x-a)}{x-a}=1$. Por tanto, vemos que no aporta información. Realizamos el siguiente cambio de variable: $(x,y)=\varphi(t)=(t,-t+t^p)$ para cierto $p\in \bb{N}$. Comprobemos que este cambio de variable es válido:
        \begin{equation*}
            \lim_{t\to 0}\varphi(t)=(0,0) \qquad \varphi(t)\in \bb{R}^2\setminus\{(0,0)\}~\forall t\neq 0.
        \end{equation*}
        Entonces, tenemos que:
        \begin{multline*}
            \lim_{t\to 0}f(\varphi(t))
            = \lim_{t\to 0}f(t,-t+t^p)
            = \lim_{t\to 0}\frac{t^2+(-t+t^p)^2}{t^p}\cdot \frac{\sen(t^p)}{t^p}
            =\\= \lim_{t\to 0}\frac{t^2+t^2+t^{2p}-2t^{p+1}}{t^p}\cdot \frac{\sen(t^p)}{t^p}
            = \lim_{t\to 0}\frac{2+t^{2p-2}-2t^{p-1}}{t^{p-2}}\cdot \frac{\sen(t^p)}{t^p}
        \end{multline*}
        Para $p>2$ (3, por ejemplo), tenemos que $(f\circ \varphi)(t)$ diverge en 0, por lo que $f$ no tiene límite en el origen. Por tanto, $f$ no es continua en el origen.  
    \end{enumerate}

    En conclusión, se tiene que $f$ solo es continua en $U$. Es decir, $f$ es continua en $(x,y)\in \bb{R}^2$ si y solo si $x+y\neq 0$.
\end{ejercicio}

\begin{ejercicio}
    Estudiar la continuidad del campo escalar $f : \bb{R}^2 \to \bb{R}$ definido, para $(x, y) \in \bb{R}^2$ , por:
    \begin{equation*}
        f(x,y)=
        \left\{\begin{array}{ccc}
            \displaystyle \frac{y\log(1+x^2)}{x^2+y^2} &\text{si}& (x,y)\neq (0,0)\\
            0 &\text{si}& x=y=0
        \end{array}\right.
    \end{equation*}

    Sea $U=\bb{R}\setminus \{(0,0)\}$. Como un único punto es un cerrado, tenemos que $U$ es abierto. 
    Veamos que la restricción de $f$ a $U$ es continua. Tenemos que es el producto de una función racional con la función $(x,y)\mapsto \log(1+x^2)$. Esta es una composición de una función polinómica que toma valores en $\bb{R}^+$ con el logaritmo, que es continua en $\bb{R}^+$. Por tanto, tenemos que la composición es continua, y por tanto $f$ es continua en $U$. Estudiemos el caso del origen.

    En primer lugar, definimos $\varphi:\bb{R}^\ast \to \bb{R}$ por:
    \begin{equation*}
        \varphi(x)=\frac{\ln(1+x^2)}{x^2}~~\forall x\in \bb{R}^\ast
    \end{equation*}    
    Es importante notar que $\lim\limits_{x\to 0}\varphi(x)=1$. Por tanto, tenemos que $\ln(1+x^2)=x^2\varphi(x)$ para todo $x\in \bb{R}^\ast$. Entonces:
    \begin{equation*}
        0 < |f(x,y)|
        = \left|\frac{y\log(1+x^2)}{x^2+y^2}\right|
        = \left|\frac{y\cdot \varphi(x)x^2}{x^2+y^2}\right|
        = \left|\frac{x^2}{x^2+y^2}\right|\cdot |y\cdot \varphi(x)| < |y\cdot \varphi(x)|
    \end{equation*}
    Como tenemos que $\lim\limits_{(x,y)\to (0,0)}y\cdot \varphi(x)=0\cdot 1=0$, por la acotación conseguida se tiene que
    $$\lim\limits_{(x,y)\to (0,0)}f(x,y)=0=f(0,0)$$

    Por tanto, tenemos que $f$ es continua en $\bb{R}^2$.
    
\end{ejercicio}

\begin{ejercicio}
    Dado $\alpha \in \bb{R}$, estudiar la existencia de límite en el punto dado por $a = (0, 1, 2)$ del campo escalar $f : \bb{R}^3 \setminus \{a\} \to \bb{R}$ definido por:
    \begin{equation*}
        f(x,y,z) = \frac{x(y-1)(z-2)}{(x^2+(y-1)^2+(z-2)^2)^\alpha} \qquad \forall (x,y,z)\in \bb{R}^3\setminus \{a\}
    \end{equation*}

    Hacemos uso de los límites direccionales según la dirección de $(u,v,w)\in \bb{R}^3$, con $\|(u,v,w)\|_2=1$:
    \begin{equation*}
        \lim_{t\to 0}f(tu, 1+tv, 2+tw)
        = \lim_{t\to 0}\frac{t^3 \cdot uvw}{(t^2(u^2+v^2+z^2))^\alpha}
        = \lim_{t\to 0}\frac{t^3 \cdot uvw}{t^{2\alpha}}
        = \lim_{t\to 0} t^{3-2\alpha} \cdot uvw
    \end{equation*}

    Distinguimos según los valores de $\alpha$:
    \begin{enumerate}
        \item Para $3-2\alpha<0 \Longleftrightarrow \alpha>\dfrac{3}{2}$:

        Tenemos que $\lim\limits_{t\to 0} t^{3-2\alpha} \cdot uvw=\pm \infty$, dependiendo de si es el límite por la derecha o por la izquierda. En cualquier caso, tenemos que no existen los límites direccionales y, por tanto, no existe $\lim\limits_{x\to a} f(x)$.

        \item Para $3-2\alpha=0 \Longleftrightarrow \alpha=\dfrac{3}{2}$:

        Tenemos que $\lim\limits_{t\to 0} t^{3-2\alpha} \cdot uvw=uvw$, por lo que el valor del límite direccional depende de la dirección. Por tanto, no existe $\lim\limits_{x\to a} f(x)$.

        \item Para $3-2\alpha>0 \Longleftrightarrow \alpha<\dfrac{3}{2}$:

        Tenemos que $\lim\limits_{t\to 0} t^{3-2\alpha} \cdot uvw=0$, por lo que todos los límites direccionales son iguales a 0.

        Intentemos realizar la acotación. Para ello, acotamos en su lugar la función resultante del límite direccional $f(tu, 1+tv, 2+tw)=t^{3-2\alpha} \cdot uvw$.

        Haciendo uso de que $\|(u,v,w)\|_2=1=\sqrt{u^2+v^2+w^2}$, veamos que $|uvw|\leq 1$:
         \begin{equation*}
             \sqrt{u^2+v^2+w^2}=1 \Longrightarrow \sqrt{u^2}=|u|\leq 1
         \end{equation*}
         Análogamente, se tiene que $|v|,|w|\leq 1$, por lo que $|uvw|\leq 1$. Por tanto, tenemos que:
         \begin{equation*}
             0<|f(tu, 1+tv, 2+tw)|
             =|t^{3-2\alpha} \cdot uvw|
             =|t^{3-2\alpha}| \cdot |uvw|
             \leq |t^{3-2\alpha}|
         \end{equation*}

         Por tanto, tenemos que $\lim\limits_{(x,y,z)\to a}f(x,y,z)=0$.
    \end{enumerate}
\end{ejercicio}


\begin{ejercicio}
    Dados $\alpha, \beta \in \bb{R}^+$, estudiar la existencia de límite en el origen del campo escalar $f : \bb{R}^+\times \bb{R}^+ \to \bb{R}$ definido por:
    \begin{equation*}
        f(x,y) = \frac{x^\alpha y^\beta}{x^2+y^2-xy} \qquad \forall (x,y)\in \bb{R}^+\times \bb{R}^+
    \end{equation*}

    En este caso, tenemos que solo está definida en el primer cuadrante. Por tanto, no tiene sentido considerar todos los límites radiales, sino tan solo los que se encuentran en el primer cuadrante. Usando coordenadas polares, sea $\theta \in \left]0,\frac{\pi}{2}\right[$ y $\rho\in \bb{R}^+$. Entonces:
    \begin{multline*}
        \lim_{\rho\to 0}f(\rho \cos \theta, \rho \sen \theta)
        = \lim_{\rho\to 0} \rho^{\alpha+\beta-2} \cdot \frac{\cos^\alpha\theta \sen^\beta\theta}{\cos^2\theta +\sen^2\theta -\cos\theta \sen\theta}
        =\\= \lim_{\rho\to 0} \rho^{\alpha+\beta-2} \cdot \frac{\cos^\alpha\theta \sen^\beta\theta}{1 -\cos\theta \sen\theta}
        =
        \lim_{\rho\to 0} 2\rho^{\alpha+\beta-2} \cdot \frac{\cos^\alpha\theta \sen^\beta\theta}{2 -\sen(2\theta)}
    \end{multline*}

    Distinguimos según los valores de $\alpha$:
    \begin{enumerate}
        \item Para $\alpha+\beta<2$:

        Tenemos que $\lim\limits_{\rho\to 0} 2\rho^{\alpha+\beta-2} \cdot \frac{\cos^\alpha\theta \sen^\beta\theta}{2 -\sen(2\theta)}=+ \infty$. No existen los límites radiales y, por tanto, no existe $\lim\limits_{x\to (0,0)} f(x)$.

        \item Para $\alpha+\beta=2$:

        Tenemos que $\lim\limits_{\rho\to 0} 2\rho^{\alpha+\beta-2} \cdot \frac{\cos^\alpha\theta \sen^\beta\theta}{2 -\sen(2\theta)}=\frac{\cos^\alpha\theta \sen^\beta\theta}{2 -\sen(2\theta)}$, por lo que el valor del límite direccional depende del ángulo $\theta$. Por tanto, no existe $\lim\limits_{x\to (0,0)} f(x)$.

        \item Para $\alpha+\beta>2$:

        Tenemos que $\lim\limits_{\rho\to 0} 2\rho^{\alpha+\beta-2} \cdot \frac{\cos^\alpha\theta \sen^\beta\theta}{2 -\sen(2\theta)}=0$, por lo que todos los límites radiales son iguales a 0.

        Intentemos realizar la acotación. Para ello, acotamos en su lugar la función resultante del límite radial $f(\rho \cos \theta, \rho \sen \theta)=2\rho^{\alpha+\beta-2} \cdot \frac{\cos^\alpha\theta \sen^\beta\theta}{2 -\sen(2\theta)}$.
         \begin{equation*}
             0<|f(\rho \cos \theta, \rho \sen \theta)|
             =\left|2\rho^{\alpha+\beta-2} \cdot \frac{\cos^\alpha\theta \sen^\beta\theta}{2 -\sen(2\theta)}\right|\leq 
             \left|2\rho^{\alpha+\beta-2} \cdot \frac{1}{2 -\sen(2\theta)}\right|
             \leq 
             \left|2\rho^{\alpha+\beta-2}\right|
         \end{equation*}

         Por tanto, tenemos que $\lim\limits_{(x,y,z)\to 0}f(x,y,z)=0$.
    \end{enumerate}
\end{ejercicio}

\begin{ejercicio}
    Estudiar la continuidad del campo escalar $f : \bb{R}^2 \to \bb{R}$ definido por:
    \begin{equation*}\begin{split}
        &f(x,y) = (x+y)\sen\left(\frac{1}{x}\right)\sen\left(\frac{1}{y}\right) \qquad \forall x,y\in \bb{R}^\ast \\
        &f(x,0)=f(0,y)=0 \qquad \forall x,y\in \bb{R}
    \end{split}\end{equation*}

    

    Sea $A=\{(x,y)\in\bb{R}^2 \mid xy=0\}$. Sea $g:\bb{R}^2\to \bb{R}$ dado por $(x,y)\mapsto xy$ la función continua producto. Tenemos que $A=g^{-1}\{0\}$, y $\{0\}$ es un cerrado. Por tanto, tenemos que $A$ es un cerrado y $U=\bb{R}\setminus A$ es un abierto.
    
    La restricción de $f$ a $U$ es un producto de tres funciones continuas. La primera es polinómica, y las otras dos funciones son composiciones de funciones racionales con el seno, que es una función continua. Por tanto, tenemos que $f_{\big|U}$ es continua. Por el carácter local de la continuidad, tenemos que $f$ es continua en cualquier punto de $U$. Veamos ahora para los puntos de $A$.

    Tenemos que $A=\{(x,y)\in\bb{R}^2 \mid xy=0\}$, por lo que al menos una de las dos coordenadas ha de ser nula:
    \begin{enumerate}
        \item Sean los puntos $(a,0)$, con $a\neq 0$. Entonces, calculamos el primer límite parcial:
        \begin{equation*}
            \lim_{y\to 0}f(a,y)
            = \lim_{y\to 0} (a+y)\sen\left(\frac{1}{a}\right)\sen\left(\frac{1}{y}\right)
        \end{equation*}

        \begin{enumerate}
            \item Supongamos $\sen\left(\frac{1}{a}\right)\neq 0\Longleftrightarrow \frac{1}{a}\neq \pi k\Longleftrightarrow a\neq \frac{1}{\pi k}$, con $k\in \bb{Z}^\ast$.
            
            Entonces, tenemos que $a\sen\left(\frac{1}{a}\right)\neq 0$. Como $\lim\limits_{y\to 0}\sen\left(\frac{1}{y}\right)$ no converge, tenemos que $\lim\limits_{y\to 0}f(a,y)$ tampoco lo hace, por lo que $f$ no es continua en estos puntos.

            \item Supongamos $\sen\left(\frac{1}{a}\right)= 0\Longleftrightarrow \frac{1}{a}= \pi k\Longleftrightarrow a= \frac{1}{\pi k}$, con $k\in \bb{Z}^\ast$.

            Entonces:
            \begin{multline*}
                \lim_{(x,y)\to (a,0)}f(x,y)
                = \lim_{(x,y)\to (a,0)} (x+y)\sen\left(\frac{1}{x}\right)\sen\left(\frac{1}{y}\right) =\\= a\sen\left(\frac{1}{a}\right)\cdot \sen(\infty) = 0\cdot \sen(\infty)=0
            \end{multline*}
        \end{enumerate}

        \item Sean los puntos $(0,a)$, con $a\neq 0$. Entonces, como la función es simétrica ($f(x,y)=f(y,x)$), se tiene que ocurre al igual que en el caso pasado. 

        Si $a=\frac{1}{\pi k}$,  con $k \in \bb{Z}^\ast$, entonces la función en $(0,a)$ converge a $0$.

        Si $a\neq \frac{1}{\pi k}$, $\forall k\in \bb{Z}^\ast$, entonces la función en $(0,a)$ no tiene límite en $(0,a)$.

        \item Estudiamos ahora el origen:

        Realizamos la siguiente acotación:
        \begin{equation*}
            0\leq \left|f(x,y)\right|
            = \left|(x+y)\sen\left(\frac{1}{x}\right)\sen\left(\frac{1}{y}\right)\right|
            \leq \left|x+y\right| \qquad \forall x,y\neq \frac{1}{\pi k},\text{ con }k\in \bb{Z}^\ast
        \end{equation*}
        Por tanto, como $\lim\limits_{(x,y)\to (0,0)}|x+y|=0$, tenemos que 
        $$\lim\limits_{(x,y)\to (0,0)}f(x,y)=0$$
        
    \end{enumerate}
\end{ejercicio}


\begin{ejercicio}
    Estudiar la existencia de límite en el origen para los campos escalares $f : \bb{R}^2 \to \bb{R}$ definidos, para $(x, y) \in \bb{R}^2 \setminus \{(0, 0)\}$, como se indica:
    \begin{enumerate}
        \item $\displaystyle f(x,y)=\frac{x^2-y^2}{x^2+y^2}$

        Comprobamos los límites direccionales en coordenadas cartesianas, que al ser el límite en el origen quedan:
        \begin{equation*}
            \lim_{x\to 0}f(x,\lambda x)
            = \lim_{x\to 0} \frac{x^2(1-\lambda^2)}{x^2(1+\lambda^2)}
            =\frac{1-\lambda^2}{1+\lambda^2}
        \end{equation*}
        Por tanto, como vemos que los límites direccionales dependen del valor de $\lambda \in \bb{R}$, tenemos que $f$ no tiene límite en el origen.

        \item $\displaystyle g(x,y)=\frac{x^2\sen y}{x^2+y^2}$

        Tenemos la siguiente acotación:
        \begin{equation*}
            0\leq |g(x,y)|
            =\left|
                \frac{x^2\sen y}{x^2+y^2}
            \right|
            =\left|
                \frac{x^2}{x^2+y^2}\cdot \sen y
            \right|
            \leq |\sen y|
        \end{equation*}

        Tenemos que $\lim\limits_{y\to 0}\sen y=0$. Por tanto, como tenemos la acotación buscada, se tiene que $$\lim\limits_{(x,y)\to(0,0)}g(x,y)=0.$$

        \item $\displaystyle h(x,y)=\frac{\ln(1+x^4)\sen^2(y)}{y^4+x^8}$

        Definimos $\varphi,\Psi:\bb{R}\to \bb{R}$ dadas por $\varphi(y)=\frac{\sen y}{y}$ y $\Psi(x)=\frac{\ln(1+x^4)}{x^4}$, tal que $\sen^2 y =\varphi^2(y)\cdot y^2$ y $\ln(1+x^4) =\Psi(x)\cdot x^4$. Además,
        \begin{equation*}
            \lim_{y\to 0}\varphi(y)
            = \lim_{y\to 0}\frac{\sen y}{y}=1
            \qquad
            \lim_{x\to 0}\Psi(x)
            = \lim_{x\to 0}\frac{\ln (1+x^4)}{x^4}=1
        \end{equation*}

        Veamos cuánto valen los límites parciales:
        \begin{equation*}
            \lim_{x\to 0}h(x,0)
            = \lim_{x\to 0} \frac{\ln(1+x^4)\cdot 0}{x^8}
            = \lim_{x\to 0} \frac{0}{x^8} = 0
        \end{equation*}

        Aplicamos ahora este cambio de variable $x=\varphi(t)=(t,t^2)$. Entonces:
        \begin{equation*}
            \lim_{t\to 0}h(\varphi(t))
            = \lim_{t\to 0}f(t,t^2)
            = \lim_{t\to 0}
            \frac{\ln(1+t^4)\sen^2(t^2)}{t^8 + t^8}
            = \lim_{t\to 0}
            \frac{\Psi(t)\cdot \cancel{t^4} \cdot\varphi^2(t^2)\cdot \cancel{t^4}}{2\cancel{t^8}}=\frac{1}{2}
        \end{equation*}

        Por tanto, tenemos que aplicando el cambio de variable $x=(t,t^2)$ obtenemos un candidato a límite distinto que usando $x=(t,0)$. Por tanto, tenemos que $h$ no tiene límite en el origen.
    \end{enumerate}
\end{ejercicio}

\begin{ejercicio}
    En cada uno de los siguientes casos, estudiar la continuidad del campo escalar $f : \bb{R}^2 \to \bb{R}$ definido, para $(x, y) \in \bb{R}^n$ , como se indica:
    \begin{enumerate}
        \item $\displaystyle
            f(x,y)=
            \left\{\begin{array}{ccc}
                \displaystyle \frac{x+y}{x-y} &\text{si}& x\neq y\\
                0 &\text{si}& x=y
            \end{array}\right. \displaystyle$

        Sea el subconjunto $U=\{(x,y)\in \bb{R}^2\mid x\neq y\}$, y consideramos su complementario $\bb{R}^2\setminus U=\{(x,y)\in \bb{R}^2\mid x-y=0\}$. Por tanto, dada $f:\bb{R}^2\to \bb{R}$ por $f(x,y)=x-y$ función continua, tenemos que $U=f^{-1}\{0\}$. Por tanto, como es la imagen inversa de un cerrado por una función continua, tenemos que $U$ es un abierto. Además, como $f_{\big|U}$ es una función racional, tenemos que es continua. Por el carácter local de continuidad, tenemos que $f$ es continua en todos los puntos de $U$.

        Para los puntos de $\bb{R}^2\setminus U$, tenemos que $\bb{R}^2\setminus U=\{(a,a)\in \bb{R}^2\mid a\in \bb{R}\}$. Entonces, para estudiar los límites laterales, realizamos la siguiente distinción:
        \begin{enumerate}
            \item Si $a\neq 0$:
            \begin{equation*}
                \lim_{x\to a}f(x,a)=
                \lim_{x\to a}\frac{x+a}{x-a} = \frac{2a}{0}=\pm \infty
            \end{equation*}
            Tenemos que $f(x,a)$ diverge en $(a,a)$, por lo que $f$ no tiene límite en $(a,a)$ para todo $a\neq 0$.

            \item Si $a=0$, tenemos:
            \begin{equation*}
                \lim_{x\to 0}f(x,0)=
                \lim_{x\to 0}\frac{x}{x} =1
                \qquad
                \lim_{y\to 0}f(0,y)=
                \lim_{y\to 0}\frac{y}{-y} =-1
            \end{equation*}
            Por tanto, tenemos que los límites laterales no coinciden, por lo que $f$ no tiene límite en el origen.
        \end{enumerate}

        Por tanto, tenemos que $f$ solo es continua en $U$.

        \item $\displaystyle
            f(x,y)=
            \left\{\begin{array}{ccc}
                \displaystyle \frac{x^3}{x^2-y^2} &\text{si}& x^2\neq y^2\\
                0 &\text{si}& x^2=y^2
            \end{array}\right. \displaystyle$

        Sea el subconjunto $U=\{(x,y)\in \bb{R}^2\mid x^2\neq y^2\}$, y consideramos su complementario $\bb{R}^2\setminus U=\{(x,y)\in \bb{R}^2\mid x^2-y^2=0\}$. Por tanto, dada $f:\bb{R}^2\to \bb{R}$ por $f(x,y)=x^2-y^2$ función continua, tenemos que $U=f^{-1}\{0\}$. Por tanto, como es la imagen inversa de un cerrado por una función continua, tenemos que $U$ es un abierto. Además, como $f_{\big|U}$ es una función racional, tenemos que es continua. Por el carácter local de continuidad, tenemos que $f$ es continua en todos los puntos de $U$.

        Para los puntos de $\bb{R}^2\setminus U$, tenemos que $\bb{R}^2\setminus U=\{(a,a), (a,-a)\in \bb{R}^2\mid a\in \bb{R}\}$. Entonces, para estudiar los límites laterales, realizamos la siguiente distinción:
        \begin{enumerate}
            \item Si $a\neq 0$:
            \begin{equation*}
                \lim_{x\to a}f(x,a)=
                \lim_{x\to a}\frac{x^3}{x^2-a^2} = \frac{a^3}{0}=\pm \infty
            \end{equation*}
            Tenemos que $f(x,a)$ diverge en $(a,a)$, por lo que $f$ no tiene límite en $(a,a)$ para todo $a\neq 0$.

            Además, $f(x,a)=f(x,-a)$, por lo que $f$ tampoco tiene límite en $(a,-a)$ para todo $a\neq 0$.

            \item Si $a=0$, tenemos:
            \begin{equation*}
                \lim_{x\to 0}f(x,0)=
                \lim_{x\to 0}\frac{x^3}{x^2} =0
                \qquad
                \lim_{y\to 0}f(0,y)=
                \lim_{y\to 0}\frac{0}{-y^2} =0
            \end{equation*}
            
            Aplicamos ahora el cambio de variable $x=(t,t+t^p)$, para cierto $p^+\in \bb{R}$. Entonces:
            \begin{equation*}
                \lim_{t \to 0}f(t,t+t^p) = 
                \lim_{t \to 0} \frac{t^3}{t^2-(t+t^p)^2} =
                \lim_{t \to 0} \frac{t^3}{t^2-t^2-t^{2p} -2t^{p+1}}=
                \lim_{t \to 0} \frac{1}{-t^{2p-3} -2t^{p-2}}
            \end{equation*}

            Para $t=2$, tenemos:
            \begin{equation*}
                \lim_{t \to 0}f(t,t+t^p) = 
                \lim_{t \to 0} \frac{1}{-t -2} = -\frac{1}{2}
            \end{equation*}
            Tenemos que no coincide con el límite dado por los límites laterales, por lo que $f$ no es continua en el origen.
        \end{enumerate}

        Por tanto, tenemos que $f$ solo es continua en $U$.

        \item $\displaystyle
            f(x,y)=
            \left\{\begin{array}{ccc}
                \displaystyle \frac{x^3-y^3}{xy} &\text{si}& xy\neq 0\\
                0 &\text{si}& xy=0
            \end{array}\right. \displaystyle$

        Sea $A=\{(x,y)\in\bb{R}^2 \mid xy=0\}$. Sea $g:\bb{R}^2\to \bb{R}$ dado por $(x,y)\mapsto xy$ función polinómica, por lo que continua. Tenemos que $A=g^{-1}\{0\}$, y $\{0\}$ es un cerrado. Por tanto, tenemos que $A$ es un cerrado y $U=\bb{R}\setminus A$ es un abierto.
    
        La restricción de $f$ a $U$ es una función polinómica, por lo que continua. Por el carácter local de la continuidad, tenemos que $f$ es continua en cualquier punto de $U$. Veamos ahora para los puntos de $A$.

        Tenemos que $A=\{(x,y)\in\bb{R}^2 \mid xy=0\}$, por lo que al menos una de las dos coordenadas ha de ser nula:
        \begin{enumerate}
            \item Sean los puntos $(a,0)$, con $a\neq 0$. Entonces, calculamos el segundo límite parcial:
            \begin{equation*}
                \lim_{y\to 0}f(a,y)
                = \lim_{y\to 0} \frac{a^3-y^3}{ay}
                = \frac{a^3}{0}\pm \infty
            \end{equation*}
            donde afirmamos que diverge ya que $a\neq 0$. Por tanto, $f$ no es continua en los puntos $(a,0)$, con $a\neq 0$.

            \item Sean los puntos $(0,a)$, con $a\neq 0$. Entonces, calculamos el primer límite parcial:
            \begin{equation*}
                \lim_{x\to 0}f(x,a)
                = \lim_{x\to 0} \frac{x^3-a^3}{xa}
                = \frac{-a^3}{0}\pm \infty
            \end{equation*}
            donde afirmamos que diverge ya que $a\neq 0$. Por tanto, $f$ no es continua en los puntos $(0,a)$, con $a\neq 0$.

            \item Sea el origen $(0,0)$. Entonces, calculamos el primer límite parcial:
            \begin{equation*}
                \lim_{x\to 0}f(x,0)
                = \lim_{x\to 0} 0 = 0
            \end{equation*}

            Aplicamos ahora el cambio de variable $x=(t,t^p)$, para cierto $p\in \bb{R}^+$. Tenemos que:
            \begin{equation*}
                \lim_{t\to 0}f(t,t^p)
                = \lim_{t\to 0} \frac{t^3-t^{3p}}{t^{p+1}}
                = \lim_{t\to 0} \frac{1-t^{3p-3}}{t^{p-2}}
                = 1 \text{\qquad para $p=2$}.
            \end{equation*}
            
            Por tanto, tenemos que según el cambio de variable del límite parcial, el candidato a límite es $0$. No obstante, en este último cambio de variable se tiene que el candidato es $1$. Por tanto, tenemos que $f$ no tiene límite en el origen, por lo que tampoco es continua en este punto.
        \end{enumerate}

        Por tanto, tenemos que $f$ solo es continua en $U$.

        \item $\displaystyle
            f(x,y)=
            \left\{\begin{array}{ccc}
                \displaystyle \frac{x}{y}\arctan(x^2+y^2) &\text{si}& y\neq 0\\
                0 &\text{si}& y=0
            \end{array}\right. \displaystyle$

        Sea $U=\{(x,y)\in \bb{R}^2\mid y\neq 0\}$. Tenemos que su complementario es la imagen inversa de $\{0\}$ por la proyección en la segunda coordenada, que es una función continua. Por ser $\{0\}$ un cerrado, tenemos que $\bb{R}^2\setminus U$ es un cerrado, por lo que $U$ es un abierto. Además, tenemos que $f$ restringido a $U$ es el producto de una función racional por la composición de una polinómica con la arcotangente, que es continua. Por tanto, tenemos que $f_{\big| U}$ es una función continua, y por el carácter local de la continuidad, tenemos que $f$ es continua en todos los puntos de $U$.

        Realizamos la siguiente distinción:
        \begin{enumerate}
            \item Para $(a,0)$, con $a\neq 0$, tenemos que el segundo límite parcial es:
            \begin{equation*}
                \lim_{y\to 0}f(a,y)
                = \lim_{y\to 0}\frac{a}{y}\arctan(a^2+y^2)
                = \frac{a}{0}\arctan(a^2)
            \end{equation*}
            Por tanto, tenemos que el segundo límite parcial diverge, ya que la arcotangente converge a un valor no nulo y el cociente, al dividir entre 0, diverge.

            \item Estudiemos ahora el límite en el origen:

            Definimos la siguiente aplicación continua, por ser el producto de una función racionar por la composición de una polinómica con la arcotangente:
            \Func{\varphi}{\bb{R}^2\setminus \{(0,0)\}}{\bb{R}}{(x,y)}{\dfrac{\arctan(x^2+y^2)}{x^2+y^2}}

            Por tanto, para $(x,y)\neq (0,0)$, tenemos la siguiente relación:
            $$\arctan(x^2+y^2)=(x^2+y^2)\cdot \varphi(x,y)$$

            Veamos en primer lugar el límite de $\varphi$ en el origen. Para ello, definimos $f,g$ tal que $\varphi = f\circ g$:
            \shorthandoff{"} % Requiere desactivar babel para el carácter " en esta sección
            \begin{figure}[H]
                \centering
                \begin{tikzcd}
                    \bb{R}^2\setminus \{(0,0)\} \arrow[r, "g"] \arrow[rr, "\varphi"', bend right] & \bb{R}^\ast \arrow[r, "f"] & \bb{R}
                \end{tikzcd}
            \end{figure}
            \shorthandon{"} % Requiere desactivar babel para el carácter " en esta sección
            Tenemos que $g(x,y)=x^2+y^2$ y $f(t)=\dfrac{\arctan t}{t}$. Veamos si podemos aplicar el cambio de variable $t=g(x,y)$.
            \begin{equation*}
                \lim_{(x,y)\to (0,0)} g(x,y)=0 \qquad g(x,y)\in \bb{R}^\ast~\forall (x,y)\in \bb{R}^2\setminus \{(0,0)\}
            \end{equation*}

            Por tanto, podemos usar el Teorema del Cambio de variable en el sentido original:
            \begin{equation*}
                \lim_{x\to 0}f(x)=1 \Longrightarrow \lim_{(x,y)\to (0,0)}f(g(x,y))=\lim_{(x,y)\to (0,0)}\varphi(x,y)=1
            \end{equation*}
            
            Realizamos ahora el cambio de variable $x=(t,t^p)$:
            \begin{multline*}
                \lim_{t\to 0}f(t,t^p)
                = \lim_{t\to 0} \frac{t}{t^p}\arctan(t^2+t^{2p})
                = \lim_{t\to 0} \frac{1}{t^{p-1}}(t^2+t^{2p})\varphi (t, t^p)
                =\\= \lim_{t\to 0} \frac{1}{t^{p-3}}(1+t^{2p-2})\varphi (t, t^p) = 1 \qquad (\text{para } p=3)
            \end{multline*}
            donde he usado al final el límite de $\varphi$ en el origen.

            Veamos ahora el valor del límite parcial:
            \begin{equation*}
                \lim_{x\to 0}f(x,0) =
                \lim_{x\to 0}0 = 0
            \end{equation*}
            Por tanto, tenemos que dos cambios de variable me dan candidatos a límite distintos, por lo que no hay límite en el origen.
        \end{enumerate}
    \end{enumerate}
\end{ejercicio}


\begin{ejercicio}
    Dado $n \in \bb{N}$, estudiar la existencia de límite en el origen para el campo escalar $f : \bb{R}^2\setminus\{(0,0)\} \to \bb{R}$ definido por
    \begin{equation*}
        f(x,y)=\frac{x^ny^n}{x^2y^2+(x-y)^2} \qquad \forall (x,y)\in \bb{R}^2\setminus \{(0,0)\}
    \end{equation*}

    Estudiamos en primer lugar los límites parciales:
    \begin{equation*}
        \lim_{x\to 0}f(x,0)=
        \lim_{x\to 0} \frac{x^n\cdot 0}{0+x^2} =
        \lim_{x\to 0} 0 = 0
        \qquad
        \lim_{y\to 0}f(0,y)=
        \lim_{y\to 0}\frac{0}{0 + y^2}=
        \lim_{x\to 0}0 = 0
    \end{equation*}
    Por tanto, tenemos que; en caso de converger, lo hará a 0. Para ver si converge, estudiamos los límites direccionales en coordenadas cartesianas:
    \begin{equation*}
        \lim\limits_{x\to 0}f(x,\lambda x)
        = \lim_{x\to 0} \frac{x^n\cdot \lambda^nx^n}{x^2\cdot \lambda^2x^2 + x^2(1-\lambda)^2}
        = \lim_{x\to 0} \frac{x^{2n}\cdot \lambda^n}{x^4\cdot \lambda^2 + x^2(1-\lambda)^2}
        = \lim_{x\to 0} \frac{x^{2n-2}\cdot \lambda^n}{x^2\cdot \lambda^2 + (1-\lambda)^2}
    \end{equation*}

    Realizamos la siguiente distinción:
    \begin{enumerate}
        \item Si $2n-2>2\Longleftrightarrow n>2$: Tenemos que $\lim\limits_{x\to 0}f(x,\lambda x)=0$.

        \item Si $2n-2=2\Longleftrightarrow n=2$: Tenemos que:
        $$\lim\limits_{x\to 0}f(x,\lambda x)
        =\lim\limits_{x\to 0}\dfrac{\lambda^2}{\lambda^2 + x^{-2}\cdot(1-\lambda)^2}
        =\lim\limits_{x\to 0}\dfrac{\lambda^2}{\lambda^2 + \frac{(1-\lambda)^2}{x^{2}}}$$

        En el caso de $\lambda=1$, tenemos:
        $$\lim\limits_{x\to 0}f(x,x)
        =\lim\limits_{x\to 0}\dfrac{1^2}{1^2 + \frac{0}{x^{2}}}
        =\lim\limits_{x\to 0}\dfrac{1^2}{1^2} = 1
        $$

        Por tanto, como para $\lambda=1$ el límite direccional difiere del resto, tenemos que no es convergente.

        \item Si $2n-2<2\Longleftrightarrow n<2$: Para $\lambda=1$, tenemos
        \begin{equation*}
            \lim\limits_{x\to 0}f(x,x)
            =\lim_{x\to 0} \frac{x^{2n-2}\cdot 1^n}{x^2\cdot 1^2}
            =\lim_{x\to 0} x^{2n-4} = \infty
        \end{equation*}
        Por tanto, como este límite direccional no existe, tenemos que $f$ no es convergente si $n<2$.
    \end{enumerate}
    \begin{observacion}
        En este caso, habiendo hecho el cambio de variable $x=(t,t)$, habría sido bastante más rápido, y no habría sido necesario calcular los límites parciales. No obstante, al seguir un procedimiento rutinario se optó por calcular los límites direccionales con coordenadas cartesianas.
    \end{observacion}

    Por tanto, tenemos que $f$ solo puede ser convergente si $n>2$, y tenemos que el candidato a límite es $0$. Intentemos acotar $f$:
    \begin{equation*}
        0\leq \left|f(x,y)\right|
        = \left|\frac{x^ny^n}{x^2y^2+(x-y)^2}\right|
        = \left|\frac{x^{n-2}y^{n-2}}{1+\left(\frac{x-y}{xy}\right)^2}\right|\leq |x^{n-2}y^{n-2}|
    \end{equation*}
    Definiendo $g:\bb{R}^2\to \bb{R}_0^+$ dado por $g(x,y)=|x^{n-2}y^{n-2}|$, tenemos la relación de orden $\|f(x)\|\leq g(x,y)$ para todo $(x,y)\in \bb{R}^2\setminus \{(0,0)\}$. Además, como $g$ es continua, tenemos que su límite en el origen debe ser igual a $g(0,0)=0$.

    Por tanto, debido a la acotación que hemos conseguido, tenemos que si $n>2$: $$\lim\limits_{(x,y)\to (0,0)}f(x,y)=0.$$
\end{ejercicio}

\begin{ejercicio}
    Dados $\alpha, b \in \bb{R}$, estudiar la existencia de límite en el punto $(0, b)$ del campo escalar $f : \bb{R}^+\times \bb{R} \to \bb{R}$ definido por:
    \begin{equation*}
        f(x,y)=x^\alpha\sen \frac{1}{x^2+y^2} \qquad \forall x\in \bb{R}^+,~y\in \bb{R}
    \end{equation*}

    Realizamos la siguiente distinción:
    \begin{enumerate}
        \item Si $\alpha>0$:

        Tenemos la siguiente acotación:
        \begin{equation*}
            0\leq |f(x,y)|
            =\left|x^\alpha\sen \frac{1}{x^2+y^2}\right|
            \leq |x^\alpha|
        \end{equation*}
        Como $\alpha>0$, tenemos que $\lim\limits_{x\to 0}x^\alpha=0$. Por tanto, $\lim\limits_{(x,y)\to (0,b)} f(x,y)=0$.

        \item Si $\alpha=0$, $b\neq 0$:

        Tenemos que $f(x,y)=\sen \frac{1}{x^2+y^2}$, que es una composición de una función racional con el seno, ambas continuas. Por tanto, tenemos que $f$ es continua en $\bb{R}^+\times \bb{R}$. Además, existe una extensión continua de $f$ que incluye al punto $(0,b)$ con $g(x,y)=f(x,y)$ para todo $x\in \bb{R}^+,~y\in \bb{R}$. Esta es:
        \Func{g}{\bb{R}^+_0\times \bb{R}}{\bb{R}}{(x,y)}{\sen \frac{1}{x^2+y^2}}

        Por tanto, se tiene que:
        \begin{equation*}
            g(0,b)=\sen \frac{1}{b^2} = \lim_{(x,y)\to (0,b)}f(x,y)
        \end{equation*}

        \item Si $\alpha=0$, $b=0$:

        Tenemos que $f(x,y)=\sen \frac{1}{x^2+y^2}$. Calculamos el primer límite parcial por la derecha:
        \begin{equation*}
            \lim_{x\to 0^+}f(x,0)=\lim_{x\to 0^+}\sen \frac{1}{x^2}
        \end{equation*}
        Tenemos que este no converge. Por tanto, se tiene que $f(x,y)$ no converge en el origen si $\alpha=0$.

        \item Si $\alpha<0$:        

        Calculamos el primer límite parcial:
        \begin{equation*}
            \lim_{x\to 0}f(x,b)
            = \lim_{x\to 0} x^\alpha \cdot \sen \frac{1}{x^2+b^2}
            = \lim_{x\to 0} \frac{\sen \frac{1}{x^2+b^2}}{x^{-\alpha}}
        \end{equation*}

        Distinguimos ahora en función de los valores de $b$. Tenemos que:
        \begin{equation*}
            \sen \frac{1}{b^2}=0 \Longleftrightarrow \frac{1}{b^2}=\pi k \Longleftrightarrow b^2=\frac{1}{\pi k}\Longleftrightarrow b=\pm \frac{1}{\sqrt{\pi k}}\qquad \text{ con } k\in \bb{Z}^\ast
        \end{equation*}

        \begin{enumerate}
            \item Si $\alpha<0$, $b\neq \pm \dfrac{1}{\sqrt{\pi k}}$, con $k\in \bb{Z}^\ast$:

            Tenemos que el primer límite parcial no existe, ya que:
            \begin{equation*}
                 \lim_{x\to 0}f(x,b) = \frac{\sen \frac{1}{b^2}}{0}=\pm \infty
            \end{equation*}
            Por tanto, tenemos que $f$ no tiene límite en $(0,b)$ para estos valores de $\alpha, b$.

            \item Si $\alpha<0$, $b= \pm \dfrac{1}{\sqrt{\pi k}}$, con $k\in \bb{Z}^\ast$:

            En este caso, lo que nos dificulta la resolución es el seno. Para poder trabajar con él, tenemos en cuesta el siguiente límite, que es la definición formal de derivada del seno:
            \begin{equation*}
                \lim_{t\to a}\frac{\sen t - \sen a}{t-a}=\cos a
            \end{equation*}
            Por tanto, para $a=\frac{1}{b^2}$, tenemos:
            \begin{equation*}
                \lim_{t\to \frac{1}{b^2}}\frac{\sen t - \sen \frac{1}{b^2}}{t-\frac{1}{b^2}}=\cos \frac{1}{b^2} = \cos(\pi k)=\pm 1
            \end{equation*}
            Por el Teorema de cambio de variable, usamos $t=\varphi(x,y)=\frac{1}{x^2+y^2}$. Tenemos que $t\to \frac{1}{b^2}$ si $(x,y)\to (0,b)$, y $t\neq \frac{1}{b^2}$ para $(x,y)\neq (0,b)$. Por tanto, 
            \begin{equation*}
                \lim_{(x,y)\to (0,b)}\frac{\sen\frac{1}{x^2+y^2} - \sen \frac{1}{b^2}}{\frac{1}{x^2+y^2} - \frac{1}{b^2}}
                = \lim_{(x,y)\to (0,b)}\frac{\sen\frac{1}{x^2+y^2}}{\frac{1}{x^2+y^2} - \frac{1}{b^2}}
                =\cos \frac{1}{b^2} = \cos(\pi k)=\pm 1
            \end{equation*}


            Haciendo uso de ese límite, podemos reescribir $f(x,y)$ de la siguiente forma:
            \begin{multline*}
                f(x,y)=x^\alpha \sen\left(\frac{1}{x^2+y^2}\right)
                = x^\alpha \left(\frac{1}{x^2+y^2} - \frac{1}{b^2}\right) \frac{\sen\left(\frac{1}{x^2+y^2}\right)}{\frac{1}{x^2+y^2} - \frac{1}{b^2}}
                =\\= x^{\nicefrac{\alpha}{2}}\cdot x^{\nicefrac{\alpha}{2}} \left(\frac{b^2-(x^2+y^2)}{b^2(x^2+y^2)}\right) \frac{\sen\left(\frac{1}{x^2+y^2}\right)}{\frac{1}{x^2+y^2} - \frac{1}{b^2}}
            \end{multline*}

            Estudiamos ahora $g:\bb{R}^2\to \bb{R}$ dada por $g(x,y)=x^{\nicefrac{\alpha}{2}} [b^2-(x^2+y^2)]$.

            Usamos el cambio de variable $(x,y)=\left(x,\sqrt{b^2-x^2+x^{\nicefrac{-\alpha}{2}}}\right)$.
            Tenemos que $(x,y)\to (0,b)$ cuando $x\to 0$, y que $(x,y)\neq (0,b)$ para $x\neq 0$.
            Entonces:
            \begin{align*}
                &g\left(x,\sqrt{b^2-x^2+x^{\nicefrac{-\alpha}{2}}}\right)
                =\\&= x^{\nicefrac{\alpha}{2}} \left[b^2 - \left(x^2 + \left(\sqrt{b^2-x^2+x^{\nicefrac{-\alpha}{2}}}\right)^2\right)\right]
                =\\&= x^{\nicefrac{\alpha}{2}} \left[b^2 - \left(x^2 + \left|b^2-x^2+x^{\nicefrac{-\alpha}{2}}\right|\right)\right]
                \AstIg \\ &\AstIg
                x^{\nicefrac{\alpha}{2}} \left[b^2 - \left(x^2 + b^2-x^2+x^{\nicefrac{-\alpha}{2}}\right)\right]
                =\\&= x^{\nicefrac{\alpha}{2}} \left[x^{\nicefrac{-\alpha}{2}}\right]
                = 1
            \end{align*}
            donde en $(\ast)$ suponemos que $b^2-x^2+x^{\nicefrac{-\alpha}{2}}\geq 0$.
            Esto es posible, ya que $b^2>0$ y, a la hora de tomar límites, siempre podremos coger $\delta\in \bb{R}^+$
            lo suficientemente pequeño como para que dicho término sea positivo.

            Por tanto, tenemos que:
            \begin{align*}
                &f(x,\sqrt{b^2-x^2+x^{\nicefrac{-\alpha}{2}}})
                =\\&= x^{\nicefrac{\alpha}{2}} \left(\frac{g(x,\sqrt{b^2-x^2+x^{\nicefrac{-\alpha}{2}}})}{b^2(x^2+\left(b^2-x^2+x^{\nicefrac{-\alpha}{2}}\right))}\right) \frac{\sen\left(\frac{1}{x^2+\left(b^2-x^2+x^{\nicefrac{-\alpha}{2}}\right)}\right)}{\frac{1}{x^2+\left(b^2-x^2+x^{\nicefrac{-\alpha}{2}}\right)} - \frac{1}{b^2}}
            \end{align*}
            donde hemos usado de nuevo que podemos presuponer que $b^2-x^2+x^{\nicefrac{-\alpha}{2}}\geq 0$.
            Por el Teorema del Cambio de Variable en su sentido original,
            tenemos que el término del cociente del seno converge a 1.
            Además, el término central converge a $\frac{1}{b^4}$ cuando $x\to 0$.
            No obstante, el término de la derecha diverge, por lo que esta función diverge cuando $x\to 0$.

            Por tanto, por el Teorema del Cambio de Variable, tenemos que $f$ no tiene límite en $(0,b)$ si $\alpha<0$, $b= \pm \dfrac{1}{\sqrt{\pi k}}$, con $k\in \bb{Z}^\ast$.
        \end{enumerate}
    \end{enumerate}
\end{ejercicio}


\begin{ejercicio}
    Dado $u = (a, b, c) \in \bb{R}^3$ , estudiar la existencia de límite en el punto $u$ del campo escalar $f : \bb{R}^3\setminus \{u\}\to \bb{R}$ definido por
    \begin{equation*}
        f(x,y,z)=\frac{xyz}{|x-a|+|y-b|+|z-c|} \qquad \forall (x,y,z)\in \bb{R}^3\setminus\{u\}
    \end{equation*}

    \begin{enumerate}
        \item Si $abc\neq 0$:

        Calculamos el primer límite parcial:
        \begin{equation*}
            \lim_{x\to a}f(x,b,c)=
            \lim_{x\to a}\frac{xbc}{|x-a|} = \frac{abc}{0}=\pm \infty
        \end{equation*}
        Por tanto, tenemos que el primer límite parcial no converge, por lo que $f$ no tiene límite en $u$ si $abc\neq 0$.

        \item Si $bc\neq 0$, $a=0$:

        Calculamos el primer límite parcial:
        \begin{equation*}
            \lim_{x\to 0^+}f(x,b,c)=
            \lim_{x\to 0^+}\frac{xbc}{|x|} =
            \lim_{x\to 0^+}bc=bc
        \end{equation*}

        Calculamos el primer segundo parcial:
        \begin{equation*}
            \lim_{y\to b}f(0,y,c)=
            \lim_{y\to b}\frac{0}{|y-b|} =
            \lim_{y\to b}0=0
        \end{equation*}
        Por tanto, como $bc\neq 0$, tenemos que los límites parciales no coinciden, por lo que $f$ no tiene límite en $u$ si $bc\neq 0$, $a=0$.

        Además, tenemos que esto se generaliza a siempre que haya 2 componentes de $u$ no nulas y la tercera nula. Sea $u_k$ la componente nula. Entonces, el límite parcial $k$-ésimo no será nulo (valdrá el producto de las otras dos componentes), mientras que el resto de límites parciales serán nulos, al anularse el denominador.

        Por tanto, si $u$ tiene dos componentes no nulas y una tercera nula, tenemos que $f$ no tiene límite en $u$.

        \item Si $a\neq 0$, $b,c=0$:

        Tenemos la siguiente acotación:
        \begin{equation*}
            0\leq |f(x,y,z)| =\left|\frac{xyz}{|x-a|+|y|+|z|}\right| = \left|\frac{z}{|x-a|+|y|+|z|}\right|\cdot |xy| \leq |xy|
        \end{equation*}
        Donde hemos usado que $\left|\frac{z}{|x-a|+|y|+|z|}\right|\leq 1$, que $|x-a|+|y|>0$. Por tanto, debido a la acotación tenemos que, si $a\neq 0$, $b,c=0$, entonces $\lim\limits_{x\to u}f(u)=0$.
        
        Además, tenemos que esto se generaliza a siempre que haya 2 componentes de $u$ nulas y la tercera no nula. Sea $u_k$ la componente no nula. Entonces, será necesario acotar por $u_k\cdot u_j$, donde $u_j$ es una de las componentes nulas.

        Por tanto, si $u$ tiene dos componentes nulas y una tercera no nula, tenemos que $\lim\limits_{x\to u}f(u)=0$.

        \item Si $a,b,c=0$:

        Tenemos la misma acotación que en el caso anterior:
        \begin{equation*}
            0\leq |f(x,y,z)| =\left|\frac{xyz}{|x|+|y|+|z|}\right| = \left|\frac{z}{|x|+|y|+|z|}\right|\cdot |xy| \leq |xy|
        \end{equation*}
        Donde hemos usado que $\left|\frac{z}{|x|+|y|+|z|}\right|\leq 1$, que $|x|+|y|>0$. Por tanto, debido a la acotación tenemos que, si $a,b,c=0$, entonces $\lim\limits_{x\to u}f(u)=0$.
    \end{enumerate}
\end{ejercicio}


\begin{ejercicio}[Prueba DGIIM 2023-24] Dados $a,b\in \bb{R}$, estudiar la continuidad del campo escalar $f:\bb{R}^2\to \bb{R}$ dado por:
\begin{equation*}
    f(x,y)=\frac{xy}{\sqrt{(x-a)^2 + (y-b)^2}}~\forall (x,y)\in \bb{R}^2\setminus \{(a,b)\} \hspace{1.5cm} f(a,b)=0
\end{equation*}

Definimos el conjunto $U=\bb{R}^2\setminus \{(a,b)\}$, que claramente es abierto por ser $\{(a,b)\}\in C_{\mathcal{T}_u}$.
En $U$, tenemos que el denominador es una función polinómica, luego continua. Además, el denominador es la composición de una función polinómica (luego continua)
que toma siempre valores en $\bb{R}^+$ con la raíz cuadrada, que es continua. Como la compoisición de funciones continuas es continua, tenemos que el denominador es una función continua en $U$.
Además, como el cociente de funciones continuas es continua, tenemos que $f_{\big | U}$ es continua.
Por el carácter local de la continuidad, tenemos que $f$ es continua en todos los puntos de $U$. Estudiemos ahora en $(a,b)$.
Para ello, realizamos la siguiente distinción de casos:
\begin{enumerate}
    \item Si $a,b\neq 0$:
    
    Calculamos el primer límite parcial. Tenemos que:
    \begin{equation*}
        f(x,b)=
        \frac{xb}{\sqrt{(x-a)^2}} =\frac{xb}{|x-a|} \qquad \forall x\in \bb{R}\setminus \{a\}
    \end{equation*}
    Que claramente diverge, por lo que $\nexists \lim\limits_{x\to a}f(x,b)$. Por tanto, como
    el primer límite parcial no existe, tenemos que $f$ no es continua en $(a,b)$.

    \item Si $a=0,~b\neq0$:
    
    Estudiamos el primer límite parcial. Tenemos que:
    \begin{equation*}
        f(x,b)= \frac{xb}{\sqrt{x^2 + \cancel{(y-b)^2}}} = \frac{xb}{|x|}
         \qquad \forall x\in \bb{R}\setminus \{0\}
    \end{equation*}
    Por tanto, tenemos que:
    \begin{equation*}
        \lim_{x\to 0^+}f(x,b)=\lim_{x\to 0^+}\frac{xb}{|x|} = \lim_{x\to 0^+}\frac{xb}{x} = b
        \qquad
        \lim_{x\to 0^-}f(x,b)=\lim_{x\to 0^-}\frac{xb}{|x|} = \lim_{x\to 0^-}\frac{xb}{-x} = -b
    \end{equation*}
    Por tanto, como $b\neq 0$, tenemos que $b\neq -b$, por lo que $\nexists \lim\limits_{x\to 0}f(x,b)$. Por tanto,
    como el primer límite parcial no existe, tenemos que $f$ no es continua en $(a,b)$.

    \item Si $a\neq 0,~b=0$:
    
    Se demuestra que no es continua de forma análoga al caso anterior. Tenemos que:
    \begin{equation*}
        f(a,y)= \frac{ay}{\sqrt{(a-a)^2 + (y-0)^2}} = \frac{ay}{|y|}
         \qquad \forall y\in \bb{R}\setminus \{0\}
    \end{equation*}
    Por tanto, tenemos que:
    \begin{equation*}
        \lim_{y\to 0^+}f(a,y)=\lim_{y\to 0^+}\frac{ay}{|y|} = \lim_{y\to 0^+}\frac{ay}{y} = a
        \qquad
        \lim_{y\to 0^-}f(a,y)=\lim_{y\to 0^-}\frac{ay}{|y|} = \lim_{y\to 0^-}\frac{ay}{-y} = -a
    \end{equation*}
    Por tanto, como $a\neq 0$, tenemos que $a\neq -a$, por lo que $\nexists \lim\limits_{y\to 0}f(a,y)$. Por tanto,
    como el primer segundo parcial no existe, tenemos que $f$ no es continua en $(a,b)$.

    \item Si $a,b=0$:
    
    Estamos en el caso del origen. Tenemos que:
    \begin{equation*}
        f(x,y)= \frac{xy}{\sqrt{x^2 + y^2}} \qquad \forall (x,y)\in \bb{R}^2\setminus \{(0,0)\}
    \end{equation*}

    Veamos que es continua usando la siguiente acotación:
    \begin{equation*}
        0\leq |f(x,y)| =\left|\frac{xy}{\sqrt{x^2 + y^2}}\right| = \left|\frac{x}{\sqrt{x^2 + y^2}}\right|\cdot |y| \leq |y|
    \end{equation*}
    Como $|y|\to 0$, tenemos que $\lim\limits_{(x,y)\to (0,0)}f(x,y)=0$. Por tanto, tenemos que $f$ es continua en $(0,0)$.
    
    
\end{enumerate}
    
\end{ejercicio}
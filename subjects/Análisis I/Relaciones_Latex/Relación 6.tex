\section{Diferenciabilidad}

\begin{ejercicio}
Probar que la norma de un espacio normado \( X \neq \{0\} \) nunca es diferenciable en 0.
\end{ejercicio}

\begin{ejercicio}
Probar que la norma euclídea es diferenciable en todo punto de \( \mathbb{R}^N \setminus \{0\} \) y calcular su diferencial. ¿En qué puntos de \( \mathbb{R}^2 \) es diferenciable la norma de la suma?
\end{ejercicio}

\begin{ejercicio}
Probar que, tanto la diferenciabilidad de una función, como su diferencial, se conservan por traslaciones. Más concretamente, sean \( X, Y \) espacios normados, \( \Omega \) un abierto no vacío de \( X \) y \( f : \Omega \rightarrow Y \) diferenciable en un punto \( a \in \Omega \). Fijados \( x_0 \in X \) y \( y_0 \in Y \), se define \( \hat{\Omega} = \{x \in X\mid x + x_0 \in \Omega\} \) y la función \( \hat{f} : \hat{\Omega} \rightarrow Y \) dada por
\[
\hat{f}(x) = f(x + x_0) + y_0 \quad \forall x \in \hat{\Omega}
\]
Probar que \( \hat{f} \) es diferenciable en \( a - x_0 \) con \( D\hat{f}(a - x_0) = Df(a) \).
\end{ejercicio}

\begin{ejercicio}
Sea \( \Omega \) un abierto no vacío de un espacio normado \( X \) y \( f : \Omega \rightarrow \mathbb{R} \) diferenciable en un punto \( a \in \Omega \). Sea \( J \) un intervalo abierto tal que \( f(\Omega) \subseteq J \) y \( g : J \rightarrow \mathbb{R} \) una función derivable en el punto \( b = f(a) \). Probar que \( g \circ f \) es diferenciable en \( a \), ¿Cuál es la relación entre \( D(g \circ f)(a) \) y \( Df(a) \)?
\end{ejercicio}

\begin{ejercicio}
Dar un ejemplo de una función \( f \in D(\mathbb{R}^N) \) tal que \( f \) no es de clase \( C^1(\mathbb{R}^N) \).
\end{ejercicio}
    
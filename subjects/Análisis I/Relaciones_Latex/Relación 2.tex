\section{Topología de un espacio métrico}


\begin{ejercicio}
    Probar que, en todo espacio métrico, la distancia queda determinada cuando se conocen las bolas abiertas. En el caso particular de un espacio normado, probar que la norma queda determinada cuando se conoce la bola abierta unidad.\\

    Sea el espacio métrico $(E,d)$. Sean $x,y\in E$, y definimos el siguiente conjunto $A=\{\veps\in \bb{R}^+\mid y\in B(x,\veps)\}$. Tenemos que $A$ no es vacío, ya que $\lim\limits_{\veps\to \infty}B(x,\veps)=E$. Por tanto, por el Teorema del Ínfimo tenemos que $\exists \inf A$. Demostramos que:
    \begin{equation*}
        d(x,y)=\inf A = \inf \{\veps\in \bb{R}^+\mid y\in B(x,\veps)\}
    \end{equation*}

    Para ello, vemos en primer lugar que $d(x,y)$ es un minorante de $A$. Demostramos por reducción al absurdo, suponiendo que $\exists \rho\in A\mid \rho<d(x,y)$. Entonces, por definición de $A$ se tiene que $y\in B(x,\rho)\subset B(x,d(x,y))$. Por tanto, se tiene que $y\in B(x,d(x,y))$, por lo que $d(x,y)<d(x,y)$, siendo esto una contradicción.\\

    Por tanto, $d(x,y)\leq \veps,~\forall \veps\in A$. Veamos además que $\exists \{\veps_n\}\to d(x,y)$, con $\veps_n\in A,~\forall n\in \bb{N}$. Sea $\veps_n:=d(x,y)+\frac{1}{n}$. Comprobemos que $\veps_n\in A,~\forall n\in \bb{N}$:
    \begin{equation*}
        \veps_n=d(x,y)+\frac{1}{n} \in A \Longleftrightarrow y\in B\left(x,d(x,y)+\frac{1}{n}\right) \Longleftrightarrow d(x,y)<d(x,y)+\frac{1}{n} \Longleftrightarrow 0<\frac{1}{n}
    \end{equation*}

    Además, tenemos que es trivial que $\{\veps_n\}\to d(x,y)$. Por tanto, por la caracterización del ínfimo con sucesiones, tenemos que
    $$d(x,y)=\inf A = \inf \{\veps\in \bb{R}^+\mid y\in B(x,\veps)\}$$

    Hemos demostrado que, a partir de las bolas abiertas, podemos conocer la distancia entre dos puntos arbitrarios de $E$.\\


    En el caso del espacio normado, tenemos que $\|x\|=d(x,0)=d(0,x)$. Entonces:
    \begin{equation*}
        \|x\|=\inf \{\veps\in \bb{R}^+\mid x\in B(0,\veps)\}=\inf \left\{\veps\in \bb{R}^+\mid \frac{x}{\veps}\in B(0,1)\right\}
    \end{equation*}
    donde la primera desigualdad se debe a lo ya demostrado, y la segunda se debe a que:
    \begin{equation*}
        x\in B(0,\veps)\Longleftrightarrow \|x\|<\veps \Longleftrightarrow \left\|\frac{x}{\veps}\right\|<1 \Longleftrightarrow \frac{x}{\veps}\in B(0,1)
    \end{equation*}
\end{ejercicio}


\begin{ejercicio}\label{Ej:Tema2_2}
    Sea $X$ un espacio normado, $x,y\in X$ y $r,\rho\in \bb{R}^+$. Probar las siguientes afirmaciones. ¿Son ciertos los resultados análogos en un espacio métrico cualquiera?
    \begin{enumerate}
        \item $B(x,r)\cap B(y,\rho)\neq \emptyset \Longleftrightarrow ||y-x|| <r+\rho$. \label{Ej:Tema2_2.1}

        Al ser un espacio normado, podemos considerar la distancia correspondiente a la norma: $d(x,y)=||y-x||,~\forall x,y\in X$. Entonces:
        \begin{description}
            \item[$\Longrightarrow)$]
            Sea $z\in B(x,r)\cap B(y,\rho)$. Entonces,
            \begin{equation*}
                ||y-x|| = d(x,y) \leq d(x,z)+ d(y,z)<r + \rho
            \end{equation*}

            \item[$\Longleftarrow)$]
            Suponemos $||y-x||=d(x,y)<r+\rho$, y buscamos $z\in B(x,r)\cap B(y,\rho)$.\\

            \begin{comment}
            Supongamos en primer lugar que $d(x,z)+d(y,z)\geq r+\rho,~\forall z\in X$. Entonces, tomando en particular $z=y$, tenemos que
            \begin{equation*}
                d(x,z)+\cancelto{0}{d(y,z)}=d(x,y)\geq r+\rho \quad \text{Contradicción.}
            \end{equation*}

            Por tanto, tenemos que $\exists z\in X$ tal que $d(x,y)\leq d(x,z)+d(y,z)< r+\rho$. Entonces:
            \begin{gather*}
                d(x,z)+d(y,z)< r+\rho 
            \end{gather*}
            \end{comment}

            Sea $\lambda:=\dfrac{\rho}{r+\rho}\in \bb{R}^+$. Consideramos $z:= \lambda x+(1-\lambda)y$. Veamos que $z$ está en la intersección:
            \begin{multline*}
                d(x,z)=\|z-x\|=\|(1-\lambda)(y-x)\| =|1-\lambda|~\|y-x\| <\\
                < (1-\lambda)(r+\rho) = \left(1-\dfrac{\rho}{r+\rho}\right)(r+\rho)
                = \left(\dfrac{r+\rho-\rho}{r+\rho}\right)(r+\rho)= r \Longrightarrow\\\Longrightarrow z\in B(x,r)
            \end{multline*}
            \begin{multline*}
                d(y,z)=\|z-y\|=\|\lambda(x-y)\| =|\lambda|~\|x-y\| <\\
                < \lambda(r+\rho) = \left(\dfrac{\rho}{r+\rho}\right)(r+\rho)
                = \rho \Longrightarrow z\in B(y,\rho)
            \end{multline*}

            Por tanto, tenemos que $z\in B(x,r)\cap B(y,\rho)$, por lo que la intersección no es nula.\\

            Este resultado no es cierto para un espacio métrico cualquiera. Sea $(X,d_{disc})$. Entonces $B_1\left(0,\frac{3}{4}\right)=\{0\}$, $B_2\left(1, \frac{3}{4}\right)=\{1\}$. Se tiene que la intersección es nula. No obstante, $$d(0, 1)=1<\frac{3}{4} + \frac{3}{4}=\frac{3}{2}$$
        \end{description}

        

        
        \item $B(y,\rho)\subset B(x,r) \Longleftrightarrow ||y-x|| <r-\rho$. \label{Ej:Tema2_2.2}\\

        TERMINAR
    \end{enumerate}
\end{ejercicio}


\begin{ejercicio}
    Dar un ejemplo de una familia numerable de abiertos de $\bb{R}$ cuya intersección no sea un conjunto abierto.\\

    Dado $x\in \bb{R}$, sea la familia $\left\{]x-\frac{1}{n}, x+\frac{1}{n}[\right\}_{n\in \bb{N}}$ que, como $\bb{N}$ es numerable, el conjunto también lo es. Es directo ver que su intersección es $\{x\}$, el cual no es un abierto porque $\nexists \veps\in \bb{R}^+$ tal que $B(x,\veps)=]x-\veps, x+\veps[\subset \{x\}$.
\end{ejercicio}

\begin{ejercicio}
    Si $A$ es un subconjunto no vacío de un espacio métrico $E$ con distancia $d$, se define la \emph{distancia} de un punto $x\in E$ al conjunto $A$ por
    $$d(x,A) = \inf\{ d(x,a) \mid a \in A\}$$

    Probar que $ \ol{A} = \{x \in E \mid d(x,A) = 0\}$.
    \begin{description}
        \item[$\subset$)] Sea $x\in \ol{A}\subset E$. Entonces, $B\left(x,\frac{1}{n}\right)\cap A\neq \emptyset,~\forall n\in \bb{N}$. Entonces, sea $z_n\in A$ tal que
        $$0\leq d\left(x,z_n\right)< \frac{1}{n}$$
        Por el lema del Sándwich, tenemos que $\{d(x,z_n)\}\to 0$. Además, como se tiene que $0\leq d(x,a)~\forall a\in A$, tenemos que $0$ es un minorante del conjunto. Por tanto, por la caracterización del ínfimo con sucesiones\footnote{El ínfimo de un conjunto es el único minorante que es límite de una sucesión de elementos del conjunto.}, tenemos que $$0=\inf\{d(x,a)\mid a\in A\}=d(x,A).$$

        \item[$\supset$)] Sea $x\in E$ tal que $d(x,A)=0=\inf\{d(x,a)\mid a\in A\}$. Entonces, $\exists \{d(x,z_n)\}\to 0$, con $z_n\in A~\forall n\in \bb{N}$. Por tanto, $\exists \{z_n\}$ tal que $\forall \veps\in \bb{R}^+,~\exists m\in \bb{N}$ tal que, tomando $n\geq m$, entonces $d(x,z_n)<\veps$.
        
        Por tanto, $\exists \{z_n\}$ tal que $\forall \veps\in \bb{R}^+,~\exists m\in \bb{N}$ tal que, si $n\geq m$, entonces $z_n\in B(x,\veps)$, $z_n\in A$. Entonces, $A\cap B(x,\veps)\neq \emptyset~\forall \veps\in \bb{R}^+$, y, consecuentemente, $x\in \ol{A}$.
    \end{description}
\end{ejercicio}


\begin{ejercicio}
    Sea $X$ un espacio normado, $x \in X$ y $r \in \bb{R}^+$, probar que
    \begin{enumerate}
        \item $\ol{B(x,r)} = \ol{B}(x,r)$,
        \begin{description}
            \item[$\subset)$] Tenemos que $B(x,r)\subset \ol{B}(x,r)\in \cc{C}$. Entonces, como $\ol{B(x,r)}$ es el mínimo cerrado que contiene a $B(x,r)$, y las bolas cerradas son cerrados, tenemos que $\ol{B(x,r)}\subset \ol{B}(x,r)$.

            \item[$\supset)$] Tenemos que $y\in \ol{B(x,r)}$ si y solo si $\exists \{y_n\}\to y$, con $d(x,y_n)<r~\forall n\in \bb{N}$.

            Sea $y\in \ol{B}(x,r)$, es decir, $d(x,y)\leq r$. Entonces, definimos la sucesión $\{y_n\}=\{y-\frac{1}{n}(y-x)\}$. Tenemos claramente que $\{y_n\}\to y$. Veamos que $d(x,y_n)<r~\forall n\in \bb{N}$:
            \begin{multline*}
                d(x,y_n) = d\left(x,y-\frac{1}{n}(y-x)\right) = \left\|y-x-\frac{1}{n}(y-x)\right\| = \left\|\left(1-\frac{1}{n}\right)(y-x)\right\|
                \leq\\\leq
                \left(1-\frac{1}{n}\right)d(x,y)<d(x,y)\leq r
            \end{multline*}
            Por tanto, tenemos que existe la sucesión buscada, por lo que $y\in \ol{B}(x,r)$.

            \item[$\supset)$] Hemos de demostrar que $\ol{B}(x,r)\subset \ol{B(x,r)}$. Tenemos que
            $$\ol{B}(x,r)=B(x,r)\amalg\footnote{Símbolo de unión de disjunta.}~ S(x,r)$$

            En el caso de $y\in B(x,r)$, tenemos claramente que $B(x,r)\subset \ol{B(x,r)}$.

            En el caso de $y\in S(x,r)$, tenemos que comprobar que $B(y,\veps)\cap B(x,r)\neq \emptyset,~\forall \veps\in \bb{R}^+$. Por el apartado \ref{Ej:Tema2_2.2} del ejercicio \ref{Ej:Tema2_2}, tenemos que esto se da si y solo si $\|x-y\|<r+\veps$, lo cual es cierto ya que $\|x-y\|=r$, por lo que $S(x,r)\subset \ol{B(x,r)}$.\\

            Por tanto, como ambos conjuntos son subconjuntos de $\ol{B(x,r)}$, se tiene de forma directa.
        \end{description}
        
        \item $B(x,r) = [\ol{B}(x,r)]^\circ$
        \begin{description}
            \item[$\subset)$] Como las bolas abiertas son abiertos métricos, tenemos la siguiente igualdad: $B(x,r)=\left[B(x,r)\right]^\circ$. Además, tenemos que:
            \begin{equation*}
                \left[B(x,r)\right]^\circ = \bigcup\{U\in \cc{T}\mid U\subset B(x,r)\}
                \qquad
                \left[\ol{B}(x,r)\right]^\circ = \bigcup\{V\in \cc{T}\mid V\subset \ol{B}(x,r)\}
            \end{equation*}

            Como $B(x,r)\subset \ol{B}(x,r)$, tenemos que $B(x,r)=\left[B(x,r)\right]^\circ\subset \left[\ol{B}(x,r)\right]^\circ$, teniendo por tanto esta inclusión.

            \item[$\supset)$] Sea $y\in \left[\ol{B}(x,r)\right]^\circ$. Entonces, $\exists \veps\in \bb{R}^+$ tal que $B(y,\veps)\subset \ol{B}(x,r)$. Entonces, por el ejercicio \ref{Ej:Tema2_2}, apartado \ref{Ej:Tema2_2.2}, tenemos que $\|x-y\|\leq r-\veps<r$. Por tanto, $d(x,y)<r$, por lo que $y\in B(x,r)$.
        \end{description}
    \end{enumerate}

    Deducir que $\Fr(B(x,r)) = \Fr(\ol{B}(x,r)) = S(x, r)$.
    \begin{gather*}
        \Fr(B(x,r)) = \ol{B(x,r)}\setminus \left[B(x,r)\right]^\circ = \ol{B}(x,r)\setminus B(x,r) = S(x,r) \\
        \Fr(\ol{B}(x,r)) = \ol{\ol{B}(x,r)}\setminus \left[\ol{B}(x,r)\right]^\circ = \ol{B}(x,r)\setminus B(x,r) = S(x,r)
    \end{gather*}
    donde he empleado que $U\in \cc{T}\Longleftrightarrow A=A^\circ$ y $C\in C_\cc{T}\Longleftrightarrow C=\ol{C}$; junto que las bolas abiertas son abiertos métricos, y las bolas cerradas son cerrados métricos.\\    
    
    ¿Son ciertos estos resultados en un espacio métrico cualquiera?

    TERMINAR
\end{ejercicio}

\begin{ejercicio}
    Para un intervalo $J \subset \bb{R}$, calcular los conjuntos $J^\circ,~\ol{J},~J'$ y $\Fr J$ .\\

    Sean $a,b\in \bb{R},~a<b$. Calculamos en primer lugar el interior para los distintos intervalos:
    \begin{equation*}\begin{split}
        ]a,b[ &= \left[a,b\right]^\circ = \left[a,b\right[^\circ= \left]a,b\right]^\circ= \left]a,b\right[^\circ \\
        ]-\infty, b[ &= \left]-\infty, b\right]^\circ = \left]-\infty,b \right[^\circ\\
        ]a, +\infty[ &= \left[a,+\infty\right[^\circ = \left]a,+\infty\right[^\circ \\
        \bb{R}&=\left[\bb{R}\right]^\circ
    \end{split}\end{equation*}


    Calculamos ahora el cierre de los distintos intervalos:
    \begin{equation*}\begin{split}
        [a,b] &= \ol{\left[a,b\right]} = \ol{\left[a,b\right[}= \ol{\left]a,b\right]}= \ol{\left]a,b\right[} \\
        ]-\infty, b] &= \ol{\left]-\infty, b\right]} = \ol{\left]-\infty,b \right[}\\
        [a, +\infty[ &= \ol{\left[a,+\infty\right[} = \ol{\left]a,+\infty\right[} \\
        \bb{R}&=\ol{\bb{R}}
    \end{split}\end{equation*}

    En este caso, al tratarse de intervalos tenemos que $J'=J^\circ$ para todo intervalo $J$. Veamos el valor de la frontera:
    \begin{equation*}\begin{split}
        \{a,b\} &= \Fr{\left[a,b\right]} = \Fr{\left[a,b\right[}= \Fr{\left]a,b\right]}= \Fr{\left]a,b\right[} \\
        \{b\} &= \Fr{\left]-\infty, b\right]} = \Fr{\left]-\infty,b \right[}\\
        \{a\} &= \Fr{\left[a,+\infty\right[} = \Fr{\left]a,+\infty\right[} \\
        \emptyset&=\Fr \bb{R}
    \end{split}\end{equation*}
\end{ejercicio}

\begin{ejercicio}
    En el espacio métrico $\bb{R}$ y para cada uno de los conjuntos $\bb{N}, \bb{Z}, \bb{Q}$ y $\bb{R}\setminus \bb{Q}$, calcular su interior y su cierre, sus puntos de acumulación, sus puntos aislados y su frontera.\\

    \textbf{Números naturales $\bb{N}$:}
    \begin{enumerate}
        \item $\bb{N}^\circ = \emptyset,$ ya que $\nexists \veps\in \bb{R}^+\mid B(x,\veps)=]x-\veps, x+\veps[~\subset \bb{N}$.
        \item $\ol{\bb{N}} = \bb{N},$ ya que para $0<\veps\leq 1$, $B(x,\veps)\cap \bb{N}\neq \emptyset \Longleftrightarrow x\in \bb{N}$
        \item ${\bb{N}}' = \emptyset,$ ya que si $0<\veps\leq 1\in \bb{R}^+$, entonces $B(x,\veps)\cap \bb{N}\setminus\{x\} = \emptyset$.
        \item $\ol{\bb{N}}\setminus {\bb{N}}' =\bb{N}$.
        \item $\Fr(\bb{N})=\ol{\bb{N}}\setminus \bb{N}^\circ=\bb{N}$.
    \end{enumerate}

    \textbf{Números enteros $\bb{Z}$:}
    \begin{enumerate}
        \item $\bb{Z}^\circ = \emptyset,$ ya que $\nexists \veps\in \bb{R}^+\mid B(x,\veps)=]x-\veps, x+\veps[~\subset \bb{Z}$.
        \item $\ol{\bb{Z}} = \bb{Z},$ ya que para $0<\veps\leq 1$, $B(x,\veps)\cap \bb{Z}\neq \emptyset \Longleftrightarrow x\in \bb{Z}$
        \item ${\bb{Z}}' = \emptyset,$ ya que si $0<\veps\leq 1\in \bb{R}^+$, entonces $B(x,\veps)\cap \bb{Z}\setminus\{x\} = \emptyset$.
        \item $\ol{\bb{Z}}\setminus {\bb{Z}}' =\bb{Z}$.
        \item $\Fr(\bb{Z})=\ol{\bb{Z}}\setminus \bb{Z}^\circ=\bb{Z}$.
    \end{enumerate}

    \textbf{Números racionales $\bb{Q}$:}
    \begin{enumerate}
        \item $\bb{Q}^\circ = \emptyset,$ ya que $\nexists \veps\in \bb{R}^+\mid B(x,\veps)=]x-\veps, x+\veps[~\subset \bb{Q}$.
        \item $\ol{\bb{Q}} = \bb{R},$ ya que $B(x,\veps)\cap \bb{Q} = ]x-\veps, x+\veps[\cap \bb{Q}\neq \emptyset ~\forall \veps\in \bb{R}^+ \Longleftrightarrow x\in \bb{R}$, ya que entre dos reales hay infinidad de racionales.
        \item ${\bb{Q}}' = \bb{R},$ ya que $B(x,\veps)\cap \bb{Q}\setminus\{x\} \neq \emptyset \quad \forall \veps\in \bb{R}^+, x\in \bb{Q}$.
        \item $\ol{\bb{Q}}\setminus {\bb{Q}}' =\emptyset$.
        \item $\Fr(\bb{Q})=\ol{\bb{Q}}\setminus \bb{Q}^\circ=\bb{R}$.
    \end{enumerate}

    \textbf{Números irracionales $\bb{R}\setminus \bb{Q}$:}
    \begin{enumerate}
        \item $\left[\bb{R}\setminus \bb{Q}\right]^\circ = \emptyset,$ ya que $\nexists \veps\in \bb{R}^+\mid B(x,\veps)=]x-\veps, x+\veps[~\subset \left[\bb{R}\setminus \bb{Q}\right]$.
        
        \item $\ol{\left[\bb{R}\setminus \bb{Q}\right]} = \bb{R},$ ya
        que $B(x,\veps)\cap \left[\bb{R}\setminus \bb{Q}\right] = ]x-\veps, x+\veps[\cap \left[\bb{R}\setminus \bb{Q}\right]\neq \emptyset ~\forall \veps\in \bb{R}^+ \Longleftrightarrow x\in \bb{R}$, ya que entre dos reales hay infinidad de irracionales.
        
        \item $\left[\bb{R}\setminus \bb{Q}\right]' = \bb{R},$ ya que $B(x,\veps)\cap \left[\bb{R}\setminus \bb{Q}\right]\setminus\{x\} \neq \emptyset \quad \forall \veps\in \bb{R}^+, x\in \bb{Q}$.
        
        \item $\ol{\left[\bb{R}\setminus \bb{Q}\right]}\setminus \left[\bb{R}\setminus \bb{Q}\right]' =\emptyset$.
        
        \item $\Fr(\left[\bb{R}\setminus \bb{Q}\right])=\ol{\left[\bb{R}\setminus \bb{Q}\right]}\setminus 
        \left[\bb{R}\setminus \bb{Q}\right]^\circ=\bb{R}$.
    \end{enumerate}
\end{ejercicio}




\begin{ejercicio}
    Si un subconjunto $A$ de un espacio métrico $E$ verifica que $A'=\emptyset$, probar que la topología inducida por $E$ en $A$ es la discreta. ¿Es cierto el recíproco?\\

    Recordamos que $\cc{T}_A=\{U\cap A\mid U\in \cc{T}\}$. Además, tenemos que el conjunto de puntos aislados de un espacio métrico es $\ol{A}\setminus A'$. Como $A'=\emptyset$, tenemos que el conjunto de puntos aislados es $\ol{A}\supset A$, por lo que todo punto de $A$ es aislado.\\

    Por tanto, $\forall x\in A,~\exists \veps\in \bb{R}^+$ tal que $A\cap B(x,\veps)=\{x\}$. Entonces, tenemos que $A\cap B_A(x,\veps)=\{x\}$. Tenemos que $A\in \cc{T}_A$ y $B_A(x,\veps)\in \cc{T}_A$, por lo que $\{x\}\in \cc{T}_A$ por ser la intersección de dos abiertos. Además, como la unión arbitraria de abiertos es abierta, tenemos que, $\cc{T}_{disc}=\cc{P}(A)\subset \cc{T}_A$.\\

    Además, trivialmente se tiene que $\cc{T}_A\subset \cc{P}(A)=\cc{T}_{disc}$. Por doble inclusión, se tiene que $\cc{T}_A=\cc{T}_{disc}$.\\

    El recíproco indica que, dado $A\subset E$, con $\cc{T}_A=\cc{T}_{disc}\Longrightarrow A'=\emptyset$. Veamos que no es cierto. Sea $E=\bb{R}$ espacio métrico, y consideremos $A=\left\{\frac{1}{n},~n\in \bb{N}\right\}\subset \bb{R}$.

    Para ver que $\cc{T}_{disc}\subset \cc{T}_A$, veamos que $\left\{\frac{1}{n}\right\}\in \cc{T}_A~\forall \frac{1}{n}\in A$, es decir, $\exists \veps\in \bb{R}^+$ tal que $B\left(\frac{1}{n},\veps\right)\cap A=\left\{\frac{1}{n}\right\}$. Para $\veps\leq \frac{1}{n-1}-\frac{1}{n}$, tenemos que:
    \begin{gather*}
        d\left(\frac{1}{n},\frac{1}{n-1}\right) = \frac{1}{n-1}-\frac{1}{n} \geq \veps \\
        d\left(\frac{1}{n},\frac{1}{n+1}\right) = \frac{1}{n}-\frac{1}{n+1} > \frac{1}{n} - \frac{1}{n-1} \geq \veps
    \end{gather*}
    Por tanto, tenemos que para $\veps\leq \frac{1}{n-1}-\frac{1}{n}$ se da. Por tanto, $\left\{x\right\}\in \cc{T}_A~\forall x\in A$. Como las uniones arbitrarias de abiertos es abierto, tenemos que $\cc{T}_{disc}=\cc{P}(A)\subset \cc{T}_{A}$. La otra inclusión es trivial, por lo que tenemos que ambas topologías son iguales.\\

    Veamos ahora que $A'\neq \emptyset$, ya que $0\in A'$. Como $\left\{\frac{1}{n}\right\}\to 0$, por definición de convergencia en $\bb{R}$ tenemos que $\forall \veps\in \bb{R}^+,~\exists n_0\in \bb{N}$ tal que si $n\in \bb{N},~n\geq n_0$, entonces $d\left(\frac{1}{n},0\right)<\veps$. Entonces, tenemos que $\forall \veps\in \bb{R}^+$ se tiene que $$B(0,\veps)\cap (A\setminus \{0\})=B(0,\veps)\cap A = \left\{\frac{1}{n},~n\geq n_0\right\}\neq \emptyset\Longrightarrow 0\in A'.$$
\end{ejercicio}

\begin{ejercicio}
    Sean $\{x_n\}$ e $\{y_n\}$ sucesiones convergentes en un espacio métrico $E$ con distancia $d$. Probar que la sucesión $\{d(x_n, y_n)\}$ es convergente y calcular su límite.\\
    
    Por ser ambas sucesiones convergentes, tenemos que:
    \begin{gather*}
        \{x_n\} \to x \Longrightarrow d(x_n, x)\to 0 \\
        \{y_n\} \to y \Longrightarrow d(y_n, y)\to 0
    \end{gather*}

    Además, aplicando las propiedades de la distancia, tenemos que:
    \begin{equation*}
        0\leq |d(x_n,x)-d(x,y)|\leq d(x_n,y)\leq d(x_n,x) + d(x,y) \qquad \forall n\in \bb{N}
    \end{equation*}
    Tomando límites y por el Lema del Sándwich, tenemos que $\{d(x_n,y)\}\to d(x,y)$.

    Vemos ahora lo siguiente:
    \begin{equation*}
        0\leq |d(x_n,y)-d(y_n,y)|\leq d(x_n,y_n)\leq d(x_n,y) + d(y,y_n) \qquad \forall n\in \bb{N}
    \end{equation*}
    Tomando límites, por el Lema del Sándwich y sabiendo que $\{d(x_n,y)\}\to d(x,y)$, tenemos que $$\{d(x_n,y_n)\}\to d(x,y).$$
\end{ejercicio}


\begin{ejercicio}
    Sea $E = \prod\limits_{k=1}^N E_k$ un producto de espacios métricos y $A =\prod\limits_{k=1}^N A_k\subset~E$, donde $A_k \subset E_k$ para todo $k \in \Delta_N$ . Probar que $A^\circ =\prod\limits_{k=1}^N A_k^\circ$ y $\ol{A} =\prod\limits_{k=1}^N \ol{A_k}$. Deducir que $A$ es un abierto de $E$ si, y sólo si, $A_k$ es un abierto de $E_k$ para todo $k \in \Delta_N$ , mientras que $A$ es un cerrado de $E$ si, y sólo si, $A_k$ es un cerrado de $E_k$ para todo $k \in \Delta_N$.

    TERMINAR


    \begin{equation*}\begin{split}
        x\in \ol{A} &\Longleftrightarrow \exists \{x_n\}\to x, \text{ con }x_n\in A=\prod\limits_{k=1}^N A_k \Longleftrightarrow \\
        & \Longleftrightarrow  \exists \{x_n(k)\}\to x(k), \text{ con }x_n(k)\in A_k\Longleftrightarrow \\
        & \Longleftrightarrow x(k)\in \ol{A_k}
    \end{split}\end{equation*}
\end{ejercicio}
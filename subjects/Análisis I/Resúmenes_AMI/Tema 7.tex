\chapter{Vector derivada}
\section{Definición del vector derivada}

\begin{definicion}[El espacio normado $L(\R,Y)$] 
    \ \newline
     Sea $Y$ espacio normado, $\Phi:L(\R,Y)\flecha Y$, $\Phi(T) = T(1)$ tenemos que $\Phi$ es lineal, biyectiva y preserva la norma, luego el espacio $L(\R,Y)$ se identifica totalmente con Y.
\end{definicion}

\begin{notacion}
    Sea $\emptyset \neq \Omega = \Omega^o \subset \R$, $Y$ espacio normado, $f:\Omega \flecha Y$
\end{notacion}

\begin{definicion}[Vector derivada]
    \ \newline
    $f$ es derivable en un punto $a\in \Omega$ cuando $t \longmapsto \dfrac{f(t)-f(a)}{t-a}$, de $\Omega\setminus \{a\}$ en $Y$ tiene límite en el punto $a$:
    $$f'(a) := \lim_{t\to a} \dfrac{f(t)-f(a)}{t-a}$$
    Este es el vector derivada de $f$ en $a$. \newline
    $f$ es derivable cuando lo es en todo punto. 
    $$f':\Omega\flecha Y, \ x \longmapsto f'(x)$$
    es la función derivada de $f$
\end{definicion}

\begin{prop}[Equivalencia entre diferenciabilidad y derivabilidad]
    \ \newline
    \begin{itemize}
        \item f diferenciable en $a\in \Omega \sii $ f derivable en a
        $$f'(a) = Df(a)(1) \ \ Df(a)(t)=f'(a)t$$
        \item f diferenciable $\sii$ f derivable
        \item $f\in C^1 \sii $ f derivable y f' continua.
        \item La distinción entre diferenciabilidad y derivabilidad es cuestión de un matiz.
    \end{itemize}
    \ \newline
\end{prop}

\begin{prop}
    Si f derivable en $a\in\Omega$, $\forall \varepsilon>0$ $\exists \delta>0 : $
    $$t_1, t_2 \in \Omega, \ t_1 \neq t_2 :\ a-\delta < t_1 \leq a \leq t_2 < a+\delta \ \then$$
    $$\left \|\dfrac{f(t_2)-f(t_1)}{t_2-t_1}-f'(a)\right \|\leq \varepsilon$$
\end{prop}

\begin{notacion}
    A partir de ahora $Y=\rm$
\end{notacion}

\begin{prop}
    $f=(f_1,...,f_M): \Omega \flecha \rm$ 
    $$f \ derivable \ en \ a\in\Omega \ \sii \ f_j \ derivable \ en \ a \ \forall j\in\Delta_M$$
    $$f'(a)=\left (f_1'(a),...,f_M'(a)\right)$$
\end{prop}


\chapter{Continuidad y límite funcional}
\section{Continuidad}

\begin{definicion}
    Se dice que una función $f:E \longrightarrow F$ es continua en un punto cuando:
    $$V \in \cc{U}(f(x)) \\ \implies \\ f^{-1}(V) \in \cc{U}(x)$$
\end{definicion}

\begin{prop}[Caracterizacion]
    Para $f:E \longrightarrow F$, $x\in E$, son equivalentes las siguientes afirmaciones:\newline
    (i)  $f$ es continua en el punto $x$\newline
    (ii)  $\forall \varepsilon > 0$  $\exists \delta > 0$  / $f(B(x,\delta)) \subset B(f(x), \varepsilon)$\newline
    (iii)  Sea ${x_n} \subset E$, $\xnx  \implies \{f(x_n)\} \longrightarrow f(x)$
\end{prop}

\begin{prop}[Caracter local]\\
    $$\fEF, \\ \emptyset \neq A \subset E, \\ x \in A$$
    \begin{itemize}
        \item $f$ continua en $x$ $\implies$ $f_{|A}$ continua en x
        \item $f_{|A}$ continua en x, $A \in \cc{U}(x)$ $\implies$ $f$ continua en $x$
    \end{itemize}
\end{prop}

\begin{definicion}
    Se dice que $f$ es continua en $A$, cuando es continua en todos los puntos de $A$. Si $f$ es continua en todo $E$ se dice simplemente que $f$ es continua.
\end{definicion}

\begin{prop}[Caracterización]
    Para $\fEF$ las siguientes afirmaciones son equivalentes:\newline
    (i)  $f$ es continua \newline
    (ii)  La preimagen de todo abierto es abierta. \newline
    (iii)  La preimagen de todo cerrado es cerrada. \newline
    (iv)  Para toda sucesión convergente $\xn$ de puntos de $E$, la sucesión $\{f(x_n)\}$ es convergente.
\end{prop}

\begin{prop}[Caracter local de la continuidad global] \\
    \begin{itemize}
        \item $A = A^o$ $$f \\ continua \\ en \\ A \sii f_{|A} \\ continua$$
        \item $E = U \cup V$ donde $U = U^o$, $V = V^o$ 
        $$f \\ continua \sii f_{|U} \\ y \\ f_{|V} \\ continuas$$
        \item $f$ continua $\sii$ $\forall x \in E\\ \exists U \in \cc{U}(x) : f_{|U}$ continua.
    \end{itemize}
\end{prop}

\section{Límite funcional}
\begin{definicion}
    $A \subset E$, $f:A \longrightarrow F$, $\alpha \in A'$.
    Se dice que $f$ tiene \textbf{límite} en el punto $\alpha$ cuando $\exists L \in F$ /\\
    $$\forall \varepsilon >0 \\ \exists \delta > 0 \\ / \\ x \in A, 0 < d(x,\delta) \implies d (f(x), L) < \varepsilon$$
    Además el $L$ es único y decimos que es el límite de $f$ en $\alpha$:
    $$L = \lim_{x\to \alpha} f(x)$$
\end{definicion}

\begin{prop}[Caracterización]
    Las siguientes afirmaciones son equivalentes:\\
    (i) $L = \lim\limits_{x \to \alpha} f(x)$ \newline
    (ii) $\forall V \in \cc{U}(L) \\ \exists U \in \cc{U}(\alpha) \\ /\\ f(U\cap (A\setminus \{\alpha\})) \subset V$\newline
    (iii) $\xn \subset A\setminus \{\alpha\}, \\ \{x_n\} \flecha \alpha \\ \implies \\ \{f(x_n)\} \flecha L$
\end{prop}

\begin{prop}[Caracter local]
    $B = \{ x\in A : 0 < d(x,\alpha) < r\}, \\ \alpha \in B'\implies$
    $$\lim\limits_{x \to \alpha} f(x) = L \sii \lim\limits_{x \to \alpha} f_{|B}(x) = L$$
\end{prop}

\begin{prop}[Relación con la continuidad] \\
    \begin{itemize}
        \item Si $a\in A\setminus A' \\ \implies$ $f$ continua en $a$
        \item Si $a\in A\cap A' \\ \implies$ $f$ continua en $a$ $\sii$ $\lim\limits_{x \to a} f(x) = f(a)$
        \item Si $\alpha\in A'\setminus A \then f$ continua $\sii \exists g:A\cup \{\alpha\} \flecha F, \\ continua \\ en \\ \alpha$ con $g(x) = f(x) \\ \forall x \in A \\ \then g$ es única y $g(\alpha) = \lim\limits_{x\to \alpha} f(x)$ 
    \end{itemize}
\end{prop}


\section{Composición de funciones}
\begin{prop}[Continuidad de la composición]
    Sean $G,E,F$ espacios métricos. $\varphi:G\flecha E$, $\fEF$, $f\circ\varphi:G\flecha F$ 
    $$\varphi \\ continua \\ z\in G \\ \land  \\ f \\ continua \\ x = \varphi (z) \\ \then \\ f\circ\varphi \\ continua \\ z$$
    $$\varphi, f \\ continuas \then f\circ \varphi \\ continua$$
\end{prop}


\begin{prop}[Cambio de variable para calcular un límite]
    $f:A\flecha F$, $A\subset E$, $\varphi :T\flecha E$, $T \subset G$\\
    $z \in T', \alpha\in E$. Si se verifica:
    $$\lim_{t\to z}\varphi(t) = \alpha \\ \land \\ \varphi(t) \in A\setminus \{\alpha\}\\ \forall t \in T\setminus \{z\} \then$$ 
    $$\alpha\in A' \\ \land \\ \lim_{x\to \alpha}f(x) = L \\ \then \\ \lim_{t\to z} f(\varphi(t)) = L$$
\end{prop}

\section{Ejemplos de funciones continuas}
\begin{itemize}
    \item $f$ constante $\then$ $f$ continua
    \item $\emptyset \neq F \subset \R$, $F^o = \emptyset$, $f:\R \flecha F$ continua $\then$ $f$ constante
    \item La función inclusión es continua
    \item La función identidad es continua
    \item La función distancia es continua
    \item La norma, la suma y el producto por escalares en un espacio normado son funciones continuas.
\end{itemize}

\begin{definicion}[Proyecciones y componentes]
    Sea $F=F_1\times...\times F_M \neq \emptyset$ 
    \begin{itemize}
        \item Proyecciones coordenadas:
        $\pi_k:F\flecha F_k, \\ \pi_k(y) = y(k) \\ \forall y \in F, \\ k \in \Delta_M$
        \item Componentes de $f$:
        $\fEF, \\ f_k=\pi_k\circ f:E\flecha F_k\\ \forall k \in \Delta_M$
    \end{itemize}
\end{definicion}

\begin{prop}[Caracterización de la continuidad y límite funcional]\\
    \begin{itemize}
        \item Si $F=F_1\times...\times F_M$ es un producto de espacios métricos $\then  \pi_k$ es continua $\forall k \in \Delta_M$
        \item $\fEF$ $\then$ $f$ continua en $x\in E$ $\sii$ $f_k$ continua en $x$ $\forall k \in \Delta_M$
        \item $f:A\flecha F$, $\alpha \in A'$, $y\in F$ $\then$ 
        $$\lim_{x\to\alpha} f(x) = y \\ \sii \\ \lim_{x\to\alpha} f_k(x) = y(k)\\  \forall k \in \Delta_M$$
    \end{itemize}
\end{prop}

\begin{definicion}
    $\cc{F}(E,Y)$ conjunto de todas las funciones de $E$ en $Y$, $\cc{F}(E) = \cc{F}(E,\R)$, 
    $Y$ espacio vectorial, $f,g \in \cc{F}(E,Y)$, $\lambda \in \R$, $\Lambda \in \cc{F}(E)$
    \begin{itemize}
        \item Suma: $(f+g)(x) = f(x) + g(x)\\ \forall x \in E$
        \item Producto: $(\Lambda g)(x) = \Lambda(x) g(x) \\\forall x \in E$
        \item Producto por escalares: $(\lambda g)(x) = \lambda g(x) \\\forall x \in E$ \newline\newline
        Si $f, g \in \cc{F}(E), g(x) \neq 0 \\\forall x \in E$
        \item Cociente: $(\frac{f}{g})(x) = \frac{f(x)}{g(x)} \\\forall x \in E$
    \end{itemize}

    Así. $\cc{F}(E,Y)$ es un \textbf{espacio vectorial} y $\cc{F}(E)$ es un \textbf{anillo conmutativo} con unidad.
\end{definicion}

\begin{prop}[Preservación de la continuidad]
    En las condiciones de la definición anterior:
    \begin{itemize}
        \item[(i)] $f,g,\Lambda \\ continuas \\ x \implies f+g, \Lambda g \\ continuas \\ x$
        \item[(ii)] $f,g$ continuas $x$, $g(E) \subset \R ^*$ $\implies \frac{f}{g}$ continua en $x$ 
    \end{itemize}
\end{prop}

\begin{definicion}[Espacio de funciones continuas]
    $\cc{C}(E,Y)$ conjunto de todas las funciones continuas de $E$ en $Y$. 
    $$\cc{C}(E) = \cc{C}(E,\R)$$
    $\cc{C}(E,Y)$ es subespacio vectorial de $\cc{F}(E,Y)$, $\cc{C}(E)$ subanillo y subespacio vectorial de $\cc{F}(E)$
\end{definicion}

\begin{prop}[Calculo de limites] \\
    \begin{enumerate}
        \item El límite de la suma es la suma de los límites.
        \item El límite del producto es el producto de los límites.
        \item El límite del cociente es el cociente de los límites.
    \end{enumerate}
\end{prop}

\begin{definicion}[Campo escalar]
    $f:A\flecha \R$ donde $\emptyset \neq A \subset \rn$
\end{definicion}

\begin{definicion}[Campo vectorial]
    $f:A\flecha \rm$ donde $\emptyset \neq A \subset \rn$
\end{definicion}



Merece la pena también destacar las funciones polinómicas y las racionales (cociente de polinomios donde el denominador no se anula) como funciones continuas. \newline
A continuación podemos ver las relaciones que tienen los distintos espacios:
$$\cc{P}(E,Y) \subset \cc{R}(E,Y) \subset \cc{C}(E,Y) \subset \cc{F}(E,Y)$$
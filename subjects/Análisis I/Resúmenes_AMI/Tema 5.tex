\chapter{Complitud y continuidad uniforme}
\section{Complitud}

\begin{definicion}[Sucesiones de Cauchy]
    $E$ espacio métrico con distancia $d$. $\xn \subset E$ es una sucesión de Cauchy cuando:
    $$\forall \varepsilon > 0 \\ \exists m \in \N : p,q \geq m \implies d(x_p,x_q) < \varepsilon$$
    Toda sucesión convergente es una sucesión de Cauchy.\newline
    No es una propiedad topológica.
\end{definicion}

\begin{definicion}[Espacios completos]
    Un espacio métrico $E$ es completo, o su distancia $d$ es completa cuando toda sucesión de Cauchy es convergente, (a un punto de $E$).
\end{definicion}

\begin{ejemplo}
    Espacio de Banach = espacio normado completo.
\end{ejemplo}

\begin{ejemplo}
    Espacio de Hilbert = espacio pre-hilbertiano completo.
\end{ejemplo}

\begin{prop}
    Dos normas equivalentes dan lugar a las mismas sucesiones de Cauchy. 
    Toda norma equivalente a una completa es completa.
\end{prop}

\begin{teo}[Complitud de $\rn$] \\
    \begin{itemize}
        \item[(i)] Todo espacio normado de dimensión finita es de Banach.
        \item[(ii)] El espacio euclídeo N-dimensional es de Hilbert.
    \end{itemize}
\end{teo}

\begin{prop}[Subespacios métricos completos]
    $E$ espacio métrico, $A$ subespacio métrico de $E$:
    \begin{itemize}
        \item $A$ completo $\implies$ $A = \overline{A}$ en $E$
        \item $E$ completo, $A = \overline{A}$ en $E$ $\implies$ $A$ completo
        \item $E$ completo $\implies$ los subconjuntos completos de $E$ son los cerrados
        \item $A\subset \rn$ completo $\sii$ $A = \overline{A}$ en $\rn$
    \end{itemize}
\end{prop}

\section{Continuidad uniforme}

\begin{definicion}[Funciones uniformemente continuas]
    Sean $E,F$ espacios métricos, $\fEF$ es uniformemente continua cuando:
    $$\forall \varepsilon > 0 \\\exists \delta >0: x,y \in E, d(x,y) > \delta \implies d(f(x),f(y)) < \varepsilon$$
\end{definicion}

\begin{prop}[Caracterización] \\
    \begin{itemize}
        \item Si f uniformemente continua $\then$
        $$\xn, \{y_n\} \subset E, \{d(x_n,y_n)\} \flecha 0 \implies \{d(f(x_n),f(y_n))\} \flecha 0$$
        \item Si f no es uniformemente continua $\then$
        $existen \\ \xn, \{y_n\} \subset E, \varepsilon > 0 :$
        $$d (x_n,y_n) < \frac{1}{n} \\\forall n \in \N \\\land\\ d(f(x_n),f(y_n)) \geq \varepsilon \\\forall n \in \N$$
    \end{itemize}    
\end{prop}

\begin{teo}[Heine]
    Sean E,F espacios métricos, $\fEF$ continua.
    $$E \\ compacto \then f \\ uniformemente \\ continua$$
\end{teo}

\begin{observacion}\\
    \begin{itemize}
        \item No es una propiedad local
        \item No es una propiedad topológica.
        \item Se conserva en espacios normados con normas equivalentes.
    \end{itemize}
\end{observacion}

\begin{definicion}[Funciones lipschitzianas]
    $E,F$ espacios métricos, $\fEF$ es lipschitziana cuando $\exists M\geq 0:$ 
    $$d(f(x),f(y)) \leq Md(x,y) \\\forall x,y \in E$$
    Se dice que $f$ es lipschitziana con constante $M$. 
    Toda función lipschitziana es uniformemente continua.
\end{definicion}

\section{Teorema del punto fijo}

\begin{definicion}[Constante de Lipschitz]
    $E,F$ espacios métricos, $\fEF$ lipschitziana:
    $$M_0 = \sup\{\frac{d(f(x),f(y))}{d(x,y)} : x,y \in E, x\neq y\}$$
\end{definicion}

\begin{definicion}[Funcion no expansiva]
    $$f\\no\\expansiva \sii M_0 \leq 1$$
\end{definicion}

\begin{definicion}[Funcion contractiva]
    $$f\\contractiva \sii f \\ lipschitziana \\ con \\ M < 1 \sii M_0 < 1$$
\end{definicion}

\begin{teo}[del punto fijo]
    Sea E un espacio métrico completo y $\fEF$ contractiva $\then$ f tiene un único punto fijo,  $\exists! x \in E : f(x) = x$
\end{teo}

\section{Aplicaciones lineales}

\begin{definicion}[Aplicaciones lineales]
    Una aplicacion $T:X\flecha Y$ se dice lineal si cumple:
    \begin{enumerate}
        \item $T(u+v) = T(u) + T(v) \\\forall u,v \in X$
        \item $T(\lambda x) = \lambda T(x) \\\forall x \in X, \lambda \in \R$
    \end{enumerate}
\end{definicion}

\begin{prop}
    X,Y espacios normados, $T:X\flecha Y$ lineal. Son equivalentes:
    \begin{enumerate}
        \item T continua.
        \item $\exists M\geq 0 : ||T(x)|| \leq M ||x|| \\\forall x\in X$
    \end{enumerate}
\end{prop}

\begin{observacion}
    Si $T$ es continua en $x_0$ es continua en todo $X$.
\end{observacion}

\begin{observacion}
    $T$ continua $\sii$ $T$ uniformemente continua $\sii$ $T$ lipschitziana.
\end{observacion}

\begin{prop}
    X espacio normado de dimensión finita, Y espacio normado. Por el teorema de Hausdorff:
    $$T:X\flecha Y \\ lineal \then T \\ continua$$
\end{prop}

\begin{definicion}[Espacio de aplicaciones lineales continuas]
    $X,Y$ espacios normados. $L(X,Y)$ conjunto de todas las aplicaciones lineales continuas de $X$ en $Y$. Es un subespacio vectorial de $\cc{C}(X,Y)$
\end{definicion}

\begin{definicion}[Norma de una aplicación lineal continua]
    Sea $T\in L(X,Y)$ se define $||T|| = M_0$, constante de Lipschitz.
\end{definicion}
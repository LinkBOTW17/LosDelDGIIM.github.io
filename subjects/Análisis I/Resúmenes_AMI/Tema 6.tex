\chapter{Diferenciabilidad}
\section{Motivación}

$$\emptyset \neq A \subset \R, \\f: A \flecha \R, \\a \in A \cap A'$$

$f$ derivable en el punto $a$ cuando $\exists \lambda \in \R:$
$$\lim_{x\to a} \frac{f(x)-f(a)- \lambda (x-a)}{x-a} = 0$$

$\lambda$ es único, lo llamamos derivada de $f$ en el punto $a$ y lo escribimos: $f'(a) = \lambda$

\begin{prop}[El espacio $L(\R, \R)$]
    $\alpha \in \R$ $T_\alpha \in L(\R,\R)$ con $T_\alpha = \alpha x$ $\forall x \in \R$
    Definimos $\Phi: \R \flecha L(\R,\R)$, con $\Phi(\alpha) = T_\alpha$ $\then$ $\Phi$ es lineal, biyectiva y conserva la norma y $\R$ y $L(\R,\R)$ son idénticos como espacios normados.
\end{prop}

\begin{definicion}[Diferencial de una función real de variable real]
    $$\emptyset \neq A \subset \R, \\f: A \flecha \R, \\a \in A \cap A'$$
    $f$ defereciable en el punto $a$ cuando $\exists T \in L(\R,\R):$
    $$\lim_{x\to a} \frac{f(x)-f(a)- T (x-a)}{x-a} = 0$$
    $T$ es único, lo llamamos diferencial de $f$ en el punto $a$ y lo denotamos: $Df(a)$
\end{definicion}

\begin{prop}[Relación entre derivada y diferencial]
    $$f \\ derivable \\ a \\ \sii \\ f \\ diferenciable \\ a$$
    $$Df(a)(x) = f'(a)x \\\\\\ f'(a) = Df(a)(1)$$
\end{prop}

\section{Funciones diferenciables}
\begin{notacion}
    $X,Y$ espacios normados, $f:A\flecha Y$ y $a\in A^o$
\end{notacion}

\begin{definicion}[Función diferenciable en un punto]
    $f$ defereciable en el punto $a$ cuando $\exists T \in L(\R,\R):$
    $$\lim_{x\to a} \frac{f(x)-f(a)- T (x-a)}{||x-a||} = 0$$ o bien
    $$\lim_{x\to a} \frac{||f(x)-f(a)- T (x-a)||}{||x-a||} = 0$$
\end{definicion}

\subsection{Observaciones importantes}

\begin{prop}[Unicidad]
    Si $f$ es diferenciable en $a$, la aplicación $T\in L(\R,\R)$ es única y la llamamos diferencial de f en a y se denota por Df(a)
\end{prop}

\begin{prop}[Relación con la continuidad]
    Si f diferenciable en a $\then $ f continua en a 
\end{prop}

\begin{prop}[Significado analítico]
    f diferenciable en a, $g:X \flecha Y:$ 
    $$g(x) = f(a) + Df(a)(x-a) = f(a) - Df(a)(a) + Df(a)(x) \\\forall x \in X$$
    g es una función afín y continua, tal que:
    $$\lim_{x\to a}\frac{f(x)-g(x)}{||x-a||} = 0$$
    Entonces g es una "buena aproximación" de f cerca del punto a.
\end{prop}

\begin{prop}[Carácter local]
    Si $U\subset X, a\in U^o \then $ 
    $$f \\ diferenciable \\ a \\ \sii \\ f_{|U} \\ diferenciable\\ a$$
    $$Df(a) = D(f_{|U})(a)$$
\end{prop}

\begin{prop}[Independencia de las normas]
    La existencia o no y la diferencial no cambia cuando cambiamos las normas por otras equivalentes.
\end{prop}

\begin{notacion}
    $X,Y$ espacios normados, $\Omega = \Omega^o \subset X$ y $f:\Omega \flecha Y$
\end{notacion}

\begin{definicion}[Función diferenciable]
    Si $f$ es diferenciable en todo punto decimos que es diferenciable.
\end{definicion}

\begin{definicion}
    $D(\Omega, Y)$ es el conjunto de todas las funciones diferenciables de $\Omega$ en $Y$.
    $$D(\Omega,Y) \subset \cc{C}(\Omega,Y)$$
\end{definicion}

\begin{definicion}[Diferencial de $f$]
    Sea $f\in D(\Omega, Y)$, tenemos la función diferencial de $f$: $Df:\Omega\flecha L(\Omega,Y)$ dada por $x\longmapsto Df(x)$
\end{definicion}

\begin{definicion}
    Decimos que $f$ es de clase $C^1$ cuando $f\in D(\Omega, Y)$ y $Df$ es continua.
\end{definicion}

\begin{definicion}
    $C^1(\Omega,Y)$ es el conjunto de todas las funciones de clase $C^1$
    $$C^1(\Omega,Y) \subset D(\Omega,Y) \subset \cc{C}(\Omega,Y)$$
\end{definicion}

\section{Reglas de diferenciación}

\begin{ejemplo}
    $f:X\flecha Y$ constante $\then$ $f\in C^1(X,Y)$ con $Df(a) = 0\\\forall a \in X$
\end{ejemplo}

\begin{ejemplo}
    $f\in L(X,Y) \then f\in C^1(X,Y)$ con $Df(a) = f\\\forall a \in X$
\end{ejemplo}

\begin{prop}[Linealidad de la diferencial]
    $$f,g \in D(\Omega,Y)\then \alpha f + \beta g \in D(\Omega,Y)$$
    $$f,g \in C^1(\Omega,Y)\then \alpha f + \beta g \in C^1(\Omega,Y)$$
\end{prop}

\begin{teo}[Regla de la cadena]
    $X,Y,Z$ espacios normados, $\Omega = \Omega^o \subset X$, $U = U^o \subset Y$, $f:\Omega\flecha U$, $g:U\flecha Z$.\newline
    Si $f$ diferenciable en $a$ y $g$ es diferenciable en $b=f(a)$ $\then$ $g\circ f$ es diferenciable en a con $D(g\circ f)(a) = Dg(b)\circ Df(a)$
    $$f\in D(\Omega,U),g\in D(U,Z)\then g\circ f \in D(\Omega,Z)$$
    $$f\in C^1(\Omega,U),g\in C^1(U,Z)\then g\circ f \in C^1(\Omega,Z)$$
\end{teo}

\begin{observacion}
    $T\in D(X,Y), S\in D(Y,Z) \then \| S\circ T\| \leq \|S\|\|T\|$
\end{observacion}

\begin{notacion}
    $Y=\prod_{j=1}^M Y_j$ producto de espacios normados, $j\in \Delta_M$
\end{notacion}

\begin{definicion}[Inyección natural]
    $I_j:Y_j\flecha Y$, $I_j(u) = (0,...,\stackrel{(j)}{u},...,0)$, $\|I_j\| = 1$
\end{definicion}

\begin{prop}[Diferenciabilidad con valores en un producto]
    X espacio normado, $\Omega = \Omega^o \subset$ X, f = $(f_1,...,f_M):\Omega\flecha$Y \newline
    $$f \\ diferenciable \\ a \sii f_j \\ diferenciable \\a\\\forall j \in \Delta_M$$
    $$Df(a) = (Df_1(a),...,Df_M(a))$$
    $$f\in D(\Omega,Y)\sii f_j \in D(\Omega,Y) \\\forall j \in \Delta_M$$
    $$f\in C^1(\Omega,Y)\sii f_j \in C^1(\Omega,Y) \\\forall j \in \Delta_M$$
\end{prop}

\begin{prop}[Producto de funciones diferenciables] 
\ 
\newline
    X espacio normado, $\Omega = \Omega^o \subset$ X, $f,g:\Omega \flecha \R$
    $$f,g\in D(\Omega) \then fg\in D(\Omega) \ D(fg) = gDf + fDg$$
    $$f,g \in C^1(\Omega) \then fg\in C^1(\Omega)$$
\end{prop}

\begin{prop}[Cociente de funciones diferenciables] \ \newline
    X espacio normado, $\Omega = \Omega^o \subset$ X, $f,g:\Omega \flecha \R$, $g(\Omega) \subset \R^*$
    $$f,g\in D(\Omega), g(\Omega) \subset \R^* \then f/g\in D(\Omega) \\ D(f/g) = \frac{1}{(g)^2}(gDf - fDg)$$
    $$f,g \in C^1(\Omega), g(\Omega) \subset \R^* \then f/g\in C^1(\Omega)$$
\end{prop}
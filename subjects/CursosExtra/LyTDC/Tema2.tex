\chapter{Lógica de Primer Orden}
Es necesario introducir ahora lenguajes en los que podamos cuantificar cosas. Como primer ejemplo, si sabemos que ``Todo hombre es mortal'' y que ``Sócrates es un hombre'', nos gustaría deducir que, entonces, ``Sócrates es mortal''. Sin embargo, para esto hemos de poder cuantificar, cosa que no es posible con los lenguajes proposicionales pero sí con los lenguajes de primer orden.\\

\noindent
Los lenguajes de primer orden estarán formados por:
\begin{itemize}
    \item Constantes: $c_1,c_2,\ldots, a,b,c,\ldots$
    \item Variables: $x_1,x_2,\ldots, x,y,z,\ldots$
    \item Símbolos de función: $f_1, f_2, \ldots, f,g,h, \ldots$
    \item Símbolos de relación: $R_1,R_2,\ldots,R,S,T,\ldots$
    \item Conectivas lógicas: $\lor$, $\land$, $\lnot$, $\to$, $\leftrightarrow $
    \item Cuantificadores: $\forall $, $\exists $
\end{itemize}

A los conjuntos de todas las constantes, de todas la variables y de todos los símbolos de función los notaremos por $Cons(\mathcal{L}), Var(\mathcal{L}), Fun(\mathcal{L})$, si $\mathcal{L}$ es nuestro lenguaje de primer orden.

\begin{notacion}
    En otros libros o contextos, en vez de denotar a los símbolos de función o variables con una letra que pueda llevar o no superíndice, estos las denotan con un superíndice:
    \begin{itemize}
        \item $f_1^{n_1},f_2^{n_2},\ldots$
        \item $R_1^{m_1},R_2^{m_2},\ldots$
    \end{itemize}
    En este caso, el superíndice indica la ariedad de la función o relación. Por ejemplo, si consideramos $f^3$, tenemos un símbolo de función que se aplica a 3 variables.
\end{notacion}

\begin{definicion}[Término]
    Un término es:
    \begin{enumerate}
        \item Cualquier constante.
        \item Cualquier variable.
        \item Si $t_1,t_2,\ldots,t_n$ son términos y $f$ es un símbolo de función $n-$ario, entonces $f(t_1,t_2,\ldots,t_n)$ es un término.
        \item No hay más términos que los que se puedan obtener siguiendo una secuencia finita de pasos a partir de las enunciadas.
    \end{enumerate}
    Al conjunto de todos los términos de nuestro lenguaje $\mathcal{L}$ lo denotamos por $Term(\mathcal{L})$.
\end{definicion}

\begin{ejemplo}\
    \begin{itemize}
        \item $f(x,f(x,y))$ es un término.
        \item $f(x,f(x))$ no es un término, ya que usamos un mismo símbolo de función, $f$, para denotar dos objetos: una función unaria y una función binaria.
    \end{itemize}
\end{ejemplo}

\begin{definicion}[Fórmulas atómicas]
    Si $t_1,\ldots,t_n$ son términos y $R$ es un símbolo de relación $n-$ario, entonces $R(t_1,\ldots,t_n)$ es una fórmula atómica (o simplemente, un átomo).
\end{definicion}

\begin{definicion}[Fórmulas]
    Son fórmulas:
    \begin{enumerate}
        \item Las fórmulas atómicas.
        \item Si $\varphi$ y $\psi$ son fórmulas, también lo son:
            \begin{equation*}
                \lnot\varphi,\ \varphi\land\psi,\ \varphi\lor\psi,\ \varphi\to\psi,\ \varphi\leftrightarrow\psi
            \end{equation*}
        \item Si $x$ es una variable y $\varphi$ es una fórmula, también lo son: $\forall x\varphi$, $\exists x\varphi$.
        \item No hay más fórmulas que las que se puedan obtener siguiendo una secuencia finita de pasos a partir de las enunciadas.
    \end{enumerate}
    Al conjunto de todas las fórmulas de nuestro lenguaje $\mathcal{L}$ lo denotamos por $Form(\mathcal{L})$.
\end{definicion}

\begin{definicion}
    Una \underline{ocurrencia} de una variable en una fórmula es una aparición de su escritura.
    \begin{itemize}
        \item En la fórmula $\forall x\varphi$, diremos que $\varphi$ es el \underline{radio de acción} de $\forall x$.
        \item En la fórmula $\exists x\varphi$, diremos que $\varphi$ es el \underline{radio de acción} de $\exists x$.
    \end{itemize}
    Diremos que $x$ se encuentra \underline{cuantificada} al ver $\forall x$ o $\exists x$.\newline
    Diremos que una ocurrencia de una variable $x$ es \underline{ligada} si aparece cuantificada o en el radio de acción de $\forall x$ o de $\exists x$.\newline
    Finalmente, diremos que una variable es \underline{libre} si no aparece ligada. Si $\varphi$ es una fórmula en la que las variables $x_1,\ldots,x_n$ aparecen libres, será usual denotar:
    \begin{equation*}
        \varphi(x_1,\ldots,x_n)
    \end{equation*}
    Que no debe confundirse con un término de una función o relación $n-$aria, ya que $\varphi$ no es ni un símbolo de función o relación, sino una fórmula.
\end{definicion}

\begin{ejemplo}
    En la siguiente fórmula:
    \begin{equation*}
        \forall x(\exists yR(x,y)\to Q(y))
    \end{equation*}
    \begin{itemize}
        \item $x$ aparece cuantificada en su primera ocurrencia.
        \item $y$ aparece cuantificada en su primera ocurrencia.
        \item $x$ aparece ligada en su segunda ocurrencia.
        \item $y$ aparece ligada en su segunda ocurrencia.
        \item $y$ aparece como variable libre en su tercera ocurrencia.
    \end{itemize}
\end{ejemplo}

\begin{definicion}[Sentencia]
    Una sentencia es una fórmula sin ocurrencias de variables libres.
\end{definicion}

\section{Semántica}
Trataremos de generalizar el concepto de ``interpretación'', ya visto para lenguajes proposicionales. Para ello, será necesario primero definir los conceptos de ``estructura'' y de ``asignación''.

\begin{definicion}[Estructura]
    Una estructura $\veps$ en un lenguaje $\cc{L}$ es una cuádrupla
    \begin{equation*}
        \veps = (D, \{c_i^\veps\}_{n\in \mathbb{N}}, \{f_i^\veps\}_{n\in \mathbb{N}}, \{R_i^\veps\}_{n\in \mathbb{N}})
    \end{equation*}
    de forma que:
    \begin{itemize}
        \item $D$ es un conjunto no vacío al que llamamos \underline{universo} o \underline{dominio}.
        \item A cada constante $c_i$ de $\mathcal{L}$ le corresponde un elemento $c_i^{\veps}$ de $D$.
        \item A cada símbolo de función $f_i$ de $\mathcal{L}$ le corresponde una función $f_i^\veps:D^n\to D$.
        \item A cada símbolo de relación $R_i$ de $\mathcal{L}$ le corresponde una aplicación ${R_i^\veps:D^m\to\mathbb{Z}_2}$, de forma que $R_i^\veps(c_1^\veps,c_2^\veps)=1$ si $c_1^\veps$ y $c_2^\veps$ están relacionados y 0 en caso contrario.
    \end{itemize}
\end{definicion}

\begin{definicion}[Asignación]
    Una asignación $v$ en $\veps$ es una aplicación ${v:Var(\mathcal{L})\to D}$. Dada una asignación $v$, podremos extenderla a $v':Term(\mathcal{L})\to D$ de la forma:
    \begin{equation*}
        v'(t) = \left\{\begin{array}{ll}
                c^\veps & \text{si\ } t=c \text{\ una constante} \\
                v(x) & \text{si\ } t=x \text{\ una variable} \\
                f^\veps(v'(t_1),\ldots,v'(t_n)) &\text{si\ } t=f(t_1,\ldots,t_n)
        \end{array}\right.
    \end{equation*}
\end{definicion}

\begin{definicion}[Interpretación]
    Una interpretación es una tupla $(\veps,v)$ con $\veps$ una estructura y $v$ una asignación que tiene asociada una aplicación\footnote{A la que próximamente denotaremos simplemente como $I^v$, por simplicidad, entendiendo que la estructura $\veps$ viene dada por el contexto.} $I_\veps^v:Form(\mathcal{L})\to\mathbb{Z}_2$ que cumple para cualesquiera fórmulas $\varphi$ y $\psi$:
    \begin{enumerate}
        \item $I^v(\lnot \varphi) = 1+ I^v(\varphi)$.
        \item $I^v(\varphi\land \psi) = I^v(\varphi)I^v(\psi)$.
        \item $I^v(\varphi\lor \psi) = I^v(\varphi) + I^v(\psi) + I^v(\varphi)I^v(\psi)$.
        \item $I^v(\varphi\to \psi) = 1 + I^v(\varphi) + I^v(\varphi)I^v(\psi)$.
        \item $I^v(\varphi\leftrightarrow \psi) = 1 + I^v(\varphi) + I^v(\psi)$.
        \item $I^v(R(t_1,\ldots,t_n)) = R^{\veps}(v(t_1), \ldots, v(t_n))$

            Con $R$ un símbolo de relación $n-$ario y $t_1,\ldots,t_n$ términos.
        \item $I^v(\forall x\varphi) = \left\{\begin{array}{ll}
                    1 & \text{si para todo\ } a\in D,\ I^{v(x\mid a)}(\varphi) = 1 \\
                    0 & \text{en caso contrario}
        \end{array}\right.$
        \item $I^v(\exists x\varphi) = \left\{\begin{array}{ll}
                    1 & \text{si existe\ } a\in D \text{\ con\ } I^{v(x\mid a)}(\varphi) = 1 \\
                    0 & \text{en caso contrario}
        \end{array}\right.$
    \end{enumerate}
    Siendo:
    \begin{equation*}
        v(x\mid a)(y) = \left\{\begin{array}{ll}
                v(y) & \text{si\ } y\neq x \\
                a & \text{si\ } y=x
        \end{array}\right.
    \end{equation*}
\end{definicion}

% // TODO: Pasar a limpio desde AQUI

\begin{definicion}
    Sea $\varphi\in Form(\cc{L})$:
    \begin{itemize}
        \item Dada una estructura $\veps$, diremos que $\varphi$ es \underline{válida} en $\veps$ si $I^v(\varphi)=1$ para toda asignación $v$ en $\veps$.
        \item Dada una estructura $\veps$, diremos que $\varphi$ es \underline{satisfacible} en $\veps$ si $I^v(\varphi)=1$ para alguna asignación $v$ en $\veps$.
        \item Diremos que $\varphi$ es \underline{universalmente válida} si $\varphi$ es válida en cualquier estructura.
        \item Diremos que $\varphi$ es \underline{satisfacible} si existe una estructura $\veps$ donde $\varphi$ es satisfacible.
        \item Diremos que $\varphi$ es \underline{refutable} si $\lnot\varphi$ es satisfacible.
        \item Diremos que $\varphi$ es una \underline{contradicción} si $\lnot\varphi$ es universalmente válida.
    \end{itemize}
\end{definicion}

\begin{lema}[de Coincidencia]\label{lema:coincidencia}
    Sea $\varphi\in Form(\cc{L})$ de forma que $x_1,\ldots,x_n\in Var(\cc{L})$ son las variables con ocurrencias libres en $\varphi$ y sea $\veps$ una estructura. Entonces, dada una asignación $v$ en $\veps$:
    \begin{equation*}
        I^v(\varphi) = I^{w}(\varphi)
    \end{equation*}
    para toda asignación $w$ en $\veps$ tal que $w(x_i)=v(x_i)$ para todo $i \in \{1,\ldots,n\}$.
\end{lema}

\begin{observacion}
    En particular, si $\varphi$ es una sentencia, entonces $I^v(\varphi)$ no depende de la asignación $v$. Por tanto, si $\varphi$ es satisfacible en $\veps$, entonces $\varphi$ es válida en $\veps$.
\end{observacion}

\begin{definicion}[Consecuencia lógica]
    Sea $\Gamma \cup \{\varphi\}\subseteq Form(\cc{L})$, diremos que $\varphi$ es consecuencia lógica de $\Gamma$, notado por $\Gamma\vDash \varphi$, si para toda interpretación $(\veps,v)$ tal que $I^v(\gamma)=1$ para toda $\gamma\in\Gamma$, fuerza a que $I^v(\varphi)=1$.
\end{definicion}

\begin{teo}[de la Deducción]
    Sea $\Gamma \cup \{\varphi,\psi\}\subseteq Form(\cc{L})$, son equivalentes:
    \begin{enumerate}
        \item $\Gamma\vDash\varphi\to\psi$.
        \item $\Gamma\cup\{\varphi\}\vDash\psi$.
    \end{enumerate}
\end{teo}

\begin{definicion}[Inconsistencia]
    Sea $\Gamma\subseteq Form(\cc{L})$, diremos que $\Gamma$ es inconsistente si no existe una interpretación $(\veps,v)$ tal que $I^v(\gamma)=1$ para toda $\gamma\in\Gamma$.
\end{definicion}

\begin{teo}
    Sea $\Gamma\cup \{\varphi\}\subseteq Form(\cc{L})$, son equivalentes:
    \begin{enumerate}
        \item $\Gamma\vDash\varphi$.
        \item $\Gamma\cup\{\neg \varphi\}$ es inconsistente.
    \end{enumerate}
\end{teo}

\begin{ejemplo}
    Demostremos las siguientes fórmulas universalmente válidas:
    \begin{enumerate}
        \item $\vDash \forall x\varphi(x) \leftrightarrow \forall y\varphi(y)$ con\footnote{Estamos usando una notación que se introduce en la siguiente sección.} $y$ libre para $x$ en $\varphi(x)$.

            Dada cualquier interpretación $(\veps, v)$, queremos ver que:
            \begin{equation*}
                I^v(\forall x\varphi(x)\leftrightarrow \forall y\varphi(y)) = 1
            \end{equation*}
            Y sabemos que eso es equivalente a ver que:
            \begin{equation*}
                I^v(\forall x\varphi(x)) = I^v(\forall y\varphi(y))
            \end{equation*}
            Que se puede ver a partir de su definición:
            \begin{align*}
                I^v(\forall x\varphi(x)) &= \left\{\begin{array}{ll}
                        1 & \text{si\ } I^{v(x\mid a)}(\varphi(x)) = 1 \text{\ para todo\ } a\in D \\
                        0 & \text{en caso contrario} 
                \end{array}\right.  \\
                                         &\AstIg \left\{\begin{array}{ll}
                        1 & \text{si\ } I^{v(y\mid a)}(\varphi(y)) = 1 \text{\ para todo\ } a\in D \\
                        0 & \text{en caso contrario} 
                \end{array}\right. \\
                                         &= I^v(\forall y\varphi(y))
            \end{align*}
            Donde en $(\ast)$ debemos tener cuidado y usar que $y$ es libre para $x$ en $\varphi(x)$, ya que $x$ aparecía libre en $\varphi$ y como $y$ es libre para $x$ en $\varphi(x)$, al hacer la sustitución de $x$ por $y$ no estaremos cambiando variables libres en $\varphi$, por lo que a partir del Lema de Coincidencia (Lema~\ref{lema:coincidencia}), al no cambiar variables libres en $\varphi$, no cambiamos su condición de verdad.
        \item $\vDash \forall x\varphi(x)\to \varphi(t)$ con $t$ libre para $x$ en $\varphi(x)$.

            Por el Teorema de la Deducción, probar esto es equivalente a ver que:
            \begin{equation*}
                \{\forall x\varphi(x)\}\vDash\varphi(t)
            \end{equation*}
        Por tanto, sea $(\veps, v)$ una interpretación de forma que $I^v(\forall x\varphi(x))=1$, entonces para todo $a\in D$, se tendrá $I^{v(x\mid a)}(\varphi(x)) = 1$. Si tomamos $a=v(t)$, entonces tendremos que:
        \begin{equation*}
            I^{v(x\mid a)}(\varphi(x)) = I^v(\varphi(t))
        \end{equation*}
    \end{enumerate}
\end{ejemplo}

\section{Demostraciones}
Trataremos ahora de generalizar lo que hicimos ya para la demostraciones en el caso de los lenguajes proposicionales, para que cualquier demostración hecha con lenguajes proposicionales siga siendo válida ahora.

\subsection{Definición de una demostración}
En lugar de dar directamente la definición de demostración, daremos primero los axiomas de nuestro sistema y las reglas de inferencia que usaremos, para posteriormente dar la definición de demostración.

\subsubsection{Axiomas}
\noindent
Sobre nuestro lenguaje $\cc{L}$ consideraremos los 3 primeros conjuntos de axiomas, que son los que ya teníamos en lenguajes proposicionales:
\begin{align*}
    \mathcal{A}_1 &= \{\varphi\to(\psi\to\varphi) : \varphi,\psi \in Form(\cc{L})\} \\
    \mathcal{A}_2 &= \{(\varphi\to(\psi\to\chi))\to((\varphi\to\psi)\to(\varphi\to\chi)) : \varphi,\psi,\chi \in Form(\cc{L})\} \\
    \mathcal{A}_3 &= \{(\lnot\varphi\to\lnot\psi)\to((\lnot\varphi\to\psi)\to\varphi) : \varphi,\psi \in Form(\cc{L})\}
\end{align*}

Ahora, será necesario considerar nuevos axiomas que nos permitan generalizar lo ya visto para lenguajes proposicionales a lenguajes de primer orden. Como el lector puede deducir, estos axiomas tendrán que contener cuantificadores, ya que es el concepto principal que introducimos en los lenguajes de primer orden. Antes de dar el cuarto axioma\footnote{Algunos autores dividen este axioma en dos, ya que no consideran la notación que vamos a considerar para poder dar este axioma.}, introduciremos la siguiente notación:

\begin{notacion}
    Sea $\varphi \in Form(\cc{L})$ y $x_1,\ldots,x_n \in Var(\cc{L})$, al notar:
    \begin{equation*}
        \varphi(x_1,\ldots,x_n)
    \end{equation*}
    Estamos diciendo que, si $x_1,\ldots,x_n$ son variables que aparecen en $\varphi$, entonces tienen todas sus ocurrencias libres en $\varphi$.
\end{notacion}

\begin{notacion}
    Sea $\varphi \in Form(\cc{L})$, $x\in Var(\cc{L})$ y $t\in Term(\cc{L})$, cuando aparezca:
    \begin{center}
        ``$t$ libre para $x$ en $\varphi(x)$''
    \end{center}
    Y notemos $\varphi(t)$, significará que estamos cambiando las ocurrencias libres de $x$ que había en $\varphi$ por $t$. Por ser $t$ un término, este puede depender de otras variables, por lo que en este proceso no se permite que variables de $t$ se queden ligadas, sino que deben aparecer libres.
\end{notacion}
Podemos dar ya el cuarto axioma:
\begin{equation*}
    \cc{A}_4 = \{\forall x\varphi(x)\to\varphi(t) \mid \varphi \in Form(\cc{L}), x\in Var(\cc{L}), t \text{\ libre para\ } x \text{\ en\ } \varphi(x)\}
\end{equation*}
Observemos que casos particulares interesantes de este axioma son:
\begin{itemize}
    \item $\forall x\varphi \to \varphi$
    \item $\forall x\varphi(x)\to \varphi(x)$
\end{itemize}

Y podemos finalmente dar el quinto axioma\footnote{Que en aquellos autores que dividen el cuarto axioma en dos, aparece como el sexto.}:
\begin{equation*}
    \cc{A}_5 = \{\forall x(\varphi\to\psi)\to(\varphi\to\forall x\psi) \mid \varphi,\psi \in Form(\cc{L}), x \text{\ no aparece libre en\ } \varphi\}
\end{equation*}

De esta forma, nuestro conjunto de axiomas vendrá dado por:
\begin{equation*}
    \mathcal{A} = \mathcal{A}_1 \cup \mathcal{A}_2 \cup \mathcal{A}_3 \cup \mathcal{A}_4 \cup \mathcal{A}_5
\end{equation*}

Notemos que en estos 5 axiomas no aparecen los conectores $\land$, $\lor$, $\leftrightarrow $ ni el cuantificador $\exists $. En caso de querer usarlos:
\begin{itemize}
    \item Los conectores los expresaremos como fórmulas semánticamente equivalentes pero usando $\lnot$ y $\to$.
    \item Usaremos que $\exists x\varphi$ es semánticamente equivalente a $\lnot\forall x\lnot\varphi$, siendo $\varphi \in Form(\cc{L})$.
\end{itemize}

\subsubsection{Reglas de inferencia}
Las reglas de inferencia que consideraremos en nuestro sistema serán las siguientes, las cuales tendremos en cuenta a la hora de realizar la definición de lo que será una demostración:
\begin{equation*}
    \begin{array}{l}
        \text{Modus ponens} \\
        \varphi\to \psi \\
        \varphi \\
        \hline
        \psi 
    \end{array} \qquad \begin{array}{l}
        \text{Generalización}\\
        \varphi \\
        \ \\
        \hline
        \forall x\varphi 
    \end{array}
\end{equation*}

\begin{definicion}[Demostración]
    Si consideramos el conjunto de fórmulas $\cc{A}$ previamente definido y sea $\Gamma\cup\{\varphi\} \subseteq Form(\cc{L})$, una demostración de $p$ a partir de $\Gamma$ (notado por $\Gamma\vdash p$) es una secuencia de fórmulas $\alpha_1,\alpha_2,\ldots,\alpha_n$ de forma que $\alpha_n = p$ y se verifica para todo $i$ menor o igual que $n$ alguna de las tres condiciones siguientes:
    \begin{itemize}
        \item $\alpha_i \in \cc{A}\cup \Gamma$.
        \item Existen $j,k$ naturales con $j<k<i$ siendo $\alpha_k = \alpha_j \to \alpha_i$ (Modus ponens).
        \item Existe un natural $j$ con $j<i$ siendo $\alpha_i = \forall x\alpha_j$ (Generalización).
    \end{itemize}
\end{definicion}

\subsection{Primeros resultados}
Como primer resultado a destacar, como los lenguajes de primer orden generalizan los lenguajes proposicionales, cualquier demostración para los lenguajes proposicionales seguirán siendo válidas para los lenguajes de primer orden.

\begin{teo}[de la Deducción]
    Sean $\Gamma\cup\{\varphi,\psi\}\subseteq Form(\cc{L})$. Si tenemos que $\Gamma\cup\{\varphi\}\vdash  \psi$ y en su demostración no usamos la regla de generalización sobre un paso en que haya intervenido $\varphi$ con una variable libre en $\varphi$, entonces:
    \begin{equation*}
        \Gamma\vdash \varphi\to\psi
    \end{equation*}
\end{teo}

\begin{observacion}
    Ha sido necesario introducir la condición extra que no teníamos en lenguajes proposicionales ya que vamos a poder demostrar, por ejemplo, ${\{A(x)\}\vdash \forall xA(x)}$:
    \begin{enumerate}
        \item $A(x)$ es hipótesis.
        \item $\forall xA(x)$ por generalización
    \end{enumerate}
    Sin embargo, $\nvdash A(x)\to \forall xA(x)$, ya si que si consideramos por ejemplo el dominio $D=\mathbb{Z}_2$ y $A$ es ``ser igual a 0'', de $A(0)$ no podemos concluir que todo elemento de $\mathbb{Z}_2$ sea $0$. Esto se debe a que semánticamente:
    \begin{equation*}
        \{A(x)\} \nvDash \forall xA(x)
    \end{equation*}
    El lector podría sospechar que la regla de generalización carece de sentido o contradice lo enunciado, pero si somos capaces de demostrar algo para un elemento $x$ arbitrario en un cierto conjunto, entonces seremos capaces de afirmar que $\forall x$ en dicho conjunto, tendremos la proposición conseguida para el elemento arbitrario anterior. Esta es la intuición detrás de la regla de generalización.
\end{observacion}

\begin{prop}
    Sean $\Gamma\cup\{\varphi,\psi\}\subseteq Form(\cc{L})$, si tenemos que $\Gamma\vdash \varphi\to\psi$, entonces $\Gamma\cup\{\varphi\}\vdash \psi$.
\end{prop}

\begin{teo}[Regla de reducción al Absurdo]
    Sean $\Gamma\cup\{\varphi,\psi\}\subseteq Form(\cc{L})$ si tenemos que $\Gamma\cup\{\lnot\varphi\}\vdash \psi$ y $\Gamma\cup\{\lnot\varphi\}\vdash \lnot\psi$ y en esas demostraciones no usamos la regla de generalización sobre un pasa en el que haya intervenido $\lnot\varphi$ con una variable libre en $\lnot\varphi$, entonces:
    \begin{equation*}
        \Gamma\vdash \varphi
    \end{equation*}
\end{teo}

\begin{ejemplo}
    Buscamos demostrar $\vdash (\varphi \to \forall x\psi)\to \forall x(\varphi\to \psi)$ con $x$ no libre en $\varphi$.\\

        Buscamos demostrar con precaución\footnote{Para poder aplicar luego el Teorema de la Deducción bajo las hipótesis correctas con la limitación extra.} que:
        \begin{equation*}
            \{\varphi \to \forall x\psi\} \vdash \forall x(\varphi\to \psi)
        \end{equation*}
        Para ello:
        \begin{enumerate}
            \item $\varphi \to \forall x\psi$ es una hipótesis.
            \item $\forall x\psi\to \psi \in \cc{A}_4$
            \item $\varphi\to \psi$ por silogismo de 1 y 2.
            \item $\forall x(\varphi\to \psi)$ generalización de $3$.
        \end{enumerate}

        Como $x$ no está libre en $\varphi$, tampoco lo estará en $\varphi\to\forall x\psi$, por lo que en esta demostración no hemos usado la regla de generalización sobre un paso en el que haya intervenido $\varphi\to\forall x\psi$ con una variable libre en la misma, por lo que podremos aplicar el Teorema de la Deducción, obteniendo lo que queríamos probar.
\end{ejemplo}

\section{Teoremas de adecuación y coherencia}
Una parte positiva de los lenguajes de primer orden es que a pesar de ser más generales que los proposicionales, seguimos contando con los teoremas de adecuación y de coherencia:

\begin{teo}[de coherencia]
    Sea $\varphi \in Form(\cc{L})$, si $\vdash \varphi$, entonces $\vDash \varphi$.
    \begin{proof}
        La demostración es similar a la del Teorema de coherencia para lenguajes proposicionales, pero ahora hemos de tener en cuenta más axiomas, así como la regla de generalización.
    \end{proof}
\end{teo}

\begin{teo}[de consistencia]
    Nuestro conjunto de axiomas $\cc{A}$ junto con las reglas de inferencia es consistente, es decir, no existe $\varphi \in Form(\cc{L})$ de forma que $\vdash \varphi$ y $\nvdash\varphi$.
\end{teo}

\begin{teo}[de adecuación]
    Sea $\varphi \in Form(\cc{L})$, si $\vDash\varphi$, entonces $\vdash \varphi$.
\end{teo}

\section{Lenguajes de Primer Orden con Igualdad}
Un lenguaje de primer orden con igualdad es un lenguaje de primer orden $\cc{L}$ en el que habrá un símbolo de relación destacado $A$.

\begin{notacion}
    Si $s,t\in Term(\cc{L})$, usaremos con frecuencia:
    \begin{equation*}
        s = t
    \end{equation*}
    Para denotar $A(s,t)$.
\end{notacion}

La única diferencia con los lenguajes de primer orden corrientas será que tendremos dos conjuntos de axiomas extras:

\begin{equation*}
    \cc{A}_6 = \{\forall x(x=x) \mid x\in Var(\cc{L})\}
\end{equation*}

Y para introducir el último axioma, hace falta introducir más notación:
\begin{notacion}
    Sea $\varphi \in Form(\cc{L})$, y $x,y\in Var(\cc{L})$, si notamos $\varphi(x,y)$ tras notar $\varphi(x,x)$, significará que estamos reemplazando algunas ocurrencias libres de $x$ por $y$ en la fórmula $\varphi$.
\end{notacion}
El último axioma será:
\begin{equation*}
    \cc{A}_7 = \{(x=y)\to (\varphi(x,x)\to \varphi(x,y))\mid x,y\in Var(\cc{L})\}
\end{equation*}

\begin{observacion}
    Notemos que con dos conjuntos de axiomas que hemos añadido, tenemos que ``$=$'' es una relación de equivalencia:
    \begin{itemize}
        \item $\cc{A}_6$ nos da la relación reflexiva.
        \item De $\cc{A}_7$ deducimos la simétrica y la transitiva.
    \end{itemize}
    Sin embargo, ``$=$'' es mucho más que eso, ya que de $\cc{A}_7$ no solo deducimos esas propiedades, sino muchas más, tal y como vemos en el siguiente ejemplo.
\end{observacion}

\begin{ejemplo}
    Demostraremos que:
    \begin{equation*}
        \vdash (x=y) \to (f(t_1,\ldots,x,\ldots,t_n) = f(t_1,\ldots,y,\ldots,t_n))
    \end{equation*}
    Para ello, bastará demostrar con cuidado que:
    \begin{equation*}
        \{x=y\} \vdash f(t_1,\ldots,x,\ldots,t_n) = f(t_1,\ldots,y,\ldots,t_n)
    \end{equation*}
    \begin{enumerate}
        \item $\forall x(x=x)\in \cc{A}_6$ 
        \item $\forall x(x=x)\to f(t_1,\dots,x,\dots,t_n)=f(t_1,\dots,x,\dots,t_n)\in \cc{A}_4$
        \item $f(t_1,\dots,x,\dots,t_n)=f(t_1,\dots,x,\dots,t_n)$ por modus ponens de 1 y 2.
        \item $(x=y)\to ((f(t_1,\dots,x,\dots,t_n)=f(t_1,\dots,x,\dots,t_n))\to \\ \to (f(t_1,\dots,x,\dots,t_n)=f(t_1,\dots,y,\dots,t_n))) \in \cc{A}_7$
        \item $x=y$ es hipótesis.
        \item $(f(t_1,\ldots,x,\ldots,t_n)=f(t_1,\ldots,x,\ldots,t_n))\to (f(t_1,\ldots,x,\ldots,t_n)=f(t_1,\ldots,y,\ldots,t_n))$ 

            por modus ponens de 4 y 5.
        \item $f(t_1,\ldots,x,\ldots,t_n)=f(t_1,\ldots,y,\ldots,t_n)$ por modus ponens de 3 y 6.
    \end{enumerate}
    Como no hemos usado ningún paso de generalización, podemos aplicar el Teorema de la Deducción, obteniendo lo que queríamos probar.
\end{ejemplo}

El presente documento es un resumen del microcredencial de ``Lógica y Teoría Descriptiva de Conjuntos'', que recoge los principales conceptos que se impartieron en el mismo. Si cursa el microcredencial se recomienda ver los recursos proporcionados por el profesorado. Si está cursando actualmente la asignatura de ``Lógica y Métodos Discretos'' del grado de Informática, los dos primeros capítulos pueden serle de gran ayuda.\\

A lo largo del curso trabajaremos en $\mathbb{Z}_2$, por lo que se recomienda al lector repasar los apuntes de Álgebra I en caso de no estar familiarizado con dicho cuerpo.

\chapter{Lógica Proposicional}
Consideraremos un conjunto finito de proposiciones atómicas, que serán para nosotros enunciados indivisibles. Nos interesará la veracidad o falsedad de cada una de estas proposiciones. Consideraremos sobre estas las conectivas $\lnot$, $\land$, $\lor$, $\to$ y $\leftrightarrow$. De esta forma, somos capaces de definir lo que es una \underline{proposición} en nuestro lenguaje.

\begin{definicion}[Proposición]
    Definimos las proposiciones de forma recursiva\footnote{Algo que será habitual en este curso.}:
    \begin{enumerate}
        \item Las proposiciones atómicas son proposiciones.
        \item Si $\alpha$ y $\beta$ son proposiciones, también lo son:
            \begin{equation*}
                \lnot\alpha,\ \alpha\land\beta,\ \alpha\lor\beta,\ \alpha\to\beta,\ \alpha\leftrightarrow\beta
            \end{equation*}
        \item No hay más proposiciones que las que se puedan obtener siguiendo una secuencia finita de pasos a partir de las enunciadas.
    \end{enumerate}
\end{definicion}

\section{Semántica}
Una vez definida lo que es una proposición, pasamos a lo que nos interesa, asignar un valor de verdad o de falsedad a cada una de las proposiciones que nos encontremos. Para ello, consideraremos una aplicación del conjunto de las proposiciones en $\mathbb{Z}_2$, e interpretaremos el valor de $0$ como falso y el valor de $1$ como verdad.

\begin{definicion}[Interpretación]
    Sea $\mathcal{P}$ el conjunto de todas las proposiciones de un lenguaje proposicional, una interpretación sobre el mismo es una aplicación $I:\mathcal{P}\rightarrow\mathbb{Z}_2$ que verifica:
    \begin{enumerate}
        \item $I(\lnot a) = 1+ I(a)$.
        \item $I(a\land b) = I(a)I(b)$.
        \item $I(a\lor b) = I(a) + I(b) + I(a)I(b)$.
        \item $I(a\to b) = 1 + I(a) + I(a)I(b)$.
        \item $I(a\leftrightarrow b) = 1 + I(a) + I(b)$.
    \end{enumerate}
    Para cualesquiera proposiciones $a,b\in \mathcal{P}$.
\end{definicion}

\begin{definicion}
    Sea $\alpha$ y $\beta$ dos proposiciones de forma que $I(\alpha)=I(\beta)$ para cualquier interpretación $I$, entonces escribiremos que $\alpha\equiv\beta$ y podemos decir que $\alpha$ y $\beta$ son \underline{semánticamente equivalentes}.
\end{definicion}

\begin{definicion}
    Sea $\alpha$ una proposición: 
    \begin{itemize}
        \item Si existe una interpretación $I$ de forma que $I(\alpha)=1$, diremos que $p$ es \textbf{satisfacible}.
        \item Si existe una interpretación $I$ de forma que $I(\alpha)=0$, diremos que $p$ es \textbf{refutable}.
        \item Si $I(\alpha)=1$ para cualquier interpretación $I$, diremos que $p$ es una \textbf{tautología}.
        \item Si $I(\alpha)=0$ para cualquier interpretación $I$, diremos que $p$ es una \textbf{contradicción}.
    \end{itemize}
\end{definicion}

\begin{definicion}[Consecuencia lógica]
    Sea $\Gamma\cup\{p\}$ un conjunto de proposiciones, decimos que $p$ es consecuencia lógica de $\Gamma$ (notado por $\Gamma\vDash p$), si dada una interpretación $I$, siempre que se tenga que $I(\gamma) = 1$ para cualquier $\gamma\in \Gamma$, entonces se tiene que $I(p)= 1$.
\end{definicion}

\begin{notacion}
    Por comodidad, si $p$ es una proposición de forma que $\emptyset \vDash p$, entonces notaremos:
    \begin{equation*}
        \vDash p
    \end{equation*}
    Notemos que en este caso $p$ es una tautología, ya que estamos diciendo que $I(p)=1$ para cualquier\footnote{Cualquiera que haga ciertos todos los elementos del vacío.} interpretación $I$.
\end{notacion}

\begin{prop}
    Se verifica que $\Gamma\vDash p$ si y solo si $(1+I(p))\displaystyle\prod_{\gamma\in \Gamma}I(\gamma) = 0$.
    \begin{proof}
        Veamos las dos implicaciones:
    \begin{description}
        \item [$\Longrightarrow)$] 
            Sea $I$ una interpretación:
            \begin{itemize}
                \item Si existe un $\gamma\in \Gamma$ de forma que $I(\gamma)=0$, entonces tenemos el resultado.
                \item En caso contrario, tendremos que $I(\gamma)=1$ para cualquier $\gamma\in \Gamma$. En dicho caso, como $\Gamma\vDash p$, se tendrá que $I(p)=1$, por lo que:
                    \begin{equation*}
                        1 + I(p) = 0 \Longrightarrow (1+I(p))\displaystyle\prod_{\gamma\in \Gamma}I(\gamma) = 0
                    \end{equation*}
            \end{itemize}
        \item [$\Longleftarrow)$] 
            Sea $I$ una interpretación que verifica $I(\gamma)=1$ para cualquier $\gamma\in \Gamma$, como $\mathbb{Z}_2$ es un dominio de integridad, de $(1+I(p))\displaystyle\prod_{\gamma\in \Gamma}I(\gamma) = 0$ deducimos que $I(p) +1=0$, por lo que $I(p) = 1$ y entonces se tiene que $\Gamma\vDash p$.
    \end{description}
    \end{proof}
\end{prop}

\begin{teo}[de la deducción]\label{teo:deduccion}
    Sea $\Gamma\cup\{\alpha,\beta\}$ un conjunto de proposiciones, equivalen:
    \begin{enumerate}
        \item $\Gamma\vDash\alpha\to\beta$
        \item $\Gamma\cup\{\alpha\}\vDash\beta$
    \end{enumerate}
    \begin{proof}
        Demostramos las dos implicaciones:
        \begin{description}
            \item [$1)\Longrightarrow 2)$] 
                Sea $I$ una interpretación de forma que $I(\alpha)=1$ y que $I(\gamma)=1$ para todo $\gamma\in \Gamma$, entonces (por 1) deducimos que $I(\alpha\to\beta)=1$, luego:
                \begin{equation*}
                    1 = I(\alpha\to\beta) = 1 + \cancelto{1}{I(\alpha)} + \cancelto{1}{I(\alpha)}I(\beta) = 1 + 1 + I(\beta) = I(\beta)
                \end{equation*}
            \item [$2)\Longrightarrow 1)$] 
                Sea $I$ una interpretación de forma que $I(\gamma)=1$ para todo $\gamma\in \Gamma$: 
                \begin{itemize}
                    \item Si $I(\alpha)=0$, entonces:
                        \begin{equation*}
                            I(\alpha\to\beta) = 1 + I(\alpha) + I(\alpha)I(\beta) = 1
                        \end{equation*}
                        Por lo que se tiene 1.
                    \item Si $I(\alpha)=1$, como $\Gamma\cup\{\alpha\}\vDash\beta$, entonces $I(\beta)=1$, por lo que:
                        \begin{equation*}
                            I(\alpha\to\beta) = 1 + I(\alpha) + I(\alpha)I(\beta) = 1 + 1 + 1 = 1
                        \end{equation*}
                \end{itemize}
        \end{description}
    \end{proof}
\end{teo}

\begin{ejemplo}
    Demostraremos ahora que varias proposiciones son tautologías:
    \begin{description}
        \item [$\vDash\alpha\to\alpha$]~\\
            Por el Teorema de la deducción (\ref{teo:deduccion}), $\vDash\alpha\to\alpha$ es equivalente a ver que $\{\alpha\}\vDash\alpha$. En efecto, sea $I$ una interpretación de forma que $I(\alpha)=1$, tenemos que $I(\alpha)=1$.
        \item [$\vDash\alpha\to(\beta\to\alpha)$]~\\
            Por el Teorema de la deducción, es equivalente ver que $\{\alpha\}\vDash \beta\to\alpha$; que nuevamente por el Teorema de la deducción es equivalente ver que $\{\alpha,\beta\}\vDash \alpha$. En efecto, sea $I$ una interpretación de forma que $I(\alpha)=I(\beta)=1$, entonces $I(\alpha)=1$.
        \item [$\vDash(\alpha\to(\beta\to\gamma))\to((\alpha\to\beta)\to(\alpha\to\gamma))$]~\\
            Por el Teorema de la deducción aplicado 3 veces, es equivalente ver que:
            \begin{equation*}
                \{\alpha\to(\beta\to\gamma),\alpha\to\beta,\alpha\}\vDash \gamma
            \end{equation*}
            Sea $I$ una interpretación de forma que:
            \begin{equation*}
                1 = I(\alpha\to(\beta\to\gamma)) = I(\alpha\to\beta) = I(\alpha)
            \end{equation*}
            Entonces:
            \begin{align*}
                1 &= I(\alpha\to\beta) = 1 + I(\alpha) + I(\alpha)I(\beta) = 1 + 1 + I(\beta) = I(\beta) \Longrightarrow I(\beta) = 1 \\
                1 &= I(\alpha\to(\beta\to\gamma)) = 1 + I(\alpha) + I(\alpha)I(\beta\to\gamma) \\
                  &= 1 + I(\alpha) + I(\alpha)(1 + I(\beta) + I(\beta)I(\gamma)) = 1 + 1 + 1(1 + 1 + I(\gamma))  \\
                  &= I(\gamma) \Longrightarrow \underline{I(\gamma) = 1}
            \end{align*}
        \item [$\vDash(\lnot\alpha\to\lnot\beta) \to ((\lnot\alpha\to\beta)\to \alpha)$]~\\
            Por el Teorema de la deducción aplicado 2 veces, es equivalente ver que:
            \begin{equation*}
                \{\lnot\alpha\to\lnot\beta,\lnot\alpha\to\beta\} \vDash \alpha
            \end{equation*}
            Sea $I$ una interpretación de forma que:
            \begin{align*}
                1 = I(\lnot\alpha\to\lnot\beta) &= 1 + I(\lnot\alpha) + I(\lnot\alpha)I(\lnot\beta) \\
                1 = I(\lnot\alpha\to\beta) &= 1 + I(\lnot\alpha) + I(\lnot\alpha)I(\beta)
            \end{align*}
            Entonces (sumando):
            \begin{equation*}
                0 = I(\lnot\alpha\to\lnot\beta) + I(\lnot\alpha\to\beta) = I(\lnot\alpha)(I(\lnot\beta) + I(\beta)) \AstIg I(\lnot\alpha)
            \end{equation*}
            Donde en $(\ast)$ hemos usado que $I(\lnot\beta) = 1 + I(\beta)\Longrightarrow I(\lnot\beta)+I(\beta)=1$.

            Como $I(\lnot\alpha)=0$, se tiene que $I(\alpha)=1$, como queríamos demostrar.
    \end{description}
\end{ejemplo}

\begin{definicion}
    Sea $\Gamma$ un conjunto de proposiciones, decimos que $\Gamma$ es \textbf{inconsistente} si para toda interpretación $I$ existe $\gamma\in \Gamma$ de forma que $I(\gamma)=0$.
\end{definicion}

\begin{prop}
    Sea $\Gamma\cup\{\alpha\}$ un conjunto de proposiciones, equivalen:
    \begin{enumerate}
        \item $\Gamma\vDash \alpha$.
        \item $\Gamma\cup\{\lnot\alpha\}$ es inconsistente.
    \end{enumerate}
    \begin{proof}
        Demostramos las dos implicaciones:
        \begin{description}
            \item [$1) \Longrightarrow 2)$] Sea $I$ una interpretación:
                \begin{itemize}
                    \item Si existe un $\gamma\in \Gamma$ de forma que $I(\gamma)=0$, entonces $\Gamma$ es inconsistente, de donde $\Gamma\cup\{\lnot\alpha\}$ también lo es.
                    \item Si $I(\gamma)=1$ para cualquier $\gamma\in \Gamma$, aplicando que $\Gamma\vDash \alpha$ deducimos que\newline $I(\alpha)=1 \Longrightarrow I(\lnot\alpha) = 1 + I(\alpha) = 0$, por lo que $\Gamma\cup\{\lnot\alpha\}$ es inconsistente.
                \end{itemize}
            \item [$2) \Longrightarrow 1)$] Sea $I$ una interpretación de forma que $I(\gamma)=1$ para cualquier $\gamma\in \Gamma$, como $\Gamma\cup\{\lnot\alpha\}$ es inconsistente, deducimos que $I(\lnot\alpha) = 0$, luego $I(\alpha) = 1$.
        \end{description}
    \end{proof}
\end{prop}

\subsection{Algoritmo de Davis \& Putnam}
\begin{definicion}
    Introducimos definiciones que nos serán útiles para llegar al algoritmo de Davis \& Putnam:
    \begin{itemize}
        \item Dada una proposición atómica $a$, entonces decimos que $a$ y $\lnot a$ son \underline{literales}.
        \item Sea $a$ una proposición atómica, denotamos $a^c = \lnot a$ y ${(\lnot a)}^{c} = a$. Para un literal $l$, decimos que $l^c$ es su \underline{complemento}.
        \item Sean $l_1,\ldots,l_n$ literales, entonces decimos que $l_1\lor \ldots \lor l_n$ es una \underline{cláusula}.
        \item Sea $\alpha$ una proposición, decimos que está en \underline{forma normal conjuntiva} (abreviado como fnc) si $\alpha$ es de la forma $c_1\land \ldots \land c_n$, con $c_1,\ldots,c_n$ cláusulas.
        \item A la cláusula sin literales (compuesta por la disyunción de 0 literales) la llamamos \underline{cláusula vacía}, y la denotamos por $\square$.
    \end{itemize}
\end{definicion}

\begin{prop}
    Sea $I$ una interpretación, entonces:
    \begin{equation*}
        I(\square) = 0
    \end{equation*}
    \begin{proof}
        Como $\square\lor a = a$ para cualquier proposición atómica $a$, entonces:
        \begin{equation*}
            I(\square\lor a) = I(\square) + I(a) + I(\square)I(a) = I(a)
        \end{equation*}
        De donde deducimos:
        \begin{equation*}
            I(\square) + I(\square)I(a) = I(\square)(1 + I(a)) = 0
        \end{equation*}
        Luego $I(\square)=0$ o $I(a)=1$, pero como la proposición atómica $a$ era arbitraria (y sabemos que hay proposiciones atómicas que no son tautologías), deducimos que ha de ser $I(\square)=0$.
    \end{proof}
\end{prop}

\begin{prop}
    Toda proposición puede expresarse en una proposición semánticamente equivalente que se encuentre en forma normal conjuntiva.
    \begin{proof}
        Aunque no lo demostraremos, el lector puede hacerse una idea de que el enunciado es cierto, con ayuda de las siguientes reglas:
        \begin{itemize}
            \item $\lnot\lnot a\equiv a$ (regla de la doble negación)
            \item $\lnot(a\lor b) \equiv \lnot a\land\lnot b$ y $\lnot(a\land b) \equiv \lnot a\lor \lnot b$ (reglas de De Morgan)
            \item $a\to b \equiv \lnot a\lor b$ y $a \leftrightarrow b \equiv (a\to b) \land (b\to a)$
            \item $a\lor (b\land c) \equiv (a\lor b) \land (a\lor c)$ (ley distributiva)
        \end{itemize}
    \end{proof}
\end{prop}

\begin{ejemplo}
    Sea $\alpha = (a\to b)\to a$, buscamos una proposición semánticamente equivalente en forma normal conjuntiva. Para ello, primero quitamos $\to$ de la proposición:
    \begin{equation*}
        (a\to b)\to a \equiv \lnot(\lnot a\lor b) \lor a
    \end{equation*}
    Posteriormente, aplicamos la regla de De Morgan, así como la de la doble negación:
    \begin{equation*}
        \lnot(\lnot a\lor b) \lor a \equiv (\lnot\lnot a\land \lnot b) \lor a \equiv (a\land \lnot b)\lor a
    \end{equation*}
    Aplicando la ley distributiva ya llegamos a una proposición semánticamente equivalente en forma normal conjuntiva:
    \begin{equation*}
        (a \lor a) \land (a \lor \lnot b)
    \end{equation*}
    Sin embargo, como $a \lor a \equiv a$, podemos seguir simplificando, obteniendo que:
    \begin{equation*}
        a \land (a \lor \lnot b)
    \end{equation*}
    Pero como:
    \begin{align*}
        I(\alpha \land (\alpha\lor \beta)) &= I(\alpha)(I(\alpha)+I(\beta)+I(\alpha)I(\beta)) \\
                                           &= I(\alpha)I(\alpha) + I(\alpha)I(\beta) + I(\alpha)I(\alpha)I(\beta) \\
                                           &= I(\alpha) + I(\alpha) I(\beta) + I(\alpha)I(\beta) = I(\alpha)
    \end{align*}
    Llegamos finalmente a que:
    \begin{equation*}
        (a\to b)\to a \equiv a
    \end{equation*}
\end{ejemplo}

\begin{prop}\label{prop:ej1}
    Dado el conjunto de proposiciones $\{\psi_1,\ldots,\psi_n\}$, son equivalentes:
    \begin{itemize}
        \item $\{\psi_1,\ldots,\psi_n\}$ es inconsistente.
        \item $\{\psi_1\land\ldots\land\psi_n\}$ es inconsistente.
    \end{itemize}
    \begin{proof}
        Notemos que decir que $\{\psi_1\land \ldots \land \psi_n\}$ sea inconsistente significa que dada una interpretación $I$, entonces:
        \begin{equation*}
            I(\psi_1\land \ldots \land \psi_n) = \prod_{k=1}^{n}I(\psi_k) = 0
        \end{equation*}
        $\{\psi_1,\ldots,\psi_n\}$ será inconsistente si y solo si dada una interpretación $I$ hay alguna proposición $\psi_i$ de forma que $I(\psi_i)=0$, si y solo si $\prod_{k=1}^{n}I(\psi_k)=0$, lo que equivale con que $\{\psi_1\land\ldots\land\psi_n\}$ sea inconsistente.
    \end{proof}
\end{prop}

\begin{prop}
    Dado un conjunto de proposiciones $\Gamma$, si consideramos el conjunto $\Gamma'$ resultante de considerar para cada fórmula de $\Gamma$ su forma normal conjuntiva y luego tomar la unión de todas ellas, se tiene que $\Gamma$ es inconsistente si y solo si $\Gamma'$ es inconsistente.
    \begin{proof}
        Puede demostrarse fácilmente usando la Proposición~\ref{prop:ej1}.
    \end{proof}
\end{prop}

El último resultado es de gran importancia, ya que recordemos que, si $\Gamma\cup\{\varphi\}$ es un conjunto de proposiciones, que $\Gamma\vDash\varphi$ es equivalente a probar que $\Gamma\cup\{\lnot\varphi\}$ es inconsistente, conjunto que puede transformarse en un conjunto de cláusulas $\Delta$ que será inconsistente si y solo si lo era $\Gamma\cup\{\lnot\varphi\}$.

De esta forma, notemos que el estudio de que una proposición sea consecuencia lógica de otra se reduce a estudiar si un conjunto de cláusulas es o no inconsistente.

\subsubsection{El algoritmo}
El algoritmo de Davis y Putnam consiste en aplicar a un conjunto de cláusulas las reglas que a continuación se exponen, intentando siempre aplicar la primera en el orden en que vienen dadas. Cada conjunto de cláusulas que obtengamos al aplicar las reglas de esta forma será inconsistente si y solo si lo era el original del que proviene (en el caso de la última regla, el conjunto de partida es inconsistente si y solo si lo son los dos conjuntos que se generan después de aplicar la regla):
\begin{description}
    \item [Regla 1 (regla de las tautologías).] Quítense todas las fórmulas que sean tautologías, es decir, las que contengan un literal y su complementario.
    \item [Regla 2 (regla de los literales).] Si hay una cláusula que es un literal $L$ en $\Delta$, obténgase $\Delta'$ a partir de $\Delta$ eliminando todas las cláusulas de $\Delta$ que contengan a $L$.
        \begin{itemize}
            \item Si $\Delta'$ es el conjunto vacío, entonces $\Delta$ no será inconsistente, ya que el vacío no lo es.
            \item En otro caso, obténgase $\Delta''$ a partir de $\Delta'$ suprimiendo $L^c$ de $\Delta'$.

                Notemos que si $L^c$ era una cláusula, entonces el resultado de suprimir $L^c$ es $\square$, la cláusula vacía, en cuyo caso, el conjunto $\{\square\}$ es inconsistente.
        \end{itemize}
    \item [Regla 3 (regla de los literales puros).] Si un literal $L$ aparece en algunas cláusulas y $L^c$ no aparece en ninguna, quítense todas las cláusulas conteniendo a $L$.
    \item [Regla 4 (regla de la generalización).] Si una cláusula $C$ tiene todos sus literales en otra $C'$ (es decir $C\subseteq C'$), quítese $C'$.
    \item [Regla 5 (regla de la subdivisión).] Si un literal $L$ y su complementario $L^c$ están presentes en el conjunto de cláusulas, construir dos nuevos conjuntos de cláusulas de la siguiente forma:
        \begin{itemize}
            \item El primero se obtiene quitando todas las cláusulas conteniendo a $L$ y borrando las ocurrencias de $L^c$.
            \item El segundo se obtiene quitando todas las cláusulas conteniendo a $L^c$ y borrando las ocurrencias de $L$.
        \end{itemize}
        En este caso, el conjunto original es inconsistente si y solo si lo son los dos conjuntos resultado de aplicar esta regla.
\end{description}

\begin{prop}
    Sea $\Delta$ un conjunto de cláusulas, $\Delta$ es inconsistente si y solo si lo son todos y cada uno de los conjuntos obtenidos tras aplicar las reglas del algoritmos de Davis y Putnam sobre $\Delta$.
    \begin{proof}
        Para ello, hemos de probar que el conjunto que obtenemos al aplicar cada regla sobre $\Delta$ es inconsistente si y solo si lo es $\Delta$:
    \end{proof}
\end{prop}

\section{Demostraciones}
\begin{definicion}[Demostración]
    Sean $\mathcal{A}$ y $\Gamma\cup\{p\}$ dos conjuntos de proposiciones (nos referiremos al conjunto $\mathcal{A}$ como ``conjunto de axiomas'' y a $\Gamma$ como ``conjunto de hipótesis''), una demostración de $p$ a partir de $\Gamma$ (notado por $\Gamma\vdash p$) es una secuencia de proposiciones $\alpha_1,\alpha_2,\ldots,\alpha_n$ de forma que $\alpha_n=p$ y se verifica para todo $i$ menor o igual que $n$:
    \begin{itemize}
        \item bien $\alpha_i \in \mathcal{A}\cup\Gamma$.
        \item bien $\exists k,j<i$ siendo $\alpha_k = \alpha_j\to \alpha_i$.
    \end{itemize}
\end{definicion}

\begin{notacion}
    Si $p$ es una proposición de forma que $\emptyset \vdash p$, podremos notar $\vdash p$ y diremos que $p$ es un teorema.
\end{notacion}

\begin{ejemplo}
    Como ejemplo demostración, veamos que $\{\alpha,\alpha\to\beta\}\vdash \beta$. Para ello, consideramos:
    \begin{align*}
        \alpha_1 &= \alpha \\
        \alpha_2 &= \alpha\to\beta \\
        \alpha_3 &= \beta
    \end{align*}
    Como vemos, es una demostración de $\beta$ a partir de $\{\alpha,\alpha\to\beta\}$ porque $\alpha_1,\alpha_2,\alpha_3$ son proposiciones, $\alpha_3=\beta$ y:
    \begin{itemize}
        \item $\alpha_1\in \Gamma$.
        \item $\alpha_2\in \Gamma$.
        \item $1,2<3$ y $\alpha_2 = \alpha_1\to \alpha_3$.
    \end{itemize}
\end{ejemplo}

\begin{notacion}
    Para abreviar las demostraciones, a partir de ahora no daremos una secuencia numerada de proposiciones $\alpha_1,\ldots,\alpha_n$, sino que numeraremos los pasos de la demostración y entenderemos que para formalizarla totalmente debemos coger como $\alpha_i$ el paso $i-$ésimo de la demostración.

    Más aún, para no pararnos a comprobar las condiciones abstractas que han de cumplir cada una de las propiedades de la demostrción, incluiremos junto a los pasos de la demostración un comentario sobre por qué dicho paso es válido.

    Con esta notación, la demostración de $\{\alpha,\alpha\to\beta\}\vdash \beta$ quedaría de la forma:
    \begin{enumerate}
        \item $\alpha$ es una hipótesis.
        \item $\alpha\to\beta$ es una hipótesis.
        \item $\beta$ por Modus Ponens de 1 y 2.
    \end{enumerate}
\end{notacion}~\\

\noindent
Finalmente, como conjunto $\mathcal{A}$ de axiomas, consideraremos:
\begin{equation*}
    \mathcal{A} = \mathcal{A}_1 \cup \mathcal{A}_2 \cup \mathcal{A}_3
\end{equation*}
Con:
\begin{align*}
    \mathcal{A}_1 &= \{\alpha\to(\beta\to\alpha) : \alpha,\beta \text{\ son proposiciones}\} \\
    \mathcal{A}_2 &= \{(\alpha\to(\beta\to\gamma))\to((\alpha\to\beta))\to(\alpha\to\gamma) : \alpha,\beta,\gamma \text{\ son proposiciones}\} \\
    \mathcal{A}_3 &= \{(\lnot\alpha\to\lnot\beta)\to((\lnot\alpha\to\beta)\to\alpha) : \alpha,\beta \text{\ son proposiciones}\}
\end{align*}

\begin{ejemplo}
    Ejemplos de algunas demostraciones:
    \begin{itemize}
        \item $\{\alpha\}\vdash \beta\to\alpha$
            \begin{enumerate}
                \item $\alpha\to(\beta\to\alpha)\in \mathcal{A}_1$
                \item $\alpha$ es una hipótesis
                \item $\beta\to\alpha$ Modus ponens de 1 y 2.
            \end{enumerate}
        \item $\vdash \alpha\to\alpha$
            \begin{enumerate}
                \item $(\alpha\to((\alpha\to\alpha)\to\alpha))\to((\alpha\to(\alpha\to\alpha))\to(\alpha\to\alpha))\in \mathcal{A}_2$
                \item $\alpha\to((\alpha\to\alpha)\to\alpha)\in \mathcal{A}_1$
                \item $(\alpha\to(\alpha\to\alpha))\to(\alpha\to\alpha)$ Modus ponens de 1 y 2
                \item $\alpha\to(\alpha\to\alpha)\in \mathcal{A}_2$
                \item $\alpha\to\alpha$ Modus ponens de 3 y 4
            \end{enumerate}
    \end{itemize}
\end{ejemplo}

\begin{teo}[de Herbrand o de la deducción]\label{teo:herbrand}
    Sea $\Gamma\cup\{\alpha,\beta\}$ un conjunto de proposiciones, equivalen:
    \begin{enumerate}
        \item $\Gamma\vdash \alpha\to\beta$
        \item $\Gamma\cup\{\alpha\}\vdash \beta$
    \end{enumerate}
    \begin{proof}
        Demostramos las dos implicaciones:
        \begin{description}
            \item [$1) \Longrightarrow 2)$]
                Como $\Gamma\vdash \alpha\to\beta$, podemos construir una demostración de $n$ pasos de $\alpha\to\beta$ a partir de $\Gamma$. En cuyo caso, podemos añadir 2 pasos más a su demostración, de forma que:
                \begin{enumerate}
                    \item \ldots \\
                        \vdots 
                    \item[$n$.] $\alpha\to\beta$
                    \item[$n+1$.] $\alpha$ es hipótesis
                    \item[$n+2$.] $\beta$ por Modus ponens de $n$ y $n+1$
                \end{enumerate}
                Como en los $n$ primeros pasos solo hemos usado como hipótesis $\Gamma$, hemos conseguido demostrar en $n+2$ pasos que $\Gamma\cup\{\alpha\}\vdash \beta$.
            \item [$2) \Longrightarrow 1)$] 
                Como $\Gamma\cup\{\alpha\}\vdash \beta$, podemos obtener una demostración $\beta$ a partir de $\Gamma\cup\{\alpha\}$ de $n$ pasos: $\beta_1,\ldots,\beta_n$ (con $\beta_n=\beta$). Por inducción sobre $n$ (el número de pasos de la demostración):
                \begin{itemize}
                    \item \underline{Si $n=1$}: Como $\Gamma\cup\{\alpha\}\vdash \beta$ gracias a la demostración $\beta_1=\beta$, distinguimos casos:
                        \begin{enumerate}[label=(\alph*)]
                            \item $\beta_1\in \mathcal{A}$. En dicho caso, podemos considerar la demostración:
                                \begin{enumerate}[label=\arabic*.]
                                    \item $\beta_1\in \mathcal{A}$
                                    \item $\beta_1\to(\alpha\to\beta_1)\in \mathcal{A}_1$
                                    \item $\alpha\to\beta_1$ por Modus ponens de 1 y 2
                                \end{enumerate}
                                Y con esto tenemos que $\Gamma\vdash \alpha\to\beta$.
                            \item $\beta_1\in \Gamma$. En dicho caso, podemos considerar una demostración similar al caso anterior:
                                \begin{enumerate}[label=\arabic*.]
                                    \item $\beta_1\in \Gamma$
                                    \item $\beta_1\to(\alpha\to\beta_1)\in \mathcal{A}_1$
                                    \item $\alpha\to\beta_1$ por Modus ponens de 1 y 2
                                \end{enumerate}
                                Y con esto también tenemos que $\Gamma\vdash \alpha\to\beta$.
                            \item $\beta_1=\alpha$. En dicho caso, podemos copiar la demostración de $\vdash \beta\to\beta$ del ejemplo anterior, llegando a que $\Gamma\vdash \alpha\to\beta$.
                        \end{enumerate}
                    \item \underline{En el paso de inducción}, supuesto que de $\Gamma\cup\{\alpha\}\vdash \beta_m$ podemos deducir que $\Gamma\vdash \alpha\to\beta_m$ para todo $m\leq n$, suponemos ahora que $\Gamma\cup\{\alpha\}\vdash \beta_{n+1}$ y queremos ver que $\Gamma\vdash \alpha\to\beta_{n+1}$.

                        En dicho caso, supuesto que $\beta_{m+1}\notin \mathcal{A}\cup\Gamma\cup\{\alpha\}$ (ya que si no la demostración es análoga al caso $n=1$), la única posibilidad es que hayan de existir $i,j<n+1$ con $\beta_i=\gamma$ y $\beta_j=\gamma\to\beta_{m+1}$.

                        Si ahora consideramos los $i$ primeros pasos de la demostración, tenemos que $\Gamma\cup\{\alpha\}\vdash \gamma$ y si consideramos los $j$ primeros pasos, tenemos que $\Gamma\cup\{\alpha\}\vdash \gamma\to\beta_{n+1}$. Por hipótesis de inducción, como $i,j<n+1$, tenemos que $\Gamma\vdash \alpha\to\gamma$ y que $\Gamma\vdash \alpha\to(\gamma\to\beta_{n+1})$. En este momento, podemos realizar la demostración (con hipótesis $\Gamma$):
                        \begin{enumerate}
                            \item[$1$.] \ldots \\
                                \vdots
                            \item[$p$.] $\alpha\to\gamma$
                            \item[$p+1$.] \ldots \\
                                \vdots
                            \item[$q$.] $\alpha\to(\gamma\to\beta_{n+1})$
                            \item[$q+1$.] $(\alpha\to(\gamma\to\beta_{n+1}))\to((\alpha\to\gamma)\to(\alpha\to\beta_{n+1}))\in \mathcal{A}_2$
                            \item[$q+2$.] $(\alpha\to\gamma)\to(\alpha\to\beta_{n+1})$ por Modus ponens de $q$ y $q+1$.
                            \item[$q+3$.] $\alpha\to\beta_{n+1}$ por Modus ponens de $p$ y $q+2$.
                        \end{enumerate}
                \end{itemize}
        \end{description}
    \end{proof}
\end{teo}

\subsection{Resultados útiles a la hora de realizar demostraciones}
\begin{prop}[regla de reducción al absurdo clásica]
    Sea $\Gamma\cup\{\alpha,\beta\}$ un conjunto de proposiciones: si $\Gamma\cup\{\lnot\alpha\}\vdash \beta$ y $\Gamma\cup\{\lnot\alpha\}\vdash \lnot\beta$, entonces $\Gamma\vdash \alpha$.
    \begin{proof}
        Supuesto que $\Gamma\cup\{\lnot\alpha\}\vdash \beta$ y que $\Gamma\cup\{\lnot\alpha\}\vdash \lnot\beta$, por el Teorema de Herbrand (\ref{teo:herbrand}), se tiene que $\Gamma\vdash \lnot\alpha\to\beta$ y que $\Gamma\vdash \lnot\alpha\to\lnot\beta$. En dicho caso:
        \begin{enumerate}
            \item[1.] \ldots \\
                \vdots
            \item[$p$.] $\lnot\alpha\to\lnot\beta$
            \item[$p+1$.] \ldots \\
                \vdots
            \item[$q$.] $\lnot\alpha\to\beta$
            \item[$q+1$.] $(\lnot\alpha\to\lnot\beta)\to((\lnot\alpha\to\beta)\to\alpha)\in \mathcal{A}_3$
            \item[$q+2$.] $((\lnot\alpha\to\beta)\to\alpha)$ por Modus ponens de $q+1$ y $p$.
            \item[$q+3$.] $\alpha$ por Modus ponens de $q+2$ y $q$.
        \end{enumerate}
        Como desde el paso 1 hasta el $q$ solo hemos usado como hipótesis $\Gamma$, deducimos que $\Gamma\vdash \alpha$.
    \end{proof}
\end{prop}

\begin{prop}[leyes de silogismo o transitividad de la flecha]
    Sean $\alpha$, $\beta$ y $\gamma$ proposiciones, se verifican:
    \begin{enumerate}
        \item $\vdash (\alpha\to\beta)\to((\beta\to\gamma)\to(\alpha\to\gamma))$
        \item $\vdash (\beta\to\gamma)\to((\alpha\to\beta)\to(\alpha\to\gamma))$
    \end{enumerate}
    \begin{proof}
        Demostraremos la primera y dejamos la segunda como ejercicio. Para ello, aplicando el Teorema de Herbrand 3 veces, llegamos a que 1 es equivalente a ver que:
        \begin{equation*}
            \{\alpha\to\beta,\beta\to\gamma,\alpha\}\vdash \gamma
        \end{equation*}
        Para ello, nos sirve con la demostración:
        \begin{enumerate}
            \item $\alpha\to\beta$ es una hipótesis
            \item $\alpha$ es una hipótesis
            \item $\beta$ por Modus ponens de 1 y 2
            \item $\beta\to\gamma$ es una hipótesis
            \item $\gamma$ por Modus ponens de 3 y 4
        \end{enumerate}
    \end{proof}
\end{prop}

\begin{coro}[regla del silogismo]
    Sea $\Gamma\cup\{\alpha,\beta\}$ un conjunto de proposiciones, si $\Gamma\vdash \alpha\to\beta$ y $\Gamma\vdash \beta\to\gamma$, entonces $\Gamma\vdash \alpha\to\gamma$.
    % \begin{proof} % // TODO: Hacer
    % \end{proof}
\end{coro}

\begin{prop}[ley de conmutación de premisas]
    Sean $\alpha$, $\beta$ y $\gamma$ proposiciones:
    \begin{equation*}
        \vdash (\alpha\to(\beta\to\gamma))\to(\beta\to(\alpha\to\gamma))
    \end{equation*}
    \begin{proof}
        Aplicando el Teorema de Herbrand 3 veces, es equivalente a ver que:
        \begin{equation*}
            \{\alpha\to(\beta\to\gamma),\beta,\alpha\}\vdash \gamma
        \end{equation*}
        Para ello, nos sirve con:
        \begin{enumerate}
            \item $\alpha\to(\beta\to\gamma)$ es una hipótesis
            \item $\alpha$ es una hipótesis
            \item $\beta\to \gamma$ por Modus ponens de 1 y 2
            \item $\beta$ es una hipótesis
            \item $\gamma$ por Modus ponenes de 3 y 4
        \end{enumerate}
    \end{proof}
\end{prop}

\begin{coro}[regla de conmutación de premisas]
    Sea $\Gamma\cup\{\alpha,\beta,\gamma\}$ un conjunto de proposiciones, si $\Gamma\vdash \alpha\to(\beta\to\gamma)$, entonces $\Gamma\vdash \beta\to(\alpha\to\gamma)$.
\end{coro}

\begin{prop}[ley de la doble negación]
    Sea $\alpha$ una proposición:
    \begin{equation*}
        \vdash \lnot\lnot\alpha\to \alpha
    \end{equation*}
    \begin{proof}
        Por el Teorema de Herbrand, es equivalente a ver que $\{\lnot\lnot\alpha\}\vdash \alpha$. Para ello, usamos la regla de la reducción al absurdo clásica, ya que:
        \begin{enumerate}
            \item $\{\lnot\lnot\alpha,\lnot\alpha\}\vdash \lnot\lnot\alpha$
            \item $\{\lnot\lnot\alpha,\lnot\alpha\}\vdash \lnot\alpha$
        \end{enumerate}
        Luego concluimos que $\{\lnot\lnot\alpha\}\vdash \alpha$.
    \end{proof}
\end{prop}

\begin{prop}[ley débil de la doble negación]
    Sea $\alpha$ una proposición:
    \begin{equation*}
        \vdash \alpha\to\lnot\lnot\alpha
    \end{equation*}
    \begin{proof}
        Por el Teorema de Herbrand, es equivalente a ver que $\{\alpha\}\vdash \lnot\lnot\alpha$. Para ello, usamos la regla de la reducción al absurdo clásica, con lo que partimos que $\{\alpha,\lnot\lnot\lnot\alpha\}$ y tenemos que demostrar una proposición y su negación. Para ello:
        \begin{enumerate}
            \item $\lnot\lnot\lnot\alpha\to\alpha$ por la ley de la doble negación
            \item $\lnot\lnot\lnot\alpha$ es una hipótesis
            \item $\alpha$ por Modus ponens de 1 y 2
            \item $\lnot\alpha$ es una hipótesis
        \end{enumerate}
        Concluimos por la regla de la reducción al absurdo que $\{\alpha\}\vdash \lnot\lnot\alpha$.
    \end{proof}
\end{prop}

\documentclass[12pt]{article}
% Idioma y codificación
\usepackage[spanish, es-tabla]{babel}       %es-tabla para que se titule "Tabla"
\usepackage[utf8]{inputenc}

% Márgenes
\usepackage[a4paper,top=3cm,bottom=2.5cm,left=3cm,right=3cm]{geometry}

% Comentarios de bloque
\usepackage{verbatim}

% Paquetes de links
\usepackage[hidelinks]{hyperref}    % Permite enlaces
\usepackage{url}                    % redirecciona a la web

% Más opciones para enumeraciones
\usepackage{enumitem}

% Personalizar la portada
\usepackage{titling}


% Paquetes de tablas
\usepackage{multirow}


%------------------------------------------------------------------------

%Paquetes de figuras
\usepackage{caption}
\usepackage{subcaption} % Figuras al lado de otras
\usepackage{float}      % Poner figuras en el sitio indicado H.


% Paquetes de imágenes
\usepackage{graphicx}       % Paquete para añadir imágenes
\usepackage{transparent}    % Para manejar la opacidad de las figuras

% Paquete para usar colores
\usepackage[dvipsnames]{xcolor}
\usepackage{pagecolor}      % Para cambiar el color de la página

% Habilita tamaños de fuente mayores
\usepackage{fix-cm}


%------------------------------------------------------------------------

% Paquetes de matemáticas
\usepackage{mathtools, amsfonts, amssymb, mathrsfs}
\usepackage[makeroom]{cancel}     % Simplificar tachando
\usepackage{polynom}    % Divisiones y Ruffini
\usepackage{units} % Para poner fracciones diagonales con \nicefrac

\usepackage{pgfplots}   %Representar funciones
\pgfplotsset{compat=1.18}  % Versión 1.18

\usepackage{tikz-cd}    % Para usar diagramas de composiciones
\usetikzlibrary{calc}   % Para usar cálculo de coordenadas en tikz

%Definición de teoremas, etc.
\usepackage{amsthm}
%\swapnumbers   % Intercambia la posición del texto y de la numeración

\theoremstyle{plain}

\makeatletter
\@ifclassloaded{article}{
  \newtheorem{teo}{Teorema}[section]
}{
  \newtheorem{teo}{Teorema}[chapter]  % Se resetea en cada chapter
}
\makeatother

\newtheorem{coro}{Corolario}[teo]           % Se resetea en cada teorema
\newtheorem{prop}[teo]{Proposición}         % Usa el mismo contador que teorema
\newtheorem{lema}[teo]{Lema}                % Usa el mismo contador que teorema

\theoremstyle{remark}
\newtheorem*{observacion}{Observación}

\theoremstyle{definition}

\makeatletter
\@ifclassloaded{article}{
  \newtheorem{definicion}{Definición} [section]     % Se resetea en cada chapter
}{
  \newtheorem{definicion}{Definición} [chapter]     % Se resetea en cada chapter
}
\makeatother

\newtheorem*{notacion}{Notación}
\newtheorem*{ejemplo}{Ejemplo}
\newtheorem*{ejercicio*}{Ejercicio}             % No numerado
\newtheorem{ejercicio}{Ejercicio} [section]     % Se resetea en cada section


% Modificar el formato de la numeración del teorema "ejercicio"
\renewcommand{\theejercicio}{%
  \ifnum\value{section}=0 % Si no se ha iniciado ninguna sección
    \arabic{ejercicio}% Solo mostrar el número de ejercicio
  \else
    \thesection.\arabic{ejercicio}% Mostrar número de sección y número de ejercicio
  \fi
}


% \renewcommand\qedsymbol{$\blacksquare$}         % Cambiar símbolo QED
%------------------------------------------------------------------------

% Paquetes para encabezados
\usepackage{fancyhdr}
\pagestyle{fancy}
\fancyhf{}

\newcommand{\helv}{ % Modificación tamaño de letra
\fontfamily{}\fontsize{12}{12}\selectfont}
\setlength{\headheight}{15pt} % Amplía el tamaño del índice


%\usepackage{lastpage}   % Referenciar última pag   \pageref{LastPage}
\fancyfoot[C]{\thepage}

%------------------------------------------------------------------------

% Conseguir que no ponga "Capítulo 1". Sino solo "1."
\makeatletter
\@ifclassloaded{book}{
  \renewcommand{\chaptermark}[1]{\markboth{\thechapter.\ #1}{}} % En el encabezado
    
  \renewcommand{\@makechapterhead}[1]{%
  \vspace*{50\p@}%
  {\parindent \z@ \raggedright \normalfont
    \ifnum \c@secnumdepth >\m@ne
      \huge\bfseries \thechapter.\hspace{1em}\ignorespaces
    \fi
    \interlinepenalty\@M
    \Huge \bfseries #1\par\nobreak
    \vskip 40\p@
  }}
}
\makeatother

%------------------------------------------------------------------------
% Paquetes de cógido
\usepackage{minted}
\renewcommand\listingscaption{Código fuente}

\usepackage{fancyvrb}
% Personaliza el tamaño de los números de línea
\renewcommand{\theFancyVerbLine}{\small\arabic{FancyVerbLine}}

% Estilo para C++
\newminted{cpp}{
    frame=lines,
    framesep=2mm,
    baselinestretch=1.2,
    linenos,
    escapeinside=||
}



\usepackage{listings} % Para incluir código desde un archivo

\renewcommand\lstlistingname{Código Fuente}
\renewcommand\lstlistlistingname{Índice de Códigos Fuente}

% Definir colores
\definecolor{vscodepurple}{rgb}{0.5,0,0.5}
\definecolor{vscodeblue}{rgb}{0,0,0.8}
\definecolor{vscodegreen}{rgb}{0,0.5,0}
\definecolor{vscodegray}{rgb}{0.5,0.5,0.5}
\definecolor{vscodebackground}{rgb}{0.97,0.97,0.97}
\definecolor{vscodelightgray}{rgb}{0.9,0.9,0.9}

% Configuración para el estilo de C similar a VSCode
\lstdefinestyle{vscode_C}{
  backgroundcolor=\color{vscodebackground},
  commentstyle=\color{vscodegreen},
  keywordstyle=\color{vscodeblue},
  numberstyle=\tiny\color{vscodegray},
  stringstyle=\color{vscodepurple},
  basicstyle=\scriptsize\ttfamily,
  breakatwhitespace=false,
  breaklines=true,
  captionpos=b,
  keepspaces=true,
  numbers=left,
  numbersep=5pt,
  showspaces=false,
  showstringspaces=false,
  showtabs=false,
  tabsize=2,
  frame=tb,
  framerule=0pt,
  aboveskip=10pt,
  belowskip=10pt,
  xleftmargin=10pt,
  xrightmargin=10pt,
  framexleftmargin=10pt,
  framexrightmargin=10pt,
  framesep=0pt,
  rulecolor=\color{vscodelightgray},
  backgroundcolor=\color{vscodebackground},
}

%------------------------------------------------------------------------

% Comandos definidos
\newcommand{\bb}[1]{\mathbb{#1}}
\newcommand{\cc}[1]{\mathcal{#1}}

% I prefer the slanted \leq
\let\oldleq\leq % save them in case they're every wanted
\let\oldgeq\geq
\renewcommand{\leq}{\leqslant}
\renewcommand{\geq}{\geqslant}

% Si y solo si
\newcommand{\sii}{\iff}

% Letras griegas
\newcommand{\eps}{\epsilon}
\newcommand{\veps}{\varepsilon}
\newcommand{\lm}{\lambda}

\newcommand{\ol}{\overline}
\newcommand{\ul}{\underline}
\newcommand{\wt}{\widetilde}
\newcommand{\wh}{\widehat}

\let\oldvec\vec
\renewcommand{\vec}{\overrightarrow}

% Derivadas parciales
\newcommand{\del}[2]{\frac{\partial #1}{\partial #2}}
\newcommand{\Del}[3]{\frac{\partial^{#1} #2}{\partial^{#1} #3}}
\newcommand{\deld}[2]{\dfrac{\partial #1}{\partial #2}}
\newcommand{\Deld}[3]{\dfrac{\partial^{#1} #2}{\partial^{#1} #3}}


\newcommand{\AstIg}{\stackrel{(\ast)}{=}}
\newcommand{\Hop}{\stackrel{L'H\hat{o}pital}{=}}

\newcommand{\red}[1]{{\color{red}#1}} % Para integrales, destacar los cambios.

% Método de integración
\newcommand{\MetInt}[2]{
    \left[\begin{array}{c}
        #1 \\ #2
    \end{array}\right]
}

% Declarar aplicaciones
% 1. Nombre aplicación
% 2. Dominio
% 3. Codominio
% 4. Variable
% 5. Imagen de la variable
\newcommand{\Func}[5]{
    \begin{equation*}
        \begin{array}{rrll}
            #1:& #2 & \longrightarrow & #3\\
               & #4 & \longmapsto & #5
        \end{array}
    \end{equation*}
}

%------------------------------------------------------------------------

\author{Arturo Olivares Martos}
\date{\today}
\title{Entrega Ejercicios Microcredencial. Parte 2}

\begin{document}
    \maketitle
    \begin{abstract}
        En el presente documento, resolveremos ejercicios de la segunda parte de la Microcredencial de Lógica y Teoría Descriptiva de Conjuntos.
    \end{abstract}

    \begin{ejercicio}
    Demostrar que $(2^\mathbb{N}, d)$ es un espacio completo. \\

    Veamos en primer lugar que $d$ es una distancia.
    \begin{enumerate}
        \item No negatividad:
            \begin{equation*}
                d(x,y) = \frac{1}{2^{n+1}} \geq 0 \qquad \forall x,y\in 2^\mathbb{N}
            \end{equation*}
            Además, se tiene que $d(x,y) = 0$ si y solo si $x = y$.

        \item Simetría:
        
            Sea $n=\min\{k\in \mathbb{N} \mid x(k) \neq y(k)\}=\min\{k\in \mathbb{N} \mid y(k) \neq x(k)\}$, entonces:
            \begin{equation*}
                d(x,y) = \frac{1}{2^{n+1}} = d(y,x)
            \end{equation*}

        \item Desigualdad triangular:
        
            Sean los siguientes tres mínimos, que suponemos que existen (ya que si no existen, la desigualdad se verifica trivialmente):
            \begin{align*}
                n_1 &= \min\{k\in \mathbb{N} \mid x(k) \neq y(k)\} \\
                n_2 &= \min\{k\in \mathbb{N} \mid y(k) \neq z(k)\} \\
                n &= \min\{k\in \mathbb{N} \mid x(k) \neq z(k)\}
            \end{align*}

            Tenemos que $\min\{n_1,n_2\} \leq n$, puesto que para $k<\min\{n_1,n_2\}$, se verifica que $x(k) = y(k)$ y $y(k) = z(k)$, por lo que $x(k) = z(k)$. Por tanto, $n\geq \min\{n_1,n_2\}$. Por tanto:
            \begin{align*}
                d(x,z) &= \frac{1}{2^{n+1}} \leq \frac{1}{2^{\min\{n_1,n_2\}+1}} = \max\left\{\frac{1}{2^{n_1+1}},\frac{1}{2^{n_2+1}}\right\} =\\&= \max\{d(x,y),d(y,z)\}\leq d(x,y) + d(y,z)
            \end{align*}
    \end{enumerate}

    Por tanto, hemos demostrado que $d$ es una distancia, por lo que consideramos el espacio métrico $(2^\mathbb{N},d)$. Este será completo si toda sucesión de Cauchy es convergente a una sucesión de cantor, lo que veremos a continuación.\\

    Sea $\{x_n\}_{n\in \mathbb{N}}$ una sucesión de Cauchy, es decir:
    \begin{equation*}
        \forall \veps\in \mathbb{R}^+\ \exists N\in \mathbb{N}\ \forall m,n\geq N\ d(x_m,x_n) < \veps
    \end{equation*}

    Veamos ahora cómo demostrar que esta sucesión es convergente, para lo cual hemos de construir la sucesión $x$ que sea el límite de la sucesión de Cauchy. Para cada $j\in \bb{N}$, consideramos $\veps=\frac{1}{2^j+1}$, y por tanto, existe $N_j\in \mathbb{N}$ tal que:
    \begin{equation*}
        \forall m,n\geq N_j\qquad d(x_m,x_n) < \frac{1}{2^j+1}
    \end{equation*}

    Por tanto, para cada $m,n\geq N_j$, tenemos que $x_m(j)=x_n(j)$. Definimos por tanto:
    \begin{equation*}
        x(j) = x_{N_j}(j) = x_m(j) \qquad \forall m\geq N_j
    \end{equation*}

    Vemos que $x$ es una sucesión de Cantor, y ahora hemos de demostrar que es el límite de la sucesión de Cauchy. Fijado $\veps\in \mathbb{R}^+$, existe $k\in \mathbb{N}$ tal que $\frac{1}{2^k+1} < \veps$, y podemos considerar $N_k\in \mathbb{N}$ tal que:
    \begin{equation*}
        \forall m,n\geq N_k\ d(x_m,x_n) < \frac{1}{2^k+1}
    \end{equation*}

    Por tanto, para todo $m\geq N_k$, veamos que $x_m(j) = x(j)$ para todo $j\leq k$. Sea $j\leq k$, luego:
    \begin{equation*}
        \frac{1}{2^{k+1}}\leq \frac{1}{2^j+1}\Longrightarrow N_j\leq N_k
    \end{equation*}

    Por tanto, para todo $m\geq N_k\geq N_j$, se tiene que $x_m(j) = x_{N_j}(j) = x(j)$. Por tanto, para todo $m\geq N_k$, se verifica que:
    \begin{equation*}
        d(x_m,x) < \frac{1}{2^{k+1}} < \veps
    \end{equation*}

    Por tanto, hemos demostrado que la sucesión de Cauchy $\{x_n\}_{n\in \mathbb{N}}$ converge a $x\in 2^\mathbb{N}$, y por tanto, $(2^\mathbb{N},d)$ es completo.
\end{ejercicio}


\begin{ejercicio}
    Sea $(X,d)$ un espacio métrico. Dados $x\in X$ y $A\subseteq X$, definimos la distancia entre $a$ y $X$ como:
    \begin{equation*}
        d(x,A) = \inf\{d(x,a) \mid a\in A\}
    \end{equation*}
    Verificar que, dado $r>0$, el siguiente conjunto es un abierto:
    \begin{equation*}
        \{x\in X\mid d(x,A) < r\}
    \end{equation*}
\end{ejercicio}
\begin{proof}
    Dado $x\in X$ con $d(x,A) < r$, veamos que $\exists \veps\in \mathbb{R}^+$ de forma que $B(x,\veps) \subseteq \{x\in X\mid d(x,A) < r\}$.\\

    Sea $\veps=r-d(x,A)>0$, y sea $y\in B(x,\veps)$, es decir, $d(x,y) < \veps$. Veamos que $d(y,A) < r$.
    \begin{align*}
        d(y,a) \leq d(y,x) + d(x,a) \forall a\in A
    \end{align*}

    Por tanto:
    \begin{align*}
        d(y,A) &= \inf\{d(y,a) \mid a\in A\} \\
        &\leq \inf\{d(y,x) + d(x,a) \mid a\in A\}
        = d(y,x) + \inf\{d(x,a) \mid a\in A\} \\
        &= d(y,x) + d(x,A)< \veps + d(x,A) = r-d(x,A) + d(x,A) = r
    \end{align*}

    Por tanto, $y\in \{x\in X\mid d(x,A) < r\}$, y hemos demostrado que:
    \begin{equation*}
        B(x,\veps) \subseteq \{x\in X\mid d(x,A) < r\}
    \end{equation*}
    Por tanto, $\{x\in X\mid d(x,A) < r\}$ es un abierto.
\end{proof}


\begin{ejercicio}
    En el cubo de Hilbert ${[0,1]}^{\mathbb{N}}$, consideramos la métrica $d$ definida como:
    \begin{equation*}
        d(x,y) = \sum_{n=0}^{+\infty} \dfrac{d(x_n,y_n)}{2^n} \qquad \forall x,y\in {[0,1]}^{\mathbb{N}}
    \end{equation*}


    Demostrar que $d$ es una métrica en ${[0,1]}^{\mathbb{N}}$.
    \begin{proof}
        En primer lugar, hemos de ver que la distancia así definida está bien definida, es decir, que la suma converge. Aplicamos para ello el Criterio de Comparación:
        \begin{equation*}
            \sum_{n=0}^{+\infty} \dfrac{d(x_n,y_n)}{2^n} \leq \sum_{n=0}^{+\infty} \dfrac{1}{2^n} = \dfrac{1}{1-\nicefrac{1}{2}} = 2
        \end{equation*}

        Por tanto, $d$ está bien definida. Ahora, veamos que $d$ es una métrica:
        \begin{itemize}
            \item \underline{No-negatividad}: Por definición de $d$, tenemos que:
                \begin{equation*}
                    d(x,y) = \sum_{n=0}^{+\infty} \dfrac{d(x_n,y_n)}{2^n} \geq 0
                \end{equation*}

                Además, se tiene que $d(x,y) = 0$ si y solo si $x = y$.
            \item \underline{Simetría}: Por definición de $d$, tenemos que:
                \begin{equation*}
                    d(x,y) = \sum_{n=0}^{+\infty} \dfrac{d(x_n,y_n)}{2^n} = \sum_{n=0}^{+\infty} \dfrac{d(y_n,x_n)}{2^n} = d(y,x)
                \end{equation*}
            \item \underline{Desigualdad triangular}: Tenemos que:
                \begin{align*}
                    d(x,z) &= \sum_{n=0}^{+\infty} \dfrac{d(x_n,z_n)}{2^n}
                    \leq \sum_{n=0}^{+\infty} \dfrac{d(x_n,y_n) + d(y_n,z_n)}{2^n}
                    = \sum_{n=0}^{+\infty} \dfrac{d(x_n,y_n)}{2^n} + \sum_{n=0}^{+\infty} \dfrac{d(y_n,z_n)}{2^n} \\
                    &= d(x,y) + d(y,z)
                \end{align*}
            \end{itemize}
        Por tanto, hemos visto que $d$ es una métrica en ${[0,1]}^{\mathbb{N}}$.
    \end{proof}
\end{ejercicio}

\end{document}

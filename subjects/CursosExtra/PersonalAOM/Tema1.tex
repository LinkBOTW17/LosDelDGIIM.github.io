\section{Lógica Proposicional}

\begin{comment}
\setcounter{ejercicio}{0}
\begin{ejercicio}\label{ej:1.1}
    Demuestra que el conjunto de proposiciones es numerable.
\end{ejercicio}
\end{comment}

\setcounter{ejercicio}{1}
\begin{ejercicio}\label{ej:1.2}
    Demuestra que las siguientes proposiciones son tautologías.
    \begin{enumerate}
        \item Ley de doble negación: $\neg\neg a \rightarrow a$.
        
        Por el Teorema de la Deducción, esto equivale a demostrar:
        \begin{equation*}
            \{\neg\neg a\} \models a.
        \end{equation*}

        Sea $I$ una interpretación tal que $I(\neg\neg a) = 1$. Entonces:
        \begin{align*}
            1=I(\neg\neg a) &= 1 + I(\neg a) = 1+1+I(a) = 2+I(a) = I(a)
        \end{align*}

        Por tanto, $I(a) = 1$, y por lo tanto, $\{\neg\neg a\} \models a$.
        \item Leyes de simplificación:
        \begin{enumerate}
            \item $(a\land b) \rightarrow a$.
            
            Por el Teorema de la Deducción, esto equivale a demostrar:
            \begin{equation*}
                \{a\land b\} \models a.
            \end{equation*}

            Sea $I$ una interpretación tal que $I(a\land b) = 1$. Entonces:
            \begin{align*}
                1=I(a\land b) &= I(a)I(b)
            \end{align*}
            
            Por ser $\bb{Z}_2$ un cuerpo, en particular es un DI. Si fuese $I(a) = 0$, entonces $I(a)I(b) = 0$, lo cual es una contradicción. Por tanto, $I(a) = 1$, y por lo tanto, $\{a\land b\} \models a$.


            \item $a \rightarrow (a\lor b)$.
            
            Por el Teorema de la Deducción, esto equivale a demostrar:
            \begin{equation*}
                \{a\} \models a\lor b.
            \end{equation*}

            Sea $I$ una interpretación tal que $I(a) = 1$. Entonces:
            \begin{align*}
                I(a\lor b) &= I(a) + I(b) + I(a)I(b) = 1 + I(b) + I(b) = 1 + 2I(b) = 1
            \end{align*}

            Por tanto, $\{a\} \models a\lor b$.
        \end{enumerate}
        
        \item Ley de conmutatividad de la conjunción: $(a\land b) \rightarrow (b\land a)$.
        
        Por el Teorema de la Deducción, esto equivale a demostrar:
        \begin{equation*}
            \{a\land b\} \models b\land a.
        \end{equation*}

        Sea $I$ una interpretación tal que $I(a\land b) = 1$. Entonces:
        \begin{align*}
            1=I(a\land b) &= I(a)I(b)\AstIg = I(b)I(a)=I(b\land a)
        \end{align*}
        donde en $(\ast)$ se usa la conmutatividad de la multiplicación en $\bb{Z}_2$. Por tanto, $\{a\land b\} \models b\land a$.
        \item Ley de conmutatividad de la disyunción: $(a\lor b) \rightarrow (b\lor a)$.
        
        Por el Teorema de la Deducción, esto equivale a demostrar:
        \begin{equation*}
            \{a\lor b\} \models b\lor a.
        \end{equation*}

        Sea $I$ una interpretación tal que $I(a\lor b) = 1$. Entonces:
        \begin{align*}
            I(a\lor b) &= I(a) + I(b) + I(a)I(b) \AstIg I(b) + I(a) + I(b)I(a) = I(b\lor a)
        \end{align*}
        donde en $(\ast)$ se usa la conmutatividad de la suma y la multiplicación en $\bb{Z}_2$. Por tanto, $\{a\lor b\} \models b\lor a$.
        \item Ley de Clavius: $(\neg a \rightarrow a) \rightarrow a$.
        
        Por el Teorema de la Deducción, esto equivale a demostrar:
        \begin{equation*}
            \{\neg a \rightarrow a\} \models a.
        \end{equation*}

        Sea $I$ una interpretación tal que $I(\neg a \rightarrow a) = 1$. Entonces:
        \begin{align*}
            1=I(\neg a \rightarrow a) &= 1+I(\neg a) + I(\neg a)I(a) = 1+1+I(a)+(1+I(a))I(a) =\\&= I(a)+I(a)+I(a)=I(a)
        \end{align*}

        Por tanto, $\{\neg a \rightarrow a\} \models a$.
        \item Ley de De Morgan: $\neg(a\land b) \rightarrow (\neg a \lor \neg b)$.
        
        Por el Teorema de la Deducción, esto equivale a demostrar:
        \begin{equation*}
            \{\neg(a\land b)\} \models \neg a \lor \neg b.
        \end{equation*}

        Sea $I$ una interpretación tal que $I(\neg(a\land b)) = 1$. Entonces:
        \begin{align*}
            1=I(\neg(a\land b)) &= 1+I(a\land b) = 1+I(a)I(b)
            \Longrightarrow
            0=I(a)I(b)
        \end{align*}

        Por tanto:
        \begin{align*}
            I(\neg a \lor \neg b) &= I(\neg a) + I(\neg b) + I(\neg a)I(\neg b) = 1+I(a) + 1+I(b) + (1+I(a))(1+I(b)) =\\&= 1+\cancel{I(a)} + 1+\bcancel{I(b)} + 1+\cancel{I(a)}+\bcancel{I(b)}+I(a)I(b) = 1+I(a)I(b)=1
        \end{align*}

        Por tanto, $\{\neg(a\land b)\} \models \neg a \lor \neg b$.
        \item Segunda ley de De Morgan: $\neg(a\lor b) \rightarrow (\neg a \land \neg b)$.
        
        Por el Teorema de la Deducción, esto equivale a demostrar:
        \begin{equation*}
            \{\neg(a\lor b)\} \models \neg a \land \neg b.
        \end{equation*}

        Sea $I$ una interpretación tal que $I(\neg(a\lor b)) = 1$. Entonces:
        \begin{align*}
            1=I(\neg(a\lor b)) &= 1+I(a\lor b) = 1+I(a) + I(b) + I(a)I(b)
        \end{align*}

        Por tanto:
        \begin{align*}
            I(\neg a \land \neg b) &= I(\neg a)I(\neg b) = (1+I(a))(1+I(b)) = 1+I(a)+I(b)+I(a)I(b) = 1
        \end{align*}

        Por tanto, $\{\neg(a\lor b)\} \models \neg a \land \neg b$.
        \item Ley de inferencia alternativa: $((a\lor b)\land \neg a) \rightarrow b$.
        
        Por el Teorema de la Deducción, esto equivale a demostrar:
        \begin{equation*}
            \{(a\lor b)\land \neg a\} \models b.
        \end{equation*}

        Sea $I$ una interpretación tal que $I((a\lor b)\land \neg a) = 1$. Entonces:
        \begin{align*}
            1=I((a\lor b)\land \neg a) &= I(a\lor b)I(\neg a) = \left(I(a)+I(b)+I(a)I(b)\right)\left(1+I(a)\right) =\\&= \bcancel{I(a)}+I(b)+I(a)I(b)\ +\ \bcancel{I(a)}+\cancel{I(a)I(b)} + \cancel{I(a)I(b)}
            =\\&= I(b)(1+I(a))
        \end{align*}

        Si fuese $I(b) = 0$, entonces $I(b)(1+I(a)) = 0$, lo cual es una contradicción. Por tanto, $I(b) = 1$, y por lo tanto, $\{(a\lor b)\land \neg a\} \models b$.
        \item Segunda ley de inferencia alternativa: $((a\lor b)\land \neg b) \rightarrow a$.
        
        Se tiene de forma directa por el apartado anterior (intercambiando los papeles de $a$ y $b$). La demostración es análoga empleando la conmutatividad de $\bb{Z}_2$.
        \item Modus ponendo ponens: $((a \rightarrow b)\land a) \rightarrow b$.
        
        Por el Teorema de la Deducción, esto equivale a demostrar:
        \begin{equation*}
            \{(a \rightarrow b)\land a\} \models b.
        \end{equation*}

        Sea $I$ una interpretación tal que $I((a \rightarrow b)\land a) = 1$. Entonces:
        \begin{align*}
            1=I((a \rightarrow b)\land a) &= I(a \rightarrow b)I(a) = \left(1+I(a)+I(a)I(b)\right)I(a) =\\&= I(a)+I(a)+I(a)I(b) = I(a)I(b)
        \end{align*}

        Si fuese $I(b) = 0$, entonces $I(a)I(b) = 0$, lo cual es una contradicción. Por tanto, $I(b) = 1$, y por lo tanto, $\{(a \rightarrow b)\land a\} \models b$.
        \item Modus tollendo tollens: $((a \rightarrow b)\land \neg b) \rightarrow \neg a$.
        
        Por el Teorema de la Deducción, esto equivale a demostrar:
        \begin{equation*}
            \{(a \rightarrow b)\land \neg b\} \models \neg a.
        \end{equation*}

        Sea $I$ una interpretación tal que $I((a \rightarrow b)\land \neg b) = 1$. Entonces:
        \begin{align*}
            1=I((a \rightarrow b)\land \neg b) &= I(a \rightarrow b)I(\neg b) = \left(1+I(a)+I(a)I(b)\right)(1+I(b)) =\\&= 1+I(a)+I(a)I(b)+I(b)+\cancel{I(a)I(b)}+\cancel{I(a)I(b)} =\\&= 1+I(a)+I(b)+I(a)I(b)
            =\\&= (1+I(b)) + I(a)(1+I(b))
            = (1+I(a))(1+I(b))
        \end{align*}
        Si fuese $1+I(a) = 0$, entonces $(1+I(a))(1+I(b)) = 0$, lo cual es una contradicción. Por tanto, $1+I(a) = 1 = I(\neg a)$. Por tanto, $\{(a \rightarrow b)\land \neg b\} \models \neg a$.
    \end{enumerate}
\end{ejercicio}

\begin{ejercicio}\label{ej:1.3}
    Dado un conjunto de proposiciones $\Gamma \cup \{\alpha, \beta\}$. Si $\Gamma \cup \{\alpha\} \models \beta$, entonces $\Gamma \models \alpha \rightarrow \beta$.\\

    Notemos que en este caso tan solo tenemos una de las implicaciones del Teorema de la Deducción. Sea $I$ una interpretación tal que $I(\gamma)=1$ para todo $\gamma\in\Gamma$.
    \begin{itemize}
        \item Si $I(\alpha)=1$, como $\Gamma \cup \{\alpha\} \models \beta$, entonces $I(\beta)=1$. Por tanto:
        \begin{align*}
            I(\alpha\rightarrow\beta) &= 1+I(\alpha)+I(\alpha)I(\beta) = 1+1+1 = 1
        \end{align*}

        \item Si $I(\alpha)=0$, entonces:
        \begin{align*}
            I(\alpha\rightarrow\beta) &= 1+I(\alpha)+I(\alpha)I(\beta) = 1+0+0 = 1
        \end{align*}
    \end{itemize}

    Por tanto, $\Gamma \models \alpha \rightarrow \beta$.
\end{ejercicio}

\begin{ejercicio}\label{ej:1.4}
    El señor Pérez, empadronador de la isla de Tururulandia, tiene como objetivo el censar la población de dicha isla. La tarea no es fácil debido al hecho de que la población se divide en dos grupos bien distinguidos: los honrados y los embusteros. Los honrados siempre dicen la verdad, mientras que un embustero solo es capaz de producir mentiras. El gobierno de la isla encarga como trabajo al señor Pérez la ardua tarea de contar los honrados y embusteros de la isla.
    He aquí cuatro de los muchos problemas con los que se encontró nuestro empadronador.
    \begin{enumerate}
        \item Llama a la puerta de una casa, en la que sabía a ciencia cierta que vivía un matrimonio, y el marido abre la puerta para ver quién es. El empadronador le dice: ``necesito información sobre usted y su esposa. ¿Cuál de ustedes, si alguno lo es, es honrado y cuál un embustero?,'' a lo que el hombre de la casa respondió ``ambos somos embusteros,'' cerrando la puerta de golpe. ¿Qué es el marido y qué es la mujer?
        
        Sean las siguientes proposiciones atómicas:
        \begin{itemize}
            \item $p\equiv$ ``El marido es honrado.''
            \item $q\equiv$ ``La mujer es honrada.''
        \end{itemize}

        Para empezar, sabemos que $p\longleftrightarrow \neg p\land \neg q$ es cierto. Por tanto, fijada una interpretación $I$ tal que $I(p\longleftrightarrow (\neg p\land \neg q))=1$, entonces:
        \begin{align*}
            1 &= I(p\longleftrightarrow (\neg p\land \neg q))
            = 1+I(p) + I(\neg p\land \neg q)
            = 1+I(p) + I(\neg p)I(\neg q)
            =\\&= 1+I(p) + (1+I(p))(1+I(q))
            = (1+I(p))(I(q))
        \end{align*}

        Por tanto:
        \begin{itemize}
            \item $I(p) = 0\Longrightarrow$ El marido es un embustero.
            \item $I(q) = 1\Longrightarrow$ La mujer es honrada.
        \end{itemize}
        \item La segunda casa que visita también está habitada por un matrimonio. Al llamar a la puerta y formular la misma pregunta que antes, el marido responde: ``Por lo menos uno de nosotros es un embustero,'' cerrando a continuación la puerta. ¿Qué es el marido y qué es la mujer?
        
        Sean las siguientes proposiciones atómicas:
        \begin{itemize}
            \item $p\equiv$ ``El marido es honrado.''
            \item $q\equiv$ ``La mujer es honrada.''
        \end{itemize}

        Para empezar, sabemos que $p\longleftrightarrow \neg p\lor \neg q$ es cierto. Por tanto, fijada una interpretación $I$ tal que $I(p\longrightarrow (\neg p\lor \neg q))=1$, entonces:
        \begin{align*}
            1 &= I(p\longleftrightarrow (\neg p\lor \neg q))
            = 1+I(p) + I(\neg p\lor \neg q)
            =\\&= 1+I(p) + I(\neg p)+I(\neg q) + I(\neg p)I(\neg q)
            =\\&= \cancel{1+I(p)} + \cancel{1+I(p)} + 1+I(q) + (1+I(p))(1+I(q))
            =\\&= (1+I(q))(1+1+I(p)) = I(p)(1+I(q))
        \end{align*}

        Por tanto:
        \begin{itemize}
            \item $I(p) = 1\Longrightarrow$ El marido es honrado.
            \item $I(q) = 0\Longrightarrow$ La mujer es una embustera.
        \end{itemize}
        \item Visita una tercera casa, y en las mismas condiciones de antes, recibe la respuesta: ``Si yo soy honrado, entonces mi mujer también lo es.'' ¿Qué es el marido y qué es la mujer?
        
        Sean las siguientes proposiciones atómicas:
        \begin{itemize}
            \item $p\equiv$ ``El marido es honrado.''
            \item $q\equiv$ ``La mujer es honrada.''
        \end{itemize}

        Para empezar, sabemos que $p\longleftrightarrow (p\longrightarrow q)$ es cierto. Por tanto, fijada una interpretación $I$ tal que $I(p\longleftrightarrow (p\longrightarrow q))=1$, entonces:
        \begin{align*}
            1 &= I(p\longleftrightarrow (p\longrightarrow q))
            = 1+I(p) + I(p\longrightarrow q)
            =\\&= 1+I(p) + 1+I(p)+I(p)I(q)
            = I(p)I(q)
        \end{align*}

        Por tanto:
        \begin{itemize}
            \item $I(p) = 1\Longrightarrow$ El marido es honrado.
            \item $I(q) = 1\Longrightarrow$ La mujer es honrada.
        \end{itemize}
        \item En la última casa que visita, pues ya estaba cansado de partirse el coco, la respuesta es ``Yo soy lo mismo que mi mujer.'' ¿Qué es el marido y qué es la mujer?
        Sean las siguientes proposiciones atómicas:
        \begin{itemize}
            \item $p\equiv$ ``El marido es honrado.''
            \item $q\equiv$ ``La mujer es honrada.''
        \end{itemize}

        Para empezar, sabemos que $p\longleftrightarrow (p\longleftrightarrow q)$ es cierto. Por tanto, fijada una interpretación $I$ tal que $I(p\longleftrightarrow (p\longleftrightarrow q))=1$, entonces:
        \begin{align*}
            1 &= I(p\longleftrightarrow (p\longleftrightarrow q))
            = 1+I(p) + I(p\longleftrightarrow q)
            =\\&= 1+I(p) + 1+I(p)+I(q)
            = I(q)
        \end{align*}

        Por tanto:
        \begin{itemize}
            \item $I(q) = 1\Longrightarrow$ La mujer es honrada.
            \item Respecto al marido, no podemos determinar si es honrado o no.
        \end{itemize}
        

        \item De vuelta a su casa se encuentra con tres individuos, A, B y C, en la calle, y pensando en que quizás podía tener más suerte con ellos decide preguntarles qué son cada uno de ellos. Le pregunta al primero, A, y no entiende la respuesta, ya que en ese momento pasa una de esas motos que hacen un ruido ensordecedor y no corren nada. El segundo, B, le aclara que lo que ha dicho el primero es que es un embustero, pero el tercero, C, le advierte que no haga caso del segundo, B, ya que es un embustero. ¿Puedes deducir algo de lo ocurrido?
        
        Sean las siguientes proposiciones atómicas:
        \begin{itemize}
            \item $a\equiv$ ``A es honrado.''
            \item $b\equiv$ ``B es honrado.''
            \item $c\equiv$ ``C es honrado.''
            \item $p\equiv$ ``A dice que es un embustero.''
        \end{itemize}

        Sabemos que:
        \begin{itemize}
            \item $a\longrightarrow \neg p$ es cierto.
            \item $\neg a\longrightarrow \neg p$ es cierto.
            \item $b\longleftrightarrow p$ es cierto.
            \item $c\longleftrightarrow \neg b$ es cierto.
        \end{itemize}

        Tenemos que:
        \begin{align*}
            1&= I(a\longrightarrow \neg p) = 1+I(a)+I(a)I(\neg p)\\
            1&= I(\neg a\longrightarrow \neg p) = 1+I(\neg a)+I(\neg a)I(\neg p)
            =\\&= 1+1+I(a) + (1+I(a))I(\neg p) =
            1+1+I(a)+I(\neg p)+I(a)I(\neg p)=\\&=1+I(\neg p)+1=I(\neg p)\Longrightarrow
            I(p)=0
        \end{align*}

        Además, sabemos que:
        \begin{align*}
            1&= I(b\longleftrightarrow p) = 1+I(b)+I(p)=1+I(b)\Longrightarrow
            I(b)=0\\
            1&= I(c\longleftrightarrow \neg b) = 1+I(c)+I(\neg b)=1+I(c)+1+I(b)=I(c)
        \end{align*}

        Por tanto:
        \begin{itemize}
            \item $I(b)=0\Longrightarrow$ B es un embustero.
            \item $I(c)=1\Longrightarrow$ C es honrado.
            \item Desconocemos el valor de $I(a)$, por lo que no podemos determinar si A es honrado o no.
        \end{itemize}
    \end{enumerate}
\end{ejercicio}


\begin{ejercicio}\label{ej:1.5}
    Probar que todo axioma del cálculo proposicional clásico es una tautología.

    Sean $\alpha,\beta,\gamma$ fórmulas proposicionales. Entonces, probaremos que cada uno de los axiomas del cálculo proposicional clásico es una tautología.
    \begin{enumerate}
        \item $\cc{A}_1 = \{\alpha\rightarrow(\beta\rightarrow\alpha)\}$.
        
        Aplicando dos veces el Teorema de la Deducción, hemos de probar que:
        \begin{equation*}
            \{\alpha,\beta\} \models \alpha
        \end{equation*}

        Sea $I$ una interpretación tal que $I(\alpha) = I(\beta) = 1$. Entonces, trivialmente $I(\alpha) = 1$. Por tanto, $\{\alpha,\beta\} \models \alpha$.
        \item $\cc{A}_2 = \{(\alpha\rightarrow(\beta\rightarrow\gamma))\rightarrow((\alpha\rightarrow\beta)\rightarrow(\alpha\rightarrow\gamma))\}$.
        
        Aplicando tres veces el Teorema de la Deducción, hemos de probar que:
        \begin{equation*}
            \{\alpha\rightarrow(\beta\rightarrow\gamma),\alpha\rightarrow\beta,\alpha\} \models \gamma
        \end{equation*}

        Sea $I$ una interpretación tal que $I\left(\alpha\rightarrow(\beta\rightarrow\gamma)\right) = I\left(\alpha\rightarrow\beta\right) = I(\alpha) = 1$. Entonces:
        \begin{align*}
            1&= I(\alpha\rightarrow\beta) = 1+I(\alpha)+I(\alpha)I(\beta)=1+1+I(\beta)\Longrightarrow I(\beta)=1\\
            1&=I\left(\alpha\rightarrow(\beta\rightarrow\gamma)\right) = 1+I(\alpha)+I(\alpha)I(\beta\rightarrow\gamma) = I(\beta\rightarrow\gamma)
            = 1+I(\beta)+I(\beta)I(\gamma) = I(\gamma)
        \end{align*}

        Por tanto, $\{\alpha\rightarrow(\beta\rightarrow\gamma),\alpha\rightarrow\beta,\alpha\} \models \gamma$.
        \item $\cc{A}_3 = \{(\neg\alpha\rightarrow\neg\beta)\rightarrow\left((\neg\alpha\rightarrow\beta)\rightarrow\alpha\right)\}$.
        
        Aplicando dos veces el Teorema de la Deducción, hemos de probar que:
        \begin{equation*}
            \{\neg\alpha\rightarrow\neg\beta,\neg\alpha\rightarrow\beta\} \models \alpha
        \end{equation*}

        Sea $I$ una interpretación tal que $I(\neg\alpha\rightarrow\neg\beta) = I(\neg\alpha\rightarrow\beta) = 1$. Entonces:
        \begin{align*}
            1&=I(\neg\alpha\rightarrow\beta) = 1+I(\neg\alpha)+I(\neg\alpha)I(\beta)\\
            1&=I(\neg\alpha\rightarrow\neg\beta) = 1+I(\neg\alpha)+I(\neg\alpha)I(\neg\beta) = 1+I(\neg\alpha)+I(\neg\alpha)\left(1+I(\beta)\right)
            =\\&= 1+I(\neg\alpha)+I(\neg\alpha)+I(\neg\alpha)I(\beta)=1+I(\neg\alpha)=1+1+I(\alpha)=I(\alpha)
        \end{align*}

        Por tanto, $\{\neg\alpha\rightarrow\neg\beta,\neg\alpha\rightarrow\beta\} \models \alpha$.
    \end{enumerate}
\end{ejercicio}

\begin{comment}
\setcounter{ejercicio}{5}
\begin{ejercicio}[Regla de ``reductio ad absurdum'' minimal o intuicionista]\label{ej:1.6}
    Si $\Gamma \cup \{\alpha\} \vdash \beta$ y $\Gamma \cup \{\alpha\} \vdash \neg\beta$, entonces $\Gamma \vdash \neg\alpha$.
\end{ejercicio}
\end{comment}

\begin{comment}
\setcounter{ejercicio}{6}
\begin{ejercicio}[Leyes de Duns Scoto]\label{ej:1.7}
    \begin{enumerate}
        \item $\vdash \neg\alpha \rightarrow (\alpha \rightarrow \beta)$.
        \item $\vdash \alpha \rightarrow (\neg\alpha \rightarrow \beta)$.
    \end{enumerate}
\end{ejercicio}
\end{comment}

\begin{comment}
\setcounter{ejercicio}{7}
\begin{ejercicio}[Principio de inconsistencia]\label{ej:1.8}
    Si $\Gamma \vdash \alpha$ y $\Gamma \vdash \neg\alpha$, entonces $\Gamma \vdash \beta$.
\end{ejercicio}
\end{comment}

\begin{comment}
\setcounter{ejercicio}{8}
\begin{ejercicio}[Leyes débiles de Duns Scoto]\label{ej:1.9}
    \begin{enumerate}
        \item $\vdash \neg\alpha \rightarrow (\alpha \rightarrow \neg\beta)$.
        \item $\vdash \alpha \rightarrow (\neg\alpha \rightarrow \neg\beta)$.
    \end{enumerate}
\end{ejercicio}
\end{comment}

\begin{comment}
\setcounter{ejercicio}{9}
\begin{ejercicio}[Principio de inconsistencia débil]\label{ej:1.10}
    Si $\Gamma \vdash \alpha$ y $\Gamma \vdash \neg\alpha$, entonces $\Gamma \vdash \neg\beta$.
\end{ejercicio}
\end{comment}

\begin{comment}
\setcounter{ejercicio}{10}
\begin{ejercicio}[Ley de contraposición fuerte o ``ponendo ponens'']\label{ej:1.11}
    $\vdash (\neg\beta \rightarrow \neg\alpha) \rightarrow (\alpha \rightarrow \beta)$.
\end{ejercicio}
\end{comment}

\begin{comment}
\setcounter{ejercicio}{11}
\begin{ejercicio}[Ley de contraposición ``ponendo tollens'']\label{ej:1.12}
    $\vdash (\beta \rightarrow \neg\alpha) \rightarrow (\alpha \rightarrow \neg\beta)$.
\end{ejercicio}
\end{comment}

\begin{comment}
\setcounter{ejercicio}{12}
\begin{ejercicio}[Ley de contraposición ``tollendo ponens'']\label{ej:1.13}
    $\vdash (\neg\alpha \rightarrow \beta) \rightarrow (\neg\beta \rightarrow \alpha)$.
\end{ejercicio}
\end{comment}

\begin{comment}
\setcounter{ejercicio}{13}
\begin{ejercicio}[Ley de contraposición débil o ``tollendo tollens'']\label{ej:1.14}
    $\vdash (\alpha \rightarrow \beta) \rightarrow (\neg\beta \rightarrow \neg\alpha)$.
\end{ejercicio}
\end{comment}

\begin{comment}
\setcounter{ejercicio}{14}
\begin{ejercicio}[Regla de prueba por casos]\label{ej:1.15}
    Si $\Gamma \cup \{\alpha\} \vdash \beta$ y $\Gamma \cup \{\neg\alpha\} \vdash \beta$, entonces $\Gamma \vdash \beta$.
\end{ejercicio}
\end{comment}

\begin{comment}
\setcounter{ejercicio}{15}
\begin{ejercicio}[Ley débil de Clavius]\label{ej:1.16}
    $\vdash (\alpha \rightarrow \neg\alpha) \rightarrow \neg\alpha$.
\end{ejercicio}
\end{comment}

\begin{comment}
\setcounter{ejercicio}{16}
\begin{ejercicio}[Ley de Clavius]\label{ej:1.17}
    $\vdash (\neg\alpha \rightarrow \alpha) \rightarrow \alpha$.
\end{ejercicio}
\end{comment}

\begin{comment}
\setcounter{ejercicio}{17}
\begin{ejercicio}[Regla de retorsión, regla de Clavius]\label{ej:1.18}
    Si $\Gamma \cup \{\neg\alpha\} \vdash \alpha$, entonces $\Gamma \vdash \alpha$.
\end{ejercicio}
\end{comment}

\begin{comment}
\setcounter{ejercicio}{18}
\begin{ejercicio}[Leyes de adjunción]\label{ej:1.19}
    \begin{enumerate}
        \item $\vdash \alpha \rightarrow \alpha \lor \beta$.
        \item $\vdash \beta \rightarrow \alpha \lor \beta$.
    \end{enumerate}
\end{ejercicio}
\end{comment}

\begin{comment}
\setcounter{ejercicio}{19}
\begin{ejercicio}[Reglas de adjunción o de introducción de la disyunción]\label{ej:1.20}
    \begin{enumerate}
        \item Si $\Gamma \vdash \alpha$, entonces $\Gamma \vdash \alpha \lor \beta$.
        \item Si $\Gamma \vdash \beta$, entonces $\Gamma \vdash \alpha \lor \beta$.
    \end{enumerate}
\end{ejercicio}
\end{comment}

\begin{comment}
\setcounter{ejercicio}{20}
\begin{ejercicio}[Ley conmutativa de la disyunción]\label{ej:1.21}
    $\vdash \alpha \lor \beta \rightarrow \beta \lor \alpha$.
\end{ejercicio}
\end{comment}

\begin{comment}
\setcounter{ejercicio}{21}
\begin{ejercicio}\label{ej:1.22}
    \begin{enumerate}
        \item $\vdash \alpha \land \beta \rightarrow \alpha$.
        \item $\vdash \alpha \land \beta \rightarrow \beta$.
    \end{enumerate}
\end{ejercicio}
\end{comment}

\begin{comment}
\setcounter{ejercicio}{22}
\begin{ejercicio}[Reglas de simplificación o de eliminación de la conjunción]\label{ej:1.23}
    Si $\Gamma \vdash \alpha \land \beta$, entonces $\Gamma \vdash \alpha$ y $\Gamma \vdash \beta$.
\end{ejercicio}
\end{comment}

\begin{comment}
\setcounter{ejercicio}{23}
\begin{ejercicio}\label{ej:1.24}
    $\vdash (\alpha \rightarrow \gamma) \rightarrow ((\beta \rightarrow \gamma) \rightarrow (\alpha \lor \beta \rightarrow \gamma))$.
\end{ejercicio}
\end{comment}

\begin{comment}
\setcounter{ejercicio}{24}
\begin{ejercicio}[Otra regla de prueba por casos]\label{ej:1.25}
    Si $\Gamma \cup \{\alpha\} \vdash \gamma$ y $\Gamma \cup \{\beta\} \vdash \gamma$, entonces $\Gamma \cup \{\alpha \lor \beta\} \vdash \gamma$.
\end{ejercicio}
\end{comment}

\begin{comment}
\setcounter{ejercicio}{25}
\begin{ejercicio}[Ley de Peirce]\label{ej:1.26}
    $\vdash ((\alpha \rightarrow \beta) \rightarrow \alpha) \rightarrow \alpha$.
\end{ejercicio}
\end{comment}

\begin{comment}
\setcounter{ejercicio}{26}
\begin{ejercicio}\label{ej:1.27}
    $\vdash \alpha \rightarrow (\beta \rightarrow \alpha \land \beta)$.
\end{ejercicio}
\end{comment}

\begin{comment}
\setcounter{ejercicio}{27}
\begin{ejercicio}[Regla del producto o de introducción de la conjunción]\label{ej:1.28}
    Si $\Gamma \vdash \alpha$ y $\Gamma \vdash \beta$, entonces $\Gamma \vdash \alpha \land \beta$.
\end{ejercicio}
\end{comment}
\subsection{Relación 2}
\setcounter{ejercicio}{0}

\begin{ejercicio}
    Dado el problema de valores iniciales
    \begin{equation*}
        \begin{cases}
            x'(t) = f(t, x(t)) \\
            x(a) = \mu
        \end{cases}
    \end{equation*}
    se pretende utilizar el siguiente método numérico para estimar el valor de $x(b)$, con $b > a$:
    \begin{equation*}
        x_{n+1} = x_n + h \left( \alpha f(t_n, x_n) + (1 - \alpha) f(t_{n+1}, x_{n+1}) \right), \quad \alpha \in [0, 1]
    \end{equation*}
    \begin{enumerate}
        \item ¿Podemos asegurar que el método es estable? Estudia su consistencia y su convergencia.
        
        Su polinomio característico es:
        \begin{equation*}
            p(\lm) = \lm -1
        \end{equation*}

        Por lo tanto, la única raíz es $\lm = 1$, que es simple y tiene módulo 1. Por tanto, el método es estable.
        Respecto a la consistencia, necesitamos que:
        \begin{itemize}
            \item $p(1) = 0$, que se tiene.
            \item $\Phi(x(t_n), x(t_n), t_n, 0) = p'(1)f(t_n, x(t_n))$:
            \begin{equation*}
                \Phi(x(t_n), x(t_n), t_n, 0) = \alpha f(t_n, x(t_n)) + (1 - \alpha) f(t_n, x(t_n))
                = f(t_n, x(t_n)) = p'(1)f(t_n, x(t_n))
            \end{equation*}
        \end{itemize}
        \item Si sabemos que la función $f$ es lipschitziana respecto a la segunda variable con constante de Lipschitz $L$, ¿cuánto debe valer $h$ como máximo para que la ecuación tenga solución para cualquier valor de $n$?
        \item\label{ap:alpha}
        Determina el valor de $\alpha$ para que el método tenga orden 2. ¿Cuál es en este caso el error de truncatura local? ¿De qué orden es el error global de discretización?
        \item Para el valor obtenido en el apartado anterior, estudia si el método es A-estable. ¿Es el método A-estable para cualquier valor de $\alpha$?
        \item ¿De qué orden sería un método predictor-corrector en el que el predictor es el método de Euler y el corrector es el correspondiente al apartado~\ref{ap:alpha}?
        \item Aplica todo lo anterior al problema:
            \begin{equation*}
                \begin{cases}
                    x'(t) = -3x + t \\
                    x(0) = 0.3
                \end{cases}
            \end{equation*}
            para estimar el valor de $x(1)$. Realiza cuatro iteraciones haciendo en cada una una predicción con el método de Euler y una corrección con el método descrito en el apartado~\ref{ap:alpha}. Muestra todas las iteraciones.
    \end{enumerate}
\end{ejercicio}

\begin{ejercicio}
    Dado el problema de valores iniciales
    \begin{equation*}
        \begin{cases}
            x'(t) = f(t, x(t)) \\
            x(a) = \mu
        \end{cases}
    \end{equation*}
    se pretende utilizar el siguiente método numérico para estimar el valor de $x(b)$, con $b > a$:
    \begin{equation*}
        x_{n+2} = x_n + h \left( \beta_0 f(t_n, x_n) + \beta_1 f(t_{n+1}, x_{n+1}) + \beta_2 f(t_{n+2}, x_{n+2}) \right)
    \end{equation*}
    \begin{enumerate}
        \item ¿Podemos asegurar que el método es estable? Estudia su consistencia y su convergencia.
        \item Determina el valor de los parámetros para que el método tenga orden al menos 3. ¿Es algún método conocido? ¿Cuál es en este caso el error de truncatura local? ¿De qué orden es el error global de discretización?
        \item ¿De qué orden sería un método predictor-corrector en el que el predictor es el método de Runge-Kutta óptimo de dos evaluaciones y el corrector es el correspondiente al apartado anterior?
        \item Aplica todo lo anterior al problema:
            \begin{equation*}
                \begin{cases}
                    x'(t) = -3x + t \\
                    x(0) = 0.3
                \end{cases}
            \end{equation*}
            para estimar el valor de $x(1)$. Realiza cuatro iteraciones haciendo en cada una predicción con el método de Runge-Kutta óptimo de dos evaluaciones y una corrección con el método descrito en el apartado anterior. Si necesitas algún valor inicial para empezar el método predictor-corrector, utiliza también Runge-Kutta óptimo de dos evaluaciones para estimarlo. Muestra los resultados de todas las iteraciones que realices.
    \end{enumerate}
\end{ejercicio}

\begin{ejercicio}
    Dado el problema de valores iniciales
    \begin{equation*}
        \begin{cases}
            x'(t) = f(t, x(t)) \\
            x(a) = \mu
        \end{cases}
    \end{equation*}
    se pretende utilizar el siguiente método numérico para estimar el valor de $x(b)$, con $b > a$:
    \begin{equation*}
        x_{n+2} = \alpha_1 x_{n+1} + \alpha_0 x_n + h \left( \beta_1 f(t_{n+1}, x_{n+1}) + \beta_0 f(t_n, x_n) \right)
    \end{equation*}
    \begin{enumerate}
        \item Determina el valor de los parámetros (en función del parámetro $\alpha_1$) para que el método tenga orden al menos 2. ¿Sería consistente en ese caso?
        \item Estima el error de truncatura local (también en función de $\alpha_1$). ¿De qué orden es el error global de discretización?
        \item Estudia la estabilidad y la convergencia en función del parámetro $\alpha_1$.
        \item Si $\alpha_1 = 0$, ¿encuentras relación con algún método conocido? ¿Y en el caso $\alpha_1 = 1$?
        \item Utiliza este método con $\alpha_1 = \nicefrac{1}{2}$ en el problema de valores iniciales
            \begin{equation*}
                \begin{cases}
                    x'(t) = -3x + t \\
                    x(0) = 0.3
                \end{cases}
            \end{equation*}
            para estimar el valor de $x(1)$. Realiza cuatro iteraciones del método haciendo uso del método de Euler para calcular los datos iniciales que necesites. Muestra todas las iteraciones.
    \end{enumerate}
\end{ejercicio}

\begin{ejercicio}~
    \begin{enumerate}
        \item Para el PVI $x' = x - |t - 2|$, $x(0) = 1$, $t \in [0, 1]$ con $h = 0.1$, calcula $x_1$ con el MML asociado al arreglo de Butcher
            \begin{equation*}
                \begin{array}{c|ccc}
                    1 & 1 & 0 & 0 \\
                    1 & 1 & 0 & 0 \\
                    1 & 0 & 1 & 0 \\ \hline
                    & \nicefrac{1}{2} & \nicefrac{1}{6} & \nicefrac{1}{3}
                \end{array}
            \end{equation*}
        \item Estudia las propiedades (estabilidad, consistencia, convergencia, orden, parte principal del error de truncatura local) del método
        \begin{equation*}
            x_{n+1} = \frac{2}{3} x_n + \frac{1}{3} x_{n-1} + \frac{h}{3} (5 f_n - f_{n-1}).
        \end{equation*}
    \end{enumerate}
\end{ejercicio}

\begin{ejercicio}
    Utiliza la fórmula de cuadratura del trapecio para deducir la fórmula del trapecio para resolución de PVI así como el error de truncatura local. Estudia también la A-estabilidad del método.
\end{ejercicio}

\begin{ejercicio}
    Dado el método multipaso lineal
    \begin{equation*}
        x_{n+2} = \frac{2a + 1}{2} x_{n+1} - \frac{a}{2} x_n + h \left( \beta_0 f(t_n, x_n) + \beta_2 f(t_{n+2}, x_{n+2}) \right)
    \end{equation*}
    \begin{enumerate}
        \item Estudia la convergencia del método.
        \item Determina el valor de los parámetros para que el método sea convergente y tenga el mayor orden posible. ¿Cuál es ese orden? Indica el término principal del error de truncatura local en este caso.
        \item Para el PVI
            \begin{equation*}
                \begin{cases}
                    x'(t) = -3x + t^2 \\
                    x(0) = 1
                \end{cases}
            \end{equation*}
            toma $h = 0.1$ y utiliza el método de Euler para aproximar $x(0.1)$. Realiza a continuación dos iteraciones del método que has obtenido en b) para aproximar $x(0.3)$.
    \end{enumerate}
\end{ejercicio}

\begin{ejercicio}
    Para aproximar la solución del PVI
    \begin{equation*}
        \begin{cases}
            x' = f(t, x) \\
            x(t_0) = \mu
        \end{cases}
    \end{equation*}
    se considera el método multipaso
    \begin{equation*}
        x_{n+3} = x_n + h \left( \beta_1 f(t_{n+1}, x_{n+1}) + \beta_2 f(t_{n+2}, x_{n+2}) \right)
    \end{equation*}
    \begin{enumerate}
        \item ¿Qué relaciones deben existir entre los parámetros $\beta_1$ y $\beta_2$ para que el método anterior sea convergente?
        \item Calcula los coeficientes $\beta_1$ y $\beta_2$ para que el método sea convergente y tenga orden máximo. Indica el error de convergencia local en este caso.
        \item Dado el PVI:
            \begin{equation*}
                \begin{cases}
                    x' = -5x + t^2 \\
                    x(0) = 1
                \end{cases}
            \end{equation*}
            realiza dos iteraciones del método de Euler y a continuación, usando esos valores como semilla, realiza otras dos iteraciones del método propuesto con $h = 0.1$.
    \end{enumerate}
\end{ejercicio}

\begin{ejercicio}
    Dado el PVI
    \begin{equation*}
        \begin{cases}
            x' = f(t, x) \\
            x(t_0) = \mu
        \end{cases}
    \end{equation*}
    utiliza la fórmula
    \begin{equation*}
        \int_a^{a+h} f(x)dx = \frac{5h}{12} f(a + h) + \frac{2h}{3} f(a) - \frac{h}{12} f(a - h) + R(f)
    \end{equation*}
    para construir razonadamente un método lineal multipaso de la forma
    \begin{equation*}
        x_{n+2} = x_{n+1} + h(\beta_2 f_{n+2} + \beta_1 f_{n+1} + \beta_0 f_n)
    \end{equation*}
    y contesta a las siguientes preguntas:
    \begin{enumerate}
        \item ¿Es el método convergente?
        \item ¿Cuál es el orden de convergencia local del método?
        \item Si queremos utilizar el método de Euler como predictor y este método como corrector para resolver el problema, ¿cuál es el número óptimo de correcciones que se deberían aplicar?
        \item Se pretende aproximar $x(1)$ donde $x(t)$ es la solución del PVI
            \begin{equation*}
                \begin{cases}
                    x' = 3x - 2 \\
                    x(0) = 1
                \end{cases}
            \end{equation*}
            Para ello, tomando $h = \nicefrac{1}{3}$, utiliza el método de Euler para la primera iteración. A continuación utiliza un método predictor-corrector donde el predictor es el método de Euler y el corrector es el método anterior con una única corrección en cada paso.
    \end{enumerate}
\end{ejercicio}

\begin{ejercicio}
    Para resolver el PVI
    \begin{equation*}
        \begin{cases}
            x' = f(t, x) \\
            x(t_0) = \mu
        \end{cases}
    \end{equation*}
    se propone el método:
    \begin{equation*}
        x_{n+1} = x_n + \frac{3}{2} h f(t_{n+1}, x_{n+1}) - \frac{1}{2} h f(t_n, x_n)
    \end{equation*}
    Estudia su A-estabilidad.
\end{ejercicio}

\begin{ejercicio}
    Para resolver numéricamente el PVI
    \begin{equation*}
        \begin{cases}
            x' = f(t, x) \\
            x(t_0) = \mu
        \end{cases}
    \end{equation*}
    se propone el método de Runge-Kutta Radau dado por el arreglo de Butcher
    \begin{equation*}
        \begin{array}{c@ccc}
            0 & \frac{1}{4} & -\frac{1}{4} \\
            \frac{2}{3} & \frac{1}{4} & \frac{5}{12} \\ 
            \frac{1}{4} & \frac{3}{4} \\ \hline
            & & 1
        \end{array}
    \end{equation*}
    Estudia la convergencia del método.
    
    Nota: No es necesario que compruebes que $\Phi$ es Lipschitziana.
\end{ejercicio}


\begin{ejercicio}
    Para aproximar la solución del PVI
    \begin{equation*}
        \begin{cases}
            x' = f(t, x) \\
            x(t_0) = \mu
        \end{cases}
    \end{equation*}
    se considera el método multipaso
    \begin{equation*}
        x_{n+3} = x_n + h(\beta_2 f_{n+2} + \beta_1 f_{n+1})
    \end{equation*}
    \begin{enumerate}
        \item ¿Qué relaciones deben existir entre los parámetros $\beta_1$ y $\beta_2$ para que el método anterior sea convergente? Justifica tu respuesta.
        \item Calcula los coeficientes $\beta_1$ y $\beta_2$ para que el orden de convergencia sea máximo. Indica el orden de convergencia y el término principal del error de curvatura local.
        \item Se pretende aproximar $x(1)$ donde $x(t)$ es la solución del PVI
            \begin{equation*}
                \begin{cases}
                    x' = x + t \\
                    x(0) = 1
                \end{cases}
            \end{equation*}
            Para ello, tomando $h = \nicefrac{1}{4}$, utiliza el método de Euler para obtener las condiciones iniciales que necesites. A continuación utiliza un método anterior hasta aproximar $x(1)$.
    \end{enumerate}
\end{ejercicio}
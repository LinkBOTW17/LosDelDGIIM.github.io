\subsection{Relación 2. Derivación Numérica}
\setcounter{ejercicio}{0}


\begin{ejercicio}\label{ej:2.2.1}~
    \begin{enumerate}
        \item Obtén la fórmula progresiva de derivación numérica de tipo interpolatorio clásico para aproximar $f'(a)$ a partir de $f(a)$ y $f(a + h)$, mediante desarrollo de Taylor de $f(a + h)$ en torno a $a$ hasta el cuarto término.
        
        Buscamos obtener $\alpha_0,\alpha_1\in \bb{R}$ tales que:
        \begin{equation*}
            f'(a) = \alpha_0 f(a) + \alpha_1 f(a + h) + R(f)
        \end{equation*}

        Desarrollando en serie de Taylor $f(a)$ y $f(a + h)$ en torno a $a$ hasta el cuarto término, tenemos:
        \begin{align*}
            f(a) &= f(a) \\
            f(a + h) &= f(a) + hf'(a) + \frac{h^2}{2}f''(a) + \frac{h^3}{6}f'''(a) + \frac{h^4}{4!}f^{(4)}(a) + \dfrac{h^5}{5!}f^{(5)}(\xi)
        \end{align*}
        donde $\xi\in\left]a,a+h\right[$. Multiplicando por $\alpha_0$ y $\alpha_1$ respectivamente y sumando, obtenemos:
        \begin{multline*}
            \alpha_0 f(a) + \alpha_1 f(a + h) = (\alpha_0 + \alpha_1)f(a) + \alpha_1hf'(a) + \frac{\alpha_1h^2}{2}f''(a) + \frac{\alpha_1h^3}{6}f'''(a) +\\+ \frac{\alpha_1h^4}{4!}f^{(4)}(a) + \dfrac{\alpha_1h^5}{5!}f^{(5)}(\xi)
        \end{multline*}

        Por tanto, tenemos que:
        \begin{equation*}
            \left\{\begin{aligned}
                \alpha_0 + \alpha_1 &= 0 \\
                \alpha_1\cdot h &= 1
            \end{aligned}\right\}
            \Longrightarrow
            \left\{\begin{aligned}
                \alpha_0 &= -\frac{1}{h} \\
                \alpha_1 &= \frac{1}{h}
            \end{aligned}\right\}
        \end{equation*}

        Por lo tanto, la fórmula progresiva de derivación numérica de tipo interpolatorio clásico para aproximar $f'(a)$ a partir de $f(a)$ y $f(a + h)$ es:
        \begin{equation*}
            f'(a) = \dfrac{f(a + h) - f(a)}{h} + R(f)
        \end{equation*}

        Respecto al error, tenemos que:
        \begin{align*}
            R(f) &= -\dfrac{h}{2}f''(a) -\dfrac{h^2}{6}f'''(a) -\dfrac{h^3}{4!}f^{(4)}(a) - \dfrac{h^4}{5!}f^{(5)}(\xi) \\
            &= -\dfrac{h}{2}f''(\mu)\qquad \text{para algún }\mu\in\left]a,a+h\right[
        \end{align*}

        % // TODO: Ese paso por qué?


        
        \item Si notamos por $F(a, h)$ la aproximación de $f'(a)$ obtenida anteriormente, expresa el valor exacto de $f'(a)$ en función de $F(a, h)$ y los restantes términos en el desarrollo de Taylor.
        
        \item A partir de una combinación de los valores $F(a, h)$ y $F(a, h = 2)$ obtén una fórmula con mayor orden de precisión que $F(a, h)$.
        
        \item Aplica las dos fórmulas obtenidas para aproximar $f'(2)$ con $h = 0.1$ para la función $f(x) = \ln(x)$, $x \in [1, 3]$.
    \end{enumerate}
\end{ejercicio}

\begin{ejercicio}\label{ej:2.2.2}
    Para evaluar el funcional $L(f) = 2f'(a) - f''(a)$ se propone una fórmula del tipo
    \[
    2f'(a) - f''(a) \approx \alpha_0 f(a - h) + \alpha_1 f(a) + \alpha_2 f(a + h) :
    \]
    \begin{enumerate}
        \item Imponiendo exactitud en el espacio correspondiente halla la fórmula anterior para que sea de tipo interpolatorio clásico.
        
        En este caso, como hay $3$ nodos, es necesario imponer exactitud en $\bb{P}_2$, o equivalentemente, en $\cc{L}\{1, x, x^2\}$. Por tanto, buscamos $\alpha_0,\alpha_1,\alpha_2\in\bb{R}$ tales que:
        \begin{align*}
            2\cdot 0 - 0 &= \alpha_0\cdot 1 + \alpha_1\cdot 1 + \alpha_2\cdot 1 \\
            2\cdot 1 - 0 &= \alpha_0\cdot (a-h) + \alpha_1\cdot a + \alpha_2\cdot (a+h) \\
            2\cdot 2a - 2 &= \alpha_0\cdot (a-h)^2 + \alpha_1\cdot a^2 + \alpha_2\cdot (a+h)^2
        \end{align*}

        Por tanto, se trata de resolver el siguiente sistema de ecuaciones:
        \begin{equation*}
            \begin{pmatrix}
                1 & 1 & 1 \\
                a-h & a & a+h \\
                (a-h)^2 & a^2 & (a+h)^2
            \end{pmatrix}
            \begin{pmatrix}
                \alpha_0 \\
                \alpha_1 \\
                \alpha_2
            \end{pmatrix}=
            \begin{pmatrix}
                0 \\
                2 \\
                4a-2
            \end{pmatrix}
        \end{equation*}

        Planteamos la matriz ampliada, y aplicamos el método de Gauss:
        \begin{align*}
            &\left(\begin{array}{ccc|c}
                1 & 1 & 1 & 0 \\
                a-h & a & a+h & 2 \\
                (a-h)^2 & a^2 & (a+h)^2 & 4a-2
            \end{array}\right) \underrightarrow{F_3'=F_3-(a-h)F_2}
            \\&\left(\begin{array}{ccc|c}
                1 & 1 & 1 & 0 \\
                a-h & a & a+h & 2 \\
                0 & ah & 2h(a+h) & 2(a+h-1)
            \end{array}\right)\underrightarrow{F_2'=F_2-(a-h)F_1}
            \\&\left(\begin{array}{ccc|c}
                1 & 1 & 1 & 0 \\
                0 & h & 2h & 2 \\
                0 & ah & 2h(a+h) & 2(a+h-1)
            \end{array}\right)\underrightarrow{F_3'=F_3-aF_2}
            \\&\left(\begin{array}{ccc|c}
                1 & 1 & 1 & 0 \\
                0 & h & 2h & 2 \\
                0 & 0 & 2h^2 & 2(h-1)
            \end{array}\right)
            \Longrightarrow
            \begin{cases}
                \alpha_2 = \dfrac{h-1}{h^2} \\
                \alpha_1 = \dfrac{2-2\cdot \frac{h-1}{h}}{h} = \dfrac{2}{h^2}\\
                \alpha_0 = -\dfrac{2+h-1}{h^2} = -\dfrac{1+h}{h^2}
            \end{cases}
        \end{align*}

        Por tanto, al ser exacta en $\bb{P}_2$, la siguiente fórmula es de tipo interpolatorio clásico:
        \begin{equation*}
            2f'(a) - f''(a) \approx -\dfrac{1+h}{h^2}f(a-h) + \dfrac{2}{h^2}f(a) + \dfrac{h-1}{h^2}f(a+h)
        \end{equation*}
        
        \item Obtén una expresión del error de la fórmula en función de unas o varias derivadas de la función de órdenes superiores a dos.
        
        Definimos en primer lugar:
        \begin{align*}
            \Pi(x) &= (x-a+h)(x-a)(x-a-h)\\
            \Pi'(x) &= (x-a)(x-a-h) + (x-a+h)(x-a-h) + (x-a+h)(x-a)\\
            \Pi''(x) &= 2(x-a+h) + 2(x-a) + 2(x-a-h)
        \end{align*}
        
        Sabemos que el error del polinomio de interpolación en los $3$ nodos dados es:
        \begin{equation*}
            E(x) = f[a-h, a, a+h, x]\Pi(x)
        \end{equation*}

        Calculemos su derivada primera y segunda:
        \begin{align*}
            \hspace{-1cm}E'(x) &= f[a-h, a, a+h, x, x]\Pi(x) + f[a-h, a, a+h, x]\Pi'(x)\\
            \hspace{-1cm}E''(x) &= 2f[a-h, a, a+h, x, x, x]\Pi(x) + 2f[a-h, a, a+h, x, x]\Pi'(x)+f[a-h, a, a+h, x]\Pi''(x)
        \end{align*}

        Evaluamos cada una de las derivadas en $a$:
        \begin{align*}
            E'(a) &= -f[a-h, a, a+h, a]h^2\\
            E''(a) &= -2f[a-h, a, a+h, a, a]h^2
        \end{align*}

        Por ser de tipo interpolatorio clásico, el error de la fórmula es:
        \begin{align*}
            R(f) &= L(E) = 2E'(a) - E''(a) \\
            &= -2f[a-h, a, a+h, a]h^2+2f[a-h, a, a+h, a, a]h^2
            =\\&= 2h^2\left(f[a-h, a, a+h, a, a]-f[a-h, a, a+h, a]\right)
            =\\&= 2h^2\left(\dfrac{f^{(4)}(\xi_1)}{4!}-\dfrac{f^{(3)}(\xi_2)}{3!}\right)\qquad \text{para algún }\xi_1,\xi_2\in\left]a-h,a+h\right[
        \end{align*}

        % // TODO: Está bien?
        
        \item Aplica la fórmula obtenida para aproximar $2f'(2) - f''(2)$ con $h = 0.1$ para la función $f(x) = \ln(x)$, $x \in [1, 3]$.
        \begin{align*}
            2f'(2) - f''(2) &\approx -\dfrac{1+0.1}{0.1^2}f(1.9) + \dfrac{2}{0.1^2}f(2) + \dfrac{0.1-1}{0.1^2}f(2.1)\\
            &= -\dfrac{1.1}{0.01}\ln(1.9) + \dfrac{2}{0.01}\ln(2) + \dfrac{-0.9}{0.01}\ln(2.1) \approx 1.2511476
        \end{align*}
        
        \item Compara el error real obtenido en en el apartado anterior con respecto a una cota deducida de 2).
        
        El valor real es:
        \begin{align*}
            2f'(2) - f''(2) &= 2\left(\dfrac{1}{2}\right) - \left(-\dfrac{1}{2^2}\right) = \dfrac{5}{4} = 1.25
        \end{align*}

        Por tanto, el error real obtenido es:
        \begin{align*}
            \text{Error real} &\approx 1.25 - 1.2511476 \approx -0.147607\cdot 10^{-3}
        \end{align*}

        Una cota del error deducida de 2) es:
        \begin{align*}
            2\cdot 0.1^2\left(\dfrac{f^{(4)}(\xi_1)}{4!}-\dfrac{f^{(3)}(\xi_2)}{3!}\right)=0.02\left(\dfrac{f^{(4)}(\xi_1)}{4!}-\dfrac{f^{(3)}(\xi_2)}{3!}\right)
        \end{align*}
        % // TODO: Acotación
        
        \item Aplica la fórmula para obtener $2f'(0) - f''(0)$ suponiendo que tienes la siguiente tabla de valores de $f$:
        \begin{center}
            \begin{tabular}{c|c}
                $x_i$ & $f(x_i)$ \\
                \hline
                $-0.2$ & $9$ \\
                $0$ & $10$ \\
                $0.2$ & $9$ \\
                $0.4$ & $12$
            \end{tabular}
        \end{center}

        Tomando $h=0.2$, tenemos que:
        \begin{align*}
            2f'(0) - f''(0) &\approx -\dfrac{1+0.2}{0.2^2}f(-0.2) + \dfrac{2}{0.2^2}f(0) + \dfrac{0.2-1}{0.2^2}f(0.2)\\
            &= -\dfrac{1.2}{0.04}\cdot 9 + \dfrac{2}{0.04}\cdot 10 + \dfrac{-0.8}{0.04}\cdot 9 = 50
        \end{align*}
        % // TODO: Error? Si tomas solo los tres puntos sale perfecto, pero si tomas 4 el error es 10
    \end{enumerate}
\end{ejercicio}

\begin{ejercicio}\label{ej:2.2.3}
    Considera la fórmula de tipo interpolatorio clásico siguiente
    \[
    f'(a) \approx \alpha_0 f(a - h) + \alpha_1 f(a + 3h)
    \]
    \begin{enumerate}
        \item Da una expresión del error de dicha fórmula.

        Definimos en primer lugar:
        \begin{align*}
            \Pi(x) &= (x-a+h)(x-a-3h)\\
            \Pi'(x) &= (x-a-3h) + (x-a+h)
        \end{align*}

        Sabemos que el error del polinomio de interpolación en los $2$ nodos dados es:
        \begin{equation*}
            E(x) = f[a-h, a+3h, x]\Pi(x)
        \end{equation*}

        Calculemos su derivada primera:
        \begin{align*}
            E'(x) &= f[a-h, a+3h, x, x]\Pi(x) + f[a-h, a+3h, x]\Pi'(x)
        \end{align*}

        Por ser de tipo interpolatorio clásico, el error de la fórmula es:
        \begin{align*}
            R(f) &= L(E) = E'(a) = f[a-h, a+3h, a,a]\Pi(a) + f[a-h, a+3h, a]\Pi'(a)\\
            &= -3h^2f[a-h, a+3h, a,a]-2hf[a-h, a+3h, a]
            =\\&= -3h^2\left(\dfrac{f^{(3)}(\xi_1)}{3!}\right)-2h\left(\dfrac{f^{(2)}(\xi_2)}{2!}\right)
            =\\&= -h^2\left(\dfrac{f^{(3)}(\xi_1)}{2}\right)-h\left(f^{(2)}(\xi_2)\right)\qquad \text{para algún }\xi_1,\xi_2\in\left]a-h,a+3h\right[
        \end{align*}


        
        \item Úsala para aproximar la derivada $f'(3)$ siendo $f(x) = x^3$ con $h = 0.1$.
        
        Calculamos los polinomios fundamentales de Lagrange:
        \begin{align*}
            \ell_0(x) &= \dfrac{x-(a+2h)}{a-h-(a+3h)} = \dfrac{x-a-2h}{-4h}
            \Longrightarrow \ell_0'(x) = -\dfrac{1}{4h} \\
            \ell_1(x) &= \dfrac{x-(a-h)}{a+3h-(a-h)} = \dfrac{x-a+h}{4h}
            \Longrightarrow \ell_1'(x) = \dfrac{1}{4h}
        \end{align*}

        Por tanto, la fórmula de derivación numérica es:
        \begin{align*}
            f'(3) &\approx -\dfrac{1}{0.4}f(3-0.1) + \dfrac{1}{0.4}f(3+0.3) = 28.87
        \end{align*}

        % // TODO: Error?
    \end{enumerate}
\end{ejercicio}

\begin{ejercicio}\label{ej:2.2.4}
    Determina razonadamente si es posible diseñar una fórmula numérica de tipo interpolatorio en el espacio generado por $\cc{L}\{1, x, x^2, x^4\}$ para aproximar
    \[
    \int_{-2}^{2} f(x) \ d{x} + \int_{-2}^{2} |x|f(x) \ d{x}
    \]
    usando para ello los datos
    \[
    \int_{-1}^{1} f(x) \ d{x}, \quad \int_{-1}^{1} |x|f(x) \ d{x}, \quad f(0) \quad \text{y} \quad f'(0).
    \]
    En particular determina el peso de $f'(0)$.
    % // TODO: x^3?
    % // Qué significa el enunciado
    % No podemos usar la caracterizacion pq es interpolacion clasica
    % No podemos interpolar pq no sabemos
\end{ejercicio}

\begin{ejercicio}\label{ej:2.2.5}
    Dada la fórmula de derivación numérica de tipo interpolatorio:
    \[
    f'(0) = \alpha_0 f(0) + \alpha_1 f(x_1) + \alpha_2 f(x_2) + R(f), \quad x_1 \neq 0, x_2 \neq 0, x_1 \neq x_2.
    \]
    \begin{enumerate}
        \item Sin realizar ningún cálculo, ¿puedes indicar el máximo grado de exactitud que puede tener la fórmula? Justifica la respuesta.
        
        Como tiene $3$ nodos y se trata de la primera derivada, sabemos que no puede ser exacta en $\bb{P}_4$, por lo que el máximo grado de exactitud que puede tener la fórmula es $3$.
        
        \item Determina los valores de $\alpha_0$, $\alpha_1$, $\alpha_2$, $x_1$ y $x_2$ para que la fórmula tenga el mayor grado de exactitud posible. ¿Cuál es ese grado de exactitud?
        
        Imponemos exactitud en $\bb{P}_3$, o equivalentemente, en $\cc{L}\{1, x, x^2,x^3\}$. Por tanto, buscamos $\alpha_0,\alpha_1,\alpha_2\in\bb{R}$ tales que:
        \begin{align*}
            0 &= \alpha_0 + \alpha_1 + \alpha_2 \\
            1 &= \alpha_1x_1 + \alpha_2x_2 \\
            0 &= \alpha_1x_1^2 + \alpha_2x_2^2 \\
            0 &= \alpha_1x_1^3 + \alpha_2x_2^3
        \end{align*}

        Veamos que el sistema para obtener $\alpha_1,\alpha_2$ es incompatible:
        \begin{align*}
            \begin{vmatrix}
                x_1 & x_2 & 1 \\
                x_1^2 & x_2^2 & 0 \\
                x_1^3 & x_2^3 & 0
            \end{vmatrix} &= x_1^2x_2^3 - x_1^3x_2^2 = x_1^2x_2^2(x_2-x_1) \neq 0
        \end{align*}

        Por tanto, por el Teorema de Rouché-Frobenius, el sistema es incompatible y no tiene solución. Por tanto, no es posible imponer exactitud en $\bb{P}_3$. Veamos si es posible imponer exactitud en $\bb{P}_2$:
        \begin{align*}
            0 &= \alpha_0 + \alpha_1 + \alpha_2 \\
            1 &= \alpha_1x_1 + \alpha_2x_2 \\
            0 &= \alpha_1x_1^2 + \alpha_2x_2^2
        \end{align*}

        Como $\alpha_1=\dfrac{-\alpha_2x_2^2}{x_1^2}$, sustituyendo en la segunda ecuación, obtenemos:
        \begin{align*}
            1 &= -\dfrac{\alpha_2x_2^2}{x_1} + \alpha_2x_2
            \Longrightarrow
            1=\alpha_2\left(x_2-\dfrac{x_2^2}{x_1}\right) = \alpha_2\left(\dfrac{x_2(x_1-x_2)}{x_1}\right)
            \Longrightarrow \alpha_2 = \dfrac{x_1}{x_2(x_1-x_2)}
        \end{align*}

        Por tanto:
        \begin{align*}
            \alpha_1 &= -\dfrac{x_2}{x_1(x_1-x_2)}
            \Longrightarrow
            \alpha_0 = \dfrac{x_2}{x_1(x_1-x_2)}-\dfrac{x_1}{x_2(x_1-x_2)}
            = \dfrac{x_2^2-x_1^2}{x_1x_2(x_1-x_2)} = -\dfrac{x_1+x_2}{x_1x_2}
        \end{align*}

        Por tanto, la fórmula de derivación numérica de tipo interpolatorio es:
        \begin{equation*}
            f'(0) = -\dfrac{x_1+x_2}{x_1x_2}f(0) -\dfrac{x_2}{x_1(x_1-x_2)}f(x_1) + \dfrac{x_1}{x_2(x_1-x_2)}f(x_2) + R(f)
        \end{equation*}

        Además, el grado máximo de exactitud es $2$.

        \item Determina la expresión del error indicando las condiciones sobre derivabilidad de la función $f$. ¿Hay alguna otra conclusión que obtengas respecto a los nodos?
        
        Definimos en primer lugar:
        \begin{align*}
            \Pi(x) &= x(x-x_1)(x-x_2)\\
            \Pi'(x) &= (x-x_1)(x-x_2) + x(x-x_2) + x(x-x_1)
        \end{align*}

        Sabemos que el error del polinomio de interpolación en los $3$ nodos dados es:
        \begin{align*}
            E(x) &= f[0, x_1, x_2, x]\Pi(x)\\
            E'(x) &= f[0, x_1, x_2, x, x]\Pi(x) + f[0, x_1, x_2, x]\Pi'(x)
        \end{align*}

        Por ser de tipo interpolatorio clásico, el error de la fórmula es:
        \begin{align*}
            R(f) &= L(E) = E'(0) = f[0, x_1, x_2, 0, 0]\Pi(0) + f[0, x_1, x_2, 0]\Pi'(0)\\
            &= f[0, x_1, x_2, 0, 0]\cdot 0 + x_1x_2f[0, x_1, x_2, 0]
            =\\&= x_1x_2\cdot \dfrac{f^{(3)}(\xi)}{3!} \qquad \text{para algún }\xi\in\left]\min\{0,x_1,x_2\},\max\{0,x_1,x_2\}\right[
        \end{align*}
        donde, además, hemos supuesto que $f\in{C}^4\left(\left]\min\{0,x_1,x_2\},\max\{0,x_1,x_2\}\right[\right)$.

        % // TODO: Otra conclusion
        
        \item Aplica el resultado para la función $x e^{x^2 + 1}$.
        
        % // TODO: Hacer
        
    \end{enumerate}
\end{ejercicio}

\begin{ejercicio}\label{ej:2.2.6}
    Dada la fórmula de derivación numérica de tipo interpolatorio:
    \[
    f'(0) = \alpha_0 f(-1) + \alpha_1 f(1) + \alpha_2 f(2) + \alpha_3 f(a) + R(f), \quad a \neq -1, 1, 2.
    \]
    \begin{enumerate}
        \item Sin realizar ningún cálculo, ¿puedes indicar el máximo grado de exactitud que puede tener la fórmula? Justifica la respuesta.
        
        Como tiene $4$ nodos y se trata de la primera derivada, sabemos que no puede ser exacta en $\bb{P}_5$, por lo que el máximo grado de exactitud que puede tener la fórmula es $4$.
        
        \item Determina los valores de $\alpha_0$, $\alpha_1$, $\alpha_2$, $\alpha_3$ y $a$ para que la fórmula tenga el mayor grado de exactitud posible. ¿Cuál es ese grado de exactitud?
        
        Imponemos exactitud en $\bb{P}_4$, o equivalentemente, en $\cc{L}\{1, x, x^2,x^3,x_4\}$. Por tanto, buscamos $\alpha_0,\alpha_1,\alpha_2,\alpha_3, a\in\bb{R}$ tales que:
        \begin{align*}
            0 &= \alpha_0 + \alpha_1 + \alpha_2 + \alpha_3 \\
            1 &= -\alpha_0 + \alpha_1 + 2\alpha_2 + a\alpha_3 \\
            0 &= \alpha_0 + \alpha_1 + 4\alpha_2 + a^2\alpha_3 \\
            0 &= -\alpha_0 + \alpha_1 + 8\alpha_2 + a^3\alpha_3 \\
            0 &= \alpha_0 + \alpha_1 + 16\alpha_2 + a^4\alpha_3
        \end{align*}

        Planteamos el sistema de ecuaciones y aplicamos el método de Gauss:
        \begin{align*}
            &\left(\begin{array}{cccc|c}
                1 & 1 & 1 & 1 & 0 \\
                -1 & 1 & 2 & a & 1 \\
                1 & 1 & 4 & a^2 & 0 \\
                -1 & 1 & 8 & a^3 & 0 \\
                1 & 1 & 16 & a^4 & 0
            \end{array}\right) \xrightarrow[F_4'=F_4+F_1]{F_2'=F_2+F_1}
            \left(\begin{array}{cccc|c}
                1 & 1 & 1 & 1 & 0 \\
                0 & 2 & 3 & a+1 & 1 \\
                1 & 1 & 4 & a^2 & 0 \\
                0 & 2 & 9 & a^3+1 & 0 \\
                1 & 1 & 16 & a^4 & 0
            \end{array}\right)\xrightarrow[F_5'=F_5-F_1]{F_3'=F_3-F_1}
            \\&\left(\begin{array}{cccc|c}
                1 & 1 & 1 & 1 & 0 \\
                0 & 2 & 3 & a+1 & 1 \\
                0 & 0 & 3 & a^2-1 & 0 \\
                0 & 2 & 9 & a^3+1 & 0 \\
                0 & 0 & 15 & a^4-1 & 0
            \end{array}\right)\xrightarrow{F_4'=F_4-F_2}
            \left(\begin{array}{cccc|c}
                1 & 1 & 1 & 1 & 0 \\
                0 & 2 & 3 & a+1 & 1 \\
                0 & 0 & 3 & a^2-1 & 0 \\
                0 & 0 & 6 & a(a^2-1) & -1 \\
                0 & 0 & 15 & a^4-1 & 0
            \end{array}\right)\xrightarrow[F_5'=F_5-5F_3]{F_4'=F_4-2F_3}
            \\&\left(\begin{array}{cccc|c}
                1 & 1 & 1 & 1 & 0 \\
                0 & 2 & 3 & a+1 & 1 \\
                0 & 0 & 3 & a^2-1 & 0 \\
                0 & 0 & 0 & (a^2-1)(a-2) & -1 \\
                0 & 0 & 0 & a^4-5a^2+4 & 0
            \end{array}\right)
        \end{align*}

        Para que el sistema tenga solución, es necesario que:
        \begin{multline*}
            a^4-5a^2+4 = 0 \iff a^2=\dfrac{5\pm\sqrt{25-16}}{2} = \dfrac{5\pm 3}{2} \iff a^2 = 4 \ \vee\  a^2 = 1 \iff\\ \iff a\in \left\{-2,2,-1,1\right\}
        \end{multline*}

        Distinguimos valores:
        \begin{itemize}
            \item Si $a\notin \left\{1,-1,2, -2\right\}$, por la última ecuación $\alpha_3=0$, pero entonces la penúltima ecuación queda $0=-1$, por lo que no hay solución.
            \item Si $a\in \left\{1,-1,2\right\}$, la última penúltima ecuación queda $0=-1$, por lo que no hay solución.
            \item Si $a=-2$, el sistema queda:
            \begin{align*}
                \left(\begin{array}{cccc|c}
                    1 & 1 & 1 & 1 & 0 \\
                    0 & 2 & 3 & -1 & 1 \\
                    0 & 0 & 3 & 3 & 0 \\
                    0 & 0 & 0 & -12 & -1 \\
                    0 & 0 & 0 & 0 & 0
                \end{array}\right)\Longrightarrow
                \begin{cases}
                    \alpha_0 = \nicefrac{-2}{3} \\
                    \alpha_1 = \nicefrac{2}{3} \\
                    \alpha_2 = \nicefrac{-1}{12} \\
                    \alpha_3 = \nicefrac{1}{12}
                \end{cases}
            \end{align*}
        \end{itemize}

        Por tanto, para que la fórmula tenga el mayor grado de exactitud posible, $a=-2$ y el grado máximo de exactitud es $4$. La fórmula es:
        \begin{equation*}
            f'(0) = -\dfrac{2}{3}f(-1) + \dfrac{2}{3}f(1) - \dfrac{1}{12}f(2) + \dfrac{1}{12}f(-2) + R(f)
        \end{equation*}
        
        \item Determina la expresión del error indicando las condiciones sobre derivabilidad de la función $f$.
        
        Definimos en primer lugar:
        \begin{align*}
            \Pi(x) &= (x+1)(x-1)(x-2)(x+2)\\
            \Pi'(x) &= (x-1)(x-2)(x+2) + (x+1)(x-2)(x+2) + (x+1)(x-1)(x+2) + (x+1)(x-1)(x-2)
        \end{align*}

        Sabemos que el error del polinomio de interpolación en los $4$ nodos dados es:
        \begin{align*}
            \hspace{-1cm}E(x) &= f[-1, 1, 2, -2, x]\Pi(x)\\
            \hspace{-1cm}E'(x) &= f[-1, 1, 2, -2, x, x]\Pi(x) + f[-1, 1, 2, -2, x]\Pi'(x)
        \end{align*}

        Por ser de tipo interpolatorio clásico, el error de la fórmula es:
        \begin{align*}
            R(f) &= L(E) = E'(0) = f[-1, 1, 2, -2, 0, 0]\Pi(0) + f[-1, 1, 2, -2, 0]\Pi'(0)\\
            &= 4f[-1, 1, 2, -2, 0, 0] + f[-1, 1, 2, -2, 0]\cdot (4-4-2+2)
            =\\&= 4f[-1, 1, 2, -2, 0, 0] = \dfrac{f^{(5)}(\xi)}{30} \qquad \text{para algún }\xi\in\left]-2,2\right[
        \end{align*}
        
        \item Aplica el resultado para la función $x e^{x^2 + 1}$.
        
        % // TODO: Hacer
        
    \end{enumerate}
\end{ejercicio}
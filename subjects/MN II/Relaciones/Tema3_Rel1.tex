\subsection{Relación 1}
\setcounter{ejercicio}{0}

En la mayoría de los ejercicios se hace referencia al problema de valores iniciales (PVI) siguiente:
\begin{equation}\label{eq:pvi}
    x' = f(t, x),\quad f : D = [a = t_0, b] \times \bb{R} \to \bb{R},\quad x(t_0) = \mu,\quad (t_0, \mu) \in D
\end{equation}
siendo $x = x(t)$ una función desconocida de $t$.

\begin{ejercicio}\label{ej:3.1.1}
    Sea $f : D \to \bb{R}$ continua y lipschitziana en su segunda variable con constante de Lipschitz $M$, y $f\in C^1(D)$. Sea $h = \frac{b-a}{N}$, con $N \in \bb{N}$. Se considera el método de Euler para resolver el PVI~\eqref{eq:pvi} con tamaño de paso $h$. Demuestre que:
    \begin{itemize}
        \item Si $M = 0$, entonces $|e_n| \leq \frac{(b-a)M^*}{2} h$ para todo $n = 0, \ldots, N$.
        \item Si $M > 0$, entonces $|e_n| \leq \dfrac{e^{(b-a)M} - 1}{M} \cdot \frac{M^*}{2} h$ para todo $n = 0, \ldots, N$,
    \end{itemize}
    donde $e_n$ es el error de truncatura global del método en el punto $t_n$, y $M^*$ es tal que $|x''(t)| \leq M^*$ para todo $t \in [a, b]$, siendo $x(t)$ la única solución de~\eqref{eq:pvi}.    
\end{ejercicio}


\begin{ejercicio}\label{ej:3.1.2}
    Demuestre que el PVI $x' = \nicefrac{-1}{2} x$, $x(0) = 2$ tiene una única solución en $[0, 1]$ y halle una cota del error de truncatura global en cada nodo del método de Euler de tamaño de paso $h$ para aproximar dicha solución.
\end{ejercicio}

\begin{ejercicio}\label{ej:3.1.3}
    Demuestre que si $f : D \to \bb{R}$ es continua y lipschitziana en su segunda variable, entonces el método del punto medio para resolver el PVI~\eqref{eq:pvi} es estable, consistente con~\eqref{eq:pvi} y converge a la solución de~\eqref{eq:pvi}.\\

    Buscamos obtener el método del punto medio para resolver el PVI~\eqref{eq:pvi}. Tenemos que:
    \begin{equation*}
        x(t_{n+1})-x(t_n) = \int_{t_n}^{t_{n+1}} x'(s)\, ds = \int_{t_n}^{t_{n+1}} f(s, x(s))\, ds
    \end{equation*}

    Para aproximar la integral anterior, utilizamos el método del punto medio visto en integración numérica:
    \begin{equation*}
        \int_{t_n}^{t_{n+1}} f(s, x(s))\, ds \approx h f\left(t_n + \frac{h}{2}, x\left(t_n + \frac{h}{2}\right)\right)
    \end{equation*}

    Para obtener el valor de $x(t_n + \nicefrac{h}{2})$, lo aproximamos como sigue:
    \begin{equation*}
        x\left(t_n + \frac{h}{2}\right) \approx x(t_n) + \frac{h}{2} f(t_n, x(t_n))
    \end{equation*}

    Por tanto, el método del punto medio para resolver el PVI~\eqref{eq:pvi} queda definido por la siguiente ecuación en diferencias:
    \begin{equation*}
        x_{n+1} = x_n + h f\left(t_n + \frac{h}{2}, x_n + \frac{h}{2} f(t_n, x_n)\right)
    \end{equation*}

    Su polinomio característico es:
    \begin{equation*}
        p(\lm) = \lm -1
    \end{equation*}

    Para ver que es estable, es necesario ver que el módulo de las raíces del polinomio característico es menor o igual que 1, y en el caso de que sea igual a 1, que la raíz sea simple. En este caso, la única raíz es $\lm = 1$, que es simple, por lo que el método del punto medio es estable.

    Para ver que es consistente con el PVI~\eqref{eq:pvi} empleando la caracterización de la consistencia, tenemos que:
    \begin{itemize}
        \item $p(1)= 0$, que se tiene.
        \item $\Phi(x(t_n), t_n, 0)=p'(1)f(t_n, x(t_n))$:
        \begin{equation*}
            \Phi(x(t_n), t_n, 0) = f\left(t_n, x(t_n)\right) = p'(1) f(t_n, x(t_n))
        \end{equation*}
        Por tanto, se tiene también.
    \end{itemize}
    
    Por tanto, el método del punto medio es consistente con el PVI~\eqref{eq:pvi}. Como es estable y consistente, converge a la solución del PVI~\eqref{eq:pvi}.
\end{ejercicio}

\begin{ejercicio}\label{ej:3.1.4}
    Razone la veracidad o falsedad de la afirmación siguiente: El método de Euler modificado (Heun) proporciona la solución exacta de la EDO dada por $x' = -2\lambda t$.

    La solución exacta de la EDO $x' = -2\lambda t$ es:
    \begin{equation*}
        x(t) = -\lambda t^2 + C
    \end{equation*}

    El método de Euler modificado (Heun) para resolver el PVI $x' = f(t, x)$ consiste en aplicar la fórmula de integración numérica del trapecio:
    \begin{equation*}
        x_{n+1} - x_n = \int_{t_n}^{t_{n+1}} f(s, x(s))\, ds \approx \frac{h}{2} \left( f(t_n, x_n) + f(t_{n+1}, x_{n+1}) \right)
    \end{equation*}

    Para obtener $x_{n+1}$, se aproxima por el método de Euler:
    \begin{equation*}
        x_{n+1} = x_n + h f(t_n, x_n)
    \end{equation*}

    Por tanto, el método de Euler modificado (Heun) para resolver el PVI $x' = -2\lambda t$ queda definido por la siguiente ecuación en diferencias:
    \begin{equation*}
        x_{n+1} = x_n + \frac{h}{2} \left( f(t_n, x_n) + f(t_{n+1}, x_n + h f(t_n, x_n)) \right)
    \end{equation*}

    Aplicada al PVI $x' = -2\lambda t$, la ecuación en diferencias queda:
    \begin{equation*}
        x_{n+1} = x_n + \frac{h}{2} \left( -2\lambda t_n + (-2\lambda (t_n + h)) \right)
        = x_n - \lambda h (t_n + t_ n + h)
        = x_n - \lambda h (2t_n + h)
    \end{equation*}

    Veamos si efectivamente el método de Euler modificado (Heun) proporciona la solución exacta de la EDO $x' = -2\lambda t$. Para ello, hemos de suponer que el dato inicial $x_0$ proporcionado cumple que $x_0 = x(t_0)$. Supuesto esto:
    \begin{align*}
        x_{n+1} &= x(t_{n+1}) = x(t_n + h) = -\lambda (t_n + h)^2 + C \\
        &= -\lambda (t_n^2 + 2t_n h + h^2) + C \\
        &= -\lambda t_n^2 - 2\lambda t_n h - \lambda h^2 + C \\
        &= x(t_n) - \lambda h (2t_n + h) \\
        &= x_n - \lambda h (2t_n + h)
    \end{align*}

    Por tanto, por inducción, demostramos que el método de Euler modificado (Heun) proporciona la solución exacta de la EDO $x' = -2\lambda t$ si el dato inicial $x_0$ es tal que $x_0 = x(t_0)$.
\end{ejercicio}

\begin{ejercicio}\label{ej:3.1.5}
    Demuestre que si $f : D \to \bb{R}$ es continua y lipschitziana en su segunda variable, entonces el método de Runge-Kutta clásico para resolver el PVI~\eqref{eq:pvi} es estable, consistente con~\eqref{eq:pvi} y converge a la solución de~\eqref{eq:pvi}.\\

    El método de Runge-Kutta clásico para resolver el PVI~\eqref{eq:pvi} se define por la siguiente ecuación en diferencias:
    \begin{align*}
        x_{n+1} &= x_n + \frac{h}{6} \left( K_1 + 2K_2 + 2K_3 + K_4 \right)\\
        K_1 &= f(t_n, x_n) \\
        K_2 &= f\left(t_n + \frac{h}{2}, x_n + \frac{h}{2} K_1\right) \\
        K_3 &= f\left(t_n + \frac{h}{2}, x_n + \frac{h}{2} K_2\right) \\
        K_4 &= f(t_n + h, x_n + h K_3)
    \end{align*}

    Su polinomio característico es:
    \begin{equation*}
        p(\lm) = \lm - 1
    \end{equation*}

    Para ver que es estable, es necesario ver que el módulo de las raíces del polinomio característico es menor o igual que 1, y en el caso de que sea igual a 1, que la raíz sea simple. En este caso, la única raíz es $\lm = 1$, que es simple, por lo que el método de Runge-Kutta clásico es estable.

    Para ver que es consistente con el PVI~\eqref{eq:pvi} empleando la caracterización de la consistencia, tenemos que:
    \begin{itemize}
        \item $p(1)= 0$, que se tiene.
        \item $\Phi(x(t_n), t_n, 0)=p'(1)f(t_n, x(t_n))$:
        
        En primer lugar, hemos de ver que si $h=0$ entonces:
        \begin{equation*}
            K_1 = K_2 = K_3 = K_4 = f(t_n, x_n)
        \end{equation*}

        Por tanto:
        \begin{align*}
            \Phi(x(t_n), t_n, 0) &= \frac{1}{6} \left( K_1 + 2K_2 + 2K_3 + K_4 \right) = \dfrac{1}{6}\cdot 6 f(t_n, x_n) = p'(1) f(t_n, x_n)
        \end{align*}
        Por tanto, se tiene también.
    \end{itemize}
    
    Por tanto, el método de Runge-Kutta clásico es consistente con el PVI~\eqref{eq:pvi}. Como es estable y consistente, converge a la solución del PVI~\eqref{eq:pvi}.
\end{ejercicio}

\begin{ejercicio}\label{ej:3.1.6}
    Razone la veracidad o falsedad de las siguientes afirmaciones:
    \begin{enumerate}
        \item Al aplicar el método de un paso del punto medio al problema $x'(t) = -x(t) + 1$, $x(0) = 2$ se obtiene la solución numérica $\{t_n, x_n\}_{n=0}^N$ donde $x_n = A^n + 1$ con $A = 1 - h + \frac{h^2}{2}$.
        
        El método del punto medio para resolver el PVI $x' = f(t, x)$ se define por la siguiente ecuación en diferencias:
        \begin{equation*}
            x_{n+1} = x_n + h f\left(t_n + \frac{h}{2}, x_n + \frac{h}{2} f(t_n, x_n)\right)
        \end{equation*}
        Aplicada al PVI $x' = -x + 1$, la ecuación en diferencias queda:
        \begin{align*}
            x_{n+1} &= x_n + h\left(-x_n - \frac{h}{2}(-x_n + 1)+1\right) \\
            &= x_n + h\left(-x_n + \frac{h}{2} x_n - \frac{h}{2} + 1\right)
        \end{align*}

        Demostramos ahora por inducción.
        \begin{itemize}
            \item Para $n=0$, tenemos que:
            \begin{equation*}
                x_0 = 2 = A^0 + 1
            \end{equation*}

            \item Supongamos que se cumple para $n$, y demostremos que se cumple para $n+1$:
            \begin{align*}
                x_{n+1} &= x_n + h\left(-x_n + \frac{h}{2} x_n - \frac{h}{2} + 1\right) \\
                &= A^n + 1 + h\left(-A^n - 1 + \frac{h}{2} (A^n+1) - \frac{h}{2} + 1\right) \\
                &= A^n + 1 + h\left(-A^n + \frac{h}{2} A^n\right) \\
                &= 1 + A^n + hA^n\left(-1 + \frac{h}{2}\right) \\
                &= 1 + A^n\left(1 - h + \frac{h^2}{2}\right) \\
                &= 1 + A^{n+1}
            \end{align*}

            Por tanto, se cumple para $n+1$.
        \end{itemize}

        Por tanto, por inducción, se tiene demostrado para todo $n \in \bb{N}$.
        \item El orden de un método explícito de un paso cuya ecuación en diferencias es:
        \begin{equation*}
            x_{n+1} = x_n + h \Phi(x_n; t_n, h),\qquad n = 0, \ldots, N-1
        \end{equation*}
        con $\Phi(x; t, h) = f(t, x) + \frac{h}{2} x''(t)$ es al menos 2.\\

        Tenemos que calcular:
        \begin{align*}
            R_{n+1} &= x(t_{n+1}) - x(t_n) -h \Phi(x(t_n); t_n, h)
            =\\&= x(t_{n+1}) - x(t_n) - h f(t_n, x(t_n)) - \frac{h^2}{2} x''(t_n) 
            =\\&= x(t_n + h) - x(t_n) - h x'(t_n) - \frac{h^2}{2} x''(t_n)
        \end{align*}

        El desarrollo de Taylor de $x(t_n + h)$ alrededor de $t_n$ es:
        \begin{align*}
            x(t_n + h) &= x(t_n) + h x'(t_n) + \frac{h^2}{2} x''(t_n) + O(h^3)
        \end{align*}

        Por tanto, tenemos que:
        \begin{align*}
            R_{n+1} &= O(h^3)
        \end{align*}

        Por tanto, el orden del método es al menos 2.
    \end{enumerate}
\end{ejercicio}

\begin{ejercicio}\label{ej:3.1.7}
    Demuestre que el PVI $x' = t - x$, $x(0) = 1$ tiene una única solución en $[0, 1]$. ¿Se puede aproximar dicha solución mediante:
    \begin{itemize}
        \item el método de Taylor de orden 2?
        \item el método de Heun?
        \item el método del punto medio?
    \end{itemize}
    ¿Por qué? Escriba la ecuación en diferencias de cada uno de estos métodos para resolver el PVI considerado. ¿Ocurre algo reseñable?\\

    Sea $x'=f(t, x)$, con:
    \Func{f}{\bb{R}^2}{\bb{R}}{(t, x)}{t - x}

    Veamos que $f$ es continua y lipschitziana en su segunda variable. La continuidad de $f$ es inmediata, y para ver que es lipschitziana, sean $x,y\in \bb{R}$:
    \begin{align*}
        |f(t, x) - f(t, y)| &= |(t - x) - (t - y)| = |y - x|\leq 1\cdot |x-y|\qquad \forall t\in \bb{R}
    \end{align*}

    Por tanto, $f$ es lipschitziana en su segunda variable con constante de Lipschitz $M = 1$. Por tanto, el PVI tiene una única solución en $[0,1]\subset \bb{R}$. Se puede aproximar mediante los tres métodos, veámoslo:
    \begin{enumerate}
        \item El método de Taylor de orden 2 para resolver el PVI $x' = f(t, x)$ es:
        \begin{align*}
            x_{n+1} &\approx x(t_n + h) \approx x(t_n) + hx'(t_n) + \frac{h^2}{2} x''(t_n)
        \end{align*}

        Para aproximar $x'(t_n)$ y $x''(t_n)$, utilizamos que:
        \begin{align*}
            x'(t_n) &= f(t_n, x(t_n)) = t_n - x(t_n)\approx t_n - x_n \\
            x''(t_n) &= \frac{\partial f}{\partial t}(t_n, x(t_n)) + \frac{\partial f}{\partial x}(t_n, x(t_n)) x'(t_n) = 1 - 1\cdot (t_n - x(t_n)) \approx 1 - t_n + x_n
        \end{align*}

        Por tanto, la ecuación en diferencias del método de Taylor de orden 2 para resolver el PVI $x' = t - x$ es:
        \begin{align*}
            x_{n+1} &= x_n + h(t_n - x_n) + \frac{h^2}{2} (1 - t_n + x_n)
        \end{align*}

        % // TODO: Cont

    \end{enumerate}
\end{ejercicio}

\begin{ejercicio}\label{ej:3.1.8}
    Demuestre que el PVI $x' = t^2 x$, $x(0) = 1$ tiene una única solución $x(t)$ en $[0, 1]$. Demuestre que el método de Taylor de orden 2 y el método de Runge-Kutta clásico convergen a $x(t)$ y halle las aproximaciones de cada uno de estos dos métodos para el tamaño de paso $h = 0.2$.
\end{ejercicio}

\begin{ejercicio}\label{ej:3.1.9}
    Utilizando el método de Runge-Kutta clásico para resolver el PVI $x' = f(t)$, $x(0) = \mu$, deduzca una conocida fórmula de integración numérica e indique de qué fórmula se trata.
\end{ejercicio}

\begin{ejercicio}\label{ej:3.1.10}
    Para el método de Runge-Kutta de 2 evaluaciones con arreglo de Butcher:
    \begin{equation*}
        \begin{array}{c|cc}
            0 & 0 & 0 \\ 
            \alpha & \alpha & 0 \\ \hline
             & \frac{1-\alpha}{2} & \frac{1+\alpha}{2}
        \end{array}
    \end{equation*}
    \begin{enumerate}
        \item Determine el orden del método según los valores del escalar $\alpha$.
        \item Para $\alpha$ adecuado para que el método tenga orden máximo, ¿cuál es el término principal del error local de truncatura?
        \item ¿Es estable (cero-estable) el método para todo $\alpha$ si $f(t, x)$ es continua y lipschitziana respecto de $x$ sobre $D$?
    \end{enumerate}
\end{ejercicio}

\begin{ejercicio}
    Considere el MML definido por:
    \begin{equation*}
        x_{n+2} + a_1 x_{n+1} + a_0 x_n = h (b_0 f_n + b_1 f_{n+1})
    \end{equation*}
    \begin{enumerate}
        \item Exprese los valores de $a_0$, $b_0$, $b_1$ respecto de $a_1$ para que el método sea de orden, al menos, 2.
        \item Para la familia de métodos obtenida en el apartado anterior, ¿qué valor(es) de $a_1$ hacen el MML estable?
        \item ¿Qué métodos particulares se obtienen si $a_1 = 0$ y $a_1 = -1$?
        \item ¿Es estable y de orden 3 el MML para algún valor de $a_1$?
    \end{enumerate}
\end{ejercicio}

\begin{ejercicio}\label{ej:3.1.12}
    Determine los parámetros $\alpha$, $\beta$ para los que el método lineal de ecuación en diferencias:
    \begin{equation*}
        x_{n+2} - (1 + \alpha) x_{n+1} + \alpha x_n = h ((1 + \beta) f_{n+2} - (\alpha + \beta + \alpha \beta) f_{n+1} + \alpha \beta f_n)
    \end{equation*}
    con $n = 0, \ldots, N-2$, tiene el mayor orden posible. ¿Cuál es dicho orden? ¿Es convergente el método obtenido? ¿Por qué?
\end{ejercicio}

\begin{ejercicio}\label{ej:3.1.13}
    Construya una familia 1-paramétrica de MML implícitos de dos pasos con el mayor orden posible. ¿Cuál es dicho orden? Si $x(t)$ es suficientemente diferenciable, ¿cuál es la parte principal de error de truncatura local? ¿Qué valores del parámetro aseguran la convergencia?
\end{ejercicio}

\begin{ejercicio}\label{ej:3.1.14}
    Obtenga la ecuación en diferencias del método de Adams-Bashforth de tres pasos y la del método de Adams-Moulton de dos pasos. Estudie la convergencia y el orden de dichos métodos.
\end{ejercicio}

\begin{ejercicio}\label{ej:3.1.15}
    Usando integración numérica sobre el intervalo $[t_{n+1}, t_{n+3}]$, deduzca dos métodos lineales de tres pasos explícitos diferentes para resolver el PVI de ecuación $x' = f(t, x)$ y condición inicial $x(t_0) = \mu$. ¿Es alguno de ellos un método óptimo? Justifique la respuesta.
\end{ejercicio}

\begin{ejercicio}\label{ej:3.1.16}
    Dado el MML:
    \begin{equation*}
        x_{n+3} + \alpha (x_{n+2} - x_{n+1}) - x_n = h \frac{2(3 + \alpha)}{2} (f_{n+1} + f_{n+2})
    \end{equation*}
    razone si es cierto que:
    \begin{enumerate}
        \item Para $\alpha = 9$, el orden es 4.
        \item Si $-3 < \alpha < 1$, el método es cero-estable.
        \item Si el MML dado es convergente, su orden es exactamente 2.
    \end{enumerate}
\end{ejercicio}

\begin{ejercicio}\label{ej:3.1.17}
    Para que un MML sea estable es necesario que las raíces del primer polinomio característico $p(\lm)$ sean de módulo no mayor que 1, y todas las de módulo 1 sean simples. Cualquier MML ya tiene la raíz $\lm = 1$ entre las de su primer polinomio característico. Podemos afinar algo más distinguiendo los MML que no tienen ninguna otra raíz de módulo 1, de aquellos en los que hay más de una raíz de módulo 1. Los primeros se denominan fuertemente estables y los segundos débilmente estables.
    \begin{enumerate}
        \item Compruebe que el método Adams-Bashforth de 4 pasos es fuertemente estable:
        \begin{equation*}
            x_{n+1} = x_n + \frac{h}{24} (-9 f_{n-3} + 37 f_{n-2} - 59 f_{n-1} + 55 f_n)
        \end{equation*}
        \item Compruebe que la fórmula abierta de 4 pasos es débilmente estable:
        \begin{equation*}
            x_{n+1} = x_{n-3} + \frac{4h}{3} (2 f_{n-2} - f_{n-1} + 2 f_n)
        \end{equation*}
        \item (Práctica) Considere el PVI $x' = -6x + 6$, $0 \leq t \leq 1$, $x(0) = 2$, que tiene como solución exacta $x(t) = 1 + e^{-6t}$. Con $h = 0.1$ y valores iniciales exactos $x_0$, $x_1$, $x_2$, $x_3$, aproxime $x(1)$ con las dos fórmulas anteriores, incluyendo los errores acumulados en cada paso para ambos métodos. Comente los resultados.
    \end{enumerate}
\end{ejercicio}
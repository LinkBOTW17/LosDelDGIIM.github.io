\subsection{Relación 1}
\setcounter{ejercicio}{0}

En la mayoría de los ejercicios se hace referencia al problema de valores iniciales (PVI) siguiente:
\begin{equation}\label{eq:pvi}
    x' = f(t, x),\quad f : D = [a = t_0, b] \times \bb{R} \to \bb{R},\quad x(t_0) = \mu,\quad (t_0, \mu) \in D
\end{equation}
siendo $x = x(t)$ una función desconocida de $t$.

\begin{ejercicio}\label{ej:3.1.1}
    Sea $f : D \to \bb{R}$ continua y lipschitziana en su segunda variable con constante de Lipschitz $M$, y $f\in C^1(D)$. Sea $h = \frac{b-a}{N}$, con $N \in \bb{N}$. Se considera el método de Euler para resolver el PVI~\eqref{eq:pvi} con tamaño de paso $h$. Demuestre que:
    \begin{itemize}
        \item Si $M = 0$, entonces $|e_n| \leq \frac{(b-a)M^*}{2} h$ para todo $n = 0, \ldots, N$.
        \item Si $M > 0$, entonces $|e_n| \leq \dfrac{e^{(b-a)M} - 1}{M} \cdot \frac{M^*}{2} h$ para todo $n = 0, \ldots, N$,
    \end{itemize}
    donde $e_n$ es el error de truncatura global del método en el punto $t_n$, y $M^*$ es tal que $|x''(t)| \leq M^*$ para todo $t \in [a, b]$, siendo $x(t)$ la única solución de~\eqref{eq:pvi}.    
\end{ejercicio}


\begin{ejercicio}\label{ej:3.1.2}
    Demuestre que el PVI $x' = \nicefrac{-1}{2} x$, $x(0) = 2$ tiene una única solución en $[0, 1]$ y halle una cota del error de truncatura global en cada nodo del método de Euler de tamaño de paso $h$ para aproximar dicha solución.
\end{ejercicio}

\begin{ejercicio}\label{ej:3.1.3}
    Demuestre que si $f : D \to \bb{R}$ es continua y lipschitziana en su segunda variable, entonces el método del punto medio para resolver el PVI~\eqref{eq:pvi} es estable, consistente con~\eqref{eq:pvi} y converge a la solución de~\eqref{eq:pvi}.
\end{ejercicio}

\begin{ejercicio}\label{ej:3.1.4}
    Razone la veracidad o falsedad de la afirmación siguiente: El método de Euler modificado (Heun) proporciona la solución exacta de la EDO dada por $x' = -2\lambda t$.
\end{ejercicio}

\begin{ejercicio}\label{ej:3.1.5}
    Demuestre que si $f : D \to \bb{R}$ es continua y lipschitziana en su segunda variable, entonces el método de Runge-Kutta clásico para resolver el PVI~\eqref{eq:pvi} es estable, consistente con~\eqref{eq:pvi} y converge a la solución de~\eqref{eq:pvi}.
\end{ejercicio}

\begin{ejercicio}\label{ej:3.1.6}
    Razone la veracidad o falsedad de las siguientes afirmaciones:
    \begin{enumerate}
        \item Al aplicar el método de un paso del punto medio al problema $x'(t) = -x(t) + 1$, $x(0) = 2$ se obtiene la solución numérica $\{t_n, x_n\}_{n=0}^N$ donde $x_n = A^n + 1$ con $A = 1 - h + \frac{h^2}{2}$.
        \item El orden de un método explícito de un paso cuya ecuación en diferencias es:
        \begin{equation*}
            x_{n+1} = x_n + h \Phi(x_n; t_n, h),\qquad n = 0, \ldots, N-1
        \end{equation*}
        con $\Phi(x; t, h) = f(t, x) + \frac{h^2}{2} x''(t)$ es al menos 2.
    \end{enumerate}
\end{ejercicio}

\begin{ejercicio}\label{ej:3.1.7}
    Demuestre que el PVI $x' = t - x$, $x(0) = 1$ tiene una única solución en $[0, 1]$. ¿Se puede aproximar dicha solución mediante:
    \begin{itemize}
        \item el método de Taylor de orden 2?
        \item el método de Heun?
        \item el método del punto medio?
    \end{itemize}
    ¿Por qué? Escriba la ecuación en diferencias de cada uno de estos métodos para resolver el PVI considerado. ¿Ocurre algo reseñable?
\end{ejercicio}

\begin{ejercicio}\label{ej:3.1.8}
    Demuestre que el PVI $x' = t^2 x$, $x(0) = 1$ tiene una única solución $x(t)$ en $[0, 1]$. Demuestre que el método de Taylor de orden 2 y el método de Runge-Kutta clásico convergen a $x(t)$ y halle las aproximaciones de cada uno de estos dos métodos para el tamaño de paso $h = 0.2$.
\end{ejercicio}

\begin{ejercicio}\label{ej:3.1.9}
    Utilizando el método de Runge-Kutta clásico para resolver el PVI $x' = f(t)$, $x(0) = \mu$, deduzca una conocida fórmula de integración numérica e indique de qué fórmula se trata.
\end{ejercicio}

\begin{ejercicio}\label{ej:3.1.10}
    Para el método de Runge-Kutta de 2 evaluaciones con arreglo de Butcher:
    \begin{equation*}
        \begin{array}{c|cc}
            0 & 0 & 0 \\ 
            \alpha & \alpha & 0 \\ \hline
             & \frac{1-\alpha}{2} & \frac{1+\alpha}{2}
        \end{array}
    \end{equation*}
    \begin{enumerate}
        \item Determine el orden del método según los valores del escalar $\alpha$.
        \item Para $\alpha$ adecuado para que el método tenga orden máximo, ¿cuál es el término principal del error local de truncatura?
        \item ¿Es estable (cero-estable) el método para todo $\alpha$ si $f(t, x)$ es continua y lipschitziana respecto de $x$ sobre $D$?
    \end{enumerate}
\end{ejercicio}




\begin{comment}
6. Razone la veracidad o falsedad de las siguientes afirmaciones.
a) Al aplicar el m´etodo de un paso del punto medio al problema x
0
(t) = −x(t) +
1, x(0) = 2 se obtiene la soluci´on num´erica {tn, xn}
N
n=0 donde xn = An + 1 con
A = 1 − h +
h
2
2
.
b) El orden de un m´etodo expl´ıcito de un paso cuya ecuaci´on en diferencias es
xn+1 = xn + hΦ(xn;tn, h), n = 0, . . . , N − 1
con Φ(x;t, h) = f(t, x) + h
2
x
00(t) es al menos 2.
7. Demuestre que el PVI x
0 = t − x, x(0) = 1 tiene una ´unica soluci´on en [0, 1]. ¿Se puede
aproximar dicha soluci´on mediante
el m´etodo de Taylor de orden 2?
el m´etodo de Heun?
el m´etodo del punto medio?
¿Por qu´e? Escriba la ecuaci´on en diferencias de cada uno de estos m´etodos para resolver
el PVI considerado. ¿Ocurre algo rese˜nable?
8. Demuestre que el PVI x
0 = t
2x, x(0) = 1 tiene una ´unica soluci´on x(t) en [0, 1]. Demuestre
que el m´etodo de Taylor de orden 2 y el m´etodo de Runge-Kutta cl´asico convergen a x(t)
y halle las aproximaciones de cada uno de los estos dos m´etodos para el tama˜no de paso
h = 0.2.
9. Utilizando el m´etodo de Runge-Kutta cl´asico para resolver el PVI x
0 = f(t), x(0) = µ,
deduzca una conocida f´ormula de integraci´on num´erica e indique de qu´e f´ormula se trata.
10. Para el m´etodo de Runge-Kutta de 2 evaluaciones con arreglo de Butcher
0 0 0
α α 0
1−α
2
1+α
2
a) Determine el orden del m´etodo seg´un los valores del escalar α.
b) Para α adecuado para que el m´etodo tenga orden m´aximo, ¿cu´al es el t´ermino
principal del error local de truncatura?
c) ¿Es estable (cero-estable) el m´etodo para todo α si f(t, x) es continua y lipschitziana
respecto de x sobre D?

\end{comment}
\subsection{Relación 1. Integración Numérica}
\setcounter{ejercicio}{8}

\begin{ejercicio}\label{ej:2.1.9}
    Se considera la siguiente fórmula de integración numérica:
    \begin{equation*}
        \int_{-1}^{1} f(x) dx \approx \alpha_0 f(-1) + \alpha_1 f(0) + \alpha_2 f(1).
    \end{equation*}
    \begin{enumerate}
        \item Halle de tres formas distintas los valores $\alpha_i \in \mathbb{R}$, $i = 0, 1, 2$ para que dicha fórmula sea de tipo interpolatorio en $\bb{P}_2$.
        \item Halle el grado de exactitud de la fórmula obtenida en el apartado anterior.
        \item Proporcione una fórmula de la forma propuesta que no sea de tipo interpolatorio en $\bb{P}_2$. Justifique la respuesta.
    \end{enumerate}
\end{ejercicio}

\begin{ejercicio}\label{ej:2.1.10}
    Para evaluar el funcional lineal $L(f) = f(\nicefrac{1}{2}) + \int_{0}^{1} f(x) dx$ con $f \in C^1[0, 1]$ se propone la siguiente fórmula de aproximación lineal:
    \begin{equation*}
        L(f) \approx \frac{1}{8} (13f(0) + 5f'(0) + 3f(1)).
    \end{equation*}
    \begin{enumerate}
        \item ¿La fórmula es exacta para $f(x) = x^3$? ¿Por qué? ¿Cuál es el grado de exactitud de la fórmula? ¿Por qué?
        \item ¿La fórmula es de tipo interpolatorio en $\bb{P}_2$? ¿Por qué? Si no lo es, deduzca los coeficientes de la fórmula para que sí lo sea.
        \item Deduzca el error de la fórmula obtenida en el apartado anterior cuando $f \in C^3[0, 1]$.
        \item ¿Qué fórmula se obtiene si deseamos que sea de tipo interpolatorio en el espacio $V = \langle 1, \sen(\pi x), \cos(\pi x)\rangle$?
    \end{enumerate}
\end{ejercicio}

\begin{ejercicio}\label{ej:2.1.11}~
    \begin{enumerate}
        \item Use el método interpolatorio para obtener la fórmula del rectángulo izquierda y deduzca la expresión del error de esta fórmula cuando $f \in C^1[a, b]$. ¿Cuál es el grado de exactitud de dicha fórmula? Justifique la respuesta.
        \item Utilice el método interpolatorio para obtener la fórmula del punto medio y deduzca la expresión del error de esta fórmula cuando $f \in C^2[a, b]$.
    \end{enumerate}
\end{ejercicio}

\begin{ejercicio}\label{ej:2.1.12}
    Obtenga la fórmula del trapecio calculando directamente sus coeficientes mediante la base de Lagrange del problema de interpolación unisolvente asociado a dicha fórmula. Halle la expresión del error de cuadratura de esta fórmula cuando $f \in C^2[a, b]$ y deduzca cuál es su grado de exactitud.
\end{ejercicio}

\begin{ejercicio}\label{ej:2.1.13}
    Halle la fórmula de Newton-Cotes abierta con dos nodos.
\end{ejercicio}

\begin{ejercicio}\label{ej:2.1.14}
    Aproxime el valor de $\ln 2 = \int_{1}^{2} \nicefrac{1}{x} dx$ usando cada una de las siguientes fórmulas, y calcule una cota del valor absoluto del error que se comete en cada una de las aproximaciones obtenidas:
    \begin{enumerate}
        \item la fórmula del punto medio,
        \item la fórmula del trapecio,
        \item la fórmula de Simpson.
    \end{enumerate}
\end{ejercicio}

\begin{ejercicio}\label{ej:2.1.15}
    Halle la expresión del error de la fórmula del trapecio compuesta cuando $f \in C^2[a, b]$.
\end{ejercicio}

\begin{ejercicio}\label{ej:2.1.16}
    La fórmula de integración numérica de tipo interpolatorio en $P_3$ de la forma
    \begin{equation*}
        \int_{a}^{b} f(x) dx \approx \alpha_0 f(a) + \alpha_1 f(b) + \alpha_2 f'(a) + \alpha_3 f'(b)
    \end{equation*}
    es conocida como \emph{fórmula del trapecio corregida}.
    \begin{enumerate}
        \item Halle dicha fórmula, así como la expresión de su error si $f \in C^4[a, b]$. ¿Cuál es su grado de exactitud? Razone la respuesta.
        \item Escriba la fórmula de cuadratura compuesta que se obtiene a partir de ella para una partición uniforme del intervalo de integración y halle la expresión de su error cuando $f \in C^4[a, b]$.
    \end{enumerate}
\end{ejercicio}

\begin{ejercicio}\label{ej:2.1.17}
    De una función $f(x)$ se conocen los valores
    \begin{center}
        \begin{tabular}{c|c|c|c|c|c}
            $x$ & $-2$ & $-1$ & $0$ & $1$ & $2$ \\
            \hline
            $f(x)$ & $2$ & $1$ & $0.5$ & $0.75$ & $1.5$
        \end{tabular}
    \end{center}
    Se pide:
    \begin{enumerate}
        \item Calcule un valor aproximado de $\int_{-2}^{2} f(x) dx$ usando las siguientes fórmulas. Interprete geométricamente cada una de las fórmulas anteriores para aproximar $\int_{-2}^{2} f(x) dx$ y haga un dibujo que ilustre la respuesta.
        \begin{enumerate}
            \item Fórmula del punto medio compuesta,
            \item Fórmula del trapecio compuesta,
            \item Fórmula de Simpson compuesta.
        \end{enumerate}
    \end{enumerate}
\end{ejercicio}

\begin{ejercicio}\label{ej:2.1.18}
    Calcule valores aproximados para $\ln 2 = \int_{1}^{2} \nicefrac{1}{x} dx$ usando las siguientes fórmulas y calcule una cota del valor absoluto del error que se comete en cada una de las aproximaciones obtenidas:
    \begin{enumerate}
        \item la fórmula del punto medio compuesta con 4 subintervalos,
        \item la fórmula del trapecio compuesta con 4 subintervalos,
        \item la fórmula de Simpson compuesta con 2 subintervalos.
    \end{enumerate}
    
    Compare los resultados obtenidos con los del problema \ref{ej:2.1.14}.
\end{ejercicio}

\begin{ejercicio}\label{ej:2.1.19}
    Se quiere calcular $\int_{0}^{1} e^{-x^2} dx$ usando la fórmula del punto medio compuesta. ¿Cuántos subintervalos deberá tener la partición uniforme a considerar para que el valor absoluto del error que se cometa al aproximar dicha integral sea menor que $5 \times 10^{-7}$?
\end{ejercicio}

\begin{ejercicio}\label{ej:2.1.20}
    Se quiere hallar un valor aproximado de $\int_{1}^{2} \ln^2(x) dx$.
    \begin{enumerate}
        \item ¿Cuántos subintervalos deberá tener la partición uniforme para que el valor absoluto del error que se cometa al aproximar dicha integral usando la fórmula del trapecio compuesta sea menor que $10^{-3}$?
        \item ¿Cuántos subintervalos deberá tener la partición uniforme para que el valor absoluto del error que se cometa al aproximar dicha integral utilizando la fórmula de Simpson compuesta sea menor que $10^{-3}$?
        \item ¿Cuál de las fórmulas compuestas de los apartados anteriores será preferible utilizar para aproximar la integral dada? Justifique la respuesta.
    \end{enumerate}
\end{ejercicio}

\begin{ejercicio}\label{ej:2.1.21}
    Se consideran las fórmulas simples del rectángulo, del punto medio, del trapecio y de Simpson. ¿Alguna de ellas es una fórmula de cuadratura gaussiana? Justifique la respuesta.
\end{ejercicio}

\begin{ejercicio}\label{ej:2.1.22}~
    \begin{enumerate}
        \item La fórmula numérica $f''(a) \approx \frac{3f(-2h) - 5f(0) + 2f(3h)}{15h^2}$:
        \begin{enumerate}
            \item ¿Es de tipo interpolatorio clásico?
            \item ¿Cuál es su grado de exactitud?
        \end{enumerate}
        \item ¿Existe alguna fórmula de derivación numérica, basada en $n + 1$ nodos distintos, de la forma $f''(a) \approx \alpha_0 f(x_0) + \alpha_1 f(x_1) + \cdots + \alpha_n f(x_n)$ que sea de tipo interpolatorio en $P_n$ y tenga grado de exactitud $n + 3$? Justifique la respuesta.
        \item Justifique que la fórmula de integración numérica $$\int_{0}^{2} f(x) dx = \frac{2}{3} (3f(0) + 3f'(0) + 2f''(0)) + R(f)$$ es de tipo interpolatorio en $P_2$ y para $f \in C^3[0, 2]$ su error se puede escribir como $R(f) = \frac{2}{3} f'''(\xi)$ con $\xi \in ]0, 2[$.
    \end{enumerate}
\end{ejercicio}

\begin{ejercicio}\label{ej:2.1.23}~
    \begin{enumerate}
        \item Calcule la fórmula de Gauss-Legendre con 2 nodos y halle la expresión de su error cuando $f \in C^4[-1, 1]$.
        \item Usando la fórmula del apartado anterior, obtenga una fórmula para aproximar $\int_{a}^{b} f(x) dx$ con $[a, b]$ un intervalo cualquiera. ¿Cuál es la expresión del error de dicha fórmula cuando $f$ es suficientemente regular?
    \end{enumerate}
\end{ejercicio}

\begin{ejercicio}\label{ej:2.1.24}
    Calcule la fórmula de Gauss-Legendre con 3 nodos y halle la expresión de su error cuando $f \in C^6[-1, 1]$.
\end{ejercicio}

\begin{ejercicio}\label{ej:2.1.25}
    Calcule la fórmula gaussiana con dos nodos de la forma $$\int_{-1}^{1} |x|f(x) dx \approx \alpha_0 f(x_0) + \alpha_1 f(x_1)$$ y halle la expresión de su error cuando $f \in C^4[-1, 1]$.
\end{ejercicio}

\begin{ejercicio}\label{ej:2.1.26}~
    \begin{enumerate}
        \item Determine los valores de $A$, $B$, $C$ de forma que la fórmula de cuadratura $\int_{0}^{3} f(x) dx \approx Af(0) + Bf(C)$ tenga el máximo grado de exactitud posible.
        \item Utilizando la fórmula obtenida en el apartado anterior, obtenga una fórmula para aproximar $\int_{a}^{b} f(x) dx$ con $[a, b]$ un intervalo cualquiera.
        \item Deduzca la fórmula de cuadratura compuesta asociada a la fórmula obtenida en el apartado anterior para una partición uniforme del intervalo de integración $[a, b]$ en $N$ subintervalos.
    \end{enumerate}
\end{ejercicio}

\begin{ejercicio}\label{ej:2.1.27}
    Halle la fórmula de cuadratura de la forma $\int_{-1}^{1} f(x) dx \approx \alpha_0 f(-1) + \alpha_1 f(x_1) + \alpha_2 f(1)$ que tiene el máximo grado de exactitud posible. ¿Cuál es ese grado de exactitud máximo? ¿Se trata de una fórmula de cuadratura gaussiana? ¿Por qué?
\end{ejercicio}

\begin{ejercicio}\label{ej:2.1.28} Las fórmulas gaussianas de Laguerre corresponden a una integral del tipo
    \begin{equation*}
        \int_{0}^{\infty} e^{-x} f(x) dx.
    \end{equation*}
    \begin{enumerate}
        \item Obtenga la fórmula de Laguerre con dos nodos.
        \item Aplique la fórmula obtenida para aproximar $\int_{0}^{\infty} e^{-x} x^4 dx$.
    \end{enumerate}
\end{ejercicio}

\begin{ejercicio}\label{ej:2.1.29}
    Se denominan fórmulas de integración de Lobatto a aquellas que usan como nodos los dos extremos del intervalo de integración, y eligen los restantes para alcanzar la máxima exactitud posible. Trapecio y Simpson son ejemplos de fórmulas de Lobatto, siendo la del trapecio la más sencilla.
    \begin{enumerate}
        \item Obtenga la fórmula de Lobatto del tipo $$\int_{-1}^{1} f(x) dx \approx \alpha_0 f(-1) + \alpha_1 f(x_1) + \alpha_2 f(x_2) + \alpha_3 f(1)$$
        \item Obtenga la expresión del error de dicha fórmula.
    \end{enumerate}
\end{ejercicio}

\begin{ejercicio}\label{ej:2.1.30}
    Se necesita calcular $$\int_{0}^{1} x f(x) dx \approx \alpha_0 f(x_0) + \alpha_1 f(x_1)$$ Obtenga $x_0$, $x_1$, $\alpha_0$, $\alpha_1$ para que la fórmula anterior tenga exactitud máxima.
\end{ejercicio}
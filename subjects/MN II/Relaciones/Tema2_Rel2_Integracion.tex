\subsection{Relación 2. Integracion Numérica}
\setcounter{ejercicio}{0}


\begin{ejercicio}~\label{ej:2.3.1}
    En la integración numérica se obtienen fórmulas simples, con pocos nodos, para aproximar la integral en un intervalo $[a, b]$. Estas fórmulas, al tener pocos nodos, no dan resultados satisfactorios en ocasiones, pero unas tienen un mayor grado de exactitud que otras.
    \begin{enumerate}
        \item Explica cómo podrías obtener fórmulas de tipo interpolatorio clásico con más exactitud de $n$, cuando puedes elegir libremente los nodos de interpolación: $x_0, \ldots, x_n$.
        \item Sea $a$ igual a la suma de los dígitos de tu DNI; sea $b = a + 3$. Calcula la fórmula con nodos $a, x_1$ de mayor grado de exactitud para aproximar la integral entre $a$ y $b$.
        \item Sea
        \begin{equation*}
            f(x) = \begin{cases}
                x^2, & x \leq 0 \\
                -x^2, & x > 0
            \end{cases}
        \end{equation*}
        Aplica la fórmula simple obtenida para aproximar, previo cambio de variable si es necesario, el valor de la integral: $\int_{-1}^{1} f(x)dx$. Aplica la fórmula compuesta asociada a la fórmula simple, haciendo dos subintervalos a partir del $[-1,1]$, para aproximar el mismo valor de la integral anterior.
        \item ¿Qué puedes decir del error de la fórmula simple obtenida, de la compuesta asociada a ella y de sus aplicaciones particulares en el apartado anterior?
    \end{enumerate}
\end{ejercicio}

\begin{ejercicio}~\label{ej:2.3.2}
    Determina razonadamente si es posible diseñar una fórmula numérica de tipo interpolatorio en el espacio generado por $\{1, x, x^2, x^4\}$ para aproximar
    \begin{equation*}
        \int_{-2}^{2} f(x) dx + \int_{-2}^{2} |x|f(x) dx
    \end{equation*}
    usando para ello los datos $\int_{-1}^{1} f(x) dx$, $\int_{-1}^{1} |x|f(x) dx$, $f(0)$ y $f'(0)$. En particular determina el peso de $f'(0)$.
\end{ejercicio}

\begin{ejercicio}~\label{ej:2.3.3}
    Considera la fórmula de cuadratura simple del trapecio en la forma
    \begin{equation*}
        \int_{a}^{b} f(x) dx = T(a, b) + R(f).
    \end{equation*}
    \begin{enumerate}
        \item Obtén la expresión del error $R(f)$ para $f$ suficientemente regular.
        \item Obtén la fórmula compuesta asociada y la correspondiente expresión del error.
        \item Llama $h = b - a$. De forma similar a la vista en clase para la integración adaptativa con la fórmula de Simpson, obtén un criterio de estimación del error
        \begin{equation*}
            \int_{a}^{b} f(x) dx - T(a, m) - T(m, b)
        \end{equation*}
        basado en $T(a, b)$, $T(a, m)$ y $T(m, b)$, siendo $m = \frac{a+b}{2}$.
        \item Estima el error cometido en la aproximación en dos subintervalos
        \begin{equation*}
            \int_{4}^{8} \left(1 + \frac{e^{-x}}{x}\right)dx \approx T(4, 6) + T(6, 8) = 4.0054471
        \end{equation*}
        sabiendo que $f(4) = 1.0045789$, $f(6) = 1.0004131$ y $f(8) = 1.0000419$.
    \end{enumerate}
\end{ejercicio}

\begin{ejercicio}~\label{ej:2.3.4}
    Se pretende aproximar una integral del tipo
    \begin{equation*}
        \int_{-1}^{1} f(x)(1 - x^2)dx
    \end{equation*}
    utilizando tres nodos, es decir:
    \begin{equation*}
        \int_{-1}^{1} f(x)(1 - x^2)dx \approx \alpha_0 f(x_0) + \alpha_1 f(x_1) + \alpha_2 f(x_2)
    \end{equation*}
    \begin{enumerate}
        \item\label{ap:1} Si fijamos los nodos $x_0 = -1$, $x_1 = 0$ y $x_2 = 1$, determina el valor de los parámetros para que sea una fórmula de tipo interpolatorio, así como el orden de exactitud de dicha fórmula.
        \item ¿Cuáles serían los nodos si utilizamos una fórmula de Newton-Cotes abierta?
        \item\label{ap:3} Determina la fórmula gaussiana correspondiente así como la expresión del error.
        \item Utiliza las fórmulas de los apartados \ref{ap:1} y \ref{ap:3} para aproximar
        \begin{equation*}
            \int_{-1}^{1} \cos(x^2)(1 - x^2)dx
        \end{equation*}
    \end{enumerate}
\end{ejercicio}

\begin{ejercicio}~\label{ej:2.3.5}
    Se considera la fórmula de integración numérica
    \begin{equation*}
        \int_{-1}^{2} f(x)x dx \sim a_0 f(0) + a_1 f(2) + a_2 f'(0) + a_3f'(2).
    \end{equation*}
    \begin{enumerate}
        \item Determina los coeficientes $a_0$, $a_1$, $a_2$ y $a_3$ para que la fórmula anterior sea de tipo interpolatorio.
        \item Indica el grado de exactitud de la fórmula anterior. ¿Es el grado de exactitud superior al esperado?
        \item Si se pretende utilizar una fórmula gaussiana con 2 nodos para aproximar la integral, determina cuáles serían dichos nodos y la expresión del error cometido en la aproximación.
    \end{enumerate}
\end{ejercicio}

\begin{ejercicio}~\label{ej:2.3.6}
    Se pretende aproximar la integral
    \begin{equation*}
        \int_{a}^{b} f(x)dx = S_n(f) + R(f) \tag{1}
    \end{equation*}
    donde $S_n(f)$ es una fórmula de integración compuesta obtenida al hacer una partición uniforme del intervalo $[a, b]$ de la forma:
    \begin{equation*}
        a = x_0 < x_1 < \ldots < x_n = b, \quad x_i = x_{i-1} + h, \quad h = \frac{b - a}{n}
    \end{equation*}
    y $R(f)$ es el error de integración numérica que tiene el siguiente desarrollo:
    \begin{equation*}
        R(f) = a_1h^3 + a_2h^6 + \cdots + a_mh^{3m} + \cdots
    \end{equation*}
    Siguiendo el mismo argumento de la integración de Romberg, combina $S_n(f)$ con $S_{3n}(f)$ para obtener una aproximación más precisa para la integral. Aplica recursivamente el método.
\end{ejercicio}

\begin{ejercicio}~\label{ej:2.3.7}
    A veces, para construir fórmulas de integración numérica es posible utilizar nodos que se encuentran fuera del intervalo de integración. Considera la fórmula:
    \begin{equation*}
        \int_{a}^{a+h} f(x)dx = \frac{5h}{12}f(a + h) + \frac{2h}{3}f(a) - \frac{h}{12}f(a - h) + R(f)
    \end{equation*}
    \begin{enumerate}
        \item Demuestra que es de tipo interpolatorio y determina el grado de exactitud.
        \item Proporciona una expresión para el error de integración numérica asociado a la fórmula.
        \item Deduce la fórmula compuesta asociada a dicha fórmula.
    \end{enumerate}
\end{ejercicio}

\begin{ejercicio}~\label{ej:2.3.8}
    Se considera la fórmula de integración numérica
    \begin{equation*}
        \int_{-1}^{1} f(x)(1 - x^2)dx = \alpha_0 f(x_0) + \alpha_1 f(x_1) + R(f).
    \end{equation*}
    \begin{enumerate}
        \item Determina los nodos y los coeficientes para que la fórmula anterior tenga grado de exactitud máximo. ¿Cuál es ese grado de exactitud?
        \item Obtén la expresión del error de dicha fórmula.
        \item Utiliza la fórmula anterior para estimar el valor de
        \begin{equation*}
            \int_{-1}^{1} \ln(x^2 + 1)(1 - x^2)dx .
        \end{equation*}
    \end{enumerate}
\end{ejercicio}

\begin{ejercicio}~\label{ej:2.3.9}
    Considera la fórmula de cuadratura de tipo interpolatorio:
    \begin{equation*}
        \int_{a}^{a+h} f(x)dx = \frac{3h}{4}f(a) + \frac{h}{4}f(a + 2h) + R(f).
    \end{equation*}
    \begin{enumerate}
        \item Proporciona una expresión para el error de integración numérica asociado a la fórmula.
        \item Deduce la fórmula compuesta asociada a dicha fórmula incluyendo una expresión del error.
        \item Deduce un método multipaso lineal para aproximar la solución del PVI
        \begin{equation*}
            \begin{cases}
                x' = f(f, x) \\
                x(t_0) = \mu
            \end{cases}
        \end{equation*}
    \end{enumerate}
\end{ejercicio}






\begin{comment}
1 En la integraci´on num´erica se obtienen f´ormulas simples, con pocos nodos, para aproximar la integral en un intervalo [a, b]. Est´as f´ormulas, al tener pocos nodos, no dan resultados satisfactorios en
ocasiones, pero unas tienen un mayor grado de exactitud que otras.
(a) Explica c´omo podr´ıas obtener f´ormulas de tipo interpolatorio cl´asico con m´as exactitud de n,
cuando puedes elegir libremente los nodos de interpolaci´on: x0, . . . , xn.
(b) Sea a igual a la suma de los d´ıgitos de tu dni; sea b = a + 3. Calcula la f´ormula con nodos a, x1
de mayor grado de exactitud para aproximar la integral entre a y b.
(c) Sea
f(x) = 
x
2
, x ≤ 0
−x
2
, x > 0
Aplica la f´ormula simple obtenida para aproximar, previo cambio de variable si es necesario,
el valor de la integral: Z 1
−1
f(x)dx. Aplica la f´ormula compuesta asociada a la f´ormula simple,
haciendo dos subintervalos a partir del [-1,1] , para aproximar el mismo valor de la integral
anterior.
(d) ¿Qu´e puedes decir del error de la f´ormula simple obtenida, de la compuesta asociada a ella y de
sus aplicaciones particulares en el apartado anterior?
2 Determina razonadamente si es posible dise˜nar una f´ormula num´erica de tipo interpolatorio en el
espacio generado por h1, x, x2
, x4
i para aproximar
Z 2
−2
f(x) dx +
Z 2
−2
|x|f(x) dx
usando para ello los datos Z 1
−1
f(x) dx,
Z 1
−1
|x|f(x) dx, f(0) y f
0
(0). En particular determina el peso
de f
0
(0).
3 Considera la f´ormula de cuadratura simple del trapecio en la forma
Z b
a
f(x) dx = T(a, b) + R(f).
a) Obt´en la expresi´on del error R(f) para f suficientemente regular.
b) Obt´en la f´ormula compuesta asociada y la correspondiente expresi´on del error.
c) Llama h = b − a. De forma similar a la vista en clase para la integraci´on adaptativa con la
f´ormula de Simpson, obt´en un criterio de estimaci´on del error




Z b
a
f(x) dx − T(a, m) − T(m, b)




basado en T(a, b), T(a, m) y T(m, b), siendo m =
a+b
2
.
d) Estima el error cometido en la aproximaci´on en dos subintervalos
Z 8
4

1 +
e
−x
x

dx ≈ T(4, 6) + T(6, 8) = 4.0054471
sabiendo que f(4) = 1.0045789, f(6) = 1.0004131 y f(8) = 1.0000419.


4 Se pretende aproximar una integral del tipo
Z 1
−1
f(x)(1 − x
2
)dx
utilizando tres nodos, es decir:
Z 1
−1
f(x)(1 − x
2
)dx ≈ α0f(x0) + α1f(x1) + α2f(x2)
a) Si fijamos los nodos x0 = −1, x1 = 0 y x2 = 1, determina el valor de los par´ametros para que
sea una f´ormula de tipo interpolatorio, as´ı como el orden de exactitud de dicha f´ormula.
b) ¿Cu´ales ser´ıan los nodos si utilizamos una f´ormula de Newton-Cotes abierta?
c) Determina la f´ormula gaussiana correspondiente as´ı como la expresi´on del error.
d) Utiliza las f´ormulas de los apartados a) y c) para aproximar
Z 1
−1
cos(x
2
)(1 − x
2
)dx
5 Se considera la f´ormula de integraci´on num´erica
Z 2
−1
f(x)x dx ∼ a0 f(0) + a1 f(2) + a2 f
0
(0) + a3f
0
(2).
a) Determina los coeficientes a0, a1, a2 y a3 para que la f´ormula anterior sea de tipo interpolatorio.
b) Indica el grado de exactitud de la f´ormula anterior. ¿Es el grado de exactitud superior al esperado?
c) Si se pretende utilizar una f´ormula gaussiana con 2 nodos para aproximar la integral, determina
cu´ales ser´ıan dichos nodos y la expresi´on del error cometido en la aproximaci´on.
6 Se pretende aproximar la integral
Z b
a
f(x)dx = Sn(f) + R(f) (1)
donde Sn(f) es una f´ormula de integraci´on compuesta obtenida al hacer una partici´on uniforme del
intervalo [a, b] de la forma:
a = x0 < x1 < . . . < xn = b, xi = xi−1 + h, h =
b − a
n
y R(f) es el error de integraci´on num´erica que tiene el siguiente desarrollo:
R(f) = a1h
3 + a2h
6 + · · · + amh
3m + . . .
Siguiendo el mismo argumento de la integraci´on de Romberg, combina Sn(f) con S3n(f) para obtener
una aproximaci´on m´as precisa para la integral. Aplica recursivamente el m´etodo.
7 A veces, para construir f´ormulas de integraci´on num´erica es posible utilizar nodos que se encuentran
fuera del intervalo de integraci´on. Considera la f´ormula:
Z a+h
a
f(x)dx =
5h
12
f(a + h) + 2h
3
f(a) −
h
12
f(a − h) + R(f)
a) Demuestra que es de tipo interpolatorio y determina el grado de exactitud.
b) Proporciona una expresi´on para el error de integraci´on num´erica asociado a la f´ormula.



c) Deduce la f´ormula compuesta asociada a dicha f´ormula.
8 Se considera la f´ormula de integraci´on num´erica
Z 1
−1
f(x)(1 − x
2
)dx = α0 f(x0) + α1 f(x1) + R(f).
a) Determina los nodos y los coeficientes para que la f´ormula anterior tenga grado de exactitud
m´aximo. ¿Cu´al es ese grado de exactitud?
b) Obt´en la expresi´on del error de dicha f´ormula.
c) Utiliza la f´ormula anterior para estimar el valor de
Z 1
−1
ln(x
2 + 1)(1 − x
2
)dx .
9 Considera la f´ormula de cuadratura de tipo interpolatorio:
Z a+h
a
f(x)dx =
3h
4
f(a) + h
4
f(a + 2h) + R(f)
a) Proporciona una expresi´on para el error de integraci´on num´erica asociado a la f´ormula.
b) Deduce la f´ormula compuesta asociada a dicha f´ormula incluyendo una expresi´on del error.
c) Deduce un m´etodo multipaso lineal para aproximar la soluci´on del PVI

x
0 = f(f, x)
x(t0) = µ
(2)
\end{comment}
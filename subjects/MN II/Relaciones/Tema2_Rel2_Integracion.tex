\subsection{Relación 2. Integracion Numérica}
\setcounter{ejercicio}{0}


\begin{ejercicio}~\label{ej:2.3.1}
    En la integración numérica se obtienen fórmulas simples, con pocos nodos, para aproximar la integral en un intervalo $[a, b]$. Estas fórmulas, al tener pocos nodos, no dan resultados satisfactorios en ocasiones, pero unas tienen un mayor grado de exactitud que otras.
    \begin{enumerate}
        \item Explica cómo podrías obtener fórmulas de tipo interpolatorio clásico con más exactitud de $n$, cuando puedes elegir libremente los nodos de interpolación: $x_0, \ldots, x_n$.
        
        Al tratarse de fórmulas de tipo interpolatorio con $n+1$ nodos, sabemos que tendrán grado mínimo de exactitud $n$. Para añadir $k$ grados de exactitud más, hemos de imponer exactitud en $\{x^i\Pi(x)\mid i\in \{0,1,\ldots,k-1\}\}$, donde $k\in \{1,2,n+1\}$.

        \item Sea $a$ igual a la suma de los dígitos de tu DNI; sea $b = a + 3$. Calcula la fórmula con nodos $a, x_1$ de mayor grado de exactitud para aproximar la integral entre $a$ y $b$.
        
        Buscamos por tanto la siguiente fórmula de tipo interpolatorio:
        \begin{equation*}
            \int_{a}^{b} f(x)dx \approx \alpha_0 f(a) + \alpha_1 f(x_1)
        \end{equation*}

        Definimos el siguiente polinomio:
        \begin{equation*}
            \Pi(x) = (x-a)(x-x_1) = x^2 - (a+x_1)x + ax_1
        \end{equation*}

        Sabemos que será exacta en $\bb{P}_1$, y el grado máximo de exactitud es $2\cdot 1+1=3$. Por tanto, buscamos por tanto imponer exactitud en $\{\Pi(x), x\Pi(x)\}$.
        \begin{align*}
            0 = \int_{a}^{b} \Pi(x)dx &= \int_{a}^{b} (x^2 - (a+x_1)x + ax_1)dx = \left[\dfrac{x^3}{3} - \dfrac{(a+x_1)x^2}{2} + ax_1x\right]_{a}^{a+3}\\
            &= \left[\dfrac{(a+3)^3}{3} - \dfrac{(a+x_1)(a+3)^2}{2} + ax_1(a+3)\right] - \left[\dfrac{a^3}{3} - \dfrac{(a+x_1)a^2}{2} + ax_1a\right]\\
            &= 3[a^2+3a+3] - \dfrac{(a+x_1)(6a+9)}{2} + 3ax_1\\
            &= \cancel{3a^2} + 9a + 9 - \cancel{3a^2}-\cancel{3ax_1} - \dfrac{9}{2}(a+x_1) + \cancel{3ax_1}\\
            &= \dfrac{9}{2}\left(2a+2-a-x_1\right) = \dfrac{9}{2}(a+2-x_1)
        \end{align*}

        Por tanto, tenemos que $x_1 = a + 2$, y ya tenemos determinados los nodos.
        \begin{equation*}
            \int_{a}^{b} f(x)dx \approx \alpha_0 f(a) + \alpha_1 f(a+2)
        \end{equation*}

        Buscamos ahora los coeficientes $\alpha_0$ y $\alpha_1$, imponiendo exactitud en $\{1,x\}$:
        \begin{align*}
            \int_{a}^{b} 1 &= b-a = 3 = \alpha_0 + \alpha_1\\
            \int_{a}^{b} x &= \left[\dfrac{x^2}{2}\right]_{a}^{a+3} = \dfrac{(a+3)^2}{2} - \dfrac{a^2}{2} = \dfrac{6a+9}{2} = \alpha_0\cdot a + \alpha_1\cdot (a+2)
        \end{align*}

        Tenemos por tanto el siguiente sistema de ecuaciones:
        \begin{equation*}
            \begin{cases}
                \alpha_0 + \alpha_1 = 3\\
                a(\alpha_0 + \alpha_1) + 2\alpha_1 = \dfrac{6a + 9}{2}
            \end{cases}
        \end{equation*}

        La segunda ecuación queda como:
        \begin{equation*}
            a\cdot 3 + 2\alpha_1 = \dfrac{6a + 9}{2} \Longrightarrow 2\alpha_1 = \dfrac{6a + 9-6a}{2} \Longrightarrow \alpha_1 = \dfrac{9}{4}
        \end{equation*}

        Sustituyendo en la primera ecuación:
        \begin{equation*}
            \alpha_0 + \dfrac{9}{4} = 3 \Longrightarrow \alpha_0 = \dfrac{3}{4}
        \end{equation*}

        Por tanto, la fórmula de integración numérica es:
        \begin{equation*}
            \int_{a}^{b} f(x)dx \approx \dfrac{3}{4}\left(f(a) + 3f(a+2)\right)
        \end{equation*}

        Sabemos que esta fórmula es exacta en $\bb{P}_2$, veamos ahora si lo es para $x^3$:
        \begin{align*}
            \int_{a}^{b} x^3 &= \left[\dfrac{x^4}{4}\right]_{a}^{a+3} = \dfrac{(a+3)^4}{4} - \dfrac{a^4}{4} = \dfrac{12a^3 + 54a^2 + 108a + 81}{4}\\
            &\neq \dfrac{3}{4}\left(a^3 + 3(a+2)^3\right) = \dfrac{3}{4}\left(a^3 + 3(a^3+6a^2 + 12a + 8)\right) = \dfrac{3}{4}\left(4a^3 + 18a^2 + 36a + 24\right)
            =\\&= \dfrac{12a^3 + 54a^2 + 108a + 72}{4}
        \end{align*}

        Por tanto, como la fórmula no es exacta en $x^3$, el grado de exactitud es $2$.

        \item Sea $f:\bb{R}\to \bb{R}$ dada por:
        \begin{equation*}
            f(x) = \begin{cases}
                x^2 & x\leq 0 \\
                -x^2 & x > 0
            \end{cases}
        \end{equation*}

        Aplica la fórmula simple obtenida para aproximar, previo cambio de variable si es necesario, el valor de la integral: $\int_{-1}^{1} f(x)dx$.

        Buscamos un cambio de variable que nos permita aplicar la fórmula obtenida, transformando el intervalo $[a,a+3]$ en $[x_0,x_0+h]$.
        Sea este cambio de variable:
        \begin{equation*}
            x = x_0 + h\cdot \frac{t-a}{3}\qquad \forall t\in [a,a+3]
        \end{equation*}

        De esta forma, tenemos que:
        \begin{equation*}
            \int_{x_0}^{x_0+h} f(x)dx = \int_{a}^{a+3} f\left(x_0 + h\cdot \frac{t-a}{3}\right)\cdot \frac{h}{3}dt
        \end{equation*}

        Por tanto, aplicando la fórmula obtenida:
        \begin{align*}
            \int_{x_0}^{x_0+h} f(x)dx &\approx \dfrac{h}{3}\cdot \dfrac{3}{4}\left(f(x_0) + 3f\left(x_0 + h\cdot \frac{2}{3}\right)\right)
            = \dfrac{h}{4}\left(f(x_0) + 3f\left(x_0 + h\cdot \frac{2}{3}\right)\right)
        \end{align*}

        En nuestro caso, tenemos que $x_0 = -1$, $h = 2$. Por tanto:
        \begin{align*}
            \int_{-1}^{1} f(x)dx &\approx \dfrac{2}{4}\left(f(-1) + 3f\left(-1+\dfrac{2\cdot 2}{3}\right)\right)
            = \dfrac{1}{2}\left(1+3f\left(\dfrac{1}{3}\right)\right)\\
            &= \dfrac{1}{2}\left(1 + 3\cdot -\left(\dfrac{1}{3}\right)^2\right) = \dfrac{1}{2}\left(1 - \dfrac{1}{3}\right) = \dfrac{1}{2}\cdot \dfrac{2}{3} = \dfrac{1}{3}
        \end{align*}

        \item Aplica la fórmula compuesta asociada a la fórmula simple, haciendo dos subintervalos a partir del $[-1,1]$, para aproximar el mismo valor de la integral anterior.
        
        Tenemos que:
        \begin{align*}
            \int_{-1}^{1} f(x)dx &= \int_{-1}^{0} f(x)dx + \int_{0}^{1} f(x)dx
            \approx \dfrac{1}{4}\left[f(-1) + 3f\left(-1+\dfrac{2}{3}\right) + f(0) + 3f\left(0+\dfrac{2}{3}\right)\right]
            =\\&= \dfrac{1}{4}\left[f(-1) + 3f\left(-\dfrac{1}{3}\right) + f(0) + 3f\left(\dfrac{2}{3}\right)\right]
            =\\&= \dfrac{1}{4}\left[1 + 3\cdot \dfrac{1}{9} + 0 -3\cdot \dfrac{4}{9}\right] = 0
        \end{align*}
        \item ¿Qué puedes decir del error de la fórmula simple obtenida, de la compuesta asociada a ella y de sus aplicaciones particulares en el apartado anterior?
        
        El error de interpolación cometido al interpolar mediante los puntos $a,a+2$ es:
        \begin{equation*}
            E(x)=f[a,a+2,x]\Pi(x)\qquad \text{donde}\qquad \Pi(x) = (x-a)(x-(a+2))
            = x^2 - 2(1+a)x + a(a+2)
        \end{equation*}

        Por tanto, el error cometido al aproximar la integral por la fórmula simple es:
        \begin{equation*}
            R(f) = \int_{a}^{b} E(x) = \int_{a}^{a+3} f[a,a+2,x]\Pi(x)dx
        \end{equation*}

        Como $\Pi(x)$ cambia de signo en $[a,a+3]$, no podemos aplicar el Teorema del Valor Medio Integral Generalizado. Usamos que:
        \begin{multline*}
            f[a,a+2,a+2,x] = \dfrac{f[a,a+2,x]-f[a,a+2,a+2]}{x-(a+2)} \Longrightarrow\\ \Longrightarrow f[a,a+2,x] = f[a,a+2,a+2,x](x-(a+2)) + f[a,a+2,a+2]
        \end{multline*}

        Por tanto, tenemos que:
        \begin{align*}
            R(f) &= \int_{a}^{a+3}\left(f[a,a+2,a+2,x](x-(a+2)) + f[a,a+2,a+2]\right)\Pi(x)dx
            =\\&= \int_{a}^{a+3} f[a,a+2,a+2,x](x-(a+2))\Pi(x)dx + f[a,a+2,a+2]\int_{a}^{a+3} \Pi(x)dx
        \end{align*}

        Tenemos que calcular ahora ambas integrales. Por un lado, como es exacta en $\Pi(x)$, sabemos que:
        \begin{equation*}
            \int_{a}^{a+3} \Pi(x)dx = 0
        \end{equation*}

        Por otro lado, como $(x-(a+2))\Pi(x)$ no cambia de signo en $[a,a+3]$ y es un polinomio de grado $3$, podemos aplicar el Teorema del Valor Medio Integral Generalizado. Por tanto, existe $\mu\in [a,a+3]$ tal que:
        \begin{equation*}
            R(f) = f[a,a+2,a+2,\mu]\int_{a}^{a+2} (x-(a+2))\Pi(x)dx
        \end{equation*}

        Calculamos dicha integral:
        \begin{align*}
            \int_{a}^{a+2} &(x-(a+2))\Pi(x)dx = \int_{a}^{a+2} (x-a)(x-(a+2))^2dx
            =\\&= \int_{a}^{a+2} (x-a)((x-a)-2)^2dx
            = \int_{a}^{a+2} (x-a)((x-a)^2 - 4(x-a) + 4)dx
            =\\&= \int_{a}^{a+2} (x-a)^3 - 4(x-a)^2 + 4(x-a)dx
            = \left[\dfrac{(x-a)^4}{4} - \dfrac{4(x-a)^3}{3} + 4\dfrac{(x-a)^2}{2}\right]_{a}^{a+2}
            =\\&= \left[\dfrac{(3)^4}{4} - \dfrac{4(3)^3}{3} + 4\dfrac{(3)^2}{2}\right] = \dfrac{9}{4}
        \end{align*}
        
        Por tanto, el error cometido al aproximar la integral por la fórmula simple es:
        \begin{equation*}
            R(f) = f[a,a+2,a+2,\mu]\cdot \dfrac{9}{4}
        \end{equation*}

        Por las propiedades de las diferencias divididas, sabemos que $\exists \xi\in [a,a+3]$ tal que:
        \begin{equation*}
            R(f) = \dfrac{9f'''(\xi)}{4\cdot 3!} = \dfrac{3f'''(\xi)}{8}
        \end{equation*}

        Por tanto, el error cometido al aproximar la integral por la fórmula compuesta es:
        \begin{equation*}
            R(f) = \dfrac{3}{8}\left(f'''(\xi_1) + f'''(\xi_2)\right)
            = \dfrac{3}{4}f'''(\xi)
        \end{equation*}
        con $\xi\in [a,a+3]$.
    \end{enumerate}
\end{ejercicio}

\begin{ejercicio}~\label{ej:2.3.2}
    Determina razonadamente si es posible diseñar una fórmula numérica de tipo interpolatorio en el espacio generado por $\{1, x, x^2, x^4\}$ para aproximar
    \begin{equation*}
        \int_{-2}^{2} f(x) dx + \int_{-2}^{2} |x|f(x) dx
    \end{equation*}
    usando para ello los datos $\int_{-1}^{1} f(x) dx$, $\int_{-1}^{1} |x|f(x) dx$, $f(0)$ y $f'(0)$. En particular determina el peso de $f'(0)$.\\


    En primer lugar, buscamos el interpolante $p\in V=\cc{L}\{1,x,x^2,x^4\}$ de los datos. Sea por tanto:
    \begin{equation*}
        p(x)=a+bx+cx^2+dx^4\in V
    \end{equation*}

    Buscamos que $L_i(p)=L_i(f)$ para $i=0,1,2,3$, donde:
    \begin{align*}
        L_0(f) &= \int_{-1}^{1} f(x)dx,\\
        L_1(f) &= \int_{-1}^{1} |x|f(x)dx,\\
        L_2(f) &= f(0),\\
        L_3(f) &= f'(0).
    \end{align*}

    Tenemos que:
    \begin{align*}
        L_0(f) &= L_0(p) = \int_{-1}^{1} p(x)dx = \int_{-1}^{1} (a + bx + cx^2 + dx^4)dx = 2\left[a + \dfrac{c}{3} + \dfrac{d}{5}\right]
    \end{align*}
\end{ejercicio}

\begin{ejercicio}~\label{ej:2.3.3}
    Considera la fórmula de cuadratura simple del trapecio en la forma
    \begin{equation*}
        \int_{a}^{b} f(x) dx = T(a, b) + R(f).
    \end{equation*}
    \begin{enumerate}
        \item Obtén la expresión del error $R(f)$ para $f$ suficientemente regular.
        \item Obtén la fórmula compuesta asociada y la correspondiente expresión del error.
        \item Llama $h = b - a$. De forma similar a la vista en clase para la integración adaptativa con la fórmula de Simpson, obtén un criterio de estimación del error
        \begin{equation*}
            \int_{a}^{b} f(x) dx - T(a, m) - T(m, b)
        \end{equation*}
        basado en $T(a, b)$, $T(a, m)$ y $T(m, b)$, siendo $m = \frac{a+b}{2}$.
        \item Estima el error cometido en la aproximación en dos subintervalos
        \begin{equation*}
            \int_{4}^{8} \left(1 + \frac{e^{-x}}{x}\right)dx \approx T(4, 6) + T(6, 8) = 4.0054471
        \end{equation*}
        sabiendo que $f(4) = 1.0045789$, $f(6) = 1.0004131$ y $f(8) = 1.0000419$.
    \end{enumerate}
\end{ejercicio}

\begin{ejercicio}~\label{ej:2.3.4}
    Se pretende aproximar una integral del tipo
    \begin{equation*}
        \int_{-1}^{1} f(x)(1 - x^2)dx
    \end{equation*}
    utilizando tres nodos, es decir:
    \begin{equation*}
        \int_{-1}^{1} f(x)(1 - x^2)dx \approx \alpha_0 f(x_0) + \alpha_1 f(x_1) + \alpha_2 f(x_2)
    \end{equation*}
    \begin{enumerate}
        \item\label{ap:1} Si fijamos los nodos $x_0 = -1$, $x_1 = 0$ y $x_2 = 1$, determina el valor de los parámetros para que sea una fórmula de tipo interpolatorio, así como el orden de exactitud de dicha fórmula.
        \item ¿Cuáles serían los nodos si utilizamos una fórmula de Newton-Cotes abierta?
        \item\label{ap:3} Determina la fórmula gaussiana correspondiente así como la expresión del error.
        \item Utiliza las fórmulas de los apartados \ref{ap:1} y \ref{ap:3} para aproximar
        \begin{equation*}
            \int_{-1}^{1} \cos(x^2)(1 - x^2)dx
        \end{equation*}
    \end{enumerate}
\end{ejercicio}

\begin{ejercicio}~\label{ej:2.3.5}
    Se considera la fórmula de integración numérica
    \begin{equation*}
        \int_{-1}^{2} f(x)x dx \sim a_0 f(0) + a_1 f(2) + a_2 f'(0) + a_3f'(2).
    \end{equation*}
    \begin{enumerate}
        \item Determina los coeficientes $a_0$, $a_1$, $a_2$ y $a_3$ para que la fórmula anterior sea de tipo interpolatorio.
        \item Indica el grado de exactitud de la fórmula anterior. ¿Es el grado de exactitud superior al esperado?
        \item Si se pretende utilizar una fórmula gaussiana con 2 nodos para aproximar la integral, determina cuáles serían dichos nodos y la expresión del error cometido en la aproximación.
    \end{enumerate}
\end{ejercicio}

\begin{ejercicio}~\label{ej:2.3.6}
    Se pretende aproximar la integral
    \begin{equation*}
        \int_{a}^{b} f(x)dx = S_n(f) + R(f) \tag{1}
    \end{equation*}
    donde $S_n(f)$ es una fórmula de integración compuesta obtenida al hacer una partición uniforme del intervalo $[a, b]$ de la forma:
    \begin{equation*}
        a = x_0 < x_1 < \ldots < x_n = b, \quad x_i = x_{i-1} + h, \quad h = \frac{b - a}{n}
    \end{equation*}
    y $R(f)$ es el error de integración numérica que tiene el siguiente desarrollo:
    \begin{equation*}
        R(f) = a_1h^3 + a_2h^6 + \cdots + a_mh^{3m} + \cdots
    \end{equation*}
    Siguiendo el mismo argumento de la integración de Romberg, combina $S_n(f)$ con $S_{3n}(f)$ para obtener una aproximación más precisa para la integral. Aplica recursivamente el método.
\end{ejercicio}

\begin{ejercicio}~\label{ej:2.3.7}
    A veces, para construir fórmulas de integración numérica es posible utilizar nodos que se encuentran fuera del intervalo de integración. Considera la fórmula:
    \begin{equation*}
        \int_{a}^{a+h} f(x)dx = \frac{5h}{12}f(a + h) + \frac{2h}{3}f(a) - \frac{h}{12}f(a - h) + R(f)
    \end{equation*}
    \begin{enumerate}
        \item Demuestra que es de tipo interpolatorio y determina el grado de exactitud.
        \item Proporciona una expresión para el error de integración numérica asociado a la fórmula.
        \item Deduce la fórmula compuesta asociada a dicha fórmula.
    \end{enumerate}
\end{ejercicio}

\begin{ejercicio}~\label{ej:2.3.8}
    Se considera la fórmula de integración numérica
    \begin{equation*}
        \int_{-1}^{1} f(x)(1 - x^2)dx = \alpha_0 f(x_0) + \alpha_1 f(x_1) + R(f).
    \end{equation*}
    \begin{enumerate}
        \item Determina los nodos y los coeficientes para que la fórmula anterior tenga grado de exactitud máximo. ¿Cuál es ese grado de exactitud?
        \item Obtén la expresión del error de dicha fórmula.
        \item Utiliza la fórmula anterior para estimar el valor de
        \begin{equation*}
            \int_{-1}^{1} \ln(x^2 + 1)(1 - x^2)dx .
        \end{equation*}
    \end{enumerate}
\end{ejercicio}

\begin{ejercicio}~\label{ej:2.3.9}
    Considera la fórmula de cuadratura de tipo interpolatorio:
    \begin{equation*}
        \int_{a}^{a+h} f(x)dx = \frac{3h}{4}f(a) + \frac{h}{4}f(a + 2h) + R(f).
    \end{equation*}
    \begin{enumerate}
        \item Proporciona una expresión para el error de integración numérica asociado a la fórmula.
        \item Deduce la fórmula compuesta asociada a dicha fórmula incluyendo una expresión del error.
        \item Deduce un método multipaso lineal para aproximar la solución del PVI
        \begin{equation*}
            \begin{cases}
                x' = f(f, x) \\
                x(t_0) = \mu
            \end{cases}
        \end{equation*}
    \end{enumerate}
\end{ejercicio}

\begin{ejercicio}
    Considerar la fórmula numérica siguiente:
    \begin{equation*}
        \int_{-1}^1f(x)(1-x^2)\ dx=\alpha_0f(-1)+\alpha_1f(0)+\alpha_2f(1)+R(f)
    \end{equation*}
    \begin{enumerate}
        \item Hallar los valores de $\alpha_0$, $\alpha_1$ y $\alpha_2$.
        
        Calculamos cada uno de los polinomios básicos de Lagrange:
        \begin{align*}
            \ell_0(x)
            &=\prod_{\substack{j=0\\j\neq 0}}^2\frac{x-x_j}{x_0-x_j}
            = \frac{x-x_1}{x_0-x_1}\cdot\frac{x-x_2}{x_0-x_2}
            = \dfrac{x(x-1)}{(-1)(-2)} = \dfrac{x(x-1)}{2}\\
            \ell_1(x)
            &= \prod_{\substack{j=0\\j\neq 1}}^2\frac{x-x_j}{x_1-x_j}
            = \frac{x-x_0}{x_1-x_0}\cdot\frac{x-x_2}{x_1-x_2}
            = \dfrac{(x+1)(x-1)}{1(-1)} = -(x+1)(x-1)\\
            \ell_2(x)
            &= \prod_{\substack{j=0\\j\neq 2}}^2\frac{x-x_j}{x_2-x_j}
            = \frac{x-x_0}{x_2-x_0}\cdot\frac{x-x_1}{x_2-x_1}
            = \dfrac{x(x+1)}{(2)1} = \dfrac{x(x+1)}{2}
        \end{align*}

        Multiplicamos ahora cada uno de los polinomios por la función peso:
        \begin{align*}
            \ell_0(x)(1-x^2) &= -\ell_0(x)(x+1)(x-1) = -\dfrac{x(x-1)^2(x+1)}{2} = -\frac{1}{2}(x^4-x^3-x^2+x)\\
            \ell_1(x)(1-x^2) &= -\ell_1(x)(x+1)(x-1) = (x+1)^2(x-1)^2 = x^4-2x^2+1\\
            \ell_2(x)(1-x^2) &= -\ell_2(x)(x+1)(x-1) = -\dfrac{x(x+1)^2(x-1)}{2} = -\frac{1}{2}(x^4+x^3-x^2-x)
        \end{align*}

        Calculamos ahora los valores de $\alpha_0$, $\alpha_1$ y $\alpha_2$:
        \begin{align*}
            \alpha_0 &= \int_{-1}^1\ell_0(x)(1-x^2)\ dx = -\frac{1}{2}\int_{-1}^1(x^4-x^3-x^2+x)\ dx = -\left(\dfrac{1}{5}-\dfrac{1}{3}\right) = \dfrac{2}{15}\\
            \alpha_1 &= \int_{-1}^1\ell_1(x)(1-x^2)\ dx = \int_{-1}^1(x^4-2x^2+1)\ dx = 2\left(\dfrac{1}{5}-\dfrac{2}{3}+1\right) = \dfrac{16}{15}\\
            \alpha_2 &= \int_{-1}^1\ell_2(x)(1-x^2)\ dx = -\frac{1}{2}\int_{-1}^1(x^4+x^3-x^2-x)\ dx = \dfrac{2}{15}
        \end{align*}

        Por tanto, tenemos que:
        \begin{align*}
            \int_{-1}^1f(x)(1-x^2)\ dx &\approx \int_{-1}^1\left[\ell_0(x)f(-1)+\ell_1(x)f(0)+\ell_2(x)f(1)\right](1-x^2)\ dx\\
            &= \left(\int_{-1}^1\ell_0(x)(1-x^2)\ dx\right)f(-1)+\left(\int_{-1}^1\ell_1(x)(1-x^2)\ dx\right)f(0)+\\&\hspace{1.5cm}+\left(\int_{-1}^1\ell_2(x)(1-x^2)\ dx\right)f(1)\\
            &= \dfrac{2}{15}f(-1)+\dfrac{16}{15}f(0)+\dfrac{2}{15}f(1)
        \end{align*}
        \item Hallar una expresión del error $R(f)$.
        
        Tenemos que la expresión del error cometido al aproximar $f(x)$ por el polinomio de interpolación de Lagrange de grado 2 es:
        \begin{equation*}
            E(x) = f[-1,0,1,x]\Pi(x)\qquad \text{donde}\qquad \Pi(x) = \prod_{j=0}^2(x-x_j)=x(x-1)(x+1)
        \end{equation*}

        Por tanto:
        \begin{equation*}
            R(f)=L(E)=\int_{-1}^1E(x)(1-x^2)\ dx = \int_{-1}^1f[-1,0,1,x]\Pi(x)(1-x^2)\ dx
        \end{equation*}

        Sabemos que $\Pi(x)$ cambia de signo en $x=0$. Para evitar esto, hacemos uso de que:
        \begin{multline*}
            f[-1, 0, 0, 1, x] = \dfrac{f[-1, 0, 1, x]-f[-1, 0, 0, 1]}{x-0}
            \Longrightarrow\\\Longrightarrow f[-1, 0, 1, x] = f[-1, 0, 0, 1, x]x + f[-1, 0, 0, 1]
        \end{multline*}

        Por tanto, tenemos que:
        \begin{align*}
            R(f) &= \int_{-1}^1\left(f[-1, 0, 0, 1, x]x + f[-1, 0, 0, 1]\right)x(x-1)(x+1)(1-x^2)\ dx
            =\\&= -\int_{-1}^1\left(f[-1, 0, 0, 1, x]x^2+f[-1, 0, 0, 1]x\right)(x-1)^2(x+1)^2\ dx
            =\\&= -\int_{-1}^1f[-1, 0, 0, 1, x]x^2(x-1)^2(x+1)^2\ dx - f[-1, 0, 0, 1]\int_{-1}^1x(x-1)^2(x+1)^2\ dx
        \end{align*}

        Por el Teorema del Valor Medio Integral Generalizado, $\exists \mu\in [-1,1]$ tal que:
        \begin{align*}
            R(f) &= -f[-1, 0, 0, 1, \mu]\int_{-1}^1x^2(x-1)^2(x+1)^2\ dx - f[-1, 0, 0, 1]\int_{-1}^1x(x-1)^2(x+1)^2\ dx
            =\\&= -f[-1, 0, 0, 1, \mu]\int_{-1}^1(x^6-2x^4+x^2)\ dx - f[-1, 0, 0, 1]\int_{-1}^1(x^5-2x^3+x)\ dx
            =\\&= -2f[-1, 0, 0, 1, \mu]\left(\dfrac{1}{7}-\dfrac{2}{5}+\dfrac{1}{3}\right) - f[-1, 0, 0, 1]\cdot 0
            \\&= -\dfrac{16}{105}\cdot f[-1, 0, 0, 1, \mu]
        \end{align*}

        Suponiendo $f\in C^4[-1,1]$, $\exists \xi\in [-1,1]$ tal que:
        \begin{equation*}
            R(f) = -\dfrac{16}{105}\cdot \dfrac{f^{(4)}(\xi)}{4!}
            = -\dfrac{2}{315}\cdot f^{(4)}(\xi)
        \end{equation*}
    \end{enumerate}
\end{ejercicio}


\begin{ejercicio}
    Se pretende aproximar mediante integración de Romberg la integral:
    \begin{equation*}
        \int_1^3 \frac{1}{x} \, dx.
    \end{equation*}
    Calcula para ello \( R(2, 2) \).\\

    Definimos la siguiente función auxiliar:
    \Func{f}{[1,3]}{\bb{R}}{x}{\nicefrac{1}{x}}

    Buscamos construir la siguiente tabla:
    \begin{equation*}
        \begin{array}{ccc}
            R(0, 0) \\
            R(1, 0) & R(1, 1) \\
            R(2, 0) & R(2, 1) & R(2, 2)
        \end{array}
    \end{equation*}

    Calculamos en primer lugar \( R(i, 0) \) para cada $i\in \{0,1,2\}$:
    \begin{align*}
        R(0, 0) &= T_1 = (3-1)\cdot \dfrac{f(3)-f(1)}{2}
        = \dfrac{4}{3}\\
        R(1, 0) &= T_2 = \dfrac{3-1}{2\cdot 2}\left(f(1)+2\cdot f\left(1+\dfrac{3-1}{2}\right)+f(3)\right)
        = \dfrac{7}{6}\\
        R(2, 0) &= T_{4} =\\&= \dfrac{3-1}{4\cdot 2}\left(f(1)+2\left[f\left(1+\dfrac{3-1}{4}\right)+f\left(1+2\cdot \dfrac{3-1}{4}\right)+f\left(1+3\cdot \dfrac{3-1}{4}\right)\right]+f(3)\right)
        =\\&= \dfrac{67}{60}
    \end{align*}

    Una vez obtenidos esos valores, calculamos \( R(i, j) \) para \( i,j=1,2 \), con $ j < i $:
    \begin{align*}
        R(1, 1) &= \dfrac{4R(1,0)-R(0,0)}{3} = \dfrac{10}{9}\\
        R(2, 1) &= \dfrac{4R(2,0)-R(1,0)}{3} = \dfrac{11}{10}\\
        R(2, 2) &= \dfrac{4^2R(2,1)-R(1,1)}{15} = \dfrac{742}{675}\approx 1.09925925
    \end{align*}
\end{ejercicio}

\begin{ejercicio}
    Dada la regla de integración numérica
    \begin{equation*}
        \int_a^b f(x) \, dx = L_n(f, h) + c_1 h + c_2 h^2 + c_3 h^3 + \ldots
    \end{equation*}
    ¿Cómo se haría un procedimiento similar a la integración de Romberg con esta fórmula?\\


    Tomando $\nicefrac{h}{2}$, llegamos a que:
    \begin{align*}
        \int_a^b f(x) \, dx &= L_n\left(f,\nicefrac{h}{2}\right) + c_1 \cdot \frac{h}{2} + c_2 \cdot \left(\frac{h}{2}\right)^2 + c_3 \cdot \left(\frac{h}{2}\right)^3 + \ldots\\
    \end{align*}

    Buscamos ahora eliminar el término del error de orden $h$. Multiplicando por $2$ esta expresión y restándole la original, obtenemos:
    \begin{align*}
        \hspace{-1cm}\int_a^b f(x) \, dx &= 2L_n\left(f,\nicefrac{h}{2}\right) - L_n(f, h) + c_1 \cdot \left(h-h\right) + c_2 \cdot \left(2\left(\frac{h}{2}\right)^2 - h^2\right) + c_3 \cdot \left(2\left(\frac{h}{2}\right)^3 - h^3\right) + \ldots\\
        \hspace{-1cm}&= 2L_n\left(f,\nicefrac{h}{2}\right) - L_n(f, h) - c_2 \cdot \left(\frac{h^2}{2}\right) - c_3 \cdot \left(\frac{3h^3}{4}\right) + \ldots
    \end{align*}

    Hemos conseguido eliminar el término de orden $h$, pero nos quedan los términos de orden $h^2$ y $h^3$. De forma similar a la integración de Romberg, podemos definir los siguientes términos:
    \begin{align*}
        L(i,0) &= L_n(f, \nicefrac{h}{2^i}),\qquad i=0,1,\ldots\\
        L(i,j) &= \dfrac{2^jL(i,j-1)-L(i-1,j-1)}{2^j-1}, \quad i,j\in \{0,1,\ldots\}, \quad j\leq i\\
    \end{align*}

    De esta forma, podemos aproximar la integral de la siguiente forma:
    \begin{align*}
        \int_a^b f(x) \, dx &\approx L(N,N)
    \end{align*}
    donde $N$ es el número de pasos que hemos dado.
\end{ejercicio}
\chapter{Transformaciones en el espacio euclídeo}

Notaremos $\bb{E}^2$ como el espacio afín; es decir, el plano.

\begin{definicion}[Movimiento]
    Definimos un movimiento en $\bb{E}^2$ como una aplicación $f:\bb{E}^2\to \bb{R}^2$ biyectiva tal que:
    \begin{equation*}
        d(P,Q) = d(f(P), f(Q)) \qquad \forall P,Q\in \bb{E}^2
    \end{equation*}
\end{definicion}
\begin{ejemplo} Algunos ejemplos de movimientos en el plano son:
\begin{enumerate}
    \item $f=1_{\bb{E}^2}$ la aplicación identidad.
    \item $f=\sigma_r$ reflexiones sobre la recta $r$. Cabe notar que $$\sigma_r\circ \sigma_r=1_{\bb{E}^2}$$
\end{enumerate}

\begin{prop}
    Si $f$ es una aplicación que deja a tres puntos no alineados invariantes, entonces $f=1_{\bb{E}^2}$.
\end{prop}
\begin{proof}
    Demostramos por reducción al absurdo. Si $f\neq 1_{\bb{E}^2}$, por lo que $\exists Q\mid f(Q)\neq Q$. Entonces, si $A$ es un punto invariante, como $d(A,Q)=d(A,f(Q))$, entonces $A$ está en la mediatriz del segmento $\ol{Q~f(Q)}$. De forma análoga, $B$ y $C$ están en la mediatriz también, por lo que están alineados. Llegamos a una contradicción, por lo que $f=1_{\bb{E}^2}$.
\end{proof}

\begin{prop}
    Si $f\neq 1_{\bb{E}^2}$ es una aplicación que deja a dos puntos invariantes, entonces $f=\sigma_r$.
\end{prop}

\begin{prop}
    Si $f\neq 1_{\bb{E}^2}$ es una aplicación que deja solamente a un punto invariante, entonces $f=\rho_{A,\alpha}$; es decir, un giro de centro $A$ y ángulo $\alpha$.
\end{prop}


\begin{teo} Para todo $f$ movimiento en $\bb{E}^2$, $\exists~m,r,s$ rectas a lo sumo tal que $$f=\sigma_m \circ \sigma_r \circ \sigma_s$$
    
\end{teo}
    
\end{ejemplo}
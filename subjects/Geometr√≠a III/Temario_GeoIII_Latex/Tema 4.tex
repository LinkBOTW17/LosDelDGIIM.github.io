\chapter{Espacio Proyectivo}

Sea $V^{n+1}(\bb{R})$ un espacio vectorial, y consideramos $V^\ast = V\setminus \{0\}$.

En $V^\ast$ definimos la siguiente relación de equivalencia:
\begin{equation*}
    v_1 \sim v_2 \Longleftrightarrow v_2=\lm v_1 \qquad \lm \in \bb{R}^\ast
\end{equation*}

Definimos el espacio proyectivo de dimensión $n$ como el espacio vectorial cociente mediante dicha relación de equivalencia $\sim$:
\begin{equation*}
    V^\ast/\sim = \{[v]\mid v\in V^\ast\}
\end{equation*}

Notamos dicho espacio cociente como $P(V)$. En el caso particular de $V=\bb{R}^{n+1}$, tenemos que se denota de la siguiente forma:
\begin{equation*}
    (\bb{R}^{n+1}\setminus \{0\})/\sim ~=~ P(\bb{R}^{n+1})=\bb{P}^n
\end{equation*}


Idea también de la Esfera. Pedirla o mirar de topo


\section{Subespacios Proyectivos}

Consideramos la proyección según dicha clase de equivalencia:
\Func{\pi}{V^\ast}{P(V)}{v}{[v]}

\begin{definicion}
    Dado el espacio proyectivo $P(V)$, y un subconjunto $X\subset P(V)$, diremos que $X$ es un subespacio proyectivo si $\pi^{-1}(X)\cup \{0\}=\wt{X}$ es un subespacio vectorial.

    Además, $\dim X = \dim \wt{X}-1$.

    Por convenio, diremos que $\emptyset$ es un subespacio proyectivo de dimensión $-1$.
\end{definicion}

\begin{prop}
    Sea $V$ un espacio vectorial, y $U\subset V$ un subespacio vectorial. Entonces:
    \begin{equation*}
        U = \pi^{-1}(\pi(U))\cup \{0\}
    \end{equation*}
\end{prop}


\subsubsection{Operaciones entre espacios proyectivos}

Sea $P(V^{n+1})$ un espacio proyectivo, y sean $X,Y$ dos subespacios proyectivos de forma que $X=\pi(\wt{X}-\{\emptyset\})$, $Y=\pi(\wt{Y}-\{\emptyset\})$.

Tenemos que:
\begin{equation*}
    X\cap Y = \pi(\wt{X}-\{\emptyset\}) \cap \pi(\wt{Y}-\{\emptyset\}) = \pi(\wt{X}\cap \wt{Y}\setminus \{0\})
\end{equation*}

\begin{definicion}
    Sea $X\subset P(V^{n+1})$. Llamamos subespacio proyectivo generado por $C$, notado por $\langle C \rangle$, al menor subespacio proyectivo que lo contenga. Es decir,
    \begin{equation*}
        \bigcap \{X\supset C\mid X\text{ es un subespacio proyectivo}\}
    \end{equation*}
\end{definicion}

A partir de esa definición, tenemos que:
\begin{equation*}
    X+Y = \langle X\cup Y\rangle = \pi(\wt{X}+\wt{Y})
\end{equation*}


\begin{prop}
    Tenemos que:
    \begin{equation*}
        \dim (X+Y) = \dim X + \dim Y - \dim(X\cap Y)
    \end{equation*}
\end{prop}


\begin{prop}
    En un plano proyectivo, dos rectas paralelas siempre se intersecan.
\end{prop}

Algunas propiedades que tenemos son:
\begin{enumerate}
    \item Si $X\subset Y$ son subespacios proyectivos, entonces:
    \begin{equation*}
        \dim X \leq \dim Y
    \end{equation*}
    Además, se da la igualdad si y solo si $X=Y$.
\end{enumerate}


\section{Coordenadas Homogéneas}

Sea $\cc{B}$ una base de $V^{n+1}$, y consideramos $v \in V^{n+1}\setminus \{0\}$. Sea $v=(x_0,\dots,x_n)_{\cc{B}}$.

Definimos las coordenadas homogéneas de $[v]\in P(V^{n+1})$ como las coordenadas de dicho $v$:
\begin{equation*}
    [v] \equiv (x_0:\dots :x_n)
\end{equation*}

Notemos que, como $[v]=[\lm v]$, tenemos que dichas coordenada son únicas \ul{salvo factores de proporcionalidad}.

Las ecuaciones paramétricas o implícitas de $X$ son las de $\wt{X}$.

\begin{ejemplo}
    Sea $p=(1:2:3), q=(0:1:2)\in \bb{P}^2$. Calcular la recta proyectiva que une ambos planos.

    Tenemos que $p=[(1,2,3)]$, $q=[(0,1,2)]$. Entonces:
    \begin{equation*}
        p+q = \pi(\cc{L}\{(1,2,3), (0,1,2)\})
    \end{equation*}

    Por tanto, tenemos que:
    \begin{equation*}
        (x:y:z)\in p+q \Longleftrightarrow (x,y,z)=\alpha(1,2,3) + \beta(0,1,2)
    \end{equation*}

    La ecuación implícita es:
    \begin{equation*}
        \left|\begin{array}{ccc}
            1 & 0 & x \\
            2 & 1 & y\\
            3 & 2 & z
        \end{array}\right|
    \end{equation*}
\end{ejemplo}


\section{Proyectividades}

\begin{figure}[H]
    \centering
    \shorthandoff{"}
    \begin{tikzcd}
        V_1 \arrow[r, "\wt{f}"] \arrow[d] & V_2 \arrow[d] \\
        P(V_1) \arrow[r, "f"]             & P(V_2)       
    \end{tikzcd}
    \shorthandon{"}
\end{figure}

\begin{definicion}[Proyectividad]
    Diremos que una aplicación $f:P(V_1)\to P(V_2)$ es una proyectividad si y solo si existe $\wt{f}:V_1\to V_2$ lineal e inyectiva tal que
    \begin{equation*}
        f([v])=[\wt{f}(v)]
    \end{equation*}

    Diremos que $\wt{f}$ es la lineal asociada a $f$.
\end{definicion}

Notemos que se pide que sea inyectiva para poder proyectar.


Notemos que esta es única salvo proporcionalidad:
\begin{prop}
    $\wt{f}$ y $\wt{g}$ son lineales asociadas a $f$ si y solo si $\wt{f}=\lm \wt{g}$.
\end{prop}

\begin{prop}
    Las proyectividades conservan la dimensión. Es decir, si $f$ es una proyectividad y $X$ es un subespacio proyectivo, entonces $f(X)$ es un subespacio proyectivo. Además,
    \begin{equation*}
        \dim f(X)=\dim X
    \end{equation*}
\end{prop}
\begin{proof}
    Tenemos que $f(X)=[\wt{f}(\wt{X})]$
\end{proof}



Determinar $f$ es equivalente a determinar $\wt{f}$.


TERMINAR
\section{Álgebra Relacional}

\begin{ejercicio}\label{ej:AR_1}
    Considere el esquema de base de datos relacional de la Figura~\ref{fig:ejAR_1}.
    \begin{figure}[H]
        \centering
        \resizebox{1.1\textwidth}{!}{
            \begin{tikzpicture}[node distance=15em]
                \node(modelo) {Modelo(id\_modelo, marca, descripcion)};
                \node(vehiculo)[right of=modelo, xshift=7em] {Vehículo(matrícula, id\_modelo, fecha\_matriculación)};
                \node(mecanico)[below of=modelo, yshift=10em] {Mecánico(id\_mecánico, nombre\_mecánico, cargo, salario)};
                \node(repara)[right of=mecanico, xshift=10em] {Repara(id\_mecánico, matrícula, fecha, número\_horas)};

                \node[CP, xshift=-6ex] (CP_modelo) at(modelo) {id\_modelo};
                \node[CP, xshift=-12ex] (CP_vehiculo) at(vehiculo) {matrícula};
                \node[CP, xshift=-11.5ex] (CP_mecanico) at(mecanico) {id\_mecánico};
                \node[CP, xshift=-3.5ex] (CP_repara) at(repara) {id\_mecánico, matrícula, fecha};

                \node[CE, xshift=11ex] (CE_vehiculo) at(CP_vehiculo) {id\_modelo};
                \node[CE, xshift=-10ex] (CE_repara_mecanico) at(CP_repara) {id\_mecánico};
                \node[CE, xshift=3ex] (CE_repara_vehiculo) at(CP_repara) {matrícula};

                \node[yshift=1.4em, xshift=1.4em, purple] at (CE_vehiculo) {(1)};
                \node[yshift=1.4em, xshift=1.4em, purple] at (CE_repara_mecanico) {(2)};
                \node[yshift=1.4em, xshift=1.4em, purple] at (CE_repara_vehiculo) {(3)};

                \node[yshift=-1.4em, xshift=1.4em, purple] at (CP_modelo) {(1)};
                \node[yshift=-1.4em, xshift=1.4em, purple] at (CP_mecanico) {(2)};
                \node[yshift=-1.4em, xshift=1.4em, purple] at (CP_vehiculo) {(3)};

            \end{tikzpicture}
        }
        \caption{Esquema de base de datos relacional del Ejercicio~\ref{ej:AR_1}.}
        \label{fig:ejAR_1}
    \end{figure}

    Cada vehículo tiene asignado un modelo de una marca determinada. La tabla \emph{Repara} registra reparaciones indicando qué mecánico repara qué vehículo en qué fecha y cuántas horas dura la reparación.\\

    Indiqué qué consulta debe realizar en álgebra relacional para obtener la información solicitada en cada uno de los siguientes puntos:
    \begin{enumerate}
        \item Reparaciones de más de 20 horas.
        \item Códigos de mecánicos que han reparado el vehículo de matrícula 1234ABC.
        \item Parejas de mecánicos que se pueden hacer en la empresa.
        \item Marca de los vehículos matriculados después del $1/1/20$.
        \item Parejas $\langle \text{cargo}, \text{marca} \rangle$ entre las que se ha dado alguna reparación.
        \item Vehículos que o tienen una fecha de matriculación posterior al $1/1/22$ o han sido reparados con posterioridad a esa misma fecha.
        \item Vehículos con fecha de matriculación posterior al $1/1/22$ que han sido reparados alguna vez.
        \item Marca de los vehículos que no han tenido ninguna reparación en el año $2022$.
        \item Código de los mecánicos que han reparado vehículos de, al menos, dos marcas distintas.
        \item Vehículos que tienen una sola reparación.
        \item Vehículos que han sufrido las reparaciones con la duración más alta.
        \item Mecánicos que tienen el salario más bajo.
        \item Mecánicos cuyo salario es uno de los dos salarios más bajos.
        \item Vehículos que han sido reparados alguna vez por cada uno de los mecánicos.
        \item Mecánicos que han reparado vehículos de todas las marcas.
        \item Vehículos a los que el mecánico de id $123$ les ha hecho todas las reparaciones.
        \item Marcas para las que todos sus vehículos han sido reparados alguna vez por un empleado con un salario superior a $30000$.
        \item Vehículos que, para todos los cargos que hay en la empresa, han tenido al menos una reparación de más de $2$ horas de duración con un empleado de ese cargo.
        \item Marcas para las que todos sus vehículos han sido reparados alguna vez por el mismo mecánico.
        \item Mecánico más joven que ha reparado vehículos de todas las marcas.
    \end{enumerate}

\end{ejercicio}




\newpage
\begin{ejercicio} \label{ej:AR_2}
    Considere el esquema de base de datos relacional de la Figura~\ref{fig:ejAR_2}.
    \begin{figure}[H]
        \centering
        \resizebox{1.1\textwidth}{!}{
            \begin{tikzpicture}[node distance=15em]
                \node(proveedor) {Proveedor(codpro, nompro, status, ciudad)};
                \node(pieza)[right of=proveedor, xshift=7em] {Pieza(codpie, nompie, color, peso, ciudad)};
                \node(proyecto)[below of=proveedor, yshift=10em] {Proyecto(codpj, nompj, ciudad)};
                \node(ventas)[right of=proyecto, xshift=8em] {Ventas(codpro, codpie, codpj, cantidad)};

                \node[CP, xshift=-7ex] (CP_proveedor) at(proveedor) {codpro};
                \node[CP, xshift=-11ex] (CP_pieza) at(pieza) {codpie};
                \node[CP, xshift=-3ex] (CP_proyecto) at(proyecto) {codpj};
                \node[CP, xshift=-1ex] (CP_ventas) at(ventas) {codpro, codpie, codpj};

                \node[CE, xshift=-8.5ex] (CE_ventas_proveedor) at(CP_ventas) {codpro};
                \node[CE, xshift=0ex] (CE_ventas_pieza) at(CP_ventas) {codpie};
                \node[CE, xshift=8ex] (CE_ventas_proyecto) at(CP_ventas) {codpj};

                \node[yshift=1.4em, xshift=1.4em, purple] at (CE_ventas_proveedor) {(1)};
                \node[yshift=1.4em, xshift=1.4em, purple] at (CE_ventas_pieza) {(2)};
                \node[yshift=1.4em, xshift=1.4em, purple] at (CE_ventas_proyecto) {(3)};

                \node[yshift=-1.5em, xshift=1.4em, purple] at (CP_proveedor) {(1)};
                \node[yshift=-1.5em, xshift=1.4em, purple] at (CP_pieza) {(2)};
                \node[yshift=-1.5em, xshift=1.4em, purple] at (CP_proyecto) {(3)};
            \end{tikzpicture}
        }
        \caption{Esquema de base de datos relacional del Ejercicio~\ref{ej:AR_2}.}
        \label{fig:ejAR_2}
    \end{figure}

    Indique qué consulta debe realizar en álgebra relacional para obtener la información solicitada en cada uno de los siguientes puntos:
    \begin{enumerate}
        \item Encontrar todas las parejas de ciudades tales que la primera sea la de un proveedor y la segunda la de un proyecto entre los cuales haya algún suministro.
        \item Encontrar los códigos de las piezas suministradas a algún proyecto por un proveedor que se encuentre en la misma ciudad que el proyecto.
        \item Encontrar los códigos de los proyectos que tienen al menos un proveedor que no se encuentre en su misma ciudad.
        \item Mostrar todas las ciudades de donde proceden piezas y las ciudades donde hay proyectos.
        \item Mostrar todas las ciudades de los proveedores en las que no se fabriquen piezas.
        \item Mostrar todas las ciudades de los proveedores en las que además se fabriquen piezas.
        \item Encontrar los códigos de los proyectos que usan una pieza que vende S1.
        \item Encontrar la cantidad más pequeña enviada en algún suministro.
        \item Encontrar los códigos de los proyectos que no utilizan una pieza roja suministrada por un proveedor de Londres.
        \item Encontrar los códigos de los proyectos que tienen como único proveedor a S1.
        \item Encontrar los códigos de las piezas que se suministran a todos los proyectos de París.
        \item Encontrar los códigos de los proveedores que venden la misma pieza a todos los proyectos.
        \item Encontrar los códigos de los proyectos a los que el proveedor S1 suministra todas las piezas existentes.
        \item Mostrar los códigos de los proveedores que suministran todas las piezas a todos los proyectos.
        \item Pieza con más peso entre las que pesan menos de 100.
        \item Entre los proyectos de Jaén, mostrar el que ha suministrado la pieza de mayor peso (puede haber más de uno).
        \item Proyectos para los que la lista de piezas que han suministrado tiene al menos dos piezas distintas.
        \item Proyectos para los que la lista de piezas que han suministrado tiene exactamente dos piezas distintas.
        \item Proveedores que han hecho una o dos ventas (y no más).
        \item Proveedores en los que todos sus suministros son de una pieza roja o de una pieza de Granada.
    \end{enumerate}
    
\end{ejercicio}
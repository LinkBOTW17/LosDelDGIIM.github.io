\section{Diagrama Entidad-Relación}

\begin{ejercicio} \label{ej:1}
    Queremos crear la BD para una biblioteca:
    \begin{itemize}
        \item Los libros se caracterizan por su ISBN, título y año de escritura.
        \item Los autores tienen código, nombre y nacionalidad.
        \item No existe más que un ejemplar de cada libro.
        \item Cada libro puede estar escrito por más de un autor.
        \item Un autor puede escribir más de un libro.
        \item Cada libro puede tratar más de un tema.
        \item Hay muchos libros de cada tema.
        \item Los usuarios de la biblioteca están caracterizados por su DNI, su nombre y su dirección.
        \item Queremos poder representar información relativa a los préstamos. Para ello registramos cuándo un usuario toma prestado un libro y eliminamos dicho registro cuando el usuario lo devuelve. Registramos también la fecha del préstamo.
        \item Cada usuario no puede tener prestado más de un libro simultáneamente.
    \end{itemize}
    
    El Diagrama Entidad-Relación correspondiente se encuentra en la Figura~\ref{fig:ej1}.
    Notemos que ambas participaciones obligatorias no están explícitamente en el enunciado,
    aunque se pueden inferir del contexto en el que trabajamos. Esto se tendría que especificar con el cliente.
    También es importante notar que, como la entrada del préstamo se borra, la fecha no es discriminante. 
    \begin{figure}
        \centering
        \begin{tikzpicture}[node distance=6.3 em]
            \node[entidad] (libro) {Libro};
            \node[atributo] (ISBN)[above left of=libro] {\key{ISBN}} edge(libro);
            \node[atributo] (titulo)[above right of=libro] {Título} edge(libro);
            \node[atributo] (anio) [above of=libro] {Año de Escritura} edge(libro);

            \node[relacion] (autoria)[left of=libro] {Autoría} edge[participacion obligatoria](libro);
            \node[entidad] (autor)[left of=autoria] {Autor} edge(autoria);
            \node[atributo] (DNI_aut) [left of=autor] {\key{DNI}} edge(autor);
            \node[atributo] (nombre_aut) [above left of=autor] {Nombre} edge(autor);
            \node[atributo] (nacionalidad) [above of=autor] {Nacionalidad} edge(autor);

            \node[relacion] (trata) [right of=libro] {Trata sobre} edge[participacion obligatoria](libro);
            \node[entidad] (tema)[right of=trata] {Tema} edge(trata);
            \node[atributo] (nombre_tema) [above of=tema] {\ul{Nombre}} edge(tema);

            
            \node[relacion] (prestamo) [below of=libro] {Préstamo} edge[-Stealth](libro);
            \node[entidad] (usuario) [below of=prestamo] {Usuario} edge[Stealth-](prestamo);
            \node[atributo] (DNI_user) [right of=usuario] {\key{DNI}} edge(usuario);
            \node[atributo] (nombre_user) [left of=usuario] {Nombre} edge(usuario);
            \node[atributo] (dir_user) [below of=usuario] {Dirección} edge(usuario);

            \node[atributo] (fecha) [left of=prestamo] {Fecha} edge(prestamo);
        \end{tikzpicture}
        \caption{Diagrama Entidad-Relación del Ejercicio \ref{ej:1}}
        \label{fig:ej1}
    \end{figure}
\end{ejercicio}

\begin{ejercicio} \label{ej:2}
    Considere el Ejercicio~\ref{ej:1}, con las siguientes modificaciones:
    \begin{itemize}
        \item Existen varios ejemplares de cada libro.
        \item Se registra información sobre el histórico de préstamos que sufre un libro.
        \item Un usuario puede tener prestados varios libros al mismo tiempo.
    \end{itemize}

    El Diagrama Entidad-Relación correspondiente se encuentra en la Figura~\ref{fig:ej2}.
    \begin{figure}
        \centering
        \begin{tikzpicture}[node distance=6.3 em]
            \node[entidad] (libro) {Libro};
            \node[atributo] (ISBN)[above left of=libro] {\key{ISBN}} edge(libro);
            \node[atributo] (titulo)[above right of=libro] {Título} edge(libro);
            \node[atributo] (anio) [above of=libro] {Año de Escritura} edge(libro);

            \node[entidad debil] (ejemplar) [below of=libro] {Ejemplar} edge(libro);
            \node[atributo] (num_ejemplar) [left of=ejemplar] {\key{Número}} edge(ejemplar);

            \node[relacion] (autoria)[left of=libro] {Autoría} edge[participacion obligatoria](libro);
            \node[entidad] (autor)[left of=autoria] {Autor} edge(autoria);
            \node[atributo] (DNI_aut) [left of=autor] {\key{DNI}} edge(autor);
            \node[atributo] (nombre_aut) [above left of=autor] {Nombre} edge(autor);
            \node[atributo] (nacionalidad) [above of=autor] {Nacionalidad} edge(autor);

            \node[relacion] (trata) [right of=libro] {Trata sobre} edge[participacion obligatoria](libro);
            \node[entidad] (tema)[right of=trata] {Tema} edge(trata);
            \node[atributo] (nombre_tema) [above of=tema] {\ul{Nombre}} edge(tema);

            
            \node[relacion] (prestamo) [right of=ejemplar] {Préstamo} edge(ejemplar);
            \node[entidad] (usuario) [right of=prestamo] {Usuario} edge[Stealth-](prestamo);
            \node[atributo] (DNI_user) [below right of=usuario] {\key{DNI}} edge(usuario);
            \node[atributo] (nombre_user) [below of=usuario] {Nombre} edge(usuario);
            \node[atributo] (dir_user) [right of=usuario] {Dirección} edge(usuario);

            \node[atributo] (fecha) [below of=prestamo] {Fecha} edge[union discriminante](prestamo);
        \end{tikzpicture}
        \caption{Diagrama Entidad-Relación del Ejercicio \ref{ej:2}}
        \label{fig:ej2}
    \end{figure}
\end{ejercicio}

\begin{ejercicio} \label{ej:3}
    Se quiere hacer una BD para una empresa de alquiler de DVDs, considerando las siguientes restricciones semánticas:
    \begin{itemize}
        \item Las películas están caracterizadas por su título, año de estreno, actores principales y tema.
        \item De los clientes se almacena su DNI, nombre, dirección y teléfono.
        \item Puede haber películas distintas con el mismo nombre (versiones), pero estas deben ser de distinto año.
        \item Hay distintas copias de cada película que se pueden alquilar.
    \end{itemize}
\end{ejercicio}

\begin{ejercicio} \label{ej:4}
    Considere el ejercicio~\ref{ej:3} y la siguiente restricción adicional:
    \begin{itemize}
        \item Las películas con el mismo título tienen el mismo tema.
    \end{itemize}
\end{ejercicio}

\begin{ejercicio} \label{ej:5}
    Se quiere gestionar información relativa a la publicación de artículos científicos en revistas:
    \begin{itemize}
        \item Una revista se identifica por un ISSN y tiene un nombre y editorial.
        \item Durante un año la revista publica uno o varios números que recogen los artículos aceptados. De cada número de la revista se recoge la fecha de publicación.
        \item Cada número contiene uno o varios artículos.
        \item Cada artículo tiene un título y una lista ordenada de autores.
        \item También se almacena la página de inicio y de fin en el número de la revista en el que se ha publicado.
        \item Cada autor se identifica por un código y se caracteriza por su nombre y nacionalidad.
        \item Un artículo puede estar escrito por varios autores y un autor puede escribir varios artículos.
        \item Un artículo puede hacer referencia a otros artículos y puede ser citado en otros artículos.
    \end{itemize}
\end{ejercicio}


Sincronizacion

- Codigo de colores:
Variables calabaza: compartidas
Color verde: cerrojos
Color fuxia: cosas nivel hardware
Variables azules: privadas
Variables en negrita: da igual si compartidas o privadas

- Nota de asm
Para hacer que lecturas o escrituras no se adelanten hay que escribir en ensamblador `dmb`:
A = 1
dmb 
k = 1

- Nombres de codigo de sincronizacion
\textbf{Codigo de sincronización de adquisición}: para adquirir el derecho a acceder a variables compartidas.
Los códigos de sincronización que ejecutan los hilos una vez que han accedido a variables compartidas par que otros hilos accedan se le llama codigo de \textbf{sincronizacion de liberacion}.

- Ejemplo de comunicacion colectiva:
Suponemos que vamos a paralelizar el codigo secuencial. En la paralelizacion tenemos una condicion de carrera tipica de hebras.
Para evitar condicion de carrera, debemos evitar que se incluya nada entre la lectura y escritura de un mismo flujo. Esto lo conseguimos con un cerrojo (primitivas de sincronizacion software a bajo nivel, implementados en el SO. Las herramientas de programacion o implementan los suyos o usan los del SO).

Funcion de adquisicion, delate del acceso a variables compartidas, lock().
Funcion de liberacion, detras del acceso a variables compartidas, unlock().
Se vera como implementarlos para que funcionen bien.

En el ejemplo no es suficiente establecer los cerrojos, tenemos que hacer que el thread 0 espere al resto a actualizar la variable.

Con cerrojos y barreras tenemos todo el problema de sincronizacion resuelto.

- Niveles de abstraccion
Monitores > Regiones criticas condicionales y Regiones criticas > Cerrojos, semaforos y barreras > Instrucciones hardware de memoria atomicas (tenemos garantizado que en medio ningun otro flujo acceda a la variable. Se procesan en paralelo pero se ven secuencialmente).
Conforme se sube en la abstraccion, hay menor probabilidad de obtener errores, aunque obtenemos un codigo mas lento.

Para implementar cerrojos, semaforos y barreras de forma eficiente, tenemos que usar instrucciones para respetar consistencia del procesador e instrucciones que incluyen los procesdores para implementar sincronizacion de forma eficiente. Hay de dos tipos.




atomic es mas rapido k critical xq critical implementa cerrojos y atomic una instruccion maquina particular, del estilo *lock addl*.

- - - Como implementar cerrojos - - -
Si no queremos usar los del SO o los de las herramientas de programacion, podemos implementarlos (al igual k las barreras).

Tiene dos funciones a implementar: cierre del cerrojo (lock()), como adquisicion; apertura del cerrojo (unlock()).
Tenemos en la diapositiva los requisitos software, a cometar sobre el ejemplo.

Un cerrojo abierto es un 0 y uno cerrado es un 1.

- lock: Se deben cumplir las 3 condiciones:
LLega un flujo y ve cerrojo abierto.
Flujo i --> ve lock(k) a 0 --> Cierra cerrojo y entra.

Llegan varios y ven cerrojo cerrado
Flujo j, k, \ldots --> ve lock(k) a 1 --> while(k == 1) espera(); 

Llegan varios y solo uno puede verlo abierto.
Flujo 0, 1, \ldots, n --> ven lock(k) a 0. 
El software debe hacer que sólo 1 vea el cerrojo abierto y pase la seccion critica, que deberia dejar el cerrojo cerrado para k el resto espere.

- unlock: SÓLO VA A LLEGAR UN HILO.
Si hay hilos esperando, deja entrar sólo a uno.
Si no hubiera hilos, deja el cerrojo abierto.

- - Alternativas para espera - -
Si el codigo critico es muy gordo, hacer llamadas al SO para suspender hilos y luego revivirlos. Espera util. Para cuando la region critica suponga un tiempo mayor a suspender y reanimar.
Si es un codigo sencillo, no renta (demasiada sobrecarga). Se implementa una espera ocupada, el procesador está en un bucle consultando el cerrojo y haciendo un trabajo inutil.


- - Metodo de adquisicion - -
Si entran varios a la vez y el cerrojo esta abierto.
Hay dos alternativas:
- x86 y otros procesadores (como ARMv6): utilizan instrucciones de lectura-modificacion-escritura atomicas. Implementan una lectura y escritura de una variable de forma atomica (indivisible, no permite que ninguna instruccion de otro flujo se introduzca en medio). Se puede notar por $(RW)^i$.
- ARMv7,v8: instruccion de carga (LL, load linked) especial (enlazado al almacenamiento condicional) y una instruccion de almacenamiento especial (SC), store condicional. Se intenta el acceso atomico a una variable compartida. Si al ejecutar el almacenamiento condicional descubre (por una linea condicional) que entre el load y esta el acceso atomico no ha surgido efecto, vuelve a intentarlo.
Si LLi --> LLk --> CCi --> CCk
La carga de k no se realiza porque en medio se situa la carga de i, vuelve a intentarse.

- - Implementacion - -  (Cerrojo Simple II)
- La mas sencilla: Se usa una variable k compartida (que toma valores 0 y 1). Cerrojo simple.
Espera ocupada con un bucle mientras cerrojo esta cerrado (ver diapositiva).

lock(k){
    while(leer-asignar_1-escribir(k) == 1){};
}

unlock(k){
    k = 0;
}

Cumple los 2 primeros y el tercero si usamos un metodo de adquisicion de los ya vistos.
Codigo asm: ldrex (carga enlazada), strexeq (almacenamiento condicional) el bucle es para volver a repetir si no ha tenido exito. El dmb establece un orden total (todos los de arriba) deben ejecutarse antes de los de abajo (los accesos a memoria).


¿Puedo implementar un unlock de un cerrojo en un x86 con k = 0?
Consistencia relajada: lectura puede adelantar a escritura.
k = 0 tenemos escritura que puede adelantar a escrituras y lecturas anteriores, pero una escritura no puede adelantar a nada.

¿ARMv7?
No, tengo que usar un dmb delante, ya que la escritura en el cerrojo de 0 podria adelantar a los accesos a variables compartidas (abrir hilo antes de liberar seccion critica => posibilidad de varios hilos en la seccion critica).

- Criticas - 
Funciona bien pero es muy ineficiente: todos los que esperan estan leyendo y escribiendo en cache (invalidando copias de otras caches, 4 paquetes por coherencia).
Podemos hacer que escriban solo cuando el cerrojo este abierto.

Cambios:
Implementar un do while de:
lock(k){
    do{
        while(k == 1){pause;}    // Solo leen
    }while( RW(1) == 1 );   // Solo escribe al salir
}

Usando una instruccion escritura-modificacion atomica

Tambien cambiamos lineas de 0 -> 1 -> 0, \ldots, consumiendo energia.
Se usa la instruccion pause para añadir retardo.

- Critica -
No establece ningun orden en el acceso a la seccion critica: No hay prioridad sobre quien llega primero, entra quien lee antes.

- - Cerrojos con etiqueta - -
Establecen un orden FIFO en el acceso.
El orden con el que llegan al lock es el orden con el que acceden a la seccion critica.
Dos contadores: de adquisicion y de liberacion.

De liberacion: el que da el turno para acceder a la seccion critica. De adquisicion: sirve para cojer un numero.

lock(k){
    contador_local_adq = contadores.adq;
    contadores.adq++;
    while(contador_local_adq != contadores.lib){;}
}
Se deben ejecutar de forma atomica

unlock(k){
    contadores.lib++;
}


- - - Implementar barreras - - -
Un codigo posible si no se reusa la barrera en el codigo varias veces (sirve si sólo sale la barrera una vez en cada hilo).
Barrera(id, numero_flujos)
Tenemos un contador y una bandera que se situan a 0 y se dejan pasar los hilos cuando bandera = 1 (sucede cuando contador = numero_flujos).

Si usamos la barrera dos veces:
Si antes de que el 3er hilo llegue a la primera, el SO suspende el 2o hilo. el 2o hilo no pasa de la barrera cuando la bandera es 1. Puede que un hilo llegue al segundo uso de la barrera, poniendo la bandera a 0. Todos se quedan bloqueados.

Solucion:
En cada uso de la barrera se complementa la bandera de espera (esperan mientras que sea 1).
Usar otra barrera (necesitamos muchas barreras).

¿Podemos quitar la variable local por una compartida?
No.



- Instrucciones atomicas

\section{Arquitecturas TLP}

\begin{ejercicio}
    En un multiprocesador SMP con 4 procesadores o nodos (N0-N3) basado en un bus, que
    implementa el protocolo MESI para mantener la coherencia, supongamos una dirección de memoria incluida
    en un bloque que no se encuentra en ninguna caché. Indique los estados de este bloque en las cachés y las
    acciones que se producen en el sistema ante la siguiente secuencia de eventos para dicha dirección:
    \begin{enumerate}
        \item Lectura generada por el procesador 1.
        \item Lectura generada por el procesador 2.
        \item Escritura generada por el procesador 1.
        \item Escritura generada por el procesador 2.
        \item Escritura generada por el procesador 3.
    \end{enumerate}
\end{ejercicio}

\begin{ejercicio}
    Para un multiprocesador de memoria distribuida con $8$ nodos se quiere implementar un
    protocolo para mantenimiento de coherencia basado en directorios. Suponiendo que se necesitan un bit de
    estado para un bloque en el directorio de memoria principal y que el tamaño de una línea de caché es de 64
    bytes, calcular el porcentaje del tamaño de memoria principal que supone el tamaño del directorio de vector
    de bits completo.
\end{ejercicio}

\begin{ejercicio}
    Suponga que en un CC-NUMA de red estática de 4 nodos (N0-N3) se implementa un protocolo
    MSI basado en directorios sin difusión con dos estados en el directorio (válido e inválido). Cada nodo tiene 8
    GBytes de memoria y una línea de caché supone 64 Bytes. Considere que el directorio utiliza vector de bits
    completo.
    \begin{enumerate}
        \item Calcule el tamaño del directorio de uno nodo en bytes.
        \item Indique cual sería el contenido del directorio, las transiciones de estados (en caché y en el directorio) y la secuencia de paquetes generados por el protocolo de coherencia en los siguientes accesos sobre una dirección D que se encuentra en la memoria del nodo 3 (inicialmente D no está en ninguna caché):
        \begin{enumerate}
            \item Lectura generada por el procesador del nodo 1.
            \item Escritura generada por el procesador del nodo 1.
            \item Lectura generada por el procesador del nodo 2.
            \item Lectura generada por el procesador del nodo 3.
            \item Escritura generada por el procesador del nodo 0.
        \end{enumerate}
    \end{enumerate}
\end{ejercicio}

\begin{ejercicio}
    Supongamos que se va a ejecutar en paralelo el siguiente código, donde \verb|x| e \verb|y| son variables
    compartidas (inicialmente \verb|x=y=0|):
    \begin{figure}[H]
        \centering
        \begin{subfigure}{0.3\textwidth}
            \begin{minted}[frame=lines,
                framesep=2mm,
                baselinestretch=1.2,
                linenos,
                escapeinside=||]{cpp}
x=1;
x=2;
print y;
            \end{minted}
            \caption{P1.}
        \end{subfigure}\hspace{3cm}
        \begin{subfigure}{0.3\textwidth}
            \begin{minted}[frame=lines,
                framesep=2mm,
                baselinestretch=1.2,
                linenos,
                escapeinside=||]{cpp}
y=1;
y=2;
print x;
            \end{minted}
            \caption{P2.}            
        \end{subfigure}
    \end{figure}

    ¿Qué resultados se pueden imprimir si (considere que el compilador no altera el código):
    \begin{enumerate}
        \item Se ejecutan P1 y P2 en un multiprocesador con consistencia secuencial.
        \item Se ejecutan en un multiprocesador basado en un bus que garantiza todos los órdenes excepto el
        orden \verb|W|$\to$\verb|R|. Esto es debido a que los procesadores tienen buffer de escritura, permitiendo el
        procesador que las lecturas en el código que ejecuta adelanten a las escrituras que tiene su buffer.
        Obsérvese que hay varios resultados posibles.
    \end{enumerate}
\end{ejercicio}


\begin{ejercicio}
    Supongamos que se va a ejecutar en paralelo el siguiente código (inicialmente \verb|x=y=0|):
    \begin{figure}[H]
        \centering
        \begin{subfigure}{0.3\textwidth}
            \begin{minted}[frame=lines,
                framesep=2mm,
                baselinestretch=1.2,
                linenos,
                escapeinside=||]{cpp}
x=30;
y=40;
flag=1;
            \end{minted}
            \caption{P1.}
        \end{subfigure}\hspace{3cm}
        \begin{subfigure}{0.3\textwidth}
            \begin{minted}[frame=lines,
                framesep=2mm,
                baselinestretch=1.2,
                linenos,
                escapeinside=||]{cpp}
while (flag==0) {};
r1=x;
r2=y;
            \end{minted}
            \caption{P2.}            
        \end{subfigure}
    \end{figure}

    ¿Qué datos puede obtener P2 en los registros \verb|r1| y \verb|r2| si (considere que el compilador no altera el código):
    \begin{enumerate}
        \item Se ejecutan P1 y P2 en un multiprocesador con consistencia secuencial.
        \item Se ejecutan en un multiprocesador con un modelo de consistencia que relaja todos los órdenes en
        los accesos a memoria. Razone su respuesta.
    \end{enumerate}
\end{ejercicio}

\begin{ejercicio}
    Se quiere implementar un cerrojo simple en un multiprocesador SMP basado en procesadores de
    la línea x86 de Intel, en particular, procesadores Intel Core. 
    \begin{enumerate}
        \item Teniendo en cuenta el modelo de consistencia de memoria que ofrece el hardware de este multiprocesador ¿podríamos implementar la función de liberación del cerrojo simple usando \verb|mov k, 0|, siendo \verb|k| la variable cerrojo? Razone su respuesta.
        \item ¿Cómo se debería implementar la función de liberación de un cerrojo simple si se usan procesadores con la arquitectura ARMv7? Razone su respuesta.
    \end{enumerate}
\end{ejercicio}

\begin{ejercicio}
    Se ha ejecutado el siguiente código en un multiprocesador con un modelo de consistencia que no
    garantiza ni \verb|W|$\to$\verb|R| ni \verb|W|$\to$\verb|W| (garantiza el resto de órdenes):
    \begin{figure}[H]
        \centering
        \begin{minted}[frame=lines,
            framesep=2mm,
            baselinestretch=1.2,
            linenos,
            escapeinside=||]{cpp}
sump = 0;
for (i=ithread ; i<8 ; i=i+nthread) {
    sump = sump + a[i];
}
while (Fetch_&_Or(k,1)==1) {};
sum = sum + sump;
k=0;
        \end{minted}
    \end{figure}
    Conteste a las siguientes preguntas (considere que el compilador no altera el código):
    \begin{enumerate}
        \item Indique qué se puede obtener en \verb|sum| si se suma la lista \verb|a={1,2,3,4,5,6,7,8}|. \verb|k| y \verb|sum| son variables
        compartidas que están inicialmente a 0 (el resto de variables son privadas), \verb|nthread = 3|, \verb|ithread| es el
        identificador del thread en el grupo (0,1,2). Si hay varios posibles resultados, se tienen que dar todos
        ellos. Justifique su respuesta.
        \item ¿Qué resultados se pueden obtener si lo único que no garantiza el modelo de consistencia es el orden
        \verb|W|$\to$\verb|R|? Justifique su respuesta.
    \end{enumerate}
\end{ejercicio}

\begin{ejercicio}
    ¿Qué ocurre si en el segundo código para implementar barreras visto en clase eliminamos la
    variable local, \verb|cont_local|, sustituyéndola en los puntos del código donde aparece por el contador
    compartido asociado a la barrera \verb|bar[id].cont|?
\end{ejercicio}

\begin{ejercicio}
    Suponiendo que la arquitectura dispone de instrucciones \verb|Fetch&Add|, simplifique el segundo
    código para barreras visto en clase.
\end{ejercicio}

\begin{ejercicio}
    Se quiere paralelizar el siguiente ciclo de forma que la asignación de iteraciones a los
    procesadores disponibles se realice en tiempo de ejecución (dinámicamente):
    \begin{figure}[H]
        \centering
        \begin{minted}[frame=lines,
            framesep=2mm,
            baselinestretch=1.2,
            linenos,
            escapeinside=||]{cpp}
for (i=0; i<100; i++) {
    // Código que usa i
}
        \end{minted}
    \end{figure}
    \begin{observacion}
        Considerar que las iteraciones del ciclo son independientes, que el único orden no garantizado por el
        sistema de memoria es \verb|W|$\to$\verb|R|, que las primitivas atómicas garantizan que sus accesos a memoria se realizan
        antes que los accesos posteriores y que el compilador no altera el código.
    \end{observacion}
    \begin{enumerate}
        \item Paralelizar el ciclo para su ejecución en un multiprocesador que implementa la primitiva \verb|Fetch&Or|
        para garantizar exclusión mutua.
        \item Paralelizar el anterior ciclo en un multiprocesador que además tiene la primitiva \verb|Fetch&Add|.
    \end{enumerate}
\end{ejercicio}

\begin{ejercicio}
    Un programador está usando el siguiente código para barreras (\verb|bar| es un vector compartido, \verb|k|
    es una variable compartida, el resto son variables locales, \verb|Fetch_&_Or(k,1)| realiza sus accesos a memoria
    antes de que puedan realizarse los accesos posteriores):
    \begin{figure}[H]
        \centering
        \begin{minted}[frame=lines,
            framesep=2mm,
            baselinestretch=1.2,
            linenos,
            escapeinside=||]{cpp}
Barrera(id, num_procesos)
{
    band_local= !(band_local)
    while (Fetch_&_Or(k,1)==1) {};
    cont_local = ++bar[id].cont;
    k=0;
    if (cont_local == num_procesos) {
        bar[id].cont=0;
        bar[id].band=band_local;
    }
    else while (bar[id].band != band_local) {};
}
        \end{minted}
    \end{figure}
    Conteste a las siguientes cuestiones (considere que el compilador no altera el código):
    \begin{enumerate}
        \item ¿Funciona bien este código como barrera en un multiprocesador en el que lo único que no garantiza su
        modelo de consistencia es el orden \verb|W|$\to$\verb|R|? Razone por qué.
        \item Funciona bien este código como barrera en un multiprocesador con modelo de consistencia que no
        garantice ningún orden en los accesos a memoria? Razone por qué.
    \end{enumerate}
\end{ejercicio}


\begin{ejercicio}
    Se quiere implementar un programa que calcule en paralelo la siguiente expresión en un
    multiprocesador en el que sólo se relaja el orden \verb|W|$\to$\verb|R| y en el que sólo se dispone de primitiva de
    sincronización \verb|test_&_set|:
    \[
        d = \frac{1}{N} \sum_{i=1}^{N} x_i^2 - \ol{x}^2, \quad \ol{x} = \frac{1}{N} \sum_{i=1}^{N} x_i
    \]
    Un programador ha implementado el código de abajo. Tenga en cuenta lo siguiente:
    \begin{itemize}
        \item El código lo ejecutan \verb|nthread| threads en paralelo.
        \item \verb|ithread| es una variable local que nota el identificador del thread.
        \item \verb|i|, \verb|medl| y \verb|varil| son variables locales.
        \item \verb|i|, \verb|med|, \verb|vari|, el vector \verb|x| y \verb|N| son variables compartidas.
        \item Inicialmente \verb|med|, \verb|vari|, \verb|medl| y \verb|varil| son 0.
    \end{itemize}
    \begin{figure}[H]
        \centering
        \begin{minted}[frame=lines,
            framesep=2mm,
            baselinestretch=1.2,
            linenos,
            escapeinside=||]{cpp}
for (i=ithread;i<N;i=i+nthread) {
    medl=medl+x[i];
    varil=varil+x[i]*x[i];
}
med = med + medl/N; vari = vari + varil/N;
vari= vari - med*med;
if (ithread==0) printf("varianza = %f", vari); //imprime en pantalla
        \end{minted}
    \end{figure}
    Conteste a las siguientes cuestiones (considere que el compilador no altera el código):
    \begin{enumerate}
        \item Se ha ejecutado este código usando varios threads y se ha visto que, aunque \verb|N| y el vector \verb|x| no
        varían, no siempre se imprime lo mismo. ¿Por qué ocurre esto?
        \item Añada lo mínimo necesario para solucionar el problema teniendo en cuenta que sólo se dispone
        para implementar sincronización de \verb|test_&_set| (tampoco se dispone de primitivas software de
        sincronización). Indique qué variables son ahora compartidas y cuáles locales.
        \item Escriba el programa suponiendo que el multiprocesador además tiene primitivas de sincronización
        \verb|fetch_&_add| (se tendrá en cuenta las prestaciones). Indique qué variables son compartidas y
        cuáles locales.
        \item Escriba el programa ahora suponiendo que el multiprocesador sólo tiene primitivas de
        sincronización \verb|compare_&_swap| (se tendrá en cuenta las prestaciones). Indique qué variables son
        compartidas y cuáles locales.
    \end{enumerate}
    \begin{observacion}
        En todos los apartados puede añadir o quitar variables si lo estima conveniente.
    \end{observacion}
\end{ejercicio}

\begin{ejercicio}
    Se ha extraído la siguiente implementación de cerrojo (\textit{spin-lock}) para x86 del kernel de Linux, el cual se muestra
    en el Código Fuente~\ref{code:spinlock}.
    \lstinputlisting[
        language=C,
        style=vscode_C,
        caption={Código de un cerrojo en el kernel de Linux.},
        label={code:spinlock}
    ]{Relacion3_Ejercicio13.c}

    Conteste a las siguientes preguntas:
    \begin{enumerate}
        \item Utiliza una implementación de cerrojo con etiquetas ¿Cuál es el contador de adquisición y cuál es el
        contador de liberación?
        \item Describa qué hace \verb|xaddw %w0, %1| ¿opera con el contador de adquisición, con el de liberación o con
        los dos? ¿qué operaciones hace con ellos?
        \item Describa qué hace \verb|cmpb %h0, %b0| ¿opera con el contador de adquisición, con el de liberación o con
        los dos? ¿qué operaciones hace con ellos?
        \item ¿Por qué cree que se usa el prefijo \verb|lock| delante de la instrucción \verb|xaddw|? 
    \end{enumerate}

    \begin{observacion} Es recomendable consultar recursos siguientes:
        \begin{itemize}
            \item Puede consultar las instrucciones en el manual de Intel con el repertorio de instrucciones
            (Volumen 2 o volúmenes 2A, 2B y 2C) que puede encontrar \href{http://www.intel.com/content/www/us/en/processors/architectures-software-developer-manuals.html}{\color{blue}\ul{aquí}}.
            \item Si no recuerda la interfaz entre C/C++ y ensamblador en \verb|gcc| (se ha presentado en Estructura de
            Computadores), consulte el manual de \verb|gcc| \href{http://gcc.gnu.org/onlinedocs/gcc-4.6.2/gcc/ExtendedAsm.html#Extended-Asm}{\color{blue}\ul{aquí}}.
        \end{itemize}
    \end{observacion}
\end{ejercicio}


\subsection{Cuestiones}

\begin{cuestion}
    Diferencias entre núcleos con multithread temporal y núcleos con multithread simultánea.
\end{cuestion}

\begin{cuestion}
    Diferencias entre núcleos con multithread temporal de grano fino y núcleos con multithread
    temporal de grano grueso.
\end{cuestion}

\begin{cuestion}
    Suponga un multiprocesador con protocolo MESI de espionaje. Si un controlador de cache
    observa en el bus un paquete de petición de lectura exclusiva de un bloque que tiene en estado C, debe
    (indique cuál sería la respuesta correcta y razone por qué es la respuesta correcta):
    \begin{enumerate}
        \item Generar un paquete de respuesta con el bloque y pasar el bloque a estado I.
        \item Pasar el bloque a estado I.
        \item Generar un paquete de respuesta con el bloque y pasar el bloque a estado E.
        \item No tiene que hacer nada.
    \end{enumerate}
\end{cuestion}

\begin{cuestion}
    Suponga un multiprocesador con protocolo MESI de espionaje. Si un nodo observa en el bus un
    paquete de petición de lectura exclusiva de un bloque que tiene en estado M, ¿qué debe hacer? Razone su
    respuesta.
\end{cuestion}

\begin{cuestion}
    Suponga un multiprocesador con el protocolo MESI de espionaje. Si el procesador de un nodo
    escribe en un bloque que tiene en su cache en estado I, debe (indique cuál sería la respuesta correcta y
    razone por qué es la respuesta correcta):
    \begin{enumerate}
        \item Generar paquete de petición de acceso E al bloque y pasar el bloque a M.
        \item Generar paquete de petición de acceso E y pasar el bloque a estado E.
        \item Generar paquete de petición de acceso E con lectura y pasar el bloque a estado E.
        \item Generar paquete de petición de acceso E al bloque con lectura y pasar el bloque a M.
    \end{enumerate}
\end{cuestion}

\begin{cuestion}
    ¿Cuál de los siguientes modelos de consistencia permite mejores tiempos de ejecución?
    Justifique su respuesta.
    \begin{enumerate}
        \item Modelo de consistencia que no garantiza los órdenes \verb|W|$\to$\verb|W| y \verb|W|$\to$\verb|R|.
        \item Modelo implementado en los procesadores de la línea x86.
        \item Modelo de consistencia secuencial.
        \item Modelo de consistencia que no garantiza ningún orden.
    \end{enumerate}
\end{cuestion}

\begin{cuestion}
    Indique qué expresión no se corresponde con la serie (justifique su respuesta):
    \begin{enumerate}
        \item \verb|Lock|.
        \item \verb|Fetch_and_Or|.
        \item \verb|Compare_and_Swap|.
        \item \verb|Test_and_Set|.
    \end{enumerate}
\end{cuestion}
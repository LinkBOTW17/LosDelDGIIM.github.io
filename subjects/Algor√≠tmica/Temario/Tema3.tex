\chapter{Algoritmos Greedy}
Los algoritmos Greedy son algoritmos que siempre toman la mejor decisión en cada momento, la cual sólo depende de datos locales. Por ejemplo, un algoritmo que busque recorrer $n$ ciudades en el plano podría implementarse mediante la técnica Greedy cogiendo en cada paso la ciudad que se encuentre más cerca de la que estamos. Notemos que de esta forma no obtenemos la solución óptima al problema, aún habiendo tomado la solución óptima en cada caso.

Los algoritmos Greedy tienen como características:
\begin{itemize}
    \item Se suelen utilizar para resolver problemas de optimización: búsqueda de máximo o mínimo.
    \item Toman decisiones en función de la información local de cada momento, tomando la mejor alternativa.
    \item Una vez tomada la decisión, no se replantea en el futuro.
    \item Son fáciles y rápidos de implementar.
    \item Los algoritmos Greedy no siempre proporcionan la solución óptima, como ya hemos podido comprobar.
\end{itemize}

\noindent
Elementos de un algoritmo voraz:
\begin{itemize}
    \item Conjunto de candidatos: representa al conjunto de posibles soluciones que se pueden tomar en cualquier momento.
    \item Conjunto de seleccionadas: representa al conjunto de decisiones tomadas hasta el momento.
    \item Función solución: determina si se ha alcanzado una solución al problema.
    \item Función de selección: determina el candidato que proporciona el mejor paso.
    \item Función de factibilidad: determina si es posible o no llegar a la solución del problema mediante el conjunto de seleccionados.
    \item Función objetivo: da el valor de la solución alcanzada.
\end{itemize}

\noindent
Un algoritmo Greedy puede resolverse de la forma:
\begin{itemize}
    \item Se parte de un conjunto de candidatos a solución vacío.
    \item De la lista de candidatos que hemos podido identificar, con la función de selección, se coge el mejor candidato posible.
    \item Vemos si con ese elemento podríamos llegar a constituir una solución, es decir, si se verifican las condiciones de factibilidad en el conjunto de candidatos.
    \item Si el candidato anterior no es válido, lo borramos de la lista de candidatos y nunca más se considera.
    \item Evaluamos la función solución (si no hemos terminado, seleccionamos otro candidato con la función de selección y repetimos el proceso hasta alcanzar una solución).
\end{itemize}

Normalmente, si conseguimos un algoritmo Greedy óptimo, es probable que sea el mejor algoritmo que resuelve el problema.

\subsubsection{Problema de dar cambio}
Normalmente, las máquinas que dan cambio están programadas de forma que nos lo devuelven con el mínimo número posible de monedas. 

En este caso, el conjunto de candidatos es el conjunto de monedas que tiene disponible la máquina. El conjunto de seleccionados son las monedas que voy incorporando a la solución. Habremos encontrado una solución cuando hayamos llegado a la cifra que se pedía. Si superamos o no llegamos a la cantidad a devolver, nuestra solución no es factible.

Como criterio de selección inicial, podemos elegir la moneda de mayor valor que sea menor o igual que el precio a pagar.\\

¿Este algoritmo consigue la solución óptima?\newline
Pues si el valor de las monedas corresponde con los billetes y monedas de euro en circulación sí; pero si tenemos otro sistema monetario, quizás no.

\subsubsection{Problema de selección de programas}
Tenemos un conjunto $T$ de $n$ programas, cada uno con tamaño $t_1, \ldots, t_n$ y un dispositivo de capacidad máxima $C$, seleccionar el mayor número de programas que pueden meterse en $C$.

% // TODO: meter demostraciones por induccion, explicando qué hay que demostrar:
% 1. caso base
% 2. subestructuras optimales.



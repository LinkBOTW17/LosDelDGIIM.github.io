\chapter{Clasificación de grupos de orden bajo}
\noindent
Clasificar grupos es una tarea dura y difícil, por lo que nos centraremos en grupos de orden bajo.

\section{Grupos abelianos}
Sabemos ya clasificar los grupos abelianos de orden menor o igual que 15:
\begin{equation*}
    \begin{array}{c|c}
        \text{Orden}  & \text{Grupo} \\
        \hline
        1 & \{1\} \\
        2 & C_2 \\
        3 & C_3 \\
        4 & C_2\oplus C_2, C_4 \\
        5 & C_5 \\
        6 & C_6 \\
        7 & C_7 \\
        8 & C_2\oplus C_2 \oplus C_2, C_2\oplus C_4 \oplus C_8 \\
        9 & C_9, C_3\oplus C_3 \\
        10 & C_{10} \\
        11 & C_{11} \\
        12 & C_{12}, C_6\oplus C_2 \\
        13 & C_{13} \\
        14 & C_{14} \\
        15 & C_{15} \\
        \vdots & \vdots 
    \end{array}
\end{equation*}
En general, si $|G| = p$ primo, tendremos $C_p$.

Si el grupo no es finito, nos dan la presentación y también sabemos clasificarlos. Nos centraremos por tanto en los grupos no abelianos.

\section{Producto semidirecto}
\noindent
Será una especie de producto que nos permitirá escribir muchos grupos no abelianos. Con esta herramienta podremos escribir muchos grupos no abelianos (aunque no todos).

\begin{ejemplo}
    En el Capítulo~\ref{cap:1} vimos que $Q_2 = \{\pm 1, \pm i, \pm j, \pm k\}$ es isomorfo a:
    \begin{equation*}
        Q_2^{\text{abs}} = \langle x,y\mid x^4=1, x^2=y^2, yxy^{-1} = x^{-1} \rangle 
    \end{equation*}
    Es decir, teníamos una aplicación (gracias a Teorema de Dyck) $f:Q_2\to Q_2^{\text{abs}}$ dada por:
    \begin{equation*}
        f(x) = i \qquad f(y) = j
    \end{equation*}
    Que además era un epimorfimos, porque $Q_2^{\text{abs}} = \langle i,j \rangle $. Veamos que $|Q_2^{\text{abs}}| = 8$, de una forma distinta que contar elementos:
    \begin{proof} % // TODO: Revisar esta demostración
        Como $x^4 = 1$, si consideramos $H = \langle x \rangle $, tendremos que $|H| \leq 4$. Ahora, como:
        \begin{equation*}
            yxy^{-1} = x^{-1}\in H
        \end{equation*}
        Tenemos que $H\lhd Q_2^{\text{abs}}$. Si escribimos $Q_2^{\text{abs}}$ en su partición de clases:
        \begin{equation*}
            Q_2^{\text{abs}} \cong H\cup yH
        \end{equation*}
        Ya que $y\notin H$, de donde al tomar cocientes:
        \begin{equation*}
            Q_2^{\text{abs}}/H\cong \langle yH \rangle 
        \end{equation*}
        Ahora, como:
        \begin{equation*}
            {(yH)}^{2} = y^2H = x^2H = H
        \end{equation*}
        Llegaremos a que:
        \begin{equation*}
            |Q_2^{\text{abs}}/H| \leq 2
        \end{equation*}
        Si aplicamos el Primer Teorema de Isomorfía sobre $f$:
        \begin{equation*}
            Q_2^{\text{abs}}/\ker(f) \cong Img(f) = Q_2
        \end{equation*}
        De donde $|Q_2^{\text{abs}}| = |Q_2||\ker(f)| \geq 8$. Concluimos que $|Q_2^{\text{abs}}| = 8$.
    \end{proof}
\end{ejemplo}

\noindent
El hecho de introducir $Q_2^{\text{abs}}$ en el Capítulo~\ref{cap:1} fue para ahora generalizar lo que hacíamos con $Q_2$ a todo grupo, con el producto semidirecto.

\begin{definicion}[Grupos dicíclicos]
    Para cada $k\in \mathbb{N}\setminus \{0\}$, definimos el $k-$ésimo grupo dicíclico como el grupo:
    \begin{equation*}
        Q_k = \left\langle x,y \mid x^{2k} = 1, y^2 = x^k, yxy^{-1} = x^{-1}  \right\rangle 
    \end{equation*}
\end{definicion}

\begin{ejemplo}
    Veamos que:
    \begin{itemize}
        \item Para $k=1$:
            \begin{equation*}
                Q_1 = \langle x,y\mid x^2 = 1, y^2 = x, yxy^{-1} = x \rangle 
            \end{equation*}
            Nos preguntamos qué grupo es. Si tratamos de describir los elementos, obtenemos:
            \begin{equation*}
                \{1,x,y,xy\} = \{1,y,y^2,y^3\}
            \end{equation*}
            Es decir, $Q_1\cong C_4$.
        \item Observemos que si $k=2$, obtenemos $Q_2^\text{abs}$.
        \item Para $k\geq 3$, tendremos que tiene un cociente isomorfo a $D_k$, por lo que no será abeliano. Sin embargo, podemos acotar el orden de $Q_k$:
            \begin{equation*}
                2k \leq |Q_k| \leq 4k \qquad \forall k\geq 3
            \end{equation*}
            Y si $k$ es impar, tendremos que $|Q_k| = 4k$.
            \begin{proof}
                Si recordamos al grupo diédrico de orden $k$:
                \begin{equation*}
                    D_k = \langle r,s \mid r^k = s^2 = 1, sr = r^{-1}s \rangle 
                \end{equation*}
                Tenemos que $r^{2k} = {(r^{k})}^{2} = 1$ y tenemos la primera relación. Para la segunda, tenemos que $s^2 = 1 = r^k$. Finalmente, $sr = r^{-1}s$ nos da la tercera (compruébese). Podemos aplicar el Teorema de Dyck, que nos da un homomorfismo $f:Q_k\to D_k$ de forma que: % // TODO: Hacer ese "compruébese"
                \begin{equation*}
                    f(x) = r \qquad f(y) = s
                \end{equation*}
                Además, sabemos que $f$ es un epimorfimo, ya que $D_k = \langle r,s \rangle $. Por el Primer Teorema de Isomorfía aplicado a $f$, podemos asegurar que:
                \begin{equation*}
                    Q_k/\ker(f)\cong D_k
                \end{equation*}
                Por tanto, el dicíclico de orden $k$ no será abeliano.  % // TODO: Justificar por qué (yo no lo se)

                \begin{description}
                    \item [$\geq)$] Sabemos que $|D_k| = 2k$, y por el epimorfismo anterior, sabemos que $2k$ divide a $|Q_k|$, de donde $2k\leq Q_k$.
                    \item [$\leq)$] Usando que $x^{2k} = 1$, si tomamos $H = \langle x \rangle $, tenemos que:
                        \begin{equation*}
                            |H| = |\langle x \rangle | \leq 2k
                        \end{equation*}
                        Como también tenemos que:
                        \begin{equation*}
                            yxy^{-1} = x^{-1}\in H
                        \end{equation*}
                        Tendremos que $H\lhd Q_k$, de donde al considerar el cociente, tendremos al igual que antes que:
                        \begin{equation*}
                            Q_k/H \cong \langle yH \rangle 
                        \end{equation*}
                        De esta forma:
                        \begin{equation*}
                            {(yH)}^{2} = y^2H = x^kH = H
                        \end{equation*}
                        Por lo que $y^2\notin yH$ y:
                        \begin{equation*}
                            |Q_k/H| \leq 2
                        \end{equation*}
                        De donde deduimos que:
                        \begin{equation*}
                            |Q_k| = |Q_k/H||H| \leq 4k
                        \end{equation*}
                \end{description}

                \noindent
                \underline{Suponiendo ahora que $k = 2t+1$} para cierto $t\in \mathbb{N}$, consideramos el cíclico de orden 4:
                \begin{equation*}
                    C_4 = \langle a \mid a^4 = 1 \rangle 
                \end{equation*}
                \begin{enumerate}
                    \item Tenemos que:
                        \begin{equation*}
                            {(a^{2})}^{2k} = {(a^4)}^{k} = 1
                        \end{equation*}
                    \item Además:
                        \begin{equation*}
                            {(a^{2})}^{k} = a^{2k} = a^{4t+2} = a^2
                        \end{equation*}
                    \item Finalmente:
                        \begin{equation*}
                            aa^2 = a^3 = a^2 a = {(a^{2})}^{-1}a
                        \end{equation*}
                \end{enumerate}
                Por el Teorema de Dyck, podemos construir el homomorfismo $f:Q_k\to C_4$ dado por:
                \begin{equation*}
                    f(x) = a^2 \qquad f(y) = a
                \end{equation*}
                Que de hecho es un epimorfismo, ya que $C_4 = \langle a \rangle $. Al igual que antes:
                \begin{equation*}
                    Q_k/\ker(f) \cong C_4
                \end{equation*}
                De donde llegamos a que 4 divide a $|Q_k|$. Como además $2k$ divide a $|Q_k|$, tenemos que $\mcm(2k,4)$ divide a $|Q_k|$ y, como $k$ era impar, tendremos que:
                \begin{equation*}
                    \mcm(2k,4) = 4
                \end{equation*}
                De donde $4k\leq |Q_k|$, que con la desigualdad anterior nos da la igualdad.
            \end{proof}
    \end{itemize}
\end{ejemplo}

\begin{ejemplo}
    Grupos no abelianos de orden 12 conocíamos:
    \begin{itemize}
        \item $A_4$.
        \item $D_6$.
    \end{itemize}
    Y ahora conocemos $Q_3$. Próximamente veremos que estos grupos son los únicos, salvo isomorfismo.
\end{ejemplo}

\begin{definicion}[Producto semidirecto]
    Dados dos grupos, $K$, $H$ y una acción $\theta: H \to Aut(K)$, consideramos el conjunto producto cartesiano:
    \begin{equation*}
        G = K\times H = \{(k,h)\mid k\in K, h\in H\}
    \end{equation*}
    Sobre el que definimos la siguiente operación:
    \begin{equation*}
        (k_1,h_1)(k_2,h_2) = (k_1 \prescript{h_1}{}{k_2}, h_1h_2)
    \end{equation*}
    Se verifica que $K\times H$ con esta operación tiene estructura de grupo, al que llamaremos \textbf{producto semidirecto de $K$ por $H$ relativo a $\theta$}, que denotaremos por:
    \begin{equation*}
        K \rtimes_\theta H
    \end{equation*}
\end{definicion}

\begin{teo}
    Se verifica que $K\times H$ con esta operación tiene estructura de grupo
    \begin{proof} % // TODO:
        \begin{itemize}
            %\item Asociativa:
            \item El elemento $(1,1)$ es el neutro:
                \begin{align*}
                    (k,h)(1,1) &= (k\prescript{h}{}{1},h) = (k,h) \\
                    (1,1)(k,h) &= (1, \prescript{1}{}{k},h) = (k,h) \\
                               &\forall (k,h)\in K\times H
                \end{align*}
            \item Para el inverso, dado $(k,h)\in K\times H$, el inverso será:
                \begin{equation*}
                    {(k,h)}^{-1} = (\prescript{h^{-1}}{}{k^{-1}},h^{-1})
                \end{equation*}
                \begin{equation*}
                    (k,h)(\prescript{h^{-1}}{}{k^{-1}},h^{-1}) = (k\prescript{h}{}{(\prescript{h^{-1}}{}{k^{-1}})}, hh^{-1}) = (k\prescript{hh^{-1}}{}{k^{-1}}, 1) = (kk^{-1},1) = (1,1)
                \end{equation*}
        \end{itemize}
    \end{proof}
\end{teo}

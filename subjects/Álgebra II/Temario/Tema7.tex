\chapter{Clasificación de grupos de orden bajo}
\noindent
Clasificar grupos es una tarea dura y difícil, por lo que nos centraremos en aprender a clasificar grupos de orden bajo. En concreto, nuestro objetivo será saber clasificar todos los grupos de oden menor o igual que 15.

\subsubsection{Grupos abelianos}
\noindent
En el Capítulo anterior aprendimos ya a clasificar todos los grupos abelianos finitos. En particular, sabemos ya clasificar todos los grupos abelianos finitos de orden menor o igual que 15:
\begin{equation*}
    \begin{array}{c|ccc}
        \text{Orden}  & \text{Grupos} & &  \\
        \hline
        1 & \{1\} &  &\\
        2 & C_2 & & \\
        3 & C_3 & & \\
        4 & C_4, &  C_2\oplus C_2& \\
        5 & C_5 & & \\
        6 & C_6 & & \\
        7 & C_7 & & \\
        8 & C_8, &  C_2\oplus C_4,  & C_2\oplus C_2 \oplus C_2\\
        9 & C_9, &  C_3\oplus C_3 & \\
        10 & C_{10} & & \\
        11 & C_{11} & & \\
        12 & C_{12}, &  C_2\oplus C_6 & \\
        13 & C_{13} & & \\
        14 & C_{14} & & \\
        15 & C_{15} & & \\
        \vdots & \vdots 
    \end{array}
\end{equation*}
\noindent
Por tanto, nos centraremos ahora en tratar de describir todos los grupos finitos no abelianos de orden menor o igual que 15.

\section{Producto semidirecto}
\noindent
Con el fin de conseguir nuestro objetivo, definiremos el producto semidirecto, herramienta que nos permitirá escribir muchos grupos no abelianos (aunque no todos).

\begin{ejemplo}
    En el Capítulo~\ref{cap:1} vimos que $Q_2 = \{\pm 1, \pm i, \pm j, \pm k\}$ es isomorfo a:
    \begin{equation*}
        Q_2^{\text{abs}} = \langle x,y\mid x^4=1, x^2=y^2, yxy^{-1} = x^{-1} \rangle 
    \end{equation*}
    Es decir, teníamos una aplicación (gracias a Teorema de Dyck) $f:Q_2\to Q_2^{\text{abs}}$ dada por:
    \begin{equation*}
        f(x) = i \qquad f(y) = j
    \end{equation*}
    Que además era un epimorfimo, porque $Q_2^{\text{abs}} = \langle i,j \rangle $. Veamos que $|Q_2^{\text{abs}}| = 8$, de una forma distinta que contar elementos:
    \begin{proof} % // TODO: Revisar esta demostración
        Como $x^4 = 1$, si consideramos $H = \langle x \rangle $, tendremos que $|H| \leq 4$. Ahora, como:
        \begin{equation*}
            yxy^{-1} = x^{-1}\in H 
        \end{equation*}
        Tenemos que $H\lhd Q_2^{\text{abs}}$. Si escribimos $Q_2^{\text{abs}}$ en su partición de clases (como $Q_2^{\text{abs}} = \langle x,y \rangle $, si un elemento no está en $H$ es porque es producto de $y$ por algo más):
        \begin{equation*}
            Q_2^{\text{abs}} \cong H\cup yH
        \end{equation*}
        Ya que $y\notin H$, de donde al tomar cocientes:
        \begin{equation*}
            Q_2^{\text{abs}}/H = \langle yH \rangle 
        \end{equation*}
        Ahora, como:
        \begin{equation*}
            {(yH)}^{2} = y^2H = x^2H = H
        \end{equation*}
        Entonces $O(yH) = 2$, por lo que:
        \begin{equation*}
            |Q_2^{\text{abs}}/H| = 2
        \end{equation*}
        Si aplicamos el Primer Teorema de Isomorfía sobre $f$:
        \begin{equation*}
            Q_2^{\text{abs}}/\ker(f) \cong Im(f) = Q_2
        \end{equation*}
        De donde $|Q_2^{\text{abs}}| = |Q_2||\ker(f)| \geq 8$. Concluimos que $|Q_2^{\text{abs}}| = 8$.
    \end{proof}
\end{ejemplo}

\noindent
El hecho de introducir $Q_2^{\text{abs}}$ en el Capítulo~\ref{cap:1} fue para ahora generalizar lo que hacíamos con $Q_2$ a todo grupo, con el producto semidirecto.

\begin{definicion}[Grupos dicíclicos]
    Para cada $k\in \mathbb{N}\setminus \{0\}$, definimos el $k-$ésimo grupo dicíclico como el grupo:
    \begin{equation*}
        Q_k = \left\langle x,y \mid x^{2k} = 1, y^2 = x^k, yxy^{-1} = x^{-1}  \right\rangle 
    \end{equation*}
\end{definicion}

\begin{ejemplo}
    Estudiemos los grupos dicíclicos:
    \begin{itemize}
        \item Para $k=1$:
            \begin{equation*}
                Q_1 = \langle x,y\mid x^2 = 1, y^2 = x, yxy^{-1} = x \rangle 
            \end{equation*}
            Nos preguntamos qué grupo es. Si tratamos de describir los elementos, obtenemos:
            \begin{equation*}
                \{1,x,y,xy\} = \{1,y,y^2,y^3\}
            \end{equation*}
            Es decir, $Q_1\cong C_4$.
        \item Observemos que si\footnote{Aquí cometemos un pequeño abuso de notación, ya que $Q_2$ no es el grupo de los cuaternios, sino el segundo grupo dicíclico.} $k=2$, obtenemos $Q_2^\text{abs}$:
            \begin{equation*}
                Q_2 = \langle x,y \mid x^4 = 1, y^2 = x^2, yx = x^{-1}y \rangle  = Q_2^{\text{abs}}
            \end{equation*}
        \item Para $k\geq 3$, veamos que:
            \begin{equation*}
                2k \leq |Q_k| \leq 4k \qquad \forall k\geq 3
            \end{equation*}
            Y si $k$ es impar, entonces $|Q_k| = 4k$.
            \begin{proof}
                Si recordamos al grupo diédrico de orden $k$:
                \begin{equation*}
                    D_k = \langle r,s \mid r^k = s^2 = 1, sr = r^{-1}s \rangle 
                \end{equation*}
                Recordamos que:
                \begin{equation*}
                    Q_k = \langle x,y \mid x^{2k} = 1, y^2 = x^k, yx=x^{-1}y \rangle 
                \end{equation*}
                Y observamos que:
                \begin{itemize}
                    \item $r^{2k} = {(r^k)}^{2} = 1$.
                    \item $s^2 = 1 = r^k$.
                    \item $sr = r^{-1}s$.
                \end{itemize}
                El Teorema de Dyck nos da un homomorfismo $f:Q_k\to D_k$ de forma que:
                \begin{equation*}
                    f(x) = r \qquad f(y) = s
                \end{equation*}
                Como además $D_k = \langle r,s \rangle $, tenemos que $f$ es un epimorfismo. Si aplicamos el Primer Teorema de Isomorfía sobre $f$, obtenemos que:
                \begin{equation*}
                    Q_k / \ker(f) \cong D_k
                \end{equation*}
                En primer lugar, observamos que $Q_k$ no es abeliano, ya que $D_k$ no lo es y cualquier cociente de un grupo abeliano es abeliano. Veamos que: 
                \begin{equation*}
                    2k \leq |Q_k| \leq 4k
                \end{equation*}
                \begin{itemize}
                    \item Para la desigualdad de la izquierda, sabemos que $|D_k| = 2k$, y por ser $[Q_k : \ker(f)] = |D_k|$, sabemos que $2k$ divide a $|Q_k|$, de donde $|Q_k| \geq 2k$.
                    \item Para la otra, si tomamos $H = \langle x \rangle $, como $x^{2k} = 1$, tendremos que $\red{|H| \leq 2k}$. Como además también tenemos que:
                        \begin{align*}
                            yxy^{-1} &= x^{-1} \in  H \\
                            y^{-1}xy &= x^{-1} \in H
                        \end{align*}
                        Tendremos que $H\lhd Q_k$, de donde al considerar el cociente, tendremos que:
                        \begin{equation*}
                            Q_k/H = \langle yH \rangle 
                        \end{equation*}
                        De esta forma, como:
                        \begin{equation*}
                            {(yH)}^{2} = y^2H = x^k H = H
                        \end{equation*}
                        Deducimos que $O(yH) \leq 2$, por lo que:
                        \begin{equation*}
                            \red{|Q_k/H| \leq 2}
                        \end{equation*}
                        De donde tenemos que:
                        \begin{equation*}
                            |Q_k| = |Q_k/H| |H| \leq 2k + 2k = 4k
                        \end{equation*}
                \end{itemize}

                Si suponemos ahora que $k = 2t + 1$ para cierto $t\in \mathbb{N}$, si consideramos el cíclico de orden 4:
                \begin{equation*}
                    C_4 = \langle a \mid a^4 = 1 \rangle 
                \end{equation*}
                Como:
                \begin{itemize}
                    \item ${(a^2)}^{2k} = {(a^4)}^{k} = 1$.
                    \item $a^2 = a^{4t+2} = a^{2k} = {(a^2)}^{k}$.
                    \item $aa^2 = a^3 = a^2a = {(a^2)}^{-1}a$.
                \end{itemize}
                Podemos aplicar el Teorema de Dyck, obteniendo un homomorfismo $f:Q_k\to C_4$ de forma que:
                \begin{equation*}
                    f(x) = a^2 \qquad f(y) = a
                \end{equation*}
                Que de hecho es un epimorfismo, ya que $C_4 = \langle a \rangle $. Al igual que antes, tendremos que:
                \begin{equation*}
                    Q_k/\ker(f)\cong C_4
                \end{equation*}
                De donde 4 divide a $|Q_k|$. Como también teníamos antes que $2k$ dividía a $|Q_k|$, tenemos que $\mcm(2k,4)$ divide a $|Q_k|$ y como $k$ era impar, tenemos que:
                \begin{equation*}
                    \mcm(2k,4) = 4k
                \end{equation*}
                De donde $4k$ divide a $|Q_k|$, por lo que $4k\leq |Q_k|$, y la otra desigualdad que teníamos ya probada nos da la igualdad.
            \end{proof}
    \end{itemize}
\end{ejemplo}

\begin{ejemplo}
    Grupos no abelianos de orden 12 conocíamos:
    \begin{itemize}
        \item $A_4$.
        \item $D_6$.
    \end{itemize}
    Y ahora conocemos $Q_3$. Próximamente veremos que estos grupos son los únicos grupos que existen de orden 12, salvo isomorfismo.
\end{ejemplo}

\noindent
Antes de proceder con la definición del producto semidirecto, daremos una útil Proposición que nos será de ayuda a la hora de construir productos semidirectos:
\begin{prop}
    Sea $p$ un primo, entonces:
    \begin{equation*}
        Aut(C_p) \cong C_{p-1}
    \end{equation*}
    \begin{proof}
        Sea $\Phi:C_{q-1}\to Aut(C_q)$ dada por:
        \begin{equation*}
            \Phi(k)(a) = ka \qquad \forall k\in C_{q-1}, \forall a\in C_q
        \end{equation*}
    \end{proof}
\end{prop}

\begin{definicion}[Producto semidirecto]
    Sean $K$, $H$ dos grupos y dado un homomorfismo $\theta:H\to Aut(K)$, sobre el producto cartesiano de $K$ por $H$ podemos definir la operación:
    \begin{equation*}
        (k_1,h_1)(k_2,h_2) = (k_1\theta(h_1)(k_2), h_1h_2) \qquad \forall (k_1,h_1),(k_2,h_2) \in K\times H
    \end{equation*}
    Se verifia que $K\times H$ con esta operación tiene estructura de grupo (como ponemos de manifiesto en la siguiente Proposición), al que llamaremos \textbf{producto semidirecto de $K$ por $H$ relativo a $\theta$}, y que denotaremos por:
    \begin{equation*}
        K\rtimes_\theta H
    \end{equation*}
    Observemos que, en particular, $\theta$ es un homomorfismo de $H$ sobre $\Perm(K)$, por lo que $\theta$ nos define una acción:
    \begin{equation*}
        \theta(h,k) = \prescript{h}{}{k} \qquad \forall h\in H, k\in K
    \end{equation*}
    Por lo que será habitual escribir:
    \begin{equation*}
        (k_1,h_1)(k_2,h_2) = (k_1 \prescript{h_1}{}{k_2}, h_1h_2)
    \end{equation*}
    Y no se nos debe olvidar que $\theta(h) \in Aut(K)$ para todo $h\in H$, ya que esta propeidad es importante a la hora de ver que $K\rtimes_\theta H$ es un grupo.
\end{definicion}

\begin{prop}
    Se verifica que $K\times H$ con la operación definida en la definición anterior es un grupo.
    \begin{proof}
        Veamos que efectivamente cumple con todas las conddiciones de ser un grupo:
        \begin{itemize}
            \item Para la propiedad asociativa, si $a,b,c\in K$ y $x,y,z\in H$:
                \begin{align*}
                    ((a,x)(b,y))(c,z) &= (a\prescript{x}{}{b},xy)(c,z) = (a\prescript{x}{}{b}\prescript{xy}{}{c}, xyz) \\
                    (a,x)((b,y)(c,z)) &= (a,x)(b\prescript{y}{}{c}, yz) = (a\prescript{x}{}{(b\prescript{y}{}{c})}, xyz) = (a\prescript{x}{}{b}\prescript{xy}{}{c},xyz)
                \end{align*}
            \item El elemento $(1,1)$ es el neutro:
                \begin{align*}
                    (k,h)(1,1) &= (k\prescript{h}{}{1},h) = (k,h) \\
                    (1,1)(k,h) &= (1, \prescript{1}{}{k},h) = (k,h) \\
                               &\forall (k,h)\in K\times H
                \end{align*}
            \item Para el inverso, dado $(k,h)\in K\times H$, el inverso será:
                \begin{equation*}
                    {(k,h)}^{-1} = \left(\prescript{h^{-1}}{}{k^{-1}},h^{-1}\right)
                \end{equation*}
                Ya que:
                \begin{equation*}
                    (k,h)\left(\prescript{h^{-1}}{}{k^{-1}},h^{-1}\right) = \left(k\prescript{h}{}{\left(\prescript{h^{-1}}{}{k^{-1}}\right)}, hh^{-1}\right) = \left(k\prescript{hh^{-1}}{}{k^{-1}}, 1\right) = (kk^{-1},1) = (1,1)
                \end{equation*}
        \end{itemize}
    \end{proof}
\end{prop}

\begin{ejemplo}
    Veamos:
    \begin{itemize}
        \item Si $\theta(k) = id_H$ para todo $k\in K$, entonces $K\rtimes_\theta H = K\times H$, ya que:
            \begin{equation*}
                (k_1,h_1)(k_2,h_2) = (k_1 \prescript{h_1}{}{k_2}, h_1h_2) = (k_1k_2, h_1h_1) \qquad \forall (k_1,h_1), (k_2,h_2) \in K\times H
            \end{equation*}
        \item Como ejemplo útil del producto semidirecto, veamos si somos capaces de escribir $S_3$ como producto semidirecto de $C_3$ por $C_2$ relativo a algún homomorfismo $\theta$:
            \begin{equation*}
                S_3\cong C_3\rtimes_{\theta} C_2
            \end{equation*}
        \item Veamos cómo escribir $S_3$ como producto semidirecto:
            \begin{equation*}
                S_3 \cong C_3 \rtimes_{\theta} C_2
            \end{equation*}
            Tenemos los elementos:
            \begin{equation*}
                C_3\times C_2 = \{(x,y) \mid x\in C_3, y\in C_2\}
            \end{equation*}
            Buscamos qué homomorfismo $\theta:C_2\to Aut(C_3)$ hemos de coger. Será:
            \begin{equation*}
                \theta(y)(x) = x^{-1} \qquad \forall y\in C_2, \forall x\in C_3
            \end{equation*}
            Ya que $Aut(C_3)\cong C_2 = \{1,x\}$. Los elementos serán:
            \begin{equation*}
                C_3 \rtimes_\theta C_2 = \langle x,y\mid x^3=1, y^2 = 1, algo \rangle 
            \end{equation*}
            Con $|C_3\rtimes_\theta C_2| = 6$. Los grupos que conocemos de orden 6 son $S_3$ y $C_6$, que podemos distinguir en función de si el grupo es abeliano o no. Veamos que no lo es:
            \begin{align*}
                (x,y^2)(1,y) &= (x^2 \prescript{y}{}{1}, y^2) = (x^2,1) \\
                (1,y)(x^2,y) &= (1\prescript{y}{}{x^2}, y^2) = (x, 1)
            \end{align*}
            Como $x\neq x^2$, no es conmutativo, por lo que $C_3\rtimes_\theta C_2\cong D_3$. Por tanto, completamos la presentación pensando en la de $D_3$:
            \begin{equation*}
                C_3 \rtimes_\theta C_2 = \langle x,y\mid x^3=1, y^2 = 1, xy = yx^{-1} \rangle 
            \end{equation*}
            En definitiva, el único producto semidirecto de dos grupos de orden 6 es $S_3$.
        \item Veamos que $Q_3 = C_3\rtimes_\theta C_4$. De nuevo, el homomorfismo a considerar será:
            \Func{\theta}{C_4}{Aut(C_3)}{y}{\theta(y)(x)=x^{-1}}
            Tendremos:
            \begin{equation*}
                C_3\rtimes_\theta C_4 = \langle x,y\mid x^3=1, y^4 = 1, \prescript{y}{}{x}= x^{-1} \rangle 
            \end{equation*}
            Y queremos ver el isomorfismo con:
            \begin{equation*}
                Q_3 = \langle c,d\mid c^6=1, d^2 = c^3, dc=c^{-1}d \rangle 
            \end{equation*}
            % // TODO: Nos ha dado mal el ejemplo, se queda incompleto
            % Si cogemos:
            % \begin{equation*}
            %     c = (x^2,y) \qquad d = (1,y)
            % \end{equation*}
            % Vemos que:
            % \begin{itemize}
            %     \item $c^6 = 1$.
            %     \item $d^2 = c^3$.
            %     \item $dc = c^{-1}d$, que equivale a ver que $cdc = d$. Para ello:
            %         \begin{equation*}
            %             (x^2,y)(1,y)(x^2,y) = (x^2,y)\left(1\prescript{y}{}{x^2}, y^2\right) = (x^2,y)(x,y^2) = (x^2\prescript{y}{}{x},y^3) = (x,y^3)
            %         \end{equation*}
            %         \begin{equation*}
            %             (x^2,1)(1,y)(x^2,1) = (x^2,y)(x^2,1) = (x^2\prescript{y}{}{x^2}, y) = (x^3,y) = (1,y)
            %         \end{equation*}
            % \end{itemize}
        \item Si $n\geq 3$, si consideramos $\theta:C_2\to Aut(C_n)$ dado por:
            \begin{equation*}
                \theta(y)(x) = x^{-1} \qquad \forall x\in C_2, y\in C_n
            \end{equation*}
            Tendremos que $C_n\rtimes_\theta C_2 \cong D_n$.
    \end{itemize}
\end{ejemplo}

% // TODO: No hemos visto que Aut(C_q) \cong C_{q-1} para q primo, ver dónde lo añado

\begin{definicion}
    En el producto semidirecto, definimos:
    \begin{figure}[H]
        \centering
        \shorthandoff{""}
\begin{tikzcd}
K \arrow[r, "\lambda_1"] & K\rtimes H \arrow[d, "\pi"] & H \arrow[l, "\lambda_2"'] \\
                         & H                           &                          
\end{tikzcd}
        \shorthandon{""}
    \end{figure}
    Por:
    \begin{align*}
        \lm_1(k) &= (k,1)  \\
        \lm_2(h) &= (1,h)  \\
        \pi(k,h) &= h
    \end{align*}
\end{definicion}

\begin{prop}
    Se verifica que:
    \begin{enumerate}
        \item $\lm_1, \lm_2, \pi$ son homomorfismos de grupos.
        \item $\pi \lm_1$ es trivial.
        \item $\pi\lm_2 = id_H$.
    \end{enumerate}
\end{prop}

\noindent
De forma análoga a la propiedad universal del producto directo, podemos tener la propiedad universal para el producto semidirecto.

\noindent
La siguiente Proposición nos será de utilidad para clasificar grupos haciéndolos isomorfoso a un producto semidirecto, a partir del orden.

\begin{teo}
    Sea $G$ un grupo y $K,H<G$ con $K\lhd G$, $KH = G$ y $K\cap H = \{1\}$, sea $\theta:H\to Aut(K)$ un homomorfismo que nos da la acción $ac:H\times K\to K$ por conjugación\footnote{La condición $K\lhd G$ nos dice que $\theta$ está bien definida}:
    \begin{equation*}
        \theta(h)(k) = hkh^{-1} \qquad \forall h\in H,\forall  k\in K
    \end{equation*}
    Entonces, $K\rtimes_\theta H \cong G$.
    \begin{proof}
        Definiremos la aplicación $f:K\rtimes_\theta H \to G$ dada por:
        \begin{equation*}
            f(k,h) = kh \qquad \forall k\in K, \forall h\in H
        \end{equation*}
        Veamos que es un isomorfismo:
        \begin{itemize}
            \item $f$ es sobreyectiva, ya que $G = KH$, de donde cualquier elemento $g\in G$ se escribe como $g = kh$  para ciertos $k\in K$, $h\in H$.
            \item Para la inyectividad, si $f(k_1,h_1) = f(k_2,h_2)$, entonces $k_1h_1 = k_2h_2$, de donde $k_2^{-1}k_1=h_2h_1^{-1}$:
                \begin{itemize}
                    \item $k_2^{-1}k_1\in K$.
                    \item $h_2h_1^{-1}\in H$.
                \end{itemize}
                Y como $H\cap K = \{1\}$, concluimos que $k_1 = k_2$ y $h_1 = h_2$, de donde $f$ es inyectiva.
            \item Para ver que $f$ es un homomorfismo, si $(k_1,h_1),(k_2,h_2)\in K\rtimes_\theta H$:
                \begin{multline*}
                    f((k_1,h_1)(k_2,h_2)) = f(k_1\prescript{h_1}{}{k_2},h_1h_2) = f(k_1h_1k_2h_1^{-1},h_1h_2) \\ = k_1h_1k_2h_1^{-1}h_1h_2 = k_1h_1k_2h_2 = f(k_1,h_1)f(k_2,h_2)
                \end{multline*}
        \end{itemize}
    \end{proof}
\end{teo}

\begin{definicion}
    Si $G$ verifica las condiciones de la Proposición anterior, decimos que $G$ es producto semidirecto interno de $K$ y $H$.
\end{definicion}

\begin{definicion}[Complemento de un subgrupo]
    Si $K<G$, un subgrupo $H<G$ se llama complemento para $K$ en $G$ si $G = KH$ con $K\cap H = \{1\}$.
\end{definicion}

\begin{observacion}
    Con esta última definición, tendremos que $G$ será un producto semidirecto interno de dos subgrupos propios si y solo si algún subgrupo normal propio tiene un complemento.
\end{observacion}

\begin{ejemplo}
    Esto último no siempre será posible. Por ejemplo, si $G$ es simple, no tendrá subgrupos normales propios, por lo que no será producto semidirecto interno de dos subgrupos.\\

    \noindent
    Si $G$ es un grupo que sí tiene subgrupos normales propios, tampoco somos capaces siempre de poner como un producto semidirecto. Por ejemplo, $Q_2$ no es un producto semidirecto interno de subgrupos propios. Si recordamos su diagrama de Hasse: % // TODO: Poner
    \begin{gather*}
        \langle i \rangle \cap \langle j \rangle  = \{1, -1\} \\
        \langle i \rangle \cap \langle k \rangle  = \{1, -1\} \\
        \langle j \rangle \cap \langle k \rangle  = \{1, -1\} 
    \end{gather*}
    Dado un subgrupo normal, no seremos capaces de complementarlo con otro.
\end{ejemplo}

\begin{ejemplo}
    Para cualquier grupo $K$, si tomamos $H = Aut(K)$ y $\theta = 1_{Aut(K)}$, si tomamos:
    \begin{equation*}
        K\rtimes_\theta Aut(K) = Hol(K)
    \end{equation*}
    Al que llamaremos grupo holomorfo de $K$.\\

    \noindent
    Por ejemplo:
    \begin{equation*}
        Hol(\mathbb{Z}_2\times \mathbb{Z}_2) = S_4
    \end{equation*}
\end{ejemplo}

% \begin{ejemplo}
%     Como ejemplos de aplicaciones inmediantas del último Teorema, tenemos:
%     \begin{itemize}
%         \item Sea $G = S_n$, $K = A_n \lhd S_n$ y $H = \langle (1\ 2) \rangle \cong \mathbb{Z}_2 $, entonces:
%             \begin{equation*}
%                 A_n H = S_n \qquad A_n\cap H = \{1\}
%             \end{equation*}
%             Por lo que estamos en las condinciones de aplicar el Teorema, con lo que:
%             \begin{equation*}
%                 S_n\cong A_n \rtimes \mathbb{Z}_2
%             \end{equation*}
%         \item En $G = S_4$, si tomamos $K = V\lhd S_4$, $H = S_3 = Stab_{S_4}(V)$, tenemos que:
%             \begin{equation*}
%                 VH = S_4 \qquad V\cap H = \{1\}
%             \end{equation*}
%             Por lo que:
%             \begin{equation*}
%                 S_4\cong V\rtimes S_3
%             \end{equation*}
%         \item Sea $G = A_4$, $K = V\lhd A_4$, $H = \langle (1\ 2\ 3) \rangle $, tenemos que:
%             \begin{equation*}
%                 A_4\cong V\rtimes H
%             \end{equation*}
%     \end{itemize}
% \end{ejemplo}

% \subsection{Grupos de orden $pq$}
% Veamos ahora cómo son los grupos $G$ con $|G| = pq$ con $p<q$ y $p,q$ primos. En dicho caso, tendremos:
% \begin{itemize}
%     \item $P\in Syl_p(G)$
%     \item $Q\in Syl_q(G)$
% \end{itemize}
% \begin{equation*}
%     \left.\begin{array}{r}
%         n_q \mid p \\
%         n_q\equiv 1 \mod p
%     \end{array}\right\} \Longrightarrow n_q \in \{1,p\}
% \end{equation*}
% Pero como $p<q$, no puede ser $n_q = p$, ya que entonces $p\equiv 1 \mod q$. De esta forma, sabemos que $Q$ es el único $q-$subgrupo de $G$, por lo que $Q\lhd G$. De esta forma, tenemos que:
% \begin{equation*}
%     Q\cap P = \{1\} \qquad QP = G
% \end{equation*}
% Por tanto, vamos a poder escribir siempre:
% \begin{equation*}
%     G \cong Q\rtimes P
% \end{equation*}
% Como $|Q| = q$, tenemos que $Q\cong C_q$. De la misma forma, como $|P| = p$, tenemos que $P\cong C_p$. En definitiva, tenemos que $n_q = 1$. Ahora:
% \begin{equation*}
%     \left.\begin{array}{r}
%         n_p \mid q \\
%         n_p \equiv 1 \mod p
%     \end{array}\right\} \Longrightarrow n_p \in \{1,q\}
% \end{equation*}
% \begin{itemize}
%     \item Si $n_q = 1 = n_p$, entonces tendremos que $P, Q\lhd G$, por lo que:
%         \begin{equation*}
%             G\cong Q\rtimes P \cong Q\times P \cong C_q \times C_p \cong C_{}
%         \end{equation*}
%     \item Si $n_q = 1$ y $n_p = q$, entonces $p\mid q-1$. Si buscamos una accion:
%         \begin{equation*}
%             C_p\to Aut(C_q) \cong U(C_q) \cong C_{q-1}
%         \end{equation*}
%         Como $p\mid q-1$, sabemos que existe un único subgrupo cíclico $\langle \alpha \rangle \subseteq Aut(C_q) $ de orden $p$.\\

%         \noindent
%         Una acción $\theta:C_p\to Aut(C_q)$ vendrá dado por:
%         \begin{equation*}
%             \theta(y) = \alpha^j
%         \end{equation*}
%         Podemos definir $\theta_i$ que a cada $y_i \longmapsto \alpha^i$, para $i \in \{0,\ldots,p-1\}$. Para $i = 0$:
%         \begin{equation*}
%             \theta_0(y_0) = 1
%         \end{equation*}
%         Luego tenemos $G\cong C_{pq}$. Para el resto de los casos, como son todas isomorfas, tomamos $\theta(y_i) = \alpha$.\\

%         \noindent
%         En conclusión:
%         \begin{equation*}
%             G\cong C_q\rtimes C_p \cong \langle x,y\mid x^q = 1, y^p = 1, yxy^{-1} = \alpha(x) \rangle 
%         \end{equation*}
%         Con lo que así se describen todos los grupos $G$ no abelianos de orden $pq$.
% \end{itemize}

% \begin{ejemplo}
%     En el caso $p = 2$, para $C_{2q}$:
%     \begin{equation*}
%         C_q\rtimes C_2 = \langle x,y\mid x^q = 1, y^2 = 1, yxy^{-1} = \alpha(x) \rangle 
%     \end{equation*}
%     Pero:
%     \begin{equation*}
%         \alpha\in Aut(C_q) \qquad O(\alpha) = 2
%     \end{equation*}
%     Y morfismos de $C_q$ en $C_q$ de orden 2 solo hay uno:
%     \begin{equation*}
%         x \longmapsto x^{-1}
%     \end{equation*}
%     Por tanto:
%     \begin{equation*}
%         C_q\rtimes C_2 = \langle x,y \mid x^q = 1, y^2 = 1, yxy^{-1}= x^{-1} \rangle  = D_q
%     \end{equation*}
%     Por lo que grupos $G$ con $|G| = 2q$ tenemos que $G$ es abeliano o que es $D_q$.
% \end{ejemplo}

% \noindent
% Repasando lo que sabemos clasificar:
% \begin{align*}
%     |G| = 6 &\longmapsto C_6, \qquad D_3\cong S_3 \\
%     |G| = 10 &\longmapsto C_{10}, \qquad D_5 \\
%     |G| = 14 &\longmapsto C_{14}, \qquad D_7 \\
%     |G| = 15 &\longmapsto C_{15}, ?
% \end{align*}

% Vamos a ver el caso que $|G| = 15$, con lo que tenemos haciendo cuentas que $n_3 = 1 = n_5$, por lo que concluimos que grupos de orden 15 solo tenemos $C_{15}$. Este razonamiento puede seguirser para $|G| = n$ para $n\geq 15$, si el lector desea hacerlo.

% \subsection{Grupos de orden 12}
% Sea $G$ un grupo de orden $|G| = 12 = 2^2 \cdot 3$.\\

% \noindent
% Sabemos que grupos abelianos de orden 12 tenemos:
% \begin{equation*}
%     C_2\times C_2\times C_3 \cong C_2\times C_6 \qquad C_4\times C_3 \cong C_{12}
% \end{equation*}

% \noindent
% Supuesto que $G$ no es abeliano:
% \begin{gather*}
%     \left.\begin{array}{r}
%             n_2 \mid 3 \\
%             n_2 \equiv 1 \mod 2
%     \end{array}\right\} \Longrightarrow n_2 \in \{1,3\} \\
%     \left.\begin{array}{r}
%         n_3 \mid 4 \\
%         n_3 \equiv 1 \mod 4
%     \end{array}\right\} \Longrightarrow n_3 \in \{1,4\}
% \end{gather*}

% \begin{itemize}
%     \item Supongamos que $n_4 = 3$ y $n_3 = 4$, entonces tendremos:
%         \begin{align*}
%             &P_1,P_2,P_3,P_4 \in Syl_3(G) \qquad |P_i| = 3 \\
%             &Q_1,Q_2,Q_3 \in Syl_2(G) \qquad |Q_i| = 4
%         \end{align*}
%         Por lo que sacamos $8$ elementos distintos de orden 3 y 9 elementos de orden 2 o 4, con lo que este caso es imposible.
%     \item Si $n_2 = 1$ o $n_3 = 1$, tendremos en cualquier caso de la existencia de un $p-$subgrupo de Sylow ($p\in \{2,3\}$) $K\lhd G$. Si consideramos su complemento, $H<G$, tendremos que:
%         \begin{equation*}
%             G\cong K\rtimes_\theta H
%         \end{equation*}
%         Si suponemos que $K\in Syl_3(G)$ y $H\in Syl_2(G)$ (en otro caso es análogo), tendremos entonces que $K\cong C_3$ y $|H| = 4$, por lo que $H \cong C_4$ o $H\cong C_2\times C_2$
%     \item Si $n_2 = 1 = n_3$, entonces tenemos dos subgrupos normales, con lo que:
%         \begin{equation*}
%             G\cong C_2\times C_6 \quad \text{o} \quad G\cong C_{12}
%         \end{equation*}
%         El primer caso si $H = C_2\times C_2$ y el segundo si $H=C_4$, por lo que volvemos al caso abeliano.
%     \item Si $n_3 = 1$ y $n_2 = 3$, tenemos entonces que:
%         \begin{equation*}
%             G\cong K\rtimes H \cong \left\{\begin{array}{l}
%                 C_3\rtimes C_4 \\
%                 C_3 \rtimes C_2 \times C_2
%             \end{array}\right.
%         \end{equation*}
%         Y vendrá por una acción:
%         \begin{align*}
%             &\theta:C_4 \to Aut(C_3) \\
%             &\theta:C_2\times C_2 \to Aut(C_3)
%         \end{align*}
%         Alguno de ellos. sin embargo, como $Aut(C_3)\cong C_2 = \{1,x^{-1}\}$
% \end{itemize}

% \begin{itemize}
%     \item En $C_3\rtimes C_4$ para la acción $\prescript{x}{}{y} = x^{-1}$, tenemos que:
%         \begin{equation*}
%             C_3 \rtimes C_4 = \langle x,y\mid x^3=1, y^4 = 1, yxy^{-1}=x^{-1} \rangle 
%         \end{equation*}
%     \item En $C_3\rtimes (C_2\times C_2)$, los automorfismos de la forma:
%         \begin{equation*}
%             C_2\times C_2\to Aut(C_3)
%         \end{equation*}
%         Solo tenemos uno no trivial, que es ($y,x$ son los generadores):
%         \begin{align*}
%             &\theta:C_2\times C_2 \to Aut(C_3) \\
%             &y\longmapsto \alpha \\
%             &x\longmapsto 1
%         \end{align*}
%         Tendremos que:
%         \begin{equation*}
%             C_3\rtimes_\theta(C_2\times C_2) = \langle x,y,z \mid x^3=1, y^2 = z^2 = 1, yxy^{-1}=x^{-1},zxz^{-1}=x, yzy^{-1}=zy \rangle 
%         \end{equation*}
%         Que es isomorfo a $D_6\cong D_3\times C_2$, tomando $r =xy$ y $s = yz$ % // Quizas este cambio está mal
% \end{itemize}

% \begin{itemize}
%     \item En el caso $n_3 = 4$ y $n_2 = 1$, hay un ejercicio en la relación de $p-$grupos que decía que si un grupo de orden 12 tiene más de $3-$subgrupos de Sylow, entonces $G\cong A_4$. Para ello:
%         \begin{gather*}
%             \phi:G\to Perm(Syl_3(G)) \cong S_4 \\
%             G/\ker(\phi) \cong G \cong Im(\phi) \subseteq S_4
%         \end{gather*}
%         Por lo que $G<S_4$ con $|G| = 12$, luego ha de ser $G\cong A_4$. Tendremos ahora:
%         \begin{equation*}
%             G \cong H\rtimes K \cong \left\{\begin{array}{l}
%                 C_4\rtimes C_3 \\
%                 (C_2\times C_2) \rtimes C_3
%             \end{array}\right.
%         \end{equation*}
%         \begin{itemize}
%             \item Si $C_4\rtimes C_3$, tendremos que la única acción no trivial es
%                 \Func{\theta}{C_3}{Aut(C_4)\cong C_2}{y}{y^{-1}}
%                 Con orden 2. Sin embargo, como su orden ha de dividir a $|C_3| = 3$, el morfismo no divide a 3, luego no hay nada no trivial ahí: todos los automorfismos son triviales. En dicho caso, tenemos:
%                 \begin{equation*}
%                     C_4\rtimes C_3 = C_4\times C_3 = C_{12}
%                 \end{equation*}
%             \item En el caso $(C_2\times C_2)\rtimes C_3$, buscamos una acción $C_3\to Aut(C_2\times C_2)\cong S_3$, por lo que tendremos dos automorfismos no triviales de $C_2\times C_2$ de orden 3.
%                 \begin{gather*}
%                     \theta_1 : x \longmapsto \alpha \\
%                     \alpha \left\{\begin{array}{l}
%                         y \longmapsto z \\
%                         z \longmapsto yz
%                     \end{array}\right.
%                 \end{gather*}
%                 \begin{gather*}
%                     \theta_2 : x \longmapsto \alpha^2 \\
%                     \alpha^2\left\{\begin{array}{l}
%                         y\longmapsto yz \\
%                         z \longmapsto y
%                     \end{array}\right.
%                 \end{gather*}
%                 Que podemos pensarlo con:
%                 \begin{equation*}
%                     (1,0), (0,1) \longmapsto (0,1), (1,1)
%                 \end{equation*}
%                 Por lo que:
%                 \begin{multline*}
%                     (C_2\times C_2)\rtimes_{\theta_1} C_3 \\ = \langle x,y,z\mid x^3=1, y^2 = z^2 = 1, xyx^{-1}=z, xzx^{-1}=zy, yz = zy \rangle  \cong A_4
%                 \end{multline*}
%                 Este último isomorfismo no sale fácil. Ver que un grupo es $A_4$ suele verse siempre viendo que tiene más de un $3-$subgrupo de Sylow.
%         \end{itemize}
% \end{itemize}

% \subsection{Grupos de orden 8}
% \noindent
% Sea $G$ un grupo de orden 8, no vamos a tener $p-$subgrupos de Sylow, porque el único es el total. Los grupos abelianos son:
% \begin{equation*}
%     C_2\times C_2\times C_2 \qquad C_2\times C_4 \qquad C_8
% \end{equation*}
% \subsubsection{No abelianos}
% \noindent
% Si $G$ es un grupo no abeliano de orden 8, entonces no existen elementos en $G$ de orden 8, ya que entonces $G$ sería cíclico, luego abeliano. Por tanto, los elementos de $G$ tendrán orden 2 o 4. Tampoco pueden ser todos de orden 2, puesto que $G$ también sería abeliano, por lo que $\exists a\in G$ de forma que $O(a) = 4$. Consideramos:
% \begin{equation*}
%     H = \langle a \rangle  = \{1,a,a^2,a^3\}
% \end{equation*}
% Tenemos que $[G:H] = 2$, por lo que $H\lhd G$. Dado $b\in G\setminus H$, tendremos dos clases en el cociente:
% \begin{equation*}
%     G = H\cup Hb 
% \end{equation*}
% De esta forma, podemos describir $G$ como:
% \begin{equation*}
%     G = \{1,a,a^2,a^3,b, ab, a^2b, a^3b\}
% \end{equation*}
% Si consideramos $b^2$, veamos en qué clase está. Supuesto que $b^2\in Hb$, entonces:
% \begin{itemize}
%     \item Puede ser que $b^2 = b \Longrightarrow b = 1$.
%     \item Puede ser $b^2 = ab \Longrightarrow b = a$.
%     \item Puede ser $b^2 = a^2b \Longrightarrow b = a^2$.
%     \item Puede ser $b^2 = a^3b \Longrightarrow b = a^3$.
% \end{itemize}
% Todas imposibles, por lo que $b^2 \in H = \{1,a,a^2, a^3\}$, de donde:
% \begin{itemize}
%     \item Si $b^2 = a$, entonces $O(b^2) = O(a) \Longrightarrow O(b) = 8$, imposible.
%     \item Si $b^2 = a^3$, entonces $O(b^2) = O(a^3) = O(a)$, imposible.
%     \item Si $b^2 = 1$, veamos que $ba = a^3b$. Como $H\lhd G$, tenemos que $bab^{-1}\in H$, pero como $O(b) = 2$, tenemos que $bab\in H$ y:
%         \begin{equation*}
%             O(bab) = O(a) = 4
%         \end{equation*}
%         De donde $bab \in \{a,a^3\}$. Si $bab = a$, entonces $G$ es abeliano, imposible, por lo que:
%         \begin{equation*}
%             bab = a^3
%         \end{equation*}
%         Por lo que en este caso tenemos:
%         \begin{equation*}
%             G = \langle a,b \mid a^4 = b^2 = 1, ba=a^3b \rangle = D_4
%         \end{equation*}
%     \item Si $b^2 = a^2$, vamos a probar la misma igualdad: $ba=a^3b$. Para ello, como $H\lhd G$, tenemos que $bab^{-1}\in H$, pero como:
%         \begin{equation*}
%             O(bab^{-1}) = O(a) = 4
%         \end{equation*}
%         Por lo que $bab^{-1}\in \{a,a^3\}$. Si $bab^{-1}=a$, entonces es abeliano, por lo que también tenemos $bab=a^3$. En este caso:
%         \begin{equation*}
%             G = \langle a,b\mid a^4 = 1, a^2 = b^2, ba = a^3b \rangle = Q_2
%         \end{equation*}
% \end{itemize}
% De esta forma, los únicos grupos no abelianos de orden 8 son:
% \begin{equation*}
%     D_4 \qquad Q_2
% \end{equation*}
% Para grupos de orden 12 podemos emplear también este mismo tipo de razonamiento.

% % Va a preguntar p-subgrupos de Sylow y torsion

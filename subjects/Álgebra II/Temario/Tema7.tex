\chapter{Clasificación de grupos de orden bajo}
\noindent
Clasificar grupos es una tarea dura y difícil, por lo que nos centraremos en grupos de orden bajo.

\section{Grupos abelianos}
Sabemos ya clasificar los grupos abelianos de orden menor o igual que 15:
\begin{equation*}
    \begin{array}{c|c}
        \text{Orden}  & \text{Grupo} \\
        \hline
        1 & \{1\} \\
        2 & C_2 \\
        3 & C_3 \\
        4 & C_2\oplus C_2, C_4 \\
        5 & C_5 \\
        6 & C_6 \\
        7 & C_7 \\
        8 & C_2\oplus C_2 \oplus C_2, C_2\oplus C_4 \oplus C_8 \\
        9 & C_9, C_3\oplus C_3 \\
        10 & C_{10} \\
        11 & C_{11} \\
        12 & C_{12}, C_6\oplus C_2 \\
        13 & C_{13} \\
        14 & C_{14} \\
        15 & C_{15} \\
        \vdots & \vdots 
    \end{array}
\end{equation*}
En general, si $|G| = p$ primo, tendremos $C_p$.

Si el grupo no es finito, nos dan la presentación y también sabemos clasificarlos. Nos centraremos por tanto en los grupos no abelianos.

\section{Producto semidirecto}
\noindent
Será una especie de producto que nos permitirá escribir muchos grupos no abelianos. Con esta herramienta podremos escribir muchos grupos no abelianos (aunque no todos).

\begin{ejemplo}
    En el Capítulo~\ref{cap:1} vimos que $Q_2 = \{\pm 1, \pm i, \pm j, \pm k\}$ es isomorfo a:
    \begin{equation*}
        Q_2^{\text{abs}} = \langle x,y\mid x^4=1, x^2=y^2, yxy^{-1} = x^{-1} \rangle 
    \end{equation*}
    Es decir, teníamos una aplicación (gracias a Teorema de Dyck) $f:Q_2\to Q_2^{\text{abs}}$ dada por:
    \begin{equation*}
        f(x) = i \qquad f(y) = j
    \end{equation*}
    Que además era un epimorfimos, porque $Q_2^{\text{abs}} = \langle i,j \rangle $. Veamos que $|Q_2^{\text{abs}}| = 8$, de una forma distinta que contar elementos:
    \begin{proof} % // TODO: Revisar esta demostración
        Como $x^4 = 1$, si consideramos $H = \langle x \rangle $, tendremos que $|H| \leq 4$. Ahora, como:
        \begin{equation*}
            yxy^{-1} = x^{-1}\in H
        \end{equation*}
        Tenemos que $H\lhd Q_2^{\text{abs}}$. Si escribimos $Q_2^{\text{abs}}$ en su partición de clases:
        \begin{equation*}
            Q_2^{\text{abs}} \cong H\cup yH
        \end{equation*}
        Ya que $y\notin H$, de donde al tomar cocientes:
        \begin{equation*}
            Q_2^{\text{abs}}/H\cong \langle yH \rangle 
        \end{equation*}
        Ahora, como:
        \begin{equation*}
            {(yH)}^{2} = y^2H = x^2H = H
        \end{equation*}
        Llegaremos a que:
        \begin{equation*}
            |Q_2^{\text{abs}}/H| \leq 2
        \end{equation*}
        Si aplicamos el Primer Teorema de Isomorfía sobre $f$:
        \begin{equation*}
            Q_2^{\text{abs}}/\ker(f) \cong Img(f) = Q_2
        \end{equation*}
        De donde $|Q_2^{\text{abs}}| = |Q_2||\ker(f)| \geq 8$. Concluimos que $|Q_2^{\text{abs}}| = 8$.
    \end{proof}
\end{ejemplo}

\noindent
El hecho de introducir $Q_2^{\text{abs}}$ en el Capítulo~\ref{cap:1} fue para ahora generalizar lo que hacíamos con $Q_2$ a todo grupo, con el producto semidirecto.

\begin{definicion}[Grupos dicíclicos]
    Para cada $k\in \mathbb{N}\setminus \{0\}$, definimos el $k-$ésimo grupo dicíclico como el grupo:
    \begin{equation*}
        Q_k = \left\langle x,y \mid x^{2k} = 1, y^2 = x^k, yxy^{-1} = x^{-1}  \right\rangle 
    \end{equation*}
\end{definicion}

\begin{ejemplo}
    Veamos que:
    \begin{itemize}
        \item Para $k=1$:
            \begin{equation*}
                Q_1 = \langle x,y\mid x^2 = 1, y^2 = x, yxy^{-1} = x \rangle 
            \end{equation*}
            Nos preguntamos qué grupo es. Si tratamos de describir los elementos, obtenemos:
            \begin{equation*}
                \{1,x,y,xy\} = \{1,y,y^2,y^3\}
            \end{equation*}
            Es decir, $Q_1\cong C_4$.
        \item Observemos que si $k=2$, obtenemos $Q_2^\text{abs}$.
        \item Para $k\geq 3$, tendremos que tiene un cociente isomorfo a $D_k$, por lo que no será abeliano. Sin embargo, podemos acotar el orden de $Q_k$:
            \begin{equation*}
                2k \leq |Q_k| \leq 4k \qquad \forall k\geq 3
            \end{equation*}
            Y si $k$ es impar, tendremos que $|Q_k| = 4k$.
            \begin{proof}
                Si recordamos al grupo diédrico de orden $k$:
                \begin{equation*}
                    D_k = \langle r,s \mid r^k = s^2 = 1, sr = r^{-1}s \rangle 
                \end{equation*}
                Tenemos que $r^{2k} = {(r^{k})}^{2} = 1$ y tenemos la primera relación. Para la segunda, tenemos que $s^2 = 1 = r^k$. Finalmente, $sr = r^{-1}s$ nos da la tercera (compruébese). Podemos aplicar el Teorema de Dyck, que nos da un homomorfismo $f:Q_k\to D_k$ de forma que: % // TODO: Hacer ese "compruébese"
                \begin{equation*}
                    f(x) = r \qquad f(y) = s
                \end{equation*}
                Además, sabemos que $f$ es un epimorfimo, ya que $D_k = \langle r,s \rangle $. Por el Primer Teorema de Isomorfía aplicado a $f$, podemos asegurar que:
                \begin{equation*}
                    Q_k/\ker(f)\cong D_k
                \end{equation*}
                Por tanto, el dicíclico de orden $k$ no será abeliano.  % // TODO: Justificar por qué (yo no lo se)

                \begin{description}
                    \item [$\geq)$] Sabemos que $|D_k| = 2k$, y por el epimorfismo anterior, sabemos que $2k$ divide a $|Q_k|$, de donde $2k\leq Q_k$.
                    \item [$\leq)$] Usando que $x^{2k} = 1$, si tomamos $H = \langle x \rangle $, tenemos que:
                        \begin{equation*}
                            |H| = |\langle x \rangle | \leq 2k
                        \end{equation*}
                        Como también tenemos que:
                        \begin{equation*}
                            yxy^{-1} = x^{-1}\in H
                        \end{equation*}
                        Tendremos que $H\lhd Q_k$, de donde al considerar el cociente, tendremos al igual que antes que:
                        \begin{equation*}
                            Q_k/H \cong \langle yH \rangle 
                        \end{equation*}
                        De esta forma:
                        \begin{equation*}
                            {(yH)}^{2} = y^2H = x^kH = H
                        \end{equation*}
                        Por lo que $y^2\notin yH$ y:
                        \begin{equation*}
                            |Q_k/H| \leq 2
                        \end{equation*}
                        De donde deduimos que:
                        \begin{equation*}
                            |Q_k| = |Q_k/H||H| \leq 4k
                        \end{equation*}
                \end{description}

                \noindent
                \underline{Suponiendo ahora que $k = 2t+1$} para cierto $t\in \mathbb{N}$, consideramos el cíclico de orden 4:
                \begin{equation*}
                    C_4 = \langle a \mid a^4 = 1 \rangle 
                \end{equation*}
                \begin{enumerate}
                    \item Tenemos que:
                        \begin{equation*}
                            {(a^{2})}^{2k} = {(a^4)}^{k} = 1
                        \end{equation*}
                    \item Además:
                        \begin{equation*}
                            {(a^{2})}^{k} = a^{2k} = a^{4t+2} = a^2
                        \end{equation*}
                    \item Finalmente:
                        \begin{equation*}
                            aa^2 = a^3 = a^2 a = {(a^{2})}^{-1}a
                        \end{equation*}
                \end{enumerate}
                Por el Teorema de Dyck, podemos construir el homomorfismo $f:Q_k\to C_4$ dado por:
                \begin{equation*}
                    f(x) = a^2 \qquad f(y) = a
                \end{equation*}
                Que de hecho es un epimorfismo, ya que $C_4 = \langle a \rangle $. Al igual que antes:
                \begin{equation*}
                    Q_k/\ker(f) \cong C_4
                \end{equation*}
                De donde llegamos a que 4 divide a $|Q_k|$. Como además $2k$ divide a $|Q_k|$, tenemos que $\mcm(2k,4)$ divide a $|Q_k|$ y, como $k$ era impar, tendremos que:
                \begin{equation*}
                    \mcm(2k,4) = 4
                \end{equation*}
                De donde $4k\leq |Q_k|$, que con la desigualdad anterior nos da la igualdad.
            \end{proof}
    \end{itemize}
\end{ejemplo}

\begin{ejemplo}
    Grupos no abelianos de orden 12 conocíamos:
    \begin{itemize}
        \item $A_4$.
        \item $D_6$.
    \end{itemize}
    Y ahora conocemos $Q_3$. Próximamente veremos que estos grupos son los únicos, salvo isomorfismo.
\end{ejemplo}

\begin{definicion}[Producto semidirecto]
    Dados dos grupos, $K$, $H$ y una acción $\theta: H \to Aut(K)$, consideramos el conjunto producto cartesiano:
    \begin{equation*}
        G = K\times H = \{(k,h)\mid k\in K, h\in H\}
    \end{equation*}
    Sobre el que definimos la siguiente operación:
    \begin{equation*}
        (k_1,h_1)(k_2,h_2) = (k_1 \prescript{h_1}{}{k_2}, h_1h_2)
    \end{equation*}
    Se verifica que $K\times H$ con esta operación tiene estructura de grupo, al que llamaremos \textbf{producto semidirecto de $K$ por $H$ relativo a $\theta$}, que denotaremos por:
    \begin{equation*}
        K \rtimes_\theta H
    \end{equation*}
\end{definicion}

\begin{teo}
    Se verifica que $K\times H$ con esta operación tiene estructura de grupo
    \begin{proof} % // TODO:
        Veamos:
        \begin{itemize}
            %\item Asociativa:
            \item El elemento $(1,1)$ es el neutro:
                \begin{align*}
                    (k,h)(1,1) &= (k\prescript{h}{}{1},h) = (k,h) \\
                    (1,1)(k,h) &= (1, \prescript{1}{}{k},h) = (k,h) \\
                               &\forall (k,h)\in K\times H
                \end{align*}
            \item Para el inverso, dado $(k,h)\in K\times H$, el inverso será:
                \begin{equation*}
                    {(k,h)}^{-1} = (\prescript{h^{-1}}{}{k^{-1}},h^{-1})
                \end{equation*}
                \begin{equation*}
                    (k,h)\left(\prescript{h^{-1}}{}{k^{-1}},h^{-1}\right) = \left(k\prescript{h}{}{\left(\prescript{h^{-1}}{}{k^{-1}}\right)}, hh^{-1}\right) = \left(k\prescript{hh^{-1}}{}{k^{-1}}, 1\right) = (kk^{-1},1) = (1,1)
                \end{equation*}
        \end{itemize}
    \end{proof}
\end{teo}

\begin{ejemplo}
    Veamos:
    \begin{itemize}
        \item Si $\theta = 1$, tenemos que el producto semidirecto coincide con el producto directo:
            \begin{equation*}
                \theta(1)(h) = \prescript{1}{}{h} = h\qquad \forall h\in H
            \end{equation*}
            De donde:
            \begin{equation*}
                (k_1,h_1)(k_2,h_2) = (k_1\prescript{h_1}{}{k_2}, h_1h_2) = (k_1k_2,h_1h_2) \qquad \forall (k_1,h_1),(k_2,h_2)\in K\times H
            \end{equation*}
        \item Veamos cómo escribir $S_3$ como producto semidirecto:
            \begin{equation*}
                S_3 \cong C_3 \rtimes_{\theta} C_2
            \end{equation*}
            Tenemos los elementos:
            \begin{equation*}
                C_3\times C_2 = \{(x,y) \mid x\in C_3, y\in C_2\}
            \end{equation*}
            Buscamos qué homomorfismo $\theta:C_2\to Aut(C_3)$ hemos de coger. Será:
            \begin{equation*}
                \theta(y)(x) = x^{-1} \qquad \forall y\in C_2, \forall x\in C_3
            \end{equation*}
            Ya que $Aut(C_3)\cong C_2 = \{1,x\}$. Los elementos serán:
            \begin{equation*}
                C_3 \rtimes_\theta C_2 = \langle x,y\mid x^3=1, y^2 = 1, algo \rangle 
            \end{equation*}
            Con $|C_3\rtimes_\theta C_2| = 6$. Los grupos que conocemos de orden 6 son $S_3$ y $C_6$, que podemos distinguir en función de si el grupo es abeliano o no. Veamos que no lo es:
            \begin{align*}
                (x^2,y)(1,y) &= (x^2 \prescript{y}{}{1}, y^2) = (x^2,1) \\
                (1,y)(x^2,y) &= (1\prescript{y}{}{x^2}, y^2) = (x, 1)
            \end{align*}
            Como $x\neq x^2$, no es conmutativo, por lo que $C_3\rtimes_\theta C_2\cong D_3$. Por tanto, completamos la presentación pensando en la de $D_3$:
            \begin{equation*}
                C_3 \rtimes_\theta C_2 = \langle x,y\mid x^3=1, y^2 = 1, xy = yx^{-1} \rangle 
            \end{equation*}
            En definitiva, el único producto semidirecto de dos grupos de orden 6 es $S_3$.
        \item Veamos que $Q_3 = C_3\rtimes_\theta C_4$. De nuevo, el homomorfismo a considerar será:
            \Func{\theta}{C_4}{Aut(C_3)}{y}{\theta(y)(x)=x^{-1}}
            Tendremos:
            \begin{equation*}
                C_3\rtimes_\theta C_4 = \langle x,y\mid x^3=1, y^4 = 1, \prescript{y}{}{x}= x^{-1} \rangle 
            \end{equation*}
            Y queremos ver el isomorfismo con:
            \begin{equation*}
                Q_3 = \langle c,d\mid c^6=1, d^2 = c^3, dc=c^{-1}d \rangle 
            \end{equation*}
            % // TODO: Nos ha dado mal el ejemplo, se queda incompleto
            % Si cogemos:
            % \begin{equation*}
            %     c = (x^2,y) \qquad d = (1,y)
            % \end{equation*}
            % Vemos que:
            % \begin{itemize}
            %     \item $c^6 = 1$.
            %     \item $d^2 = c^3$.
            %     \item $dc = c^{-1}d$, que equivale a ver que $cdc = d$. Para ello:
            %         \begin{equation*}
            %             (x^2,y)(1,y)(x^2,y) = (x^2,y)\left(1\prescript{y}{}{x^2}, y^2\right) = (x^2,y)(x,y^2) = (x^2\prescript{y}{}{x},y^3) = (x,y^3)
            %         \end{equation*}
            %         \begin{equation*}
            %             (x^2,1)(1,y)(x^2,1) = (x^2,y)(x^2,1) = (x^2\prescript{y}{}{x^2}, y) = (x^3,y) = (1,y)
            %         \end{equation*}
            % \end{itemize}
        \item Si $n\geq 3$, si consideramos $\theta:C_2\to Aut(C_n)$ dado por:
            \begin{equation*}
                \theta(y)(x) = x^{-1} \qquad \forall x\in C_2, y\in C_n
            \end{equation*}
            Tendremos que $C_n\rtimes_\theta C_2 \cong D_n$.
    \end{itemize}
\end{ejemplo}

% // TODO: No hemos visto que Aut(C_q) \cong C_{q-1} para q primo, ver dónde lo añado

\begin{definicion}
    En el producto semidirecto, definimos:
    \begin{figure}[H]
        \centering
        \shorthandoff{""}
\begin{tikzcd}
K \arrow[r, "\lambda_1"] & K\rtimes H \arrow[d, "\pi"] & H \arrow[l, "\lambda_2"'] \\
                         & H                           &                          
\end{tikzcd}
        \shorthandon{""}
    \end{figure}
    Por:
    \begin{align*}
        \lm_1(k) &= (k,1)  \\
        \lm_2(h) &= (1,h)  \\
        \pi(k,h) &= h
    \end{align*}
\end{definicion}

\begin{prop}
    Se verifica que:
    \begin{enumerate}
        \item $\lm_1, \lm_2, \pi$ son homomorfismos de grupos.
        \item $\pi \lm_1$ es trivial.
        \item $\pi\lm_2 = id_H$.
    \end{enumerate}
\end{prop}

\noindent
De forma análoga a la propiedad universal del producto directo, podemos tener la propiedad universal para el producto semidirecto.

\noindent
La siguiente Proposición nos será de utilidad para clasificar grupos haciéndolos isomorfoso a un producto semidirecto, a partir del orden.

\begin{prop}
    Sea $G$ un grupo y $K,H<G$ con $K\lhd G$, $KH = G$ y $K\cap H = \{1\}$, sea $\theta:H\to Aut(K)$ un homomorfismo que nos da la acción $ac:H\times K\to K$ por conjugación\footnote{La condición $K\lhd G$ nos dice que $\theta$ está bien definida}:
    \begin{equation*}
        \theta(h)(k) = hkh^{-1} \qquad \forall h\in H,\forall  k\in K
    \end{equation*}
    Entonces, $K\rtimes_\theta H \cong G$.
    \begin{proof}
        Definiremos la aplicación $f:K\rtimes_\theta H \to G$ dada por:
        \begin{equation*}
            f(k,h) = kh \qquad \forall k\in K, \forall h\in H
        \end{equation*}
        Veamos que es un isomorfismo:
        \begin{itemize}
            \item $f$ es sobreyectiva, ya que $G = KH$, de donde cualquier elemento $g\in G$ se escribe como $g = kh$  para ciertos $k\in K$, $h\in H$.
            \item Para la inyectividad, si $f(k_1,h_1) = f(k_2,h_2)$, entonces $k_1h_1 = k_2h_2$, de donde $k_2^{-1}k_1=h_2h_1^{-1}$:
                \begin{itemize}
                    \item $k_2^{-1}k_1\in K$.
                    \item $h_2h_1^{-1}\in H$.
                \end{itemize}
                Y como $H\cap K = \{1\}$, concluimos que $k_1 = k_2$ y $h_1 = h_2$, de donde $f$ es inyectiva.
            \item Para ver que $f$ es un homomorfismo, si $(k_1,h_1),(k_2,h_2)\in K\rtimes_\theta H$:
                \begin{multline*}
                    f((k_1,h_1)(k_2,h_2)) = f(k_1\prescript{h_1}{}{k_2},h_1h_2) = f(k_1h_1k_2h_1^{-1},h_1h_2) \\ = k_1h_1k_2h_1^{-1}h_1h_2 = k_1h_1k_2h_2 = f(k_1,h_1)f(k_2,h_2)
                \end{multline*}
        \end{itemize}
    \end{proof}
\end{prop}

\begin{definicion}
    Si $G$ verifica las condiciones de la Proposición anterior, decimos que $G$ es producto semidirecto interno de $K$ y $H$.
\end{definicion}

\begin{definicion}[Complemento de un subgrupo]
    Si $K<G$, un subgrupo $H<G$ se llama complemento para $K$ en $G$ si $G = KH$ con $K\cap H = \{1\}$.
\end{definicion}

\begin{observacion}
    Con esta última definición, tendremos que $G$ será un producto semidirecto interno de dos subgrupos propios si y solo si algún subgrupo normal propio tiene un complemento.
\end{observacion}

\begin{ejemplo}
    Esto último no siempre será posible. Por ejemplo, si $G$ es simple, no tendrá subgrupos normales propios, por lo que no será producto semidirecto interno de dos subgrupos.\\

    \noindent
    Si $G$ es un grupo que sí tiene subgrupos normales propios, tampoco somos capaces siempre de poner como un producto semidirecto. Por ejemplo, $Q_2$ no es un producto semidirecto interno de subgrupos propios. Si recordamos su diagrama de Hasse: % // TODO: Poner
    \begin{gather*}
        \langle i \rangle \cap \langle j \rangle  = \{1, -1\} \\
        \langle i \rangle \cap \langle k \rangle  = \{1, -1\} \\
        \langle j \rangle \cap \langle k \rangle  = \{1, -1\} 
    \end{gather*}
    Dado un subgrupo normal, no seremos capaces de complementarlo con otro.
\end{ejemplo}

\begin{ejemplo}
    Para cualquier grupo $K$, si tomamos $H = Aut(K)$ y $\theta = 1_{Aut(K)}$, si tomamos:
    \begin{equation*}
        K\rtimes_\theta Aut(K) = Hol(K)
    \end{equation*}
    Al que llamaremos grupo holomorfo de $K$.\\

    \noindent
    Por ejemplo:
    \begin{equation*}
        Hol(\mathbb{Z}_2\times \mathbb{Z}_2) = S_4
    \end{equation*}
\end{ejemplo}

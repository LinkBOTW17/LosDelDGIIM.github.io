\chapter{Grupos: definición, generalidades y ejemplos}

\begin{definicion}[Operación binaria]
    Sea $G$ un conjunto, una operación binaria en $G$ es una aplicación
    \Func{\ast}{G\times G}{G}{(a,b)}{a\ast b}
\end{definicion}

\begin{ejemplo}
    Ejemplos de operaciones binarias sobre conjuntos son:
    \begin{enumerate}
        \item La suma y el producto de números en $\mathbb{N}$, $\mathbb{Z}$, $\mathbb{Q}$, $\mathbb{R}$, \ldots
        \item Dado un conjunto $X$, consideramos las operaciones $\bigcup$, $\bigcap$ sobre $\mathcal{P}(X)$.
    \end{enumerate}
\end{ejemplo}

\begin{definicion}[Monoide]
    Un monoide es un conjunto $G$ no vacío junto con una operación binaria $\ast$ que verifica:
    \begin{enumerate}
        \item[i)] Asociatividad: $(x\ast y)\ast z = x\ast (y\ast z)$ $\forall x,y,z\in G$.
        \item[ii)] Existencia de elemento neutro: $\exists e\in G \mid e\ast x = x$ $\forall x\in G$
    \end{enumerate}
\end{definicion}

\begin{observacion}
    En un monoide, el elemento neutro es único.
    \begin{proof}
        % // TODO: es fácil
    \end{proof}
\end{observacion}

\begin{notacion}
    Si $X$ es un monoide con una operación binaria $\ast$ y un elemento neutro $e\in X$, será común hacer referencia al monoide por la tripleta:
    \begin{equation*}
        (X,\ast,e)
    \end{equation*}
\end{notacion}

\begin{ejemplo}
    Ejemplos de monoides son (notando):
    \begin{enumerate}
        \item $(\mathbb{N}, +, 0)$, $(\mathbb{N}, \cdot, 1)$
        \item $(\mathcal{P}(X), \cap, X)$, $(\mathcal{P}(X), \cup, \emptyset )$
    \end{enumerate}
\end{ejemplo}

\begin{definicion}[grupo]
    Un grupo es un conjunto $G$ no vacío junto con una operación binaria $\ast$ que verifica:
    \begin{enumerate}
        \item[i)] Asociatividad: $(x\ast y)\ast z = x\ast (y\ast z)$ $\forall x,y,z\in G$.
        \item[ii)] Existencia de elemento neutro: $\exists e\in G \mid e\ast x = x$ $\forall x\in G$.
        \item[iii)] Existencia de elemento simétrico\footnote{Al que luego llamaremos inverso en algunos casos.}: $\forall x\in G$ $\exists x'\in G \mid x'\ast x = e$.
    \end{enumerate}
    Si además se cumple que:
    \begin{enumerate}
        \item[iv)] La propiedad conmutativa de $\ast$: $x\ast y=y\ast x$ $\forall x,y\in G$.
    \end{enumerate}
    Entonces, diremos que $(G,\ast,e)$ es un grupo conmutativo o abeliano.
\end{definicion}

\begin{notacion}
    Nos permitimos los siguientes abusos del lenguaje:
    \begin{enumerate}
        \item Por abuso de lenguaje, admitimos escribir $G$ en lugar de $(G,\ast,e)$, en los casos en los que $\ast$ y $e$ estén claros por el contexto.
        \item Usaremos una notación multiplicativa usualmente, es decir, sustituiremos $\ast$ por $\cdot$ o por la yuxtaposición:
            \begin{equation*}
                x\ast y = x\cdot y = xy
            \end{equation*}
            Con esta notación, notaremos $e = 1$ y al elemento simétrico de $x$ lo notaremos por $x^{-1}$.
        \item En los casos con notación aditiva, escribiremos como operación $\ast$ el símbolo $+$, $\forall x\in G$.

            En estos casos, notaremos $e=0$ y al elemento simétrico de $x$ lo notaremos por $-x$, $\forall x\in G$.
    \end{enumerate}
\end{notacion}

\begin{ejemplo} Consideramos los siguientes ejemplos:
     \begin{enumerate}
         \item $\mathbb{Z},\mathbb{Q},\mathbb{R},\mathbb{C}$ con su respectiva suma son grupos abelianos.
         \item $\mathbb{Q}^\ast,\mathbb{R}^\ast,\mathbb{C}^\ast$ con su respectivo producto son grupos abelianos.
         \item $\{1,-1,i,-i\}\subseteq \mathbb{C}$ con el producto heredado de $\mathbb{C}$ también es un grupo abeliano.
         \item $(\mathcal{M}_2(\mathbb{R}),+)$ es un grupo abeliano.
         \item $GL_2(\mathbb{R})$, el grupo lineal\footnote{Es decir, el conjunto formado por todas las matrices regulares.} de orden 2 (con coeficientes en $\mathbb{R}$) con el producto de matrices es un grupo que no es abeliano.
         \item $\mathbb{Z}_n$ con la suma es un grupo abeliano, $\forall n\in \mathbb{N}$.
         \item $U(\mathbb{Z}_n)=\{[a]\in \mathbb{Z}_n \mid mcd(a,n)=1\}$ con el producto es un grupo abeliano, $\forall n\in \mathbb{N}$.
         \item Dado $n\geq 1$, $\mu_n=\{\text{raíces complejas de\ } x^n-1\} = \{\xi_n = \cos\frac{2k\pi}{2}+i\sen\frac{2k\pi}{2} \mid k\in \mathbb{N}\}$ es un grupo abeliano con el proudcto.
             \begin{equation*}
                 \mu_n = \left\{1,\xi,\xi^2, \ldots, \xi^{n-1} \mid \xi = \cos\frac{2\pi}{n} + i\sen\frac{2\pi}{n}\right\}i
             \end{equation*}
         \item El grupo lineal especial de orden 2 sobre el cuerpo $\mathbb{K}$: 
             \begin{equation*}
                 SL_2(\mathbb{K}) = \{\text{matrices con determinante 1}\}
             \end{equation*}
             siendo $\mathbb{K}$ un cuerpo con el producto de matrices es un grupo no abeliano.
         \item Sean $G$ y $H$ dos grupos, $G\times H$ es un grupo, considerando la operación binaria $\ast:(G\times H)\times(G\times H)\rightarrow G\times H$.
             \begin{equation*}
                 (x,y)\ast(x',y') = (xx',yy')
             \end{equation*}
             A $G\times H$ lo llamaremos \underline{grupo directo} de $G$ y $H$.
         \item Si $X$ es un conjunto no vacío y consideramos
             \begin{equation*}
                 S(X) = \{f:X\rightarrow X \mid f \text{\ biyectiva}\} = Perm(X)
             \end{equation*}
             será un grupo (no abeliano\footnote{Compruébese}) con la operación de composición $\circ$.

             En el caso en el que $X$ sea finito y tenga $n$ elementos: $X = \{x_1, x_2, \ldots, x_n\}$, notamos:
             \begin{equation*}
                 S_n = S(X)
             \end{equation*}
         \item Sea $G$ un grupo y $X$ un conjunto, consideramos el conjunto:
             \begin{equation*}
                 Apl(X,G) = G^X = \{f:X\rightarrow G \mid f \text{\ aplicación}\}
             \end{equation*}
             junto con la operación binaria de multiplicación de aplicaciones:
             \begin{equation*}
                 (f\ast g)(x) = f(x)g(x) \qquad \forall x\in X
             \end{equation*}
             De forma que la aplicación simétrica la calculamos de la forma\footnote{En cada punto, la aplicacion simétrica es el simétrico del elemento $f(x)$.}:
             \begin{equation*}
                 f'(x) = {(f(x))}'
             \end{equation*}
             Es un grupo. Casos a destacar son:
             \begin{enumerate}
                 \item Si $X=\emptyset $, entonces $G^X = \{\emptyset \}$.
                 \item SI $X = \{1,2\}$, entonces $G^X$ se identifica con $G\times G$.
             \end{enumerate}
         \item El conjunto $\{1\}$ con cualquier operación binaria es un grupo conmutativo, al que llamaremos grupo trivial.
     \end{enumerate}
\end{ejemplo}

\subsubsection{Propiedades}

\begin{prop}
    En un grupo $G$, el neutro y el simétrico de cada elemento son únicos.
    \begin{proof}
        % // TODO: Hacer la demostración
    \end{proof}
\end{prop}

\begin{prop}
    Sea $G$ un grupo, entonces:
    \begin{enumerate}
        \item[i)] $xx^{-1} = e$ $\forall x\in G$
        \item[ii)]  $xe = x$ $\forall x\in G$
    \end{enumerate}
    \begin{proof}
        Veamos cada una de las propiedades:
        \begin{enumerate}
            \item[$i)$] Usando la unicidad del neutro $e\in G$:
                \begin{equation*}
                    x^{-1}(xx^{-1}) = (x^{-1}x)x^{-1} = ex^{-1} = x^{-1} \Longrightarrow xx^{-1} = e
                \end{equation*}
            \item[$ii)$]
                \begin{equation*}
                    xe = x(x^{-1}x) = (xx^{-1})x = ex = x
                \end{equation*}
        \end{enumerate}
    \end{proof}
\end{prop}

\begin{prop}
    En un grupo $G$ se verifica la propiedad cancelativa (tanto a la izquierda como a la derecha):
    \begin{equation*}
        \forall x,y,z\in G:\ \left\{\begin{array}{l}
            xy = xz \Longrightarrow y = z \\
            xy = zy \Longrightarrow x = z
        \end{array}\right.
    \end{equation*}
    \begin{proof}
        Para la primera, supongamos que $xy=xz$:
        \begin{equation*}
            y = ey = (x^{-1}x)y = x^{-1}(xy) = x^{-1}(xz) = (x^{-1}x)z = ez = z
        \end{equation*}
        % // TODO: demostrar la segunda
    \end{proof}
\end{prop}

\begin{prop}
    Sea $G$ un grupo, entones:
    \begin{enumerate}
        \item $e^{-1} = e$.
        \item ${(x^{-1})}^{-1} = x$, $\forall x\in G$.
        \item ${(xy)}^{-1} = y^{-1}x^{-1}$, $\forall x,y\in G$.
    \end{enumerate}
    \begin{proof} Cada caso se demuestra observando sencillamante:
        \begin{enumerate}
            \item $e e = e$.
            \item $xx^{-1} = e$.
            \item $(y^{-1}x^{-1})(xy) = y^{-1}x^{-1}xy = y^{-1} e y = e$.
        \end{enumerate}
    \end{proof}
\end{prop}

\begin{prop}
    Sea $G$ un conjunto no vacío con una operación binaria $\ast$ asociativa, son equivalentes:
    \begin{enumerate}
        \item[i)] $G$ es un grupo.
        \item[ii)] Para cada par de elementos $a,b\in G$, las ecuaciones:
            \begin{equation*}
                aX = b \qquad Xa = b
            \end{equation*}
            Tienen solución en $G$ ($\exists c,d\in G\mid ac=b \land da = b$).
    \end{enumerate}
    \begin{proof}
        \begin{description}
            \item [$i)\Rightarrow ii)$] Tomando $c=a^{-1}b$ y $d = ba^{-1}$ se tiene.
            \item [$i)\Rightarrow ii)$] % // TODO: Hacer
        \end{description}
    \end{proof}
\end{prop}

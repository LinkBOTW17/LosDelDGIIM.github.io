En Álgebra I el objeto principal de estudio fueron los anillos conmutativos, conjuntos en los que teníamos definidas dos operaciones, una usualmente denotada con notación aditiva y otra con notación multiplicativa.

Posteriormente, el estudio se centró en los dominios de integridad (DI), anillos conmutativos donde teníamos más propiedades con las que manejar nuestros elementos (como la tan característica propiedad cancelativa). Después, el objeto de estudio fueron los dominios euclídeos (DE), donde ya podíamos realizar un estudio sobre la divisibilidad de los elementos del conjunto.

Finalmente, nos centramos en los dominios de factorización única (DFU), donde realizamos una breve introducción a la irreducibilidad de los polinomios.\\

En esta asignatura el principal objeto de estudio serán los grupos, conjuntos en los que hay definida una sola operación que entendemos por ``buena\footnote{La operación cumplirá ciertas propiedades deseables.}''. Por tanto, los grupos serán estructuras menos restrictivas que los anillos conmutativos, aunque su estudio no será menos interesante.

\chapter{Grupos: definición, generalidades y ejemplos}
Comenzamos realizando la primera definición necesaria para entender el concepto de grupo, que es entender qué es una operación dentro de un conjunto.

\begin{definicion}[Operación binaria]
    Sea $G$ un conjunto, una operación binaria en $G$ es una aplicación
    \Func{\ast}{G\times G}{G}{(a,b)}{a\ast b}
\end{definicion}

\begin{ejemplo}
    Ejemplos de operaciones binarias sobre conjuntos que ya conocemos son:
    \begin{enumerate}
        \item La suma y el producto de números en $\mathbb{N}$, $\mathbb{Z}$, $\mathbb{Q}$, $\mathbb{R}, \mathbb{C}$.
        \item Dado un conjunto $X$, los operadores $\cap$ y $\cup$ son operaciones binarias sobre el conjunto $\mathcal{P}(X)$.
    \end{enumerate}
\end{ejemplo}

Antes de dar la definición de grupo, daremos la de monoide, que es menos restrictiva que la de grupo.

\begin{definicion}[Monoide]
    Un monoide es una tripleta $(G,\ast,e)$ donde $G$ es un conjunto no vacío, $\ast$ es una operación binaria en $G$ y $e$ es un elemento destacado de $G$ de forma que se verifica:
    \begin{enumerate}
        \item[$i)$] La propiedad asociativa de $\ast$:
            \begin{equation*}
                (x\ast y) \ast z = x \ast (y\ast z) \qquad \forall x,y,z\in G
            \end{equation*}
        \item[$ii)$] La existencia de un elemento neutro (el elemento destacado de $G$):
            \begin{equation*}
                \exists e\in G \mid e\ast x = x\ast e = x \qquad \forall x\in G
            \end{equation*}
    \end{enumerate}
\end{definicion}

\begin{prop}\label{prop:neutro_unico_monoide}
    En un monoide, el elemento neutro es único.
    \begin{proof}
        Sea $(G,\ast,e)$ un monoide y sea $f\in G$ tal que $f\ast x = x\ast f = x$ $\forall x\in G$:
        \begin{equation*}
            f = f\ast e = e
        \end{equation*}
    \end{proof}
\end{prop}

\begin{ejemplo}
    Ejemplos de monoides ya conocidos son:
    \begin{enumerate}
        \item $(\mathbb{N}, +, 0)$, $(\mathbb{N}, \cdot, 1)$
        \item Dado un conjunto $X$: $(\mathcal{P}(X), \cap, X)$, $(\mathcal{P}(X), \cup, \emptyset )$
    \end{enumerate}
\end{ejemplo}

\begin{definicion}[Grupo]
    Un grupo es una tripleta $(G,\ast,e)$ donde $G$ es un conjunto no vacío, $\ast$ es una operación binaria en $G$ y $e$ es un elemento destacado de $G$ de forma que se verifica:
    \begin{enumerate}
        \item[$i)$] La propiedad asociativa de $\ast$:
            \begin{equation*}
                (x\ast y) \ast z = x \ast (y\ast z) \qquad \forall x,y,z\in G
            \end{equation*}
        \item[$ii)$] La existencia de un elemento neutro por la izquierda (el elemento destacado de $G$):
            \begin{equation*}
                \exists e\in G \mid e\ast x = x \qquad \forall x\in G
            \end{equation*}
        \item[$iii)$] La existencia de un elemento simétrico por la izquierda para cada elemento de $G$:
            \begin{equation*}
                \forall x\in G \quad \exists x'\in G\mid x'\ast x = e
            \end{equation*}
    \end{enumerate}
    Si además se cumple:
    \begin{enumerate}
        \item[$iv)$] La propiedad conmutativa de $\ast$: 
            \begin{equation*}
                x\ast y = y \ast x \qquad \forall x,y\in G
            \end{equation*}
    \end{enumerate}
    Entonces, diremos que $(G,\ast,e)$ es un grupo conmutativo o abeliano.
\end{definicion}

\begin{notacion}
    Para una mayor comodidad a la hora de manejar grupos, introducimos las siguientes notaciones:
    \begin{enumerate}
        \item Cuando dado un conjunto no vacío $G$ sepamos por el contexto a qué grupo $(G,\ast,e)$ nos estamos refiriendo, indicaremos simplemente $G$ (o en algunos casos $(G,\ast)$, para hacer énfasis en la operación binaria) para referirnos al grupo $(G,\ast,e)$.
        \item En algunos casos, usaremos (por comodidad) la notación multiplicativa de los grupos. De esta forma, dado un grupo $(G,\cdot,1)$, en ciertos casos notaremos la operación binaria $\cdot$ simplemente por yuxtaposición:
            \begin{equation*}
                x \cdot y = xy \qquad \forall x,y\in G
            \end{equation*}
            Además, nos referiremos al elemento neutro como ``uno'' y al simétrico de cada elemento como ``inverso'', sustituyendo la notación de $x'$ por la de $x^{-1}$.
        \item Otra notación que también usaremos (aunque de forma menos frecuente que la multiplicativa) será la aditiva. Dado un grupo $(G,+,0)$, nos referiremos al elemento neutro como ``cero'' y al simétrico de cada elemento como ``opuesto'', sustituyendo la notación de $x'$ por la de $-x$.
    \end{enumerate}
\end{notacion}

\begin{ejemplo} Ejemplos de grupos que se usarán con frecuencia en la asignatura son:
     \begin{enumerate}
         \item $\mathbb{Z},\mathbb{Q},\mathbb{R},\mathbb{C}$ con su respectiva suma son grupos abelianos.
         \item $\mathbb{Q}^\ast,\mathbb{R}^\ast,\mathbb{C}^\ast$ con su respectivo producto son grupos abelianos.

             Notemos la importancia de eliminar el $0$ de cada conjunto para que todo elemento tenga inverso, así como que $\mathbb{Z}^\ast$ no es un grupo, ya que el inverso de cada elemento (para el producto al que estamos acostumbrados) no está dentro de $\mathbb{Z}^\ast$.
         \item $\{1,-1,i,-i\}\subseteq \mathbb{C}$ con el producto heredado\footnote{Será común hablar de ``operación heredada'' cuando consideramos un subconjunto de un conjunto en el que ya hay definida una operación interna, haciendo referencia a la restricción en dominio y recorrido de dicha operación interna al subconjunto considerado.} de $\mathbb{C}$ también es un grupo abeliano.
         \item $(\mathcal{M}_2(\mathbb{R}),+)$ es un grupo abeliano.
         \item Dado un cuerpo $\mathbb{K}$, el grupo lineal de orden 2 con coeficientes en dicho cuerpo:
             \begin{equation*}
                 \GL_2(\mathbb{K}) = \{M\in \mathcal{M}_2(\mathbb{K}) : \det(M)\neq 0\}
             \end{equation*}
             con el producto heredado de $\mathcal{M}_2(\mathbb{K})$ es un grupo que no es conmutativo.
         \item $\mathbb{Z}_n$ con su suma es un grupo abeliano, $\forall n\in \mathbb{N}$.
         \item $\cc{U}(\mathbb{Z}_n)=\{[a]\in \mathbb{Z}_n \mid \mcd(a,n)=1\}$ con el producto es un grupo abeliano, $\forall n\in \mathbb{N}$. También lo notaremos por $\mathbb{Z}_n^\times$.
         \item Dado $n\geq 1$, consideramos:
             \begin{align*}
                 \mu_n &= \{\text{raíces complejas de\ } x^n-1\} = \left\{\xi_k = \cos\frac{2k\pi}{n} + i\sen\frac{2k\pi}{n} : k\in \{0,\dots,n-1\}\right\} \\
                       &= \left\{1,\xi,\xi^2, \ldots, \xi^{n-1} : \xi = \cos\frac{2\pi}{n} + i\sen\frac{2\pi}{n}\right\}
             \end{align*}
             Este conjunto es un grupo abeliano con el producto heredado de $\mathbb{C}$.
         \item Dado un cuerpo $\mathbb{K}$, el grupo lineal especial de orden 2 sobre dicho cuerpo:
             \begin{equation*}
                 \SL_2(\mathbb{K}) = \{M\in \mathcal{M}_2(\mathbb{K}) : \det(M) = 1\}
             \end{equation*}
             con el producto heredado de $\mathcal{M}_2(\mathbb{K})$ es un grupo que no es conmutativo.
         \item Sean $(G,\square,e),(H,\triangle,f)$ dos grupos, si consideramos sobre $G\times H$ la operación binaria $\ast:(G\times H)\times(G\times H)\rightarrow G\times H$  dada por:
             \begin{equation*}
                 (x,u) \ast (y,v) = (x\square y, u\triangle v) \qquad \forall (x,u),(y,v)\in G\times H
             \end{equation*}
             Entonces, $G\times H$ es un grupo, al que llamaremos \underline{grupo directo} de $G$ y $H$. Este será abeliano si y solo si $G$ y $H$ lo son.
         \item Si $X$ es un conjunto no vacío y consideramos
             \begin{equation*}
                 S(X) = \{f:X\rightarrow X \mid f \text{\ biyectiva}\} = \Perm(X)
             \end{equation*}
             es un grupo no abeliano con la operación de composición de funciones $\circ$.

             En el caso en el que $X$ sea finito y tenga $n$ elementos: $X = \{x_1, x_2, \ldots, x_n\}$, notaremos:
             \begin{equation*}
                 S_n = S(X)
             \end{equation*}
         \item Sea $(G,\ast,e)$ un grupo y $X$ un conjunto, consideramos el conjunto:
             \begin{equation*}
                 \Apl(X,G) = G^X = \{f:X\rightarrow G \mid f \text{\ aplicación}\}
             \end{equation*}
             junto con la operación binaria $\ast:G^X\times G^X \rightarrow G^X$ dada por:
             \begin{equation*}
                 (f\ast g)(x) = f(x)\ast g(x) \qquad \forall x\in X, \quad \forall f,g\in G^X
             \end{equation*}
             Entonces, $(G^X, \ast, g)$ es un grupo, con elemento neutro:
             \begin{equation*}
                 g(x) = e \qquad \forall x\in X
             \end{equation*}
             de esta forma, dada $f\in G^X$, la aplicación simétrica de $f$ será:
             \begin{equation*}
                 f'(x) = {(f(x))}' \qquad \forall x\in X
             \end{equation*}

             Casos a destacar son:
             \begin{enumerate}
                 \item Si $X=\emptyset $, entonces $G^X = \{\emptyset \}$.
                 \item Si $X = \{1,2\}$, entonces $G^X$ se identifica con $G\times G$.
             \end{enumerate}
         \item El grupo más pequeño que se puede considerar es el único grupo válido sobre un conjunto unitario $X=\{e\}$. Es decir, el grupo $(X,\ast,e)$ con $X = \{e\}$ y $\ast:X\times X\rightarrow X$ dada por:
             \begin{equation*}
                 e\ast e = e \qquad e\in X
             \end{equation*}
             A este grupo (independientemente de cual sea el conjunto $X$, ya que todos tendrán la misma\footnote{Concepto que luego formalizaremos.} estructura) lo llamaremos \underline{grupo trivial}.
     \end{enumerate}
\end{ejemplo}

\begin{ejemplo}
    Consideramos en $\mathbb{Z}$ la operación binaria $\ast:\mathbb{Z}\times\mathbb{Z} \rightarrow \mathbb{Z}$ dada por:
    \begin{equation*}
        a\ast b = a + b + 1 \qquad \forall a,b\in \mathbb{Z}
    \end{equation*}
    Donde usamos $+$ para denotar la suma de $\mathbb{Z}$. Se pide demostrar que $(\mathbb{Z},\ast)$ es un grupo abeliano.
    \begin{proof}
        Demostramos cada una de las propiedades de la definición de grupo abeliano:
        \begin{itemize}
            \item La propiedad asociativa de $\ast$ es consecuencia de las propiedades asociativa y conmutativa de $+$:
            \begin{align*}
                (a\ast b) \ast c &= (a+b+1) \ast c = a + b + 1 + c + 1 = a + b + c + 2 \\
                a \ast (b\ast c) &= a\ast (b+c+1) = a+b+c+1+1 = a + b + c + 2 \\
                                 &\forall a,b,c\in \mathbb{Z}
            \end{align*}
            \item Buscamos $x\in \mathbb{Z}$ de forma que $x\ast a = a$ para todo $a\in \mathbb{Z}$, por lo que queremos resolver la ecuación:
                \begin{equation*}
                    X\ast a = a \Longleftrightarrow  X + a + 1 = a \Longrightarrow X = -1
                \end{equation*}
                Por lo que $-1\in \mathbb{Z}$ es el elemento neutro para $\ast$:
                \begin{equation*}
                    -1\ast a = -1 + a +1 = a \qquad \forall a\in \mathbb{Z}
                \end{equation*}
            \item Fijado $x\in \mathbb{Z}$, tratamos de buscar un elemento simétrico para $x$, por lo que buscamos resolver la ecuación:
                \begin{equation*}
                    X\ast x = -1 \Longleftrightarrow X + x + 1 = -1 \Longleftrightarrow X = -x-2
                \end{equation*}
                Por lo que dado $x\in \mathbb{Z}$, su elemento simétrico es $-x-2\in \mathbb{Z}$:
                \begin{equation*}
                    (-x-2)\ast x = -x-2+x+1 = -1 \qquad \forall x\in \mathbb{Z}
                \end{equation*}
            \item La propiedad conmutativa de $\ast$ es consecuencia de la propiedad conmutativa de $+$:
                \begin{equation*}
                    a\ast b = a + b + 1 = b + a + 1 = b \ast a \qquad \forall a,b\in \mathbb{Z}
                \end{equation*}
        \end{itemize}
    \end{proof}
\end{ejemplo}
\subsubsection{Propiedades}
Aunque estas propiedades parezcan ya conocidas y familiares (por ejemplo para el caso $(\mathbb{Z},+,0)$), es una buena observación darnos cuenta de que son válidas para \textbf{cualquier grupo} que consideremos, por raros y difíciles que sean sus elementos y operación interna.

\begin{prop}
    Sea $(G,\ast,e)$ un grupo, destacamos sus primeras propiedades:
    \begin{enumerate}
        \item[$i)$] $x\ast x' = e$ $\forall x\in G$.
        \item[$ii)$] $x\ast e = x$ $\forall x\in G$.
        \item[$iii)$] El elemento neutro de $\ast$ es único. Simbólicamente:
            \begin{equation*}
                \exists_1 e\in G \mid e\ast x = x \qquad \forall x\in G
            \end{equation*}
        \item[$iv)$] Fijado $x\in G$, el simétrico de $x$ es único. Simbólicamente:
            \begin{equation*}
                \forall x\in G \quad \exists_1 x' \in G \mid x' \ast x = e
            \end{equation*}
    \end{enumerate}
    \begin{proof}
        Demostramos cada una a partir de la anterior:
        \begin{enumerate}
            \item[$i)$] En primer lugar, observemos que:
                \begin{equation}\label{eq:primera_ec}
                    x'\ast (x\ast x') = (x'\ast x) \ast x' = e \ast x' = x'
                \end{equation}
                Ahora:
                \begin{equation*}
                    x\ast x' = e\ast (x \ast x') = \left({(x')}'\ast x'\right) \ast (x\ast x') = {(x')}' \ast (x'\ast (x\ast x')) \AstIg {(x')}' \ast x' = e
                \end{equation*}
                Donde en $(\ast)$ hemos usado~(\ref{eq:primera_ec}).
            \item[$ii)$] Usando $i)$ en $(\ast)$:
                \begin{equation*}
                    x\ast e = x\ast (x' \ast x) = (x \ast x') \ast x \AstIg e \ast x = x
                \end{equation*}
            \item[$iii$)] Sea $f\in G$ de forma que $f\ast x = x$ $\forall x\in G$, entonces:
                \begin{equation*}
                    f = f \ast e \AstIg e
                \end{equation*}
                Donde en $(\ast)$ hemos usado $ii)$.

                De otra forma, podríamos haber argumentado que gracias a $ii)$, todo grupo es un monoide, por lo que podemos aplicar la Proposición~\ref{prop:neutro_unico_monoide} y ya habríamos terminado.
            \item[$iv)$] Dado $x\in G$, sea $x'' \in G$ de forma que $x'' \ast x = e$, entonces:
                \begin{equation*}
                    x '' = x'' \ast e \AstIg x '' \ast (x \ast x') = (x '' \ast x) \ast x' = e \ast x' = x'
                \end{equation*}
                Donde en $(\ast)$ hemos usado $i)$.
        \end{enumerate}
    \end{proof}
\end{prop}


\begin{notacion}
    A partir de ahora, dado un grupo $(G,\ast,e)$, comenzaremos a usar (por comodidad) la notacion multiplicativa de los grupos:
    \begin{equation*}
        xy = x\ast y \qquad \forall x,y\in G
    \end{equation*}
    Y denotando a $x'$ (el elemento simétrico de $x$) por $x^{-1}$.
\end{notacion}

\begin{prop}
    En un grupo $G$ se verifica la propiedad cancelativa (tanto a la izquierda como a la derecha):
    \begin{equation*}
        \forall x,y,z\in G:\ \left\{\begin{array}{l}
            xy = xz \Longrightarrow y = z \\
            xy = zy \Longrightarrow x = z
        \end{array}\right.
    \end{equation*}
    \begin{proof}
        Para la primera, supongamos que $xy=xz$:
        \begin{equation*}
            y = ey = (x^{-1}x)y = x^{-1}(xy) = x^{-1}(xz) = (x^{-1}x)z = ez = z
        \end{equation*}
        Ahora, para la segunda, supongamos que $xy = zy$ y la demostración es la misma que la anterior pero en el otro sentido y tomando $e = yy^{-1}$.
        \begin{equation*}
            x = xe = x(yy^{-1}) = (xy)y^{-1} = (zy)y^{-1} = z(yy^{-1}) = z
        \end{equation*}
    \end{proof}
\end{prop}

\begin{prop}
    Sea $G$ un grupo, entones:
    \begin{enumerate}
        \item $e^{-1} = e$.
        \item ${(x^{-1})}^{-1} = x$, $\forall x\in G$.
        \item ${(xy)}^{-1} = y^{-1}x^{-1}$, $\forall x,y\in G$.
    \end{enumerate}
    \begin{proof} Cada caso se demuestra observando sencillamante que:
        \begin{enumerate}
            \item $e e = e$.
            \item $xx^{-1} = e$.
            \item $(y^{-1}x^{-1})(xy) = y^{-1}x^{-1}xy = y^{-1} e y = e$.
        \end{enumerate}
    \end{proof}
\end{prop}

\begin{prop}\label{prop:carac_grupo_ec}
    Sea $G$ un conjunto no vacío con una operación binaria $\ast$ asociativa, son equivalentes:
    \begin{enumerate}
        \item[i)] $G$ es un grupo.
        \item[ii)] Para cada par de elementos $a,b\in G$, las ecuaciones\footnote{Donde hemos usado $X$ para denotar la incógnita y que no se confunda con un elemento de $G$.}:
            \begin{equation*}
                aX = b \qquad Xa = b
            \end{equation*}
            Tienen solución en $G$, es decir: $\exists c,d\in G\mid ac=b \land da = b$.
    \end{enumerate}
    \begin{proof}
        Demostramos las dos implicaciones:
        \begin{description}
            \item [$i)\Rightarrow ii)$] Tomando $c=a^{-1}b,d = ba^{-1}\in G$ se tiene.
            \item [$ii)\Rightarrow i)$] Basta demostrar que $\exists e\in G$ con $ex = x$ $\forall x\in G$ y que fijado $x\in G$, entonces $\exists x'\in G$ con $x'x = e$:

                \begin{enumerate}
                    \item Dado $a\in G$, sabemos que la ecuación $Xa=a$ tiene solución, por lo que existe $e\in G$ de forma que $ea = a$.

                        Veamos que no depende de la elección de $a$; es decir, que es un elemento neutro para cualquier elemento de $G$. Para ello, dado cualquier $b\in G$, sabemos que la ecuación $aX=b$ tiene solución, por lo que existirá un $x_b\in G$ de forma que $ax_b=b$. Finalmente:
                        \begin{equation*}
                            eb = e(ax_b) = (ea)x_b = ax_b = b \qquad \forall b\in G
                        \end{equation*}
                    \item Fijado $x\in G$, sabemos que la ecuación $Xx=e$ tiene solución, por lo que existe $x'\in G$ de forma que $x'x = e$, para cualquier $x\in G$.
                \end{enumerate}
        \end{description}
    \end{proof}
\end{prop}

\begin{prop}[Ley asociativa general]
    Sea $G$ un grupo, dados $n,m\in \mathbb{N}$ con $n>m>0$, se tiene que:
    \begin{equation*}
        \left(\prod_{i=1}^m x_i\right) \left(\prod_{i=m+1}^nx_i\right) = \prod_{i=1}^n x_i \qquad \forall x_i \in G, \quad i \in \{1,\ldots,n\}
    \end{equation*}
    \begin{proof}
        Por inducción sobre $n\in \mathbb{N}$:
        \begin{itemize}
            \item \underline{Para $n=0,n=1$}: No hay nada que probar: $\nexists m\in \mathbb{N}$ con $0<m<n$.
            \item \underline{Para $n=2$}: Dado $m\in \mathbb{N}$ con $0<m<n$ (entonces $m=1$):
            \begin{equation*}
                \left(\prod_{i=1}^m x_i\right) \left(\prod_{i=m+1}^nx_i\right) = x_1 x_2 =\prod_{i=1}^n x_i \qquad \forall x_1,x_2\in G
            \end{equation*}
            \item \underline{Supuesto para $n$}, veámoslo para $n+1$: Dado $m\in \mathbb{N}$ con $0<m<n+1$:
            \begin{align*}
                \left(\prod_{i=1}^m x_i\right) \left(\prod_{i=m+1}^{n+1}x_i\right) &= \left[x_1\left(\prod_{i=2}^m x_i\right)\right]\left[\left(\prod_{i=m+1}^nx_i\right) x_{n+1}\right]\\  
                                                                                   &= x_1\left(\prod_{i=2}^m x_i \prod_{i=m+1}^nx_i\right)x_{n+1} \AstIg x_1 \left(\prod_{i=2}^n x_i\right) x_{n+1} = \prod_{i=1}^{n+1} x_i \\
                                                                                   &\forall x_i \in G, \quad i \in \{1,\ldots,n+1\}
            \end{align*}
            Donde en $(\ast)$ hemos usado la hipótesis de inducción, ya que $0<m-1<n$.
        \end{itemize}
    \end{proof}
\end{prop}

\begin{definicion}[Potencia]
    Sea $(G,\cdot,e)$ un grupo, dado $x\in G$ y $n\in \mathbb{Z}$, podemos definir:
    \begin{equation*}
        x^n = \left\{\begin{array}{cr}
                \prod\limits_{i=1}^n x & \text{si\ }n > 0 \\
                e & \text{si\ }n = 0 \\
                {(x^{-1})}^{-n} & \text{si\ }n < 0
        \end{array}\right.
    \end{equation*}
\end{definicion}

\begin{notacion}
    En grupos aditivos $(G,+,0)$, en lugar de $x^n$ escribiremos $n\cdot x$, que se define de igual forma pero en el caso $n>0$, en lugar de escribir $\prod$, escribiremos $\sum$.
\end{notacion}

\begin{prop}
    Sea $G$ un grupo, se verifica que:
    \begin{equation*}
        x^{n+m} = x^n \cdot x^m \qquad \forall x\in G, \quad n,m\in \mathbb{Z}
    \end{equation*}
    \begin{proof} % // TODO: Comprobar que el pardillo de Arturo no se ha equivocado
 Aunque la demostración es sencilla, hemos de distinguir bastantes casos, pues hemos de asegurarnos de que el límite superior de cada producto sea siempre un número positivo.
    Fijado $x\in G$, distinguimos en función de los valores de $n,m\in \mathbb{Z}$:
    \begin{enumerate}
        \item \ul{$n>0$}:
        \begin{enumerate}
            \item \ul{$m>0$}:
            \begin{equation*}
                x^{n+m} = \prod_{i=1}^{n+m} x =\left(\prod_{i=1}^n x\right) \cdot\left(\prod_{i=n+1}^{n+m} x\right)
                =\left(\prod_{i=1}^n x\right) \cdot\left(\prod_{i=1}^m x\right) = x^n \cdot x^m
            \end{equation*}

            \item \ul{$m=0$}:
            \begin{equation*}
                x^{n+0} = x^n = x^n \cdot e = x^n \cdot x^0
            \end{equation*}

            \item \ul{$m<0$}:
            
            En este caso, no sabemos el signo de $n+m$. Por tanto, hemos de distinguir casos:
            \begin{enumerate}
                \item \ul{$n+m>0$}:
                Entonces, $n>-m$. Tenemos:
                \begin{equation*}
                    x^n \cdot x^m =\left(\prod_{i=1}^n x\right) \cdot \left(x^{-1}\right)^{-m}
                    =\left(\prod_{i=1}^n x\right) \cdot\left(\prod_{i=1}^{-m} x^{-1}\right) =\prod_{i=1}^{n-(-m)} x
                    =\prod_{i=1}^{n+m} x = x^{n+m}
                \end{equation*}

                \item \ul{$n+m=0$}:
                Entonces, $n=-m$. Tenemos:
                \begin{equation*}
                    x^{n+m} = x^0 = e =\left(\prod_{i=1}^n x\right) \cdot\left(\prod_{i=1}^{n} x^{-1}\right) = x^n \cdot\left(\prod_{i=1}^{-m} x^{-1}\right) = x^n \cdot \left(x^{-1}\right)^{-m} = x^n \cdot x^m
                \end{equation*}

                \item \ul{$n+m<0$}:
                Entonces, $n<-m$. Tenemos:
                \begin{multline*}
                    x^n \cdot x^m =\left(\prod_{i=1}^n x\right)\cdot \left(x^{-1}\right)^{-m}
                    =\left(\prod_{i=1}^n x \right)\cdot\left(\prod_{i=1}^{-m} x^{-1}\right) =\prod_{i=1}^{-m-n} x^{-1}
                    =\\=\prod_{i=1}^{-(n+m)} x^{-1} = (x^{-1})^{-(n+m)} = x^{n+m}
                \end{multline*}
            \end{enumerate}

        \end{enumerate}

        \item \ul{$n=0$}:
        \begin{equation*}
            x^{0+m} = x^m = e \cdot x^m = x^0 \cdot x^m
        \end{equation*}

        \item \ul{$n<0$}:
        \begin{enumerate}
            \item \ul{$m>0$}:
            \begin{equation*}
                x^{n+m} = x^{m+n} = x^{m} \cdot x^{n}
                = \prod_{i=1}^{m} x \cdot \prod_{i=1}^{-n} x^{-1}
                = x^n \cdot x^m
            \end{equation*}
            donde en la primera igualdad hemos usado la propiedad conmutativa de la suma en $\mathbb{Z}$, en la segunda hemos empleado el caso anteriormente demostrado, y en la última igualdad hemos empleado que $xx^{-1} = e = x^{-1}x$.

            \item \ul{$m=0$}:
            \begin{equation*}
                x^{n+0} = x^n = x^n \cdot e = x^n \cdot x^0
            \end{equation*}

            \item \ul{$m<0$}:
            \begin{multline*}
                x^n \cdot x^m =\left(x^{-1}\right)^{-n} \cdot\left(x^{-1}\right)^{-m}
                = \left(\prod_{i=1}^{-n} x^{-1}\right) \cdot\left(\prod_{i=-1}^{-m} x^{-1}\right) =\\
                = \prod_{i=1}^{-n-m} x^{-1} = \left(x^{-1}\right)^{-(n+m)} = x^{n+m}
            \end{multline*}
        \end{enumerate}
    \end{enumerate}
    \end{proof}
\end{prop}

\begin{definicion}[Grupos finitos e infinitos]
    Sea $G$ un grupo, si $G$ como conjunto tiene\footnote{Excluimos $n= 0$ ya que en la definición de grupo exigimos que $G\neq \emptyset $.} $n\in \mathbb{N}\setminus\{0\}$ elementos, diremos que es un grupo finito. En dicho caso, diremos que $n$ es el ``orden del grupo'', notado por: $|G| = n$.\newline
    Si $G$ no fuera finito, decimos que es un grupo infinito.
\end{definicion}

\begin{definicion}[Tabla de Cayley]
    En un grupo finito $G=\{x_1,x_2,\ldots,x_n\}$, se llama tabla de Cayley (o de multiplicar\footnote{Entendiendo que en este caso hacemos uso de la notación multiplicativa.}) a la matriz $n\times n$ de forma que su entrada $(i,j)$ es $x_ix_j$.
\end{definicion}

\begin{ejemplo}
    A continuación, mostramos ejemplos de posibles tablas de Cayley para ciertas operaciones sobre determinados grupos. Como podemos ver, la finalidad de la tabla es mostrar en cada caso cómo se comporta la operación binaria cuando se aplica a distintos elementos del grupo.
    \begin{enumerate}
        \item Si $G=\{0,1\}$, podemos considerar sobre $G$ las operaciones $\ast_1$ y $\ast_2$, cuya definición puede obtenerse a partir de sus tablas de Cayley:
            \begin{equation*}
                \begin{array}{c|cc}
                    \ast_1 & 0 & 1 \\
                    \hline 
                    0 & 0 & 1 \\
                    1 & 1 & 0
                \end{array} \qquad 
                \begin{array}{c|cc}
                    \ast_2 & 0 & 1 \\
                    \hline 
                    0 & 1 & 0 \\
                    1 & 0 & 1
                \end{array}
            \end{equation*}
        \item Si $G=\{0,1,2\}$, podemos considerar sobre $G$ la siguiente operación binaria:
            \begin{equation*}
                \begin{array}{c|ccc}
                     & 0 & 1 & 2 \\
                     \hline
                    0 & 0 & 1 & 2 \\
                    1 & 1 & 2 & 0 \\
                    2 & 2 & 0 & 1 
                \end{array}
            \end{equation*}
        \item Si $G=\{0,1,2,3\}$, podemos considerar sobre $G$ las siguientes operaciones binarias:
            \begin{equation*}
                \begin{array}{c|cccc}
                     & 0 & 1 & 2 & 3 \\
                     \hline 
                    0 & 0 & 1 & 2 & 3\\
                    1 & 1 & 2 & 3 & 0 \\
                    2 & 2 & 3 &  0 & 1 \\
                    3 & 3 & 0 & 1 & 2
                \end{array} \qquad 
                \begin{array}{c|cccc}
                     & 0 & 1 & 2 & 3 \\
                     \hline 
                    0 & 0 & 1 & 2 & 3 \\
                    1 & 1 & 0 & 3  & 2\\
                    2& 2 & 3 & 0 & 1 \\ 
                    3 & 3 & 2 & 1 & 0
                \end{array}
            \end{equation*}
    \end{enumerate}
    A partir de la definición de la tabla de Cayley para la operación binaria de un grupo pueden deducirse ciertas propiedades que estas tienen, las cuales no demostraremos, entendiendo que pueden deducirse de forma fácil a partir de la definición de grupo:
    \begin{itemize}
        \item Si consideramos un grupo abeliano, su tabla de Cayley será una matriz simétrica.
        \item Todos los elementos del grupo aparecen en todas las filas o columnas de la tabla de Cayley, ya que en la Proposición~\ref{prop:carac_grupo_ec} vimos que las ecuaciones $aX=b$ y $Xa=b$ tenían que tener solución $\forall a,b\in G$, para que $G$ fuese un grupo.
        \item Como para que $G$ sea un grupo tiene que haber un elemento que actúe de neutro, esto se refleja en la tabla con un elemento que mantiene igual los encabezados en una fila y en una columna.
    \end{itemize}
\end{ejemplo}

\begin{definicion}[Orden de un elemento]
    Sea $(G,\cdot,1)$ un grupo, el orden de un elemento $x\in G$ es el menor $n\in \mathbb{N}\setminus\{0\}$ (en caso de existir) que verifica: $x^n = 1$. En cuyo caso, notaremos\footnote{Podremos encontrarnos cualquiera de las dos notaciones.}: $O(x) = \ord(x) = n$.\newline
    Si para un elemento $x\in G$ dicho $n$ no existe, se dice que su orden es infinito: $O(x) = +\infty$.
\end{definicion}

\begin{notacion}
    Si consideramos un grupo con notación aditiva, $(G,+,0)$, interpretando la anterior definición con esta notación diremos que $x\in G$ tendrá orden $n\in \mathbb{N}\setminus\{0\}$ si $n$ es el menor natural no nulo de forma que verifica $n\cdot x = \sum\limits_{i=1}^{n}x = 0$.
\end{notacion}

\begin{prop}\label{prop:divide_orden}
    Sea $G$ un grupo, $x\in G$ con $O(x) = n$ y sea $m\in \mathbb{N}\setminus\{0\}$:
    \begin{equation*}
        x^m = 1 \Longleftrightarrow n\mid m
    \end{equation*}
    \begin{proof}
        Demostramos las dos implicaciones:
        \begin{description}
            \item [$\Longrightarrow)$] 
                Si $O(x)=n$, entonces no puede ser $m<n$, ya que si no el orden de $x$ no sería $n$ sino $m$, por lo que $m\geq n$. En cuyo caso, $\exists q,r\in \mathbb{N}$ de forma que:
                \begin{equation*}
                    m = nq + r \qquad \text{con\ }0\leq r < n 
                \end{equation*}
                Pero entonces:
                \begin{equation*}
                    1 = x^m = x^{nq+r} = x^{nq} x^r = x^r \stackrel{(\ast)}{\Longrightarrow} r = 0
                \end{equation*}
                Donde en $(\ast)$ hemos usado que $r<n$, ya que si $r$ no fuese $0$, tendríamos que $O(x)=r$.
            \item [$\Longleftarrow)$] Si $n\mid m$, entonces $\exists q\in \mathbb{N}$ de forma que $m = qn$, luego:
                \begin{equation*}
                    x^m = x^{qn} = {(x^n)}^{q} = 1^q = 1
                \end{equation*}
        \end{description}
    \end{proof}
\end{prop}

\begin{prop}\label{prop:orden_grupo}
    Sea $G$ un grupo, se verifica que:
    \begin{enumerate}
        \item $O(x)=1\Longleftrightarrow x=1$.
        \item $O(x)=O(x^{-1})$ $\forall x\in G$. 
        \item Si $O(x)=+\infty$ para cierto $x\in G$, entonces todas las potencias de $x$ son elementos distintos de $G$.
        \item Si $G$ es finito, entonces $O(x)\neq +\infty$ para todo $x\in G$. 
        \item Si $O(x) = n\in \mathbb{N}\setminus\{0\}$ para cierto $x\in G$, entonces $x$ tiene $n$ potencias distintas. Más aún, sean $p,q\in \mathbb{N}$ de forma que $x^p = x^q$ con $q>p$, entonces:
            \begin{equation*}
                x^{q-p} = 1 \Longleftrightarrow n\mid (q-p)
            \end{equation*}
    \end{enumerate}
    \begin{proof}
        Demostramos todas las propiedades:
        \begin{enumerate}
            \item Por doble implicación:
                \begin{description}
                    \item [$\Longleftarrow)$] Trivial.
                    \item [$\Longrightarrow)$] Si aplicamos la definición de $O(x)$ y de $x^1$:
                        \begin{equation*}
                            1 = x^1 = \prod_{i=1}^{1} x = x
                        \end{equation*}
                \end{description}
            \item % // TODO: Revisar pardillo de Arturo
                Distinguimos dos casos:
                \begin{itemize}
                    \item Fijado $x\in G$ con $O(x)=n$, entonces $x^n = 1$, por lo que:
                    \begin{equation*}
                        x^{-1} = x^{n-1}
                    \end{equation*}

                    Veamos en primer lugar que $O(x^{-1})\leq n$. Para ello, vemos que $\left(x^{-1}\right)^{n}=1$:
                    \begin{equation*}
                        \left(x^{-1}\right)^{n} = \left(x^{n-1}\right)^{n} = x^{n(n-1)} = {\left(x^n\right)}^{n-1}= 1
                    \end{equation*}

                    Veamos ahora que $O(x^{-1})\geq n$. Supongamos ahora que $O(x^{-1})=k$, entonces:
                    \begin{equation*}
                        (x^{-1})^k = 1 \Longrightarrow x^{(n-1)k} = 1 \Longrightarrow n\mid (n-1)k
                    \end{equation*}

                    Por tanto, como $n \nmid (n-1)$ y $\mcd(n,n-1)=1$, entonces $n\mid k$, por lo que $n\leq k=O(x^{-1})$. Por tanto, tenemos que:
                    \begin{equation*}
                        n\leq O(x^{-1})\leq n \Longrightarrow O(x^{-1}) = n
                    \end{equation*}

                    \item Si tenemos que $O(x) = +\infty$, por reducción al absurdo, supongamos que $\exists n\in \mathbb{N}\setminus\{0\}$ de forma que $O(x^{-1}) = n$.

                        Que $O(x)=+\infty$ significa que $\nexists m\in \mathbb{N}\setminus\{0\}$ de forma que $x^m = 1$.

                        Como $O(x^{-1})=n$, tenemos que:
                        \begin{equation*}
                            {\left(x^{-1}\right)}^{n} = 1 \Longrightarrow x = {\left(x^{-1}\right)}^{-1} = {\left(x^{-1}\right)}^{n-1}
                        \end{equation*}
                        De donde llegamos a que:
                        \begin{equation*}
                            x^n = {\left({\left(x^{-1}\right)}^{n-1}\right)}^{n} = {\left({\left(x^{-1}\right)}^{n}\right)}^{n-1} = 1^{n-1} = 1
                        \end{equation*}
                        Contradicción, puesto que $O(x)=+\infty$. Deducimos que si $O(x)=+\infty$, entonces ha de ser $O(x^{-1})=+\infty$. 
                \end{itemize}
            \item Por reducción al absurdo, supongamos que existen $p,q\in \mathbb{N}$ con $p< q$ de forma que $x^p = x^q$, luego:
                \begin{equation*}
                    x^{q-p} = 1
                \end{equation*}
                De donde deducimos que $O(x)<+\infty$, contradicción, luego $x^p \neq x^q$ para todo $p,q\in \mathbb{N}$ con $p\neq q$.
            \item Por reducción al absurdo, supongamos que $\exists x\in G$ con $O(x) = +\infty$. En este caso, podemos construir una aplicación $\phi:\mathbb{N}\to G$ dada por $\phi(n) = x^n$ $\forall n\in \mathbb{N}$. Esta aplicación es inyectiva gracias al punto 3, lo que contradice que $G$ sea un grupo finito. Concluimos que $O(x) \in \mathbb{N}\setminus\{0\}$ para todo $x\in G$.
            \item Si $O(x)=n$, consideramos la sucesión:
                \begin{equation*}
                    x, x^2, x^3, \ldots, x^{n-1}, x^n = 1
                \end{equation*}
                Si seguimos calculando potencias, está claro que se repetirá este patrón, por lo que tratamos de ver que todos los elementos de la sucesión son distintos entre sí. Por reducción al absurdo, supuesto que existen $p,q\in \mathbb{N}$ de forma que $p<q\leq n$ con $x^p = x^q$, entonces $x^{q-p} = 1$ con $q-p \leq n$, lo que contradice que $O(x) = n$.\\

                Para ver que si $p,q\in \mathbb{N}$ con $x^p = x^q$, entonces:
                \begin{equation*}
                    x^{q-p} = 1 \Longleftrightarrow n\mid (q-p)
                \end{equation*}
                Basta aplicar la Proposición~\ref{prop:divide_orden} con $m=q-p$.
        \end{enumerate}
    \end{proof}
\end{prop}

\begin{ejemplo}
    Mostramos ahora ejemplos de órdenes de ciertos elementos en distintos grupos, entendiendo que cuando consideramos conjuntos susceptibles de ser anillos (conjuntos con suma y multiplicación), si dejamos el $0$ en el conjunto consideramos el grupo con su suma ($e=0$) y que cuando quitamos el $0$ del conjunto consideramos el grupo con su multiplicación ($e=1$).
    \begin{enumerate}
        \item Si cogemos $x\neq 1$ en $\mathbb{Z}^\ast, \mathbb{Q}^\ast, \mathbb{R}^\ast$ con la multiplicación: $O(x)=+\infty$.
        \item Si consideramos $\mathbb{C}^\ast$ con su multiplicación: $O(i)=4$, ya que $i^4 = 1$.
        \item En $\mathbb{Z}_9$, $O\left(\overline{6}\right) = 3$:
            \begin{gather*}
                \overline{6} \neq \overline{0} \\
                \overline{6+6} = \overline{12} = \overline{3} \neq \overline{0} \\
                \overline{6+6+6} = \overline{18} = \overline{0}
            \end{gather*}
        \item En $\mathbb{Z}_{7}^{\ast}= \mathcal{U}(\mathbb{Z}_7):$ 
            \begin{itemize}
                \item $O\left(\overline{2}\right)=3$:
                    \begin{gather*}
                        \overline{2} \neq \overline{1} \\
                        \overline{2\cdot 2} = \overline{4} \neq \overline{1} \\
                        \overline{2\cdot 2\cdot 2} = \overline{8} = \overline{1}
                    \end{gather*}
                \item $O\left(\overline{3}\right)=6$.
                \begin{gather*}
                    \overline{3} \neq \overline{1} \\
                    \overline{3\cdot 3} = \overline{9} = \ol{2} \neq \overline{1} \\
                    \overline{3\cdot 3\cdot 3} = \overline{27} = \ol{6} \neq \overline{1} \\
                    \overline{3\cdot 3\cdot 3\cdot 3} = \overline{81} = \ol{3} \neq \overline{1}\\
                    \overline{3\cdot 3\cdot 3\cdot 3\cdot 3} = \overline{243} = \ol{5} \neq \ol{1}\\
                    \overline{3\cdot 3\cdot 3\cdot 3\cdot 3\cdot 3} = \overline{729} = \ol{1}
                \end{gather*}
            \end{itemize}
    \end{enumerate}
\end{ejemplo}

\section{Grupos diédricos $D_n$}
A continuación, estaremos interesados en el estudio de una familia\footnote{Donde con ``familia'' hacemos referencia a un conjunto de grupos que guardan cierta similitud entre ellos.} de grupos conocida como los ``grupos diédricos'', cuyo estudio se desarrollará a lo largo de la asignatura.

\subsection{Motivación}
Para entender estos grupos, conviene destacar la forma en la que surgieron ciertos objetos geométricos que luego fueron interesantes desde el punto de vista algebraico, por formar un grupo.\\

\begin{ejemplo}
    Si pensamos en un triángulo rectángulo (el menor polígono regular) sobre el plano centrado en el origen como el de la Figura~\ref{fig:triangulo}, donde hemos numerado los vértices del mismo, es interesante preguntarnos sobre las isometrías del plano en el plano que dejan invariante al mismo.
    \begin{figure}[H]
        \centering
       \begin{tikzpicture}
            % Dibujar el triángulo equilátero centrado en el origen
            \draw (-1,0) -- (1,0) -- (0,1.732) -- cycle;
            
            % Etiquetar los vértices
            \node[below] at (-1.3,0.1) {1};
            \node[below] at (1.3,0.1) {2};
            \node[above] at (0,1.732) {3};
        \end{tikzpicture}     
        \caption{Triángulo equilátero con centro en el origen de coordenadas.}
        \label{fig:triangulo}
    \end{figure}

    En Geometría II se vio que las únicas isometrías que podemos considerar en el plano son los giros y las simetrías axiales o centrales, por lo que procedemos a distinguir casos:
    \begin{description}
        \item [Giros.] Como vemos en la Figura~\ref{fig:giros_triangulo}, de forma intuitiva vemos que giros (pensando que todos son en sentido antihorario) que dejan el triángulo invariante solo hay 3:
            \begin{itemize}
                \item El giro de ángulo $\frac{2\pi}{3}$.
                \item El giro de ángulo $\frac{4\pi}{3}$.
                \item El giro de ángulo $2\pi$.
            \end{itemize}
            \begin{figure}[H]
                \centering
            \begin{tikzpicture}
                % Dibujar el triángulo equilátero
                \draw (-1,0) -- (1,0) -- (0,1.732) -- cycle;
                
                % Etiquetar los vértices
                \node[below=5pt] (A) at (-1.3,0.1) {1};
                \node[below=5pt] (B) at (1.3,0.1) {2};
                \node[above=5pt] (C) at (0,1.732) {3};
                
                % Dibujar flechas curvas EXTERIORES conectando las etiquetas
                \draw[-Stealth, bend right=30] (A) to (B);
                \draw[-Stealth, bend right=30] (B) to (C);
                \draw[-Stealth, bend right=30] (C) to (A);
            \end{tikzpicture} \hspace{1cm}
            \begin{tikzpicture}
                % Dibujar el triángulo equilátero
                \draw (-1,0) -- (1,0) -- (0,1.732) -- cycle;
                
                % Etiquetar los vértices
                \node[below=5pt] (A) at (-1.3,0.1) {1};
                \node[below=5pt] (B) at (1.3,0.1) {2};
                \node[above=5pt] (C) at (0,1.732) {3};
                
                % Dibujar flechas curvas EXTERIORES conectando las etiquetas
                \draw[-Stealth, bend left=30] (A) to (C);
                \draw[-Stealth, bend left=30] (B) to (A);
                \draw[-Stealth, bend left=30] (C) to (B);
            \end{tikzpicture} \hspace{1cm}
            \begin{tikzpicture}
                % Dibujar el triángulo equilátero
                \draw (-1,0) -- (1,0) -- (0,1.732) -- cycle;
                
                % Etiquetar los vértices
                \node[below=5pt] (A) at (-1.3,0.1) {1};
                \node[below=5pt] (B) at (1.3,0.1) {2};
                \node[above=5pt] (C) at (0,1.732) {3};
                
                % Dibujar flechas curvas EXTERIORES conectando las etiquetas
                \draw[-Stealth, loop left] (A) to (A);
                \draw[-Stealth, loop left] (B) to (B);
                \draw[-Stealth, loop left] (C) to (C);
            \end{tikzpicture}
            \caption{Todos los giros que dejan invariánte al triángulo.}
            \label{fig:giros_triangulo}
            \end{figure}
        \item [Simetrías.] Como vemos en la Figura~\ref{fig:simetrias_triangulo}, de forma intuitiva vemos que hay 3 simetrías axiales que dejan invariante al triángulo y que no hay ninguna simetría central que lo deje invariante:
            \begin{itemize}
                \item La simetría respecto a la mediatriz del segmento 2, 3.
                \item La simetría respecto a la mediatriz del segmento 3, 1.
                \item La simetría respecto a la mediatriz del segmento 1, 2.
            \end{itemize}
            Notemos la forma en la que hemos nombrado las rectas respecto a las cuales se hace la simetría: la recta $l_i$ contiene al vértice $i$-ésimo.
            \begin{figure}[H]
                \centering
                \begin{tikzpicture}
                    % Dibujar el triángulo equilátero
                    \draw (-1,0) -- (1,0) -- (0,1.732) -- cycle;
                    
                    % Etiquetar los vértices
                    \node[below=5pt](A) at (-1.3,0.1) {1};
                    \node[below=5pt](B) at (1.3,0.1) {2};
                    \node[above=5pt](C) at (0,1.732) {3};

                    % flechas
                    \draw[-Stealth, loop left] (A) to (A);
                    \draw[Stealth-Stealth, bend right=30] (B) to (C);
                    %\draw[-Stealth, bend left=30] (C) to (B);
                    
                    % Dibujar la mediatriz del segmento 1-2
                    \draw(-1,0) -- (0.7,1); % Línea punteada vertical en x=0
                    \node[below] at (0.1,0.6) {$l_1$};
                \end{tikzpicture}\hspace{1cm}
                \begin{tikzpicture}
                    % Dibujar el triángulo equilátero
                    \draw (-1,0) -- (1,0) -- (0,1.732) -- cycle;
                    
                    % Etiquetar los vértices
                    \node[below=5pt] at (-1.3,0.1) {1};
                    \node[below=5pt] at (1.3,0.1) {2};
                    \node[above=5pt] at (0,1.732) {3};

                    % flechas
                    \draw[-Stealth, loop left] (B) to (B);
                    \draw[Stealth-Stealth, bend left=30] (A) to (C);
                    
                    % Dibujar la mediatriz del segmento 1-2
                    \draw(1,0) -- (-0.7,1); % Línea punteada vertical en x=0
                    \node[below] at (0.1,0.6) {$l_2$};
                \end{tikzpicture}\hspace{1cm}
                \begin{tikzpicture}
                    % Dibujar el triángulo equilátero
                    \draw (-1,0) -- (1,0) -- (0,1.732) -- cycle;
                    
                    % Etiquetar los vértices
                    \node[below=5pt] at (-1.3,0.1) {1};
                    \node[below=5pt] at (1.3,0.1) {2};
                    \node[above=5pt] at (0,1.732) {3};

                    % flechas
                    \draw[-Stealth, loop left=10] (C) to (C);
                    \draw[Stealth-Stealth, bend right=30] (A) to (B);
                    
                    % Dibujar la mediatriz del segmento 1-2
                    \draw(0,-0.3) -- (0,1.732); % Línea punteada vertical en x=0
                    \node[below] at (0.35,0.8) {$l_3$};
                \end{tikzpicture}
                \caption{Todas las reflexiones que dejan invariante al triángulo.}
                \label{fig:simetrias_triangulo}
            \end{figure}
    \end{description}
\end{ejemplo}

Con el fin de estudiar las isometrías que mantienen polígonos regulares en el plano, conviene introducir las siguientes definiciones y notaciones:

\begin{definicion}[Permutación]
    Sea $X$ un conjunto, una permutación del mismo es cualquier aplicación biyectiva $f:X\rightarrow X$.\\

    Si $X$ es el conjunto $\{1,2,\ldots,n\}$, es usual notar:
    \begin{equation*}
        S_n = \Perm(X) = \{f:X\rightarrow X \mid f \text{\ es una permutación}\}
    \end{equation*}
\end{definicion}

\begin{definicion}[Ciclo]
    Sea $\{a_1,a_2,\ldots,a_m\}\subseteq \{1,2,\ldots,n\}$, un ciclo de longitud $m\leq n$ es una permutación $\sigma\in S_n$ de forma que:
    \begin{enumerate}
        \item $\sigma(a_i) = a_{i+1}$ para todo $i \in \{1,\ldots,m-1\}$.
        \item $\sigma(a_m) = a_1$.
        \item $\sigma(a_j) = a_j$ para todo $a_j \notin \{a_1,a_2,\ldots,a_m\}$.
    \end{enumerate}
    En dicho caso, representaremos a $\sigma$ por:
    \begin{equation*}
        \sigma = (a_1\ a_2\ \ldots\ a_m)
    \end{equation*}
\end{definicion}

\begin{observacion}
    Notemos que podemos notar a un ciclo de longitud $m$, $\sigma$, de $m$ formas distintas:
    \begin{equation*}
        \sigma = (a_1\ a_2\ \ldots\ a_m) = (a_2\ \ldots\ a_m\ a_1) = \ldots = (a_m\ a_1\ a_2\ \ldots\ a_{m-1})
    \end{equation*}
    De esta forma, el número de ciclos de longitud $m$ son todas las posibles combinaciones de los $m$ elementos entre $n$, pero como cada vez aparecen $m$:
    \begin{equation*}
        \dfrac{V_m^n}{m}
    \end{equation*}
    A los $2-$ciclos los llamaremos \underline{transposiciones}.
\end{observacion}

\begin{ejemplo}
    Para familiarizarnos con los ciclos, observamos que:
    \begin{itemize}
        \item En $S_3$, los ciclos de longitud 2 que podemos considerar son: $(1\ 2)$, $(1\ 3)$ y $(2\ 3)$. Estos se interpretan respectivamente como:
            \begin{itemize}
                \item Mantener el 3 fijo e intercambiar el 1 con el 2.
                \item Mantener el 2 fijo e intercambiar el 1 con el 3.
                \item Mantener el 1 fijo e intercambiar el 2 con el 3.
            \end{itemize}
        \item En $S_3$, los únicos ciclos de longitud 3 que podemos considerar son: $(1\ 2\ 3)$ y $(3\ 2\ 1)$, cuya definición debe estar clara.
    \end{itemize}
\end{ejemplo}

\begin{notacion}
    Es claro que no toda permutación es un ciclo. Sin embargo, hay ciertas permutaciones como por ejemplo la aplicación $\sigma:\{1,2,3,4\}\rightarrow \{1,2,3,4\}$ dada por:
    \begin{align*}
        \sigma(1) &= 2 \\
        \sigma(2) &= 1 \\
        \sigma(3) &= 4 \\
        \sigma(4) &= 3 \\
    \end{align*}
    Que restringida a $\{1,2\}$ da el ciclo $(1\ 2)$ y que restringida al $\{3, 4\}$ da el ciclo $(3\ 4)$. Será usual denotar permutaciones como esta por\footnote{Más adelante formalizaremos bien esta notación, aunque por ahora empecemos a usarla desde un punto de vista más intuitivo.}:
    \begin{equation*}
        \sigma = (1\ 2)(3\ 4)
    \end{equation*}
    Aprovechando la notación para los ciclos previamente definida, si por ejemplo extendemos $\sigma$ a $\{1,2,3,4,5\}$ definiendo:
    \begin{equation*}
        \sigma(5) = 5
    \end{equation*}
    Entonces, la notación para $\sigma$ será la misma: $(1\ 2)(3\ 4)$, ya que el $5$ ``no se mueve''.
\end{notacion}

\begin{ejemplo}
    Volviendo al ejemplo anterior del triángulo y de las isometrías que lo dejan invariante, si notamos por:
    \begin{itemize}
        \item $r$ al giro de ángulo $\frac{2\pi}{3}$.
        \item $s$ a la simetría axial cuya recta pasa por el vértice $1$.
    \end{itemize}
    Puede comprobarse de forma geométrica que a partir de composiciones de $r$ y de $s$ obtenemos los otros 4 movimientos restantes (notaremos la composición de aplicaciones por yuxtaposición, ya que estamos buscando un grupo con estas aplicaciones):
    \begin{itemize}
        \item El giro de ángulo $\frac{4\pi}{3}$ es $r^2 = rr$.
        \item El giro de ángulo $2\pi$ es $r^3$.
        \item La simetría respecto a la recta $l_2$ es $sr^2$.
        \item La simetría respecto a la recta $l_3$ es $sr$.
    \end{itemize}
    Notemos que el giro de ángulo $2\pi$ es la identidad, que es el elemento neutro para la composición, por lo que el elemento neutro del futuro grupo que definamos será $r^3$, que podemos denotar por $1$. Además, la composición de aplicaciones es una operación asociativa y se deja como ejercicio demostrar que cada elemento del conjunto:
    \begin{equation*}
        D_3 = \{1, r, r^2, s, sr, sr^2\}
    \end{equation*}
    Tiene un elemento simétrico respecto de la composición. Podemos ver que $(D_3,\circ,1)$ es un grupo.
\end{ejemplo}

\begin{ejemplo}
    Continuando con la motivación para los grupos diédricos, nos preguntamos ahora qué pasa si en vez de considerar las isometrías que mantienen invariante a un triángulo equilátero, consideramos las isometrías del plano que mantienen invariantes los vértices de un cuadrado sobre el plano; un cuadrado como el de la Figura~\ref{fig:cuadrado}.
    \begin{figure}[H]
        \centering
        \begin{tikzpicture}
            % Dibujar el triángulo equilátero centrado en el origen
            \draw (-1,-1) -- (1,-1) -- (1,1) -- (-1,1) -- cycle;
            
            % Etiquetar los vértices
            \node[below] at (-1.3,-1.1) {1};
            \node[below] at (1.3,-1.1) {2};
            \node[above] at (1.3,1.1) {3};
            \node[above] at (-1.3,1.1) {4};
        \end{tikzpicture}     
        \caption{Cuadrado con centro en el origen de coordenadas.}
        \label{fig:cuadrado}
    \end{figure}
    Es fácil ver que las únicas isometrías que dejan invariante al cuadrado son (Véase la Figura~\ref{fig:inv_cuadrado}):
    \begin{itemize}
        \item Los giros de ángulos $\frac{\pi}{2}$, $\pi$, $\frac{3\pi}{2}$ y $2\pi$.
        \item Las simetrías axiales respecto a las rectas:
            \begin{itemize}
                \item La recta que une los vértices 1 y 3.
                \item La recta que une los vértices 2 y 4.
                \item La recta que es mediatriz del segmento 1, 2.
                \item La recta que es mediatriz del segmento 2, 3.
            \end{itemize}
    \end{itemize}
    \begin{figure}
        \centering
       \begin{tikzpicture}
            % Dibujar el cuadrado centrado en el origen
            \draw (-1,-1) -- (1,-1) -- (1,1) -- (-1,1) -- cycle;
            
            % Etiquetar los vértices
            \node[below] (A) at (-1.3,-1.1) {1};
            \node[below] (B) at (1.3,-1.1) {2};
            \node[above] (C) at (1.3,1.1) {3};
            \node[above] (D) at (-1.3,1.1) {4};

            \draw[-Stealth, bend right=10] (A) to (B);
            \draw[-Stealth, bend right=10] (B) to (C);
            \draw[-Stealth, bend right=10] (C) to (D);
            \draw[-Stealth, bend right=10] (D) to (A);
            \end{tikzpicture}\hspace{.5cm}
       \begin{tikzpicture}
            % Dibujar el cuadrado centrado en el origen
            \draw (-1,-1) -- (1,-1) -- (1,1) -- (-1,1) -- cycle;
            
            % Etiquetar los vértices
            \node[below] (A) at (-1.3,-1.1) {1};
            \node[below] (B) at (1.3,-1.1) {2};
            \node[above] (C) at (1.3,1.1) {3};
            \node[above] (D) at (-1.3,1.1) {4};

            \draw[Stealth-Stealth, left] (-0.9,-0.9) to (0.9,0.9);
            \draw[Stealth-Stealth, left] (0.9,-0.9) to (-0.9,0.9);
            \end{tikzpicture}\hspace{.5cm}
       \begin{tikzpicture}
            % Dibujar el cuadrado centrado en el origen
            \draw (-1,-1) -- (1,-1) -- (1,1) -- (-1,1) -- cycle;
            
            % Etiquetar los vértices
            \node[below] (A) at (-1.3,-1.1) {1};
            \node[below] (B) at (1.3,-1.1) {2};
            \node[above] (C) at (1.3,1.1) {3};
            \node[above] (D) at (-1.3,1.1) {4};

            \draw[-Stealth, bend left=10] (B) to (A);
            \draw[-Stealth, bend left=10] (C) to (B);
            \draw[-Stealth, bend left=10] (D) to (C);
            \draw[-Stealth, bend left=10] (A) to (D);
            \end{tikzpicture}\hspace{.5cm}
       \begin{tikzpicture}
            % Dibujar el cuadrado centrado en el origen
            \draw (-1,-1) -- (1,-1) -- (1,1) -- (-1,1) -- cycle;
            
            % Etiquetar los vértices
            \node[below] (A) at (-1.3,-1.1) {1};
            \node[below] (B) at (1.3,-1.1) {2};
            \node[above] (C) at (1.3,1.1) {3};
            \node[above] (D) at (-1.3,1.1) {4};

            \draw[-Stealth, loop left] (A) to (A);
            \draw[-Stealth, loop left] (B) to (B);
            \draw[-Stealth, loop left] (C) to (C);
            \draw[-Stealth, loop left] (D) to (D);
            \end{tikzpicture}
       \begin{tikzpicture}
            % Dibujar el cuadrado centrado en el origen
            \draw (-1,-1) -- (1,-1) -- (1,1) -- (-1,1) -- cycle;
            
            % Etiquetar los vértices
            \node[below] (A) at (-1.3,-1.1) {1};
            \node[below] (B) at (1.3,-1.1) {2};
            \node[above] (C) at (1.3,1.1) {3};
            \node[above] (D) at (-1.3,1.1) {4};

            \draw[-Stealth, loop left] (A) to (A);
            \draw[-Stealth, loop left] (C) to (C);
            \draw[Stealth-Stealth, left=30] (0.9,-0.9) to (-0.9,0.9);

                % recta
                \draw(-1.1,-1.1) -- (1.1,1.1);
                \end{tikzpicture}\hspace{.4cm}
       \begin{tikzpicture}
            % Dibujar el cuadrado centrado en el origen
            \draw (-1,-1) -- (1,-1) -- (1,1) -- (-1,1) -- cycle;
            
            % Etiquetar los vértices
            \node[below] (A) at (-1.3,-1.1) {1};
            \node[below] (B) at (1.3,-1.1) {2};
            \node[above] (C) at (1.3,1.1) {3};
            \node[above] (D) at (-1.3,1.1) {4};

            \draw[-Stealth, loop left] (B) to (B);
            \draw[-Stealth, loop left] (D) to (D);
            \draw[Stealth-Stealth, left=30] (0.9,0.9) to (-0.9,-0.9);

                % recta
                \draw(-1.1,1.1) -- (1.1,-1.1);
                \end{tikzpicture}\hspace{.4cm}
       \begin{tikzpicture}
            % Dibujar el cuadrado centrado en el origen
            \draw (-1,-1) -- (1,-1) -- (1,1) -- (-1,1) -- cycle;
            
            % Etiquetar los vértices
            \node[below] (A) at (-1.3,-1.1) {1};
            \node[below] (B) at (1.3,-1.1) {2};
            \node[above] (C) at (1.3,1.1) {3};
            \node[above] (D) at (-1.3,1.1) {4};

            \draw[Stealth-Stealth, bend left=10] (B) to (A);
            \draw[Stealth-Stealth, bend right=10] (C) to (D);

                % recta
                \draw(0,-1.25) -- (0,1.25);
                \end{tikzpicture}\hspace{.4cm}
       \begin{tikzpicture}
            % Dibujar el cuadrado centrado en el origen
            \draw (-1,-1) -- (1,-1) -- (1,1) -- (-1,1) -- cycle;
            
            % Etiquetar los vértices
            \node[below] (A) at (-1.3,-1.1) {1};
            \node[below] (B) at (1.3,-1.1) {2};
            \node[above] (C) at (1.3,1.1) {3};
            \node[above] (D) at (-1.3,1.1) {4};

            \draw[Stealth-Stealth, bend left=10] (A) to (D);
            \draw[Stealth-Stealth, bend left=10] (C) to (B);

                % recta
                \draw(-1.25,0) -- (1.25,0);
                \end{tikzpicture}
        \caption{Giros y simetrías que dejan invariante al cuadrado}
        \label{fig:inv_cuadrado}
    \end{figure}
    Todos estos movimientos pueden verse como aplicaciones lineales $f:\mathbb{R}^2\rightarrow\mathbb{R}^2$ tal y como se hace en geometría o aprovecharnos de que todas ellas mantienen el cuadrado invariante, por lo que podemos pensar en ellas como si fueran permutaciones del conjunto $\{1,2,3,4\}$. Aprovechando esta dualidad, vemos que:
    \begin{itemize}
        \item El giro de ángulo $\frac{\pi}{2}$ es $(1\ 2\ 3\ 4)$.
        \item El giro de ángulo $\pi$ es $(1\ 3)(2\ 4)$.
        \item El giro de ángulo $\frac{3\pi}{2}$ es $(1\ 4\ 3\ 2)$.
        \item El giro de ángulo $2\pi$ es la identidad, $(1)$.
        \item La simetría respecto a la recta que une 1 y 3 es $(2\ 4)$.
        \item La simetría respecto a la recta que une 2 y 4 es $(1\ 3)$.
        \item La simetría respecto a la mediatriz de 1 y 2 es $(1\ 2)(3\ 4)$.
        \item La simetría respecto a la mediatriz de 2 y 3 es $(1\ 4)(2\ 3)$.
    \end{itemize}
    Dejamos como ejercicio hacer esta correspondencia (notar las isometrías como su correspondiente permutación) con los movimientos que teníamos en el triángulo. Si ahora hacemos como hicimos anteriormente con el triángulo y notamos por:
    \begin{itemize}
        \item $r$ al giro de ángulo $\frac{\pi}{2}$.
        \item $s$ a la reflexión respecto a la recta que pasa por el vértice $1$.
    \end{itemize}
    Podemos obtener los otros 6 movimientos (o permutaciones desde el punto de vista algebráico) con la composición de $r$ y $s$:
    \begin{itemize}
        \item $r^2$ es $(1\ 3)(2\ 4)$.
        \item $r^3$ es $(1\ 4\ 3\ 2)$.
        \item $r^4$ es $1$ (la aplicación identidad).
        \item $sr$ es $(1\ 4)(2\ 3)$.
        \item $sr^2$ es $(1\ 3)$.
        \item $sr^3$ es $(1\ 2)(3\ 4)$.
    \end{itemize}
    De esta forma, si consideramos el conjunto:
    \begin{equation*}
        D_4 = \{1, r, r^2, r^3, s, sr, sr^2, sr^3\}
    \end{equation*}
    Tenemos que $(D_4,\circ,1)$ es un grupo. Más aún, podemos completar su tabla de Cayley para observar cómo se comporta $\circ$ dentro de $D_4$:
    \begin{equation*}
        \begin{array}{c|cccccccc}
             \circ & 1 & r & r^2 & r^3 & s & sr & sr^2 & sr^3 \\
             \hline
                1 & 1 & r & r^2 & r^3 & s & sr & sr^2 & sr^3 \\
                r & r & r^2 & r^3 & 1 & sr^3 & s & sr & sr^2 \\
                r^2 & r^2 & r^3 & 1 & r & sr^2 & sr^3 & s & sr \\
                r^3 & r^3 & 1 & r & r^2 & sr & sr^2 & sr^3 & s \\
                s & s & sr & sr^2 & sr^3 & 1 & r & r^2 & r^3 \\
                sr & sr & sr^2 & sr^3 & s & r^3 & 1 & r & r^2 \\
                sr^2 & sr^2 & sr^3 & s & sr & r^2 & r^3 & 1 & r \\
                sr^3 & sr^3 & s & sr & sr^2 & r & r^2 & r^3 & 1 
        \end{array}
    \end{equation*}
\end{ejemplo}

\subsection{Definición y primeras propiedades}
Una vez comprendida la motivación de los grupos diédricos, estamos preparados para dar su definición. No demostraremos que, dado $n\in \mathbb{N}$, el conjunto de isometrías que dejan invariante al polígono regular de $n$ lados forma un grupo si consideramos sobre dicho conjunto la composición de aplicaciones, ya que no es interesante para esta asignatura.

Sin embargo, aceptaremos la definición como válida (animamos al lector a investigar más sobre los grupos diédricos y su definición) y procedemos a destacar las propiedades algebraicas de estos grupos, que es lo que nos interesa.

\begin{definicion}[Grupos diédricos $D_n$]
    Sea $D_n$ el conjunto de isometrías que dejan invariante al polígono regular de $n$ lados. Sabemos que $D_n$ tiene $2n$ elementos:
    \begin{itemize}
        \item $n$ rotaciones de ángulo $\frac{2k\pi}{n}$, con $k\in \{1,\ldots,n\}$.
        \item $n$ simetrías axiales:
            \begin{itemize}
                \item Si $n$ es par, tenemos:
                    \begin{itemize}
                        \item $\nicefrac{n}{2}$ simetrías respecto a las mediatrices.
                        \item $\nicefrac{n}{2}$ simetrías respecto a unir vértices opuestos.
                    \end{itemize}
                \item Si $n$ es impar, tenemos $n$ simetrías respecto a las mediatrices.
            \end{itemize}
    \end{itemize}
    Se verifica que $(D_n,\circ,1)$ es un grupo. Además, destacamos dos elementos suyos:
    \begin{itemize}
        \item $r$, la rotación de ángulo $\frac{2\pi}{n}$.
        \item $s$, la simetría axial respecto a la recta que pasa por el origen de coordenadas y el vértice nombrado 1.
    \end{itemize}
    De esta forma, todos los elementos de $D_n$ son:
    \begin{equation*}
        D_n = \left\{1, r, r^2, \ldots, r^{n-1}, s, sr, sr^2, \ldots, sr^{n-1}\right\}
    \end{equation*}
\end{definicion}

\begin{prop}\label{prop:reglas_diedricos}
    Dado $n\in \mathbb{N}$, en $D_n$ se cumple que:
    \begin{enumerate}
        \item $1,r,r^2,\ldots,r^{n-1}$ son todos distintos y $r^n =1$, es decir, $O(r)=n$.
        \item $s^2 = 1$. En particular, $O(s) = 2$.
        \item $s\neq r^i$, $\forall\ 0 \leq i \leq n-1$.
        \item $sr^i$ con $0\leq i\leq n-1$ son simetrías. 
        \item $sr^i \neq sr^j$ para todo $i\neq j$, con $i,j \in \{1,\ldots,n-1\}$.
        \item $sr = r^{-1}s$.
        \item $sr^i = r^{-i}s$.
    \end{enumerate}
    \begin{proof}
        Demostramos cada una de las propiedades:
        \begin{enumerate}
            \item La primera parte es compentencia de Geometría. Para la segunda, basta ver que $r^n$ es componer $n$ veces el giro de ángulo $\frac{2\pi}{n}$, que es lo mismo que considerar el giro de ángulo $n\cdot \frac{2\pi}{n} = 2\pi$, que es la identidad.
            \item Es competencia de Geometría.
            \item Es competencia de Geometría, que puede probarse de distintas formas:
                \begin{itemize}
                    \item Viendo que $s$ tiene puntos fijos y $r^i$ no.
                    \item Viendo que $s$ es un movimiento inverso y que $r^i$ es directo.
                \end{itemize}
            \item Es competencia de Geometría.
            \item Basta aplicar 1.
            \item[6, 7.] Son competencia de Geometría.
        \end{enumerate}
    \end{proof}
\end{prop}

Usaremos los resultados de la Proposición~\ref{prop:reglas_diedricos} con frecuencia, como las propiedades básicas de los grupos diédricos. Notemos que a partir de estas puede construirse la tabla de Cayley para cualquier grupo diédrico $D_n$.

\begin{ejercicio*}
    Construya la tabla de Cayley para $D_4$ y $D_5$ usando los resultados de la Proposición~\ref{prop:reglas_diedricos}.
\end{ejercicio*}

\section{Generadores de un grupo}
\begin{definicion}[Conjunto de generadores de un grupo]
    Sea $G$ un grupo, diremos que $S\subseteq G$ es un conjunto de generadores de $G$ si todo elemento $x\in G$ puede escribirse como producto finito de elementos de $S$ y de sus inversos. En dicho caso, notaremos: $G = \langle S \rangle $.\newline
    Si $S$ es un conjunto finito, $S= \{x_1,x_2,\ldots,x_n\}\subseteq G$, podemos escribir:
    \begin{equation*}
        G = \langle x_1,x_2,\ldots,x_n \rangle 
    \end{equation*}
    Y diremos que $G$ es finitamente generado.\\
    \noindent
    Si $S$ está formado solo por un elemento, diremos que $G$ es un grupo cíclico.
\end{definicion}

\begin{observacion}
    Sea $G$ un grupo y $S\subseteq G$, equivalen:
    \begin{enumerate}
        \item[$i)$] $S$ es un conjunto de generadores de $G$.
        \item[$ii)$] Dado $x\in G$, $\exists x_1,x_2,\ldots,x_p \in S$ de forma que:
            \begin{equation*}
                x = x_1^{\gamma_1}x_2^{\gamma_2} \ldots x_p^{\gamma_p} \qquad \gamma_i \in  \mathbb{Z}, \quad i \in \{1,\ldots,p\}
            \end{equation*}
    \end{enumerate}
\end{observacion}

\begin{ejemplo}
    Como ejemplos a destacar, vemos que:
    \begin{enumerate}
        \item $\mathbb{Z} = \langle 1 \rangle $ si pensamos en $(\mathbb{Z}, +, 0)$, ya que dado $x\in \mathbb{Z}$:
            \begin{itemize}
                \item Si $x>0$, entonces:
                    \begin{equation*}
                        x = \underbrace{1+1+\ldots +1}_{x \text{\ veces}}
                    \end{equation*}
                \item Si $x<0$, entonces ($-1$ es el simétrico de $1$):
                    \begin{equation*}
                        x = \underbrace{-1-1-\ldots-1}_{x \text{\ veces}}
                    \end{equation*}
                \item Si $x=0$, entonces: $x = 1 - 1$.
            \end{itemize}
        \item $D_n = \langle r,s \rangle $, ya que $D_n = \{1, r, r^2, \ldots, r^{n-1}, s, sr, sr^2, \ldots, sr^{n-1}\}$.
    \end{enumerate}
\end{ejemplo}

\begin{definicion}[Presentación de un grupo]
    Sea $G$ un grupo y $S\subseteq G$, si $G=\langle S \rangle $ y existe un conjunto de relaciones $R_1,R_2,\ldots,R_m$ (igualdades entre elementos de $S$, $\{1\}$ y los elementos simétricos de $S$) tal que cualquier relación entre los elementos de $S$ puede deducirse de estas, entonces, decimos que estos generadores y relaciones constituyen una presentación de $G$, notado:
    \begin{equation*}
        G=\langle S \mid R_1,R_2,\ldots, R_n \rangle 
    \end{equation*}
\end{definicion}

\begin{ejemplo}
    Veamos algunos ejemplos de presentaciones, observando que dar una presentación es equivalente a dar la definición del propio grupo, ya que a partir de la presentación pueden deducirse todos los elementos del grupo y las relaciones que estos guardan entre sí.
    \begin{enumerate}
        \item En el diédrico $D_n$, tenemos que:
        \begin{equation*}
            D_n = \langle r,s \mid rs=sr^{-1}, r^n = 1, s^2 = 1 \rangle 
        \end{equation*}
        \item $D_1 := \langle s\mid s^2 = 1 \rangle$.

            En este caso, vemos que $D_1 = \{1,s\}$.
        \item $D_2 := \langle r,s\mid r^2 = s^2 = 1, sr=rs \rangle$.

            Ahora, tenemos: $D_2 = \{1, r, s, rs\}$.
        \item $C_n = \langle x \mid x^n = 1 \rangle $ es un grupo cíclico de orden $n$.
            
            Vemos que: $C_n = \{1, x, x^2, x^3, \ldots, x^{n-1}\}$
        \item $V^{\text{abs}} = \langle x,y \mid x^2=1,y^2 = 1, {(xy)}^{2}=1 \rangle $ es el grupo de Klein abstracto.

            En primer lugar, sabemos que $\{1,x,y\}\subseteq V^{\text{abs}}$. Como $x$ e $y$ son de orden 2, sabemos que $x^{-1} = x$ y que $y^{-1} = y$. Además, vemos que $xy\in V^{\text{abs}}$ y que:
            \begin{equation*}
                {(xy)}^{2} = 1 \Longleftrightarrow xyxy = 1 \Longleftrightarrow xy = yx
            \end{equation*}
            Por lo que $xy$ también está en $V^{\text{abs}}$, con ${(xy)}^{-1} = yx$. Vemos que no hay más elementos que puedan estar en $V^{\text{abs}}$, con lo que:
            \begin{equation*}
                V^{\text{abs}} = \{1,x,y,xy\}
            \end{equation*}
            Observamos que el grupo nos recuerda a $D_2$.
        \item $Q_2^{\text{abs}} = \langle x,y\mid x^4 = 1, y^2 = x^2, yxy^{-1} = x^{-1} \rangle $.

            Inicialmente, $\{1,x,y\}\subseteq Q^{\text{abs}}_2$. De la primera relación vemos que también tenemos $\{x^2,x^3\}\subseteq Q^{\text{abs}}_2$. Reescribimos la última relación, para buscar más elementos de forma cómoda:
            \begin{equation*}
                yxy^{-1} = x^{-1} \Longleftrightarrow yx = x^{-1}y
            \end{equation*}
            Como $yx$ no guarda ninguna relación con $x$ e $y$, sabemos que también está en el grupo, junto con $yx^2$ y $yx^3$. De esta forma:
            \begin{equation*}
                Q^{\text{abs}}_2 = \{1,x,x^2,x^3,y,yx,yx^2,yx^3\}
            \end{equation*}
    \end{enumerate}
\end{ejemplo}

\begin{ejemplo}
    Las similitudes que hemos encontrado entre distintos grupos como entre $V^{\text{abs}}$ y $D_2$ las formalizaremos con ayuda de un concepto algebraico que luego definiremos, pero merece la pena destacar ahora una similaritud entre $Q_2^{\text{abs}}$, el grupo de los cuaternios $Q_2 = \{\pm 1, \pm i,\pm j,\pm k\}$ y unos elementos del grupo $\SL_2(\mathbb{C})$. Para familiarizarnos con los cuaternios, estos cumplen que:
    \begin{gather*}
        i^2 = j^2 = k^2 = -1 \\
        \begin{array}{ccc}
            ij = k & jk = i & ki = j \\
            ji = -k & kj = -i & ik = -j
        \end{array}
    \end{gather*}
    Productos que pueden recordarse observando la Figura~\ref{fig:cuaternios}
    \begin{figure}[H]
        \centering
        \begin{tikzpicture}
            % Etiquetar los vértices
            \node[below=5pt] (A) at (-1,0.1) {$i$};
            \node[below=5pt] (B) at (1,0.1) {$j$};
            \node[above=5pt] (C) at (0,0.7) {$k$};
            
            % Dibujar flechas curvas EXTERIORES conectando las etiquetas
            \draw[-Stealth, bend right=30] (A) to (B);
            \draw[-Stealth, bend right=30] (B) to (C);
            \draw[-Stealth, bend right=30] (C) to (A);
        \end{tikzpicture} \hspace{1cm}
        \caption{Dirección en la que se multiplican los cuaternios de forma positiva.}
        \label{fig:cuaternios}
    \end{figure}
    Se deja como ejercicio ver en qué forma podemos entender que los grupos $Q_2$, $Q_2^{\text{abs}}$ y el subconjunto de matrices de $\SL_2(\mathbb{C})$ con la operación heredada del mismo:
    \begin{equation*}
        C = \left\{\begin{array}{c}
            \left(\begin{array}{cc}
                1 & 0 \\
                0 & 1 
            \end{array}\right), \left(\begin{array}{cc}
                i & 0 \\
                0 & -i 
            \end{array}\right),\left(\begin{array}{cc}
                0 & -1 \\
                1 & 0 
            \end{array}\right),\left(\begin{array}{cc}
                -1 & 0 \\
                0 & -1 
            \end{array}\right) \\
            \left(\begin{array}{cc}
                -i & 0 \\
                0 & i 
            \end{array}\right) , \left(\begin{array}{cc}
                0 & -i \\
                -i & 0 
            \end{array}\right), \left(\begin{array}{cc}
                0 & 1 \\
                -1 & 0 
            \end{array}\right), \left(\begin{array}{cc}
                0 & i \\
                i & 0 
            \end{array}\right)
        \end{array}\right\} \subseteq \SL_2(\mathbb{C})
    \end{equation*}
    Si pensamos en relacionar los elementos de la Tabla~\ref{tab:elementos_relacionados}.

    \begin{table}[H]
    \centering
    \begin{tabular}{c|c|c}
        $Q_2^{\text{abs}}$ & C & $Q_2$ \\
        \hline
        1 & $\left(\begin{array}{cc}
            1 & 0 \\
            0 & 1 
    \end{array}\right)$  & 1 \\
        $x$ & $\left(\begin{array}{cc}
            i & 0 \\
            0 & -i 
    \end{array}\right)$& $i$ \\
        $y$ & $\left(\begin{array}{cc}
            0 & -1 \\
            1 & 0 
    \end{array}\right)$& $j$ \\
        $x^2$ & $\left(\begin{array}{cc}
            -1 & 0 \\
            0 & -1 
    \end{array}\right)$ & $-1$ \\
        $x^3$ & $\left(\begin{array}{cc}
            -i & 0 \\
            0 & i 
    \end{array}\right)$ & $-i$ \\
        $xy$ & $\left(\begin{array}{cc}
            0 & -i \\
            -i & 0 
    \end{array}\right)$ & $k$ \\
        $x^2y$ & $\left(\begin{array}{cc}
            0 & 1 \\
            -1 & 0 
    \end{array}\right)$ & $-j$ \\
        $x^3y$ & $\left(\begin{array}{cc}
            0 & i \\
             i &  0
     \end{array}\right)$ & $-k$
    \end{tabular}
    \caption{Elementos que se relacionan.}
    \label{tab:elementos_relacionados}
    \end{table}
\end{ejemplo}

\section{Grupos Simétricos $S_n$}\label{sec:grupos_simetricos}
Recordamos que dado un conjunto $X$, podemos considerar el conjunto de todas sus permutaciones:
\begin{equation*}
    S(X) = \{f:X\rightarrow X\mid f \text{\ biyectiva}\}
\end{equation*}

\begin{definicion}[Grupos Simétricos $S_n$]
    Dado $n\in \mathbb{N}$, consideramos $X=\{1,2,\ldots,n\}$ y definimos $S_n = S(X)$, el conjunto de todas las permutaciones de $X$. Se verifica que $S_n$ junto con la operación de composición de aplicaciones es un grupo:
    \begin{itemize}
        \item La composición de aplicaciones es asociativa.
        \item La aplicación $id:X\rightarrow X$ es el elemento neutro.
        \item Como las permutaciones son biyecciones, cada una tiene su elemento simétrico.
    \end{itemize}
    Llamaremos a $(S_n,\circ,id)$ el $n-$ésimo grupo simétrico, que recordamos tiene orden:
    \begin{equation*}
        |S_n| = n!
    \end{equation*}
\end{definicion}

\begin{notacion}
Estaremos interesados en ver cómo se comportan de forma algebraica las permutaciones de conjuntos de $n$ elementos, por lo que tendremos que conocer en cada caso cuáles son las aplicaciones con las que estamos trabajando.

Para abreviar, en muchos casos usaremos la notación matricial de las permutaciones. Sea $\sigma\in S_n$, sabemos que dar $\sigma$ es equivalente a dar $\sigma(a)$ para cualquier $a\in X$. De esta forma, podemos dar una matriz $2\times n$ de la forma:
\begin{equation*}
    \left(\begin{array}{ccccc}
        1 & 2 & 3 & \cdots & n \\
        \sigma(1) & \sigma(2) & \sigma(3) & \cdots & \sigma(n)
    \end{array}\right)
\end{equation*}
Observemos que, conocida la matriz anterior, conocemos $\sigma$.
\end{notacion}

\begin{ejemplo}
En este ejemplo, vemos los grupos simétricos más pequeños:
\begin{enumerate}
    \item Si consideramos $S_0$, son todas las permutaciones del $\emptyset $ en el $\emptyset $, que solo hay una: $\sigma:\emptyset\rightarrow\emptyset $.
    \item Si consideramos $S_1$, solo hay una permutación: $id:\{1\}\rightarrow\{1\}$.
    \item En $S_2$, tenemos $S_2 = \{\sigma_1, \sigma_2\}$, con:
        \begin{equation*}
            \sigma_1 = id = \left(\begin{array}{cc}
                1 & 2 \\
                1 & 2 
            \end{array}\right), \quad \sigma_2 = \left(\begin{array}{cc}
                1 & 2 \\
                2 & 1 
            \end{array}\right)
        \end{equation*}
        Hasta ahora, todos estos grupos son abelianos.
    \item En $S_3$, tenemos:
        \begin{multline*}
            S_3 = \left\{\left(\begin{array}{ccc}
                1 & 2 & 3 \\
                1 & 2 & 3
            \end{array}\right), \left(\begin{array}{ccc}
                1 & 2 & 3 \\
                1 & 3 & 2
            \end{array}\right), \left(\begin{array}{ccc}
                1 & 2 & 3 \\
                2 & 1 & 3
            \end{array}\right), \left(\begin{array}{ccc}
                1 & 2 & 3 \\
                2 & 3 & 1
            \end{array}\right), \right. \\ \left. \left(\begin{array}{ccc}
                1 & 2 & 3 \\
                3 & 1 & 2
            \end{array}\right), \left(\begin{array}{ccc}
                1 & 2 & 3 \\
                3 & 2 & 1
            \end{array}\right)\right\}
        \end{multline*}
        Que ya es un ejemplo de grupo simétrico no abeliano, ya que si tomamos:
        \begin{equation*}
            \sigma = \left(\begin{array}{ccc}
                1 & 2 & 3 \\
                1 & 3 & 2
            \end{array}\right), \quad \tau = \left(\begin{array}{ccc}
                1 & 2 & 3 \\
                2 & 1 & 3
            \end{array}\right)
        \end{equation*}
        Vemos que $\sigma\tau \neq \tau\sigma$:
        \begin{equation*}
            \sigma\tau = \left(\begin{array}{ccc}
                1 & 2 & 3 \\
                3 & 1 & 2
            \end{array}\right) \neq \left(\begin{array}{ccc}
                1 & 2 & 3 \\
                2 & 3 & 1
            \end{array}\right) = \tau\sigma
        \end{equation*}
        De esta forma, acabamos de probar que $S_n$ con $n\geq 3$ no es abeliano, ya que si estamos en $S_n$, podemos considerar las extensiones de $\sigma$ y $\tau$ a $S_n$:
        \begin{equation*}
            \sigma=\left(\begin{array}{cccccc}
                    1 & 2 & 3 & 4 & \cdots & n \\
                    1 & 3 & 2 & 4 & \cdots & n 
            \end{array}\right), \quad \tau = \left(\begin{array}{cccccc}
                    1 & 2 & 3 & 4 & \cdots & n \\
                    2 & 1 & 3 & 4 & \cdots & n
            \end{array}\right)
        \end{equation*}
        Y tendremos que $\sigma\tau \neq \tau\sigma$.
\end{enumerate}
\end{ejemplo}

\begin{ejemplo}
    Sean $s_1,s_2\in S_7$ dadas por:
    \begin{equation*}
        s_2 = \left(\begin{array}{ccccccc}
                1 & 2 & 3 & 4 & 5 & 6 & 7 \\
                5 & 7 & 6 & 2 & 1 & 4 & 3
        \end{array}\right), \quad s_1 = \left(\begin{array}{ccccccc}
                1 & 2 & 3 & 4 & 5 & 6 & 7 \\
                3 & 2 & 4 & 5 & 1 & 7 & 6
        \end{array}\right)
    \end{equation*}
    Se pide calcular $s_1s_2$, $s_2s_1$ y $s_2^2$.
    \begin{align*}
        s_1s_2 &= \left(\begin{array}{ccccccc}
                1 & 2 & 3 & 4 & 5 & 6 & 7 \\
                1 & 6 & 7 & 2 & 3 & 5 & 4
        \end{array}\right) \\
        s_2s_1 &= \left(\begin{array}{ccccccc}
                1 & 2 & 3 & 4 & 5 & 6 & 7 \\
                6 & 7 & 2 & 1 & 5 & 3 & 4
        \end{array}\right) \\
        s_2^2 &= \left(\begin{array}{ccccccc}
                1 & 2 & 3 & 4 & 5 & 6 & 7 \\
                1 & 3 & 4 & 7 & 5 & 2 & 6
        \end{array}\right)
    \end{align*}
\end{ejemplo}

\begin{prop}\label{prop:propiedades_sn}
    Se verifica que:
    \begin{enumerate}
        \item Dado $\sigma\in S_n$, existe $m\in \mathbb{N}$ de forma que $\sigma^{m+1}(x) = x$, $\forall x\in X=\{1,\ldots,n\}$.
        \item Todo ciclo es una permutación.
        \item El orden de un ciclo de longitud $m$ es $m$.
        \item Si $\sigma = (x_1\ x_2\ \ldots\ x_{m-1}\ x_m)$, entonces: $\sigma^{-1} = (x_m\ x_{m-1}\ \ldots\ x_2\ x_1)$.
    \end{enumerate}
    \begin{proof}
        Demostramos cada propiedad:
        \begin{enumerate}
            \item Por la Proposición~\ref{prop:orden_grupo}, como $S_n$ es un grupo finito, sabemos que $\exists n\in \mathbb{N}\setminus\{0\}$ de forma que $O(\sigma)=n$. Tomando $m=n-1$, tenemos que:
                \begin{equation*}
                    \sigma^{m+1}(x) = \sigma^{n}(x) = x \qquad \forall x\in X
                \end{equation*}

                % Consideramos la sucesión de elementos $\{\sigma^n(x)\}$:
                % \begin{equation*}
                %     x,\sigma(x),\sigma^2(x),\sigma^3(x),\ldots
                % \end{equation*}
                % Como $\sigma^n(x)\in X$ para todo $n\in \mathbb{N}$ y $X$ es un conjunto finito de $n$ elementos, tarde o temprano ha de repetirse algún valor en la sucesión, es decir, existen $k>j\geq 0$ de forma que:
                % \begin{equation*}
                %     \sigma^k(x) = \sigma^j(x)
                % \end{equation*}
                % Si ahora tomamos $m=k-j-1$, tenemos que:
                % \begin{equation*}
                %     \sigma^k(x) = \sigma^{j+m+1}(x) = \sigma^j(x)
                % \end{equation*}
                % Aplicando $\sigma^{-j}$ a ambos lados, llegamos a que:
                % \begin{equation*}
                %     \sigma^{m+1}(x) = \sigma^{j-j}(x) = x
                % \end{equation*}
            \item Se tiene directamente por la definición de ciclo.
            \item Sea $\sigma\in S_n$ un ciclo de longitud $m$:
                \begin{equation*}
                    \sigma = (x_1\ x_2\ \ldots\ x_m) \qquad x_1,x_2,\ldots,x_m \in X
                \end{equation*}
                Queremos ver que $O(\sigma)=m$. Para ello:
                \begin{itemize}
                    \item En primer lugar, veamos que $\sigma^m = 1$:
                        \begin{itemize}
                            \item Si $x\in X$ con $x\neq x_i$ para todo $i \in \{1,\ldots,m\}$, entonces $\sigma(x)=x$, luego:
                                \begin{equation*}
                                    \sigma^m(x) = \sigma^{m-1}(\sigma(x)) = \sigma^{m-1}(x) = \sigma^{m-2}(\sigma(x)) =  \ldots = x
                                \end{equation*}
                            \item Si ahora consideramos $x_i$ con $i \in \{1,\ldots,m\}$, tendremos que:
                                \begin{equation*}
                                    x_i \underbrace{\stackrel{\sigma}{\longmapsto} x_{i+1} \stackrel{\sigma}{\longmapsto} \ldots \stackrel{\sigma}{\longmapsto}x_{m-1}\stackrel{\sigma}{\longmapsto}}_{\sigma^{m-i}} x_m \underbrace{\stackrel{\sigma}{\longmapsto} x_1 \stackrel{\sigma}{\longmapsto}  \ldots \stackrel{\sigma}{\longmapsto} x_{i-1} \stackrel{\sigma}{\longmapsto}}_{\sigma^{i}} x_i
                                \end{equation*}
                        \end{itemize}
                        Luego: $1=\sigma^{m-i}\sigma^{i} = \sigma^{m-i+i} = \sigma^m$
                    \item Supongamos ahora que existe $k<m$ de forma que $\sigma^k = 1$, esto significaría que $\sigma^k(x_1) = x_1$, pero como $\sigma$ es un ciclo de longitud $m$, se tiene que $\sigma^k(x_1) = x_k$ y $x_k \neq x_1$, contradicción, con lo que $k\geq m$.
                \end{itemize}
            \item Recordamos por la definción de ciclo que si $\sigma=(a_1\ a_2\ \ldots\ a_{m-1}\ a_m)$, entonces se ha de cumplir que:
                \begin{align*}
                    \sigma(x) &= x \qquad x\neq x_i, \quad i \in \{1,\ldots,m\} \\
                    \sigma(x_i) &= x_{i+1} \qquad i \in \{1,\ldots,m-1\} \\
                    \sigma(x_m) &= x_1
                \end{align*}
                Si vemos $\sigma$ como aplicación y tratamos de buscarle su aplicación inversa $\sigma^{-1}$, esta ha de cumplir que:
                \begin{align*}
                    \sigma^{-1}(x) &= x \qquad x\neq x_i, \quad i \in \{1,\ldots,m\} \\
                    \sigma^{-1}(x_{i+1}) &= x_i \qquad i \in \{1,\ldots,m-1\} \\
                    \sigma^{-1}(x_1) &= x_m
                \end{align*}
                Sin embargo, vemos que entonces $\sigma^{-1}$ también es un ciclo:
                \begin{equation*}
                    \sigma^{-1} = (x_m\ x_{m-1}\ \ldots\ x_2\ x_1)
                \end{equation*}
        \end{enumerate}
    \end{proof}
\end{prop}

Con el siguiente teorema veremos que los ciclos son una parte interesante de los grupos simétricos, tanto que cualquier permutación pueda expresarse como una composición de ciertos ciclos de longitud mayor o igual que 2. Para ello, será necesario primero realizar una definición:

\begin{definicion}[Ciclos disjuntos]
    Sean $\sigma_1,\sigma_2\in S_n$ ciclos, decimos que son disjuntos si no existe $i \in X=\{1,2,\ldots,n\}$ de forma que:
    \begin{equation*}
        \sigma_1(i) = j, \quad \sigma_2(i) = k \quad \text{con\ } j,k \in X, i\neq j \neq k \neq i
    \end{equation*}
    Es decir, si no hay ningún elemento que se mueva en ambos ciclos.
\end{definicion}

\begin{ejemplo}
    Ejemplos de ciclos disjuntos son:
    \begin{equation*}
        \sigma_1 = (1\ 3\ 5), \quad \sigma_2 = (2\ 4\ 6), \quad \sigma_3 = (7\ 8)
    \end{equation*}
    Un ejemplo de dos ciclos que no son disjuntos son:
    \begin{equation*}
        \tau_1 = (1\ 3\ 5\ 8), \quad \tau_2 = (2\ 4\ 5\ 9)
    \end{equation*}
    Ya que $\tau_1(5) = 8$ y $\tau_2(5) = 9$, con $5\neq 8 \neq 9 \neq 5$. Es decir, el 5 se mueve en ambos ciclos.
\end{ejemplo}

\begin{teo}\label{teo:ciclos_disjuntos}
    Toda permutación $\sigma\in S_n$ con $\sigma\neq 1$ se expresa en la forma:
    \begin{equation*}
        \sigma = \gamma_1\gamma_2 \ldots \gamma_k
    \end{equation*}
    siendo los $\gamma_i$ con $i \in \{1,\ldots,k\}$ ciclos disjuntos de longitud mayor o igual que 2. Además, dicha descomposición es única, salvo el orden de los factores.
    \begin{proof}
        Supuesto que estamos trabajando con permutaciones sobre el conjunto $X=\{1,2,\ldots,n\}$, sea $\sigma\in S_n$ con $\sigma\neq 1$, definimos la relación:
        \begin{equation*}
            yRx \Longleftrightarrow \exists m\in \mathbb{Z} \mid y = \sigma^m(x)
        \end{equation*}
        Que es una relación de equivalencia:
        \begin{itemize}
            \item \underline{Propiedad reflexiva}. Se tiene gracias a la Proposición~\ref{prop:propiedades_sn}.
            \item \underline{Propiedad simétrica}. Sean $x,y\in X$ de forma que $yRx$, tenemos que $\exists m\in \mathbb{Z}$ de forma que $y=\sigma^m(x)$, pero entonces:
                \begin{equation*}
                    \sigma^{-m}(y) = \sigma^{-m}(\sigma^{m}(x)) = x \Longrightarrow xRy
                \end{equation*}
            \item \underline{Propiedad transitiva}. Sean $x,y,z\in X$ de forma que $yRx$ y que $zRx$, entonces: $\exists p,q\in \mathbb{Z}$ de forma que:
                \begin{equation*}
                    \left.\begin{array}{r}
                        y = \sigma^p(x) \\
                        z = \sigma^q(y)
                    \end{array}\right\} \Longrightarrow z = \sigma^q(\sigma^p(x)) = \sigma^{p+q}(x) \Longrightarrow zRx
                \end{equation*}
        \end{itemize}
        De esta forma, dado $x\in X$, podemos considerar su clase de equivalencia:
        \begin{equation*}
            \overline{x} = \{\sigma^{m}(x) \mid m\in \mathbb{Z}\} \in X/R
        \end{equation*}
        Que es un conjunto finito, ya que gracias a la Proposición~\ref{prop:propiedades_sn}, existe $m\in \mathbb{N}$ de forma que $\sigma^{m+1}(x)=x$, con lo que:
        \begin{equation*}
            C_x = \overline{x} = \{x, \sigma(x), \sigma^2(x), \ldots, \sigma^{m}(x)\}
        \end{equation*}
        Si consideramos ahora el ciclo:
        \begin{equation*}
            \gamma_x = (x\ \sigma(x)\ \sigma^2(x)\ \cdots\ \sigma^m(x)) \in S_n
        \end{equation*} 
        Tenemos que:
        \begin{equation*}
            \gamma_x(y) = \left\{\begin{array}{cr}
                    \sigma(y) & \text{si\ } y \in C_x \\
                    y & \text{si\ } y\notin C_x
            \end{array}\right.
        \end{equation*}
        De esta forma, tenemos una partición de $X$ en clases de equivalencia, cada una de las $C_x$ con $x\in X$, que llevan asociado un ciclo $\gamma_x$.\\

        \begin{enumerate}
            \item Sean $\overline{i},\overline{j}\in X/R$ con $\overline{i}\neq \overline{j}$, entonces los elementos que se mueven en $\gamma_i$ son los elementos de $C_i$, mientras los elementos que se mueven en $\gamma_j$ son los de $C_j$. Como se tiene que $C_i\cap C_j = \emptyset $ por ser $C_i$ y $C_j$ clases de equivalencia distintas, llegamos a que $\gamma_i$ y $\gamma_j$ son ciclos disjuntos, para $\overline{i}\neq \overline{j}$.
            \item Sea $\tau = \gamma_1\gamma_2\ldots\gamma_n$, sea $y\in X$, entonces:
                \begin{equation*}
                    \tau(y) = \gamma_1\gamma_2\ldots\gamma_n(y) = \gamma_1\gamma_2\ldots\gamma_y(y) = \gamma_1\gamma_2\ldots\gamma_{y-1}(\sigma(y)) = \gamma_1(\sigma(y)) = \sigma(y)
                \end{equation*}
                Ya que anteriormente vimos que:
                \begin{equation*}
                    \gamma_j(y) = \left\{\begin{array}{cr}
                            \sigma(y) & \text{si\ } y\in C_j \\
                            y & \text{si\ } y\notin C_j \\
                    \end{array}\right. \qquad \forall j\in X
                \end{equation*}
                Y se verifica que $y,\sigma(y)\in C_y$. Por tanto, tenemos que $\tau = \sigma$. Si ahora despreciamos de la expresión de $\tau$ los ciclos de longitud menor que 2 (la identidad), la permutación $\sigma$ no cambia y tenemos que $\sigma$ se expresa como producto de ciclos disjuntos (por el apartado 1) de longitud mayor o igual que 2.\qedhere
                % // TODO: Hacer unicidad de descomposición
            % \item Supuesto que tenemos dos descomposiciones de $\sigma$ en ciclos disjuntos:
            %     \begin{equation*}
            %         \sigma &= \gamma_1\gamma_2 \ldots \gamma_p \qquad 
            %         \sigma &= \alpha_1\alpha_2 \ldots \alpha_q
            %     \end{equation*}
            %     Para la unicidad: Supongamos que hay otra descomposición en ciclos y se llega a que en realidad es una renumeración de las anteriores porque se llega a que las $C$s son las mismas viendo las imágenes.
        \end{enumerate}
    \end{proof}
\end{teo}

\begin{notacion}
    A partir del Teorema~\ref{teo:ciclos_disjuntos}, podemos introducir una nueva notación basada en los ciclos disjuntos. Dado $\sigma\in S_n$, como existe una única descomposición en ciclos disjuntos:
    \begin{equation*}
        \sigma=\gamma_1\gamma_2\ldots\gamma_k
    \end{equation*}
    Teníamos una notación estandar para cada ciclo. Ahora podemos notar $\sigma$ como el producto de todas esas notaciones.
\end{notacion}

Como acabamos de decir, a partir del Teorema~\ref{teo:ciclos_disjuntos}, podremos notar a las permutaciones como su descomposición en ciclos disjuntos. Sin embargo, merece la pena preguntarse sobre el orden de los ciclos en esta descomposición, pregunta a la que contestamos con la siguiente proposición:

\begin{prop}
    Se verifican:
    \begin{enumerate}
        \item Si $\gamma_1,\gamma_2\in S_n$ son dos ciclos disjuntos, entonces:
            \begin{equation*}
                \gamma_1\gamma_2 = \gamma_2\gamma_1
            \end{equation*}
            Es decir, el producto de ciclos disjuntos es conmutativo.
        \item Sea $\sigma\in S_n$ una permutación, si consideramos su descompoisición en ciclos disjuntos:
            \begin{equation*}
                \sigma = \gamma_1\gamma_2\ldots\gamma_k
            \end{equation*}
            Entonces, se tiene que:
            \begin{equation*}
                \sigma^{-1} = \gamma_1^{-1}\gamma_2^{-1}\ldots\gamma_k^{-1}
            \end{equation*}
    \end{enumerate}
    \begin{proof}
        Demostramos cada uno de los resultados:
        \begin{enumerate}
            \item Supongamos que:
                \begin{gather*}
                    \gamma_1 = (x_1\ x_2\ \ldots\ x_n), \qquad \gamma_2 = (y_1\ y_2\ \ldots\ y_m) \\
                    x_1,x_2,\ldots,x_n,y_1,y_2,\ldots,y_m \in X \text{\ todos ellos distintos}
                \end{gather*}
                Tenemos entonces que:
                \begin{itemize}
                    \item Si $x\neq x_i, x\neq y_j$ para $i \in \{1,\ldots,n\}, j\in \{1,\ldots,m\}$:
                        \begin{equation*}
                            \gamma_1(\gamma_2(x)) = \gamma_1(x) = x = \gamma_2(x) = \gamma_2(\gamma_1(x))
                        \end{equation*}
                    \item Si consideramos $i \in \{1,\ldots,n-1\}$:
                        \begin{equation*}
                            \gamma_1(\gamma_2(x_i)) = \gamma_1(x_i) = x_{i+1} = \gamma_2(x_{i+1}) = \gamma_2(\gamma_1(x_i))
                        \end{equation*}
                        Donde hemos usado que $x_i \neq y_j$ para todo $j \in \{1,\ldots,m\}$.
                    \item Si consideramos ahora $j\in \{1,\ldots,m-1\}$:
                        \begin{equation*}
                            \gamma_1(\gamma_2(y_j)) = \gamma_1(y_{j+1}) = y_{j+1} = \gamma_2(y_j) = \gamma_2(\gamma_1(y_j))
                        \end{equation*}
                        Donde hemos usado que $y_j \neq x_j$ para todo $i \in \{1,\ldots,n\}$.
                    \item Faltan los casos de $x_n$ y $y_m$, que son análogos:
                        \begin{align*}
                            \gamma_1(\gamma_2(x_n)) = \gamma_1(x_n) &= x_1 = \gamma_2(x_1) = \gamma_2(\gamma_1(x_n)) \\
                            \gamma_1(\gamma_2(y_m)) = \gamma_1(y_1) &= y_1 = \gamma_2(y_m) = \gamma_2(\gamma_1(y_m))
                        \end{align*}
                \end{itemize}
                Como hemos visto que $\gamma_1(\gamma_2(x)) = \gamma_2(\gamma_1(x))$ para todo $x\in X$, concluimos que $\gamma_1\gamma_2=\gamma_2\gamma_1$.
            \item Dado $\sigma = \gamma_1\gamma_2\ldots\gamma_k$, buscamos una permutación $\tau\in S_n$ que verifique que:
                \begin{equation*}
                    \sigma\tau = \gamma_1\gamma_2\ldots\gamma_{k-1}\gamma_k\tau = 1
                \end{equation*}
                Observamos que como $\tau$ podemos tomar:
                \begin{equation*}
                    \tau = \gamma_k^{-1}\gamma_{k-1}^{-1}\ldots\gamma_2^{-1}\gamma_1^{-1}
                \end{equation*}
                sin embargo, como los $\gamma_i$ con $i \in \{1,\ldots,k\}$ eran ciclos disjuntos, por la Proposición~\ref{prop:propiedades_sn}, sabemos que los $\gamma_i^{-1}$ también seguirán siendo ciclos disjuntos y por 1 sabemos que su producto es conmutativo, por lo que podemos escribir:
                \begin{equation*}
                    \tau = \gamma_1^{-1}\gamma_2^{-1}\ldots\gamma_k^{-1}
                \end{equation*}
                Como $\sigma\tau=1$, concluimos que $\tau = \sigma^{-1}$.
        \end{enumerate}
    \end{proof}
\end{prop}

\begin{ejemplo}
    En $S_{13}$, consideramos:
    \begin{equation*}
        \sigma = \left(\begin{array}{ccccccccccccc}
                1 & 2 & 3 & 4 & 5 & 6 & 7 & 8 & 9 & 10 & 11 & 12 & 13 \\
                12 & 13 & 3 & 1 & 11 & 9 & 5 & 10 & 6 & 4 & 7 & 8 & 2
        \end{array}\right)
    \end{equation*}
    De forma por ciclos disjuntos, podemos notar:
    \begin{equation*}
        \sigma = (1\ 12\ 8\ 10\ 4)(2\ 13)(5\ 11\ 7)(6\ 9)
    \end{equation*}
    Dada una permutacion en notación de ciclos disjuntos, sabemos que para calcular la permutación inversa basta calcular la inversa de cada uno de los ciclos:
    \begin{equation*}
        \sigma^{-1} = (4\ 10\ 8\ 12\ 1)(2\ 13)(7\ 11\ 5)(6\ 9)
    \end{equation*}
\end{ejemplo}~\\

Del Teorema~\ref{teo:ciclos_disjuntos} deducimos el siguiente corolario:
\begin{coro}\label{cor:orden_permutacion}
    El orden de una permutación $\sigma\in S_n$ es el mínimo común múltiplo de las longitudes de los ciclos disjuntos en los que se descompone.
    \begin{proof}
        Supongamos que $\sigma$ se descompone de la forma:
        \begin{equation*}
            \sigma = \gamma_1\gamma_2\ldots \gamma_k
        \end{equation*}
        como $\gamma_i \gamma_j = \gamma_j \gamma_i$ para $i,j\in \{1,\ldots,k\}$, tenemos que $\forall m\in \mathbb{N}$:
        \begin{equation*}
            \sigma^m = \gamma_1^m \gamma_2^m \ldots \gamma_k^m
        \end{equation*}
        Si $m = O(\sigma)$, entonces:
        \begin{equation*}
            \sigma^m = 1 \Longleftrightarrow \gamma_i^m = 1 \stackrel{(\ast)}{\Longrightarrow} O(\gamma_i) | m \quad \forall i \in \{1,\ldots,k\}
        \end{equation*}
        Donde en $(\ast)$ hemos usado la Proposición~\ref{prop:divide_orden}. Concluimos que $m$ es el mínimo común múltiplo de los órdenes de los ciclos, que por la Proposición~\ref{prop:propiedades_sn}, coincide con el mínimo común múltiplo de las longitudes de los ciclos.
    \end{proof}
\end{coro}

\begin{ejemplo}
    Para familizarnos con la notación de permutaciones por ciclos disjuntos, vamos a enumerar todos los elementos de $S_n$ para $n=2,3,4$:
    \begin{enumerate}
        \item Para $n=2$, tenemos $X=\{1,2\}$ y por tanto:
            \begin{equation*}
                S_2 = \{id, (1\ 2)\}
            \end{equation*}
        \item Para $n=3$, tenemos $X=\{1,2,3\}$ y:
            \begin{equation*}
                S_3 = \{id, (1\ 2\ 3), (1\ 3\ 2), (1\ 2), (1\ 3), (2, 3)\}
            \end{equation*}
        \item Para $n=4$, tenemos $X=\{1,2,3,4\}$ y:
            \begin{multline*}
                S_4 = \{id, (1\ 2), (1\ 3), (1\ 4), (2\ 3), (2\ 4), (3\ 4), (1\ 2\ 3), (1\ 3\ 2), (1\ 2\ 4), \\
                    (1\ 4\ 2), (1\ 3\ 4), (1\ 4\ 3), (2\ 3\ 4), (2\ 4\ 3), (1\ 2\ 3\ 4), (1\ 3\ 2\ 4), (1\ 2\ 4\ 3), \\
                (1\ 3\ 4\ 2), (1\ 4\ 2\ 3), (1\ 4\ 3\ 2), (1\ 2)(3\ 4), (1\ 3)(2\ 4), (1\ 4)(2\ 3)\}
            \end{multline*}
    \end{enumerate}
\end{ejemplo}

% Ejercicio 33 del T1: % // TODO: Ya está hecho, puedes poner una referencia
% De euler pq todos de grado par
% De hamilton por la segunda formula
% es k44 al dibujarlo => contiene a k33
% \begin{ejemplo}
%     Se considera:
%     \begin{equation*}
%         Q_2 = \langle x,y\mid x^4=1,x^2=y^2,yx=x^{-1}y \rangle  = \{1,x,x^2,x^3,y,xy,x^2y,x^3y\}
%     \end{equation*}
%     Consideramos $G$ un grafo cuyos vértices son los elementos de $Q_2$ y los lados son:
%     \begin{itemize}
%         \item Si $a$ es un vértice, tiene una arista al vértice $ax$.
%         \item Si $a$ es un vértice, tiene una arista al vértice $ay$.
%     \end{itemize}
% \end{ejemplo}

\begin{definicion}[Elementos conjugados]
    Sea $G$ un grupo y $a,c\in G$, decimos que son conjugados si $\exists b \in G$ de forma que $a = bcb^{-1}$.
\end{definicion}

\begin{prop}\label{prop:conjugar_ciclos}
    Si $\gamma\in S_n$ es un ciclo de longitud $m$, también lo será cualquier conjugado suyo. Es decir, si $\tau\in S_n$ y $\gamma$ es un ciclo, entonces $\tau\gamma\tau^{-1}$ es un ciclo de longitud $m$.
    \begin{proof}
        Si $\gamma=(x_1\ x_2\ \ldots\ x_m)$, sea $\tau \in S_n$, entonces veamos que:
        \begin{equation*}
            \alpha = \tau \gamma\tau^{-1} = (\tau(x_1)\ \tau(x_2)\ \ldots\ \tau(x_m))
        \end{equation*}
        Luego $\alpha$ será un ciclo de longitud $m$. Para ello, sea $y\in \{1,\ldots,n\}$:
        \begin{itemize}
            \item Si $\tau^{-1}(y) = x_i\Longrightarrow y=\tau(x_i)$ con $i \in \{1,\ldots,m-1\}$:
                \begin{equation*}
                    y \stackrel{\tau^{-1}}{\longmapsto} x_i \stackrel{\gamma}{\longmapsto} x_{i+1} \stackrel{\tau}{\longmapsto} \tau(x_{i+1}) = \alpha(\tau(x_i))
                \end{equation*}
            \item Si $\tau^{-1}(y) = x_{m}\Longrightarrow y=\tau(x_m)$:
                \begin{equation*}
                    y \stackrel{\tau^{-1}}{\longmapsto} x_m \stackrel{\gamma}{\longmapsto} x_1 \stackrel{\tau}{\longmapsto} \tau(x_1) = \alpha(\tau(x_m))
                \end{equation*}
            \item Si $\tau^{-1}(y) = x\Longrightarrow y=\tau(x)$ con $x\neq x_i$ para todo $i \in \{1,\ldots,m\}$:
                \begin{equation*}
                    y \stackrel{\tau^{-1}}{\longmapsto} x \stackrel{\gamma}{\longmapsto} x \stackrel{\tau}{\longmapsto} \tau(x) = \alpha(\tau(x))
                \end{equation*}
        \end{itemize}
        Concluimos que $\alpha = (\tau(x_1)\ \tau(x_2)\ \ldots\ \tau(x_m))$.
    \end{proof}
\end{prop}

\begin{ejemplo}
    Veamos la última Proposición en un caso práctico. Si consideramos:
    \begin{equation*}
        \tau = (1\ 3\ 4), \quad \gamma = (2\ 4\ 5\ 3), \quad \tau^{-1} = (4\ 3\ 1)
    \end{equation*}
    Y tratamos de estudiar la imagen de $X=\{1,2,3,4,5\}$ bajo $\alpha=\tau\gamma\tau^{-1}$:
    \begin{align*}
        &1 \stackrel{\tau^{-1}}{\longmapsto} 4 \stackrel{\sigma}{\longmapsto} 5 \stackrel{\tau}{\longmapsto} 5 \\
        &2 \longmapsto2 \longmapsto 4 \longmapsto 1 \\
        &3 \longmapsto1 \longmapsto1 \longmapsto 3 \\
        &4 \longmapsto 3 \longmapsto 2 \longmapsto 2 \\
        &5\longmapsto 5 \longmapsto 3 \longmapsto 4
    \end{align*} 
    Tenemos entonces que $\alpha$ es también un ciclo de longitud 4:
    \begin{equation*}
        \alpha=\tau\gamma\tau^{-1} = \left(\begin{array}{ccccc}
            1 & 2 & 3 & 4 & 5 \\
            5 & 1 & 3 & 2 & 4 
        \end{array}\right) = (1\ 5\ 4\ 2)
    \end{equation*}
\end{ejemplo}

\begin{prop}\label{prop:conjugar_permutaciones}
    Sea $\sigma\in S_n$ una permutación de forma que se descompone en ciclos disjuntos de la forma:
    \begin{equation*}
        \sigma = \gamma_1\ldots\gamma_k
    \end{equation*}
    Entonces, podemos calcular su conjugado mediante $\tau\in S_n$ componiendo el conjugado de cada uno de los ciclos disjuntos en los que se descompone:
    \begin{equation*}
        \tau\sigma\tau^{-1} = \tau\gamma_1\tau^{-1}\ldots \tau\gamma_k\tau^{-1}
    \end{equation*}
    \begin{proof}
        \begin{equation*}
            \tau\sigma\tau^{-1} = \tau\gamma_1\ldots\gamma_k\tau^{-1} = \tau \gamma_1 id \gamma_2 id \ldots id \gamma_k \tau^{-1} = \tau\gamma_1\tau^{-1}\tau \gamma_2 \tau^{-1} \ldots \tau\gamma_k\tau^{-1}
        \end{equation*}
    \end{proof}
\end{prop}

\begin{ejemplo}
    Para practicar la conjugación de ciclos aplicando las Proposiciones~\ref{prop:conjugar_ciclos} y~\ref{prop:conjugar_permutaciones}, se plantea dados:
    \begin{equation*}
        \sigma = (1\ 12\ 8\ 10\ 4)(2\ 13)(5\ 11\ 7)(6\ 9),\qquad \tau = (4\ 8\ 12\ 7\ 5\ 9)
    \end{equation*}
    calcular $\tau \sigma \tau^{-1}$. Para ello, sabemos por la Proposición~\ref{prop:conjugar_permutaciones} que si\footnote{Observar la descomposición hecha ya de $\sigma$.} $\sigma=\gamma_1\gamma_2\gamma_3\gamma_4$, entonces basta calcular:
    \begin{equation*}
        \tau\gamma_1\tau^{-1},\quad\tau\gamma_2\tau^{-1},\quad\tau\gamma_3\tau^{-1},\quad\tau\gamma_4\tau^{-1}
    \end{equation*}
    Por la Proposición~\ref{prop:conjugar_ciclos}, sabemos que:
    \begin{align*}
        \tau\gamma_1\tau^{-1} &= (\tau(1)\ \tau(12)\ \tau(8)\ \tau(10)\ \tau(4)) = (1\ 7\ 12\ 10\ 8) \\
        \tau\gamma_2\tau^{-1} &= (\tau(2)\ \tau(13)) = (2\ 13) \\
        \tau\gamma_3\tau^{-1} &= (\tau(5)\ \tau(11)\ \tau(7)) = (9\ 11\ 5) \\
        \tau\gamma_4\tau^{-1} &= (\tau(6)\ \tau(9)) = (6\ 4)
    \end{align*}
    Si lo escribimos todo junto:
    \begin{equation*}
        \tau\sigma\tau^{-1} = (1\ 7\ 12\ 10\ 8)(2\ 13)(9\ 11\ 5)(6\ 4)
    \end{equation*}
\end{ejemplo}

\begin{prop}\label{prop:perm_prod_transp}
    Toda permutación es un producto de transposiciones.
    \begin{proof}
        Dada $\sigma\in S_n$, esta tiene su descomposición en ciclos disjuntos:
        \begin{equation*}
            \sigma = \gamma_1\ldots\gamma_k
        \end{equation*}
        Basta demostrar que todo ciclo es producto de transposiciones.\newline En efecto, sea $\gamma = (x_1\ x_2\ \ldots\ x_m)$, podemos escribir:
        \begin{equation*}
            (x_1\ x_2\ \ldots\ x_m) = (x_1\ x_m)(x_1\ x_{m-1}) \ldots (x_1 x_3)(x_1 x_2)
        \end{equation*}
        Para verlo, observemos qué hace la aplicación de la derecha con cada elemento (léase la descomposición de derecha a izquierda):
        \begin{align*}
            &x_1 \longmapsto x_2 \\
            &x_2 \longmapsto x_1 \longmapsto x_3 \\
            &x_3 \longmapsto x_1 \longmapsto x_4 \\
            &\vdots \\
            &x_i \longmapsto x_1 \longmapsto x_{i+1} \\
            &\vdots \\
            &x_m \longmapsto x_1
        \end{align*}
        O también podemos escribir:
        \begin{equation*}
            (x_1\ x_2\ \ldots\ x_m) = (x_1\ x_2)(x_2\ x_3) \ldots (x_{m-1}\ x_m)
        \end{equation*}
        Para verlo:
        \begin{align*}
            &x_1 \longmapsto x_2 \\
            &x_2 \longmapsto x_3 \\
            &x_3 \longmapsto x_4 \\
            &\vdots \\
            &x_i \longmapsto x_{i+1} \\
            &\vdots \\
            &x_m \longmapsto x_{m-1} \longmapsto x_{m-2} \longmapsto \ldots \longmapsto x_3 \longmapsto x_2 \longmapsto x_1
        \end{align*}
    \end{proof}
\end{prop}

\begin{ejemplo}
    Sea $\sigma = (1\ 2\ 3\ 4\ 5)$, veamos que $\sigma$ se puede descomponer en transposiciones de la forma:
    \begin{equation*}
        \sigma= t_1t_2t_3t_4
    \end{equation*}
    Con $t_1 = (1\ 5),\  t_2 = (1\ 4),\  t_3 = (1\ 3),\  t_4 = (1\ 2)$.\newline
    Para ello, escribamos la imagen de $X=\{1,2,3,4,5\}$ mediante la permutación resultante de componer las 4 transposiciones $\gamma=t_1t_2t_3t_4$ y veamos que coincide con la de $\sigma$:
    \begin{equation*}
        \left.\begin{array}{cc}
            &1 \stackrel{t_4}{\longmapsto} 2 \stackrel{t_3}{\longmapsto} 2 \stackrel{t_2}{\longmapsto} 2 \stackrel{t_1}{\longmapsto} 2 \\
            &2 \longmapsto 1 \longmapsto 3 \longmapsto 3 \longmapsto 3 \\
            &3 \longmapsto 3 \longmapsto 1 \longmapsto 4 \longmapsto 4 \\
            &4 \longmapsto 4 \longmapsto 4 \longmapsto 1 \longmapsto 5 \\
            &5 \longmapsto 5 \longmapsto 5 \longmapsto 5 \longmapsto 1 
        \end{array}\right\} \Longrightarrow 
        \left\{\begin{array}{cc}
            &1 \stackrel{\gamma}{\longmapsto} 2 \\
            &2 \longmapsto 3 \\
            &3 \longmapsto 4 \\
            &4 \longmapsto 5 \\
            &5 \longmapsto 1 
        \end{array}\right.
    \end{equation*}
    De esta forma:
    \begin{equation*}
        \gamma = t_1t_2t_3t_4 = \left(\begin{array}{ccccc}
            1 & 2 & 3 & 4 & 5 \\
            2 & 3 & 4 & 5 & 1 
        \end{array}\right) = (1\ 2\ 3\ 4\ 5) = \sigma
    \end{equation*}
    Si ahora consideramos la descomposición de la forma:
    \begin{equation*}
        \sigma = r_1r_2r_3r_4
    \end{equation*}
    Con $r_1 = (1\ 2)$, $r_2 = (2\ 3)$, $r_3 = (3\ 4)$, $r_4 = (4\ 5)$, escribimos ahora la imagen de $X$ mediante la permutación $\tau=r_1r_2r_3r_4$:
    \begin{equation*}
        \left.\begin{array}{cc}
            &1 \stackrel{r_4}{\longmapsto} 1 \stackrel{r_3}{\longmapsto} 1 \stackrel{r_2}{\longmapsto} 1 \stackrel{r_1}{\longmapsto} 2 \\
            &2 \longmapsto 2 \longmapsto 2 \longmapsto 3 \longmapsto 3 \\
            &3 \longmapsto 3 \longmapsto 4 \longmapsto 4 \longmapsto 4 \\
            &4 \longmapsto 5 \longmapsto 5 \longmapsto 5 \longmapsto 5 \\
            &5 \longmapsto 4 \longmapsto 3 \longmapsto 2 \longmapsto 1 
        \end{array}\right\} \Longrightarrow \left\{\begin{array}{cc}
            &1 \stackrel{\gamma}{\longmapsto} 2 \\
            &2 \longmapsto 3 \\
            &3 \longmapsto 4 \\
            &4 \longmapsto 5 \\
            &5 \longmapsto 1 
        \end{array}\right.
    \end{equation*}
    Vemos igual que antes que $\tau = \sigma$.
\end{ejemplo}

\begin{prop}\label{prop:paridad_transposiciones}
    Una permutación admite varias descomposiciones en productos de transposiciones, pero todas ellas coinciden en la paridad del número de transposiciones.
\end{prop}

\subsection{Signatura}
\begin{definicion}[Signatura]
    Consideraremos el siguiente polinomio de $n$ variables:
    \begin{equation*}
        \Delta = \prod_{1\leq i<j\leq n} (x_i-x_j) \in \mathbb{Z}[x_1,\ldots,x_n]
    \end{equation*}
    Y definimos para cada $\sigma\in S_n$:
    \begin{equation*}
        \sigma(\Delta) = \prod_{1\leq i<j\leq n}(x_{\sigma(i)}-x_{\sigma(j)})
    \end{equation*}
    Podemos ahora definir la aplicación signatura $\veps:S_n\longrightarrow \{-1,1\}$ dada por:
    \begin{equation*}
        \veps(\sigma) = \left\{\begin{array}{cl}
                1 & \text{si\ } \sigma(\Delta) = \Delta \\
                -1 & \text{si\ } \sigma(\Delta) = -\Delta 
        \end{array}\right.
    \end{equation*}
    \begin{itemize}
        \item Si $\veps(\sigma)=1$, diremos que $\sigma$ es una permutación par.
        \item Si $\veps(\sigma)=-1$, diremos que $\sigma$ es una permutación impar.
    \end{itemize}
\end{definicion}

\begin{observacion}
    A partir de la definición anterior, tenemos que $\sigma(\Delta)=\veps(\sigma)\Delta$.
\end{observacion}

\begin{ejemplo}
    Sea $n=4$, estaremos interesados en el polinomio:
    \begin{equation*}
        \Delta = \prod_{1\leq i<j\leq n} (x_i-x_j) = (x_1-x_2)(x_1-x_3)(x_1-x_4)(x_2-x_3)(x_2-x_4)(x_3-x_4)
    \end{equation*}
    Si consideramos $\sigma=(1\ 2\ 3\ 4)$, queremos comprobar cual es la signatura de $\sigma$. Como\footnote{Marcamos en rojo los factores que se invierten.}:
    \begin{equation*}
        \sigma(\Delta) = (x_2 - x_3)(x_2-x_4)\red{(x_2-x_1)}(x_3-x_4)\red{(x_3-x_1)}\red{(x_4-x_1)} = -\Delta
    \end{equation*}
    Deducimos que $\veps(\sigma)=-1$, es decir, $\sigma$ es una permutación impar.
\end{ejemplo}

\begin{prop}\label{prop:producto_signaturas}
    La aplicación signatura verifica que:
    \begin{equation*}
        \veps\left(\prod_{i=1}^{m}\sigma_i\right) = \prod_{i=1}^{m} \veps(\sigma_i)
    \end{equation*}
    Con $\sigma_1,\sigma_2,\ldots,\sigma_m\in S_n$.
    \begin{proof}
        Por inducción sobre $m$:
        \begin{itemize}
            \item \underline{Para $m=2$}: Queremos ver que dadas $\sigma,\tau\in S_n$, entonces:
                \begin{equation*}
                    \veps(\sigma\tau) = \veps(\sigma)\veps(\tau)
                \end{equation*}
                Para ello, si vemos que $(\sigma\tau)(\Delta) = \sigma(\tau(\Delta))$ y que $\sigma(-\Delta) = -\sigma(\Delta)$,
            basta distinguir casos:
            \begin{itemize}
                \item Si $\sigma$ es par:
                    \begin{itemize}
                        \item Si $\tau$ es par, se tendrá $\sigma(\tau(\Delta)) = \sigma(\Delta)=\Delta$, con lo que $\sigma\tau$ es par.
                        \item Si $\tau$ es impar, se tendrá $\sigma(\tau(\Delta)) = \sigma(-\Delta)=-\sigma(\Delta)=-\Delta$, con lo que $\sigma\tau$ es impar.
                    \end{itemize}
                \item Si $\sigma$ es impar:
                    \begin{itemize}
                        \item Si $\tau$ es par, se tendrá $\sigma(\tau(\Delta)) = \sigma(\Delta) = -\Delta$, con lo que $\sigma\tau$ es impar.
                        \item Si $\tau$ es impar, se tendrá $\sigma(\tau(\Delta)) = \sigma(-\Delta)=-\sigma(\Delta)=\Delta$, con lo que $\sigma\tau$ es par.
                    \end{itemize}
            \end{itemize}
            \item \underline{Supuesto para $m-1$}:
                \begin{equation*}
                    \veps\left(\prod_{i=1}^{m}\sigma_1\right) = \veps\left(\left(\prod_{i=1}^{m-1}\sigma_i\right)\sigma_m\right) = \veps\left(\prod_{i=1}^{m-1}\sigma_i\right)\veps(\sigma_m) \AstIg \prod_{i=1}^{m-1}\left(\veps(\sigma_i)\right)\veps(\sigma_m) = \prod_{i=1}^{m}\veps(\sigma_i)
                \end{equation*}
                Donde en $(\ast)$ hemos usado la hipótesis de inducción.
        \end{itemize}
    \end{proof}
\end{prop}

\begin{prop}
    Se verifican los siguientes resultados:
    \begin{enumerate}
        \item Las transposiciones son permutaciones impares.
        \item Una permutación es par si y solo si se descompone en el producto de un número par de transposiciones.
        \item Un ciclo de longitud $m\geq 2$ es par si y solo si $m$ es impar.
        \item Una permutación es par si y solo si el número de ciclos de longitud par en su descomposición en ciclos disjuntos es par.
    \end{enumerate}
    \begin{proof}
        Demostramos cada uno de los resultados:
        \begin{enumerate}
            \item Sea $\sigma=(i\ j)$ una transposición (con $1\leq i <j\leq n$), estudiemos qué sucede con $\sigma(\Delta)$:
                \begin{itemize}
                    \item Por una parte, está claro que hay un cambio de signo tras aplicar $\sigma$ al factor $(x_i - x_j)$, ya que este pasa a ser $(x_j - x_i)$.
                    \item Está claro que los factores de la forma $(x_a - x_b)$ con $a,b \notin \{i,j\}$ se mantienen invariantes ante $\sigma$, por lo que no hay cambio de signo en estos.
                    \item Además, los factores de la forma $(x_a - x_y)$ con $y \in \{i,j\}$ y $a < i$ tampoco alteran el signo de $\Delta$, ya que al aplicar $\sigma$:
                        \begin{align*}
                            (x_a - x_i) \stackrel{\sigma}{\longmapsto} (x_a - x_j) \\
                            (x_a - x_j) \longmapsto (x_a - x_i) 
                        \end{align*}
                        Tenemos que un factor va al otro, por lo que no alteran el signo.
                    \item De forma análoga, los factores de la forma $(x_y - x_b)$ con $y \in \{i,j\}$ y $b > j$ tampoco alteran el signo de $\Delta$:
                        \begin{align*}
                            (x_i - x_b) \stackrel{\sigma}{\longmapsto} (x_j - x_b) \\
                            (x_j - x_b) \longmapsto (x_i - x_b) 
                        \end{align*}
                    \item Finalmente, los únicos factores que nos quedan por considerar son los de la forma $(x_i - x_a)$ y $(x_a - x_j)$, con $i < a < j$. En este caso:
                        \begin{align*}
                            (x_i - x_a) \stackrel{\sigma}{\longmapsto} (x_j - x_a) = -(x_a - x_j) \\
                            (x_a - x_j) \longmapsto (x_a - x_i) = -(x_i - x_a) 
                        \end{align*}
                        Fijado $a$ con $i<a<j$, tanto el factor $(x_i - x_a)$ como el $(x_a - x_j)$ cambian de signo, por lo que el doble cambio de signo se compensa, luego estos factores no alteran el signo de $\Delta$ al aplicar $\sigma$.
                \end{itemize}
                Concluimos que al aplicar $\sigma = (i\ j)$ sobre $\Delta$, el signo obtenido es el mismo salvo por el factor $(x_i - x_j)$, que cambia de signo, por lo que:
                \begin{equation*}
                    \sigma(\Delta) = -\Delta
                \end{equation*}
                y llegamos a que $\sigma$ es impar.
            \item[2.] Sea $\sigma\in S_n$ una permutación, sabemos que puede descomponerse en $k$ transposiciones:
                \begin{equation*}
                    \sigma = \gamma_1\gamma_2\ldots\gamma_k
                \end{equation*}
                Usando la Proposición~\ref{prop:producto_signaturas} y 1, tenemos que:
                \begin{equation*}
                    \veps(\sigma) = \prod_{i=1}^{k}\veps(\gamma_i) = \prod_{i=1}^{k}(-1) = {(-1)}^{k}
                \end{equation*}
                Por lo que:
                \begin{itemize}
                    \item Si $k$ es par, entonces $\veps(\sigma)=1$.
                    \item Si $k$ es impar, entonces $\veps(\sigma)=-1$.
                \end{itemize}
            \item[3.] Para $m=2$, un ciclo de longitud $m$ es una transposición, que ya sabemos que es impar. Sea $\tau$ un ciclo de longitud $m\geq 3$, en la Proposición~\ref{prop:perm_prod_transp} vimos que $\tau$ se podía descomponer como producto de $m-1$ transposiciones:
                \begin{equation*}
                    \tau = \gamma_1\gamma_2\ldots \gamma_{m-1}
                \end{equation*}
                Por tanto, y aplicando 2, tenemos que:
                \begin{itemize}
                    \item Si $m$ es par, entonces $m-1$ es impar, con lo que $\tau$ es impar.
                    \item Si $m$ es impar, entonces $m-1$ es par, con lo que $\tau$ es par.
                \end{itemize}
            \item[4.] Sea $\sigma\in S_n$, esta se puede descomponer como producto de $k$ ciclos disjuntos de longitud mayor o igual que 2:
                \begin{equation*}
                    \sigma = \gamma_1\gamma_2 \ldots \gamma_k
                \end{equation*}
                Usando la Proposición~\ref{prop:producto_signaturas}, tenemos que:
                \begin{equation*}
                    \veps(\sigma) = \prod_{i=1}^{k}\veps(\gamma_i)
                \end{equation*}
                Si consideramos la siguiente partición de $\{1,\ldots,k\}$:
                \begin{gather*}
                    A = \{i \in \{1,\ldots,k\} \mid \gamma_i \text{\ tiene longitud impar}\} \\
                    B = \{i \in \{1,\ldots,k\} \mid \gamma_i \text{\ tiene longitud par}\} 
                \end{gather*}
                Por 3 tenemos que $\veps(\gamma_i) = 1$ para todo $i \in A$ y que $\veps(\gamma_j) = -1$ para todo $j\in B$. De esta forma:
                \begin{equation*}
                    \veps(\sigma) = \left(\prod_{i \in A}\veps(\gamma_i)\right) \left(\prod_{i \in B}\veps(\gamma_i)\right) = \left(\prod_{i \in A}1\right) \left(\prod_{i \in B}-1\right) = \prod_{i \in B}-1 = {(-1)}^{|B|}
                \end{equation*}
                Por tanto:
                \begin{itemize}
                    \item Si $|B|$ es par, tenemos que $\sigma$ es par.
                    \item Si $|B|$ es impar, tenemos que $\sigma$ es impar.
                \end{itemize}
        \end{enumerate}
    \end{proof}
\end{prop}

\noindent
Con esta Proposición, la demostración de la Proposición~\ref{prop:paridad_transposiciones} se hace ya evidente.

\begin{ejemplo}
    Ahora, es fácil determinar la signatura de cualquier permutación. Por ejemplo, si consideramos:
    \begin{equation*}
        \sigma = (1\ 12\ 8\ 10\ 4)(2\ 13)(5\ 11\ 7)(6\ 9)
    \end{equation*}
    Como tiene 2 ciclos de longitud par (un número par), $\sigma$ es una permutación par.
\end{ejemplo}

\subsection{Grupos Alternados $A_n$}
\begin{definicion}[Grupos Alternados $A_n$]
    En $S_n$ consideramos el conjunto:
    \begin{equation*}
        A_n = \{\sigma\in S_n \mid \sigma \text{\ es par}\}
    \end{equation*}
    Se verifica que $(A_n,\circ,1)$ es un grupo:
    \begin{itemize}
        \item La asociatividad de $\circ$ es heredada de la de $\circ$ en $S_n$.
        \item El producto de dos permutaciones pares es par, luego está bien definido el grupo.
        \item La identidad es una permutación par, que es el neutro de la operación binaria.
        \item Dado $\sigma\in A_n$, escribimos su descomposición en ciclos disjuntos e invertimos cada ciclo. La longitud de los ciclos no cambia, luego la paridad del ciclo inverso tampoco, por lo que $\sigma^{-1}$ sigue siendo una permutación par.
    \end{itemize}
    Al grupo $A_n$ lo llamamos el grupo alternado de grado $n$, que verifica:
    \begin{equation*}
        |A_n| = \dfrac{n!}{2}
    \end{equation*}
\end{definicion}

% Esto último se debe a que si consideramos todas las permutaciones pares, las podemos componer con $(1\ 2)$ y obtendremos todas las impares.

\begin{observacion}
    Notemos que si definimos $B_n = \{\sigma\in S_n \mid \sigma \text{\ es impar}\}$, entonces sobre $B_n$ no podemos tener una estructura de grupo con la operación $\circ$, ya que el neutro para $\circ$ de $S_n$ no está en $B_n$, sino en $A_n$.
\end{observacion}

\begin{ejemplo}
    Listar todos los elementos de los grupos alternados es fácil si previamente listamos todos los elementos de su grupo simétrico correspondiente:
    \begin{enumerate}
        \item Para $n=3$:
        \begin{gather*}
            S_3 = \{1, (1\ 2), (1\ 3), (2\ 3), (1\ 2\ 3), (1\ 3\ 2)\} \\
            A_3 = \{1, (1\ 2\ 3), (1\ 3\ 2)\}
        \end{gather*}
        \item Para $n=4$:
            \begin{multline*}
                S_4 = \{1, (1\ 2), (1\ 3), (1\ 4), (2\ 3), (2\ 4), (3\ 4), (1\ 2\ 3), (1\ 3\ 2), (1\ 2\ 4), \\
                    (1\ 4\ 2), (1\ 3\ 4), (1\ 4\ 3), (2\ 3\ 4), (2\ 4\ 3), (1\ 2\ 3\ 4), (1\ 2\ 4\ 3), (1\ 3\ 2\ 4),\\
                     (1\ 3\ 4\ 2), (1\ 4\ 2\ 3), (1\ 4\ 3\ 2), (1\ 2)(3\ 4), (1\ 3)(2\ 4), (1\ 4)(2\ 3)\}
            \end{multline*}
            \begin{multline*}
                A_4 = \{1, (1\ 2\ 3), (1\ 2\ 4), (1\ 3\ 2), (1\ 4\ 2), (1\ 3\ 4), (1\ 4\ 3), (2\ 3\ 4), (2\ 4\ 3), \\
                (1\ 2)(3\ 4), (1\ 3)(2\ 4), (1\ 4)(2\ 3)\}
            \end{multline*}
    \end{enumerate}
\end{ejemplo}

\begin{prop}\label{prop:generadores_grupos}
    Se tiene que:
    \begin{enumerate}[label=(\alph*)]
        \item $S_n = \langle (1\ 2),(2\ 3),\ldots,(n-1, n) \rangle $
        \item $S_n = \langle (1\ 2), (1\ 2\ \ldots\ n) \rangle $
        \item $S_n = \langle (1\ 2),(1\ 3), \ldots, (1\ n)\rangle $
        \item $A_n = \langle (x_1\ x_2\ x_3) \rangle $ con $n\geq 3$
        \item $A_n = \langle (1\ x\ y) \rangle $ con $n\geq 3$
    \end{enumerate}
    % // TODO: Ver ejercicios 21 y 22 de la relación 2
    \begin{proof}
        Veamos cada uno de los enunciados:
        \begin{enumerate}[label=(\alph*)]
            \item Sabemos que (por la Proposición~\ref{prop:perm_prod_transp}):
                \begin{equation*}
                    S_n = \langle (i\ j) \mid i,j \in \{1,\ldots,n\},\ i<j \rangle 
                \end{equation*}
                Supuesto que $i<j$, vemos que:
                \begin{equation*}
                    (i\ j) = (i\ i+1)(i+1\ i+2) \ldots (j-2\ j-1)(j-1\ j)(j-1\ j-2) \ldots (i+2\ i+1)(i+1\ i)
                \end{equation*}
            \item Por el apartado anterior, basta obtener cualquier transposición de la forma $(i\ i+1)$ con $i \in \{1,\ldots,n-1\}$ a partir de $\sigma=(1\ 2\ \ldots\ n)$ y $(1\ 2)$. Para ello, como se tiene que:
                \begin{equation*}
                    \sigma^{i-1}(1) = i \qquad \sigma^{i-1}(2) = i+1 
                \end{equation*}
                Podemos considerar el conjugado de $(1\ 2)$ mediante $\sigma^{i-1}$:
                \begin{equation*}
                    \sigma^{i-1}(1\ 2){(\sigma^{i-1})}^{-1} = \sigma^{i-1}(1\ 2)\sigma^{1-i} = (\sigma^{i-1}(1)\ \sigma^{i-1}(2)) = (i\ i+1)
                \end{equation*}
            \item Basta ver que $(1\ 2\ \ldots\ n)$ se puede obtener por composición de transposiciones de la forma $(1\ j)$ con $j \in \{2,\ldots,n\}$, lo que ya se hizo en la Proposición~\ref{prop:perm_prod_transp}:
                \begin{equation*}
                    (1\ 2\ \ldots\ n) = (1\ n)(1\ n-1)\ldots (1\ 3)(1\ 2)
                \end{equation*}
            \item Podemos suponer que $x_1<x_2<x_3$, ya que:
                \begin{equation*}
                    (x_1\ x_3\ x_2) = {(x_1\ x_2\ x_3)}^{2}
                \end{equation*}
                Sabemos que si $\sigma\in A_n$, entonces será producto de un número par de transposiciones, por lo que basta expresar estos productos en función de ciclos de la forma $(x_1\ x_2\ x_3)$.
                \begin{itemize}
                    \item Si hay elementos comunes, escribiremos:
                        \begin{equation*}
                            (x_1\ x_2)(x_2\ x_3) = (x_1\ x_2\ x_3)
                        \end{equation*}
                    \item Si no hay elementos comunes (tenemos dos transposiciones disjuntas), entonces:
                        \begin{equation*}
                            (x_1\ x_2)(x_3\ x_4) = (x_1\ x_2\ x_3)(x_2\ x_3\ x_4)
                        \end{equation*}
                \end{itemize}
            \item Usando el apartado anterior, tenemos que cualquier terna ordenada $(x_1\ x_2\ x_3)$ podemos escribirla de la forma:
                \begin{equation*}
                    (x_1\ x_2\ x_3) = (1\ x_3\ x_2)(1\ x_1\ x_2)(1\ x_1\ x_3)
                \end{equation*}
        \end{enumerate}
    \end{proof}
\end{prop}

\begin{ejemplo}
    Usando la Proposición~\ref{prop:generadores_grupos}, veamos distintos conjuntos generadores para varios grupos:
    \begin{enumerate}[label=(\alph*)]
        \item Destacamos:
        \begin{itemize}
            \item $S_3 = \langle (1\ 2), (2\ 3) \rangle $ y buscamos expresar la última transposición como producto de estas:
                \begin{equation*}
                    (1\ 3) = (1\ 2)(2\ 3)(2\ 1)
                \end{equation*}
            \item En $S_4 = \langle (1\ 2),(2\ 3),(3\ 4) \rangle $ mostramos por ejemplo que:
                \begin{equation*}
                    (1\ 4) = (1\ 2)(2\ 3)(3\ 4)(3\ 2)(2\ 1)
                \end{equation*}
        \end{itemize}
        \item Ahora:
            \begin{itemize}
                \item En $S_3 = \langle (1\ 2), (1\ 2\ 3) \rangle $:
                    \begin{equation*}
                        (2\ 3) = (1\ 2\ 3)(1\ 2){(1\ 2\ 3)}^{-1}
                    \end{equation*}
                \item En $S_4 = \langle (1\ 2), (1\ 2\ 3\ 4) \rangle $:
                    \begin{align*}
                        (2\ 3) &= (1\ 2\ 3\ 4)(1\ 2){(1\ 2\ 3\ 4)}^{-1} \\
                        (3\ 4) &= {(1\ 2\ 3\ 4)}^{2}(1\ 2){(1\ 2\ 3\ 4)}^{-2}
                    \end{align*}
            \end{itemize}
        \item[(d)] Recordamos los elementos de $A_4$:
                \begin{multline*}
                     A_4 = \{ 1, (1\ 2\ 3), (1\ 3\ 2), (1\ 2\ 4), (1\ 4\ 2), (1\ 3\ 4), (1\ 4\ 3), (2\ 3\ 4), (2\ 4\ 3),\\ (1\ 2)(3\ 4), (1\ 3)(2\ 4), (1\ 4)(2\ 3) \} 
                \end{multline*}
                Tenemos que:
                \begin{equation*}
                    A_4 = \langle (1\ 2\ 3), (1\ 2\ 4), (1\ 3\ 4), (2\ 3\ 4) \rangle 
                \end{equation*}
                Por ejemplo, podemos escribir:
                \begin{equation*}
                    (1\ 2)(3\ 4) = (1\ 2\ 3)(2\ 3\ 4)
                \end{equation*}
            \item[(e)] Tenemos:
                \begin{equation*}
                    A_4 = \langle (1\ 2\ 3), (1\ 2\ 4), (1\ 3\ 4) \rangle 
                \end{equation*}
    \end{enumerate}
\end{ejemplo}

\section{Grupos de matrices}
Sea $\mathbb{F}$ un cuerpo, las matrices cuadradas de orden $n$ sobre $\mathbb{F}$ las denotaremos por:
\begin{equation*}
    \mathcal{M}_n(\mathbb{F})
\end{equation*}
Sabemos que $(\mathcal{M}_n(\mathbb{F}), +, \cdot)$ es un anillo, aunque estaremos interesados en ver el conjunto $\mathcal{M}_n(\mathbb{F})$ como un grupo en su forma más interesante, es decir, como grupo con notación multiplicativa.

\subsection{Grupo lineal $\GL_n(\mathbb{F})$}
\begin{definicion}[Grupo lineal $\GL_n(\mathbb{F})$] Sea $\mathbb{F}$ un cuerpo finito, en $\mathcal{M}_n(\mathbb{F})$ consideramos el conjunto:
    \begin{equation*}
        \GL_n(\mathbb{F}) = \{A\in \mathcal{M}_n(\mathbb{F}) \mid \det(A) \neq 0\}
    \end{equation*}
    Se verifica que $(\GL_n(\mathbb{F}),\cdot,I)$ es un grupo:
    \begin{itemize}
        \item La asociatividad de $\cdot $ viene heredada de la de $\cdot $ en $\mathcal{M}_n(\mathbb{F})$.
        \item Como el determinante del producto es el producto de los determinantes, $\cdot $ es una operación cerrada para $\GL_n(\bb{F})$.
        \item $\det(I)=1\neq 0$ y se tiene que $I$ es el elemento neutro para $\cdot $.
        \item Como consideramos las matrices con determinante no nulo, sabemos que todas estas tienen inversa.
    \end{itemize}
    A $\GL_n(\mathbb{F})$ lo llamamos el grupo lineal de orden $n$.
\end{definicion}

\begin{prop}
    Sea $n\in \mathbb{N}$, si $|\bb{F}| = q$, entonces se verifica que:
    \begin{equation*}
        |\GL_n(\bb{F})| = (q^n-1)(q^n -q)\ldots (q^n - q^{n-1}) = \prod_{k=1}^{n}\left(q^n - q^{k-1}\right)
    \end{equation*}
    \begin{proof}
        Como $A\in \GL_n(\bb{F})\Longleftrightarrow A$ es regular $\Longleftrightarrow $ sus filas son vectores linealmente independientes, basta contar de cuántas formas podemos elegir $n$ vectores linealmente independientes con $n$ entradas en $\bb{F}$ (que recordamos tenía $q$ elementos). Para ello:
        \begin{itemize}
            \item Para elegir el primer vector $v_1\in \bb{F}^n$, podemos elegir cualquiera, luego el problema es elegir $n$ números de entre $q$ posibilidades, $q^n$ posibles elecciones.
                
                Sin embargo, como queremos que $v_1$ sea linealmente independiente con el resto de vectores que forman las filas de una matriz, hemos de exigir $v_1\neq 0$, con lo que tenemos $q^n-1$ posibilidades para $v_1$.
            \item Una vez elegido $v_1$, para elegir $v_2$ no podemos elegir un vector de $\cc{L}(v_1)\cong \bb{F}$, por lo que tenemos $q$ vectores que no podemos elegir; elegimos un vector de los $q^n-q$ restantes.
            \item Repitiendo el proceso, una vez elegido $v_{k-1}$, para elegir $v_k$ (con $k \in \{2, \ldots, n\}$), no podemos elegir ningún vector de $\cc{L}(v_1,\ldots,v_{k-1})\cong\bb{F}^{k-1}$, por lo que tenemos $q^{k-1}$ vectores que no podemos elegir y elegimos entre los $q^n-q^{k-1}$ restantes.
        \end{itemize}
        Este proceso ilusta que las posibles elecciones totales de vectores para las filas de una matriz de $\GL_n(\bb{F})$ son:
        \begin{equation*}
            \prod_{k=1}^{n}\left(q^n-q^{k-1}\right)
        \end{equation*}
        Por lo que este debe ser el cardinal de $|\GL_n(\bb{F})|$.
    \end{proof}
\end{prop}

\begin{ejemplo}
    Veamos:
    \begin{itemize}
        \item En $|\GL_2(\mathbb{Z}_2)| = (2^2 - 1)(2^2-2) = 6$:
            \begin{equation*}
                \GL_2(\mathbb{Z}) = \left\{\left(\begin{array}{cc}
                    1 & 0 \\
                    0 & 1 
                \end{array}\right), \left(\begin{array}{cc}
                    1 & 1 \\
                    0 & 1 
                \end{array}\right),\left(\begin{array}{cc}
                    1 & 0 \\
                    1 & 1 
                \end{array}\right), \left(\begin{array}{cc}
                    1 & 1 \\
                    1 & 0 
                \end{array}\right),\left(\begin{array}{cc}
                    0 & 1 \\
                    1 & 0 
                \end{array}\right), \left(\begin{array}{cc}
                    0 & 1 \\
                    1 & 1 
                \end{array}\right)\right\}
            \end{equation*}
            Podemos escribirlos sin que se nos olvide ninguna pensando en que tenemos que escribir todas las matrices de forma que los vectores formados por las columnas sean linealmente independientes entre sí (para así conseguir un determinante no nulo).
        \item Tenemos $|\GL_3(\mathbb{Z}_2)| = 168$. Se deja como ejercicio escribir todas las matrices.
        \item Tenemos $|\GL_2(\mathbb{Z}_3)| = 48$.
    \end{itemize}
\end{ejemplo}

\subsection{Grupo lineal especial $\SL_n(\mathbb{F})$}
\begin{definicion}[Grupo lineal especial $\SL_n(\mathbb{F})$] Sea $\mathbb{F}$ un cuerpo finito, en $\mathcal{M}_n(\mathbb{F})$ consideramos el conjunto:
    \begin{equation*}
        \SL_n(\mathbb{F}) = \{A\in \mathcal{M}_n(\mathbb{F}) \mid \det(A)=1\}
    \end{equation*}
    Se verifica que $(\SL_n(\mathbb{F}),\cdot,I)$ es un grupo:
    \begin{itemize}
        \item La asociatividad de $\cdot $ viene heredada de la de $\cdot $ en $\mathcal{M}_n(\mathbb{F})$.
        \item Como el determinante del producto es el producto de los determinantes, $\cdot $ es una operación cerrada para $\SL_n(\bb{F})$.
        \item $\det(I)=1$ y se tiene que $I$ es el elemento neutro para $\cdot $.
        \item Como consideramos las matrices con determinante $1\neq 0$, sabemos que todas estas tienen inversa.
    \end{itemize}
    A $\SL_n(\mathbb{F})$ lo llamamos el grupo lineal especial de orden $n$.
\end{definicion}

\begin{prop}\label{prop:orden_sl}
    Sea $n\in \mathbb{N}$, si $|\bb{F}| = q$, entonces se verifica que:
    \begin{equation*}
        |\SL_n(\bb{F})| = \dfrac{|\GL_n(\bb{F})|}{q-1}
    \end{equation*}
    \begin{proof}
    Sea $A\in \GL_n(\bb{F})$, observemos que $\det(A)$ puede\footnote{De hecho los toma, es fácil comprobar que $\det:\GL_n(\bb{F})\to\bb{F}^\ast$ es una aplicación sobreyectiva.} tomar $q-1$ valores distintos, uno por cada elemento de $\bb{F}^\ast$. De esta forma, si dado $k\in \bb{F}^\ast$ definimos:
        \begin{equation*}
            D_k = \{A\in \GL_n(\bb{F}) : \det(A) = k\}
        \end{equation*}
        Es claro que estos conjuntos forman una partición de $\GL_n(\bb{F})$:
        \begin{equation*}
            \GL_n(\bb{F}) = \biguplus_{k\in \bb{F}^\ast} D_k 
        \end{equation*}
        Veamos que $|D_k| = |D_1|$ para todo $k\in \bb{F}^\ast$. Para ello, sea $k\in \bb{F}^\ast$, definimos la aplicación $\varphi_k:\GL_n(\bb{F})\to\GL_n(\bb{F})$ dada por:
        \begin{equation*}
            \varphi_k(A) = \varphi_k({(a_{ij})}_{i,j}) = {(\overline{a_{ij}})}_{i,j} \qquad \forall A={(a_{ij})}_{i,j} \in \GL_n(\bb{F})
        \end{equation*}
        Donde:
        \begin{equation*}
            \overline{a_{ij}} = \left\{\begin{array}{ll}
                    ka_{ij} & \text{si\ } i = 1 \\
                    a_{ij} & \text{si\ } i \neq 1 
            \end{array}\right.
        \end{equation*}
        Es decir, $\varphi_k$ multiplica la primera fila de una matriz por $k$. De esta forma, las propiedades de los determinantes nos dicen que si $A\in D_1$, entonces:
        \begin{equation*}
            \det(\varphi_k(A)) = k\cdot \det(A) = k
        \end{equation*}
        Por lo que $\varphi_k(A)\in D_k$ para todo $k\in \bb{F}^\ast$. Por tanto, podemos definir la aplicación $\psi_k:D_1\to D_k$ de forma que $\psi_k = \varphi_{k_{|D_1}}$. Veamos que $\psi_k$ es biyectiva para terminar el razonamiento. Para ello, dada $\psi_k$ para un cierto $k\in \bb{F}^\ast$, consideramos $\psi_{k^{-1}}$. Como:
        \begin{equation*}
            kk^{-1}a_{ij} = k^{-1}ka_{ij} = a_{ij} \qquad \forall a_{ij}\in \bb{F}^\ast
        \end{equation*}
        Concluimos que $\psi_k^{-1} = \psi_{k^{-1}}$ y además vemos que $\varphi_{k^{-1}}(D_k) \subseteq D_1$. Llegamos a que $\psi_k$ es biyectiva, por lo que $|D_k| = |D_1|$ para todo $k\in \bb{F}^\ast$.\\

        \noindent
        Sea ahora $\phi:\{1,\ldots,q-1\}\to\bb{F}^\ast$ cualquier biyección de forma que $\phi(1)=1$, como los $D_k$ formaban una partición finita de $\GL_n(\bb{F})$, tenemos que:
        \begin{equation*}
            |\GL_n(\bb{F})| = \sum_{k=1}^{q-1}|D_{\phi(k)}| = \sum_{k=1}^{q-1}|D_1| = (q-1)|D_1| = (q-1)|\SL_n(\bb{F})|
        \end{equation*}
        De donde deducimos que:
        \begin{equation*}
            |\SL_n(\bb{F})| = \dfrac{|\GL_n(\bb{F})|}{q-1}
        \end{equation*}
    \end{proof}
\end{prop}

\begin{ejemplo}
    Tenemos:
    \begin{itemize}
        \item $|\SL_2(\mathbb{Z}_3)| = 24$.
        \item $\SL_n(\mathbb{Z}_2)=\GL_n(\mathbb{Z}_2)$ $\forall n\in \mathbb{N}$
    \end{itemize}
\end{ejemplo}

\section{Homomorfismos de grupos}
\begin{definicion}[Homomorfismo]
    Dados dos grupos $G$ y $H$, un homomorfismo de grupos de $G$ en $H$ es una aplicación $f:G\rightarrow H$ que verifica:
    \begin{equation*}
        f(xy) = f(x)f(y) \qquad \forall x,y\in G
    \end{equation*}
\end{definicion}

\begin{prop}
    Si $f:G\to H$ es un homomorfismo de grupos, entonces:
    \begin{enumerate}
        \item $f(1) = 1$
        \item $f(x^{-1}) = {(f(x))}^{-1}$
        \item $f(x^n) = {(f(x))}^{n}$ $\forall n\in \mathbb{N}$
    \end{enumerate}
    \begin{proof}
        Veamos cada una:
        \begin{enumerate}
            \item $f(1) = f(1\cdot 1) = f(1)f(1)\Longrightarrow f(1) = 1$
            \item $1 = f(1) = f(xx^{-1}) = f(x)f(x^{-1}) \Longrightarrow f(x^{-1}) = {(f(x))}^{-1}$
            \item $f(x^n) = f(\underbrace{x\cdot \ldots \cdot x}_{n \text{\ veces}}) = \underbrace{f(x)\cdot \ldots\cdot f(x)}_{n \text{\ veces}} = {(f(x))}^{n}$
        \end{enumerate}
    \end{proof}
\end{prop}

\begin{definicion}
Sea $f:G\to H$ un homomorfismo de grupos, distinguimos:
\begin{itemize}
    \item $\ker f = \{x\in G \mid f(x) = 1\}$
    \item $Im f = \{f(x) \mid x\in G\}$
\end{itemize}
\end{definicion}

\begin{ejemplo}
    Ejemplos de homomorfismos de grupos son:
    \begin{enumerate}
        \item Dado $G$ un grupo, $id:G\to G$.
        \item Dados $G,H$ grupos, consideramos el siguiente homomorfismo, denominado \emph{homomorfismo trivial}:
            \Func{f}{G}{H}{x}{1}
        \item La exponencial es también un homomorfismo:
            \Func{\exp}{(\mathbb{R},+)}{(\mathbb{R}^+,\cdot)}{x}{e^x}
        
        \item La aplicación determinante de matrices con determinante no nulo:
            \Func{\det}{\GL_n(\mathbb{F})}{\mathbb{F}^\ast}{A}{\det(A)}
        \item La aplicación signatura:
            \Func{\veps}{S_n}{\cc{U}(\mathbb{Z})=\{-1,1\}}{\sigma}{\veps(\sigma)}
    \end{enumerate}
\end{ejemplo}

\begin{prop}\label{prop:comp_homomorf}
    Sean $f:G\to H$ y $g:H\to T$ dos homomorfismos de grupos, entonces la aplicación $g\circ f:G\to T$ es un homomorfismo de grupos.
    \begin{proof}
        Sean $x,y\in G$, entonces:
        \begin{equation*}
            (g\circ f)(xy) = g(f(xy)) = g(f(x)f(y)) = g(f(x)) g(f(y)) = (g\circ f)(x)(g\circ f)(y)
        \end{equation*}
    \end{proof}
\end{prop}

\begin{definicion}
    Dado $f:G\to H$ un homomorfismo de grupos, decimos que:
    \begin{itemize}
        \item $f$ es un monomorfismo si es inyectiva.
        \item $f$ es un epimorfismo si es sobreyectiva.
        \item $f$ es un isomorfismo si es biyectiva.
        \item Si $G=H$, diremos que $f$ es un endomorfismo.
        \item Si $f$ es un endomorfismo biyectivo, diremos que es un automorfismo.
    \end{itemize}
\end{definicion}

\begin{prop}\label{prop:propiedades_homorf}
    Sea $f:G\to H$ un homomorfismo de grupos, entonces:
    \begin{enumerate}
        \item[$i)$] $f$ es monomorfismo $\Longleftrightarrow \ker(f)=\{1\}$
        \item[$ii)$] $f$ es isomorfismo $\Longleftrightarrow f^{-1}$ es un isomorfismo.
    \end{enumerate}
    \begin{proof}
            Veamos los dos resultados:
        \begin{enumerate}
            \item[$i)$] Para el primero, demostramos las dos implicaciones:
                \begin{description}
                    \item [$\Longrightarrow)$] $x\in \ker(f) \Longrightarrow f(x) = 1 = f(1)$, pero como $f$ es inyectiva, tenemos que $x=1$.
                    \item [$\Longleftarrow)$] Sean $x,y\in G$ de forma que $f(x)=f(y)$, entonces:
                        \begin{equation*}
                            f(x){(f(y))}^{-1} = 1 \Longrightarrow f(xy^{-1}) = 1 \Longrightarrow xy^{-1} = 1 \Longrightarrow x = y
                        \end{equation*}
                        Concluimos que $f$ es inyectiva.
                \end{description}
            \item[$ii)$] Demostramos las dos implicaciones:
                \begin{description}
                    \item [$\Longrightarrow)$] Si $f$ es un isomorfismo, entonces es biyectiva, por lo que tendrá una aplicación inversa $f^{-1}$, que por lo pronto ya sabemos que es biyectiva. Basta ver que esta aplicación es un homomorfismo. Para ello, sean ${y,y'\in H}$, por ser $f$ un biyectiva, existirán $x,x'\in G$ de forma que $f(x) = y$ y $f(x')=y'$, luego $x = f^{-1}(y)$ y $x' = f^{-1}(y')$. Por tanto:
                        \begin{equation*}
                            f^{-1}(yy') = f^{-1}(f(x)f(x')) = f^{-1}(f(xx')) = xx' = f^{-1}(y)f^{-1}(y')
                        \end{equation*}
                        Lo que demuestra que $f^{-1}$ es un homomorfismo biyectivo, luego isomorfismo.
                    \item [$\Longleftarrow)$] Si $f^{-1}$ es un isomorfismo, entonces por la implicación que acabamos de demostrar, ${(f^{-1})}^{-1} = f$ también es un isomorfismo.
                    \qedhere
                \end{description}
        \end{enumerate}
    \end{proof}
\end{prop}

\begin{definicion}[Grupos isomorfos]
    Sean $G$ y $H$ dos grupos, decimos que son isomorfos si existe un isomorfismo entre ellos, que se denotará por $G\cong H$.
\end{definicion}

\begin{prop}
    La propiedad de ser isomorfo es una relación de equivalencia.
    \begin{proof}
        Demostramos cada una de las propiedades:
        \begin{itemize}
            \item \underline{Propiedad reflexiva.} Sea $G$ un grupo, como $id:G\to G$ es un homomorfismo, tenemos que $G\cong G$.
            \item \underline{Propiedad simétrica.} Sean $G$ y $H$ dos grupos de forma que $G\cong H$, entonces existe un isomorfismo $f:G\to H$. Por la Proposición~\ref{prop:propiedades_homorf}, $f^{-1}:H\to G$ también será un isomorfismo, por lo que $H \cong G$.
            \item \underline{Propiedad transitiva.} Sean $G$, $H$ y $T$ tres grupos de forma que $G\cong H$ y $H\cong T$, entonces existen dos isomorfismos: $f:G\to H$ y $g:H\to T$. Si consideramos $g\circ f:G\to T$, tenemos por la Proposición~\ref{prop:comp_homomorf} que $g\circ f$ es un isomorfismo de $G$ en $T$, por lo que $G\cong T$.
        \end{itemize}
    \end{proof}
\end{prop}

\begin{prop}\label{prop:propiedades_grupos_isomorf}
    Se verifican:
    \begin{enumerate}
        \item[$i)$] Si $f:X\to Y$ es una aplicación biyectiva, se tiene que la aplicación siguiente es un isomorfismo de grupos:
        \Func{\varphi}{\Perm(X)}{\Perm(Y)}{\sigma}{f\sigma f^{-1}}
        \item[$ii)$] $Aut(G) = \{f:G\to G \mid f \text{\ automorfismo}\}$ con la composición forman un grupo.
        \item[$iii)$] Si $f:G\to H$ es un isomorfismo, entonces $|G| = |H|$.
        \item[$iv)$] Si $G$ y $H$ son isomorfos, entonces $G$ es abeliano $\Longleftrightarrow H$ es abeliano.
        \item[$v)$] Si $f:G\to H$ es un isomorfismo, entonces se mantiene el orden:
            \begin{equation*}
                O(x) = O(f(x)) \qquad \forall x\in G
            \end{equation*}
        \item[$vi)$] Si $f:G\to H$ es un epimorfismo y $S=\{s_1,\ldots,s_n\}\subseteq G$ cumple que $G= \langle S \rangle $, entonces $H=\langle f(S) \rangle $.
    \end{enumerate}
    \begin{proof}
        Veamos cada una:
        \begin{enumerate}
            \item[$i)$] Hemos de ver que $\varphi$ es un homomorfismo biyectivo:
                \begin{itemize}
                    \item Sean $\sigma,\tau \in \Perm(X)$, entonces:
                        \begin{equation*}
                            \varphi(\sigma\tau) = f\sigma\tau f^{-1} \AstIg f\sigma f^{-1}f\tau f^{-1} = \varphi(\sigma)\varphi(\tau)
                        \end{equation*}
                        Donde podemos ver $(\ast)$ descomponiendo en ciclos disjuntos tanto $\sigma$ como $\tau$ y aplicando la Proposición~\ref{prop:conjugar_permutaciones}.
                    \item Definimos la siguiente aplicación:
                        \Func{\psi}{\Perm(Y)}{\Perm(X)}{\tau}{f^{-1}\tau f}

                        Veamos que $\psi$ es la inversa de $\varphi$:
                        \begin{align*}
                            \psi(\varphi(\sigma)) &= \psi(f\sigma f^{-1}) = f^{-1}f\sigma f^{-1}f = \sigma \\
                            \varphi(\psi(\tau)) &= \varphi(f^{-1}\tau f) = f(f^{-1}\tau f)f^{-1} = \tau
                        \end{align*}

                        Por tanto, $\varphi$ es biyectiva.
                \end{itemize}
                Como $\varphi$ es un homomorfismo biyectivo, es un isomorfismo.
            \item[$ii)$] La asociatividad viene heredada de la asociatividad de funciones, el neutro del grupo es $id:G\to G$ y como son automorfismos, son aplicaciones biyectivas, con lo que cada una tiene inversa.
            \item[$iii)$] Por ser $f$ biyectiva, se tiene $|G| = |H|$.
            \item[$iv)$] Veamos las dos implicaciones:
                \begin{description}
                    \item [$\Longrightarrow)$] Sean $x,y\in H$, existirá un isomorfismo $f:G\to H$, luego:
                        \begin{equation*}
                            xy = f(f^{-1}(xy)) = f(f^{-1}(x)f^{-1}(y)) = f(f^{-1}(y) f^{-1}(x)) = f(f^{-1}(yx)) = yx
                        \end{equation*}
                    \item [$\Longleftarrow)$] Como $G\cong H \Longleftrightarrow H\cong G$ por la propiedad simétrica, se tiene la otra implicación.
                \end{description}
            \item[$v)$] Si $O(x) = n$, entonces:
                \begin{equation*}
                    {(f(x))}^{n} = f(x^n) = f(1) = 1
                \end{equation*}
                Por tanto, tenemos que $O(f(x))\leq n$.
                Si suponemos ahora que $\exists m\in \bb{N}$ tal que ${(f(x))}^{m}=1$, entonces $f(x^m)= 1 = f(1)$ y por inyectividad tenemos que $x^m = 1$, luego $n\leq m$. De todo esto deducimos que $O(f(x)) = n$.

                Si $O(x)=+\infty$, basta observar que $f(x^n) = {(f(x))}^{n}$ para todo $n\in \mathbb{N}\setminus\{0\}$, para concluir que $O(f(x))=+\infty$. Si $O(f(x))=+\infty$, basta usar $f^{-1}$.
            \item[$vi)$] Sea $y\in H$, buscamos una descomposición de $y$ en función de los elementos $f(s_i)$. Para ello, como $f$ es sobreyectiva, existirá $x\in G$ de forma que $y = f(x)$. Como $G=\langle S \rangle $, tendremos que existen $\gamma_1,\ldots,\gamma_k \in \mathbb{Z}$ de forma que:
                \begin{equation*}
                    x = s_1^{\gamma_1}s_2^{\gamma_2}\ldots s_k^{\gamma_k}
                \end{equation*}
                Luego:
                \begin{equation*}
                    y = f(x) = f(s_1^{\gamma_1}s_2^{\gamma_2}\ldots s_k^{\gamma_k}) = {f(s_1)}^{\gamma_1}{f(s_2)}^{\gamma_2} \ldots {f(s_k)}^{\gamma_k}
                \end{equation*}
                Por lo que $H = \langle f(s_1),f(s_2),\ldots,f(s_n) \rangle = \langle f(S) \rangle $.
        \end{enumerate}
    \end{proof}
\end{prop}

\subsection{Ejemplos}
\begin{teo}[de Dyck]\label{teo:Dyck}
    Sea $G$ un grupo finito con una presentación
    \begin{equation*}
        G = \langle S\mid R_1,R_2,\ldots,R_k \rangle \qquad S = \{s_1,\ldots,s_m\}
    \end{equation*}
    Sea $H$ otro grupo finito con $\{r_1,\ldots,r_m\}\subseteq H$, y supongamos que cualquier relación satisfecha en $G$ por los $s_i$ con $i \in \{1,\ldots,m\}$ es también satisfecha en $H$ para los $r_i$ con $i \in \{1,\ldots,m\}$. Entonces existe un único homomorfismo de grupos $f:G\to H$ de forma que:
    \begin{equation*}
        f(s_i) = r_i \qquad i \in \{1,\ldots,n\}
    \end{equation*}
    \begin{itemize}
        \item Si además $\{r_1,\ldots,r_m\}$ son un conjunto de generadores de $H$, entonces $f$ es un epimorfismo.
        \item Más aún, si $|G| = |H|$, entonces $f$ es un isomorfismo.
    \end{itemize}
    % \begin{proof}
        % // TODO: Hacer
        % Hace falta saber cosas de grupos libres
    % \end{proof}
\end{teo}

\begin{ejemplo}
    Usando el Teorema~\ref{teo:Dyck}, podemos dar muchos ejemplos de grupos isomorfos:
    \begin{enumerate}
        \item Si consideramos el grupo cíclico de orden $n$: $C_n = \langle x \mid x^n = 1 \rangle $.

            Observamos que en $\mathbb{Z}_n$ el elemento $\overline{1}$ también verifica la propiedad $x^n = 1$, ya que:
            \begin{equation*}
                n\cdot \overline{1} = \underbrace{\overline{1} + \ldots + \overline{1}}_{n \text{\ veces}} = 0
            \end{equation*}
            De esta forma, por el Teorema~\ref{teo:Dyck}, sabemos que existe un homomorfismo $f:C_n\to \mathbb{Z}_n$, de forma que $f(x) = 1$.

            Más aún, como $\mathbb{Z}_n = \left\langle \overline{1} \right\rangle $ y $|C_n| = n = |\mathbb{Z}_n|$, tenemos que $f$ es un isomorfismo de grupos, por lo que $C_n\cong \mathbb{Z}_n$.
        \item Si ahora consideramos el grupo de Klein abstracto:
            \begin{equation*}
                V^{\text{abs}} = \langle x,y \mid x^2=y^2=1, xy=yx \rangle 
            \end{equation*}
            Podemos intentar relacionarlo con el grupo directo $\mathbb{Z}_2\times\mathbb{Z}_2$, ya que los elementos $(0,1)$ y $(1,0)$ cumplen las relaciones enunciadas:
            \begin{align*}
                &2\cdot (0, 1) = (0, 1) + (0, 1) = (0, 0) \\
                &2\cdot (1, 0) = (1, 0) + (1, 0) = (0, 0) \\
                &(0, 1) + (1, 0) = (1, 1) = (1, 0) + (0, 1)
            \end{align*}
            Por lo que existirá un homomorfismo $f:V^{\text{abs}}\to\mathbb{Z}_2\times\mathbb{Z}_2$ de forma que $f(x) = (0, 1)$ y $f(y) = (1, 0)$.

            Más aún, como $\mathbb{Z}_2\times\mathbb{Z}_2 = \langle (0,1),(1,0) \rangle $ y es claro que $|\mathbb{Z}_2\times\mathbb{Z}_2| = 4 = |V^{\text{abs}}|$, tenemos que $f$ es un isomorfismo, por lo que $V^{\text{abs}}\cong \mathbb{Z}_2\times\mathbb{Z}_2$.
        \item Si tratamos ahora de relacionar el grupo de Klein abstracto (visto en el ejemplo anterior) con el grupo de Klein:
            \begin{equation*}
                V = \langle (1\ 2)(3\ 4), (1\ 3)(2\ 4) \rangle  = \{1, (1\ 2)(3\ 4), (1\ 3)(2\ 4), (1\ 4)(2\ 3)\}
            \end{equation*}
            Como $(1\ 2)(3\ 4)$ y $(1\ 3)(2\ 4)$ verifican que:
            \begin{align*}
                &{(1\ 2)(3\ 4)}^{2} = (1\ 2)(3\ 4)(1\ 2)(3\ 4) = 1 \\
                &{(1\ 3)(2\ 4)}^{2} = (1\ 3)(2\ 4)(1\ 3)(2\ 4) = 1 \\
                &(1\ 2)(3\ 4)(1\ 3)(2\ 4) = (1\ 4)(2\ 3) = (1\ 3)(2\ 4)(1\ 2)(3\ 4)
            \end{align*}
            Por el Teorema de Dyck, existe un homomorfismo $g:V^{\text{abs}}\to V$ de forma que $g(x) = (1\ 2)(3\ 4)$ y $g(y) = (1\ 3)(2\ 4)$.

            Como hemos visto ya que $V=\langle g(x), g(y) \rangle $ y que $|V^\text{abs}| = 4 = |V|$, $g$ es un isomorfismo. Tenemos que $V^{\text{abs}}\cong V$.

            Como vimos que $\cong$ es una relación de equivalencia, también tendremos que $V\cong \mathbb{Z}_2\times\mathbb{Z}_2$.
        \item Consideramos ahora el grupo diédrico de orden 3:
            \begin{equation*}
                D_3 = \langle r,s\mid r^3 = 1, s^2=1, sr = r^2s \rangle 
            \end{equation*}
            Que vamos a intentar relacionar con $S_3$. Como $(1\ 2)$ y $(1\ 2\ 3)$ verifican que:
            \begin{align*}
                &{(1\ 2\ 3)}^{3} = (1\ 2\ 3)(1\ 2\ 3)(1\ 2\ 3) = 1 \\
                &{(1\ 2)}^{2} = (1\ 2)(1\ 2) = 1 \\
                &(1\ 2)(1\ 2\ 3) = (2\ 3) = (1\ 3\ 2)(1\ 2) = {(1\ 2\ 3)}^{2}(1\ 2)
            \end{align*}
            Tenemos que existe un homomorfismo $f:D_3\to S_3$ de forma que $f(r)=(1\ 2\ 3)$ y $f(s) = (1\ 2)$. Como además tenemos que\footnote{Esto se vio en la Proposición~\ref{prop:generadores_grupos}.} $S_3 = \langle (1\ 2)(1\ 2\ 3) \rangle $ y que $|D_3| = 2\cdot 3 = 6 = 3! = |S_3|$, concluimos que $f$ es un isomorfismo, por lo que $D_3 \cong S_3$.
            % // TODO: Ejercicio 18
        \item Si consideramos el grupo lineal de orden 2 sobre $\mathbb{Z}_2$:
            \begin{equation*}
                \GL_2(\mathbb{Z}_2) = \left\{\left(\begin{array}{cc}
                    1 & 0 \\
                    0 & 1 
                \end{array}\right), \left(\begin{array}{cc}
                    1 & 1 \\
                    0 & 1 
                \end{array}\right), \left(\begin{array}{cc}
                    1 & 0 \\
                    1 & 1 
                \end{array}\right), \left(\begin{array}{cc}
                    1 & 1 \\
                    1 & 0 
                \end{array}\right), \left(\begin{array}{cc}
                    0 & 1 \\
                    1 & 0 
                \end{array}\right), \left(\begin{array}{cc}
                    0 & 1 \\
                    1 & 1 
                \end{array}\right)\right\}
            \end{equation*}
            Y tratamos de relacionarlo con $S_3 = \langle r,s\mid r^3 = 1, s^2 = 1, sr = r^2s \rangle$, como tenemos que:
            \begin{align*}
                {\left(\begin{array}{cc}
                    0 & 1 \\
                    1 & 1 
                \end{array}\right)}^{3} &= \left(\begin{array}{cc}
                    1 & 0 \\
                    0 & 1 
                \end{array}\right) \\
                {\left(\begin{array}{cc}
                    0 & 1 \\
                    1 & 0 
                \end{array}\right)}^2 &= \left(\begin{array}{cc}
                    1 & 0 \\
                    0 & 1 
                \end{array}\right) \\
                \left(\begin{array}{cc}
                    0 & 1 \\
                    1 & 0 
                \end{array}\right)\left(\begin{array}{cc}
                    0 & 1 \\
                    1 & 1 
                    \end{array}\right) &= {\left(\begin{array}{cc}
                    0 & 1 \\
                    1 & 1 
                \end{array}\right)}^2 \left(\begin{array}{cc}
                    0 & 1 \\
                    1 & 0 
                \end{array}\right)
            \end{align*}
            Entonces, existe un homomorfismo $f:S_3\to \GL_2(\mathbb{Z}_2)$ de forma que:
            \begin{equation*}
                f(r) = \left(\begin{array}{cc}
                    0 & 1 \\
                    1 & 1 
                \end{array}\right)\qquad 
                f(s) = \left(\begin{array}{cc}
                    0 & 1 \\
                    1 & 0 
                \end{array}\right) 
            \end{equation*}
            Además, como (ver el Ejercicio~\ref{ej:2.10}):
            \begin{equation*}
                \GL_2(\mathbb{Z}_2) = \left\langle  \left(\begin{array}{cc}
                    0 & 1 \\
                    1 & 1 
                \end{array}\right), \left(\begin{array}{cc}
                    0 & 1 \\
                    1 & 0 
                \end{array}\right)\right\rangle 
            \end{equation*}
            Y ambos tienen el mismo número de elementos, $f$ es un isomorfismo.
        \item Fijado $n\in \mathbb{N}\setminus\{0,3\}$, si ahora consideramos el grupo simétrico de orden $n$, $S_n$ y el grupo diédrico de orden $n$, $D_n$, como $|D_n| = 2n \neq n! = |S_n|$ no vamos a tener un isomorfismo de grupos. Sin embargo, los elementos:
            \begin{equation*}
                (1\ 2\ \ldots\ n), \left(\begin{array}{cccccc}
                        1 & 2 & 3 & \ldots & n-1 & n \\
                        1 & n & n-1 & \ldots & 3 & 2
                \end{array}\right) \in S_n
            \end{equation*}
            Verifican todas las propiedades de la presentación de $D_n$, por lo que existirá un homomorfismo $f:D_n\to S_n$ de forma que 
            \begin{gather*}
                f(r) = (1\ 2\ \ldots\ n) \\
                f(s) = \left(\begin{array}{cccccc}
                        1 & 2 & 3 & \ldots & n-1 & n \\
                        1 & n & n-1 & \ldots & 3 & 2
                \end{array}\right)
            \end{gather*}
        \item Si consideramos ahora:
            \begin{equation*}
                Q_2^{\text{abs}} = \langle x,y\mid x^4=1, y^2=x^2, yxy^{-1}=x^{-1} \rangle 
            \end{equation*}
            Y pensamos en relacionarlo con $Q_2 = \{\pm 1, \pm i, \pm j, \pm k\}$, como tenemos que:
            \begin{align*}
                &i^4 = 1 \\
                &j^2 = -1 = i^2 \\
                &ji(-j) = j(-k) = -i 
            \end{align*}
            Sabemos que existe un homomorfismo $f:Q_2^{\text{abs}}\to Q_2$ de forma que $f(x) = i$ y $f(y) = j$. Además, como $Q_2 = \langle i, j \rangle $ y $|Q_2^{\text{abs}}| = 4 = |Q_2|$, tenemos que $f$ es un isomorfismo, por lo que $Q_2^{\text{abs}}\cong Q_2$.
        \item Como último ejemplo, si consideramos $k,n\in \mathbb{N}$, $k\geq 3$ con $k\mid n$ y consideramos los grupos diédricos:
            \begin{align*}
                D_n &= \langle r,s\mid r^n=1, s^2 = 1, sr=r^{-1}s \rangle  \\
                D_k &= \langle r_1,s_1\mid r_1^k=1, s_1^2 = 1, s_1r_1=r_1^{-1}s_1 \rangle  
            \end{align*}
            Y tratamos de relacionarlos, como $k\mid n$, existirá $p\in \mathbb{N}$ de forma que $n = kp$.

            Como $r_1,s_1\in D_k$ verifican que:
            \begin{align*}
                &r_1^n = r_1^{kp} = {(r_1^k)}^{p} = 1^p = 1 \\
                &s_1^2 = 1 \\
                &s_1r_1 = r_1^{-1}s_1
            \end{align*}
            Tenemos por el Teorema~\ref{teo:Dyck} que existe un homomorfismo $f:D_n\to D_k$ de forma que $f(r) = r_1$ y $f(s) = s_1$.
    \end{enumerate}
\end{ejemplo}

\section{Resumen de grupos}
Para finalizar este capítulo, haremos un breve repaso de los grupos vistos hasta el momento, ya que los usaremos de forma constante a lo largo de la asignatura, por lo que conviene tenerlos siempre presentes.

\begin{description}
    \item [Grupo Trivial.] $(\{e\},\ast, e)$.
    \item [Grupos de los enteros módulo $n$.] $(\mathbb{Z}_n, +)$, $(\cc{U}(\mathbb{Z}_n),\cdot )$.
    \item [Grupo de raíces $n-$ésimas de la unidad.]~\\
        \begin{equation*}
            \mu_n = \left\{1,\xi,\xi^2, \ldots, \xi^{n-1}\mid \xi = \cos\left(\dfrac{2\pi}{n}\right) + i\sen\left(\dfrac{2\pi}{n}\right)\right\} \subseteq \mathbb{C}
        \end{equation*}
    \item [Grupo lineal de orden $n$.] Sea $\bb{F}$ un cuerpo:
        \begin{equation*}
            \GL_n(\bb{F}) = \{A\in \cc{M}_n(\bb{F}) \mid \det(A) \neq 0\}
        \end{equation*}
    \item [Grupo lineal especial de orden $n$.] Sea $\bb{F}$ un cuerpo:
        \begin{equation*}
            \SL_n(\bb{F}) = \{A\in \cc{M}_n(\bb{F}) \mid \det(A) = 1\}
        \end{equation*}
    \item [Potencias de grupos.] Sea $G$ un grupo y $X$ un conjunto:
        \begin{equation*}
            G^X = Apl(X, G) = \{f:X\to G \mid f \text{\ aplicación}\}
        \end{equation*}
    \item [$n-$ésimo grupo diédrico.] Sea $n\in \mathbb{N}$:
        \begin{equation*}
            D_n = \{1,r,r^2, \ldots, r^{n-1},s,sr,sr^2, \ldots, sr^{n-1}\}
        \end{equation*}
    \item [$n-$ésimo grupo simétrico.] Sea $X$ un conjunto con $|X| = n\in \mathbb{N}$:
        \begin{equation*}
            S_n = \Perm(X) = \{f:X\to X \mid f \text{\ biyectiva}\}
        \end{equation*}
    \item [$n-$ésimo grupo alternado.] Sea $n\in \mathbb{N}$:
        \begin{equation*}
            A_n = \{\sigma\in S_n \mid \sigma \text{\ es par}\}
        \end{equation*}
    \item [Grupo cíclico de orden $n$.] Sea $n\in \mathbb{N}$:
        \begin{equation*}
            C_n = \langle x\mid x^n = 1 \rangle  = \{1,x,x^2,x^3,\ldots,x^{n-1}\}
        \end{equation*}
    \item [Grupo de los cuaternios.] 
        \begin{equation*}
            Q_2 = \{\pm 1, \pm i,\pm j,\pm k\}
        \end{equation*}
    \item [Grupo abstracto $Q_2^\text{abs}$.] 
        \begin{align*}
            Q_2^\text{abs} &= \langle x,y\mid x^4=1, y^2 = x^2, yxy^{-1}=x^{-1} \rangle \\
                           &= \{1,x,x^2,x^3,y,yx,yx^2,yx^3\}
        \end{align*}
    \item [Grupo de Klein.] Sea $n\in \mathbb{N}$ con $n\geq 4$:
        \begin{equation*}
            V = \{1, (1\ 2)(3\ 4), (1\ 3)(2\ 4), (1\ 4)(2\ 3)\} \subseteq S_n
        \end{equation*}
    \item [Grupo de Klein abstracto.]
        \begin{equation*}
            V^{\text{abs}} = \langle x,y\mid x^2=y^2=1, xy=yx \rangle  = \{1,x,y,xy\}
        \end{equation*}
\end{description}

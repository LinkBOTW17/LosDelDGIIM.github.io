\chapter{Clasificación de grupos abelianos finitos}
\noindent
El objetivo final del tema es demostrar los teoremas de estructura de los grupos abelianos finitos, que permiten clasificar todos estos grupos según su orden (para cada orden tendremos una clasificación), salvo isomorfismos.\\

Serán de especial relevancia varios resultados que ya hemos visto: % // TODO: Buscarlos para referenciarlos
\begin{itemize}
    \item $C_n\times C_m \cong C_{nm} \Longleftrightarrow \mcd(n,m) = 1$, en la Proposición~\ref{prop:carac_prod_finito_ciclicos}.
    \item Si $|G| = p_1^{n_1} \ldots p_k^{n_k}$ y $G$ tenía un único $P_i$ $p_i-$subgrupo de Sylow para cada $i \in \{1,\ldots,k\}$, entonces $G\cong P_1\times P_2\times \ldots \times P_k$.
\end{itemize}
Como trabajaremos con subgrupos abelianos, recordamos que la notación que usábamos para el producto directo de grupos abelianos era $\oplus$.

% // TODO: Cambiar posiblemente el enunciado:
% - Para cada particion de n hay una forma unica
\begin{teo}[Estructura de los $p-$grupos abelianos finitos]\label{teo:1_tema6}\ \\
    Sea $A$ un $p-$grupo abeliano finito con orden $|A| = p^n$ para $n\geq 1$, entonces existen enteros $\beta_1\geq \beta_2 \geq \ldots \geq \beta_t \geq 1$ de forma que:
    \begin{equation*}
        \beta_1 + \beta_2 + \ldots + \beta_t = n \quad \text{y} \quad A\cong C_{p^{\beta_1}} \oplus C_{p^{\beta_2}} \oplus \ldots \oplus C_{p^{\beta_t}}
    \end{equation*}
    Además, esta expresión es única, es decir, si existen $\alpha_1\geq \alpha_2\geq \ldots \geq \alpha_s \geq 1$ de forma que:
    \begin{equation*}
        \alpha_1 + \alpha_2 + \ldots + \alpha_s = n \quad \text{y} \quad A\cong C_{p^{\alpha_1}} \oplus C_{p^{\alpha_2}} \oplus \ldots \oplus C_{p^{\alpha_s}}
    \end{equation*}
    entonces $s = t$ y $\alpha_k = \beta_k$, para todo $k \in \{1,\ldots,t\}$.
    % \begin{proof} % // TODO: Copiar de Prado
    % \end{proof}
\end{teo}

\begin{observacion}
    Notemos que lo que estamos haciendo es tomar particiones de $n$ de la forma $\beta_i$, y este Teorema nos dice que el $p-$grupo puede escribirse de forma única salvo isomorfismos como producto de ciertos subgrupos cíclicos.\newline

    Es decir, existen tantos $p-$grupos abelianos de orden $p^n$ como particiones tengamos del número $n$.
\end{observacion}

\begin{ejemplo}
    Por ejemplo:
    \begin{itemize}
        \item Grupos abelianos finitos de orden $8 = 2^3$, tenemos como particiones:
            \begin{align*}
                3& \Longrightarrow A\cong C_8 \\
                1, 2& \Longrightarrow A\cong C_4\oplus C_2 \\
                1, 1, 1& \Longrightarrow A\cong C_2 \oplus C_2 \oplus C_2
            \end{align*}
        \item Los grupos abelianos finitos de orden $81 = 3^4$, tenemos como particiones:
            \begin{align*}
                A &\cong C_{81} \\
                A &\cong C_{27} \oplus C_3 \\
                A &\cong C_9 \oplus C_9 \\
                A &\cong C_9 \oplus C_3 \oplus C_3 \\
                A &\cong C_3 \oplus C_3 \oplus C_3 \oplus C_3 
            \end{align*}
    \end{itemize}
\end{ejemplo}

\begin{teo}[Estructura de los grupos abelianos finitos]\label{teo:2_tema6}\ \\
    Sea $A$ un grupo abeliano finito con $|A| = p_1^{\gamma_1}\ldots p_k^{\gamma_k}$ siendo $p_i$ primo $\forall i \in \{1,\ldots,k\}$, entonces:
    \begin{equation*}
        A \cong \bigoplus_{i=1}^k \left(\bigoplus_{j=1}^{t_i} C_{p_i^{n_{ij}}}\right)
    \end{equation*}
    Donde para cada $i \in \{1,\ldots,k\}$ tenemos:
    \begin{align*}
        n_{i1} \geq n_{i2} \geq \ldots \geq n_{it_{i}} \geq 1 \\
        n_{i1} + n_{i2} + \ldots + n_{it_{i}} = r_i
    \end{align*}
    Y la descomposición es única salvo el orden.\\

    \noindent
    Esta última recibe el nombre de descomposición cíclica primaria, y a los $p_i^{n_{ij}}$ con $i \in \{1,\ldots,k\}$ y $j \in \{1,\ldots t_i\}$ se les llama divisores elementales de $A$.\newline
    \begin{proof}
        Si $A$ es abeliano y finito, entonces todos sus $p-$subgrupos de Sylow son normales, luego podemos escribir:
        \begin{equation*}
            A = P_1 \oplus P_2 \oplus \ldots \oplus P_k
        \end{equation*}
        Siendo ${P_1, P_2, \ldots, P_k}$ el conjunto de todos sus $p-$subgrupos de Sylow, de forma que $|P_i| = p_i^{r_i}$, para todo $i \in \{1,\ldots,k\}$. Como cada $P_i$ es un $p_i-$subgrupo abeliano finito, aplicando el Teorema~\ref{teo:1_tema6}, podemos escribir:
        \begin{equation*}
            P_i = \bigoplus_{j=1}^{t_i} C_{p_i^{n_{ij}}} \qquad \forall i \in \{1,\ldots,k\}
        \end{equation*}
        De donde tenemos la expresión de la tesis.\newline
        A cada $P_i$ con $i \in \{1,\ldots,k\}$ lo llamaremos componente $p_i-$primaria de $A$.
    \end{proof}
\end{teo}

\begin{ejemplo}
    Si tenemos un subgrupo finito abeliano $A$ con $|A| = 360 = 2^3\cdot 3^2\cdot 5$, veamos los divisores elementales:
    \begin{equation*}
        \begin{array}{r|l}
            \text{Div. elementales} & \text{Descomp. cíclica primaria} \\
            \hline
            2^3\ 3^2\ 5 & C_8\oplus C_9 \oplus C_5 \\
            2^2\ 2\ 3^2\ 5 & C_4\oplus C_2 \oplus C_9 \oplus C_5\\
            2\ 2\ 2\ 3^2\ 5 & C_2 \oplus C_2 \oplus C_2 \oplus C_9 \oplus C_5\\
            2^3\ 3\ 3\ 5 & C_8\oplus C_3 \oplus C_3 \oplus C_5 \oplus C_8\\
            2\ 2^2\ 3\ 3\ 5 & C_2 \oplus C_4 \oplus C_3 \oplus C_3 \oplus C_5\\
            2\ 2\ 2\ 3\ 3\ 5 & C_2 \oplus C_2 \oplus C_2 \oplus C_3 \oplus C_3 \oplus C_5
        \end{array}
    \end{equation*}
    Serían todas las descomposiciones cíclicas primarias de $A$. Es decir, $A$ será isomorfo a cualquiera de esos.
\end{ejemplo}

\noindent
Sin embargo, si recordamos la Proposición~\ref{prop:carac_prod_finito_ciclicos}, esto nos llevará a la descomposición cíclica, donde observaremos por ejemplo que:
\begin{align*}
    C_8\oplus C_9 \oplus C_5 &\cong C_{360} \\
    C_4\oplus C_2 \oplus C_9 \oplus C_5 &\cong C_{180} \oplus C_2 \\
    C_2 \oplus C_2 \oplus C_2 \oplus C_9 \oplus C_5 &\cong C_{90} \oplus C_2 \oplus C_2 \\
    C_8\oplus C_3 \oplus C_3 \oplus C_5 \oplus C_8 &\cong C_{120} \oplus C_3 \\
    C_2 \oplus C_4 \oplus C_3 \oplus C_3 \oplus C_5 &\cong C_{60} \oplus C_6 \\
    C_2 \oplus C_2 \oplus C_2 \oplus C_3 \oplus C_3 \oplus C_5 &\cong C_{30} \oplus C_6 \oplus C_2
\end{align*}

Ahora, usaremos que:
\begin{equation*}
    C_n\times C_m\cong C_{nm} \Longleftrightarrow \mcd(n,m) = 1
\end{equation*}

\begin{teo}[Descomposición cíclica de un grupo abeliano finito]\label{teo:3_tema6}\ \\
    Si $A$ es un grupo abeliano finito, entonces:
    \begin{equation*}
        A\cong C_{d_1} \oplus C_{d_2} \oplus \ldots \oplus C_{d_t}
    \end{equation*}
    Donde los $d_i$ son enteros positivos de forma que:
    \begin{equation*}
        d_1d_2\ldots d_t = |A|
    \end{equation*}
    Y $d_i \mid d_j$ para cada $j \leq i$. Además, la descomposición es única salvo el orden, para cada partición.
    \begin{proof}
        Supuesto que $|A| = p_1^{r_1}\ldots p_k^{r_k}$, si usamos la descomposición que nos da el Teorema~\ref{teo:2_tema6}:
        \begin{equation*}
            A \cong \bigoplus_{i=1}^k \left(\bigoplus_{j=1}^{t_i} C_{p_i^{n_{ij}}}\right)
        \end{equation*}
        Para ciertos:
        \begin{align*}
            n_{i1} \geq n_{i2} \geq \ldots \geq n_{it_{i}} \geq 1 \\
            n_{i1} + n_{i2} + \ldots + n_{it_{i}} = r_i
        \end{align*}
        Sea $t = \max{t_1, t_2, \ldots, t_k}$, si $t_i < l \leq t$, tendremos entonces que $n_{il} = 0$.

        Lo que estamos haciendo es dada una partición, como por ejemplo la $\{2, 2^2, 3^2, 5\}$, denotar por $t_i$ al número de particiones de cada número y por $n_{ij}$ a los exponentes de cada una de las particiones, construyendo la tabla:
        \begin{equation*}
            \left\{\begin{array}{c|cc}
                    t_1 & n_{11} & n_{12} \\
                    t_2 & n_{21} & n_{22} \\
                    t_3 & n_{31} & n_{32} 
            \end{array}\right.
        \end{equation*}
        De esta forma, tenemos:
        \begin{equation*}
            \left(\begin{array}{cccc}
                p_1^{n_{11}} & p_2^{n_{21}} & \ldots & p_k^{n_{k1}} \\
                p_1^{n_{12}} & p_2^{n_{22}} & \ldots  & p_k^{n_{k2}}  \\
                 \vdots & \vdots & \ddots & \vdots \\
                 p_1^{n_{1k}} & p_2^{n_{2k}} & \ldots & p_k^{n_{kk}}  
            \end{array}\right)
        \end{equation*}
        Y $A$ es la suma directa de los cíclicos con órdenas las entradas de las columnas. 
        Si tomamos el producto por columnas obtenemos la cíclica primaria y si la hacemos por filas la que estamos interesados:
        \begin{align*}
            d_1 &= p_1^{n_{11}}p_2^{n_{21}} \ldots p_k^{n_{k1}} \\
                &\vdots \\
            d_t &= p_1^{n_{1t}}p_2^{n_{2t}} \ldots p_k^{n_{kt}} 
        \end{align*}
        Efectivamente, tendremos que:
        \begin{equation*}
            d_1d_2 \ldots d_t = p_1^{r_1} p_2^{r_2} \ldots p_k^{r_k} = |A|
        \end{equation*}
        Como $n_{ij}\geq n_{ij+1}$, tendremos entonces que $d_i \mid d_j$, para todo $j\leq i$. Además, tendremos que:
        \begin{align*}
            C_{d_1} &\cong C_{p_1^{n_{11}}} \oplus C_{p_2^{n_{21}}} \oplus \ldots \oplus C_{p_k^{n_{k1}}} \\
                    &\vdots \\
            C_{d_t} &\cong C_{p_1^{n_{1t}}} \oplus C_{p_2^{n_{2t}}} \oplus \ldots \oplus C_{p_k^{n_{kt}}} 
        \end{align*}
        De donde $A \cong C_{d_1}\oplus C_{d_2}\oplus \ldots \oplus C_{d_t}$. La unicidad viene de la unicidad dada por la descomposición del Teorema~\ref{teo:2_tema6}.
    \end{proof} 
    Los $d_i$ reciben el nombre de \underline{factores invariantes}.
\end{teo}

\begin{ejemplo}
    Sea $A$ un grupo abeliano finito con $|A| = 360 = 2^3\cdot 3^2\cdot 5$:
    \begin{itemize}
        \item Para la partición $\{2^3, 3^2, 5\}$, tenemos que:
            \begin{equation*}
                A\cong C_8\oplus C_9\oplus C_5 
            \end{equation*}
            Los factores invariantes serán:
            \begin{align*}
                d_1 = 2^3\cdot 3^2\cdot 5
            \end{align*}
            Por lo que la descomposición cíclica será $A\cong C_{360}$.
        \item Para la partición $\{2^2, 2, 3^2, 5\}$, la descomposición cíclica primaria fue:
            \begin{equation*}
                A \cong C_4 \oplus C_2 \oplus C_9 \oplus C_5
            \end{equation*}
            En este caso, tendremos $t = \max\{2, 1, 1\} = 2$, por lo que tendremos dos factores invariantes:
            \begin{equation*}
                \left(\begin{array}{ccc}
                    2^2 & 3^2 & 5 \\
                     2 & 1 & 1
                \end{array}\right)
            \end{equation*}
            Por lo que tendremos (los productos de las filas):
            \begin{align*}
                d_1 &= 2^2 \cdot 3^2 \cdot 5 = 180 \\
                d_2 &= 2\cdot 1\cdot 1 = 2
            \end{align*}
            Y la descomposición cíclica es:
            \begin{equation*}
                A\cong C_{180}\oplus C_2
            \end{equation*}
        \item Para la descomposición $\{2,2,2,3^2,5\}$, tenemos:
            \begin{equation*}
                A\cong C_2 \oplus C_2 \oplus C_2 \oplus C_9 \oplus C_5
            \end{equation*}
            Y tendremos $t=3$:
            \begin{equation*}
                \left(\begin{array}{cccc}
                        d_1 =& 2 & 3^2 & 5 \\
                        d_2 =&2 & 1 & 1 \\
                        d_3 =&2 & 1 & 1
                \end{array}\right)
            \end{equation*}
            Por lo que:
            \begin{equation*}
                A\cong C_{90} \oplus C_2 \oplus C_2
            \end{equation*}
        \item Para $\{2^3, 3,3,5\}$:
            \begin{equation*}
                A\cong C_8\oplus C_3\oplus C_3\oplus C_5
            \end{equation*}
            Y tenemos:
            \begin{equation*}
                \left(\begin{array}{ccc}
                    2^3 & 3 & 5 \\
                    1 & 3 & 1
                \end{array}\right)
            \end{equation*}
            Y la descomposición cíclica será:
            \begin{equation*}
                A\cong C_{120} \oplus C_3
            \end{equation*}
        \item Para $\{2^2,2,3,3,5\}$:
            \begin{equation*}
                A\cong C_4\oplus C_2\oplus C_3 \oplus C_3 \oplus C_5
            \end{equation*}
            Y tenemos:
            \begin{equation*}
                \left(\begin{array}{ccc}
                    2^2 & 3 & 5 \\
                    2 & 3 & 1
                \end{array}\right)
            \end{equation*}
            Por lo que tenemos la descomposición cíclica:
            \begin{equation*}
                A\cong C_{60} \oplus C_6
            \end{equation*}
        \item Para $\{2, 2, 2, 3, 3, 5\}$:
            \begin{equation*}
                A\cong C_2\oplus C_2\oplus C_2\oplus C_3\oplus C_3\oplus C_5
            \end{equation*}
            Tenemos:
            \begin{equation*}
                \left(\begin{array}{ccc}
                    2 & 3 & 5 \\
                    2 & 3 & 1 \\
                    2 & 1 & 1
                \end{array}\right)
            \end{equation*}
            Y:
            \begin{equation*}
                A\cong C_{30} \oplus C_6 \oplus C_2
            \end{equation*}
    \end{itemize}
\end{ejemplo}

En el caso particular de que todos los primos tengan exponente 1:

\begin{coro}
    Si $A$ es un grupo abeliano finito con $|A| = p_1p_2 \ldots p_k = n$, entonces salvo isomorfismo, el único grupo abeliano de orden $n$ es el cíclico $C_n$.
    \begin{proof}
        Utilizando el Teorema~\ref{teo:2_tema6}, podemos escribir:
        \begin{equation*}
            A \cong C_{p_1} \oplus C_{p_2} \oplus \ldots \oplus C_{p_k}
        \end{equation*}
        Y como $\mcd(p_i, p_j) = 1$ para cada $i,j\in \{1,\ldots,k\}$ con $i\neq j$, tenemos que:
        \begin{equation*}
            C_{p_1} \oplus C_{p_2} \oplus \ldots \oplus C_{p_k} = C_{p_1p_2\ldots p_k} = C_n
        \end{equation*}
    \end{proof}
\end{coro}

\begin{ejemplo}
    Sea $A$ un grupo abeliano finito con $|A| = 580 = 2^2\cdot 3^2\cdot 5$, buscamos clasificarlo según la descomposición cíclica:
    \begin{equation*}
        \begin{array}{c|c|c|c}
            \text{Descomposición} & \text{desc. cíclica primaria} & \text{factores invariantes} & \text{desc. cíclica} \\
            \hline
            \{2^2, 3^2, 5\} & C_4\oplus C_9 \oplus C_5 & 2^2\cdot 3^2\cdot 5 = 180 & C_{180} \\
            \hline
            \{2,2,3^2,5\} & C_2\oplus C_2\oplus C_9\oplus C_5 & \begin{array}{c}
                    d_1 = 2\cdot 9\cdot 5 = 90 \\
                    d_2 = 2
            \end{array}& C_{90}\oplus C_2 \\
            \hline
            \{2^3, 3, 3, 5\} & C_4\oplus C_3\oplus C_3\oplus C_5 & \begin{array}{c}
                    d_1 = 2^2\cdot 3\cdot 5 = 60 \\
                    d_2 = 3
            \end{array}& C_{60} \oplus C_3 \\
            \hline
                    \{2,2,3,3,5\} & C_2\oplus C_2\oplus C_3\oplus C_3\oplus C_5 & \begin{array}{c}
                            d_1 = 2\cdot 3\cdot 5 = 30 \\
                            d_2 = 2\cdot 3 = 6
                        \end{array} & C_{30} \oplus C_6
        \end{array}
    \end{equation*}
\end{ejemplo}

\begin{ejemplo} % // TODO: EJercicio 1
    Listar los órdenes de todos los elementos de un grupo de orden 8.
    Sea $A$ un grupo abeliano finito de orden $8$, entonces lo podemos clasificar en:
    \begin{itemize}
        \item $C_8$:
            \begin{itemize}
                \item Los elementos $\{1,3,5,7\}$ tienen orden 8.
                \item $O(0) = 1$.
                \item $O(2) = \nicefrac{8}{\mcd(2,8)} = 4 = O(6)$.
                \item $O(4) = 2$.
            \end{itemize}
        \item $C_4\oplus C_2$, aplicamos que $O(a,b) = \mcm(O(a), O(b))$:
            Como los órdenes de los elementos en $C_4$ son: $\{1,2,4\}$ y en $C_2$ son $\{1,2\}$, las posibilidades son: $\{1,2,4\}$:
            \begin{itemize}
                \item $O(0,0) = 1$.
                \item $O(0,1) = 2$.
                \item $O(1,b) = 4 = O(3,b)$, $\forall b\in C_2$
                \item $O(2,b) = 2 $, $\forall b\in C_2$.
            \end{itemize}
        \item $C_2\oplus C_2\oplus C_2$, los órdenes son $\{1,2\}$ y todos tienen orden 2 salvo el elemento $(0,0,0)$, que tiene orden 1.
    \end{itemize}
\end{ejemplo}

\begin{ejemplo}
    Listar los ódenes de todos los elementos de un grupo abeliano $A$ de orden 12.\\

    \noindent
    Sea $A$ con $|A| = 12 = 2^2\cdot 3$, tenemos $A\cong \mathbb{Z}_{12}$ o $A\cong \mathbb{Z}_6\oplus \mathbb{Z}_2$.
    \begin{itemize}
        \item En $\mathbb{Z}_{12}$:
            \begin{itemize}
                \item $U(\mathbb{Z}_{12}) = \{1,5,7,11\}$.
                \item $O(2) = 6$.
                \item $O(3) = 4 = O(9)$.
                \item $O(4) = 3 = O(8)$.
                \item $O(6) = 2$.
            \end{itemize}
        \item En $\mathbb{Z}_6\oplus\mathbb{Z}_2$:
            \begin{equation*}
                O(a,b) = \mcm(Div(6), Div(2)) = \mcm(\{1,2,3,6\}, \{1,2\}) = \{1,2,3,6\}
            \end{equation*}
            El orden de los elementos de $\mathbb{Z}_6$ son:
            \begin{itemize}
                \item $U(\mathbb{Z}_6) = \{1,5\}$, luego $O(1) = O(5) = 6$.
                \item $O(2) = 3 = O(4)$.
                \item $O(3) = 2$.
                \item $O(0) = 1$.
            \end{itemize}
            Ahora:
            \begin{itemize}
                \item $O(0,0) = 1$.
                \item $O(1,b) = O(5,b) = 6$ $\forall b\in \mathbb{Z}_2$.
                \item $O(3, b) = 2$ $\forall b\in \mathbb{Z}_2$.
                \item $O(2, 0) = O(4, 0) = 3$.
                \item $O(2, 1) = O(4, 1) = 6$.
            \end{itemize}
    \end{itemize}
\end{ejemplo}

\section{Clasificación de grupos abelianos no finitos}
\noindent
Buscamos hayar la descomposición cíclica y la descomposición cíclica primaria de dos grupos cualesquiera. Para ello, recordamos varias definiciones que ya vimos.

\begin{notacion}
    Como trabajaremos con grupos abelianos finitos, usaremos la notación aditiva.
\end{notacion}

\begin{definicion}
    Un grupo abeliano $A$ se dice que es finitamente generado si existe un conjunto:
    \begin{equation*}
        X = \{x_1,\ldots,x_r\} \subseteq A
    \end{equation*}
    De forma que para todo $a\in A$, existirán $\lm_1, \ldots, \lm_r \in \mathbb{Z}$ de forma que:
    \begin{equation*}
        a = \sum_{k=1}^{r} \lm_k x_k
    \end{equation*}
    En dicho caso, diremos que $X$ es un \underline{sistema de generadores de A}, y notaremos:
    \begin{equation*}
        A = \langle x_1, \ldots, x_r \rangle 
    \end{equation*}
\end{definicion}

\begin{definicion}[Base]
Sea $A$ un grupo abeliano, un conjunto de generadores $X = \{x_1,\ldots,x_r\}$ de $A$ es una \underline{base} si son $\mathbb{Z}-$linealmente independientes.\\ % // TODO: Meter def

\noindent
En dicho caso $A$ es un \underline{grupo abeliano libre de rango $r$}.
\end{definicion}

% Un grupo finito no puede tener bases por la independencia !!!

\begin{observacion}
    Observemos que si $A$ es un grupo abeliano libre de rango $r$, entonces tendremos que:
    \begin{equation*}
        A \cong \mathbb{Z}^r
    \end{equation*}
    Además, si $H < A$, tendremos entonces que $H\cong \mathbb{Z}^s$, para cierta $s\leq r$.
\end{observacion}

\noindent
De esta forma, si $A$ es un grupo finitamente generado, podemos descomponerlo en:
\begin{equation*}
    A\cong F \oplus T(A)
\end{equation*}
Que será la \underline{descomposición estándar} de $A$. $F$ será un grupo abeliano libre de rango finito y:
\begin{equation*}
    T(A) = \{a\in A \mid O(a) < +\infty\}
\end{equation*}
Que recibe el nombre de \underline{subgrupo de torsión de $A$}. % // TODO: Meter defs y props

\begin{prop}
    El subgrupo de torsión de un grupo es un grupo abeliano finito.
\end{prop}
De esta forma, existirán $r\geq 0$ y $d_1,\ldots,d_s$ con $d_i\mid d_j$ con $j\leq i$ de forma que:
\begin{equation*}
    d_1d_2\ldots d_s = |T(A)|
\end{equation*}
Por lo que:
\begin{equation*}
    A \cong \mathbb{Z}^r \oplus \mathbb{Z}_{d_1} \oplus \mathbb{Z}_{d_2} \oplus \ldots \oplus \mathbb{Z}_{d_s}
\end{equation*}

\begin{itemize}
    \item Llamaremos $r$ al rango de $A$.
    \item A los $d_i$ los llamaremos factores invariantes de $A$.
\end{itemize}

\begin{ejemplo}
    Si tomamos:
    \begin{equation*}
        A = \langle x,y,z \mid x^3=y^4, x^2z = z^{-1}y, xy=yx, xz=zx, yz=zy \rangle 
    \end{equation*}
    Si lo escribimos en notación aditiva:
    \begin{equation*}
        A = \langle x,y,z \mid 3x=4y, 2x+z = y-z, x+y=y+x, x+z = z+x, y+z=z+y\rangle 
    \end{equation*}
    Si nos olvidamos de las últimas y pensamos que el grupo es abeliano, así como despejando:
    \begin{equation*}
        A = \langle x,y,z \mid 3x-4y = 0, 2x-y+2z = 0 \rangle 
    \end{equation*}
    Y tenemos el sistema:
    \begin{equation*}
        M = \left(\begin{array}{ccc}
            3 & -4 & 0 \\
            2 & -1 & 2
        \end{array}\right)
    \end{equation*}
    Tenemos 3 incógnitas y $rg(M) = 2$, un Sistema Compatible Indeterminado, con un parámetro libre. Veremos que transformaremos $M$ en:
    \begin{equation*}
        \left(\begin{array}{ccc}
            3 & -4 & 0 \\
            2 & -1 & 2
        \end{array}\right) \sim \left(\begin{array}{ccc}
            1 & 0 & 0 \\
            0 & 2 & 0
        \end{array}\right)
    \end{equation*}
    Que es la \underline{forma normal de Smith} (parecido a Hermite pero en $\mathbb{Z}$). De esta forma, tendremos que:
    \begin{equation*}
        A\cong \mathbb{Z} \oplus \mathbb{Z} \oplus \mathbb{Z}_2 \cong \mathbb{Z} \oplus \mathbb{Z}_2
    \end{equation*} % // TODO: Se explicará la absorción del Z
\end{ejemplo}

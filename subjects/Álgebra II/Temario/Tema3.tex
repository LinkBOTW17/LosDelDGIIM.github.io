\chapter{Grupos cocientes y Teoremas de isomorfía}
Este tema se centrará en las relaciones de equivalencia $\prescript{}{H}{\sim}$ y $\sim_H$ definidas en el capítulo anterior, donde ya vimos propiedades de estas relaciones (recordamos la Proposición~\ref{prop:biyecciones_conj_cocientes}), como que $G/\prescript{}{H}{\sim}$ y $G/\sim_H$ eran biyectivos o el Teorema de Lagrange. Estaremos especialmente interesados en el caso de que los conjuntos cocientes de estas dos relaciones de equivalencia coincidan, propiedad que nos dará los Teoremas de Isomorfía, que son el principal objeto de estudio de este tema.

\begin{definicion}[Subgrupos normales]
    Sea $G$ un grupo y $H<G$, diremos que $H$ es un subgrupo normal de $G$, denotado por $H \lhd G$, si las clases laterales de cada elemento coinciden, es decir, si:
    \begin{equation*}
        xH = Hx \qquad \forall x\in G
    \end{equation*}
    En cuyo caso, tendremos que $G/\prescript{}{H}{\sim\ } = G/\sim_H$, y notaremos a este conjunto como $G/H$, al llamaremos \underline{conjunto de las clases laterales de $H$ en $G$}. 
\end{definicion}

\begin{definicion}[Conjugado]
    Sea $G$ un grupo, $H\subseteq G$ y $x\in G$, definimos el conjugado de $H$ por $x$ como el conjunto:
    \begin{equation*}
        xHx^{-1} = \{xhx^{-1} \mid h\in H\}
    \end{equation*}
\end{definicion}

\begin{prop}
    Sea $G$ un grupo, $H<G$ y $x\in G$, entonces $xHx^{-1}< G$.
    \begin{proof}
        Para ello, sean $xh_1x^{-1}, xh_2x^{-1}\in xHx^{-1}$, entonces:
        \begin{equation*}
            xh_1x^{-1}{(xh_2x^{-1})}^{-1} = xh_1x^{-1}xh_2^{-1}x^{-1} = xh_1h_2^{-1}x^{-1} \in xHx^{-1}
        \end{equation*}
        Ya que como $H$ es un subgrupo de $G$, entonces $h_1h_2^{-1}\in H$.
    \end{proof}
\end{prop}~\\

\noindent
Buscamos ahora formas cómodas de detectar cuándo un subgrupo de un grupo es normal o no, ya que es tedioso comprobar la igualdad $xH=Hx$ para todo elemento $x$ del grupo que estemos considerando en cada caso.
\begin{prop}[Caracterización de subgrupos normales]\ \\
    Sea $G$ un grupo y ${H<G}$, son equivalentes:
    \begin{enumerate}
        \item[$i)$] $H\lhd G$.
        \item[$ii)$] $xhx^{-1}\in H$ $\forall x\in G, \forall h\in H$.
        \item[$iii)$] $xHx^{-1}\subseteq H$ $\forall x\in G$.
        \item[$iv)$] $xHx^{-1} = H$ $\forall x\in G$.
    \end{enumerate}
    \begin{proof}
        Veamos todas las implicaciones:
        \begin{description}
            \item [$i)\Longrightarrow ii)$] Por ser $H\lhd G$, tenemos que $xH = Hx$ para todo $x\in G$, con lo que ${xh\in xH = Hx}$, por lo que $\exists h'\in H$ de forma que $xh = h'x$. Si multiplicamos por $x^{-1}$ a la derecha:
                \begin{equation*}
                    xhx^{-1} = h' \in H
                \end{equation*}
            \item [$ii)\Longleftrightarrow iii)$] Es claro.
            \item [$iii)\Longrightarrow iv)$] Falta ver que $H\subseteq xHx^{-1}$. Para ello, si cogemos $x\in G$, en particular tendremos que $x^{-1}\in G$, con lo que (por hipótesis):
                \begin{equation*}
                    x^{-1}H{(x^{-1})}^{-1} = x^{-1}Hx \subseteq H
                \end{equation*}
                Y si multiplicamos estas relaciones por $x$ a la izquierda y por $x^{-1}$ a la derecha, llegamos a que:
                \begin{equation*}
                    H = x(x^{-1}Hx)x^{-1} \subseteq xHx^{-1}
                \end{equation*}
            \item [$iv)\Longrightarrow i)$] Fijado $x\in G$, veamos que $xH=Hx$:
                \begin{description}
                    \item [$\subseteq)$] Si $xh\in xH$, entonces tendremos que:
                        \begin{equation*}
                            xhx^{-1} \in xHx^{-1}= H
                        \end{equation*}
                        Con lo que existirá $h'\in H$ de forma que $xhx^{-1}=h'$. Si multilicamos por $x$ a la derecha, obtenemos que:
                        \begin{equation*}
                            xh = h'x \in Hx
                        \end{equation*}
                    \item [$\supseteq)$] Para la otra inclusión, si $hx\in Hx$, tendremos que:
                        \begin{equation*}
                            x^{-1}hx \in x^{-1}Hx = H
                        \end{equation*}
                        Por lo que existirá $h'\in H$ de forma que $x^{-1}hx = h'$. Si multiplicamos por $x$ a la izquierda:
                        \begin{equation*}
                            hx = xh' \in xH
                        \end{equation*}
                \end{description}
        \end{description}
    \end{proof}
\end{prop}

Comprobar que $xhx^{-1}\in H$ para todo $x\in G$ y para todo $h\in H$ puede ser una labor tediosa, por lo que presentamos la siguiente Proposición, que puede resultar de utilidad a la hora de comprobar si un subgrupo $H$ de un grupo $G$ es normal o no.
% \begin{prop}
%     Sea $G$ un grupo, $H<G$ y $S\subseteq G$ de forma que $G=\langle S \rangle $, entonces:
%     \begin{equation*}
%         xhx^{-1}\in H \quad \forall x\in G, \forall h\in H \Longleftrightarrow shs^{-1}\in H \quad \forall s\in S, \forall h\in H
%     \end{equation*}
%     \begin{proof}
%         Veamos las dos implicaciones:
%         \begin{description}
%             \item [$\Longrightarrow)$] En particular, tenemos que $s\in S\subseteq G$.
%             \item [$\Longleftarrow)$] Sea $x\in G=\langle S \rangle $, entonces existirán $s_1,\ldots,s_n\in S$ y $\gamma_1,\ldots,\gamma_n \in \{\pm 1\}$ de forma que:
%                 \begin{equation*}
%                     x = s_1^{\gamma_1}\ldots s_n^{\gamma_n}
%                 \end{equation*}
%                 Por inducción sobre $n$:
%                 \begin{itemize}
%                     \item \underline{Si $n=1$:} Entonces $x=s^{\gamma}$ con $s\in S$ y $\gamma\in \{\pm 1\}$. Distinguimos casos:
%                         \begin{itemize}
%                             \item Si $\gamma=1$, entonces:
%                                 \begin{equation*}
%                                     xhx^{-1} = shs^{-1} \in H \qquad \forall h\in H
%                                 \end{equation*}
%                             \item Si $\gamma=-1$, por el punto superior tenemos que:
%                                 \begin{equation*} % // TODO: Terminar
%                                     shs^{-1}\in H \qquad \forall h\in H
%                                 \end{equation*}
%                         \end{itemize}
%                     \item 
%                 \end{itemize}
%         \end{description}
%     \end{proof}
% \end{prop}

\begin{ejemplo}
    Hemos caracterizado ya a los grupos normales, pero veamos ejemplos de ellos:
    \begin{enumerate}
        \item Dado un grupo $G$, los dos subgrupos impropios de $G$ siempre son subgrupos normales del mismo:
            \begin{itemize}
                \item Para el caso $H=\{e\}$:
                    \begin{equation*}
                        xex^{-1} = xx^{-1} = e \in \{e\} \qquad \forall x\in G
                    \end{equation*}
                    Y por la Proposición anterior, tenemos que $\{e\}\lhd G$.
                \item Para el caso $H=G$:
                    \begin{equation*}
                        xhx^{-1} \in G \qquad \forall x\in G, \forall h\in G
                    \end{equation*}
                    Y por la misma razón, también tenemos que $G\lhd G$.
            \end{itemize}
        \item En un grupo abeliano $G$, todos sus subgrupos son normales (sea $H<G$):
            \begin{equation*}
                xH = \{xh \mid h \in H\} = \{hx \mid h \in H\} = Hx \qquad \forall x\in G
            \end{equation*}
        \item Todo subgrupo de índice 2 es normal, es decir, si $H<G$ con $[G:H] = 2$, entonces $H\lhd G$.

            Para verlo, si tomamos $x\in G\setminus H$, como $[G:H] = 2$, tenemos que:
                \begin{equation*}
                    H\cup xH = G = H\cup Hx
                \end{equation*}
                En ambos casos, como son particiones disjuntas, tenemos que $xH = Hx$ para todo $x\in G\setminus H$ (y si $x\in H$, entonces $xH = H = Hx$), con lo que $H\lhd G$.
        \item En $S_3$, si consideramos $H = \langle (1\ 2) \rangle $, $H$ no es un subgrupo normal de $S_3$, como se vio en el correspondiente ejemplo del tema anterior, y podemos volverlo a comprobar con la caracterización, ya que:
            \begin{equation*}
                (2\ 3)(1\ 2){(2\ 3)}^{-1} = (1\ 3)\notin H
            \end{equation*}
            Igual les pasa a los subgrupos $\langle (2\ 3) \rangle $ y $\langle (1\ 3) \rangle $. Sea ahora $K = \{1, (1\ 2\ 3), (1\ 3\ 2)\}$, como $[S_3:K]=2$, tenemos que $K\lhd S_3$:
            \begin{equation*}
                S_3 / K = \{H, H(1\ 2)\} = \{K, (1\ 2)H\}
            \end{equation*}
        \item La relación de ``ser un subgrupo normal de'' no es transitiva, es decir, si $G$ es un grupo con $K<H<G$, $K\lhd H$ y $H\lhd G$, entonces no necesariamente se tiene que $K\lhd G$. La situación es la descrita en la Figura~\ref{fig:situacion}
            \begin{figure}[H]
                \centering
                \begin{tikzpicture}
                    \node (G) {$G$};
                    \node[below right=of G] (H) {$H$};
                    \node[below left=of H] (K) {$K$};

                    \draw (G) -- (H);
                    \draw (H) -- (K);
                    \draw (K) -- (G);
                \end{tikzpicture}
                \caption{Situación descrita.}
                \label{fig:situacion}
            \end{figure}

            Por ejemplo, en $A_4$ consideramos el grupo de Klein $V$ y $\langle (1\ 2)(3\ 4) \rangle $. Vamos a ver que $\langle (1\ 2)(3\ 4) \rangle \lhd V$ y que $V \lhd A_4 $ pero no se cumple que $\langle (1\ 2)(3\ 4) \rangle \lhd A_4 $:
            \begin{figure}[H]
                \centering
                \begin{tikzpicture}
                    \node (G) {$A_4$};
                    \node[below right=of G] (H) {$V$};
                    \node[below=of G, yshift=-2cm] (K) {$\langle (1\ 2)(3\ 4) \rangle $};

                    \draw (G) -- (H);
                    \draw (H) -- (K);
                    \draw (K) -- (G);
                \end{tikzpicture}
            \end{figure}
            \begin{itemize}
                \item En primer lugar, $\langle (1\ 2)(3\ 4) \rangle \lhd V $, por ser $[V:\langle (1\ 2)(3\ 4) \rangle ] = 2$.
                \item Veamos ahora que $V\lhd A_4$. Para ello, consideramos:
                    \begin{equation*}
                        A_4 = \langle (1\ 2\ 3), (1\ 2\ 4) \rangle 
                    \end{equation*}
                    Ya que $(1\ 3\ 4) = (1\ 2\ 4)(1\ 2\ 3)$. Basta comprobar la caracterización para todos los generadores de $A_4$:
                    \begin{align*}
                        &(1\ 2\ 3)(1\ 2)(3\ 4){(1\ 2\ 3)}^{-1} \in  V \\
                        &(1\ 2\ 3)(1\ 3)(2\ 4){(1\ 2\ 3)}^{-1} \in  V \\
                        &(1\ 2\ 3)(1\ 4)(2\ 3){(1\ 2\ 3)}^{-1} \in  V \\
                        &(1\ 2\ 4)(1\ 2)(3\ 4){(1\ 2\ 4)}^{-1} \in  V \\
                        &(1\ 2\ 4)(1\ 3)(2\ 4){(1\ 2\ 4)}^{-1} \in  V \\
                        &(1\ 2\ 4)(1\ 4)(2\ 3){(1\ 2\ 4)}^{-1} \in  V 
                    \end{align*}
                \item Veremos ahora que no se tiene que $\langle (1\ 2)(3\ 4) \rangle\lhd A_4 $, ya que:
                    \begin{equation*}
                        (1\ 2\ 3)(1\ 2)(3\ 4){(1\ 2\ 3)}^{-1} = (1\ 4)(2\ 3)\notin H
                    \end{equation*}
            \end{itemize}
    \end{enumerate}
\end{ejemplo}

\begin{definicion}[Centro]
    Sea $G$ un grupo, definimos el \underline{centro de $G$} como el conjunto de los elementos de $G$ que conmutan con todos los demás, es decir, el conjunto:
    \begin{equation*}
        Z(G) = \{a\in G \mid ax = xa, \forall x\in G\}
    \end{equation*}
Podemos entener $Z(G)$ como ``la parte abeliana del grupo'' $G$.
\end{definicion}

\begin{prop}
    Sea $G$ un grupo, se verifica:
    \begin{enumerate}
        \item[$i)$] $Z(G)<G$.
        \item[$ii)$] $Z(G)\lhd G$.
        \item[$iii)$] Si $G$ es abeliano, entonces $L(G) = G$.
    \end{enumerate}
    \begin{proof}
        Demostramos las propiedades:
        \begin{enumerate}
            \item[$i)$] Sean $a,b\in Z(G)$ y dado $x\in G$, entonces:
                \begin{equation*}
                    (ab^{-1})x = a(b^{-1}x) = a{(x^{-1}b)}^{-1} = a{(bx^{-1})}^{-1} = a(xb^{-1}) = (ax)b = (xa)b = x(ab^{-1})
                \end{equation*}
                Por lo que $ab^{-1}\in Z(G)$, lo que nos dice que $Z(G)$ es un subgrupo de $G$.
            \item[$ii)$] Sea $x\in G$, entonces:
                \begin{equation*}
                    xZ(G) = \{xz \mid z\in Z(G)\} = \{zx \mid z\in Z(G)\} = Z(G)x
                \end{equation*}
            \item[$iii)$] Es evidente.
        \end{enumerate}
    \end{proof}
\end{prop}

\begin{ejemplo} % // TODO: Es el ejercicio 6 ARTURITO
    Ejemplos interesantes:
    \begin{itemize}
        \item Veamos que $Z(S_n) = 1$ cuando $n\geq 3$. Para ello, supongamos que $n\geq 3$ y consideremos $1\neq \sigma\in S_n$, con lo que existirán $i,j\in \{1,\ldots,n\}$ con $i\neq j$ de forma que $\sigma(i) = j$.

            En dicho caso, $\exists k\in \{1,\ldots,n\}$ con $i\neq k \neq j$. Si consideramos $\tau = (j\ k)$:
            \begin{equation*}
                \left.\begin{array}{r}
                    \sigma\tau(i) = \sigma(i) = j \\
                    \tau\sigma(i) = \tau(j) = k
                \end{array}\right\} \Longrightarrow \sigma\tau \neq \tau \sigma
            \end{equation*}
            Por tanto, $\sigma\notin Z(S_n)$, para todo $\sigma\in S_n\setminus\{1\}$.
        \item Veamos que  que $Z(A_n) = 1$ cuando $n\geq 4$. Para $n\geq 4$, $\exists i,j\in \{1,\ldots,n\}$ con $i\neq j$ de forma que $\sigma(i) = j$, con lo que podemos encontrar $k,l\in \{1,\ldots,n\}$, distintos entre sí y distintos de $i$ y $j$. Consideramos:
            \begin{equation*}
                \tau = (j\ k\ l) \in A_4
            \end{equation*}
            Y tenemos de la misma forma que:
            \begin{equation*}
                \left.\begin{array}{:}
                        \sigma\tau(i) = k \\
                        \tau\sigma(i) = j
                \end{array}\right\} \Longrightarrow Z(A_n) = 1
            \end{equation*}
    \end{itemize}
\end{ejemplo}

\begin{prop}
    Sea $G$ un grupo, $H<G$, entonces, equivalen:
    \begin{enumerate}
        \item[$i)$] $H\lhd G$.
        \item[$ii)$] $\forall x,y\in G \mid xy \in H$, entonces $yx \in H$
    \end{enumerate}
    \begin{proof}
        Veamos las dos implicaciones:
        \begin{description}
            \item [$i)\Longrightarrow ii)$] Sean $x,y\in G$ con $xy \in H$, entonces:
                \begin{equation*}
                    x\sim_H y^{-1} \Longrightarrow Hx = Hy^{-1}
                \end{equation*}
                Por lo que podemos encontrar $h,h'\in H$ de forma que $hx = h'y^{-1}$. Al ser $H\lhd G$, tenemos que:
                \begin{equation*}
                    \left.\begin{array}{rcl}
                            xH &=& Hx \\
                            y^{-1}H &=& Hy^{-1}
                        \end{array}\right\} \Longrightarrow \left\{\begin{array}{rcll}
                            hx &=& xh'' & h'' \in H \\
                            h'y^{-1} &=& y^{-1} h ''' &h''' \in H
                    \end{array}\right.
                \end{equation*}
                En conclusión:
                \begin{equation*}
                    xh'' = y^{-1}h'''\Longrightarrow yx = h''' {(h'')}^{-1} \in H
                \end{equation*}
            \item [$ii)\Longrightarrow i)$] Sean $x\in G$ y $h\in H$, tenemos que:
                \begin{equation*}
                    h = x^{-1}(xh) \in H
                \end{equation*}
                De donde deducimos por hipótesis que $xhx^{-1} \in H$, lo que nos dice que $H\lhd G$.
        \end{description}
    \end{proof}
\end{prop}

\begin{teo}
    Sea $G$ un grupo y $H\lhd G$, entonces en el conjunto $G/H$ podemos definir una operación binaria $G/H\times G/H\longrightarrow G/H$ que dota a $G/H$ de estructura de grupo, de forma que la proyección canónica $p:G\to G/H$ sea un homomorfismo de grupos. De esta forma, llamaremos a $G/H$ \underline{grupo cociente}.
    \begin{proof}
        Definimos la operación binaria $\cdot: G/H\times G/H\longrightarrow G/H$ dada por:
        \begin{equation*}
            xH\cdot yH = xyH \qquad \forall xH,yH\in G/H
        \end{equation*}
        A esta operación la denotaremos a partir de ahora por yuxtaposición.
        \begin{itemize}
            \item En primer lugar, comprobemos que está bien definida, es decir, si $xH = x'H$ y $yH=y'H$, entonces $xyH = x'y'H$. Para ello:
                \begin{equation*}
                    \left.\begin{array}{l}
                        xH = x'H \\
                        yH = y'H
                    \end{array}\right\} \Longrightarrow \left\{\begin{array}{l}
                        x'= xh_1 \\
                        y' = yh_2 \\
                        h_1,h_2\in H
                    \end{array}\right.
                \end{equation*}
                Vemos ahora que dado $h\in H$:
                \begin{equation*}
                    x'y'h = xh_1yh_2h \stackrel{H\lhd\ G}{=} xyh_1'h_2h \in xyH
                \end{equation*}
                Para cierto $h_1'\in H$, por lo que tenemos $\subseteq$.
            \item Que la operación está definida es clara, ya que la operación de $G$ es asociativa.
            \item En neutro es $1H = H$.
            \item Fijado un elemento $xH \in G/H$, tendremos que ${(xH)}^{-1} = x^{-1}H$.
        \end{itemize}
        Concluimos que $G/H$ es un grupo.\\

        \noindent
        Ahora, consideramos $p:G\rightarrow G/H$ que viene definida por $p(x) = xH$ para todo $x\in G$, gracias a la definición de la operación de $G/H$, tenemos que:
        \begin{equation*}
            p(xy) = xyH = xHyH = p(x)p(y)
        \end{equation*}
    \end{proof}
\end{teo}

Como propiedades a destacar del ahora grupo $G/H$:

\begin{itemize}
    \item Sabemos por el capítulo anterior que el orden del grupo $G/H$ es (si $G$ es finito):
        \begin{equation*}
            |G/H| = [G:H] = \dfrac{|G|}{|H|}
        \end{equation*}
    \item Además, tenemos que: 
        \begin{equation*}
            \ker(p) = \{x\in G\mid p(x) = H\} = \{x\in G\mid xH = H\} = \{x\in H\} = H
        \end{equation*}
\end{itemize}

\begin{ejemplo}
    Algunas consecuencias de que $G/H$ sea un grupo:
    \begin{enumerate}
        \item[$i)$] En $S_3$, si consideramos $H = \{1, (1\ 2\ 3), (1\ 3\ 2)\}$, tenemos que:
            \begin{equation*}
                S_3/H = \{H, (1\ 2)H\} \cong \mathbb{Z}_2
            \end{equation*}
        \item[$ii)$] Si consideramos $H<\mathbb{Z}$, entonces $H\lhd \mathbb{Z}$, ya que $\mathbb{Z}$ es abeliano. Además, sabemos que $\exists n\in \mathbb{Z}$ de forma que $H = n\mathbb{Z}$. De esta forma, tendremos que:
            \begin{equation*}
                \mathbb{Z}/n\mathbb{Z} \cong \mathbb{Z}_n
            \end{equation*}
        \item[$iii)$] Veamos otra vez que $A_4$ no tiene subgrupos de orden 6. Si $H<A_4$ con $|H| = 6$, entonces:
            \begin{equation*}
                [A_4:H] = \dfrac{A_4}{H} = 2
            \end{equation*}
            Por tanto, $H\lhd A_4$. De esta forma, $A_4/H\cong \mathbb{Z}_2$, por ser el único grupo de orden 2. Si el cociente es $\mathbb{Z}_2$ y consideramos $xH\in A_4/H$, entonces:
            \begin{equation*}
                {(xH)}^{2} = x^2H = H \qquad \forall x\in A_4/H
            \end{equation*}
            Por tanto, los cuadrados de los 8 $3-$ciclos de $A_4$ serían los $3-$ciclos de $H$, de donde $|H| \geq 8$, \underline{contradicción}.
    \end{enumerate}
\end{ejemplo}

\begin{prop}
    Sea $G$ un grupo, $H<G$ es normal si y solo si existe $f:G\to G'$ un homomorfismo de grupos de forma que $\ker(f) = H$.
    \begin{proof}
        Veamos las dos implicaciones:
        \begin{description}
            \item [$\Longrightarrow)$] Si $H\lhd G$, entonces la proyección canónica $p:G\to G/H$ es un homomorfismo de grupos de forma que $\ker(p) = H$, gracias a lo visto anteriormente.
            \item [$\Longleftarrow)$] Supongamos ahora que existe un homomorfismo $f:G\to G'$ de grupos de forma que $\ker(f) = H$. Sea $x\in G$ y $h\in H$, tenemos que:
                \begin{equation*}
                    f(xhx^{-1}) = f(x)f(h){(f(x))}^{-1} = f(x) {(f(x))}^{-1} = 1
                \end{equation*}
                De donde deducimos que $xhx^{-1}\in \ker(f) = H$.
        \end{description}
    \end{proof}
\end{prop}

\begin{teo}[Propiedad universal del grupo cociente]\label{teo:prop_universal}
    Sea $G$ un grupo, $H\lhd G$, $p:G\to G/H$ la proyección canónica al cociente, entonces para cualquier homomorfismo $f:G\to G'$ tal que $H\subseteq \ker(f)$, existe un único homomorfismo de grupos $\varphi:G/H\to G'$ de forma que $\varphi\circ p = f$.\\

    Más aún, tendremos que:
    \begin{align*}
        f \text{\ sobreyectiva} &\Longleftrightarrow \varphi \text{\ sobreyectiva}\\
        H = \ker(f) &\Longleftrightarrow \varphi \text{\ inyectiva}
    \end{align*}

    \noindent
    La situación es la descrita podemos observarla en la Figura~\ref{fig:teo_propiedad_universal}, nos dice que el diagrama conmuta.
    \begin{proof}
        Definimos $\varphi:G/H\to G'$ de la forma:
        \begin{equation*}
            \varphi(xH) = f(x) \qquad \forall xH \in G/H
        \end{equation*}
        Veamos que está bien definido. Para ello, sean $x,y\in G$ de forma que $xH = yH$, entonces $y^{-1}x\in H\subseteq \ker(f)$, de donde:
        \begin{equation*}
            f(y^{-1}x) = {(f(y))}^{-1}f(x) = 1 \Longrightarrow f(x) = f(y)
        \end{equation*}
        \begin{itemize}
            \item Veamos ahora que $\varphi$ es un homomorfismo:
                \begin{equation*}
                    \varphi(xHyH) = \varphi(xyH) = f(xy) = f(x) f(y) = \varphi(xH)\varphi(xy) \qquad \forall x,y\in G
                \end{equation*}
            \item Veamos que $\varphi\circ p = f$. Para ello, sea $x\in G$:
                \begin{equation*}
                    (\varphi \circ p)(x) = \varphi(p(x)) = \varphi(xH) = f(x)
                \end{equation*}
            \item Supongamos que existe otra función $\psi:G/H\to G'$ de forma que $\psi\circ p = f$. En cuyo caso, sea $x\in G$:
                \begin{equation*}
                    (\psi\circ p)(x) = \psi(p(x)) = \psi(xH) = f(x) = \varphi(xH)
                \end{equation*}
                Por lo que $\psi = \varphi$.
        \end{itemize}
        Veamos ahora las dos implicaciones:
        \begin{equation*}
            f \text{\ sobreyectivo} \Longleftrightarrow \varphi \text{\ sobreyectivo}
        \end{equation*}
        \begin{description}
            \item [$\Longleftarrow)$] Como $f = \varphi\circ p$ y la composición de aplicaciones sobreyectivas es sobreyectiva, concluimos que $f$ será sobreyectiva.
            \item [$\Longrightarrow)$] Supongamos que $f$ es sobreyectiva y sea $y\in G'$, por lo que $\exists x\in G$ de forma que $f(x) = y$, pero:
                \begin{equation*}
                    y = f(x) = \varphi(p(x)) = \varphi(xH)
                \end{equation*}
                Concluimos que $\varphi$ es sobreyectiva.
        \end{description}
        Veamos ahora la relación de inyectividad:
        \begin{equation*}
            H = \ker(f) \Longleftrightarrow \varphi \text{\ inyectiva}
        \end{equation*}
        \begin{description}
            \item [$\Longrightarrow)$] Si $H=\ker(f)$ y $\varphi(xH) = 1$, entonces:
                \begin{equation*}
                    1 = \varphi(xH) = f(x) \Longrightarrow x\in \ker(f) = H
                \end{equation*}
                Con lo que $xH = H$, lo que nos dice que $\varphi$ es inyectiva.
            \item [$\Longleftarrow)$] Vamos a ver que $\ker(f) \subseteq H$, ya que la otra ya la tenemos. Para ello, sea $x\in \ker(f)$, entonces:
                \begin{equation*}
                    1 = f(x) = \varphi(p(x)) = \varphi(xH) \Longrightarrow xH\in \ker(\varphi)
                \end{equation*}
                Pero como $\varphi$ es inyectiva, tenemos que $\ker(\varphi) = \{H\}$, con lo que $xH = H$, de donde $x\in H$.
        \end{description}
    \end{proof}
\end{teo}

Cualquier homomorfismo podemos factorizarlo pasando por el grupo cociente.

\begin{figure}[H]
    \centering
    \shorthandoff{""}
    \begin{tikzcd}
    G \arrow[r, "p"] \arrow[rd, "f"'] & G/H \arrow[d, "\varphi", dotted] \\
                                      & G'                              
    \end{tikzcd}
    \shorthandon{""}
    \caption{Situación del Teorema~\ref{teo:prop_universal}.}
    \label{fig:teo_propiedad_universal}
\end{figure}

\section{Teoremas de isomorfía}
\begin{teo}[Primer Teorema de Isomorfía para grupos]
    Sea $f:G\to G'$ un homomorfismo de grupos, entonces existe un isomorfismo de grupos de forma que
    \begin{equation*}
        G/\ker(f) \cong Imf
    \end{equation*}
    Y vendrá definido por $x\ker(f) \mapsto f(x)$.
    \begin{proof}
        En primer lugar, como $\ker(f) \lhd G$, podemos considerar la proyección canónica $p:G\to G/\ker(f)$. Definimos consideramos la restriccion de $f$ a su codominio, lo que nos da un homomorfismo sobreyectivo. Por la propiedad universal del grupo cociente, tenemos que ($f$ es sobreyectiva y conseguimos una biyección):
    \begin{figure}[H]
        \centering
        \shorthandoff{""}
        \begin{tikzcd}
        G \arrow[r, "p"] \arrow[rd, "f"'] & G/\ker(f) \arrow[d, "\varphi", dotted] \\
                                          & Im(f)                              
        \end{tikzcd}
        \shorthandon{""}
    \end{figure}
    De donde deducimos que $G/\ker(f) \cong Im(f)$.
    \end{proof}
\end{teo}

\begin{ejemplo}
    Como consecuencias del primer teorema de isomorfía: consideramos $\bb{K}$ un cuerpo finito con $|K| = q$ elementos. La aplicación $\det:\GL_n(\bb{K})\to \bb{K}^\ast$ es un homomorfismo de grupos. Tenemos que:
    \begin{equation*}
        \ker(\det) = \SL_n(\bb{K})
    \end{equation*}
    Con lo que $\GL_n(\bb{K})/\SL_n(\bb{K}) \cong Im(\det) = \bb{K}^\ast$. Usémoslo para calcular el orden de $|\GL_n(\bb{K})|$ (de otra forma):
    \begin{equation*}
        |\GL_n(\bb{K})| = (q^n-1)(q^n-q) \ldots (q^n-q^{n-1}) = \prod_{k=1}^{n} (q^n - q^{k-1})
    \end{equation*}
    Por lo que:
    \begin{equation*}
        |\SL_n(\bb{K})| = \dfrac{|\GL_n(\bb{K})|}{|\bb{K}^\ast|} = \dfrac{|\GL_n(\bb{K})|}{q-1}
    \end{equation*}
\end{ejemplo}

\begin{teo}[Segundo Teorema de Isomorfía para grupos]
    Sea $G$ un grupo, $H,K<G$ de forma que $K\lhd G$, entonces:
    \begin{equation*}
        H\cap K \lhd H
    \end{equation*}
    Y existe un isomorfismo de grupos de forma que
    \begin{equation*}
        H/H\cap K \cong HK/K
    \end{equation*}
    % \begin{proof} % // TODO: Se hará
    % \end{proof}
\end{teo}

\begin{figure}[H]
    \centering
    \begin{tikzpicture}
        \node (G) {$G$};
        \node (HK) [below=of G] {$HK$};
        \node (H) [below left=of HK] {$H$};
        \node (K) [below right=of HK] {$K$};
        \node (HcapK) [below right=of H] {$H \cap K$};

        \draw (G) -- (HK);
        \draw (HK) -- (H);
        \draw (HK) -- (K);
        \draw (H) -- (HcapK);
        \draw (K) -- (HcapK);
    \end{tikzpicture}
\end{figure}

\begin{ejemplo} % // TODO: Ejercicio 1 de la relacion
    Si $H<S_n$ conteniendo $H$ una permutación impar, entonces $[H:H\cap A_n]=2$. Es decir, $H$ tiene el mismo número de permutaciones pares que de impares.

    Como $H$ tiene una pertmuación impar, sabemos que $A_n\lhd S_n$ y tenemos que $H\nsubseteq A_n$ y se tiene que:
    \begin{equation*}
        HA_n = S_n
    \end{equation*}
    Que se puede deducir observando el retículo de subgrupos de $S_n$. Por el Segundo Teorema de Isomorfía, tenemos que:
    \begin{equation*}
        H/H\cap A_n \cong S_n/A_n \cong \mathbb{Z}_2
    \end{equation*}
\end{ejemplo}

% G
% HK
% H   K
%   H\cap K


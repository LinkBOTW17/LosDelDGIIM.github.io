\section{Grupos resolubles}

\begin{ejercicio}
    Sea $N\lhd G$ un subgrupo normal y simple de un grupo $G$. Demostrar que si $G/N$ tiene una serie de composición entonces $G$ tiene una serie de composición.    
\end{ejercicio}

\begin{ejercicio}
    Sea $G$ un grupo abeliano. Demostrar que $G$ tiene series de composición si y sólo si $G$ es finito.
\end{ejercicio}

\begin{ejercicio}
    Sea $H$ un subgrupo normal de un grupo finito $G$. Demostrar que existe una serie de composición de $G$ uno de cuyos términos es $H$.
\end{ejercicio}

\begin{ejercicio}
    Se define la longitud de un grupo finito $G$, denotada $l(G)$, como la longitud de cualquiera de sus series de composición. Demostrar que si $H$ es un subgrupo normal de $G$ entonces $l(G) = l(H) + l(G/H)$.
\end{ejercicio}

\begin{ejercicio}
    Encontrar todas las series de composición, calcular la longitud y la lista de factores de composición de los siguientes grupos:
    \begin{enumerate}
        \item El grupo diédrico $D_4$.
        \item El grupo alternado $A_4$.
        \item El grupo simétrico $S_4$.
        \item El grupo diédrico $D_5$.
        \item El grupo de cuaterniones $Q_2$.
        \item El grupo cíclico $C_{24}$.
        \item El grupo simétrico $S_5$.
    \end{enumerate}
\end{ejercicio}

\begin{ejercicio}
    Sea $G$ un grupo finito, y
    \[
        G = G_0 \rhd G_1 \rhd \cdots \rhd G_{r-1} \rhd G_r = \{1\}
    \]
    una serie normal de $G$. Demostrar que
    \[
        l(G) = \sum_{i=0}^{r-1} l\left(\frac{G_i}{G_{i+1}}\right), \quad \fact(G) = \prod_{i=0}^{r-1} \fact\left(\frac{G_i}{G_{i+1}}\right).
    \]
\end{ejercicio}

\begin{ejercicio}
    Si $G_1, G_2, \ldots, G_r$ son grupos finitos, demostrar que
    \[
        l(G_1 \times G_2 \times \cdots \times G_r) = \sum_{i=1}^{r} l(G_i), \quad \fact(G_1 \times G_2 \times \cdots \times G_r) = \prod_{i=1}^{r} \fact(G_i).
    \]
\end{ejercicio}

\begin{ejercicio}
    Sea $G$ un grupo cíclico de orden $p^n$ con $p$ primo. Demostrar que $l(G) = n$ y que $\fact(G) = (\bb{Z}_p, \bb{Z}_p, \stackrel{(n)}{\ldots}, \bb{Z}_p)$ ($n$ veces).
\end{ejercicio}

\begin{ejercicio}
    Sea $G$ un grupo cíclico de orden $n$. Si la descomposición de $n$ en factores primos es $n = p_1^{e_1} p_2^{e_2} \cdots p_r^{e_r}$, demostrar que
    \[
        l(G) = e_1 + e_2 + \cdots + e_r,
    \]
    y que
    \[
        \fact(G) = (\bb{Z}_{p_1}, \bb{Z}_{p_1}, \stackrel{(e_1)}{\ldots}, \bb{Z}_{p_r},\stackrel{(e_r)}{\ldots}, \bb{Z}_{p_r}).
    \]
    Aplica el resultado cuando $n = 12$ y compara su longitud y factores de composición con los del grupo $\bb{Z}_2 \times \bb{Z}_6$.
\end{ejercicio}

\begin{ejercicio}
    Sea $D_n$ el grupo diédrico de orden $2n$. Si la descomposición de $n$ en factores primos es $n = p_1^{e_1} p_2^{e_2} \cdots p_r^{e_r}$, demostrar que
    \[
        l(D_n) = e_1 + e_2 + \cdots + e_r + 1,
    \]
    y que
    \[
        \fact(D_n) = (\bb{Z}_{p_1}, \bb{Z}_{p_1}, \stackrel{(e_1)}{\ldots}, \bb{Z}_{p_r},\stackrel{(e_r)}{\ldots}, \bb{Z}_{p_r}, \bb{Z}_2).
    \]
\end{ejercicio}

\begin{ejercicio}
    Demostrar que $D_4$, $D_5$, $S_2$, $S_3$ y $S_4$ son grupos resolubles.
\end{ejercicio}

\begin{ejercicio}
    Sean $H$ y $K$ subgrupos normales de un grupo $G$ tales que $G/H$ y $G/K$ son ambos resolubles. Demostrar que $G/(H \cap K)$ también es resoluble.
\end{ejercicio}

\begin{ejercicio}
    Sea $G$ un grupo resoluble y sea $H$ un subgrupo normal no trivial de $G$. Demostrar que existe un subgrupo no trivial $A$ de $H$ que es abeliano y normal en $G$.
\end{ejercicio}

\begin{ejercicio}
    Demuestra que todo $p$-grupo finito es resoluble.
\end{ejercicio}

\begin{ejercicio}
    Demuestra que todo grupo de orden $pq$, con $p$ y $q$ primos, es un grupo resoluble.
\end{ejercicio}

\begin{ejercicio}
    Demuestra que todo grupo de orden $p^2q$, con $p$ y $q$ primos, es un grupo resoluble.
\end{ejercicio}

\begin{ejercicio}
    Demuestra que si $p_1$, $p_2$, $p_3$ son tres primos tales que $p_3 > p_1 p_2$ entonces cualquier grupo de orden $p_1 p_2 p_3$ es resoluble.
\end{ejercicio}

\begin{ejercicio}~
    \begin{enumerate}
        \item Demuestra que todo grupo de orden $70$ es resoluble.
        \item Demuestra que todo grupo de orden $24$ es resoluble.
        \item Demuestra que todo grupo de orden $100$ es resoluble.
        \item Demuestra que todo grupo de orden $48$ es resoluble.
        \item Sea $G$ un grupo de orden $200$. Demuestra que $G \times D_{41}$ es resoluble.
        \item Demuestra que todo grupo de orden $63$ es soluble (sin usar que es un caso particular de un grupo de orden $p^2q$ con $p$ y $q$ primos).
    \end{enumerate}
\end{ejercicio}
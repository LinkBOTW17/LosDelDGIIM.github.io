\section{Combinatoria y Teoría de Grafos}

\begin{ejercicio}\label{ej:1.1}
    Diez personas están sentadas alrededor de una mesa circular. Cada persona estrecha la mano a todos los demás excepto a la persona sentada directamente enfrente de la mesa. Dibuja un grafo que modele la situación.
\end{ejercicio}

\begin{ejercicio}\label{ej:1.2}
    Seis hermanos (Alonso, Bernardo, Carlos, Daniel, Enrique y Fernando) tienen que emparejarse para compartir habitación en el próximo curso escolar. Cada uno de ellos ha elaborado una lista con los nombres de aquellos con los que quiere emparejarse:
    \begin{itemize}
        \item \ul{Lista de Alonso:} Daniel.
        \item \ul{Lista de Bernardo:} Alonso, Enrique.
        \item \ul{Lista de Carlos:} Daniel, Enrique.
        \item \ul{Lista de Daniel:} Carlos.
        \item \ul{Lista de Enrique:} Daniel, Bernardo, Fernando.
        \item \ul{Lista de Fernando:} Alonso, Bernardo.
    \end{itemize}
    Dibuja el grafo dirigido que modela esta situación.
\end{ejercicio}

\begin{ejercicio}\label{ej:1.3}
    Expresa en forma matricial los grafos de la Figura~\ref{fig:1.3}.
    \begin{figure}
        \centering
        \begin{subfigure}[b]{0.4\textwidth}
            \centering
            \begin{tikzpicture}
                % Nodos con posiciones relativas
                \node[draw, circle] (A) {A};
                \node[draw, circle, below left=of A] (B) {B};
                \node[draw, circle, below right=of A] (C) {C};
                \node[draw, circle, below=of B] (D) {D};
                \node[draw, circle, below=of C] (E) {E};
                
                % Aristas
                \draw (A) -- (B);
                \draw (A) -- (C);
                \draw (B) -- (C);
                \draw (B) -- (D);
                \draw (C) -- (E);
                \draw (D) -- (E);
            \end{tikzpicture}
            \caption{Grafo~\ref{fig:1.3a}.}
            \label{fig:1.3a}
        \end{subfigure}
        \begin{subfigure}[b]{0.4\textwidth}
            \centering
            \begin{tikzpicture}
                % Nodos con posiciones relativas
                \node[draw, circle] (B) {B};
                \node[draw, circle, right=of B] (C) {C};
                \node[draw, circle, below left=of B] (A) {A};
                \node[draw, circle, below right=of C] (D) {D};
                \node[draw, circle, below right=of A] (F) {F};
                \node[draw, circle, right=of F] (E) {E};
                
                % Aristas: A-B-C-D-E-F
                \draw (A) -- (B);
                \draw (B) -- (C);
                \draw (C) -- (D);
                \draw (D) -- (E);
                \draw (E) -- (F);
                \draw (F) -- (A);
                \draw (D) -- (A);
            \end{tikzpicture}
            \caption{Grafo~\ref{fig:1.3b}.}
            \label{fig:1.3b}
        \end{subfigure}
        \caption{Grafos para el ejercicio~\ref{ej:1.3}.}
        \label{fig:1.3}
    \end{figure}
\end{ejercicio}

\begin{ejercicio}\label{ej:1.4}
    Sea $G$ un grafo completo con cuatro vértices. Construye todos sus subgrafos salvo isomorfismo.
\end{ejercicio}

\begin{ejercicio}\label{ej:1.5}
    ¿Son isomorfos los grafos de la Figura~\ref{fig:1.5_1}? ¿Y los de la Figura~\ref{fig:1.5_2}? ¿Y los de la Figura~\ref{fig:1.5_3}?
    
    \begin{figure}
        \centering
        \begin{subfigure}[b]{0.4\textwidth}
            \centering
            \begin{tikzpicture}
                % Nodos con posiciones relativas
                \node[draw, circle] (A) {A};
                \node[draw, circle, right=of A] (B) {B};
                \node[draw, circle, below=of B] (C) {C};
                \node[draw, circle, below=of A] (D) {D};
                
                % Aristas
                \draw (A) -- (B);
                \draw (A) -- (C);
                \draw (A) -- (D);
                \draw (C) -- (D);
                \draw (C) -- (B);
            \end{tikzpicture}
            \caption{Grafo~\ref{fig:1.5_1.a}.}
            \label{fig:1.5_1.a}
        \end{subfigure}
        \begin{subfigure}[b]{0.4\textwidth}
            \centering
            \begin{tikzpicture}
                % Nodos con posiciones relativas
                \node[draw, circle] (A) {A};
                \node[draw, circle, right=of A] (B) {B};
                \node[draw, circle, below=of B] (C) {C};
                \node[draw, circle, below=of A] (D) {D};
                
                % Aristas
                \draw (A) -- (B);
                \draw (A) -- (C);
                \draw (A) -- (D);
                \draw (C) -- (D);
                \draw (D) -- (B);
            \end{tikzpicture}
            \caption{Grafo~\ref{fig:1.5_1.b}.}
            \label{fig:1.5_1.b}
        \end{subfigure}
        \caption{Primer par de grafos para el ejercicio~\ref{ej:1.5}.}
        \label{fig:1.5_1}
    \end{figure}


    \begin{figure}
        \centering
        \begin{subfigure}[b]{0.4\textwidth}
            \centering
            \begin{tikzpicture}
                % Nodos con posiciones relativas
                \node[draw, circle] (A) {A};
                \node[draw, circle, below left=of A] (B) {B};
                \node[draw, circle, below right=of A] (E) {E};
                \node[draw, circle, below=of B] (C) {C};
                \node[draw, circle, below=of E] (D) {D};
                
                % Aristas
                \draw (A) -- (B);
                \draw (A) -- (E);
                \draw (B) -- (C);
                \draw (B) -- (D);
                \draw (B) -- (E);
                \draw (C) -- (E);
                \draw (C) -- (D);
                \draw (D) -- (E);
            \end{tikzpicture}
            \caption{Grafo~\ref{fig:1.5_2.a}.}
            \label{fig:1.5_2.a}
        \end{subfigure}
        \begin{subfigure}[b]{0.4\textwidth}
            \centering
            \begin{tikzpicture}
                % Nodos con posiciones relativas
                \node[draw, circle] (A) {A};
                \node[draw, circle, below left=of A] (B) {B};
                \node[draw, circle, below right=of A] (E) {E};
                \node[draw, circle, below=of B] (C) {C};
                \node[draw, circle, below=of E] (D) {D};
                
                % Aristas
                \draw (A) -- (B);
                \draw (A) -- (E);
                \draw (B) -- (C);
                \draw (A) -- (D);
                \draw (B) -- (E);
                \draw (C) -- (A);
                \draw (C) -- (D);
                \draw (D) -- (E);
            \end{tikzpicture}
            \caption{Grafo~\ref{fig:1.5_2.b}.}
            \label{fig:1.5_2.b}
        \end{subfigure}
        \caption{Segundo par de grafos para el ejercicio~\ref{ej:1.5}.}
        \label{fig:1.5_2}
    \end{figure}


    \begin{figure}
        \centering
        \begin{subfigure}[b]{0.4\textwidth}
            \centering
            \begin{tikzpicture}
                % Nodos con posiciones relativas
                \node[draw, circle] (B) {B};
                \node[draw, circle, right=of B] (C) {C};
                \node[draw, circle, below left=of B] (A) {A};
                \node[draw, circle, below right=of C] (D) {D};
                \node[draw, circle, below right=of A] (F) {F};
                \node[draw, circle, right=of F] (E) {E};
                
                % Aristas: A-B-C-D-E-F
                \draw (A) -- (B);
                \draw (B) -- (C);
                \draw (C) -- (D);
                \draw (D) -- (E);
                \draw (E) -- (F);
                \draw (F) -- (A);
                \draw (D) -- (A);
                \draw (B) -- (E);
                \draw (F) -- (C);
            \end{tikzpicture}
            \caption{Grafo~\ref{fig:1.5_3.a}.}
            \label{fig:1.5_3.a}
        \end{subfigure}
        \begin{subfigure}[b]{0.4\textwidth}
            \centering
            \begin{tikzpicture}
                % Nodos con posiciones relativas
                \node[draw, circle] (B) {B};
                \node[draw, circle, right=of B] (C) {C};
                \node[draw, circle, left=of B] (A) {A};
                \node[draw, circle, below =of A, yshift=-3em] (D) {D};
                \node[draw, circle, right =of D] (E) {E};
                \node[draw, circle, right =of E] (F) {F};
                
                
                % Aristas: A-D,E,F
                \draw (A) -- (D);
                \draw (A) -- (E);
                \draw (A) -- (F);
                \draw (B) -- (D);
                \draw (B) -- (E);
                \draw (B) -- (F);
                \draw (C) -- (D);
                \draw (C) -- (E);
                \draw (C) -- (F);
            \end{tikzpicture}
            \caption{Grafo~\ref{fig:1.5_3.b}.}
            \label{fig:1.5_3.b}
        \end{subfigure}
        \caption{Tercer par de grafos para el ejercicio~\ref{ej:1.5}.}
        \label{fig:1.5_3}
    \end{figure}
\end{ejercicio}

\begin{ejercicio}\label{ej:1.6}
    Demostrar que, en cualquier grafo, el número de vértices de grado impar es par.
    (Así, en un grupo de personas, el número total de personas que estrechan la mano de un número impar de otras personas es siempre par).
\end{ejercicio}

\begin{ejercicio}\label{ej:1.7}
    Demostrar que si cada vértice de un grafo $G$ es de grado 2, cada componente conexa de $G$ es un ciclo.
\end{ejercicio}

\begin{ejercicio}\label{ej:1.8}
    Los siguientes hechos se conocen de las personas A, B, C, D, E, F, G:
    \begin{itemize}
        \item A habla inglés.
        \item B habla inglés y español.
        \item C habla inglés, italiano y ruso.
        \item D habla japonés y español.
        \item E habla alemán e italiano.
        \item F habla francés, japonés y ruso.
        \item G habla francés y alemán.
    \end{itemize}
    Demostrar que cada par de personas entre estas siete puede comunicarse (con la ayuda de intérpretes, si es necesario, tomados de los cinco restantes).
\end{ejercicio}

\begin{ejercicio}\label{ej:1.9}
    Demuestra que en todo grafo con más de un vértice existen dos vértices con el mismo grado.
\end{ejercicio}

\begin{ejercicio}\label{ej:1.10}
    Prueba que si un grafo $G$ contiene solo dos vértices de grado impar entonces ambos han de encontrarse en la misma componente conexa.
\end{ejercicio}

\begin{ejercicio}\label{ej:1.11}
    ¿Existe algún grafo regular de grado 5 con 25 vértices?
\end{ejercicio}

\begin{ejercicio}\label{ej:1.12}
    ¿Existe un grafo completo con 595 lados?
\end{ejercicio}

\begin{ejercicio}\label{ej:1.13}
    ¿Existe un grafo con 6 vértices cuyos grados sean 1, 2, 2, 3, 4 y 4 respectivamente?
\end{ejercicio}

\begin{ejercicio}\label{ej:1.14}
    En cada uno de los siguientes casos, dibuja un grafo de Euler que verifique las condiciones, o prueba que tal grafo no existe:
    \begin{enumerate}
        \item Con un número par de vértices y un número par de lados.
        \item Con un número par de vértices y un número impar de lados.
        \item Con un número impar de vértices y un número par de lados.
        \item Con un número impar de vértices y un número impar de lados.
    \end{enumerate}
\end{ejercicio}

\begin{ejercicio}\label{ej:1.15}
    Encuentra un circuito de Euler para los grafos de la Figura~\ref{fig:1.15}.
    

    \begin{figure}
        \centering
        \begin{subfigure}[b]{0.4\textwidth}
            \centering
            \resizebox{1.2\textwidth}{!}{
            \begin{tikzpicture}[node distance=0.5cm]
                % Nodos con posiciones relativas
                \node[draw, circle] (A) {A};
                \node[draw, circle, below left=of A] (B) {B};
                \node[draw, circle, below right=of A] (C) {C};
                \node[draw, circle, below left=of B] (D) {D};
                \node[draw, circle, below right=of B] (E) {E};
                \node[draw, circle, below right=of C] (F) {F};
                \node[draw, circle, below left=of D] (G) {G};
                \node[draw, circle, below right=of D] (H) {H};
                \node[draw, circle, below left=of F] (I) {I};
                \node[draw, circle, below right=of F] (J) {J};
        
                % Aristas
                \draw (A) -- (B);
                \draw (A) -- (C);
                \draw (B) -- (D);
                \draw (B) -- (C);
                \draw (B) -- (E);
                \draw (C) -- (E);
                \draw (C) -- (F);
                \draw (D) -- (G);
                \draw (D) -- (H);
                \draw (D) -- (E);
                \draw (E) -- (H);
                \draw (E) -- (I);
                \draw (E) -- (F);
                \draw (F) -- (I);
                \draw (F) -- (J);
                \draw (G) -- (H);
                \draw (H) -- (I);
                \draw (I) -- (J);
            \end{tikzpicture}}
            \caption{Grafo~\ref{fig:1.15a}.}
            \label{fig:1.15a}
        \end{subfigure}
        \begin{subfigure}[b]{0.4\textwidth}
            \centering
            \begin{tikzpicture}
                % Nodos con posiciones relativas
                \node[draw, circle] (A) {A};
                \node[draw, circle, below left=of A] (B) {B};
                \node[draw, circle, below right=of A] (C) {C};
                \node[draw, circle, below=of B] (D) {D};
                \node[draw, circle, below=of C] (E) {E};
                \node[draw, circle, below right=of D] (F) {F};
                
                % Aristas
                \draw (A) -- (B);
                \draw (A) -- (C);
                \draw (B) -- (C);
                \draw (B) -- (D);
                \draw (B) -- (E);
                \draw (C) -- (A);
                \draw (C) -- (D);
                \draw (C) -- (E);
                \draw (D) -- (E);
                \draw (D) -- (F);
                \draw (F) -- (E);
            \end{tikzpicture}
            \caption{Grafo~\ref{fig:1.15b}.}
            \label{fig:1.15b}
        \end{subfigure}
        
        \caption{Grafos para el ejercicio~\ref{ej:1.15}.}
        \label{fig:1.15}
    \end{figure}
\end{ejercicio}


\begin{ejercicio}\label{ej:1.16}
    Encuentra un camino de Euler para los grafos de la Figura~\ref{fig:1.16}.
    

    \begin{figure}
        \centering
        \begin{subfigure}[b]{0.4\textwidth}
            \centering
            \begin{tikzpicture}
                % Nodos con posiciones relativas
                \node[draw, circle] (A) {A};
                \node[draw, circle, below left=of A] (C) {C};
                \node[draw, circle, below right=of A] (D) {D};
                \node[draw, circle, below=of C] (F) {F};
                \node[draw, circle, below=of D] (G) {G};
                \node[draw, circle, below right=of F] (I) {I};

                \node[draw, circle, above right=of D] (B) {B};
                \node[draw, circle, below right=of B] (E) {E};
                \node[draw, circle, below=of E] (H) {H};
                \node[draw, circle, below right=of G] (J) {J};
                
                % Aristas
                \draw (A) -- (C);
                \draw (A) -- (D);
                \draw (B) -- (D);
                \draw (B) -- (E);
                \draw (C) -- (D);
                \draw (C) -- (F);
                \draw (C) -- (G);
                \draw (D) -- (E);
                \draw (D) -- (H);
                \draw (D) -- (G);
                \draw (E) -- (H);
                \draw (F) -- (G);
                \draw (F) -- (I);
                \draw (G) -- (J);
                \draw (G) -- (I);
                \draw (H) -- (J);
                \draw (G) -- (H);
                \draw (F) -- (D);
                \draw (G) -- (E);
            \end{tikzpicture}
            \caption{Grafo~\ref{fig:1.16a}.}
            \label{fig:1.16a}
        \end{subfigure}\qquad 
        \begin{subfigure}[b]{0.4\textwidth}
            \centering
            \begin{tikzpicture}
                % Nodos con posiciones relativas
                \node[draw, circle] (A) {A};
                \node[draw, circle, below left=of A] (B) {B};
                \node[draw, circle, below right=of A] (C) {C};
                \node[draw, circle, below=of B] (D) {D};
                \node[draw, circle, below=of C] (E) {E};
                \node[draw, circle, below=of A, yshift=-3.9em] (F) {F};
                \node[draw, circle, below=of D] (G) {G};
                \node[draw, circle, below=of E] (H) {H};
                
                % Aristas
                \draw (A) -- (B);
                \draw (A) -- (C);
                \draw (A) -- (F);
                \draw (B) -- (C);
                \draw (B) -- (D);
                \draw (B) -- (F);
                \draw (C) -- (F);
                \draw (C) -- (E);
                \draw (D) -- (F);
                \draw (E) -- (F);
                \draw (G) -- (F);
                \draw (H) -- (F);
                \draw (G) -- (H);
            \end{tikzpicture}
            \caption{Grafo~\ref{fig:1.16b}.}
            \label{fig:1.16b}
        \end{subfigure}
        
        \caption{Grafos para el ejercicio~\ref{ej:1.16}.}
        \label{fig:1.16}
    \end{figure}
\end{ejercicio}

\begin{ejercicio}\label{ej:1.17}
    Encontrar un circuito de Euler en el grafo de la Figura~\ref{fig:1.17_1} y un camino de Euler en el grafo de la Figura~\ref{fig:1.17_2}.


    \begin{figure}
        \centering
        \begin{tikzpicture}
            % Nodos con posiciones relativas
            \node[draw, circle] (A) {A};
            \node[draw, circle, right=of A] (B) {B};
            \node[draw, circle, right=of B] (C) {C};
            \node[right=of C] (Ph) {\phantom{C}};
            \node[draw, circle, right=of Ph] (E) {E};
            \node[below=of A] (Ph2) {};
            \node[draw, circle, right=of E] (F) {F};
            \node[draw, circle, below=of Ph2] (G) {G};
            \node[draw, circle, right=of G] (H) {H};
            \node[draw, circle, right=of H] (I) {I};
            \node[draw, circle, below=of Ph] (D) {D};
            \node[draw, circle, right=of I] (J) {J};
            \node[draw, circle, below=of E] (K) {K};
            \node[below=of F] (Ph3) {\phantom{C}};
            \node[draw, circle, below =of Ph3, yshift=-3em] (L) {L};
        
            % Aristas
            \draw (A) -- (B);
            \draw (A) -- (G);
            \draw (B) -- (C);
            \draw (B) -- (G);
            \draw (B) -- (H);
            \draw (B) -- (I);
            \draw (B) -- (J);
            \draw (C) -- (D);
            \draw (C) -- (J);
            \draw (C) -- (E);
            \draw (E) -- (D);
            \draw (E) -- (J);
            \draw (E) -- (F);
            \draw (H) -- (I);
            \draw (I) -- (J);
            \draw (J) -- (L);
            \draw (J) -- (K);
            \draw (K) -- (L);
            \draw (F) -- (L);
            \draw (I) -- (L);
        \end{tikzpicture}
        
        \caption{Primer grafo para el ejercicio~\ref{ej:1.17}.}
        \label{fig:1.17_1}
    \end{figure}


    \begin{figure}
        \centering
        \begin{tikzpicture}
            % Nodos con posiciones relativas
            \node[draw, circle] (A) {A};
            \node[draw, circle, right=of A] (B) {B};
            \node[draw, circle, right=of B] (C) {C};
            \node[draw, circle, right=of C] (D) {D};
            \node[draw, circle, above=of A] (E) {E};
            \node[draw, circle, above=of B] (F) {F};
            \node[draw, circle, above=of C] (G) {G};
            \node[draw, circle, above=of D] (H) {H};
        
            % Aristas
            \draw (A) -- (B);
            \draw (A) -- (F);
            \draw (A) -- (E);
            \draw (B) -- (E);
            \draw (B) -- (F);
            \draw (E) -- (F);
            \draw (B) -- (C);
            \draw (F) -- (G);
            \draw (C) -- (G);
            \draw (C) -- (D);
            \draw (C) -- (H);
            \draw (D) -- (G);
            \draw (G) -- (H);
        \end{tikzpicture}
        
        
        \caption{Segundo grafo para el ejercicio~\ref{ej:1.17}.}
        \label{fig:1.17_2}
    \end{figure}
\end{ejercicio}

\begin{ejercicio}\label{ej:1.18}
    ¿Para qué valores de $n$ el grafo $K_n$ es un circuito de Euler?
\end{ejercicio}

\begin{ejercicio}\label{ej:1.19}
    Un viajante vive en la ciudad A y se supone que visita las ciudades B, C y D antes de volver a A. Encontrar la ruta más corta que consuma este viaje si las distancias entre las cuatro ciudades son, en Km:
    \begin{itemize}
        \item 120 entre A y B.
        \item 70 entre B y C.
        \item 140 entre A y C.
        \item 180 entre A y D.
        \item 100 entre B y D.
        \item 110 entre C y D.
    \end{itemize}
\end{ejercicio}

\begin{ejercicio}\label{ej:1.20}
    El grafo línea $L(G)$ de un un grafo $G$ se define como sigue: Los vértices de $L(G)$ son los lados de $G$, $V(L(G)) = E(G)$; y dos vértices en $L(G)$ son adyacentes si y solo si los lados correspondientes en $G$ comparten un vértice. Demostrar:
    \begin{enumerate}
        \item Si $G$ es un grafo conexo regular de grado $r$, entonces $L(G)$ es un grafo de Euler.
        \item Si $G$ es un grafo de Euler entonces $L(G)$ es Hamiltoniano.
    \end{enumerate}
\end{ejercicio}

\begin{ejercicio}\label{ej:1.21}
    De entre los grafos de la Figura~\ref{fig:1.21_1} y la Figura~\ref{fig:1.21_2}, ¿cuáles contienen un circuito de Hamilton?
    
    \begin{figure}
        \centering
        \begin{tikzpicture}
            % Nodos con posiciones relativas
            \node[draw, circle] (A) {A};
            \node[draw, circle, below=of A] (C) {C};
            \node[draw, circle, right=of C] (D) {D};
            \node[draw, circle, right=of D] (E) {E};
            \node[draw, circle, above=of E] (B) {B};
            \node[draw, circle, below=of D] (H) {H};
            \node[draw, circle, left=of H] (G) {G};
            \node[draw, circle, left=of G] (F) {F};
            \node[draw, circle, right=of H] (I) {I};
            \node[draw, circle, right=of I] (J) {J};
            \node[draw, circle, below=of F] (L) {L};
            \node[draw, circle, below=of G] (M) {M};
            \node[draw, circle, left=of L] (K) {K};
            \node[draw, circle, below=of I] (N) {N};
            \node[draw, circle, below=of J] (O) {O};
            \node[draw, circle, right=of O] (P) {P};
            \node[draw, circle, below=of L] (Q) {Q};
            \node[draw, circle, right=of Q] (R) {R};
            \node[draw, circle, below=of N, yshift=-0.1em] (S) {S};
            \node[draw, circle, right=of S] (T) {T};
            \node[draw, circle, below right=of R] (U) {U};
            \node[draw, circle, below=of U] (V) {V};


            % Aristas
            \draw (A) -- (B);
            \draw (A) -- (C);
            \draw (C) -- (D);
            \draw (D) -- (E);
            \draw (E) -- (B);
            \draw (D) -- (H);
            \draw (C) -- (F);
            \draw (E) -- (J);
            \draw (F) -- (G);
            \draw (G) -- (H);
            \draw (H) -- (I);
            \draw (I) -- (J);
            \draw (F) -- (L);
            \draw (L) -- (Q);
            \draw (G) -- (M);
            \draw (M) -- (R);
            \draw (I) -- (N);
            \draw (N) -- (S);
            \draw (J) -- (O);
            \draw (O) -- (T);
            \draw (Q) -- (R);
            \draw (R) -- (S);
            \draw (S) -- (T);
            \draw (Q) -- (U);
            \draw (U) -- (T);
            \draw (U) -- (V);
            \draw (K) -- (L);
            \draw (O) -- (P);

            % A-K,   flecha curvada
            \draw (A) to [bend right=35] (K);
            \draw (K) to [bend right=35] (V);
            % B-P,   flecha curvada
            \draw (B) to [bend right=-35] (P);
            \draw (P) to [bend right=-35] (V);
        \end{tikzpicture}
        
        
        \caption{Primer grafo para el ejercicio~\ref{ej:1.21}.}
        \label{fig:1.21_1}
    \end{figure}




    \begin{figure}
        \centering
        \begin{tikzpicture}
            % Nodos con posiciones relativas
            \node[draw, circle] (A) {A};
            \node[draw, circle, below=of A] (C) {C};
            \node[draw, circle, below=of C] (E) {E};
            \node[draw, circle, left=of C] (B) {B};
            \node[draw, circle, right=of C] (D) {D};

            % Aristas
            \draw (A) -- (B);
            \draw (A) -- (D);
            \draw (B) -- (C) -- (D) -- (E) -- (B);
        \end{tikzpicture}
        
        
        \caption{Segundo grafo para el ejercicio~\ref{ej:1.21}.}
        \label{fig:1.21_2}
    \end{figure}
\end{ejercicio}

\begin{ejercicio}\label{ej:1.22}~
    \begin{enumerate}
        \item Prueba, utilizando el algoritmo explicado en clase, que la sucesión $4 \geq 4 \geq 4 \geq 3 \geq 3 \geq 3 \geq 2 \geq 1$ es gráfica y, utilizando dicho algoritmo, encuentra un grafo que la tenga como sucesión de grados.
        
        \item El grafo con matriz de adyacencia $M$ dada por:
        \[
            M=\begin{pmatrix}
                0 & 1 & 1 & 1 & 0 & 1 & 1 & 0 \\
                1 & 0 & 1 & 1 & 1 & 1 & 1 & 0 \\
                1 & 1 & 0 & 1 & 0 & 1 & 0 & 0 \\
                1 & 1 & 1 & 0 & 0 & 1 & 1 & 1 \\
                0 & 1 & 0 & 0 & 0 & 1 & 1 & 1 \\
                1 & 1 & 1 & 1 & 1 & 0 & 1 & 0 \\
                1 & 1 & 0 & 1 & 1 & 1 & 0 & 1 \\
                0 & 0 & 0 & 1 & 1 & 0 & 1 & 0
            \end{pmatrix}
        \]
        es de Euler o en él hay un camino de Euler entre dos vértices. Razona cuál es la situación y encuentra, en su caso, el circuito o el camino de Euler que existe.
    \end{enumerate}
\end{ejercicio}

\begin{ejercicio}\label{ej:1.23}~
    \begin{enumerate}
        \item En el grafo $G$ cuya matriz de adyacencia es
        \[
            M=\begin{pmatrix}
                0 & 1 & 0 & 1 & 1 & 1 & 0 & 0 \\
                1 & 0 & 0 & 0 & 0 & 0 & 0 & 1 \\
                0 & 0 & 0 & 1 & 1 & 1 & 1 & 0 \\
                1 & 0 & 1 & 0 & 0 & 1 & 0 & 1 \\
                1 & 0 & 1 & 0 & 0 & 0 & 0 & 0 \\
                1 & 0 & 1 & 1 & 0 & 0 & 0 & 1 \\
                0 & 0 & 1 & 0 & 0 & 0 & 0 & 1 \\
                0 & 1 & 0 & 1 & 0 & 1 & 1 & 0
            \end{pmatrix}
        \]
        determina el número de aristas y la sucesión de grados de los vértices y, caso de que $G$ sea de Euler, describe un circuito de Euler en él usando el algoritmo apropiado.
        
        \item Calcula el número de vértices de un grafo plano, conexo y regular de grado 5 con 20 caras.
    \end{enumerate}
\end{ejercicio}

\begin{ejercicio}\label{ej:1.24}~
    \begin{enumerate}
        \item La siguiente matriz es la matriz de incidencia o adyacencia de un grafo. Razona qué caso es y dibuja el correspondiente grafo.
        \[
            M=\begin{pmatrix}
                1 & 0 & 1 & 0 & 0 & 0 & 0 & 0 \\
                1 & 1 & 0 & 0 & 0 & 0 & 0 & 0 \\
                0 & 1 & 0 & 1 & 1 & 0 & 0 & 0 \\
                0 & 0 & 1 & 1 & 0 & 0 & 0 & 0 \\
                0 & 0 & 0 & 0 & 1 & 1 & 1 & 0 \\
                0 & 0 & 0 & 0 & 0 & 1 & 0 & 1 \\
                0 & 0 & 0 & 0 & 0 & 0 & 1 & 1
            \end{pmatrix}
        \]
        ¿Es el grafo anterior de Euler o Hamilton? Razona la respuesta y da un circuito de Euler o Hamilton en caso de que los haya.
        
        \item Aplica el algoritmo para comprobar si la siguiente sucesión $$6 \geq 4 \geq 4 \geq 3 \geq 3 \geq 3 \geq 3 \geq 3$$ es, o no es, una sucesión gráfica y, en caso de serlo, también aplica el algoritmo para encontrar un grafo que la tenga como sucesión de grados.
    \end{enumerate}
\end{ejercicio}


\begin{ejercicio}\label{ej:1.25}
    Razona cuál es la respuesta correcta en cada una de las siguientes cuestiones (todos los grafos a los que se hace referencia son simples, no tienen lazos ni lados paralelos):
    \begin{enumerate}
        \item El grafo completo $K_n$:
        \begin{enumerate}
            \item Es siempre de Euler.
            \item Es siempre de Hamilton.
            \item Dependiendo de $n$ puede ser, o no, de Hamilton o de Euler.
        \end{enumerate}
        \item He encontrado un grafo plano y conexo con 200 vértices y:
        \begin{enumerate}
            \item Un número par de caras y un número impar de lados.
            \item Un número par de lados y un número impar de caras.
            \item Un número par de lados y caras.
        \end{enumerate}
        \item Tengo un grafo con un solo vértice de grado impar $v$:
        \begin{enumerate}
            \item Puedo encontrar un camino que empiece en ese vértice $v$, recorra todos los lados del grafo solo una vez y vuelva a él.
            \item Si añado un lado que conecte ese vértice con otro cualquiera del grafo, pongamos $w$, puedo encontrar un camino que empiece en $v$, recorra todos los lados del grafo (incluido el que he añadido) solo una vez y termine en $w$.
            \item Es imposible tener un grafo como ese.
        \end{enumerate}
        \item En un grafo plano con cinco componentes conexas y 24 lados:
        \begin{enumerate}
            \item El número de vértices y el número de caras son opuestos módulo 30.
            \item El número de vértices y el número de caras son congruentes módulo 30.
            \item Ninguna de las anteriores es cierta.
        \end{enumerate}
        \item Dado un grafo regular de grado 1, entonces:
        \begin{enumerate}
            \item El grafo no puede ser conexo.
            \item El grafo tiene tantas componentes conexas como vértices.
            \item El grafo tiene tantas componentes conexas como lados.
        \end{enumerate}
        \item Un grafo regular conexo de grado 11 con veinte vértices:
        \begin{enumerate}
            \item Es siempre de Euler.
            \item Es siempre de Hamilton.
            \item Ninguna de las dos respuestas anteriores es cierta.
        \end{enumerate}
        \item Elija la respuesta correcta:
        \begin{enumerate}
            \item Sólo hay dos grafos con cuatro vértices y cuatro lados no isomorfos.
            \item Todos los grafos con cuatro vértices y cuatro lados son isomorfos.
            \item Sólo hay tres grafos con cuatro vértices y cuatro lados no isomorfos.
        \end{enumerate}
        \item Un grafo cuya matriz de adyacencia es
        \[
            \begin{pmatrix}
                0 & 1 & 1 & 0 & 0 & 0 & 0 \\
                1 & 0 & 1 & 0 & 0 & 0 & 0 \\
                1 & 1 & 0 & 0 & 0 & 0 & 0 \\
                0 & 0 & 0 & 0 & 1 & 1 & 0 \\
                0 & 0 & 0 & 0 & 0 & 1 & 1 \\
                0 & 0 & 0 & 1 & 1 & 0 & 0 \\
                0 & 0 & 0 & 1 & 1 & 0 & 0
            \end{pmatrix}
        \]
        \begin{enumerate}
            \item Es de Euler.
            \item No es de Euler pero hay un camino de Euler entre dos vértices.
            \item No es de Euler pero sus componentes conexas sí lo son.
        \end{enumerate}
        \item Un grafo cuya matriz de incidencia es
        \[
            \begin{pmatrix}
                1 & 1 & 0 & 0 & 0 \\
                1 & 0 & 1 & 0 & 0 \\
                0 & 1 & 1 & 1 & 1 \\
                0 & 0 & 0 & 1 & 0 \\
                0 & 0 & 0 & 0 & 1
            \end{pmatrix}
        \]
        \begin{enumerate}
            \item Es de Hamilton.
            \item No es de Hamilton pero sus componente conexas sí lo son.
            \item No es de Hamilton y tampoco lo son sus componentes conexas.
        \end{enumerate}
        \item La siguiente matriz
        \[
            \begin{pmatrix}
                0 & 1 & 0 & 0 & 0 \\
                1 & 0 & 1 & 0 & 0 \\
                1 & 1 & 0 & 1 & 1 \\
                0 & 0 & 1 & 0 & 1 \\
                0 & 0 & 1 & 1 & 0
            \end{pmatrix}
        \]
        \begin{enumerate}
            \item Puede ser la matriz de adyacencia de un grafo pero no la de incidencia.
            \item Puede ser la matriz de incidencia de un grafo pero no la de adyacencia.
            \item No puede ser la matriz de adyacencia ni la de incidencia de un grafo.
        \end{enumerate}
    \end{enumerate}
\end{ejercicio}

\begin{ejercicio}\label{ej:1.26}~
    \begin{enumerate}
        \item Prueba, utilizando el algoritmo explicado en clase, que la sucesión dada por $3 \geq 3 \geq 2 \geq 2 \geq 2 \geq 2 \geq 2$ es gráfica y, utilizando dicho algoritmo, encuentra un grafo en que los grados de sus vértices sean los términos de esa sucesión. Prueba que el grafo es plano y que satisface el teorema de la característica de Euler.
        
        \item Considera los grafos $G_1$ dado por el diagrama de la Figura~\ref{fig:1.26_1} y $G_2$ con matriz de incidencia
        \[
            \begin{pmatrix}
                1 & 1 & 0 & 0 & 0 & 0 & 0 & 1 \\
                1 & 0 & 1 & 1 & 1 & 0 & 0 & 0 \\
                0 & 0 & 0 & 1 & 0 & 1 & 1 & 1 \\
                0 & 1 & 1 & 0 & 0 & 1 & 0 & 0 \\
                0 & 0 & 0 & 0 & 1 & 0 & 1 & 0 \\
            \end{pmatrix}
        \]
        Estudia si son o no isomorfos, si son o no planos, si son o no de Euler o si hay un camino de Euler (en caso afirmativo aplica el algoritmo para calcular un circuito o un camino de Euler) y si son o no de Hamilton (encontrando el camino en caso afirmativo).
        \begin{figure}
            \centering
            \begin{tikzpicture}
                % Nodos con posiciones relativas
                \node[draw, circle] (1) {$1$};
                \node[draw, circle, right=of 1] (2) {$2$};
                \node[draw, circle, below right=of 2] (3) {$3$};
                \node[draw, circle, below left=of 1] (5) {$5$};
                \node[draw, circle, below left=of 3] (4) {$4$};

                % Aristas
                \draw (1) -- (5);
                \draw (1) -- (4);
                \draw (1) -- (3);
                \draw (2) -- (5);
                \draw (2) -- (4);
                \draw (2) -- (3);
                \draw (5) -- (4) -- (3);
            \end{tikzpicture}
            \caption{Grafo para el ejercicio~\ref{ej:1.26}.}
            \label{fig:1.26_1}
        \end{figure}
    \end{enumerate}
\end{ejercicio}


\begin{ejercicio}\label{ej:1.27}~
    \begin{enumerate}
        \item Si $G$ es un grafo completo con 6 vértices entonces:
        \begin{enumerate}
            \item $G$ es regular de grado 5.
            \item $G$ tiene 20 aristas.
            \item $G$ es de Euler y de Hamilton.
        \end{enumerate}
        \item Sea $G'$ un subgrafo completo (pleno) de un grafo $G$. Entonces:
        \begin{enumerate}
            \item Si $G$ es de Euler también $G'$ es de Euler.
            \item Si $G$ es de Hamilton también $G'$ es de Hamilton.
            \item Ninguna de las anteriores.
        \end{enumerate}
        \item Seleccione la respuesta correcta:
        \begin{enumerate}
            \item Sólo hay dos grafos con cuatro vértices y 5 lados no isomorfos.
            \item Todos los grafos con cuatro vértices y 5 lados son isomorfos.
            \item Todos los grafos con cuatro vértices y cinco lados son de Euler.
        \end{enumerate}
        \item Sea $G$ un grafo plano conexo regular de grado 6 con 15 caras. Entonces:
        \begin{enumerate}
            \item $G$ tiene 13 vértices.
            \item El número de vértices es el triple del de aristas.
            \item No existe un tal grafo.
        \end{enumerate}
        \item Salvo isomorfismos, grafos con 50 vértices y 1225 aristas:
        \begin{enumerate}
            \item Sólo hay 1.
            \item Hay 2.
            \item No existen grafos en esas condiciones.
        \end{enumerate}
    \end{enumerate}
\end{ejercicio}


\begin{ejercicio}\label{ej:1.28}~
    \begin{enumerate}
        \item Considera la sucesión $4,4,4,4,4$.
        \begin{enumerate}
            \item Utiliza el algoritmo dado en clase para probar que la sucesión es una sucesión gráfica y para dibujar un grafo $G$ que la tenga como sucesión gráfica.
            \item Calcula las matrices incidencia y adyacencia del grafo $G$ obtenido en el apartado anterior.
            \item ¿Es $G$ de Euler o tiene un camino de Euler? En caso afirmativo, utiliza el algoritmo dado en clase para calcular el circuito o el camino de Euler.
            \item ¿Es $G$ de Hamilton? En caso afirmativo calcula el circuito de Hamilton.
            \item ¿Es $G$ plano? En caso afirmativo comprueba la fórmula de la característica de Euler.
        \end{enumerate}
        \item Demuestra que si $G$ es un grafo de Euler con $n$ vértices que solo tiene 2 vértices de grado 2 entonces el número de aristas es $\geq 2n - 2$.
    \end{enumerate}
\end{ejercicio}

\begin{ejercicio}\label{ej:1.29}~
    \begin{enumerate}
        \item Considera el subconjunto $X = \{(12),(13),(23)\} \subset S_3$ y el siguiente grafo $G$: Los vértices de $G$ son los elementos de $S_3$ y hay un lado entre dos vértices $x$ e $y$ si $xy^{-1} \in X$.
        \begin{enumerate}
            \item Dibuja el grafo.
            \item Calcula sus matrices de incidencia y adyacencia.
            \item ¿Es de Euler o tiene un camino de Euler? En caso afirmativo aplica el algoritmo dado en clase para calcular un ciclo o un camino de Euler.
            \item ¿Es de Hamilton? En caso afirmativo calcula el ciclo de Hamilton.
            \item ¿Es plano? En caso afirmativo comprueba la fórmula de Euler.
        \end{enumerate}
        \item Si $G$ es un grafo con $n$ vértices y $m$ lados. Prueba que $m \leq \frac{n(n-1)}{2}$ y que se da la igualdad si y solo si $G = K_n$ es el grafo completo.
    \end{enumerate}
\end{ejercicio}

\begin{ejercicio}\label{ej:1.30}
    Demuestra, utilizando el algoritmo explicado en clase, que la sucesión de grados de los vértices de un octaedro (poliedro regular con 6 vértices, 8 caras y 12 aristas) es gráfica y, utilizando dicho algoritmo, encuentra un grafo $G$ en que los grados de sus vértices sean los términos de esa sucesión. Encuentra las matrices de adyacencia e incidencia de $G$.
    
    Comprueba que el grafo $G$ es plano y estudia si es de Euler y, en caso afirmativo, determina por algún algoritmo explicado en clase un circuito de Euler para $G$. ¿Es $G$ un grafo de Hamilton? Razona la respuesta.
\end{ejercicio}


\begin{ejercicio}\label{ej:1.31}
    Razona cuál es la respuesta correcta en cada una de las siguientes cuestiones (todos los grafos a los que se hace referencia son simples, no tienen lazos ni lados paralelos):
    \begin{enumerate}
        \item La sucesión $70, 69, 68, \ldots, 3, 2, 1$.
        \begin{enumerate}
            \item Es una sucesión gráfica y su grafo asociado es el completo $K_{70}$.
            \item Es una sucesión gráfica pero su grafo asociado no es $K_{70}$.
            \item No es una sucesión gráfica.
        \end{enumerate}
        \item Tengo un grafo conexo con 6 vértices y 9 lados:
        \begin{enumerate}
            \item Puedo asegurar que es plano.
            \item Puedo asegurar que no es plano.
            \item Puede ser plano o no serlo.
        \end{enumerate}
        \item La sucesión $4, 4, 4, 4$:
        \begin{enumerate}
            \item No es una sucesión gráfica pero si le añadimos al final un 2 si lo es.
            \item No es una sucesión gráfica pero si le añadimos al final un 3 si lo es.
            \item No es una sucesión gráfica pero si le añadimos al final un 4 si lo es.
        \end{enumerate}
        \item Puedo encontrar un grafo plano conexo con:
        \begin{enumerate}
            \item Un número impar de vértices, un número impar de lados y un número impar de caras.
            \item Un número par de vértices, un número par de lados y un número impar de caras.
            \item Un número impar de vértices, un número par de lados y un número impar de caras.
        \end{enumerate}
        \item La sucesión $4, 2, 2, 2, 2$:
        \begin{enumerate}
            \item Es la sucesión de grados de un grafo de Euler y de Hamilton.
            \item Es la sucesión de grados de un grafo de Hamilton y no de Euler.
            \item Es la sucesión de grados de un grafo de Euler y no de Hamilton.
        \end{enumerate}
        \item Un grafo regular de grado 7:
        \begin{enumerate}
            \item Tiene que tener al menos 8 vértices y un número impar de lados.
            \item Tiene que tener al menos 8 vértices pero puede tener un número impar o par de lados.
            \item Lo único que puedo afirmar sobre él es que tiene un número par de vértices.
        \end{enumerate}
    \end{enumerate}
\end{ejercicio}


\begin{ejercicio}\label{ej:1.32}
    Considera el grupo simétrico $S_4$ y el subgrupo suyo $H = \langle (1 2 3) \rangle$.
    \begin{enumerate}
        \item Construye el conjunto cociente $S_4/H$ de clases laterales por la izquierda $xH$.
        \item Para cada clase $xH$ denotamos $m(xH)$ al máximo común divisor de los órdenes de los elementos en $xH$. Considera el grafo $G$ con vértices las clases $xH$ y en el que hay un lado entre $xH$ e $yH$ si $m(xH)$ divide a $m(yH)$ o $m(yH)$ divide a $m(xH)$. Identifica el grafo $G$ dando la sucesión de grados de sus vértices y su matriz de adyacencia. ¿Es $G$ de Euler, de Hamilton o plano?
        \item Considera el subgrafo $G'$ obtenido a partir de $G$ eliminando la clase $1H$, ¿es $G'$ de Euler? En caso afirmativo aplica el algoritmo dado en clase para calcular un circuito de Euler.
    \end{enumerate}
\end{ejercicio}

\begin{ejercicio}\label{ej:1.33}
    Se considera el grupo $Q_2 = \langle x, y \mid x^4 = 1, x^2 = y^2, yx = x^{-1}y \rangle$ y el grafo $G$ cuyos vértices son los elementos de $Q_2$ y en el que, para cualquier $a \in Q_2$, hay un lado entre $a$ y $ax$ y también un lado entre $a$ y $ay$.
    \begin{enumerate}
        \item Comprueba que $G$ es un grafo regular dando la sucesión de grados de sus vértices y calcula su matriz de adyacencia.
        \item Razona si $G$ es un grafo de Hamilton o plano.
        \item Razona si $G$ es un grafo de Euler y, en caso afirmativo, aplica el algoritmo dado en clase para calcular un circuito de Euler.
    \end{enumerate}
\end{ejercicio}

\begin{ejercicio}\label{ej:1.34}
    Se considera el grupo $D_4 = \langle r, s \mid r^4 = 1, s^2 = 1, sr = r^{-1}s \rangle$ y el grafo $G$ cuyos vértices son los elementos de $D_4$ y en el que, para cualquier $a \in D_4$, hay un lado entre $a$ y $ar$ y también un lado entre $a$ y $as$.
    \begin{enumerate}
        \item Comprueba que $G$ es un grafo regular dando la sucesión de grados de sus vértices y calcula su matriz de adyacencia.
        \item Razona si $G$ es un grafo de Hamilton o plano.
        \item Razona si $G$ es un grafo de Euler y, en caso afirmativo, aplica el algoritmo dado en clase para calcular un circuito de Euler.
    \end{enumerate}
\end{ejercicio}


\begin{ejercicio}\label{ej:1.35}
    Se considera el grupo diédrico $D_5 = \langle r, s \mid r^5 = 1, s^2 = 1, sr = r^{-1}s \rangle$ y el grafo $G$ cuyos vértices son los elementos de $D_5$ y en el que, para cualquier $a \in D_5$, hay un lado entre $a$ y $ar$ y también un lado entre $a$ y $as$.
    \begin{enumerate}
        \item Calcula la sucesión de grados de $G$ y razona si $G$ es un grafo de Euler, de Hamilton o plano.
        \item Considera un nuevo grafo $G'$ obtenido añadiendo a $G$ un nuevo vértice adyacente a todos los de $G$. Razona si $G'$ es un grafo de Euler y, en caso afirmativo, aplica algún algoritmo dado en clase para calcular un circuito de Euler.
    \end{enumerate}
\end{ejercicio}

\begin{ejercicio}\label{ej:1.36}
    Razona cuál es la respuesta correcta en cada una de las siguientes cuestiones. Todos los grafos a los que se hace referencia son simples (es decir, no tienen lazos ni lados paralelos).
    \begin{enumerate}
        \item La matriz
        \[
            \begin{pmatrix}
                0 & 1 & 1 & 1 & 1 \\
                1 & 0 & 1 & 1 & 1 \\
                1 & 1 & 0 & 1 & 0 \\
                1 & 1 & 1 & 0 & 1 \\
                1 & 1 & 0 & 1 & 0
            \end{pmatrix}
        \]
        es la de adyacencia de un grafo que:
        \begin{enumerate}
            \item Es de Euler.
            \item No es de Hamilton.
            \item Es plano.
        \end{enumerate}
        \item Un grafo plano conexo regular de grado 8 con 23 caras:
        \begin{enumerate}
            \item No existe.
            \item Tiene 12 aristas.
            \item Tiene 9 vértices.
        \end{enumerate}
        \item Se tiene que:
        \begin{enumerate}
            \item Un grafo que es de Euler y de Hamilton siempre es plano.
            \item Un grafo que es plano y de Euler siempre es de Hamilton.
            \item Ninguna de las respuestas anteriores es cierta.
        \end{enumerate}
        \item Se tiene que:
        \begin{enumerate}
            \item La sucesión $5, 5, 4, 2, 2, 2$ es la sucesión gráfica de un grafo plano.
            \item La sucesión $5, 5, 4, 4, 4, 4$ es la sucesión gráfica de un grafo de Hamilton.
            \item La sucesión $5, 4, 4, 3, 3, 3$ es la sucesión gráfica de un grafo de Euler.
        \end{enumerate}
    \end{enumerate}
\end{ejercicio}

\begin{ejercicio}\label{ej:1.37}
    Considera el grupo simétrico $S_4$ y el subgrupo suyo $H = \langle (1 2 3) \rangle$.
    \begin{enumerate}
        \item Construye el conjunto cociente $\nicefrac{S_4}{\sim_H}$ de clases laterales por la derecha $Hx$, $x \in S_4$.
        \item Para cada clase $Hx$ denotamos $n(Hx)$ al mínimo común múltiplo de los órdenes de los elementos en $Hx$. Considera el grafo $G$ con vértices las clases $Hx$ y en el que hay un lado entre $Hx$ e $Hy$ si $n(Hx)$ divide a $n(Hy)$ o $n(Hy)$ divide a $n(Hx)$. Identifica el grafo $G$ dando la sucesión de grados de sus vértices y su matriz de adyacencia. ¿Es $G$ de Euler, de Hamilton o plano?
        \item Considera, si es posible, un subgrafo $G'$ de $G$ obtenido al suprimir una arista entre dos vértices de $G$ de grado impar. ¿Es $G'$ de Euler? ¿Hay un camino de Euler entre dos vértices de $G'$? En caso afirmativo aplica algún algoritmo dado en clase para calcular un circuito o camino de Euler en $G'$.
    \end{enumerate}
\end{ejercicio}

\begin{ejercicio}\label{ej:1.38}
    Se considera el grupo $A_4$ y su subgrupo $H = \langle (1 2)(3 4) \rangle$. Se considera el grafo $G$ con vértices las clases laterales por la izquierda de $H$ en $A_4$, $xH$, y en el que hay un lado entre $xH$ e $yH$ si $m(xH)$ divide a $m(yH)$ o $m(yH)$ divide a $m(xH)$, donde $m(Hx)$ denota el máximo común divisor de los órdenes de los elementos en $xH$.
    Razone cuál de las siguientes es la respuesta correcta:
    \begin{enumerate}
        \item $G$ es plano pero no es de Hamilton.
        \item $G$ no es plano y tiene dos vértices conectados por un camino de Euler.
        \item $G$ es de Hamilton pero no es de Euler.
    \end{enumerate}
\end{ejercicio}


\begin{ejercicio}\label{ej:1.39}
    Considera el grupo simétrico $S_4$ y el subgrupo suyo $H = \langle (1 2 3 4) \rangle$.
    \begin{enumerate}
        \item Construye el conjunto cociente $S_4/H$ de clases laterales por la izquierda $xH$. ¿Es $H \triangleleft S_4$?
        \item Para cada clase $xH$ denotamos $m(xH)$ al máximo común divisor de los órdenes de los elementos en $xH$. Considera el grafo $G$ con vértices las clases $xH$ y en el que hay un lado entre dos clases $xH$ e $yH$ si $m(xH) = m(yH)$. Identifica el grafo $G$ dando la sucesión de grados de sus vértices y su matriz de adyacencia. ¿Es $G$ de Euler, de Hamilton o plano?
        \item Considera el subgrafo $G'$ obtenido a partir de $G$ eliminando la clase $(1 3)H$. ¿Es $G'$ de Euler? En caso afirmativo aplica algún algoritmo dado en clase para calcular un circuito o camino de Euler en $G'$.
    \end{enumerate}
\end{ejercicio}


\begin{ejercicio}\label{ej:1.40}
    Razona cuál es la respuesta correcta en cada una de las siguientes cuestiones. Todos los grafos a los que se hace referencia son simples (es decir, no tienen lazos ni lados paralelos).
    \begin{enumerate}
        \item Se tiene que:
        \begin{enumerate}
            \item Hay un grafo conexo regular de grado 6 con 22 caras y 24 aristas.
            \item La sucesión $4, 4, 4, 3, 3$ es la sucesión gráfica de un grafo plano que tiene un camino de Euler entre dos vértices.
            \item Un grafo conexo y plano es de Euler si y solo si es de Hamilton.
        \end{enumerate}
        \item La matriz
        \[
            \begin{pmatrix}
                0 & 1 & 1 & 1 & 1 & 1 \\
                1 & 0 & 0 & 0 & 0 & 0 \\
                1 & 0 & 0 & 1 & 1 & 1 \\
                1 & 0 & 1 & 0 & 1 & 1 \\
                1 & 0 & 1 & 1 & 0 & 1 \\
                1 & 0 & 1 & 1 & 1 & 0
            \end{pmatrix}
        \]
        es la de adyacencia de un grafo:
        \begin{enumerate}
            \item Con 11 aristas y que es de Euler y de Hamilton.
            \item Que es conexo y plano pero no de Hamilton.
            \item Que no es de Hamilton ni plano ni de Euler.
        \end{enumerate}
    \end{enumerate}
\end{ejercicio}


\begin{ejercicio}\label{ej:1.41}
    Considera el grupo simétrico $S_4$ y el subgrupo suyo $H = \langle (1 3 4) \rangle$.
    \begin{enumerate}
        \item Construye el conjunto cociente $S_4/\sim_H$ de clases laterales por la derecha $Hx$, $x \in S_4$.
        \item Para cada clase $Hx$ denotamos $n(Hx)$ al mínimo común múltiplo de los órdenes de los elementos en $Hx$. Considera el grafo $G$ con vértices las clases $Hx$ y en el que hay un lado entre $Hx$ y $Hy$ si $n(Hx)$ divide a $n(Hy)$ o $n(Hy)$ divide a $n(Hx)$. Identifica el grafo $G$ dando la sucesión de grados de sus vértices y su matriz de adyacencia.
        \item ¿Hay alguna condición suficiente que asegure que $G$ es de Hamilton? ¿Y necesaria para ser plano? ¿Es $G$ de Euler, de Hamilton o plano?
        \item Considera el subgrafo $G'$ de $G$ obtenido al suprimir la arista entre las clases $H(2 3)$ y $H(2 4)$. ¿Es $G'$ de Hamilton, plano o de Euler? ¿Hay un camino de Euler entre dos vértices de $G'$? En caso afirmativo aplica algún algoritmo dado en clase para calcular un circuito o camino de Euler en $G'$.
    \end{enumerate}
\end{ejercicio}
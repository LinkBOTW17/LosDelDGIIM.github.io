\section{Combinatoria y Teoría de Grafos}

\begin{ejercicio}\label{ej:1.1}
    Diez personas están sentadas alrededor de una mesa circular. Cada persona estrecha la mano a todos los demás excepto a la persona sentada directamente enfrente de la mesa. Dibuja un grafo que modele la situación.
\end{ejercicio}

\begin{ejercicio}\label{ej:1.2}
    Seis hermanos (Alonso, Bernardo, Carlos, Daniel, Enrique y Fernando) tienen que emparejarse para compartir habitación en el próximo curso escolar. Cada uno de ellos ha elaborado una lista con los nombres de aquellos con los que quiere emparejarse:
    \begin{itemize}
        \item \ul{Lista de Alonso:} Daniel.
        \item \ul{Lista de Bernardo:} Alonso, Enrique.
        \item \ul{Lista de Carlos:} Daniel, Enrique.
        \item \ul{Lista de Daniel:} Carlos.
        \item \ul{Lista de Enrique:} Daniel, Bernardo, Fernando.
        \item \ul{Lista de Fernando:} Alonso, Bernardo.
    \end{itemize}
    Dibuja el grafo dirigido que modela esta situación.
\end{ejercicio}

\begin{ejercicio}\label{ej:1.3}
    Expresa en forma matricial los grafos de la Figura~\ref{fig:1.3}.
    \begin{figure}
        \centering
        \begin{subfigure}[b]{0.4\textwidth}
            \centering
            \begin{tikzpicture}
                % Nodos con posiciones relativas
                \node[draw, circle] (A) {A};
                \node[draw, circle, below left=of A] (B) {B};
                \node[draw, circle, below right=of A] (C) {C};
                \node[draw, circle, below=of B] (D) {D};
                \node[draw, circle, below=of C] (E) {E};
                
                % Aristas
                \draw (A) -- (B);
                \draw (A) -- (C);
                \draw (B) -- (C);
                \draw (B) -- (D);
                \draw (C) -- (E);
                \draw (D) -- (E);
            \end{tikzpicture}
            \caption{Grafo~\ref{fig:1.3a}}
            \label{fig:1.3a}
        \end{subfigure}
        \begin{subfigure}[b]{0.4\textwidth}
            \centering
            \begin{tikzpicture}
                % Nodos con posiciones relativas
                \node[draw, circle] (B) {B};
                \node[draw, circle, right=of B] (C) {C};
                \node[draw, circle, below left=of B] (A) {A};
                \node[draw, circle, below right=of C] (D) {D};
                \node[draw, circle, below right=of A] (F) {F};
                \node[draw, circle, right=of F] (E) {E};
                
                % Aristas: A-B-C-D-E-F
                \draw (A) -- (B);
                \draw (B) -- (C);
                \draw (C) -- (D);
                \draw (D) -- (E);
                \draw (E) -- (F);
                \draw (F) -- (A);
                \draw (D) -- (A);
            \end{tikzpicture}
            \caption{Grafo~\ref{fig:1.3b}}
            \label{fig:1.3b}
        \end{subfigure}
        \caption{Grafos para el ejercicio \ref{ej:1.3}.}
        \label{fig:1.3}
    \end{figure}
\end{ejercicio}

\begin{ejercicio}\label{ej:1.4}
    Sea $G$ un grafo completo con cuatro vértices. Construye todos sus subgrafos salvo isomorfismo.
\end{ejercicio}

\begin{ejercicio}\label{ej:1.5}
    
    
    \begin{figure}
        \centering
        \begin{subfigure}[b]{0.4\textwidth}
            \centering
            \begin{tikzpicture}
                % Nodos con posiciones relativas
                \node[draw, circle] (A) {A};
                \node[draw, circle, right=of A] (B) {B};
                \node[draw, circle, below=of B] (C) {C};
                \node[draw, circle, below=of A] (D) {D};
                
                % Aristas
                \draw (A) -- (B);
                \draw (A) -- (C);
                \draw (A) -- (D);
                \draw (C) -- (D);
                \draw (C) -- (B);
            \end{tikzpicture}
            \caption{Grafo~\ref{fig:1.5_1.a}.}
            \label{fig:1.5_1.a}
        \end{subfigure}
        \begin{subfigure}[b]{0.4\textwidth}
            \centering
            \begin{tikzpicture}
                % Nodos con posiciones relativas
                \node[draw, circle] (A) {A};
                \node[draw, circle, right=of A] (B) {B};
                \node[draw, circle, below=of B] (C) {C};
                \node[draw, circle, below=of A] (D) {D};
                
                % Aristas
                \draw (A) -- (B);
                \draw (A) -- (C);
                \draw (A) -- (D);
                \draw (C) -- (D);
                \draw (D) -- (B);
            \end{tikzpicture}
            \caption{Grafo~\ref{fig:1.5_1.b}.}
            \label{fig:1.5_1.b}
        \end{subfigure}
        \caption{Primeros grafos para el ejercicio \ref{ej:1.5}.}
        \label{fig:1.5_1}
    \end{figure}


    \begin{figure}
        \centering
        \begin{subfigure}[b]{0.4\textwidth}
            \centering
            \begin{tikzpicture}
                % Nodos con posiciones relativas
                \node[draw, circle] (A) {A};
                \node[draw, circle, below left=of A] (B) {B};
                \node[draw, circle, below right=of A] (C) {C};
                \node[draw, circle, below=of B] (D) {D};
                \node[draw, circle, below=of C] (E) {E};
                
                % Aristas
                \draw (A) -- (B);
                \draw (A) -- (C);
                \draw (B) -- (C);
                \draw (B) -- (D);
                \draw (C) -- (E);
                \draw (D) -- (E);
            \end{tikzpicture}
            \caption{Grafo~\ref{fig:1.5_2.a}.}
            \label{fig:1.5_2.a}
        \end{subfigure}
        \begin{subfigure}[b]{0.4\textwidth}
            \centering
            \begin{tikzpicture}
                % Nodos con posiciones relativas
                \node[draw, circle] (B) {B};
                \node[draw, circle, right=of B] (C) {C};
                \node[draw, circle, below left=of B] (A) {A};
                \node[draw, circle, below right=of C] (D) {D};
                \node[draw, circle, below right=of A] (F) {F};
                \node[draw, circle, right=of F] (E) {E};
                
                % Aristas: A-B-C-D-E-F
                \draw (A) -- (B);
                \draw (B) -- (C);
                \draw (C) -- (D);
                \draw (D) -- (E);
                \draw (E) -- (F);
                \draw (F) -- (A);
                \draw (D) -- (A);
            \end{tikzpicture}
            \caption{Grafo~\ref{fig:1.5_2.b}.}
            \label{fig:1.5_2.b}
        \end{subfigure}
        \caption{Segundos grafos para el ejercicio \ref{ej:1.5}.}
        \label{fig:1.5_2}
    \end{figure}
\end{ejercicio}


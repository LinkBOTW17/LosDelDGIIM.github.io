\section{Grupos: generalidades y ejemplos}

\begin{ejercicio}\label{ej:2.1}
    Describir explícitamente la tabla de multiplicar de los grupos $\mathbb{Z}^{\times}_n$ para $n = 4$, $n = 6$ y $n = 8$, donde por $\mathbb{Z}^{\times}_n$ denotamos al grupo de las unidades del anillo $\mathbb{Z}_n$.\\

    Sabemos que, fijado $n\in \bb{N}$, las unidades del anillo $\bb{Z}_n$ son:
    \[
        \cc{U}(\bb{Z}_n) = \bb{Z}^{\times}_n = \{a\in \bb{Z}_n \mid \mcd(a,n) = 1\}
    \]

    Describimos entonces a continuación las tablas de multiplicar de los grupos $\bb{Z}^{\times}_4$, $\bb{Z}^{\times}_6$ y $\bb{Z}^{\times}_8$.
    \begin{itemize}
        \item \ul{Para $n = 4$}:
        \begin{equation*}
            \begin{array}{c|cc}
                \cdot & 1 & 3 \\
                \hline
                1 & 1 & 3 \\
                3 & 3 & 1
            \end{array}
        \end{equation*}

        \item \ul{Para $n = 6$}:
        \begin{equation*}
            \begin{array}{c|cc}
                \cdot & 1 & 5 \\ \hline
                1 & 1 & 5 \\
                5 & 5 & 1
            \end{array}
        \end{equation*}

        \item \ul{Para $n = 8$}:
        \begin{equation*}
            \begin{array}{c|cccc}
                \cdot & 1 & 3 & 5 & 7 \\ \hline
                1 & 1 & 3 & 5 & 7 \\
                3 & 3 & 1 & 7 & 5 \\
                5 & 5 & 7 & 1 & 3 \\
                7 & 7 & 5 & 3 & 1
            \end{array}
        \end{equation*}
    \end{itemize}

\end{ejercicio}

\begin{ejercicio}\label{ej:2.2}
    Describir explícitamente la tabla de multiplicar de los grupos $\mathbb{Z}^{\times}_p$ para $p = 2$, $p = 3$, $p = 5$ y $p = 7$.
    \begin{itemize}
        \item \ul{Para $p = 2$}:
        \begin{equation*}
            \begin{array}{c|c}
                \cdot & 1 \\ \hline
                1 & 1
            \end{array}
        \end{equation*}

        \item \ul{Para $p = 3$}:
        \begin{equation*}
            \begin{array}{c|cc}
                \cdot & 1 & 2 \\ \hline
                1 & 1 & 2 \\
                2 & 2 & 1
            \end{array}
        \end{equation*}

        \item \ul{Para $p = 5$}:
        \begin{equation*}
            \begin{array}{c|cccc}
                \cdot & 1 & 2 & 3 & 4 \\ \hline
                1 & 1 & 2 & 3 & 4 \\
                2 & 2 & 4 & 1 & 3 \\
                3 & 3 & 1 & 4 & 2 \\
                4 & 4 & 3 & 2 & 1
            \end{array}
        \end{equation*}

        \item \ul{Para $p = 7$}:
        \begin{equation*}
            \begin{array}{c|ccccccc}
                \cdot & 1 & 2 & 3 & 4 & 5 & 6 \\ \hline
                1 & 1 & 2 & 3 & 4 & 5 & 6 \\
                2 & 2 & 4 & 6 & 1 & 3 & 5 \\
                3 & 3 & 6 & 2 & 5 & 1 & 4 \\
                4 & 4 & 1 & 5 & 2 & 6 & 3 \\
                5 & 5 & 3 & 1 & 6 & 4 & 2 \\
                6 & 6 & 5 & 4 & 3 & 2 & 1
            \end{array}
        \end{equation*}
    \end{itemize}
\end{ejercicio}

\begin{ejercicio}\label{ej:2.3}
    Calcular el inverso de $7$ en los grupos $\mathbb{Z}^{\times}_{11}$ y $\mathbb{Z}^{\times}_{37}$.

    Para calcular el inverso de un elemento $a$ en un grupo $\bb{Z}^{\times}_n$, basta con encontrar un elemento $b$ tal que $ab=1$ en $\bb{Z}_n$.
    \begin{itemize}
        \item \ul{Para $\bb{Z}^{\times}_{11}$}:
        \begin{equation*}
            7\cdot 8 = 56=1\Longrightarrow 7^{-1} = 8
        \end{equation*}

        \item \ul{Para $\bb{Z}^{\times}_{37}$}:
        \begin{equation*}
            7\cdot 16 = 112 = 1\Longrightarrow 7^{-1} = 16
        \end{equation*}
    \end{itemize}
\end{ejercicio}

\begin{ejercicio}\label{ej:2.4}
    Describir explícitamente los grupos $\mu_n$ (de raíces $n$-ésimas de la unidad) para $n = 3$, $n = 4$ y $n = 8$, dando su tabla de multiplicar.
    \begin{itemize}
        \item \ul{Para $n = 3$}:
        \begin{align*}
            \mu_3 &= \left\{1, \xi_3,\xi_3^2 \mid \xi_3 = \cos\left(\frac{2\pi}{3}\right) + i\sen\left(\frac{2\pi}{3}\right)\right\}
            =\\&= \left\{1, -\frac{1}{2} + i\frac{\sqrt{3}}{2}, -\frac{1}{2} - i\frac{\sqrt{3}}{2}\right\}
        \end{align*}
        \begin{equation*}
            \begin{array}{c|ccc}
                \cdot & 1 & \xi_3 & \xi_3^2 \\ \hline
                1 & 1 & \xi_3 & \xi_3^2 \\
                \xi_3 & \xi_3 & \xi_3^2 & 1 \\
            \end{array}
        \end{equation*}

        \item \ul{Para $n = 4$}:
        \begin{align*}
            \mu_4 &= \left\{1, \xi_4, \xi_4^2, \xi_4^3 \mid \xi_4 = \cos\left(\frac{\pi}{2}\right) + i\sen\left(\frac{\pi}{2}\right)\right\}
            =\\&= \left\{1, \xi_4, \xi_4^2, \xi_4^3 \mid \xi_4 = i\right\}
            = \left\{1, i, -1, -i\right\}
        \end{align*}
        \begin{equation*}
            \begin{array}{c|cccc}
                \cdot & 1 & i & -1 & -i \\ \hline
                1 & 1 & i & -1 & -i \\
                i & i & -1 & -i & 1 \\
                -1 & -1 & -i & 1 & i \\
                -i & -i & 1 & i & -1
            \end{array}
        \end{equation*}

        \item \ul{Para $n = 8$}:
        \begin{align*}
            \mu_8 &= \left\{1, \xi_8, \xi_8^2, \xi_8^3, \xi_8^4, \xi_8^5, \xi_8^6, \xi_8^7 \mid \xi_8 = \cos\left(\frac{\pi}{4}\right) + i\sen\left(\frac{\pi}{4}\right)\right\}
            =\\&= \left\{1, \xi_8, \xi_8^2, \xi_8^3, \xi_8^4, \xi_8^5, \xi_8^6, \xi_8^7 \mid \xi_8 = \frac{1}{\sqrt{2}} + i\frac{1}{\sqrt{2}}\right\}
            =\\&= \left\{1, \frac{1}{\sqrt{2}} + i\frac{1}{\sqrt{2}}, i, -\frac{1}{\sqrt{2}} + i\frac{1}{\sqrt{2}}, -1, -\frac{1}{\sqrt{2}} - i\frac{1}{\sqrt{2}}, -i, \frac{1}{\sqrt{2}} - i\frac{1}{\sqrt{2}}\right\}
        \end{align*}
        \begin{equation*}
            \begin{array}{c|cccccccc}
                \cdot & 1 & \xi_8 & \xi_8^2 & \xi_8^3 & \xi_8^4 & \xi_8^5 & \xi_8^6 & \xi_8^7 \\ \hline
                1 & 1 & \xi_8 & i & \xi_8^3 & -1 & \xi_8^5 & -i & \xi_8^7 \\
                \xi_8 & \xi_8 & i & \xi_8^3 & -1 & \xi_8^5 & -i & \xi_8^7 & 1 \\
                \xi_8^2 & i & \xi_8^3 & -1 & \xi_8^5 & -i & \xi_8^7 & 1 & \xi_8 \\
                \xi_8^3 & \xi_8^3 & -1 & \xi_8^5 & -i & \xi_8^7 & 1 & \xi_8 & i \\
                \xi_8^4 & -1 & \xi_8^5 & -i & \xi_8^7 & 1 & \xi_8 & i & \xi_8^3 \\
                \xi_8^5 & \xi_8^5 & -i & \xi_8^7 & 1 & \xi_8 & i & \xi_8^3 & -1 \\
                \xi_8^6 & -i & \xi_8^7 & 1 & \xi_8 & i & \xi_8^3 & -1 & \xi_8^5 \\
                \xi_8^7 & \xi_8^7 & 1 & \xi_8 & i & \xi_8^3 & -1 & \xi_8^5 & -i
            \end{array}
        \end{equation*}
    \end{itemize}
\end{ejercicio}

\begin{ejercicio}\label{ej:2.5}
    En el conjunto $\mathbb{Q}^{\times} := \{q \in \mathbb{Q} \mid q \neq 0\}$ de los números racionales no nulos, se considera la operación de división, dada por $(x, y) \mapsto \nicefrac{x}{y} = xy^{-1}$. ¿Nos da esta operación una estructura de grupo en $\mathbb{Q}^{\times}$?\\

    Veamos qué condiciones han de cumplirse para que se tenga la propiedad asociativa. Sean $a, b, c \in \mathbb{Q}^{\times}$, entonces:
    \begin{equation*}
        \frac{\nicefrac{a}{b}}{c} =  
        \frac{a}{\nicefrac{b}{c}}
        \iff \frac{a}{bc} = \frac{ac}{b}
        \iff ab = abc^2
        \iff 1=c^2
    \end{equation*}

    Por tanto, tomando por ejemplo $2,3,4\in\mathbb{Q}^{\times}$ no se tiene la propiedad asociativa, por lo que no se tiene un grupo.
\end{ejercicio}

\begin{ejercicio}\label{ej:2.6}
    Sea $G$ un grupo en el que $x^2 = 1$ para todo $x \in G$. Demostrar que el grupo $G$ es abeliano.\\

    Dados $x, y \in G$, se tiene que:
    \begin{align*}
        (xy)(xy) &= (xy)^2 = 1 \Longrightarrow (xy)^{-1} = xy\\
        (xy)(yx) &= x(yy)x = xy^2x = x1x = xx = x^2 = 1 \Longrightarrow (xy)^{-1} = yx
    \end{align*}

    Por tanto, como en un grupo se tiene la unicidad del inverso, se tiene que $xy = yx$ para todo $x, y \in G$, por lo que $G$ es abeliano.
\end{ejercicio}

\begin{ejercicio}\label{ej:2.7}
    Sea $G$ un grupo. Demostrar que son equivalentes:
    \begin{enumerate}
        \item $G$ es abeliano.
        \item $\forall x, y \in G$ se verifica que $(xy)^2 = x^2y^2$.
        \item $\forall x, y \in G$ se verifica que $(xy)^{-1} = x^{-1}y^{-1}$.
    \end{enumerate}
    \begin{proof}~
        \begin{description}
            \item[$1\Longrightarrow 2$)] Dados $x, y \in G$, se tiene que:
            \begin{equation*}
                (xy)^2 = xyxy \AstIg x^2y^2
            \end{equation*}
            donde en $(\ast)$ se ha usado que $G$ es abeliano.

            \item[$2\Longrightarrow 1$)] Dados $x, y \in G$, se tiene que:
            \begin{align*}
                \left(xy\right)^{2}
                &= (xy)(xy) = xyxy\\
                &\AstIg x^2y^2
            \end{align*}
            donde en $(\ast)$ se ha usado la hipótesis. Por la propiedad cancelativa, se tiene que:
            \begin{equation*}
                \cancel{x}yx\cancel{y} = x^{\cancel{2}}y^{\cancel{2}} \Longrightarrow xy = yx
            \end{equation*}

            Como se tiene para todo $x, y \in G$, entonces $G$ es abeliano.

            \item[$1\Longrightarrow 3$)] Dados $x, y \in G$, se tiene que:
            \begin{align*}
                (xy)^{-1}
                &= y^{-1}x^{-1} \AstIg x^{-1}y^{-1}
            \end{align*}
            donde en $(\ast)$ se ha usado que $G$ es abeliano.

            \item[$3\Longrightarrow 1$)] Dados $x, y \in G$, tenemos que:
            \begin{align*}
                (xy)^{-1} &\AstIg x^{-1}y^{-1} = (yx)^{-1}
                \Longrightarrow \left((xy)^{-1}\right)^{-1} = \left((yx)^{-1}\right)^{-1}
                \Longrightarrow xy = yx
            \end{align*}
            donde en $(\ast)$ se ha usado la hipótesis. Por tanto, como se tiene para todo $x, y \in G$, entonces $G$ es abeliano.
        \end{description}
    \end{proof}
\end{ejercicio}

\begin{ejercicio}\label{ej:2.8}
    Demostrar que si en un grupo $G$, $x, y \in G$ verifican que $xy = yx$ entonces, para todo $n \in \bb{N}\setminus \{0\}$, se tiene que $(xy)^n = x^ny^n$.\\

    Demostramos por inducción sobre $n$.
    \begin{itemize}
        \item \ul{Caso base}: $n = 1$.
        \begin{equation*}
            (xy)^1 = xy = yx = x^1y^1
        \end{equation*}

        \item \ul{Paso inductivo}: Supuesto cierto para $n$, veamos que se cumple para $n+1$.
        \begin{align*}
            (xy)^{n+1}
            &= (xy)^n(xy) = x^ny^nxy\\
            &= x^nxy^nx = x^{n+1}y^{n+1}
        \end{align*}
    \end{itemize}
    Por tanto, por inducción, se tiene que $(xy)^n = x^ny^n$ para todo $n \in \bb{N}\setminus \{0\}$.
\end{ejercicio}

\begin{ejercicio}\label{ej:2.9}
    Demostrar que el conjunto de las aplicaciones $f : \mathbb{R} \to \mathbb{R}$, tales que $f(x) = ax + b$ para algún $a,b\in \bb{R}$, $a\neq 0$, es un grupo con la composición como ley de composición.\\

    Definimos el conjunto siguiente:
    \begin{equation*}
        G = \{f : \mathbb{R} \to \mathbb{R} \mid \exists a,b\in \bb{R},~a\neq 0 \text{ tales que } f(x) = ax + b\ \forall x\in \bb{R}\}
    \end{equation*}

    En primer lugar, tomando $a=1$ y $b=0$, se tiene que $\Id_{\bb{R}}\in G$. Veamos que $(G,\circ, \Id_{\bb{R}})$ es un grupo.
    \begin{itemize}
        \item \ul{Asociatividad}: Se tiene de forma directa por serlo la composición de funciones.

        \item \ul{Elemento neutro}: Se tiene de forma directa.

        \item \ul{Elemento inverso}: Dado $f\in G$, entonces existen $a,b\in \bb{R}$, $a\neq 0$ tales que $f(x) = ax + b$.
        Entonces, definimos su elemento inverso como:
        \begin{equation*}
            f^{-1}(z) = a^{-1}\left(z - b\right)
        \end{equation*}

        Comprobémoslo (notemos que tan solo hace falta comprobar que $f\circ f^{-1} = \Id_{\bb{R}}$, puesto que en la definición no se impone $f^{-1}\circ f = \Id_{\bb{R}}$):
        \begin{align*}
            (f\circ f^{-1})(z) &= a\left(a^{-1}\left(z - b\right)\right) + b = z& \forall z\in \bb{R}
        \end{align*}

        Por tanto, para todo $f\in G$, existe $f^{-1}\in G$ tal que $f\circ f^{-1}=\Id_{\bb{R}}$.
    \end{itemize}
\end{ejercicio}

\begin{ejercicio}\label{ej:2.10}~
    \begin{enumerate}
        \item Demostrar que $|\GL_2(\mathbb{Z}_2)| = 6$, describiendo explícitamente todos los elementos que forman este grupo.\\
        
        Sea $A\in \GL_2(\bb{Z}_2)$:
        \begin{align*}
            A = \begin{pmatrix} a & b \\ c & d \end{pmatrix} \Longrightarrow |A| = ad - bc \neq 0\Longrightarrow ad \neq bc
        \end{align*}

        Por tanto, los elementos de $\GL_2(\bb{Z}_2)$ son:
        \begin{align*}
            A_1 &= \begin{pmatrix} 1 & 0 \\ 0 & 1 \end{pmatrix}, & A_2 &= \begin{pmatrix} 1 & 1 \\ 0 & 1 \end{pmatrix}, & A_3 &= \begin{pmatrix} 1 & 0 \\ 1 & 1 \end{pmatrix},\\
            A_4 &= \begin{pmatrix} 0 & 1 \\ 1 & 0 \end{pmatrix}, & A_5 &= \begin{pmatrix} 0 & 1 \\ 1 & 1 \end{pmatrix}, & A_6 &= \begin{pmatrix} 1 & 1 \\ 1 & 0 \end{pmatrix}
        \end{align*}
        \item Sea $\alpha = \begin{pmatrix} 0 & 1 \\ 1 & 1 \end{pmatrix}$ y $\beta = \begin{pmatrix} 0 & 1 \\ 1 & 0 \end{pmatrix}$. Demostrar que
        $$\GL_2(\mathbb{Z}_2) = \{1, \alpha, \alpha^2, \beta, \alpha\beta, \alpha^2\beta\}.$$

        Tenemos que:
        \begin{align*}
            1&= A_1, & \alpha &= A_5, & \alpha^2 &= A_6,& \beta &= A_4, & \alpha\beta &= A_3, & \alpha^2\beta &= A_2
        \end{align*}
        \item Escribir, utilizando la representación anterior, la tabla de multiplicar de $\GL_2(\mathbb{Z}_2)$.
        
        \begin{equation*}
            \begin{array}{c|cccccc}
                \cdot & 1 & \alpha & \alpha^2 & \beta & \alpha\beta & \alpha^2\beta \\ \hline
                1 & 1 & \alpha & \alpha^2 & \beta & \alpha\beta & \alpha^2\beta \\
                \alpha & \alpha & \alpha^2 & 1 & \alpha\beta & \alpha^2\beta & \beta \\
                \alpha^2 & \alpha^2 & 1 & \alpha & \alpha^2\beta & \beta & \alpha\beta \\
                \beta & \beta & \alpha^2\beta & \alpha\beta & 1 & \alpha^2 & \alpha \\
                \alpha\beta & \alpha\beta & \beta & \alpha^2\beta & \alpha & 1 & \alpha^2 \\
                \alpha^2\beta & \alpha^2\beta & \alpha\beta & \beta & \alpha^2 & \alpha & 1
            \end{array}
        \end{equation*}
    \end{enumerate}
\end{ejercicio}

\begin{ejercicio}\label{ej:2.11}
    Dar las tablas de grupo para los grupos $D_3$, $D_4$, $D_5$ y $D_6$.\\

    Recordamos que:
    \begin{equation*}
        D_n = \langle r, s \mid r^n = s^2 = 1, rs = sr^{-1}\rangle
    \end{equation*}

    \begin{itemize}
        \item \ul{Para $D_3$}:
        \begin{equation*}
            \begin{array}{c|cccccc}
                \cdot & 1 & r & r^2 & s & sr & sr^2 \\ \hline
                1 & 1 & r & r^2 & s & sr & sr^2 \\
                r & r & r^2 & 1 & sr^2 & s & sr \\
                r^2 & r^2 & 1 & r & sr & sr^2 & s \\
                s & s & sr & sr^2 & 1 & r & r^2 \\
                sr & sr & sr^2 & s & r^2 & 1 & r \\
                sr^2 & sr^2 & s & sr & r & r^2 & 1
            \end{array}
        \end{equation*}

        \item \ul{Para $D_4$}:
        \begin{equation*}
            \begin{array}{c|cccccccc}
                 \cdot & 1 & r & r^2 & r^3 & s & sr & sr^2 & sr^3 \\
                 \hline
                    1 & 1 & r & r^2 & r^3 & s & sr & sr^2 & sr^3 \\
                    r & r & r^2 & r^3 & 1 & sr^3 & s & sr & sr^2 \\
                    r^2 & r^2 & r^3 & 1 & r & sr^2 & sr^3 & s & sr \\
                    r^3 & r^3 & 1 & r & r^2 & sr & sr^2 & sr^3 & s \\
                    s & s & sr & sr^2 & sr^3 & 1 & r & r^2 & r^3 \\
                    sr & sr & sr^2 & sr^3 & s & r^3 & 1 & r & r^2 \\
                    sr^2 & sr^2 & sr^3 & s & sr & r^2 & r^3 & 1 & r \\
                    sr^3 & sr^3 & s & sr & sr^2 & r & r^2 & r^3 & 1 
            \end{array}
        \end{equation*}

        \item \ul{Para $D_5$}:
        \begin{equation*}
            \begin{array}{c|cccccccccc}
                \cdot & 1 & r & r^2 & r^3 & r^4 & s & sr & sr^2 & sr^3 & sr^4 \\ \hline
                1 & 1 & r & r^2 & r^3 & r^4 & s & sr & sr^2 & sr^3 & sr^4 \\
                r & r & r^2 & r^3 & r^4 & 1 & sr^4 & s & sr & sr^2 & sr^3 \\
                r^2 & r^2 & r^3 & r^4 & 1 & r & sr^3 & sr^4 & s & sr & sr^2 \\
                r^3 & r^3 & r^4 & 1 & r & r^2 & sr^2 & sr^3 & sr^4 & s & sr \\
                r^4 & r^4 & 1 & r & r^2 & r^3 & sr & sr^2 & sr^3 & sr^4 & s \\
                s & s & sr & sr^2 & sr^3 & sr^4 & 1 & r & r^2 & r^3 & r^4 \\
                sr & sr & sr^2 & sr^3 & sr^4 & s & r^4 & 1 & r & r^2 & r^3 \\
                sr^2 & sr^2 & sr^3 & sr^4 & s & sr & r^3 & r^4 & 1 & r & r^2 \\
                sr^3 & sr^3 & sr^4 & s & sr & sr^2 & r^2 & r^3 & r^4 & 1 & r \\
                sr^4 & sr^4 & s & sr & sr^2 & sr^3 & r & r^2 & r^3 & r^4 & 1
            \end{array}
        \end{equation*}

        \item \ul{Para $D_6$}:
        \begin{equation*}
            \begin{array}{c|cccccccccccc}
                \cdot & 1 & r & r^2 & r^3 & r^4 & r^5 & s & sr & sr^2 & sr^3 & sr^4 & sr^5 \\ \hline
                1 & 1 & r & r^2 & r^3 & r^4 & r^5 & s & sr & sr^2 & sr^3 & sr^4 & sr^5 \\
                r & r & r^2 & r^3 & r^4 & r^5 & 1 & sr^5 & s & sr & sr^2 & sr^3 & sr^4 \\
                r^2 & r^2 & r^3 & r^4 & r^5 & 1 & r & sr^4 & sr^5 & s & sr & sr^2 & sr^3 \\
                r^3 & r^3 & r^4 & r^5 & 1 & r & r^2 & sr^3 & sr^4 & sr^5 & s & sr & sr^2 \\
                r^4 & r^4 & r^5 & 1 & r & r^2 & r^3 & sr^2 & sr^3 & sr^4 & sr^5 & s & sr \\
                r^5 & r^5 & 1 & r & r^2 & r^3 & r^4 & sr & sr^2 & sr^3 & sr^4 & sr^5 & s \\
                s & s & sr & sr^2 & sr^3 & sr^4 & sr^5 & 1 & r & r^2 & r^3 & r^4 & r^5 \\
                sr & sr & sr^2 & sr^3 & sr^4 & sr^5 & s & r^5 & 1 & r & r^2 & r^3 & r^4 \\
                sr^2 & sr^2 & sr^3 & sr^4 & sr^5 & s & sr & r^4 & r^5 & 1 & r & r^2 & r^3 \\
                sr^3 & sr^3 & sr^4 & sr^5 & s & sr & sr^2 & r^3 & r^4 & r^5 & 1 & r & r^2 \\
                sr^4 & sr^4 & sr^5 & s & sr & sr^2 & sr^3 & r^2 & r^3 & r^4 & r^5 & 1 & r \\
                sr^5 & sr^5 & s & sr & sr^2 & sr^3 & sr^4 & r & r^2 & r^3 & r^4 & r^5 & 1
            \end{array}
        \end{equation*}

    \end{itemize}
\end{ejercicio}

\begin{ejercicio}\label{ej:2.12}
    Demostrar que el conjunto de rotaciones respecto al origen del plano euclídeo junto con el conjunto de simetrías respecto a las rectas que pasan por el origen, es un grupo.\\

    Denotamos por $G$ al conjunto de rotaciones respecto al origen del plano euclídeo junto con el conjunto de simetrías respecto a las rectas que pasan por el origen. Notemos que no se trata de ningún grupo diédrico:
    \begin{equation*}
        D_n\subsetneq G\qquad \forall n\in\bb{N}
    \end{equation*}

    Además, $\Id_{\bb{R}^2}\in G$. Veamos que $(G,\circ, \Id_{\bb{R}^2})$ es un grupo.
    \begin{itemize}
        \item \ul{Asociatividad}: Se tiene de forma directa por serlo la composición de funciones.

        \item \ul{Elemento neutro}: Se tiene de forma directa.

        \item \ul{Elemento inverso}: Dado $f\in G$, veamos que existe $f^{-1}\in G$ tal que $f\circ f^{-1} = \Id_{\bb{R}^2}$.
        \begin{itemize}
            \item Si $f$ es una rotación de ángulo $\theta$ respecto al origen, entonces $f^{-1}$ es la rotación de ángulo $-\theta$ respecto al origen.
            \item Si $f$ es una simetría respecto a una recta que pasa por el origen, entonces $f^{-1}$ es la misma simetría.
        \end{itemize}
        En ambos casos, se tiene que $f\circ f^{-1} = \Id_{\bb{R}^2}$.
    \end{itemize}

    Por tanto, $(G,\circ, \Id_{\bb{R}^2})$ es un grupo.
\end{ejercicio}

\begin{ejercicio}\label{ej:2.13}
    Sea $G$ un grupo y sean $a, b \in G$ tales que $ba = ab^k$, $a^n = 1 = b^m$ con $n, m> 0$.
    \begin{enumerate}
        \item Demostrar que para todo $i = 0, \ldots, m - 1$ se verifica $b^ia = ab^{ik}$.
        
        Demostramos para todo $i\in \bb{N}$ por inducción sobre $i$.
        \begin{itemize}
            \item  \ul{Caso base}: $i = 0$.
            \begin{equation*}
                b^0a = a = ab^0
            \end{equation*}

            \item \ul{Caso base}: $i = 1$.
            \begin{equation*}
                b^1a = ba = ab^k = ab^{1\cdot k}
            \end{equation*}

            \item \ul{Paso inductivo}: Supuesto cierto para $i$, veamos que se cumple para $i+1$.
            \begin{align*}
                b^{i+1}a
                &= bb^ia = bab^{ik} = ab^kb^{ik} = ab^{k(i+1)}
            \end{align*}
        \end{itemize}
        \item Demostrar que para todo $j = 0, \ldots, n - 1$ se verifica $ba^j = a^jb^{k^j}$.
        
        Demostramos para todo $j\in \bb{N}$ por inducción sobre $j$.
        \begin{itemize}
            \item  \ul{Caso base}: $j = 0$.
            \begin{equation*}
                ba^0 = b = a^0b^{k^0}
            \end{equation*}

            \item \ul{Caso base}: $j = 1$.
            \begin{equation*}
                ba = ab^k = a^1b^{k^1}
            \end{equation*}

            \item \ul{Paso inductivo}: Supuesto cierto para $j$, veamos que se cumple para $j+1$.
            \begin{align*}
                ba^{j+1}
                &= ba^ja = a^jb^{k^j}a \AstIg a^jab^{k^jk}
                = a^{j+1}b^{k^{j+1}}
            \end{align*}
            donde en $(\ast)$ se ha usado el apartado anterior.
        \end{itemize}
        \item Demostrar que para todo $i = 0, \ldots, m - 1$ y todo $j = 0, \ldots, n - 1$ se verifica $b^ia^j = a^jb^{ik^j}$.\\
        Fijado $i\in \bb{N}$, demostramos por inducción sobre $j$.
        \begin{itemize}
            \item  \ul{Caso base}: $j = 0$.
            \begin{equation*}
                b^ia^0 = b^i = a^0b^{ik^0}
            \end{equation*}

            \item \ul{Caso base}: $j = 1$.
            \begin{equation*}
                b^ia = ab^{ik} = a^1b^{ik^1}
            \end{equation*}

            \item \ul{Paso inductivo}: Supuesto cierto para $j$, veamos que se cumple para $j+1$.
            \begin{align*}
                b^ia^{j+1}
                &= b^ia^ja = a^jb^{ik^j}a \AstIg a^jab^{ik^jk}
                = a^{j+1}b^{ik^{j+1}}
            \end{align*}
            donde en $(\ast)$ se ha usado el apartado anterior.
        \end{itemize}
        Por tanto, se tiene para todo $i, j\in \bb{N}$.
        \item Demostrar que todo elemento de $\langle a, b \rangle$ puede escribirse como $a^rb^s$ cumpliendo $0 \leq r < n$, $0 \leq s < m$.\\
        
        Dado $x\in \langle a, b \rangle$, entonces $x$ es producto de elementos de $\{a,b,a^{-1},b^{-1}\}$. Como $a^n = 1 = b^m$, entonces $a^{-1} = a^{n-1}$ y $b^{-1} = b^{m-1}$. Por tanto, se tiene que $x$ es producto de elementos de $\{a, b\}$. Usando el apartado anterior, podemos ``llevar'' los $a$'s a la izquierda y los $b$'s a la derecha, obteniendo lo siguiente:
        \begin{equation*}
            x = a^{r'}b^{s'}\qquad r', s'\in \bb{N}\cup \{0\}
        \end{equation*}

        Supuesto $r'\geq n$, sea $r = r'\mod n$ ($r'=nk+r$) y se tiene que:
        \begin{equation*}
            a^{r'} = a^{nk + r} = (a^{n})^k \cdot a^r = a^r
        \end{equation*}
        Además, se cumple que $0\leq r < n$. Análogamente, supuesto $s'\geq m$, sea $s = s'\mod m$ ($s'=mk+s$) y se tiene que:
        \begin{equation*}
            b^{s'} = b^{mk + s} = (b^{m})^k \cdot b^s = b^s
        \end{equation*}
        Además, se cumple que $0\leq s < m$. Por tanto:
        \begin{equation*}
            x = a^{r'}b^{s'} = a^rb^s\qquad 0\leq r < n,\ 0\leq s < m
        \end{equation*}
    \end{enumerate}
    \begin{observacion}
        Notemos que $D_n$ es un caso particular de este grupo, donde:
        \begin{align*}
            a=r,\qquad b=s,\qquad k=n-1,\qquad m=2,\qquad n=n
        \end{align*}
    \end{observacion}
\end{ejercicio}

\begin{ejercicio}\label{ej:2.14}
    Sean $s_1, s_2 \in S_7$ las permutaciones dadas por
    \begin{align*}
        s_2 &= \begin{pmatrix} 1 & 2 & 3 & 4 & 5 & 6 & 7 \\ 5 & 7 & 6 & 2 & 1 & 4 & 3 \end{pmatrix}, &
        s_1 &= \begin{pmatrix} 1 & 2 & 3 & 4 & 5 & 6 & 7 \\ 3 & 2 & 4 & 5 & 1 & 7 & 6 \end{pmatrix}.
    \end{align*}
    Calcular los productos $s_1s_2$, $s_2s_1$ y $s_2^2$, y su representación como producto de ciclos disjuntos.\\

    En notación matricial, se tiene que:
    \begin{align*}
        s_1s_2 &= \begin{pmatrix} 1 & 2 & 3 & 4 & 5 & 6 & 7 \\ 1 & 6 & 7 & 2 & 3 & 5 & 4 \end{pmatrix}\\
        s_2s_1 &= \begin{pmatrix} 1 & 2 & 3 & 4 & 5 & 6 & 7 \\ 6 & 7 & 2 & 1 & 5 & 3 & 4 \end{pmatrix}\\
        s_2^2 &= \begin{pmatrix} 1 & 2 & 3 & 4 & 5 & 6 & 7 \\ 1 & 3 & 4 & 7 & 5 & 2 & 6 \end{pmatrix}
    \end{align*}

    Descomponiendo en ciclos disjuntos, se tiene que:
    \begin{align*}
        s_2 &= (1\ 5)(2\ 7\ 3\ 6\ 4)\\
        s_1 &= (1\ 3\ 4\ 5)(6\ 7)\\
        s_1s_2 &= (2\ 6\ 5\ 3\ 7\ 4)\\
        s_2s_1 &= (1\ 6\ 3\ 2\ 7\ 4)\\
        s_2^2 &= (2\ 3\ 4\ 7\ 6)
    \end{align*}
\end{ejercicio}

\begin{ejercicio}\label{ej:2.15}
    Dadas las permutaciones
    \begin{align*}
        p_1 &= (1\ 3\ 2\ 8\ 5\ 9)(2\ 6\ 3), &
        p_2 &= (1\ 3\ 6)(2\ 5\ 3)(1\ 9\ 2\ 8\ 5),
    \end{align*}
    hallar la descomposición de la permutación producto $p_1p_2$ como producto de ciclos disjuntos.
\end{ejercicio}

\begin{ejercicio}\label{ej:2.16}
    Sean $s_1, s_2, p_1$ y $p_2$ las permutaciones dadas en los ejercicios anteriores.
    \begin{enumerate}
        \item Descomponer la permutación $s_1s_2s_1s_2$ como producto de ciclos disjuntos.
        \item Expresar matricialmente la permutación $p_3 = p_2p_1p_2$ y obtener su descomposición como ciclos disjuntos.
        \item Descomponer la permutación $s_2p_2$ como producto de ciclos disjuntos y expresarla matricialmente.
    \end{enumerate}
    \begin{observacion}
        Aquí tratamos a $S_7$ como un subgrupo de $S_9$, donde consideramos cada permutación del conjunto $\{1, 2, 3, 4, 5, 6, 7\}$ como una permutación del conjunto $\{1, \ldots, 9\}$ que deja fijos a los elementos $8$ y $9$.
    \end{observacion}
\end{ejercicio}

\begin{ejercicio}\label{ej:2.17}
    Sean $s_1, s_2, p_1$ y $p_2$ las permutaciones dadas en los ejercicios anteriores.
    \begin{enumerate}
        \item Calcular el orden de la permutación producto $s_1s_2$. ¿Coincide dicho orden con el producto de los órdenes de $s_1$ y $s_2$?
        \item Calcular el orden de $s_1(s_2)^{-1}(s_1)^{-1}$.
        \item Calcular la permutación $(s_1)^{-1}$, y expresarla como producto de ciclos disjuntos.
        \item Calcular la permutación $(p_1)^{-1}$ y expresarla matricialmente.
        \item Calcular la permutación $p_2(s_2)^2(p_1)^{-1}$. ¿Cuál es su orden?
    \end{enumerate}
\end{ejercicio}

\begin{ejercicio}\label{ej:2.18}
    Sean $s_1, s_2, p_1$ y $p_2$ las permutaciones dadas anteriormente. Sean también $s_3 = (2\ 4\ 6)$ y $s_4 = (1\ 2\ 7)(2\ 4\ 6\ 1)(5\ 3)$. ¿Cuál es la paridad de las permutaciones $s_1$, $s_4p_1p_2$ y $p_2s_3$?
\end{ejercicio}

\begin{ejercicio}\label{ej:2.19}
    En el grupo $S_3$, se consideran las permutaciones $\sigma = (1\ 2\ 3)$ y $\tau = (1\ 2)$.
    \begin{enumerate}
        \item Demostrar que
        $$S_3 = \{1, \sigma, \sigma^2, \tau, \sigma\tau, \sigma^2\tau\}.$$
        \item Reescribir la tabla de multiplicar de $S_3$ empleando la anterior expresión de los elementos de $S_3$.
        \item Probar que
        $$\sigma^3 = 1, \quad \tau^2 = 1, \quad \tau\sigma = \sigma^2\tau.$$
        \item Observar que es posible escribir toda la tabla de multiplicar de $S_3$ usando simplemente la descripción anterior y las relaciones anteriores.
    \end{enumerate}
\end{ejercicio}

\begin{ejercicio}\label{ej:2.20}
    Describir los diferentes ciclos del grupo $S_4$. Expresar todos los elementos de $S_4$ como producto de ciclos disjuntos.
\end{ejercicio}

\begin{ejercicio}\label{ej:2.21}
    Demostrar que el conjunto de transposiciones
    $$\{(1, 2), (2, 3), \ldots, (n - 1, n)\}$$
    genera al grupo simétrico $S_n$.
\end{ejercicio}

\begin{ejercicio}\label{ej:2.22}
    Demostrar que el conjunto $\{(1, 2, \ldots, n), (1, 2)\}$ genera al grupo simétrico $S_n$.
\end{ejercicio}

\begin{ejercicio}\label{ej:2.23}
    Demostrar que para cualquier permutación $\alpha \in S_n$ se verifica que $s(\alpha) = s(\alpha^{-1})$, donde $s$ denota la signatura, o paridad, de una permutación.
\end{ejercicio}

\begin{ejercicio}\label{ej:2.24}
    Demostrar que si $(x_1x_2 \cdots x_r) \in S_n$ es un ciclo de longitud $r$, entonces
    $$s(x_1x_2 \cdots x_r) = (-1)^{r - 1}.$$
\end{ejercicio}

\begin{ejercicio}\label{ej:2.25}
    Encontrar un isomorfismo $\mu_2 \cong \mathbb{Z}^{\times}_3$.
\end{ejercicio}

\begin{ejercicio}\label{ej:2.26}~
    \begin{enumerate}
        \item Demostrar que la aplicación
        \begin{align*}
            1 &\mapsto 1, & -1 &\mapsto 4, & i &\mapsto 2, & -i &\mapsto 3,
        \end{align*}
        da un isomorfismo entre el grupo $\mu_4$ de las raíces cuárticas de la unidad y el grupo $\mathbb{Z}^{\times}_5$ de las unidades en $\mathbb{Z}_5$.
        \item Encontrar otro isomorfismo entre estos dos grupos que sea distinto del anterior.
    \end{enumerate}
\end{ejercicio}

\begin{ejercicio}\label{ej:2.27}
    Encontrar un isomorfismo $\mu_2 \times \mu_2 \cong \mathbb{Z}^{\times}_8$.
\end{ejercicio}

\begin{ejercicio}\label{ej:2.28}
    Demostrar, haciendo uso de las representaciones conocidas, que $D_3 \cong S_3 \cong \GL_2(\mathbb{Z}_2)$.
\end{ejercicio}

\begin{ejercicio}\label{ej:2.29}
    Sea $K$ un cuerpo y considérese la operación binaria
    \Func{\otimes}{K \times K}{K}{(a, b)}{a \otimes b = a + b - ab.}
    Demostrar que $(K - \{1\}, \otimes)$ es un grupo isomorfo al grupo multiplicativo $K^{\ast}$.
\end{ejercicio}

\begin{ejercicio}\label{ej:2.30}~
    \begin{enumerate}
        \item Probar que si $f : G \cong G'$ es un isomorfismo de grupos, entonces $o(a) = o(f(a))$, para todo elemento $a \in G$.
        \item Listar los órdenes de los diferentes elementos del grupo $Q_2$ y del grupo $D_4$ y concluir que $D_4$ y $Q_2$ no son isomorfos.
    \end{enumerate}
\end{ejercicio}

\begin{ejercicio}\label{ej:2.31}
    Calcular el orden de:
    \begin{enumerate}
        \item la permutación $\sigma = (1\ 8\ 10\ 4\ 5\ 9)(2\ 6\ 3) \in S_{15}$.
        \item cada elemento del grupo $\mathbb{Z}^{\times}_{11}$.
    \end{enumerate}
\end{ejercicio}

\begin{ejercicio}\label{ej:2.32}
    Demostrar que un grupo generado por dos elementos distintos de orden dos, que conmutan entre sí, consiste del $1$, de esos elementos y de su producto y es isomorfo al grupo de Klein.
\end{ejercicio}

\begin{ejercicio}\label{ej:2.33}
    Sea $G$ un grupo y sean $a, b \in G$.
    \begin{enumerate}
        \item Demostrar que $o(b) = o(aba^{-1})$ (un elemento y su conjugado tienen el mismo orden).
        \item Demostrar que $o(ba) = o(ab)$.
    \end{enumerate}
\end{ejercicio}

\begin{ejercicio}\label{ej:2.34}
    Sea $G$ un grupo y sean $a, b \in G$, $a \neq 1 \neq b$, tales que $a^2 = 1$ y $ab^2 = b^3a$. Demostrar que $o(a) = 2$ y que $o(b) = 5$.
\end{ejercicio}

\begin{ejercicio}\label{ej:2.35}
    Sea $f : G \to H$ un homomorfismo de grupos.
    \begin{enumerate}
        \item $f(x^n) = f(x)^n$ $\forall n \in \mathbb{Z}$.
        \item Si $f$ es un isomorfismo entonces $G$ y $H$ tienen el mismo número de elementos de orden $n$. ¿Es cierto el resultado si $f$ es sólo un homomorfismo?
        \item Si $f$ es un isomorfismo entonces $G$ es abeliano $\Leftrightarrow$ $H$ es abeliano.
    \end{enumerate}
\end{ejercicio}

\begin{ejercicio}\label{ej:2.36}~
    \begin{enumerate}
        \item Demostrar que los grupos multiplicativos $\mathbb{R}^{\ast}$ (de los reales no nulos) y $\mathbb{C}^{\ast}$ (de los complejos no nulos) no son isomorfos.
        \item Demostrar que los grupos aditivos $\mathbb{Z}$ y $\mathbb{Q}$ no son isomorfos.
    \end{enumerate}
\end{ejercicio}

\begin{ejercicio}\label{ej:2.37}
    Sea $G$ un grupo. Demostrar:
    \begin{enumerate}
        \item $G$ es abeliano $\iff$ la aplicación $f : G \to G$ dada por $f(x) = x^{-1}$ es un homomorfismo de grupos.
        \item $G$ es abeliano $\iff$ la aplicación $f : G \to G$ dada por $f(x) = x^2$ es un homomorfismo de grupos.
    \end{enumerate}
\end{ejercicio}


\begin{ejercicio}\label{ej:2.38}
    Si $G$ es un grupo cíclico demostrar que cualquier homomorfismo de grupos $f : G \to H$ está determinado por la imagen del generador.
\end{ejercicio}

\begin{ejercicio}\label{ej:2.39}
    Demostrar que no existe ningún cuerpo $K$ tal que sus grupos aditivo $(K, +)$ y $(K^{\ast}, \cdot)$ sean isomorfos.
\end{ejercicio}
\section{Grupos: generalidades y ejemplos}

\begin{ejercicio}\label{ej:2.1}
    Describir explícitamente la tabla de multiplicar de los grupos $\mathbb{Z}^{\times}_n$ para $n = 4$, $n = 6$ y $n = 8$, donde por $\mathbb{Z}^{\times}_n$ denotamos al grupo de las unidades del anillo $\mathbb{Z}_n$.\\

    Sabemos que, fijado $n\in \bb{N}$, las unidades del anillo $\bb{Z}_n$ son:
    \[
        \cc{U}(\bb{Z}_n) = \bb{Z}^{\times}_n = \{a\in \bb{Z}_n \mid \mcd(a,n) = 1\}
    \]

    Describimos entonces a continuación las tablas de multiplicar de los grupos $\bb{Z}^{\times}_4$, $\bb{Z}^{\times}_6$ y $\bb{Z}^{\times}_8$.
    \begin{itemize}
        \item \ul{Para $n = 4$}:
        \begin{equation*}
            \begin{array}{c|cc}
                \cdot & 1 & 3 \\
                \hline
                1 & 1 & 3 \\
                3 & 3 & 1
            \end{array}
        \end{equation*}

        \item \ul{Para $n = 6$}:
        \begin{equation*}
            \begin{array}{c|cc}
                \cdot & 1 & 5 \\ \hline
                1 & 1 & 5 \\
                5 & 5 & 1
            \end{array}
        \end{equation*}

        \item \ul{Para $n = 8$}:
        \begin{equation*}
            \begin{array}{c|cccc}
                \cdot & 1 & 3 & 5 & 7 \\ \hline
                1 & 1 & 3 & 5 & 7 \\
                3 & 3 & 1 & 7 & 5 \\
                5 & 5 & 7 & 1 & 3 \\
                7 & 7 & 5 & 3 & 1
            \end{array}
        \end{equation*}
    \end{itemize}

\end{ejercicio}

\begin{ejercicio}\label{ej:2.2}
    Describir explícitamente la tabla de multiplicar de los grupos $\mathbb{Z}^{\times}_p$ para $p = 2$, $p = 3$, $p = 5$ y $p = 7$.
    \begin{itemize}
        \item \ul{Para $p = 2$}:
        \begin{equation*}
            \begin{array}{c|c}
                \cdot & 1 \\ \hline
                1 & 1
            \end{array}
        \end{equation*}

        \item \ul{Para $p = 3$}:
        \begin{equation*}
            \begin{array}{c|cc}
                \cdot & 1 & 2 \\ \hline
                1 & 1 & 2 \\
                2 & 2 & 1
            \end{array}
        \end{equation*}

        \item \ul{Para $p = 5$}:
        \begin{equation*}
            \begin{array}{c|cccc}
                \cdot & 1 & 2 & 3 & 4 \\ \hline
                1 & 1 & 2 & 3 & 4 \\
                2 & 2 & 4 & 1 & 3 \\
                3 & 3 & 1 & 4 & 2 \\
                4 & 4 & 3 & 2 & 1
            \end{array}
        \end{equation*}

        \item \ul{Para $p = 7$}:
        \begin{equation*}
            \begin{array}{c|ccccccc}
                \cdot & 1 & 2 & 3 & 4 & 5 & 6 \\ \hline
                1 & 1 & 2 & 3 & 4 & 5 & 6 \\
                2 & 2 & 4 & 6 & 1 & 3 & 5 \\
                3 & 3 & 6 & 2 & 5 & 1 & 4 \\
                4 & 4 & 1 & 5 & 2 & 6 & 3 \\
                5 & 5 & 3 & 1 & 6 & 4 & 2 \\
                6 & 6 & 5 & 4 & 3 & 2 & 1
            \end{array}
        \end{equation*}
    \end{itemize}
\end{ejercicio}

\begin{ejercicio}\label{ej:2.3}
    Calcular el inverso de $7$ en los grupos $\mathbb{Z}^{\times}_{11}$ y $\mathbb{Z}^{\times}_{37}$.

    Para calcular el inverso de un elemento $a$ en un grupo $\bb{Z}^{\times}_n$, basta con encontrar un elemento $b$ tal que $ab=1$ en $\bb{Z}_n$.
    \begin{itemize}
        \item \ul{Para $\bb{Z}^{\times}_{11}$}:
        \begin{equation*}
            7\cdot 8 = 56=1\Longrightarrow 7^{-1} = 8
        \end{equation*}

        \item \ul{Para $\bb{Z}^{\times}_{37}$}:
        \begin{equation*}
            7\cdot 16 = 112 = 1\Longrightarrow 7^{-1} = 16
        \end{equation*}
    \end{itemize}
\end{ejercicio}

\begin{ejercicio}\label{ej:2.4}
    Describir explícitamente los grupos $\mu_n$ (de raíces $n$-ésimas de la unidad) para $n = 3$, $n = 4$ y $n = 8$, dando su tabla de multiplicar.
    \begin{itemize}
        \item \ul{Para $n = 3$}:
        \begin{align*}
            \mu_3 &= \left\{1, \xi_3,\xi_3^2 \mid \xi_3 = \cos\left(\frac{2\pi}{3}\right) + i\sen\left(\frac{2\pi}{3}\right)\right\}
            =\\&= \left\{1, -\frac{1}{2} + i\frac{\sqrt{3}}{2}, -\frac{1}{2} - i\frac{\sqrt{3}}{2}\right\}
        \end{align*}
        \begin{equation*}
            \begin{array}{c|ccc}
                \cdot & 1 & \xi_3 & \xi_3^2 \\ \hline
                1 & 1 & \xi_3 & \xi_3^2 \\
                \xi_3 & \xi_3 & \xi_3^2 & 1 \\
                \xi_3^2 & \xi_3^2 & 1 & \xi_3 \\
            \end{array}
        \end{equation*}

        \item \ul{Para $n = 4$}:
        \begin{align*}
            \mu_4 &= \left\{1, \xi_4, \xi_4^2, \xi_4^3 \mid \xi_4 = \cos\left(\frac{\pi}{2}\right) + i\sen\left(\frac{\pi}{2}\right)\right\}
            =\\&= \left\{1, \xi_4, \xi_4^2, \xi_4^3 \mid \xi_4 = i\right\}
            = \left\{1, i, -1, -i\right\}
        \end{align*}
        \begin{equation*}
            \begin{array}{c|cccc}
                \cdot & 1 & i & -1 & -i \\ \hline
                1 & 1 & i & -1 & -i \\
                i & i & -1 & -i & 1 \\
                -1 & -1 & -i & 1 & i \\
                -i & -i & 1 & i & -1
            \end{array}
        \end{equation*}

        \item \ul{Para $n = 8$}:
        \begin{align*}
            \mu_8 &= \left\{1, \xi_8, \xi_8^2, \xi_8^3, \xi_8^4, \xi_8^5, \xi_8^6, \xi_8^7 \mid \xi_8 = \cos\left(\frac{\pi}{4}\right) + i\sen\left(\frac{\pi}{4}\right)\right\}
            =\\&= \left\{1, \xi_8, \xi_8^2, \xi_8^3, \xi_8^4, \xi_8^5, \xi_8^6, \xi_8^7 \mid \xi_8 = \frac{1}{\sqrt{2}} + i\frac{1}{\sqrt{2}}\right\}
            =\\&= \left\{1, \frac{1}{\sqrt{2}} + i\frac{1}{\sqrt{2}}, i, -\frac{1}{\sqrt{2}} + i\frac{1}{\sqrt{2}}, -1, -\frac{1}{\sqrt{2}} - i\frac{1}{\sqrt{2}}, -i, \frac{1}{\sqrt{2}} - i\frac{1}{\sqrt{2}}\right\}
        \end{align*}
        \begin{equation*}
            \begin{array}{c|cccccccc}
                \cdot & 1 & \xi_8 & \xi_8^2 & \xi_8^3 & \xi_8^4 & \xi_8^5 & \xi_8^6 & \xi_8^7 \\ \hline
                1 & 1 & \xi_8 & i & \xi_8^3 & -1 & \xi_8^5 & -i & \xi_8^7 \\
                \xi_8 & \xi_8 & i & \xi_8^3 & -1 & \xi_8^5 & -i & \xi_8^7 & 1 \\
                \xi_8^2 & i & \xi_8^3 & -1 & \xi_8^5 & -i & \xi_8^7 & 1 & \xi_8 \\
                \xi_8^3 & \xi_8^3 & -1 & \xi_8^5 & -i & \xi_8^7 & 1 & \xi_8 & i \\
                \xi_8^4 & -1 & \xi_8^5 & -i & \xi_8^7 & 1 & \xi_8 & i & \xi_8^3 \\
                \xi_8^5 & \xi_8^5 & -i & \xi_8^7 & 1 & \xi_8 & i & \xi_8^3 & -1 \\
                \xi_8^6 & -i & \xi_8^7 & 1 & \xi_8 & i & \xi_8^3 & -1 & \xi_8^5 \\
                \xi_8^7 & \xi_8^7 & 1 & \xi_8 & i & \xi_8^3 & -1 & \xi_8^5 & -i
            \end{array}
        \end{equation*}
    \end{itemize}
\end{ejercicio}

\begin{ejercicio}\label{ej:2.5}
    En el conjunto $\mathbb{Q}^{\times} := \{q \in \mathbb{Q} \mid q \neq 0\}$ de los números racionales no nulos, se considera la operación de división, dada por $(x, y) \mapsto \nicefrac{x}{y} = xy^{-1}$. ¿Nos da esta operación una estructura de grupo en $\mathbb{Q}^{\times}$?\\

    Veamos qué condiciones han de cumplirse para que se tenga la propiedad asociativa. Sean $a, b, c \in \mathbb{Q}^{\times}$, entonces:
    \begin{equation*}
        \frac{\nicefrac{a}{b}}{c} =  
        \frac{a}{\nicefrac{b}{c}}
        \iff \frac{a}{bc} = \frac{ac}{b}
        \iff ab = abc^2
        \iff 1=c^2
    \end{equation*}

    Por tanto, tomando por ejemplo $2,3,4\in\mathbb{Q}^{\times}$ no se tiene la propiedad asociativa, por lo que no se tiene un grupo.
\end{ejercicio}

\begin{ejercicio}\label{ej:2.6}
    Sea $G$ un grupo en el que $x^2 = 1$ para todo $x \in G$. Demostrar que el grupo $G$ es abeliano.\\

    Dados $x, y \in G$, se tiene que:
    \begin{align*}
        (xy)(xy) &= (xy)^2 = 1 \Longrightarrow (xy)^{-1} = xy\\
        xy=(xy)^{-1} &= y^{-1}x^{-1} = yx
    \end{align*}

    Por tanto, $xy = yx$ para todo $x, y \in G$, por lo que $G$ es abeliano.
\end{ejercicio}

\begin{ejercicio}\label{ej:2.7}
    Sea $G$ un grupo. Demostrar que son equivalentes:
    \begin{enumerate}
        \item $G$ es abeliano.
        \item $\forall x, y \in G$ se verifica que $(xy)^2 = x^2y^2$.
        \item $\forall x, y \in G$ se verifica que $(xy)^{-1} = x^{-1}y^{-1}$.
    \end{enumerate}
    \begin{proof}~
        \begin{description}
            \item[$1\Longrightarrow 2$)] Dados $x, y \in G$, se tiene que:
            \begin{equation*}
                (xy)^2 = xyxy \AstIg x^2y^2
            \end{equation*}
            donde en $(\ast)$ se ha usado que $G$ es abeliano.

            \item[$2\Longrightarrow 1$)] Dados $x, y \in G$, se tiene que:
            \begin{align*}
                \left(xy\right)^{2}
                &= (xy)(xy) = xyxy\\
                &\AstIg x^2y^2
            \end{align*}
            donde en $(\ast)$ se ha usado la hipótesis. Por la propiedad cancelativa, se tiene que:
            \begin{equation*}
                \cancel{x}yx\cancel{y} = x^{\cancel{2}}y^{\cancel{2}} \Longrightarrow xy = yx
            \end{equation*}

            Como se tiene para todo $x, y \in G$, entonces $G$ es abeliano.

            \item[$1\Longrightarrow 3$)] Dados $x, y \in G$, se tiene que:
            \begin{align*}
                (xy)^{-1}
                &= y^{-1}x^{-1} \AstIg x^{-1}y^{-1}
            \end{align*}
            donde en $(\ast)$ se ha usado que $G$ es abeliano.

            \item[$3\Longrightarrow 1$)] Dados $x, y \in G$, tenemos que:
            \begin{align*}
                (xy)^{-1} &\AstIg x^{-1}y^{-1} = (yx)^{-1}
                \Longrightarrow \left((xy)^{-1}\right)^{-1} = \left((yx)^{-1}\right)^{-1}
                \Longrightarrow xy = yx
            \end{align*}
            donde en $(\ast)$ se ha usado la hipótesis. Por tanto, como se tiene para todo $x, y \in G$, entonces $G$ es abeliano.
        \end{description}
    \end{proof}
\end{ejercicio}

\begin{ejercicio}\label{ej:2.8}
    Demostrar que si en un grupo $G$, $x, y \in G$ verifican que $xy = yx$ entonces, para todo $n \in \bb{N}\setminus \{0\}$, se tiene que $(xy)^n = x^ny^n$.\\

    Demostramos por inducción sobre $n$.
    \begin{itemize}
        \item \ul{Caso base}: $n = 1$.
        \begin{equation*}
            (xy)^1 = xy = yx = x^1y^1
        \end{equation*}

        \item \ul{Paso inductivo}: Supuesto cierto para $n$, veamos que se cumple para $n+1$.
        \begin{align*}
            (xy)^{n+1}
            &= (xy)^n(xy) = x^ny^nxy\\
            &= x^nxy^nx = x^{n+1}y^{n+1}
        \end{align*}
    \end{itemize}
    Por tanto, por inducción, se tiene que $(xy)^n = x^ny^n$ para todo $n \in \bb{N}\setminus \{0\}$.
\end{ejercicio}

\begin{ejercicio}\label{ej:2.9}
    Demostrar que el conjunto de las aplicaciones $f : \mathbb{R} \to \mathbb{R}$, tales que $f(x) = ax + b$ para algún $a,b\in \bb{R}$, $a\neq 0$, es un grupo con la composición como ley de composición.\\

    Definimos el conjunto siguiente:
    \begin{equation*}
        G = \{f : \mathbb{R} \to \mathbb{R} \mid \exists a,b\in \bb{R},~a\neq 0 \text{ tales que } f(x) = ax + b\ \forall x\in \bb{R}\}
    \end{equation*}

    En primer lugar, hemos de comprobar que $G$ es cerrado bajo la composición de funciones, algo que tendremos gracias a ser $\bb{R}$ cerrado para el producto y la suma. Dados $f, g\in G$, entonces existen $a,b,c,d\in \bb{R}$, $a,c\neq 0$ tales que:
    \begin{align*}
        f(x) &= ax + b, & g(x) &= cx + d
    \end{align*}

    Entonces, se tiene que:
    \begin{align*}
        (f\circ g)(x) &= f(g(x)) = a(cx + d) + b = acx + ad + b\in G\\
        (g\circ f)(x) &= g(f(x)) = c(ax + b) + d = acx + cb + d\in G
    \end{align*}
    
    Por tanto, $G$ es cerrado bajo la composición de funciones.
    Ahora, tomando $a=1$ y $b=0$, se tiene que $\Id_{\bb{R}}\in G$. Veamos que $(G,\circ, \Id_{\bb{R}})$ es un grupo.
    \begin{itemize}
        \item \ul{Asociatividad}: Se tiene de forma directa por serlo la composición de funciones.

        \item \ul{Elemento neutro}: Se tiene de forma directa.

        \item \ul{Elemento inverso}: Dado $f\in G$, entonces existen $a,b\in \bb{R}$, $a\neq 0$ tales que $f(x) = ax + b$.
        Entonces, definimos su elemento inverso como:
        \begin{equation*}
            f^{-1}(z) = a^{-1}\left(z - b\right)\in G
        \end{equation*}

        Comprobémoslo (notemos que tan solo hace falta comprobar que $f\circ f^{-1} = \Id_{\bb{R}}$, puesto que en la definición no se impone $f^{-1}\circ f = \Id_{\bb{R}}$):
        \begin{align*}
            (f\circ f^{-1})(z) &= a\left(a^{-1}\left(z - b\right)\right) + b = z& \forall z\in \bb{R}
        \end{align*}

        Por tanto, para todo $f\in G$, existe $f^{-1}\in G$ tal que $f\circ f^{-1}=\Id_{\bb{R}}$.
    \end{itemize}
\end{ejercicio}

\begin{ejercicio}\label{ej:2.10}~
    \begin{enumerate}
        \item Demostrar que $|\GL_2(\mathbb{Z}_2)| = 6$, describiendo explícitamente todos los elementos que forman este grupo.\\
        
        Sea $A\in \GL_2(\bb{Z}_2)$:
        \begin{align*}
            A = \begin{pmatrix} a & b \\ c & d \end{pmatrix} \Longrightarrow |A| = ad - bc \neq 0\Longrightarrow ad \neq bc
        \end{align*}

        Por tanto, los elementos de $\GL_2(\bb{Z}_2)$ son:
        \begin{align*}
            A_1 &= \begin{pmatrix} 1 & 0 \\ 0 & 1 \end{pmatrix}, & A_2 &= \begin{pmatrix} 1 & 1 \\ 0 & 1 \end{pmatrix}, & A_3 &= \begin{pmatrix} 1 & 0 \\ 1 & 1 \end{pmatrix},\\
            A_4 &= \begin{pmatrix} 0 & 1 \\ 1 & 0 \end{pmatrix}, & A_5 &= \begin{pmatrix} 0 & 1 \\ 1 & 1 \end{pmatrix}, & A_6 &= \begin{pmatrix} 1 & 1 \\ 1 & 0 \end{pmatrix}
        \end{align*}
        \item Sea $\alpha = \begin{pmatrix} 0 & 1 \\ 1 & 1 \end{pmatrix}$ y $\beta = \begin{pmatrix} 0 & 1 \\ 1 & 0 \end{pmatrix}$. Demostrar que
        $$\GL_2(\mathbb{Z}_2) = \{1, \alpha, \alpha^2, \beta, \alpha\beta, \alpha^2\beta\}.$$

        Tenemos que:
        \begin{align*}
            1&= A_1, & \alpha &= A_5, & \alpha^2 &= A_6,& \beta &= A_4, & \alpha\beta &= A_3, & \alpha^2\beta &= A_2
        \end{align*}
        \item Escribir, utilizando la representación anterior, la tabla de multiplicar de $\GL_2(\mathbb{Z}_2)$.
        
        \begin{equation*}
            \begin{array}{c|cccccc}
                \cdot & 1 & \alpha & \alpha^2 & \beta & \alpha\beta & \alpha^2\beta \\ \hline
                1 & 1 & \alpha & \alpha^2 & \beta & \alpha\beta & \alpha^2\beta \\
                \alpha & \alpha & \alpha^2 & 1 & \alpha\beta & \alpha^2\beta & \beta \\
                \alpha^2 & \alpha^2 & 1 & \alpha & \alpha^2\beta & \beta & \alpha\beta \\
                \beta & \beta & \alpha^2\beta & \alpha\beta & 1 & \alpha^2 & \alpha \\
                \alpha\beta & \alpha\beta & \beta & \alpha^2\beta & \alpha & 1 & \alpha^2 \\
                \alpha^2\beta & \alpha^2\beta & \alpha\beta & \beta & \alpha^2 & \alpha & 1
            \end{array}
        \end{equation*}
    \end{enumerate}
\end{ejercicio}

\begin{ejercicio}\label{ej:2.11}
    Dar las tablas de grupo para los grupos $D_3$, $D_4$, $D_5$ y $D_6$.\\

    Recordamos que:
    \begin{equation*}
        D_n = \langle r, s \mid r^n = s^2 = 1, rs = sr^{-1}\rangle
    \end{equation*}

    \begin{itemize}
        \item \ul{Para $D_3$}:
        \begin{equation*}
            \begin{array}{c|cccccc}
                \cdot & 1 & r & r^2 & s & sr & sr^2 \\ \hline
                1 & 1 & r & r^2 & s & sr & sr^2 \\
                r & r & r^2 & 1 & sr^2 & s & sr \\
                r^2 & r^2 & 1 & r & sr & sr^2 & s \\
                s & s & sr & sr^2 & 1 & r & r^2 \\
                sr & sr & sr^2 & s & r^2 & 1 & r \\
                sr^2 & sr^2 & s & sr & r & r^2 & 1
            \end{array}
        \end{equation*}

        \item \ul{Para $D_4$}:
        \begin{equation*}
            \begin{array}{c|cccccccc}
                 \cdot & 1 & r & r^2 & r^3 & s & sr & sr^2 & sr^3 \\
                 \hline
                    1 & 1 & r & r^2 & r^3 & s & sr & sr^2 & sr^3 \\
                    r & r & r^2 & r^3 & 1 & sr^3 & s & sr & sr^2 \\
                    r^2 & r^2 & r^3 & 1 & r & sr^2 & sr^3 & s & sr \\
                    r^3 & r^3 & 1 & r & r^2 & sr & sr^2 & sr^3 & s \\
                    s & s & sr & sr^2 & sr^3 & 1 & r & r^2 & r^3 \\
                    sr & sr & sr^2 & sr^3 & s & r^3 & 1 & r & r^2 \\
                    sr^2 & sr^2 & sr^3 & s & sr & r^2 & r^3 & 1 & r \\
                    sr^3 & sr^3 & s & sr & sr^2 & r & r^2 & r^3 & 1 
            \end{array}
        \end{equation*}

        \item \ul{Para $D_5$}:
        \begin{equation*}
            \begin{array}{c|cccccccccc}
                \cdot & 1 & r & r^2 & r^3 & r^4 & s & sr & sr^2 & sr^3 & sr^4 \\ \hline
                1 & 1 & r & r^2 & r^3 & r^4 & s & sr & sr^2 & sr^3 & sr^4 \\
                r & r & r^2 & r^3 & r^4 & 1 & sr^4 & s & sr & sr^2 & sr^3 \\
                r^2 & r^2 & r^3 & r^4 & 1 & r & sr^3 & sr^4 & s & sr & sr^2 \\
                r^3 & r^3 & r^4 & 1 & r & r^2 & sr^2 & sr^3 & sr^4 & s & sr \\
                r^4 & r^4 & 1 & r & r^2 & r^3 & sr & sr^2 & sr^3 & sr^4 & s \\
                s & s & sr & sr^2 & sr^3 & sr^4 & 1 & r & r^2 & r^3 & r^4 \\
                sr & sr & sr^2 & sr^3 & sr^4 & s & r^4 & 1 & r & r^2 & r^3 \\
                sr^2 & sr^2 & sr^3 & sr^4 & s & sr & r^3 & r^4 & 1 & r & r^2 \\
                sr^3 & sr^3 & sr^4 & s & sr & sr^2 & r^2 & r^3 & r^4 & 1 & r \\
                sr^4 & sr^4 & s & sr & sr^2 & sr^3 & r & r^2 & r^3 & r^4 & 1
            \end{array}
        \end{equation*}

        \item \ul{Para $D_6$}:
        \begin{equation*}
            \begin{array}{c|cccccccccccc}
                \cdot & 1 & r & r^2 & r^3 & r^4 & r^5 & s & sr & sr^2 & sr^3 & sr^4 & sr^5 \\ \hline
                1 & 1 & r & r^2 & r^3 & r^4 & r^5 & s & sr & sr^2 & sr^3 & sr^4 & sr^5 \\
                r & r & r^2 & r^3 & r^4 & r^5 & 1 & sr^5 & s & sr & sr^2 & sr^3 & sr^4 \\
                r^2 & r^2 & r^3 & r^4 & r^5 & 1 & r & sr^4 & sr^5 & s & sr & sr^2 & sr^3 \\
                r^3 & r^3 & r^4 & r^5 & 1 & r & r^2 & sr^3 & sr^4 & sr^5 & s & sr & sr^2 \\
                r^4 & r^4 & r^5 & 1 & r & r^2 & r^3 & sr^2 & sr^3 & sr^4 & sr^5 & s & sr \\
                r^5 & r^5 & 1 & r & r^2 & r^3 & r^4 & sr & sr^2 & sr^3 & sr^4 & sr^5 & s \\
                s & s & sr & sr^2 & sr^3 & sr^4 & sr^5 & 1 & r & r^2 & r^3 & r^4 & r^5 \\
                sr & sr & sr^2 & sr^3 & sr^4 & sr^5 & s & r^5 & 1 & r & r^2 & r^3 & r^4 \\
                sr^2 & sr^2 & sr^3 & sr^4 & sr^5 & s & sr & r^4 & r^5 & 1 & r & r^2 & r^3 \\
                sr^3 & sr^3 & sr^4 & sr^5 & s & sr & sr^2 & r^3 & r^4 & r^5 & 1 & r & r^2 \\
                sr^4 & sr^4 & sr^5 & s & sr & sr^2 & sr^3 & r^2 & r^3 & r^4 & r^5 & 1 & r \\
                sr^5 & sr^5 & s & sr & sr^2 & sr^3 & sr^4 & r & r^2 & r^3 & r^4 & r^5 & 1
            \end{array}
        \end{equation*}

    \end{itemize}
\end{ejercicio}

\begin{ejercicio}\label{ej:2.12}
    Demostrar que el conjunto de rotaciones respecto al origen del plano euclídeo junto con el conjunto de simetrías respecto a las rectas que pasan por el origen, es un grupo.\\

    Denotamos por $G$ al conjunto de rotaciones respecto al origen del plano euclídeo junto con el conjunto de simetrías respecto a las rectas que pasan por el origen. Notemos que no se trata de ningún grupo diédrico:
    \begin{equation*}
        D_n\subsetneq G\qquad \forall n\in\bb{N}
    \end{equation*}

    En primer lugar, sería necesario demostrar que es cerrado por la composición, algo que dejamos como ejercicio al lector por ser competencia de Geometría II.

    Además, $\Id_{\bb{R}^2}\in G$. Veamos que $(G,\circ, \Id_{\bb{R}^2})$ es un grupo.
    \begin{itemize}
        \item \ul{Asociatividad}: Se tiene de forma directa por serlo la composición de funciones.

        \item \ul{Elemento neutro}: Se tiene de forma directa.

        \item \ul{Elemento inverso}: Dado $f\in G$, veamos que existe $f^{-1}\in G$ tal que se tiene $f\circ f^{-1} = \Id_{\bb{R}^2}$.
        \begin{itemize}
            \item Si $f$ es una rotación de ángulo $\theta$ respecto al origen, entonces $f^{-1}$ es la rotación de ángulo $-\theta$ respecto al origen.
            \item Si $f$ es una simetría respecto a una recta que pasa por el origen, entonces $f^{-1}$ es la misma simetría.
        \end{itemize}
        En ambos casos, se tiene que $f\circ f^{-1} = \Id_{\bb{R}^2}$.
    \end{itemize}

    Por tanto, $(G,\circ, \Id_{\bb{R}^2})$ es un grupo.
\end{ejercicio}

\begin{ejercicio}\label{ej:2.13}
    Sea $G$ un grupo y sean $a, b \in G$ tales que $ba = ab^k$, $a^n = 1 = b^m$ con $n, m> 0$.
    \begin{enumerate}
        \item Demostrar que para todo $i = 0, \ldots, m - 1$ se verifica $b^ia = ab^{ik}$.
        
        Demostramos para todo $i\in \bb{N}$ por inducción sobre $i$.
        \begin{itemize}
            \item  \ul{Caso base}: $i = 0$.
            \begin{equation*}
                b^0a = a = ab^0
            \end{equation*}

            \item \ul{Caso base}: $i = 1$.
            \begin{equation*}
                b^1a = ba = ab^k = ab^{1\cdot k}
            \end{equation*}

            \item \ul{Paso inductivo}: Supuesto cierto para $i$, veamos que se cumple para $i+1$.
            \begin{align*}
                b^{i+1}a
                &= bb^ia = bab^{ik} = ab^kb^{ik} = ab^{k(i+1)}
            \end{align*}
        \end{itemize}
        \item Demostrar que para todo $j = 0, \ldots, n - 1$ se verifica $ba^j = a^jb^{k^j}$.
        
        Demostramos para todo $j\in \bb{N}$ por inducción sobre $j$.
        \begin{itemize}
            \item  \ul{Caso base}: $j = 0$.
            \begin{equation*}
                ba^0 = b = a^0b^{k^0}
            \end{equation*}

            \item \ul{Caso base}: $j = 1$.
            \begin{equation*}
                ba = ab^k = a^1b^{k^1}
            \end{equation*}

            \item \ul{Paso inductivo}: Supuesto cierto para $j$, veamos que se cumple para $j+1$.
            \begin{align*}
                ba^{j+1}
                &= ba^ja = a^jb^{k^j}a \AstIg a^jab^{k^jk}
                = a^{j+1}b^{k^{j+1}}
            \end{align*}
            donde en $(\ast)$ se ha usado el apartado anterior.
        \end{itemize}
        \item Demostrar que para todo $i = 0, \ldots, m - 1$ y todo $j = 0, \ldots, n - 1$ se verifica $b^ia^j = a^jb^{ik^j}$.\\
        Fijado $i\in \bb{N}$, demostramos por inducción sobre $j$.
        \begin{itemize}
            \item  \ul{Caso base}: $j = 0$.
            \begin{equation*}
                b^ia^0 = b^i = a^0b^{ik^0}
            \end{equation*}

            \item \ul{Caso base}: $j = 1$.
            \begin{equation*}
                b^ia = ab^{ik} = a^1b^{ik^1}
            \end{equation*}

            \item \ul{Paso inductivo}: Supuesto cierto para $j$, veamos que se cumple para $j+1$.
            \begin{align*}
                b^ia^{j+1}
                &= b^ia^ja = a^jb^{ik^j}a \AstIg a^jab^{ik^jk}
                = a^{j+1}b^{ik^{j+1}}
            \end{align*}
            donde en $(\ast)$ se ha usado el apartado anterior.
        \end{itemize}
        Por tanto, se tiene para todo $i, j\in \bb{N}$.
        \item Demostrar que todo elemento de $\langle a, b \rangle$ puede escribirse como $a^rb^s$ cumpliendo $0 \leq r < n$, $0 \leq s < m$.\\
        
        Dado $x\in \langle a, b \rangle$, entonces $x$ es producto de elementos de $\{a,b,a^{-1},b^{-1}\}$. Como $a^n = 1 = b^m$, entonces $a^{-1} = a^{n-1}$ y $b^{-1} = b^{m-1}$. Por tanto, se tiene que $x$ es producto de elementos de $\{a, b\}$. Usando el apartado anterior, podemos ``llevar'' los $a$'s a la izquierda y los $b$'s a la derecha, obteniendo lo siguiente:
        \begin{equation*}
            x = a^{r'}b^{s'}\qquad r', s'\in \bb{N}\cup \{0\}
        \end{equation*}

        Supuesto $r'\geq n$, sea $r = r'\mod n$ ($r'=nk+r$) y se tiene que:
        \begin{equation*}
            a^{r'} = a^{nk + r} = (a^{n})^k \cdot a^r = a^r
        \end{equation*}
        Además, se cumple que $0\leq r < n$. Análogamente, supuesto $s'\geq m$, sea $s = s'\mod m$ ($s'=mk+s$) y se tiene que:
        \begin{equation*}
            b^{s'} = b^{mk + s} = (b^{m})^k \cdot b^s = b^s
        \end{equation*}
        Además, se cumple que $0\leq s < m$. Por tanto:
        \begin{equation*}
            x = a^{r'}b^{s'} = a^rb^s\qquad 0\leq r < n,\ 0\leq s < m
        \end{equation*}
    \end{enumerate}
    \begin{observacion}
        Notemos que $D_n$ es un caso particular de este grupo, donde:
        \begin{align*}
            a=r,\qquad b=s,\qquad k=n-1,\qquad m=2,\qquad n=n
        \end{align*}
    \end{observacion}
\end{ejercicio}

\begin{ejercicio}\label{ej:2.14}
    Sean $s_1, s_2 \in S_7$ las permutaciones dadas por
    \begin{align*}
        s_2 &= \begin{pmatrix} 1 & 2 & 3 & 4 & 5 & 6 & 7 \\ 5 & 7 & 6 & 2 & 1 & 4 & 3 \end{pmatrix}, &
        s_1 &= \begin{pmatrix} 1 & 2 & 3 & 4 & 5 & 6 & 7 \\ 3 & 2 & 4 & 5 & 1 & 7 & 6 \end{pmatrix}.
    \end{align*}
    Calcular los productos $s_1s_2$, $s_2s_1$ y $s_2^2$, y su representación como producto de ciclos disjuntos.\\

    En notación matricial, se tiene que:
    \begin{align*}
        s_1s_2 &= \begin{pmatrix} 1 & 2 & 3 & 4 & 5 & 6 & 7 \\ 1 & 6 & 7 & 2 & 3 & 5 & 4 \end{pmatrix}\\
        s_2s_1 &= \begin{pmatrix} 1 & 2 & 3 & 4 & 5 & 6 & 7 \\ 6 & 7 & 2 & 1 & 5 & 3 & 4 \end{pmatrix}\\
        s_2^2 &= \begin{pmatrix} 1 & 2 & 3 & 4 & 5 & 6 & 7 \\ 1 & 3 & 4 & 7 & 5 & 2 & 6 \end{pmatrix}
    \end{align*}

    Descomponiendo en ciclos disjuntos, se tiene que:
    \begin{align*}
        s_2 &= (1\ 5)(2\ 7\ 3\ 6\ 4)\\
        s_1 &= (1\ 3\ 4\ 5)(6\ 7)\\
        s_1s_2 &= (2\ 6\ 5\ 3\ 7\ 4)\\
        s_2s_1 &= (1\ 6\ 3\ 2\ 7\ 4)\\
        s_2^2 &= (2\ 3\ 4\ 7\ 6)
    \end{align*}
\end{ejercicio}

\begin{ejercicio}\label{ej:2.15}
    Dadas las permutaciones
    \begin{align*}
        p_1 &= (1\ 3\ 2\ 8\ 5\ 9)(2\ 6\ 3), &
        p_2 &= (1\ 3\ 6)(2\ 5\ 3)(1\ 9\ 2\ 8\ 5),
    \end{align*}
    hallar la descomposición de la permutación producto $p_1p_2$ como producto de ciclos disjuntos.\\

    Usando la notación matricial, se tiene que:
    \begin{align*}
        p_1p_2 &=
        \begin{pmatrix}
            1 & 2 & 3 & 4 & 5 & 6 & 7 & 8 & 9 \\
            1 & 5 & 6 & 4 & 8 & 3 & 7 & 2 & 9
        \end{pmatrix}
    \end{align*}

    Descomponiendo en ciclos disjuntos, se tiene que:
    \begin{align*}
        p_1p_2 &= (2\ 5\ 8)(3\ 6)
    \end{align*}
\end{ejercicio}

\begin{ejercicio}\label{ej:2.16}
    Sean $s_1, s_2, p_1$ y $p_2$ las permutaciones dadas en los ejercicios anteriores.
    \begin{observacion}
        Aquí tratamos a $S_7$ como un subgrupo de $S_9$, donde consideramos cada permutación del conjunto $\{1, 2, 3, 4, 5, 6, 7\}$ como una permutación del conjunto $\{1, \ldots, 9\}$ que deja fijos a los elementos $8$ y $9$.
    \end{observacion}
    \begin{enumerate}
        \item Descomponer la permutación $s_1s_2s_1s_2$ como producto de ciclos disjuntos.
        \begin{align*}
            s_1s_2s_1s_2 &=
            \begin{pmatrix}
                1 & 2 & 3 & 4 & 5 & 6 & 7 & 8 & 9 \\
                1 & 5 & 4 & 6 & 7 & 3 & 2 & 8 & 9
            \end{pmatrix}\\
            &= (2\ 5\ 7)(3\ 4\ 6)
        \end{align*}

        \item Expresar matricialmente la permutación $p_3 = p_2p_1p_2$ y obtener su descomposición como ciclos disjuntos.
        \begin{align*}
            p_3 = p_2p_1p_2 &=
            \begin{pmatrix}
                1 & 2 & 3 & 4 & 5 & 6 & 7 & 8 & 9 \\
                9 & 3 & 1 & 4 & 6 & 2 & 7 & 8 & 5
            \end{pmatrix}\\
            &= (1\ 9\ 5\ 6\ 2\ 3)
        \end{align*}
        \item Descomponer la permutación $s_2p_2$ como producto de ciclos disjuntos y expresarla matricialmente.
        \begin{align*}
            s_2p_2 &=
            \begin{pmatrix}
                1 & 2 & 3 & 4 & 5 & 6 & 7 & 8 & 9 \\
                9 & 8 & 7 & 2 & 6 & 5 & 3 & 4 & 1
            \end{pmatrix}\\
            &= (1\ 9)(2\ 8\ 4)(3\ 7)(5\ 6)
        \end{align*}
    \end{enumerate}
\end{ejercicio}

\begin{ejercicio}\label{ej:2.17}
    Sean $s_1, s_2, p_1$ y $p_2$ las permutaciones dadas en los ejercicios anteriores.
    \begin{enumerate}
        \item Calcular el orden de la permutación producto $s_1s_2$. ¿Coincide dicho orden con el producto de los órdenes de $s_1$ y $s_2$?
        \begin{align*}
            s_1s_2 &= (2\ 6\ 5\ 3\ 7\ 4)\\
            s_1 &= (1\ 3\ 4\ 5)(6\ 7)\\
            s_2 &= (1\ 5)(2\ 7\ 3\ 6\ 4)
        \end{align*}

        Por el Corolario~\ref{cor:orden_permutacion}, se tiene que:
        \begin{align*}
            O(s_1s_2) &= 6\\
            O(s_1) &= \mcm(4, 2) = 4\\
            O(s_2) &= \mcm(2, 5) = 10
        \end{align*}

        Por tanto, $O(s_1s_2) \neq O(s_1)O(s_2)$.
        
        \item Calcular el orden de $s_1(s_2)^{-1}(s_1)^{-1}$.
        \begin{align*}
            s_1(s_2)^{-1}(s_1)^{-1}
            &=\begin{pmatrix}
                1 & 2 & 3 & 4 & 5 & 6 & 7 \\
                3 & 5 & 1 & 6 & 7 & 2 & 4
            \end{pmatrix}
            =\\&= (1\ 3)(2\ 5\ 7\ 4\ 6)
            O(s_1(s_2)^{-1}(s_1)^{-1}) = \mcm(2, 5) = 10
        \end{align*}

        \item Calcular la permutación $(s_1)^{-1}$, y expresarla como producto de ciclos disjuntos.
        \begin{align*}
            s_1 &= (1\ 3\ 4\ 5)(6\ 7)\\
            (s_1)^{-1} &= (5\ 4\ 3\ 1)(7\ 6)
        \end{align*}
        \item Calcular la permutación $(p_1)^{-1}$ y expresarla matricialmente.
        \begin{align*}
            p_1^{-1} &=
            \begin{pmatrix}
                1 & 2 & 3 & 4 & 5 & 6 & 7 & 8 & 9 \\
                9 & 6 & 1 & 4 & 8 & 2 & 7 & 3 & 5
            \end{pmatrix}
            =\\&= (1\ 9\ 5\ 8\ 3)(2\ 6)
        \end{align*}
        \item Calcular la permutación $p_2(s_2)^2(p_1)^{-1}$. ¿Cuál es su orden?
        \begin{align*}
            p_2 &= (1\ 3\ 6)(2\ 5\ 3)(1\ 9\ 2\ 8\ 5)\\
            (s_2)^2 &= (2\ 3\ 4\ 7\ 6)\\
            (p_1)^{-1} &= (1\ 9\ 5\ 8\ 3)(2\ 6)\\
            p_2(s_2)^2(p_1)^{-1} &= (1\ 3\ 6)(2\ 5\ 3)(1\ 9\ 2\ 8\ 5)(2\ 3\ 4\ 7\ 6)(1\ 9\ 5\ 8\ 3)(2\ 6)\\
            &= (1\ 5\ 6\ 2\ 8\ 4\ 7)(3\ 9)\\
            O(p_2(s_2)^2(p_1)^{-1}) &= \mcm(7, 2) = 14
        \end{align*}
    \end{enumerate}
\end{ejercicio}

\begin{ejercicio}\label{ej:2.18}
    Sean $s_1, s_2, p_1$ y $p_2$ las permutaciones dadas anteriormente. Sean también $s_3 = (2\ 4\ 6)$ y $s_4 = (1\ 2\ 7)(2\ 4\ 6\ 1)(5\ 3)$. ¿Cuál es la paridad de las permutaciones $s_1$, $s_4p_1p_2$ y $p_2s_3$?
    \begin{align*}
        s_1 &= (1\ 3\ 4\ 5)(6\ 7)\\
        s_4p_1p_2 &= (1\ 7)(2\ 3)(4\ 6\ 5\ 8)\\
        p_2s_3 &= (1\ 9\ 5\ 3\ 2\ 4)(6\ 8)
    \end{align*}

    Por tanto:
    \begin{align*}
        \veps(s_1) &= 1\\
        \veps(s_4p_1p_2) &= -1\\
        \veps(p_2s_3) &= 1
    \end{align*}
\end{ejercicio}

\begin{ejercicio}\label{ej:2.19}
    En el grupo $S_3$, se consideran las permutaciones $\sigma = (1\ 2\ 3)$ y $\tau = (1\ 2)$.
    \begin{enumerate}
        \item Demostrar que
        $$S_3 = \{1, \sigma, \sigma^2, \tau, \sigma\tau, \sigma^2\tau\}.$$


        Sabemos que $|S_3|=3!=6$. Dividimos $S_3$ en dos conjuntos, uno con las permutaciones pares $(P)$ y otro con las impares $(I)$.
        \begin{align*}
            P&=\{1,\sigma,\sigma^2\}\\
            I&=\{\tau,\sigma \tau,\sigma^2 \tau\}
        \end{align*}

        Como $O(\sigma)=3$, tenemos que las tres permutaciones pares son distintas. Supongamos ahora que dos permutaciones impares son iguales. Entonces, componiendo por la derecha con $\tau^{-1}$, obtenemos que dos permutaciones pares serían iguales, algo que hemos descartado. Por tanto, las tres permutaciones impares son distintas.
        \begin{equation*}
            |P|=|I|=3
        \end{equation*}
        
        Como una permutación par no puede ser igual a una impar, tenemos que $P\cap I=\emptyset$. Por tanto:
        \begin{align*}
            \left.\begin{array}{rcl}
                |P\cup I|&=&|P|+|I|=6=|S_3|\\&\land&\\
                P\cup I&\subset&S_3
            \end{array}\right\}\Longrightarrow S_3=P\cup I
        \end{align*}

        Por tanto, $S_3 = \{1, \sigma, \sigma^2, \tau, \sigma\tau, \sigma^2\tau\}$.
        
        \item Reescribir la tabla de multiplicar de $S_3$ empleando la anterior expresión de los elementos de $S_3$.
        \begin{equation*}
            \begin{array}{c|cccccc}
                \cdot & 1 & \sigma & \sigma^2 & \tau & \sigma\tau & \sigma^2\tau \\ \hline
                1 & 1 & \sigma & \sigma^2 & \tau & \sigma\tau & \sigma^2\tau \\
                \sigma & \sigma & \sigma^2 & 1 & \sigma\tau & \sigma^2\tau & \tau \\
                \sigma^2 & \sigma^2 & 1 & \sigma & \sigma^2\tau & \tau & \sigma\tau \\
                \tau & \tau & \sigma^2\tau & \sigma\tau & 1 & \sigma^2 & \sigma \\
                \sigma\tau & \sigma\tau & \tau & \sigma^2\tau & \sigma & 1 & \sigma^2 \\
                \sigma^2\tau & \sigma^2\tau & \sigma\tau & \tau & \sigma^2 & \sigma & 1
            \end{array}
        \end{equation*}
        \item Probar que
        $$\sigma^3 = 1, \quad \tau^2 = 1, \quad \tau\sigma = \sigma^2\tau.$$

        Como $O(\sigma)=3$, tenemos que $\sigma^3=1$. Por otro lado, como $O(\tau)=2$, tenemos que $\tau^2=1$.
        El último caso hay que calcularlo, y se ha visto ya en la tabla de multiplicar.
        \item Observar que es posible escribir toda la tabla de multiplicar de $S_3$ usando simplemente la descripción anterior y las relaciones anteriores.
    \end{enumerate}
\end{ejercicio}

\begin{ejercicio}\label{ej:2.20}
    Describir los diferentes ciclos del grupo $S_4$. Expresar todos los elementos de $S_4$ como producto de ciclos disjuntos.\\

    Veamos cuántos ciclos de longitud $m$ hay en un $S_n$. Cada una de las elecciones es una variación de $n$ elementos tomados de $m$ en $m$. Como además un mismo ciclo de longitud $m$ puede empezar en $m$ posiciones distintas, tenemos que el número de ciclos de longitud $m$ es:
    \begin{equation*}
        \frac{V_n^m}{m}=\frac{n!}{m(n-m)!}
    \end{equation*}

    Por tanto, los ciclos son:
    \begin{equation*}
        \begin{array}{c|c|l}
            \text{l} & \text{Nº} & \text{Ciclos}\\\hline
            1 & 1 & id\\
            2 & 6 & (1\ 2),\ (1\ 3),\ (1\ 4),\ (2\ 3),\ (2\ 4),\ (3\ 4)\\
            3 & 8 & (1\ 2\ 3),\ (1\ 2\ 4),\ (1\ 3\ 2),\ (1\ 3\ 4),\ (1\ 4\ 2),\ (1\ 4\ 3),\ (2\ 3\ 4),\ (2\ 4\ 3)\\
            4 & 6 & (1\ 2\ 3\ 4),\ (1\ 2\ 4\ 3),\ (1\ 3\ 2\ 4),\ (1\ 3\ 4\ 2),\ (1\ 4\ 2\ 3),\ (1\ 4\ 3\ 2)
        \end{array}
    \end{equation*}

    Tenemos ahora que $|S_4|=4!=24$. Como ya hemos dado 21 elementos, nos faltan 3. Estos son los elementos que no son ciclos, y son los siguientes:
    \begin{equation*}
        (1\ 2)(3\ 4),\ (1\ 3)(2\ 4),\ (1\ 4)(2\ 3)
    \end{equation*}
\end{ejercicio}

\begin{ejercicio}\label{ej:2.21}
    Demostrar que el conjunto de transposiciones
    $$\{(1, 2), (2, 3), \ldots, (n - 1, n)\}$$
    genera al grupo simétrico $S_n$.\\
    Demostramos por doble inclusión que:
    \begin{equation*}
        \langle (1, 2), (2, 3), \ldots, (n - 1, n) \rangle = S_n
    \end{equation*}
    \begin{description}
        \item[$\subset$)] Dado $\sigma\in \langle (1, 2), (2, 3), \ldots, (n - 1, n) \rangle$, entonces como $S_n$ es cerrado por producto, se tiene que $\sigma\in S_n$.
        
        \item[$\supset$)] Dado $\sigma\in S_n$, veamos que $\sigma\in \langle (1, 2), (2, 3), \ldots, (n - 1, n) \rangle$. Por ser una permutación, tenemos que $\sigma$ es producto de transposiciones. Por tanto, basta con demostrar que cualquier transposición se puede escribir como producto de elementos de $\{(1, 2), (2, 3), \ldots, (n - 1, n)\}$.\\
        
        Sea una transposición $(i, j)$, y sin pérdida de generalidad, supongamos que $i<j$. Entonces, se tiene que:
        \begin{equation*}
            (i, j) = (i, i+1)(i+1, i+2)\cdots(j-2,j-1)(j-1, j)(j-2, j-1)\cdots(i+1, i+2)(i, i+1)
        \end{equation*}

        Por tanto, $\sigma\in \langle (1, 2), (2, 3), \ldots, (n - 1, n) \rangle$.
    \end{description}
\end{ejercicio}

\begin{ejercicio}\label{ej:2.22}
    Demostrar que el conjunto $\{(1, 2, \ldots, n), (1, 2)\}$ genera al grupo simétrico $S_n$.

    Demostramos por doble inclusión que:
    \begin{equation*}
        \langle (1, 2, \ldots, n), (1, 2) \rangle = S_n
    \end{equation*}

    \begin{description}
        \item[$\subset$)] Dado $\sigma\in \langle (1, 2, \ldots, n), (1, 2) \rangle$, entonces como $S_n$ es cerrado por producto, se tiene que $\sigma\in S_n$.
        
        \item[$\supset$)] Dado $\sigma\in S_n$, veamos que $\sigma\in \langle (1, 2, \ldots, n), (1, 2) \rangle$. En primer lugar, definimos $\tau=(1, 2, \ldots, n)$. Entonces, se tiene que:
        \begin{equation*}
            \tau^k(j) = j+k\qquad \forall k,j\in \{1, \ldots, n\},~k+j\leq n
        \end{equation*}

        Además, por las propiedades de los conjugados, tenemos que:
        \begin{equation*}
            \tau^{(k-1)}(1, 2)\tau^{-(k-1)} = (\tau^{k-1}(1), \tau^{k-1}(2)) = (k, k+1)\qquad \forall k\in\bb{N},~k< n
        \end{equation*}

        Entonces, tenemos que:
        \begin{equation*}
            \left\{(1,2), (2,3), \ldots, (n-1, n)\right\} \subset \langle (1, 2, \ldots, n), (1, 2) \rangle
        \end{equation*}

        Por tanto:
        \begin{equation*}
            \sigma\in S_n=\langle (1, 2), (2, 3), \ldots, (n-1, n) \rangle \subset \langle (1, 2, \ldots, n), (1, 2) \rangle
        \end{equation*}
    \end{description}
\end{ejercicio}

\begin{ejercicio}\label{ej:2.23}
    Demostrar que para cualquier permutación $\alpha \in S_n$ se verifica que $\veps(\alpha) = \veps(\alpha^{-1})$, donde $\veps$ denota la signatura, o paridad, de una permutación.\\

    Sabemos que la paridad depende del número de ciclos de longitud par que tiene una permutación en su descomposición en ciclos disjuntos. Como este valor es el mismo para una permutación y su inversa, se tiene que $\veps(\alpha) = \veps(\alpha^{-1})$.
\end{ejercicio}

\begin{ejercicio}\label{ej:2.24}
    Demostrar que si $(x_1\ x_2\ \cdots\ x_r) \in S_n$ es un ciclo de longitud $r$, entonces
    $$\veps(x_1x_2 \cdots x_r) = (-1)^{r - 1}.$$

    \begin{itemize}
        \item Si $r$ es par, entonces hay un solo ciclo de longitud par, y por tanto $\veps(x_1x_2 \cdots x_r) = -1$. Como además $r-1$ es impar, se tiene que $(-1)^{r-1}=-1$.
        
        \item Si $r$ es impar, entonces hay un solo ciclo de longitud impar, y 0 ciclos de longitud par. Por tanto, $\veps(x_1x_2 \cdots x_r) = 1$. Como además $r-1$ es par, se tiene que $(-1)^{r-1}=1$.
    \end{itemize}
\end{ejercicio}

\begin{ejercicio}\label{ej:2.25}
    Encontrar un isomorfismo $\mu_2 \cong \mathbb{Z}^{\times}_3$.\\

    Definimos la aplicación $f:\mu_2\to \mathbb{Z}^{\times}_3$ dada por:
    \begin{align*}
        1 &\mapsto 1\\
        -1 &\mapsto 2
    \end{align*}

    Vemos de forma directa que es biyectiva. Veamos además que se trata de un homomorfismo. Para ello, a priori deberíamos de comprobar que, para todas las parejas $x,y\in \mu_2$, se cumple que $f(xy)=f(x)f(y)$. Sin embargo, por tratarse de grupos conmutativos, podemos ahorrarnos la comprobación de algunas de ellas. Además, en todas las parejas en las que aparezca el elemento neutro, puesto que $f(1)=1$, se tiene que:
    \begin{equation*}
        f(x)=f(1\cdot x)=f(1)\cdot f(x)=1\cdot f(x)=f(x)\qquad \forall x\in \mu_2
    \end{equation*}
    Por tanto, todas estas también se tienen ya comprobadas (idea que repetiremos en ejercicios posteriores). Comprobamos las restantes:
    \begin{align*}
        1=f(1)=f((-1)\cdot (-1))&=f(-1)\cdot f(-1)=2\cdot 2=4=1
    \end{align*}
    Por tanto, $f$ es un isomorfismo entre ambos grupos.
\end{ejercicio}

\begin{ejercicio}\label{ej:2.26}~
    \begin{enumerate}
        \item Demostrar que la aplicación $f:\mu_4\to \mathbb{Z}^{\times}_5$ dada por:
        \begin{align*}
            1 &\mapsto 1, & -1 &\mapsto 4, & i &\mapsto 2, & -i &\mapsto 3,
        \end{align*}
        da un isomorfismo entre el grupo $\mu_4$ de las raíces cuárticas de la unidad y el grupo $\mathbb{Z}^{\times}_5$ de las unidades en $\mathbb{Z}_5$.

        De forma directa, vemos que es biyectiva. Para ver que es un homomorfismo, tendremos que comprobar que se da la condición para las 16 posibles parejas. Por tratarse de grupos conmutativos, podremos ahorrarnos la comprobación de algunas de ellas.
        \begin{align*}
            1=f(1)=f((-1)\cdot (-1))&= f(-1)\cdot f(-1)=4\cdot 4=16=1\\
            4=f(-1)=f(i\cdot i)&= f(i)\cdot f(i)=2\cdot 2=4\\
            4=f(-1)=f((-i)\cdot (-i))&= f(-i)\cdot f(-i)=3\cdot 3=9=4\\
            3=f(-i)=f((-1)\cdot i)&= f(-1)\cdot f(i)=4\cdot 2=8=3\\
            2=f(i)=f((-1)\cdot (-i))&= f(-1)\cdot f(-i)=4\cdot 3=12=2\\
            1=f(1)=f(i\cdot (-i))&= f(i)\cdot f(-i)=2\cdot 3=6=1
        \end{align*}

        Por tanto, $f$ es un isomorfismo entre ambos grupos.
        \item Encontrar otro isomorfismo entre estos dos grupos que sea distinto del anterior.
        
        Sea $g:\mu_4\to \mathbb{Z}^{\times}_5$ otra aplicación que a continuación definiremos de forma que sea un isomorfismo. En primer lugar, hemos de imponer que $g(1)=1$, por ser este el elemento neutro en ambos grupos. Por otro lado, en $\bb{Z}^{\times}_5$ tenemos que:
        \begin{equation*}
            O(2)=O(3)=4\qquad O(4)=2
        \end{equation*}

        Como en $\mu_2$ tenemos que $O(-1)=2$ y sabemos que el orden se conserva en un isomorfismo, tenemos que ha de ser $g(-1)=4$. Por tanto, solo nos quedan dos opciones para $i$ y $-i$ de forma que $g$ sea biyectiva. Una de ellas opciones nos daría $f$, por lo que consideramos la otra alternativa. Definimos $g$ entonces como sigue:
        \begin{align*}
            1 &\mapsto 1, & -1 &\mapsto 4, & i &\mapsto 3, & -i &\mapsto 2,
        \end{align*}

        La biyección la tenemos de forma directa, y hemos de comprobar que se trata de un homomorfismo. Comprobamos tan solo los pares en los que intervienen los elementos $i$ o $-i$:
        \begin{align*}
            4=g(-1)=g(i\cdot i)&= g(i)\cdot g(i)=3\cdot 3=9=4\\
            4=g(-1)=g((-i)\cdot (-i))&= g(-i)\cdot g(-i)=2\cdot 2=4\\
            3=g(i)=g((-1)\cdot (-i))&= g(-1)\cdot g(-i)=4\cdot 2=8=3\\
            2=g(-i)=g((-1)\cdot i)&= g(-1)\cdot g(i)=4\cdot 3=12=2\\
            1=g(1)=g(i\cdot (-i))&= g(i)\cdot g(-i)=3\cdot 2=6=1
        \end{align*}
    \end{enumerate}
\end{ejercicio}

\begin{ejercicio}\label{ej:2.27}
    Encontrar un isomorfismo $\mu_2 \times \mu_2 \cong \mathbb{Z}^{\times}_8$.\\

    Sea $f:\mu_2 \times \mu_2 \to \mathbb{Z}^{\times}_8$ la aplicación definida por:
    \begin{align*}
        (1, 1) &\mapsto 1\\
        (1, -1) &\mapsto 3\\
        (-1, 1) &\mapsto 5\\
        (-1, -1) &\mapsto 7
    \end{align*}

    Comprobamos que es biyectiva de forma directa. Veamos ahora que es un homomorfismo:
    \begin{align*}
        1=f(1,1)=f[(1,-1)(1,-1)]&= f(1,-1)f(1,-1)=3\cdot 3=9=1\\
        1=f(1,1)=f[(-1,1)(-1,1)]&= f(-1,1)f(-1,1)=5\cdot 5=25=1\\
        1=f(1,1)=f[(-1,-1)(-1,-1)]&= f(-1,-1)f(-1,-1)=7\cdot 7=49=1\\
        7=f(-1,-1)=f[(1,-1)(-1,1)]&= f(1,-1)f(-1,1)=3\cdot 5=15=7\\
        5=f(-1,1)=f[(1,-1)(-1,-1)]&= f(1,-1)f(-1,-1)=3\cdot 7=21=5\\
        3=f(1,-1)=f[(-1,1)(-1,-1)]&= f(-1,1)f(-1,-1)=5\cdot 7=35=3
    \end{align*}

    Por tanto, $f$ es un isomorfismo entre ambos grupos.
\end{ejercicio}

\begin{ejercicio}\label{ej:2.28}
    Demostrar, haciendo uso de las representaciones conocidas, que $D_3 \cong S_3 \cong \GL_2(\mathbb{Z}_2)$.\\

    En primer lugar, tenemos que:
    \begin{align*}
        |D_3|&=2\cdot 3=6\\
        |S_3|&=3!=6\\
        |\GL_2(\mathbb{Z}_2)|&=(2^2-1)(2^2-2)=6
    \end{align*}

    Ahora, damos generadores para cada grupo. El generador de $S_3$ se ha visto en el Ejercicio~\ref{ej:2.22}, mientras que el generador de $\GL_2(\mathbb{Z}_2)$ se ha visto en el Ejercicio~\ref{ej:2.10}.
    \begin{align*}
        D_3 &= \langle r, s \mid r^3=1,\ s^2=1,\ sr=r^{-1}s \rangle\\
        S_3 &= \langle (1\ 2\ 3), (1\ 2) \rangle\\
        \GL_2(\mathbb{Z}_2) &= \left\langle\begin{pmatrix}0&1\\1&1\end{pmatrix},\ \begin{pmatrix}0&1\\1&0\end{pmatrix}\right\rangle
    \end{align*}

    Comprobemos en primer lugar que el generador de $S_3$ cumple las relaciones de $D_3$.
    \begin{itemize}
        \item Como $O((1\ 2\ 3))=3$, se tiene que $(1\ 2\ 3)^3=1$.
        \item Como $O((1\ 2))=2$, se tiene que $(1\ 2)^2=1$.
        \item Comprobemos que $(1\ 2)(1\ 2\ 3)=(3\ 2\ 1)(1\ 2)$.
        \begin{align*}
            (1\ 2)(1\ 2\ 3) &= (2\ 3)\\
            (3\ 2\ 1)(1\ 2) &= (2\ 3)
        \end{align*}
    \end{itemize}

    Por tanto, por el Teorema de Dyck (Teorema~\ref{teo:Dyck}), se tiene que existe un único homomorfismo $f$ de $D_3$ en $S_3$ dado por:
    \begin{align*}
        r &\mapsto (1\ 2\ 3)\\
        s &\mapsto (1\ 2)
    \end{align*}

    Como además $\{f(r),f(s)\}$ son un generador de $S_3$, tenemos que se trata de un epimorfismo, y como además $|D_3|=|S_3|$, se trata de un isomorfismo. Por tanto, $D_3\cong S_3$.\\

    Comprobemos ahora que el generador de $\GL_2(\mathbb{Z}_2)$ cumple las relaciones de $D_3$.
    \begin{align*}
        {\left(\begin{array}{cc}
            0 & 1 \\
            1 & 1 
        \end{array}\right)}^{3} &= \left(\begin{array}{cc}
            1 & 0 \\
            0 & 1 
        \end{array}\right) \\
        {\left(\begin{array}{cc}
            0 & 1 \\
            1 & 0 
        \end{array}\right)}^2 &= \left(\begin{array}{cc}
            1 & 0 \\
            0 & 1 
        \end{array}\right) \\
        \left(\begin{array}{cc}
            0 & 1 \\
            1 & 0 
        \end{array}\right)\left(\begin{array}{cc}
            0 & 1 \\
            1 & 1 
            \end{array}\right) &= {\left(\begin{array}{cc}
            0 & 1 \\
            1 & 1 
        \end{array}\right)}^2 \left(\begin{array}{cc}
            0 & 1 \\
            1 & 0 
        \end{array}\right)
    \end{align*}
    Entonces, existe un único homomorfismo $g:S_3\to \GL_2(\mathbb{Z}_2)$ de forma que:
    \begin{equation*}
        g(r) = \left(\begin{array}{cc}
            0 & 1 \\
            1 & 1 
        \end{array}\right)\qquad 
        g(s) = \left(\begin{array}{cc}
            0 & 1 \\
            1 & 0 
        \end{array}\right) 
    \end{equation*}

    Como además $\{g(r),g(s)\}$ son un generador de $\GL_2(\mathbb{Z}_2)$, tenemos que se trata de un epimorfismo, y como además $|S_3|=|\GL_2(\mathbb{Z}_2)|$, se trata de un isomorfismo. Por tanto, $S_3\cong \GL_2(\mathbb{Z}_2)$.\\

    Por ser $\cong$ una relación de equivalencia, tenemos que:
    \begin{equation*}
        D_3\cong S_3\cong \GL_2(\mathbb{Z}_2)
    \end{equation*}
\end{ejercicio}

\begin{ejercicio}\label{ej:2.29}
    Sea $\bb{K}$ un cuerpo y considérese la operación binaria
    \Func{\otimes}{\bb{K} \times \bb{K}}{\bb{K}}{(a, b)}{a \otimes b = a + b - ab.}
    Demostrar que $(\bb{K}\setminus \{1\}, \otimes)$ es un grupo isomorfo al grupo multiplicativo $\bb{K}^{\ast}$.\\

    En primer lugar, hemos de ver que es cerrado para el producto así definido. Dados $a,b\in \bb{K}\setminus \{1\}$, veamos que $a\otimes b\neq 1$. Tenemos que:
    \begin{align*}
        a\otimes b=1 &\iff a+b-ab=1\iff a(1-b)=1-b\iff a=1
    \end{align*} 
    donde, en la última implicación, hemos usado que $\bb{K}$ es un cuerpo y $b\neq 1$, por lo que $1-b\neq 0$ y por tanto tiene inverso. Por tanto, se tiene que $a\otimes b\neq 1$ y por tanto es cerrado para dicho producto. Veamos ahora que se trata de un grupo (donde hemos de tener en cuenta que no tenemos garantizada la conmutatividad de la suma):
    \begin{enumerate}
        \item \textbf{Asociatividad:} Dados $a,b,c\in \bb{K}\setminus \{1\}$, hemos de comprobar que se da la igualdad $(a\otimes b)\otimes c=a\otimes(b\otimes c)$. Tenemos que:
        \begin{align*}
            (a\otimes b)\otimes c &= (a+b-ab)\otimes c = a+b-ab+c-(a+b-ab)c\\
            a\otimes(b\otimes c) &= a\otimes(b+c-bc) = a+b+c-bc-a(b+c-bc)
        \end{align*}

        Por tanto, tenemos que:
        \begin{align*}
            (a\otimes b)\otimes c = a\otimes(b\otimes c) &\iff -ab -ac -bc -abc = -bc -ab -ac - abc
        \end{align*}
        Por tanto, se tiene que la asociatividad se cumple.

        \item \textbf{Elemento neutro:} Hemos de encontrar un elemento neutro $e\in \bb{K}\setminus \{1\}$ tal que $a\otimes e=a$ para todo $a\in \bb{K}\setminus \{1\}$. Tenemos que:
        \begin{align*}
            a\otimes e = a &\iff a+e-ae=a\iff e=ae\iff e=0
        \end{align*}
        Por tanto, el elemento neutro es el elemento neutro para la suma en $\bb{K}$, $e=0$.

        \item \textbf{Elemento inverso:} Dado $a\in \bb{K}\setminus \{1\}$, hemos de encontrar un elemento inverso $a^{-1}\in \bb{K}\setminus \{1\}$ tal que $a\otimes a^{-1}=e$. Tenemos que:
        \begin{align*}
            a\otimes a^{-1} = 0 &\iff a+a^{-1}-aa^{-1}=0\iff a=a^{-1}(-1+a)\iff a^{-1}=a(-1+a)^{-1}
        \end{align*}
        donde hemos usado que $a\neq 1$ y por tanto $-1+a\neq 0$, por lo que podemos considerar su inverso en $\bb{K}$.
    \end{enumerate}
\end{ejercicio}

\begin{ejercicio}\label{ej:2.30}~
    \begin{enumerate}
        \item Probar que si $f : G \cong G'$ es un isomorfismo de grupos, entonces se mantiene el orden; es decir, $O(a) = O(f(a))$ para todo elemento $a \in G$.
        
        Probado en la Proposición~\ref{prop:propiedades_grupos_isomorf}.
        \item Listar los órdenes de los diferentes elementos del grupo $Q_2$ y del grupo $D_4$ y concluir que $D_4$ y $Q_2$ no son isomorfos.
        
        En primer lugar, tenemos que:
        \begin{align*}
            Q_2 &= \{\pm 1, \pm i, \pm j, \pm k\}\\
            D_4 &= \{1, r, r^2, r^3, s, sr, sr^2, sr^3\}
        \end{align*}

        Calculamos los órdenes de $D_4$:
        \begin{gather*}
            O(1) =1\qquad O(r)=O(r^3)=4\\
            O(r^2)=O(s)=O(sr)=O(sr^2)=O(sr^3)=2
        \end{gather*}

        Por otro lado, calculamos los órdenes de $Q_2$:
        \begin{gather*}
            O(1)=1\qquad O(-1)=2\\
            O(\pm i)=O(\pm j)=O(\pm k)=4
        \end{gather*}

        Por tanto no es posible establecer un isomorfismo $f:D_4\to Q_2$ de forma que cumpla
        \begin{equation*}
            O(x)=O(f(x))\qquad \forall x\in D_4
        \end{equation*}

        Por tanto, $D_4$ y $Q_2$ no son isomorfos.
    \end{enumerate}
\end{ejercicio}

\begin{ejercicio}\label{ej:2.31}
    Calcular el orden de:
    \begin{enumerate}
        \item La permutación $\sigma = (1\ 8\ 10\ 4)(2\ 8)(5\ 1\ 4\ 8)\in S_{15}$.
        \begin{align*}
            \sigma &= (2\ 10\ 4)(5\ 8)\\
            O(\sigma) &= \mcm(3,2)=6
        \end{align*}
        \item Cada elemento del grupo $\mathbb{Z}^{\times}_{11}$.
        \begin{align*}
            O(1)&=1\\
            O(3)&=O(4)=O(5)=O(9)=5\\
            O(2)&=O(6)=O(7)=O(8)=10\\
            O(10)&=2
        \end{align*}
    \end{enumerate}
\end{ejercicio}

\begin{ejercicio}\label{ej:2.32}
    Demostrar que un grupo generado por dos elementos distintos de orden dos, que conmutan entre sí, consiste del $1$, de esos elementos y de su producto y es isomorfo al grupo de Klein.\\

    Sea $G=\langle a,b\mid a^2=b^2=1, ab=ba\rangle$. Entonces, por el Ejercicio~\ref{ej:2.13} tenemos:
    \begin{equation*}
        G=\{1,a,b,ab\}
    \end{equation*}

    Sea ahora el grupo de Klein el siguiente:
    \begin{align*}
        V&=\{1,\ (1\ 2)(3\ 4),\ (1\ 3)(2\ 4),\ (1\ 4)(2\ 3)\}\\
        &= \langle (1\ 2)(3\ 4),\ (1\ 3)(2\ 4) \rangle
    \end{align*}

    Por tanto, hemos de encontrar un isomorfismo entre ambos grupos. Comprobemos que los elementos generadores de $V$ cumplen las relaciones de $G$:
    \begin{align*}
        O((1\ 2)(3\ 4))&= \mcm(2,2)=2\Longrightarrow [(1\ 2)(3\ 4)]^2=1\\
        O((1\ 3)(2\ 4))&= \mcm(2,2)=2\Longrightarrow [(1\ 3)(2\ 4)]^2=1\\
        (1\ 2)(3\ 4)\ (1\ 3)(2\ 4)&=(1\ 4)(2\ 3)\\
        (1\ 3)(2\ 4)\ (1\ 2)(3\ 4)&=(1\ 4)(2\ 3)
    \end{align*}

    Por tanto, por el Teorema de Dyck (Teorema~\ref{teo:Dyck}), se tiene que existe un único homomorfismo $f:G\to V$ cumpliendo:
    \begin{align*}
        a &\mapsto (1\ 2)(3\ 4)\\
        b &\mapsto (1\ 3)(2\ 4)
    \end{align*}

    Como además $\{f(a),f(b)\}$ son un generador de $V$, tenemos que se trata de un epimorfismo, y como además $|G|=|V|$, se trata de un isomorfismo. Por tanto, $G\cong V$.
\end{ejercicio}

\begin{ejercicio}\label{ej:2.33}
    Sea $G$ un grupo y sean $a, b \in G$.
    \begin{enumerate}
        \item Demostrar que $O(b) = O(aba^{-1})$ (un elemento y su conjugado tienen el mismo orden).
        
        Para todo $n\in \bb{N}$, se tiene que:
        \begin{align*}
            1=(aba^{-1})^n&=ab^n a^{-1}\iff a^{-1}=b^n a^{-1}\iff 1=b^n
        \end{align*}

        Comprobemos ahora que $O(b)=O(aba^{-1})$:
        \begin{itemize}
            \item Si $O(b)=\infty$, supongamos por reducción al absurdo que $\exists n\in \bb{N}$ tal que $(aba^{-1})^n=1$. Entonces, se tiene que $b^n=1$, lo que contradice que $O(b)=\infty$.
            
            \item Si $O(b)=n$, entonces se tiene que $b^n=1$, por lo que $(aba^{-1})^n=1$ y por tanto $O(aba^{-1})\leq n$. Por otro lado, supongamos que $\exists m\in \bb{N}$, con $m<n$, tal que $(aba^{-1})^m=1$. Entonces, se tiene que $b^m=1$, lo que contradice que $O(b)=n$. Por tanto, $O(aba^{-1})=n$.
        \end{itemize}

        En cualquier caso, se tiene que $O(b)=O(aba^{-1})$.
        \item Demostrar que $O(ba) = O(ab)$.
        
        Por el apartado anterior, considerando ahora $ba\in G$, se tiene:
        \begin{equation*}
            O(ba) = O(a\ ba\ a^{-1}) = O(ab)
        \end{equation*}
    \end{enumerate}
\end{ejercicio}

\begin{ejercicio}\label{ej:2.34}
    Sea $G$ un grupo y sean $a, b \in G$, $a \neq 1 \neq b$, tales que $a^2 = 1$ y $ab^2 = b^3a$. Demostrar que $O(a) = 2$ y que $O(b) = 5$.\\

    Comprobemos en primer lugar que $O(a)=2$. Por hipótesis, tenemos que $a^2=1$, por lo que $O(a)\mid 2$. Por tanto, $O(a)=1$ o $O(a)=2$. Como $a\neq 1$, se tiene que $O(a)=2$. Veamos ahora que $O(b)=5$. Tenemos que:
    \begin{multline*}
        ab^2 = b^3a
        \Longrightarrow
        b^2=ab^3a
        \Longrightarrow \\ \Longrightarrow
        b^4=(ab^3a)(ab^3a)=ab^6a=a(ab^3a)(ab^3a)(ab^3a)a
        =b^9\Longrightarrow 1=b^5
    \end{multline*}

    Por tanto, $O(b)\mid 5$. Por tanto, $O(b)=1$ o $O(b)=5$. Como $b\neq 1$, se tiene que $O(b)=5$.
\end{ejercicio}

\begin{ejercicio}\label{ej:2.35}
    Sea $f : G \to H$ un homomorfismo de grupos.
    \begin{enumerate}
        \item $f(x^n) = f(x)^n$ $\forall n \in \mathbb{Z}$.
        
        Por inducción, se tiene que:
        \begin{itemize}
            \item \ul{Caso base:} $n=1$.
            \begin{equation*}
                f(x^1)=f(x)=f(x)^1
            \end{equation*}

            \item \ul{Paso inductivo:} Supongamos que se cumple para $n$, y veamos que se cumple para $n+1$.
            \begin{align*}
                f(x^{n+1})&=f(x^n x)=f(x^n)f(x)=f(x)^n f(x)=f(x)^{n+1}
            \end{align*}
        \end{itemize}

        Por tanto, se tiene que $f(x^n)=f(x)^n$ $\forall n\in \bb{Z}$.
        \item Si $f$ es un isomorfismo entonces $G$ y $H$ tienen el mismo número de elementos de orden $n$. ¿Es cierto el resultado si $f$ es sólo un homomorfismo?
        
        Consideramos la aplicación inclusión dada por:
        \Func{i}{\bb{R}^*}{\bb{C}^*}{x}{x}

        Comprobemos que se trata de un homomorfismo:
        \begin{align*}
            i(x\cdot y)&=x\cdot y=i(x)\cdot i(y)\qquad \forall x,y\in \bb{R}^*
        \end{align*}

        No obstante, tenemos que en $\bb{C}^*$ hay elementos de orden 4 ($O(i)=4$), mientras que en $\bb{R}^*$ no los hay. Por tanto, no se cumple el resultado si $f$ es solo un homomorfismo.
        \item Si $f$ es un isomorfismo entonces $G$ es abeliano $\Leftrightarrow$ $H$ es abeliano.
        
        Probado en la Proposición~\ref{prop:propiedades_grupos_isomorf}.
    \end{enumerate}
\end{ejercicio}

\begin{ejercicio}\label{ej:2.36}~
    \begin{enumerate}
        \item Demostrar que los grupos multiplicativos $\mathbb{R}^{\ast}$ (de los reales no nulos) y $\mathbb{C}^{\ast}$ (de los complejos no nulos) no son isomorfos.
        
        En $\bb{R}^*$, tenemos que:
        \begin{equation*}
            O(x)=\infty\qquad \forall x\in \bb{R}^*\setminus\{1\}
        \end{equation*}

        No obstante, en $\bb{C}^*$, tenemos que $O(i)=4$. Por tanto, no pueden ser isomorfos.
        \item Demostrar que los grupos aditivos $\mathbb{Z}$ y $\mathbb{Q}$ no son isomorfos.
    \end{enumerate}
\end{ejercicio}

\begin{ejercicio}\label{ej:2.37}
    Sea $G$ un grupo. Demostrar:
    \begin{enumerate}
        \item $G$ es abeliano $\iff$ la aplicación $f : G \to G$ dada por $f(x) = x^{-1}$ es un homomorfismo de grupos.
        \item $G$ es abeliano $\iff$ la aplicación $f : G \to G$ dada por $f(x) = x^2$ es un homomorfismo de grupos.
    \end{enumerate}
\end{ejercicio}


\begin{ejercicio}\label{ej:2.38}
    Si $G$ es un grupo cíclico demostrar que cualquier homomorfismo de grupos $f : G \to H$ está determinado por la imagen del generador.
\end{ejercicio}

\begin{ejercicio}\label{ej:2.39}
    Demostrar que no existe ningún cuerpo $K$ tal que sus grupos aditivo $(K, +)$ y $(K^{\ast}, \cdot)$ sean isomorfos.
\end{ejercicio}
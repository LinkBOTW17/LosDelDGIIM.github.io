\section{Grupos cociente. Teoremas de isomorfismo. Productos}

\begin{ejercicio}
    Demostrar que si $G\leq S_n$, entonces $G\subseteq A_n$ o bien se tiene que $[G:G\cap A_n]=2$. Concluir que un subgrupo de $S_n$ consiste sólo en permutaciones pares, o bien contiene el mismo número de permutaciones pares que de impares.\\

    Sea $G\leq S_n$ un subgrupo de $S_n$. O bien $G\subseteq A_n$ (en cuyo caso consiste sólo de permutaciones pares); o bien $\exists \sigma\in G$ tal que $\sigma\notin A_n$, es decir, $\veps(\sigma)=-1$.
    Para ver que $[G:G\cap A_n]=2$, hay varias posibilidades.
    \begin{description}
        \item[Opción 1:] Consideramos el homomorfismo $\veps:G\to \{-1,1\}$ dado por la aplicación signatura. Calculemos su núcleo y su imagen:
        \begin{align*}
            \ker(\veps) &= \{\sigma\in G\mid \veps(\sigma)=1\} = A_n\cap G,\\
            Im(\veps) &= \{\veps(\sigma)\mid \sigma\in G\}\AstIg \{-1,1\}
        \end{align*}
        donde vamos a razonar el por qué de $(\ast)$.
        Como $1\in G$ por ser este un grupo, entonces $1\in Im(\veps)$, y como $\exists \sigma\in G$ tal que $\veps(\sigma)=-1$, entonces $-1\in Im(\veps)$. Por lo tanto, $Im(\veps)=\{-1,1\}$. Por el Primer Teorema de Isomorfía, se tiene que:
        \begin{equation*}
            \frac{G}{\ker(\veps)}\cong Im(\veps) \implies \frac{G}{A_n\cap G}\cong \{-1,1\}\cong \bb{Z}_2.
        \end{equation*}

        Por definición de índice, se tiene que:
        \begin{equation*}
            [G:A_n\cap G] = \left|\dfrac{G}{A_n\cap G}\right| = |\bb{Z}_2| = 2
        \end{equation*}

        \item[Opción 2:] Por el Teorema de Lagrange, sabemos que:
        \begin{equation*}
            [S_n : A_n] = \dfrac{|S_n|}{|A_n|} = 2\Longrightarrow A_n \lhd S_n
        \end{equation*}

        Aplicando el Segundo Teorema de Isomorfía a $G$ y $A_n$, tenemos que:
        \begin{equation*}
            \dfrac{G}{G\cap A_n} \cong \dfrac{GA_n}{A_n}
        \end{equation*}

        Veamos qué grupo es $GA_n$. En primer lugar, para $1\in G$ vemos que $A_n\subset GA_n$. No obstante, como $\exists \sigma\in G$ con $\veps(\sigma)=-1$, entonces $A_n\neq GA_n$, por lo que $|GA_n|>|A_n|=\nicefrac{|S_n|}{2}$. Como $|GA_n| \mid |S_n|$, ha de ser $|GA_n|=|S_n|$, por lo que $GA_n=S_n$. Por tanto:
        \begin{equation*}
            \dfrac{G}{G\cap A_n} \cong \dfrac{S_n}{A_n}
        \end{equation*}

        Por definición de índice, se tiene que:
        \begin{equation*}
            [G:A_n\cap G] = \left|\dfrac{G}{A_n\cap G}\right| = \left|\dfrac{S_n}{A_n}\right| = \dfrac{|S_n|}{|A_n|} = 2
        \end{equation*}
    \end{description}

    En cualquier caso, hemos visto que $[G:A_n\cap G]=2$. Por el Teorema de Lagrange, se tiene que $|G|=2\cdot |G\cap A_n|$, por lo que la mitad de las permutaciones de $G$ son pares. Como una permutación o bien es par o es impar, entonces la otra mitad ha de tener signatura impar. Por tanto, contiene el mismo número de permutaciones pares que de impares.
\end{ejercicio}

\begin{ejercicio}
    Sea $\bb{K}$ un cuerpo.
    \begin{enumerate}
        \item Se considera la siguiente aplicación:
        \Func{\det}{\GL_n(\bb{K})}{\bb{K}^\times}{G}{\det(G)}
        Demostrar que dicha aplicación es un epimorfismo de grupos. ¿Cuál es el núcleo de este homomorfismo?

        Para comprobar que se trata de un homomorfismo, tomamos $A,B\in \GL_n(\bb{K})$ y por las propiedades de la determinante, se tiene que:
        \begin{equation*}
            \det(AB) = \det(A)\cdot\det(B)
        \end{equation*}

        Por otro lado, para cada $a\in \bb{K}^\times$, se considera la siguiente matriz:
        \begin{equation*}
            A_a = \begin{pmatrix}
                a & 0 & \cdots & 0 \\
                0 & 1 & \cdots & 0 \\
                \vdots & \vdots & \ddots & \vdots \\
                0 & 0 & \cdots & 1
            \end{pmatrix}
        \end{equation*}

        Como $\det(A_a)=a\neq 0$, entonces $A_a\in \GL_n(\bb{K})$. Como $\det(A_a)=a$, se tiene que $\det$ es sobreyectiva. Por lo tanto, $\det$ es un epimorfismo de grupos. Su núcleo es:
        \begin{equation*}
            \ker(\det) = \{A\in \GL_n(\bb{K})\mid \det(A)=1\} = \SL_n(\bb{K})
        \end{equation*}
        \item Si $\bb{K}$ es un cuerpo finito con $q$ elementos, determinar el orden del grupo $\SL_n(\bb{K})$.\\
        
        Por el Primer Teorema de Isomorfía, se tiene que:
        \begin{equation*}
            \dfrac{\GL_n(\bb{K})}{\SL_n(\bb{K})}\cong \bb{K}^\times
        \end{equation*}

        Por el Teorema de Lagrange, se tiene que:
        \begin{equation*}
            |\SL_n(\bb{K})| = \dfrac{|\GL_n(\bb{K})|}{|\bb{K}^\times|} = \dfrac{|\GL_n(\bb{K})|}{q-1} = \dfrac{(q^n-1)(q^n-q)\cdots(q^n-q^{n-1})}{q-1}
        \end{equation*}
    \end{enumerate}
\end{ejercicio}

\begin{ejercicio}
    Sea $n\in \bb{N}\setminus \{0\}$, y sea $G$ un grupo verificando que para todo par de elementos $x,y\in G$ se tiene que $(xy)^n=x^ny^n$. Se definen:
    \begin{align*}
        H &= \{x\in G\mid x^n=1\},\\
        K &= \{x^n\mid x\in G\}.
    \end{align*}
    Demostrar que $H,K\lhd G$, y que $|K|=[G:H]$.\\

    Definimos en primer lugar la siguiente aplicación:
    \Func{f}{G}{G}{x}{x^n}

    Para demostrar que se trata de un homomorfismo emplearemos la propiedad dada en el enunciado $(\ast)$:
    \begin{align*}
        f(xy) &= (xy)^n \AstIg x^ny^n = f(x)f(y)\qquad \forall x,y\in G
    \end{align*}

    Por tanto, $f$ es un homomorfismo. Como $\{1\},G<G$, entonces los siguientes grupos son subgrupos de $G$:
    \begin{align*}
        f^*(\{1\}) &= \{x\in G\mid f(x)=1\} = \{x\in G\mid x^n=1\} = H = \ker(f),\\
        f_*(G) &= \{f(x)\mid x\in G\} = \{x^n\mid x\in G\} = K = Im(f).
    \end{align*}

    Por tanto, tenemos que $H,K<G$. Probamos ahora que son grupos normales en $G$. En primer lugar, para $H$ tomamos $x\in G$ y $h\in H$ (por lo que $h^n=1$). Entonces:
    \begin{align*}
        (xhx^{-1})^n = xh^nx^{-1} = x\cdot 1\cdot x^{-1} = 1 \implies xhx^{-1}\in H
    \end{align*}

    Por tanto, $H\lhd G$. Ahora probamos que $K$ es normal en $G$. Tomamos $x\in G$ y $k\in K$ (por lo que $\exists y\in G$ tal que $k=y^n$). Entonces, consideramos $xyx^{-1}\in G$ y calculamos:
    \begin{align*}
        xkx^{-1} &= x(y^n)x^{-1} = (xyx^{-1})^n \in K
    \end{align*}
    Por tanto, $K\lhd G$. Para probar que $|K|=[G:H]$, tomamos el homomorfismo $f$ anteriormente descrito. Por el Primer Teorema de Isomorfía, se tiene que:
    \begin{equation*}
        \dfrac{G}{H}\cong K \implies |K| = \left|\dfrac{G}{H}\right| = [G:H]
    \end{equation*}
\end{ejercicio}

\begin{ejercicio}
    Para un grupo $G$ se define su centro como
    \[
        Z(G) = \{a\in G\mid ax=xa\ \forall x\in G\}.
    \]
    \begin{enumerate}
        \item Demostrar que $Z(G)< G$.
        
        Como $Z(G)\subset G$, hay dos principales posibilidades, ambas equivalentes:
        \begin{description}
            \item[Opción 1:]
            Comprobamos las tres condiciones que caracterizan a los subgrupos:
            \begin{itemize}
                \item $1\in Z(G)$: Para todo $x\in G$, se tiene que $1x=x1=x$.
                \item $a,b\in Z(G)\implies ab\in Z(G)$: Para todo $x\in G$, se tiene que:
                \begin{equation*}
                    (ab)x = a(bx) = a(xb) = (ax)b = (xa)b = x(ab).
                \end{equation*}
                \item $a\in Z(G)\implies a^{-1}\in Z(G)$: Para todo $x\in G$, se tiene que:
                \begin{equation*}
                    a^{-1}x = (x^{-1}a)^{-1} = (ax^{-1})^{-1} = xa^{-1}.
                \end{equation*}
            \end{itemize}

            \item[Opción 2:]
            
            Dados $a,b\in Z(G)$, comprobemos que $ab^{-1}\in Z(G)$:
            \begin{align*}
                \hspace{-1cm}(ab^{-1})x &= a(b^{-1}x) = a(x^{-1}b)^{-1} = a(bx^{-1})^{-1} = a(xb^{-1}) = (ax)b^{-1} = (xa)b^{-1} = x(ab^{-1}).
            \end{align*}
        \end{description}

        En cualquier caso, $Z(G)$ es un subgrupo de $G$.
        \item Demostrar que $Z(G)\lhd G$.
        
        De nuevo, hay dos posibilidades:
        \begin{description}
            \item[Opción 1:]
            Para $x\in G$. Entonces:
            \begin{equation*}
                xZ(G) = \{xz\mid z\in Z(G)\} = \{zx\mid z\in Z(G)\} = Z(G)x.
            \end{equation*}

            \item[Opción 2:]
            Empleamos la caracterización de subgrupo normal. Para $x\in G$ y $z\in Z(G)$, buscamos ver que $xzx^{-1}\in Z(G)$:
            \begin{align*}
                xzx^{-1}y &= zxx^{-1}y = zy = yz = yzxx^{-1} = yxzx^{-1}\qquad \forall y\in G.
            \end{align*}
        \end{description}
        En ambos casos, se tiene que $Z(G)\lhd G$.
        \item Demostrar que $G$ es abeliano si, y sólo si, $G=Z(G)$.
        \begin{equation*}
            \hspace{-1cm}G=Z(G) \iff G=\{a\in G\mid ax=xa\ \forall x\in G\} \iff ax=xa\ \forall a,x\in G \iff G\text{ es abeliano.}
        \end{equation*}
        \item Demostrar que si $G/Z(G)$ es cíclico, entonces $G$ es abeliano.\\
        
        Si $G$ no es finito, entonces $G\cong \bb{Z}$, por lo que $G$ es abeliano. Supongamos por tanto que $G$ es finito. Como $G/Z(G)$ es cíclico, entonces existe $x\in G$ tal que $G/Z(G)=\langle xZ(G)\rangle$. Por tanto, si $O(x)=n$, entonces:
        \begin{equation*}
            G/Z(G) = \{x^kZ(G)\mid k\in \{0,\ldots,n-1\}\} = \{Z(G),xZ(G),\ldots,x^{n-1}Z(G)\}
        \end{equation*}

        Como las clases de equivalencia forman una partición disjunta de $G$, para cada $x\in G$ existe $k\in \{0,\ldots,n-1\}$ tal que $x\in x^kZ(G)$. Buscamos ahora demostrar que $G$ es abeliano. Dados $a,b\in G$, entonces existen $p,q\in \{0,\ldots,n-1\}$ tales que $a\in x^pZ(G)$ y $b\in x^qZ(G)$. Entonces, existen $z_a,z_b\in Z(G)$ tales que $a=x^pz_a$ y $b=x^qz_b$. Por tanto:
        \begin{align*}
            ab &= (x^pz_a)(x^qz_b) = x^{p+q}z_az_b = x^qz_b x^pz_a = (x^qz_b)(x^pz_a) = ba
        \end{align*}

        Por tanto, $ab=ba$ para todo $a,b\in G$, por lo que $G$ es abeliano.        
    \end{enumerate}
\end{ejercicio}

\begin{ejercicio}
    Determinar el centro del grupo diédrico $D_4$. Observar que el cociente $D_4/Z(D_4)$ es abeliano, aunque $D_4$ no lo sea (compárese este hecho con el tercer apartado del ejercicio anterior).
\end{ejercicio}

\begin{ejercicio}
    Determinar el centro de los grupos $S_n$ y $A_n$ para $n\geq 2$.
\end{ejercicio}

\begin{ejercicio}
    Determinar el centro del grupo $D_n$ para $n\geq 3$.
\end{ejercicio}

\begin{ejercicio}
    Sean $H$ y $K$ dos subgrupos finitos de un grupo $G$, uno de ellos normal. Demostrar que
    \[
        |H||K| = |HK||H\cap K|.
    \]
\end{ejercicio}

\begin{ejercicio}
    Sea $G$ finito y $N\lhd G$. Probar que $G/N\cong G$ si, y sólo si, $N=\{1\}$, y que $G/N\cong \{1\}$ si, y sólo si, $N=G$.
\end{ejercicio}

\begin{ejercicio}
    Sean $G$ y $H$ dos grupos cuyos órdenes sean primos relativos. Probar que si $f:G\to H$ es un homomorfismo, entonces necesariamente $f(x)=1$ para todo $x\in G$, es decir, que el único homomorfismo entre ellos es el trivial.
\end{ejercicio}

\begin{ejercicio}
    Sean $H,K\leq G$, y sea $N\lhd G$ un subgrupo normal de $G$ tal que $HN=KN$. Demostrar que
    \[
        \frac{H}{H\cap N}\cong \frac{K}{K\cap N}.
    \]
\end{ejercicio}

\begin{ejercicio}
    Sea $N\lhd G$ tal que $N$ y $G/N$ son abelianos. Sea $H$ un subgrupo cualquiera de $G$. Demostrar que existe un subgrupo normal $K\lhd H$ tal que $K$ y $H/K$ son abelianos.
\end{ejercicio}

\begin{ejercicio}
    Sea $G$ un grupo finito, y sean $H,K\leq G$, con $K\lhd G$ y tales que $|H|$ y $[G:K]$ son primos relativos. Demostrar que $H\subseteq K$.
\end{ejercicio}

\begin{ejercicio}
    Sea $G$ un grupo.
    \begin{enumerate}
        \item Demostrar que para cada $a\in G$ la aplicación $\varphi_a:G\to G$ definida por $\varphi_a(x)=axa^{-1}$, es un automorfismo de $G$. $\varphi_a$ se llama automorfismo interno o de conjugación de $G$ definido por $a$.
        \item Demostrar que la siguiente aplicación es un homomorfismo:
        \Func{\varphi}{G}{\Aut(G)}{a}{\varphi_a}
        \item Demostrar que el conjunto de automorfismos internos de $G$, que se denota $\Int(G)$, es un subgrupo normal de $\Aut(G)$.
        \item Demostrar que $G/Z(G)\cong \Int(G)$.
        \item Demostrar que $\Int(G)=1$ si y sólo si $G$ es abeliano.
    \end{enumerate}
\end{ejercicio}

\begin{ejercicio}
    Demostrar que el grupo de automorfismos de un grupo no abeliano no puede ser cíclico.
\end{ejercicio}

\begin{ejercicio}
    Demostrar que $\Aut(\bb{Z}_2\times \bb{Z}_2)\cong S_3$.
\end{ejercicio}

\begin{ejercicio}
    Demostrar que los grupos $S_3$, $\bb{Z}_{p^n}$ (con $p$ primo) y $\bb{Z}$ no son producto directo internos de subgrupos propios.
\end{ejercicio}

\begin{ejercicio}
    En cada uno de los siguientes casos, decidir si el grupo $G$ es o no producto directo de los subgrupos $H$ y $K$.
    \begin{enumerate}
        \item $G=\bb{R}^\times$, $H=\{\pm 1\}$, $K=\{x\in \bb{R}\mid x>0\}$.
        \item $G=\left\{\begin{pmatrix} a & b \\ 0 & c \end{pmatrix}\in \GL_2(\bb{R})\right\}$, $H=\left\{\begin{pmatrix} a & 0 \\ 0 & c \end{pmatrix}\in \GL_2(\bb{R})\right\}$, $K=\left\{\begin{pmatrix} 1 & b \\ 0 & 1 \end{pmatrix}\in \GL_2(\bb{R})\right\}$.
        \item $G=\bb{C}^\times$, $H=\left\{z\in \bb{C}\mid |z|=1\right\}$, $K=\left\{x\in \bb{R}\mid x>0\right\}$.
    \end{enumerate}
\end{ejercicio}

\begin{ejercicio}
    Sean $G,H$ y $K$ grupos. Demostrar que:
    \begin{enumerate}
        \item $H\times K\cong K\times H$.
        \item $G\times (H\times K)\cong (G\times H)\times K$.
    \end{enumerate}
\end{ejercicio}

\begin{ejercicio}
    Dados isomorfismos de grupos $H\cong J$ y $K\cong L$, demostrar que $H\times K\cong J\times L$.
\end{ejercicio}

\begin{ejercicio}
    Sean $H,K,L$ y $M$ grupos tales que $H\times K\cong L\times M$. ¿Se verifica necesariamente que $H\cong L$ y $K\cong M$?
\end{ejercicio}

\begin{ejercicio}
    Demostrar que no todo subgrupo de un producto directo $H\times K$ es de la forma $H_1\times K_1$, con $H_1\leq H$ y $K_1\leq K$.
\end{ejercicio}

\begin{ejercicio}
    Sean $H,K$ dos grupos y sean $H_1\lhd H$, $K_1\lhd K$. Demostrar que $H_1\times K_1\lhd H\times K$ y que
    \[
        \frac{H\times K}{H_1\times K_1}\cong \frac{H}{H_1}\times \frac{K}{K_1}.
    \]
\end{ejercicio}

\begin{ejercicio}
    Sean $H,K\lhd G$ tales que $H\cap K=\{1\}$. Demostrar que $G$ es isomorfo a un subgrupo de $G/H\times G/K$.
\end{ejercicio}

\begin{ejercicio}
    Sean $H,K\lhd G$ tales que $HK=G$. Demostrar que
    \[
        \frac{G}{H\cap K}\cong \frac{H}{H\cap K}\times \frac{K}{H\cap K}\cong \frac{G}{H}\times \frac{G}{K}.
    \]
\end{ejercicio}

\begin{ejercicio}
    Demostrar que si $G$ es un grupo que es producto directo interno de subgrupos $H$ y $K$, y $N\lhd G$ tal que $N\cap H=\{1\}=N\cap K$, entonces $N$ es abeliano.
\end{ejercicio}

\begin{ejercicio}
    Dar un ejemplo de un grupo $G$ que sea producto directo interno de dos subgrupos propios $H$ y $K$, y que contenga a un subgrupo normal no trivial $N$ tal que $N\cap H=\{1\}=N\cap K$. Concluir que para $N\lhd H\times K$ es posible que se tenga
    \[
        N\neq (N\cap (H\times \{1\}))\times (N\cap (\{1\}\times K)).
    \]
\end{ejercicio}

\begin{ejercicio}
    Sea $G$ un grupo finito que sea producto directo interno de dos subgrupos $H$ y $K$ tales que $\mcd(|H|,|K|)=1$. Demostrar que para todo subgrupo $N\leq G$ se verifica que $N=(N\cap H)\times (N\cap K)$.
\end{ejercicio}

\begin{ejercicio}
    Sea $G$ un grupo y sea $f:G\to G$ un endomorfismo idempotente (esto es, verificando que $f^2=f$) y tal que $Im(f)\lhd G$. Demostrar que
    \[
        G\cong Im(f)\times \ker(f).
    \]
\end{ejercicio}

\begin{ejercicio}
    Sea $S$ un subconjunto de un grupo $G$. Se llama \emph{centralizador} de $S$ en $G$ al conjunto
    \[
        C_G(S) = \{x\in G\mid xs=sx\ \forall s\in S\}
    \]
    y se llama \emph{normalizador} de $S$ en $G$ al conjunto
    \[
        N_G(S) = \{x\in G\mid xS=Sx\}.
    \]
    \begin{enumerate}
        \item Demostrar que $N_G(S)\leq G$.
        \item Demostrar que $C_G(S)\lhd N_G(S)$.
        \item Demostrar que si $S\leq G$ entonces $S\lhd N_G(S)$.
    \end{enumerate}
\end{ejercicio}

\begin{ejercicio}
    Sea $G$ un grupo y $H$ y $K$ subgrupos suyos con $H\subseteq K$. Entonces demostrar que $H$ es normal en $K$ si y sólo si $K< N_G(H)$. (Así, el normalizador $N_G(H)$ queda caracterizado como el mayor subgrupo de $G$ en el que $H$ es normal.)
\end{ejercicio}

\begin{ejercicio} Sea $G$ un grupo.
    \begin{enumerate}
        \item Demostrar que $C_G(Z(G))=G$ y que $N_G(Z(G))=G$.
        \item Si $G$ es un grupo y $H<G$ ¿Cuándo es $G=N_G(H)$? ¿Y cuándo es $G=C_G(H)$?
        \item Si $H\leq G$ con $|H|=2$, demostrar que $N_G(H)=C_G(H)$. Deducir que $H\lhd G$ si y sólo si $H\subset Z(G)$.
    \end{enumerate}
\end{ejercicio}

\begin{ejercicio}
    Sea $G$ un grupo arbitrario. Para dos elementos $x,y\in G$ se define su \emph{conmutador} como el elemento
    \[
        [x,y] = xyx^{-1}y^{-1}.
    \]
    \begin{observacion}
        (El conmutador recibe tal nombre porque $[x,y]yx=xy$.)
    \end{observacion}
    
    Como $[x,y]^{-1}=[y,x]$, el inverso de un conmutador es un conmutador. Sin embargo el producto de dos conmutadores no tiene porqué ser un conmutador. Entonces se define el \emph{subgrupo conmutador} o (primer) \emph{subgrupo derivado} de $G$, denotado $[G,G]$, como el subgrupo generado por todos los conmutadores de $G$.
    \begin{enumerate}
        \item Demostrar que, $\forall a,x,y\in G$, se tiene que $a[x,y]a^{-1}=[axa^{-1},aya^{-1}]$.
        \item Demostrar que $[G,G]\lhd G$.
        \item demostrar que el grupo cociente $G/[G,G]$, que se representa por $G^{ab}$, es un grupo abeliano (que se llama el abelianizado de $G$).
        \item Demostrar que $G$ es abeliano si y sólo si $[G,G]=1$.
        \item Sea $N\lhd G$. Demostrar que el grupo cociente $G/N$ es abeliano si y sólo si $N>[G,G]$ (así que el grupo $[G,G]$ es el menor subgrupo normal de $G$ tal que el cociente es abeliano).
    \end{enumerate}
\end{ejercicio}

\begin{ejercicio}~
    \begin{enumerate}
        \item Calcular el subgrupo conmutador de los grupos $S_3$, $A_4$, $D_4$ y $Q_2$.
        \item Demostrar que, para $n\geq 3$, el subgrupo conmutador de $S_n$ es $A_n$ y que éste es el único subgrupo de $S_n$ de orden $\nicefrac{n!}{2}$.
    \end{enumerate}
\end{ejercicio}
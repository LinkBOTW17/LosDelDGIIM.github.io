\section{Grupos resolubles}

\begin{ejercicio}
    Sea $N\lhd G$ un subgrupo normal y simple de un grupo $G$. Demostrar que si $G/N$ tiene una serie de composición entonces $G$ tiene una serie de composición.\\
    
    Consideramos la serie de composición de $G/N$:
    \[
        G/N = N_0 \rhd N_1 \rhd \cdots \rhd N_{r-1} \rhd N_r = \{1N\}.
    \]

    Por el Tercer Teorema de Isomorfía, como $N_i<G/N$ para todo $i\in \{0,1,\ldots,r\}$ existe $G_i<G$ tal que $N\lhd G_i$ cumpliendo que:
    \begin{equation*}
        N_i = G_i/N \quad \forall i\in \{0,1,\ldots,r\}.
    \end{equation*}

    Para considerar la serie buscada, hemos de probar que $G_i < G_{i-1}$ para todo $i\in \{1,\ldots,r\}$. Como $N_i\subset N_{i-1}$ por la biyección del Tercer Teorema de Isomorfía se tiene que $G_i\subset G_{i-1}$; y como $G_i$ es un grupo, se tiene que $G_i<G_{i-1}$. Consideramos por tanto la siguiente serie (donde notemos que hemos añadido el $\{1\}$):
    \[
        G = G_0 > G_1 > \cdots > G_{r-1} > G_r = N > \{1\}
    \]

    Nos falta ahora por ver que $G_i \lhd G_{i-1}$ para todo $i\in \{1,\ldots,r\}$. Por el Tercer Teorema de Isomorfía, como $G_i/N \lhd G/N$, se tiene que $G_i \lhd G$. Como además $G_i<G_{i-1}$, se tiene que $G_i \lhd G_{i-1}$. Por último, sabemos que $\{1\} \lhd N$.    
    Por tanto, consideramos la siguiente serie normal:
    \[
        G = G_0 \rhd G_1 \rhd \cdots \rhd G_{r-1} \rhd G_r = N \rhd \{1\}.
    \]

    Veamos que dicha serie es de composición. Para ello, hemos de ver que los factores son simples. Por ser la serie de partida de composición, sabemos que el siguiente grupo cociente es simple:
    \begin{equation*}
        \dfrac{G_{i-1}/N}{G_i/N}\qquad \forall i\in \{1,\ldots,r\}
    \end{equation*}

    Por el Tercer Teorema de Isomorfía, se tiene que:
    \begin{equation*}
        \dfrac{G_{i-1}}{G_i} \cong \dfrac{G_{i-1}/N}{G_i/N}\qquad \forall i\in \{1,\ldots,r\}
    \end{equation*}

    Como el ``ser simple'' es una propiedad que se conserva bajo isomorfismos, se tiene que:
    \begin{equation*}
        G_{i-1}/G_i \text{ es simple } \forall i\in \{1,\ldots,r\}.
    \end{equation*}

    Falta por compronar que $N/\{1\}$ es simple, algo que se tiene de forma directa puesto que $N/\{1\}\cong N$ y $N$ es simple. Por tanto, la serie
    \[
        G = G_0 \rhd G_1 \rhd \cdots \rhd G_{r-1} \rhd G_r = N \rhd \{1\}
    \]
    tiene todos sus factores simples, y por tanto es de composición. Notemos además que $l(G)=l(G/N)+1$.    
\end{ejercicio}

\begin{ejercicio}
    Sea $G$ un grupo abeliano. Demostrar que $G$ tiene series de composición si y sólo si $G$ es finito.
    \begin{description}
        \item[$\Longleftarrow$)] Si $G$ es finito, entonces hemos visto que $G$ tiene series de composición.
        
        \item[$\Longrightarrow$)] Si $G$ tiene una serie de composición, veamos ahora que $G$ es finito. Consideramos la serie de composición:
        \[
            G = G_0 \rhd G_1 \rhd \cdots \rhd G_{r-1} \rhd G_r = \{1\}.
        \]

        Como $G$ es abeliano, todos sus subgrupos son abelianos, y por tanto todos sus factores son abelianos. Además, por ser serie de composición, todos los factores son simples. Por la caracterización de los grupos abelianos y simples, todos los factores son de orden primo (en particular, finitos).\\

        A continuación, desarrollamos la siguiente idea. Dado un grupo $A$ y un subgrupo suyo $B\lhd A$, si $B$ es finito y $A/B$ es finito, entonces $A$ es finito. Notemos que a priori no podemos aplicar el Teorema de Lagrange, puesto que $A$ no es necesariamente finito. Sin embargo como las clases de equivalencia del cociente $A/B$ forman una partición de $A$, se tiene que:
        \begin{equation*}
            A=\bigcup_{i=1}^{|A/B|} a_iB
        \end{equation*}
        Como $B$ es finito, entonces $|a_iB|=|B|$ para todo $i\in \{1,\ldots,|A/B|\}$; luego $|A|=|B|\cdot |A/B|$, y en particular $A$ es finito.\\

        Aplicando esta idea a la serie de composición de $G$, obtenemos en primer que $G_{r-1}$ es finito, puesto que $G_{r-1}/\{1\}$ y $\{1\}$ son finitos. Análogamente, $G_{r-2}$ es finito, puesto que $G_{r-2}/G_{r-1}$ y $G_{r-1}$ son finitos. Por inducción sobre $r$, se tiene que $G_{0}=G$ es finito.
    \end{description}
\end{ejercicio}

\begin{ejercicio}\label{ej:5.3}
    Sea $H$ un subgrupo normal de un grupo finito $G$. Demostrar que existe una serie de composición de $G$ uno de cuyos términos es $H$.\\

    Como $G$ es finito, entonces $G$ tiene una serie de composición. Consideramos ahora la siguiente serie normal:
    \[
        G \rhd H \rhd \{1\}.
    \]

    Como $G$ admite una serie de composición, por el Teorema de Jordan-Holder dicha serie normal puede refinarse a una serie de composición.
\end{ejercicio}

\begin{ejercicio}\label{ej:5.4}
    Se define la longitud de un grupo finito $G$, denotada $l(G)$, como la longitud de cualquiera de sus series de composición. Demostrar que si $H$ es un subgrupo normal de $G$ entonces:
    \begin{equation*}
        l(G) = l(H) + l(G/H)\qquad \fact(G)=\fact(H)\cup\fact(G/H).
    \end{equation*}
    \begin{observacion}
        Notemos que los factores de composición de $G$ no tienen por qué ser únicos, por lo que a priori no podemos hablar de $\fact(G)$ como un conjunto. No obstante, son únicos salvo isomorfismos (y reordenamientos, pero al trabajar con conjuntos no es necesario tener en cuenta el orden). Por tanto, dos conjuntos $\fact(G)$ pueden ser distintos, pero sus elementos son isomorfos entre sí.
    \end{observacion}

    Por el Ejercicio~\ref{ej:5.3}, $G$ tiene una serie de composición que contiene a $H$. Consideramos la serie de composición:
    \[
        G = G_0 \rhd G_1 \rhd \cdots \rhd G_{r-1} \rhd G_r = H \rhd G_{r+1} \rhd \cdots \rhd G_{r+m-1} \rhd G_{r+m} = \{1\}.
    \]

    Vemos claramente que $l(G)=r+m$ y $l(H)=r+m-r=m$. Además:
    \begin{equation*}
        \fact(H)=\bigcup_{i=r+1}^{r+m-1} G_i/G_{i+1} \qquad \fact(G)=\bigcup_{i=0}^{r+m-1} G_i/G_{i+1}.
    \end{equation*}    
    
    
    Hemos de calcular ahora una serie de composición de $G/H$. Como $H\lhd G$ se tiene que $H\lhd G_i$ para todo $i\in \{0,1,\ldots,r\}$. Por el Tercer Teorema de Isomorfía, como $G_{i-1}\rhd G_i$, se tiene que $G_{i-1}/H \rhd G_i/H$ para todo $i\in \{1,\ldots,r\}$. Por tanto, la serie
    \[
        G/H = G_0/H \rhd G_1/H \rhd \cdots \rhd G_{r-1}/H \rhd G_r/H = H/H = \{1H\}
    \]
    es una serie normal de $G/H$. Además, por el Tercer Teorema de Isomorfía, se tiene que:
    \begin{equation*}
        \dfrac{G_{i-1}/H}{G_i/H} \cong {G_{i-1}/}{G_i} \qquad \forall i\in \{1,\ldots,r\}.
    \end{equation*}
    Como $G_{i-1}/G_i$ es simple por ser un factor de composición, y los factores se conservan bajo isomorfismos, se tiene que los factores de la serie de composición de $G/H$ son simples. Por tanto, la serie de $G/H$ es de composición, luego se cumple que $l(G/H)=r$ y se tiene que:
    \begin{equation*}
        l(H) + l(G/H) = m + r = l(G).
    \end{equation*}

    Por otro lado, los factores de la serie de composición de $G/H$ son isomorfos por el Tercer Teorema de Isomorfía a:
    \begin{equation*}
        \fact(G/H)=\bigcup_{i=0}^r G_i/G_{i+1}
    \end{equation*}

    Por tanto, se tiene que:
    \begin{equation*}
        \fact(G/H)\cup \fact(H) = \bigcup_{i=0}^{r+m-1} G_i/G_{i+1} = \fact(G).
    \end{equation*}
\end{ejercicio}

\begin{ejercicio}
    Encontrar todas las series de composición, calcular la longitud y la lista de factores de composición de los siguientes grupos:
    \begin{enumerate}
        \item El grupo diédrico $D_4$.
        
        Conviene tener presente el Diagrama de Hasse de $D_4$, presente en la Figura~\ref{fig:ej11_D4}. Simplemente lo usaremos para buscar todas las series normales de $D_4$ que no admitan refinamientos, consiguiendo así todas las series de composición. Para ello, iremos desde $D_4$ hasta $\{1\}$ por el grafo del retículo sin saltarnos vértices (evitando así los refinamientos) y yendo solo por los subgrupos normales en el anterior.\\

        En este caso, como todos los índices de un grupo en su subgrupo adyacente son $2$, todas las relaciones de inclusión dadas en el grafo son en realidad de normalidad. De hecho, todas las series de composición son las siguientes:
        \begin{align*}
            D_4 &\rhd \langle r^2, s \rangle \rhd  \langle s \rangle  \rhd \{1\} \\
            D_4 &\rhd \langle r^2, s \rangle \rhd \langle sr^2 \rangle  \rhd \{1\} \\
            D_4 &\rhd \langle r^2, s \rangle \rhd \langle r^2 \rangle  \rhd \{1\} \\
            D_4 &\rhd \langle r^2, sr \rangle  \rhd \langle r^2 \rangle  \rhd \{1\} \\
            D_4 &\rhd \langle r^2, sr \rangle  \rhd \langle sr \rangle  \rhd \{1\} \\
            D_4 &\rhd \langle r^2, sr \rangle  \rhd \langle sr^3 \rangle  \rhd \{1\} \\
            D_4 &\rhd \langle r \rangle \rhd \langle r^2 \rangle  \rhd \{1\}
        \end{align*}
        \item El grupo alternado $A_4$.
        
        El Diagrama de Hasse de $A_4$ está presente en la Figura~\ref{fig:ej11_A4}. Además, se vió que el único subgrupo normal propio de $A_4$ es $V$. Por tanto, las series de composición son las siguientes:
        \begin{align*}
            A_4 &\rhd V \rhd \langle (1\ 2)(3\ 4) \rangle  \rhd \{1\} \\
            A_4 &\rhd V \rhd \langle (1\ 3)(2\ 4) \rangle  \rhd \{1\} \\
            A_4 &\rhd V \rhd \langle (1\ 4)(2\ 3) \rangle  \rhd \{1\}
        \end{align*}
        \item El grupo simétrico $S_4$.
        
        En primer lugar, sabemos que las siguientes son series de composición de $S_4$:
        \begin{align*}
            S_4 &\rhd A_4 \rhd V \rhd \langle (1\ 2)(3\ 4) \rangle  \rhd \{1\} \\
            S_4 &\rhd A_4 \rhd V \rhd \langle (1\ 3)(2\ 4) \rangle  \rhd \{1\} \\
            S_4 &\rhd A_4 \rhd V \rhd \langle (1\ 4)(2\ 3) \rangle  \rhd \{1\}
        \end{align*}

        No obstante, podría suceder que tuviese más grupos normales. Supongamos que existe $N\lhd S_4$ tal que $N\neq A_4$ y $N\neq \{1\}$.
        \begin{itemize}
            \item Si este contiene a un $n-$ciclo $\gamma\in N$, veamos que contiene a todos los $n-$ciclos. Dado otro $n-$ciclo $\sigma\in S_4$, sean:
            \begin{align*}
                \gamma &= (x_1\ \dots\ x_n)\in N \\
                \sigma &= (y_1\ \dots\ y_n)\in S_4
            \end{align*}

            Definimos ahora $\tau\in S_4$ como:
            \begin{align*}
                \tau(x_n) &= y_n\qquad \forall n\in \{1,\ldots,n\} \\
                \tau(k) &= k\qquad \forall k\in \{1,\ldots,4\}\setminus\{x_1,\ldots,x_n\}
            \end{align*}

            De esta forma, tenemos que $\sigma=\tau\gamma\tau^{-1}$. Como $N$ es normal, se tiene que $\sigma=\tau\gamma\tau^{-1}\in N$. Por tanto, $N$ contiene todos los $n-$ciclos.

            \item Si $N$ contiene un producto de dos transposiciones disjuntas $\gamma\in N$, veamos que contiene a todos los productos de dos transposiciones disjuntas. Sea $\gamma=\gamma_1\gamma_2\in N$ un producto de dos transposiciones disjuntas, y sea $\sigma=\sigma_1\sigma_2\in S_4$ un producto de dos transposiciones disjuntas.
            \begin{align*}
                \gamma &= \gamma_1\gamma_2=(x_1\ x_2)(x_3\ x_4)\in N \\
                \sigma &= \sigma_1\sigma_2=(y_1\ y_2)(y_3\ y_4)\in S_4
            \end{align*}
            
            Definimos $\tau\in S_4$ como:
            \begin{align*}
                \tau(x_n) &= y_n\qquad \forall n\in \{1,\ldots,4\} \\
                \tau(k) &= k\qquad \forall k\in \{1,\ldots,4\}\setminus\{x_1,x_2,y_1,y_2\}
            \end{align*}
            De esta forma, tenemos que:
            \begin{equation*}
                \tau\gamma\tau^{-1} = \tau\gamma_1\tau^{-1}\tau\gamma_2\tau^{-1} = (\tau(x_1)\ \tau(x_2))(\tau(x_3)\ \tau(x_4)) = (y_1\ y_2)(y_3\ y_4)=\sigma\in N.
            \end{equation*}
            Por tanto, $N$ contiene todos los productos de dos transposiciones disjuntas.
        \end{itemize}
        Este es el concepto de clase de conjugación, concepto que no se ha tratado pero no es difícil de entender. En $S_4$ hay:
        \begin{itemize}
            \item $1$ $1-$ciclo (la identidad).
            \item $6$ $2-$ciclos.
            \item $8$ $3-$ciclos.
            \item $6$ $4-$ciclos.
            \item $3$ productos de dos transposiciones disjuntas.
        \end{itemize}

        Efecetivamente, se tiene que $|A_4|=12=1+8+3$. Sea entonces $N$ un subgrupo normal propio de $S_4$.
        \begin{itemize}
            \item Supongamos que $N$ contiene un $2-$ciclo. Entonces, $|N|\geq 1+6=7$. Como $|N|$ es divisor de $|S_4|=24$, se tiene que $|N|=12$, luego faltarían $5$ elementos. No obstante, esto no es posible (puesto que no hay ninguna clase de conjugación con $5$ elementos). Por tanto, ningún $2-$ciclo pertenece a $N$.
            \item De forma análoga, se ve que no hay $4-$ciclos en $N$.
        \end{itemize}
        Como no hay $2-$ciclos ni $4-$ciclos, se tiene que $N\subset A_4$. Como $N$ es un grupo, se tiene que $N<A_4$. Si $N$ no es normal en $A_4$, entonces tampoco lo es en $S_4$, por lo que $N$ es normal en $A_4$ y entonces será necesario pasar por $A_4$ en la serie de composición.
        Por tanto, las únicas series de composición de $S_4$ son las anteriormente vistas:
        \begin{align*}
            S_4 &\rhd A_4 \rhd V \rhd \langle (1\ 2)(3\ 4) \rangle  \rhd \{1\} \\
            S_4 &\rhd A_4 \rhd V \rhd \langle (1\ 3)(2\ 4) \rangle  \rhd \{1\} \\
            S_4 &\rhd A_4 \rhd V \rhd \langle (1\ 4)(2\ 3) \rangle  \rhd \{1\}
        \end{align*}
        \item El grupo diédrico $D_5$.
        
        Calculamos el orden de cada elemento de $D_5$:
        \begin{align*}
            O(r) &= O(r^2) = O(r^3) = O(r^4) = 5 \\
            O(s) &= O(sr) = O(sr^2) = O(sr^3) = O(sr^4) = 2
        \end{align*}

        Todo subgrupo de $D_5$ será de orden primo, luego será cíclico. El diagrama de Hasse de $D_5$ está presente en la Figura~\ref{fig:Hasse_D5}.
        \begin{figure}
            \centering
            \begin{tikzpicture}[node distance=1cm]
                \node (D5) {$D_5$};
                \node (r) [below left=of D5] {$\langle r \rangle$};
                \node (s) [below right=of r] {$\langle s \rangle$};
                \node (sr) [right=of s] {$\langle sr \rangle$};
                \node (sr2) [right=of sr] {$\langle sr^2 \rangle$};
                \node (sr3) [right=of sr2] {$\langle sr^3 \rangle$};
                \node (sr4) [right=of sr3] {$\langle sr^4 \rangle$};
                \node (1) [below=of s] {$\{1\}$};


                \draw (D5) -- (r);
                \draw (D5) -- (s);
                \draw (D5) -- (sr);
                \draw (D5) -- (sr2);
                \draw (D5) -- (sr3);
                \draw (D5) -- (sr4);
                \draw (r) -- (1);
                \draw (s) -- (1);
                \draw (sr) -- (1);
                \draw (sr2) -- (1);
                \draw (sr3) -- (1);
                \draw (sr4) -- (1);
            \end{tikzpicture}    
            \caption{Diagrama de Hasse para los subgrupos del grupo $D_5$.}
            \label{fig:Hasse_D5}
        \end{figure}     
        
        Veamos que no los de orden $2$ no son normales.
        \begin{itemize}
            \item $r\ s\ r^4 = sr^3 \notin \langle s \rangle$.
            \item $r\ sr \ r^4 = sr^4 \notin \langle sr \rangle$.
            \item $r\ sr^2 \ r^4 = sr^{10} = s \notin \langle sr^2 \rangle$.
            \item $r\ sr^3 \ r^4 = sr^{11} = sr \notin \langle sr^3 \rangle$.
            \item $r\ sr^4 \ r^4 = sr^{12} = sr^2 \notin \langle sr^4 \rangle$.
        \end{itemize}

        Por tanto, el único subgrupo normal de $D_5$ es $\langle r\rangle$. Por tanto, la única serie de composición es la siguiente:
        \begin{align*}
            D_5 &\rhd \langle r \rangle \rhd \{1\}
        \end{align*}
        \item El grupo de cuaterniones $Q_2$.
        
        El Diagrama de Hasse de $Q_2$ está presente en la Figura~\ref{fig:ej11_Q2}. Como todos los índices son $2$, todas las relaciones de inclusión son de normalidad. Por tanto, las series de composición son las siguientes:
        \begin{align*}
            Q_2 &\rhd \langle i \rangle \rhd \{-1\} \rhd \{1\} \\
            Q_2 &\rhd \langle j \rangle \rhd \{-1\} \rhd \{1\} \\
            Q_2 &\rhd \langle k \rangle \rhd \{-1\} \rhd \{1\} \\
        \end{align*}
        \item El grupo cíclico $C_{24}$.
        
        Sabemos que los subgrupos de $C_{24}$ son cíclicos, y por tanto abelianos. Por tanto, todos los subgrupos son normales. El Diagrama de Hasse de $C_{24}$ está presente en la Figura~\ref{fig:Hasse_C24}.
        \begin{figure}
            \centering
            \begin{tikzpicture}[node distance=1cm]
                \node (C24) {$C_{24}$};
                \node (C12) [below right=of C24, xshift=-1cm, yshift=0.2cm] {$\langle x^2 \rangle \cong C_{12}$};
                \node (C8) [below left=of C12, yshift=0.5cm] {$\langle x^3 \rangle \cong C_8$};
                \node (C6) [below right=of C8, yshift=0.5cm] {$\langle x^4 \rangle \cong C_6$};
                \node (C4) [below left=of C6, yshift=0.5cm] {$\langle x^6 \rangle \cong C_4$};
                \node (C3) [below right=of C4, yshift=0.5cm] {$\langle x^8 \rangle \cong C_3$};
                \node (C2) [below left=of C3, yshift=0.5cm] {$\langle x^{12} \rangle \cong C_2$};
                \node (C1) [below=of C2, xshift=1cm, yshift=0.5cm] {$\{1\}$};

                \draw (C24) -- (C12) -- (C6) -- (C3) -- (C1);
                \draw (C24) -- (C8) -- (C4) -- (C2) -- (C1);

                \draw (C12) -- (C4);
                \draw (C6) -- (C2);

            \end{tikzpicture}
            \caption{Diagrama de Hasse para los subgrupos del grupo $C_{24}$.}
            \label{fig:Hasse_C24}
        \end{figure}
        
        Las series de composición son, por tanto, las siguientes:
        \begin{align*}
            C_{24} &\rhd \langle C_{12} \rangle \rhd \langle C_{6} \rangle \rhd \langle C_{3} \rangle \rhd \{1\} \\
            C_{24} &\rhd \langle C_{12} \rangle \rhd \langle C_{6} \rangle \rhd \langle C_{2} \rangle \rhd \{1\} \\
            C_{24} &\rhd \langle C_{12} \rangle \rhd \langle C_{4} \rangle \rhd \langle C_{2} \rangle \rhd \{1\} \\
            C_{24} &\rhd \langle C_{8} \rangle \rhd \langle C_{4} \rangle \rhd \langle C_{2} \rangle \rhd \{1\}
        \end{align*}
        \item El grupo simétrico $S_5$.
        
        Como $A_5$ es normal en $S_5$, se tiene que la siguiente es una serie normal:
        \begin{align*}
            S_5 &\rhd A_5 \rhd \{1\}
        \end{align*}

        Además, $S_5/A_5$ es simple por ser de orden primo, mientras que $A_5/\{1\}\cong A_5$ es simple por el Lema de Abel. Por tanto, la serie es de composición.

        No obstante, podría suceder que tuviese más grupos normales. Supongamos que existe $N\lhd S_5$ tal que $N\neq A_5$ y $N\neq \{1\}$. Por una demostración análoga a la de $S_4$, las clases de conjugación de $S_5$ son las siguientes:
        \begin{itemize}
            \item $1$ $1-$ciclo (la identidad).
            \item $10$ $2-$ciclos.
            \item $20$ $3-$ciclos.
            \item $30$ $4-$ciclos.
            \item $24$ $5-$ciclos.
            \item $15$ productos de dos transposiciones disjuntas.
            \item $20$ productos de un $2-$ciclo y un $3-$ciclo.
        \end{itemize}

        Efecetivamente, se tiene que $|A_5|=60=1+20+24+15$. Sea entonces $N$ un subgrupo normal propio de $S_4$.
        \begin{itemize}
            \item Supongamos que $N$ contiene un $2-$ciclo. Entonces, $|N|\geq 1+10=11$, que no divide a $120$. Como la siguiente clase de conjugación más pequeña es de $15$ elementos, sabemos que $|N|>26$. Por tanto, $|N|\in \{30,40,60\}$. Para, desde $11$ podemos sumár un múltiplo de $10$, es necesario que contenga a los $24$ $5-$ciclos y a los $15$ productos de dos transposiciones disjuntas, luego $|N|\geq 11+15+24=50$, luego $|N|=60$, por lo que tan solo nos falta por determinar $10$ elementos. No obstante, todas las clases restantes son de más de $10$ elementos. Por tanto, no puede contener ningún $2-$ciclo.
            
            \item Supongamos que $N$ contiene un $4-$ciclo. Entonces, $|N|\geq 1+30=31$, que no divide a $120$. Por tanto, $|N|\in \{40,60\}$. Para, desde $31$ podemos sumar un múltiplo de $10$, es necesario que contenga a los $24$ $5-$ciclos y a los $15$ productos de dos transposiciones disjuntas, pero $31+24+15>60$. Por tanto, no puede contener ningún $4-$ciclo.
            

            \item Supongamos que $N$ contiene un producto de un $2-$ciclo y un $3-$ciclo. Entonces, $|N|\geq 1+20=21$, que no divide a $120$. Como la siguiente clase de conjugación más pequeña es de $10$ elementos, sabemos que $|N|>31$. Por tanto, $|N|\in \{40,60\}$. Para, desde $21$ podemos sumar un múltiplo de $10$, es necesario que contenga a los $24$ $5-$ciclos y a los $15$ productos de dos transposiciones disjuntas, luego $|N|\geq 21+15+24=60$, luego $|N|=60$. Por tanto, $N$ está formado por:
            \begin{itemize}
                \item $1$ $1-$ciclo (la identidad).
                \item $20$ productos de un $2-$ciclo y un $3-$ciclo.
                \item $24$ $5-$ciclos.
                \item $15$ productos de dos transposiciones disjuntas.
            \end{itemize}

            No obstante, veamos que $N$ no es un subgrupo de $S_5$ puesto que no es cerrado por producto:
            \begin{equation*}
                (1\ 2)(3\ 4\ 5)\ (1\ 2)(3\ 4) = (1\ 2)(1\ 2)(3\ 4\ 5)(3\ 4)
                = (3\ 4\ 5)(3\ 4) = (3\ 5)\notin N
            \end{equation*}
            Por tanto, no puede contener ningún producto de un $2-$ciclo y un $3-$ciclo.
        \end{itemize}
        Por tanto, $N\subset A_5$. Como $N$ es un grupo, se tiene que $N<A_5$. Si $N$ no es normal en $A_5$, entonces tampoco lo es en $S_5$, por lo que $N$ es normal en $A_5$. No obstante, $A_5$ es simple, luego $N=A_5$.
        Por tanto, la única serie de composición de $S_5$ es la siguiente:
        \begin{align*}
            S_5 &\rhd A_5 \rhd \{1\}
        \end{align*}
    \end{enumerate}
\end{ejercicio}

\begin{ejercicio}\label{ej:5.6}
    Sea $G$ un grupo finito, y
    \[
        G = G_0 \rhd G_1 \rhd \cdots \rhd G_{r-1} \rhd G_r = \{1\}
    \]
    una serie normal de $G$. Demostrar que
    \[
        l(G) = \sum_{i=0}^{r-1} l\left(\frac{G_i}{G_{i+1}}\right), \quad \fact(G) = \bigcup_{i=0}^{r-1} \fact\left(\frac{G_i}{G_{i+1}}\right).
    \]

    Como $G$ es finito y $G_1\lhd G$, por el Ejercicio~\ref{ej:5.4} se tiene que:
    \begin{align*}
        l(G) &= l(G_1) + l(G/G_1) \\
        \fact(G) &= \fact(G_1)\cup\fact(G/G_1).
    \end{align*}

    Como $G_1$ es finito y $G_2\lhd G_1$, por el Ejercicio~\ref{ej:5.4} se tiene que:
    \begin{align*}
        l(G) = l(G_1) + l(G/G_1) &= l(G_2) + l(G_1/G_2) + l(G/G_1) \\
        &= l(G_2) + l(G_1/G_2) + l(G/G_1) \\
        \fact(G) &= \fact(G_1)\cup\fact(G/G_1) = \fact(G_2)\cup\fact(G_1/G_2)\cup\fact(G/G_1).
    \end{align*}

    Iterando hasta usar que $G_{r}\lhd G_{r-1}$, se tiene que:
    \begin{align*}
        l(G) &= \sum_{i=0}^{r-1} l(G_i/G_{i+1}) + \cancel{l(G_r)} \\
        \fact(G) &= \bigcup_{i=0}^{r-1} \fact(G_i/G_{i+1}) \cup \cancel{\fact(G_r)}.
    \end{align*}
\end{ejercicio}

\begin{ejercicio}\label{ej:5.7}
    Si $G_1, G_2, \ldots, G_r$ son grupos finitos, demostrar que
    \[
        l(G_1 \times G_2 \times \cdots \times G_r) = \sum_{i=1}^{r} l(G_i), \quad \fact(G_1 \times G_2 \times \cdots \times G_r) = \bigcup_{i=1}^{r} \fact(G_i).
    \]

    Demostramos por inducción sobre $r$.
    \begin{itemize}
        \item Para $r=1$ se tiene trivialmente.
        \item Supuesto cierto para $r$, demostrémoslo para $r+1$.\\
        
        Buscamos demostrarlo aplicando el Ejercicio~\ref{ej:5.4}. Para ello, necesitamos un subgrupo normal de $G_1 \times \cdots \times G_r \times G_{r+1}$. Definimos:
        \Func{\pi}{G_1 \times \cdots \times G_r \times G_{r+1}}{G_1\times \cdots \times G_r}{(g_1, g_2, \ldots, g_r, g_{r+1})}{(g_1, g_2, \ldots, g_r)}

        Tenemos que $\pi$ es un homomorfismo con:
        \begin{align*}
            \ker(\pi) &= \{1\}\times \cdots \times \{1\} \times G_{r+1} \\
            Im(\pi) &= G_1\times \cdots \times G_r.
        \end{align*}

        Por el Primer Teorema de Isomorfía, se tiene que:
        \begin{align*}
            \dfrac{G_1 \times \cdots \times G_r \times G_{r+1}}{\{1\}\times \cdots \times \{1\} \times G_{r+1}} &\cong G_1\times \cdots \times G_r.
        \end{align*}

        Veamos ahora que $\{1\}\times \cdots \times \{1\} \times G_{r+1}$ es isomorfo a $G_{r+1}$. Definimos:
        \Func{\phi}{\{1\}\times \cdots \times \{1\} \times G_{r+1}}{G_{r+1}}{(1, \ldots, 1, g_{r+1})}{g_{r+1}}

        Vemos claramente que $\phi$ es un isomorfismo, luego $\{1\}\times \cdots \times \{1\} \times G_{r+1} \cong G_{r+1}$.\\

        Vistos ambos aspectos, como $\{1\}\times \cdots \times \{1\} \times G_{r+1}=\ker(\pi) \lhd G_1 \times \cdots \times G_r \times G_{r+1}$ por el Ejercicio~\ref{ej:5.4}, se tiene que:
        \begin{align*}
            l(G_1 \times \cdots \times G_r \times G_{r+1}) &= l(\{1\}\times \cdots \times \{1\} \times G_{r+1}) + l\left(\dfrac{G_1 \times \cdots \times G_r \times G_{r+1}}{\{1\}\times \cdots \times \{1\} \times G_{r+1}}\right)
        \end{align*}

        Como las series de composición de dos grupos isomorfas son isomorfas, tenemos que:
        \begin{align*}
            l(G_1 \times \cdots \times G_r \times G_{r+1}) &= l(G_{r+1}) + l\left(G_1 \times \cdots \times G_r\right) \AstIg
             l(G_{r+1}) + \sum_{i=1}^{r} l(G_i)
            = \sum_{i=1}^{r+1} l(G_i).
        \end{align*}
        donde en $(\ast)$ hemos usado la hipótesis de inducción.\\

        De igual forma, usando de nuevo el Ejercicio~\ref{ej:5.4} se tiene que:
        \begin{align*}
            \fact(G_1 \times \cdots \times G_r \times G_{r+1}) &= \fact(\{1\}\times \cdots \times \{1\} \times G_{r+1}) \cup \fact\left(\dfrac{G_1 \times \cdots \times G_r \times G_{r+1}}{\{1\}\times \cdots \times \{1\} \times G_{r+1}}\right)\\
            &\AstIg \fact(G_{r+1})\cup\fact(G_1\times\cdots\times G_r)\\
            &\stackrel{(\ast\ast)}{=} \fact(G_{r+1})\cup \bigcup_{i=1}^{r} \fact(G_i) = \bigcup_{i=1}^{r+1} \fact(G_i).
        \end{align*}
        donde en $(\ast\ast)$ hemos usado la hipótesis de inducción y en $(\ast)$ hemos empleado que las series de composición de dos grupos isomorfas son isomorfas, luego sus factores de composición son isomorfos y por tanto el conjunto $\fact$ de ambos grupos es el mismo (salvo la observacón que hicimos de isomorfismos en el Ejercicio~\ref{ej:5.4}).
    \end{itemize}

    Por tanto, se ha demostrado el resultado por inducción.
\end{ejercicio}

\begin{ejercicio}\label{ej:5.8}
    Sea $G$ un grupo cíclico de orden $p^n$ con $p$ primo. Demostrar que $l(G) = n$ y que $\fact(G) = (\bb{Z}_p, \bb{Z}_p, \stackrel{(n)}{\ldots}, \bb{Z}_p)$ ($n$ veces).\\

    Conviene tener presente el diagrama de Hasse de los subgrupos de $G=\langle g\rangle$, presente en la Figura~\ref{fig:ej12}. Además, como $G$ es cíclico, en particular es abeliano y todos sus subgrupos son abelianos, luego todas las relaciones de inclusión son de normalidad. Por tanto, la única serie de composición es la siguiente:
    \begin{align*}
        G=\langle g^{p^0}\rangle &\rhd \langle g^{p} \rangle \rhd \langle g^{p^2} \rangle \rhd \cdots \rhd \langle g^{p^{n-1}} \rangle \rhd \langle g^{p^n} \rangle = \{1\}
    \end{align*}

    De esta serie de composición se deduce que $l(G)=n$. Veamos cuáles son los factores de composición:
    \begin{equation*}
        \left|\dfrac{\langle g^{p^i} \rangle}{\langle g^{p^{i+1}} \rangle}\right| = \dfrac{|\langle g^{p^i} \rangle|}{|\langle g^{p^{i+1}} \rangle|} = \dfrac{O(g^{p^i})}{O(g^{p^{i+1}})} = \dfrac{\nicefrac{p^n}{\mcd(p^n, p^i)}}{\nicefrac{p^n}{\mcd(p^n, p^{i+1})}} = \dfrac{\mcd(p^n, p^{i+1})}{\mcd(p^n, p^i)} = \dfrac{p^{i+1}}{p^i} = p.\quad \forall i\in\{0,\ldots,n-1\}
    \end{equation*}

    Por tanto, se tiene que:
    \begin{equation*}
        \dfrac{\langle g^{p^i} \rangle}{\langle g^{p^{i+1}} \rangle} \cong \bb{Z}_p\qquad \forall i\in\{0,\ldots,n-1\}
    \end{equation*}

    Por tanto, los factores de composición son:
    \begin{equation*}
        \fact(G) = \left(\bb{Z}_p, \bb{Z}_p, \stackrel{(n)}{\ldots}, \bb{Z}_p\right).
    \end{equation*}
\end{ejercicio}

\begin{ejercicio}\label{ej:5.9}
    Sea $G$ un grupo cíclico de orden $n$. Si la descomposición de $n$ en factores primos es $n = p_1^{e_1} p_2^{e_2} \cdots p_r^{e_r}$, demostrar que
    \[
        l(G) = e_1 + e_2 + \cdots + e_r,
    \]
    y que
    \[
        \fact(G) = (\bb{Z}_{p_1}, \stackrel{(e_1)}{\ldots}\bb{Z}_{p_1}, \dots, \bb{Z}_{p_r},\stackrel{(e_r)}{\ldots}, \bb{Z}_{p_r}).
    \]
    Aplica el resultado cuando $n = 12$ y compara su longitud y factores de composición con los del grupo $\bb{Z}_2 \times \bb{Z}_6$.\\

    Sabemos que $\mcd(p_1,\dots,p_r)=1$, luego $\mcd(p_1^{e_1},\dots,p_r^{e_r})=1$. Por tanto, se tiene que:
    \begin{equation*}
        \prod_{i=1}^{r} C_{p_i^{e_i}}\qquad \text{es cíclico}
    \end{equation*}

    Además, se tiene que:
    \begin{equation*}
        \left|\prod_{i=1}^{r} C_{p_i^{e_i}}\right| = \prod_{i=1}^{r} |C_{p_i^{e_i}}| = \prod_{i=1}^{r} p_i^{e_i} = n.
    \end{equation*}

    Por tanto, $G\cong \prod\limits_{i=1}^{r} C_{p_i^{e_i}}$. Como dos grupos isomorfos tienen series de composición isomorfas, se tiene que:
    \begin{align*}
        l(G) &= l\left(\prod_{i=1}^{r} C_{p_i^{e_i}}\right) 
        \AstIg \sum_{i=1}^{r} l\left(C_{p_i^{e_i}}\right) 
        \stackrel{(\ast\ast)}{=} \sum_{i=1}^{r} e_i
    \end{align*}
    donde en $(\ast)$ hemos usado el Ejercicio~\ref{ej:5.7} y en $(\ast\ast)$ el Ejercicio~\ref{ej:5.8}.\\

    Veamos ahora cuáles son los factores de composición. Como las series de composición de dos grupos isomorfos son isomorfas y, por tanto, sus factores de composición son isomorfos, se tiene que:
    \begin{align*}
        \fact(G) &= \fact\left(\prod_{i=1}^{r} C_{p_i^{e_i}}\right) 
        \AstIg \bigcup_{i=1}^{r} \fact\left(C_{p_i^{e_i}}\right) 
        \stackrel{(\ast\ast)}{=} \bigcup_{i=1}^{r} \left(\bb{Z}_{p_i}, \bb{Z}_{p_i}, \stackrel{(e_i)}{\ldots}, \bb{Z}_{p_i}\right)\\
        &= (\bb{Z}_{p_1}, \stackrel{(e_1)}{\ldots}\bb{Z}_{p_1}, \dots, \bb{Z}_{p_r},\stackrel{(e_r)}{\ldots}, \bb{Z}_{p_r}).
    \end{align*}
    donde en $(\ast)$ hemos usado el Ejercicio~\ref{ej:5.7} y en $(\ast\ast)$ el Ejercicio~\ref{ej:5.8}. Por tanto, se ha demostrado el resultado.\\

    Aplicándolo ahora a $n=12$, se tiene que $12=2^2\cdot 3^1$, luego:
    \begin{align*}
        l(\bb{Z}_{12}) &= 2+1 = 3 \\
        \fact(\bb{Z}_{12}) &= (\bb{Z}_2, \bb{Z}_2, \bb{Z}_3).
    \end{align*}

    Queremos calcular ahora la longitud y factores de composición de $\bb{Z}_2 \times \bb{Z}_6$. Como este no es cíclico, calculamos su longitud y factores de composición usando el Ejercicio~\ref{ej:5.7}:
    \begin{align*}
        l(\bb{Z}_2 \times \bb{Z}_6) &= l(\bb{Z}_2) + l(\bb{Z}_6) = 1 + 1+1 = 3 \\
        \fact(\bb{Z}_2 \times \bb{Z}_6) &= \fact(\bb{Z}_2)\cup\fact(\bb{Z}_6) = (\bb{Z}_2)\cup(\bb{Z}_2, \bb{Z}_3) = (\bb{Z}_2, \bb{Z}_2, \bb{Z}_3).
    \end{align*}

    Comprobamos por tanto que, aun no siendo isomorfos (puesto que uno es cíclico y el otro no), se cumple que:
    \begin{align*}
        l(\bb{Z}_{12}) &= l(\bb{Z}_2 \times \bb{Z}_6) = 3 \\
        \fact(\bb{Z}_{12}) &= \fact(\bb{Z}_2 \times \bb{Z}_6) = (\bb{Z}_2, \bb{Z}_2, \bb{Z}_3).
    \end{align*}

    Notemos que si dos grupos son isomorfos entonces tienen la misma longitud y los mismos factores de composición, pero el recíproco no es cierto.
\end{ejercicio}

\begin{ejercicio}
    Sea $D_n$ el grupo diédrico de orden $2n$. Si la descomposición de $n$ en factores primos es $n = p_1^{e_1} p_2^{e_2} \cdots p_r^{e_r}$, demostrar que
    \[
        l(D_n) = e_1 + e_2 + \cdots + e_r + 1,
    \]
    y que
    \[
        \fact(D_n) = (\bb{Z}_{p_1}, \stackrel{(e_1)}{\ldots}\bb{Z}_{p_1}, \dots, \bb{Z}_{p_r},\stackrel{(e_r)}{\ldots}, \bb{Z}_{p_r}, \bb{Z}_2).
    \]

    Sabemos que la siguiente serie es una serie normal de $D_n$:
    \begin{align*}
        D_n &\rhd \langle r \rangle \rhd \{1\}
    \end{align*}

    Por tanto, por el Ejercicio~\ref{ej:5.6} se tiene que:
    \begin{align*}
        l(D_n) &= l\left(\dfrac{D_n}{\langle r \rangle}\right) + l\left(\dfrac{\langle r \rangle}{\{1\}}\right) \\
        \fact(D_n) &= \fact\left(\dfrac{D_n}{\langle r \rangle}\right) \cup \fact\left(\dfrac{\langle r \rangle}{\{1\}}\right)
    \end{align*}

    Sabemos que $|D_n/\langle r \rangle|=\nicefrac{2n}{n}=2$, luego $D_n/\langle r \rangle \cong \bb{Z}_2$. Por otro lado, sabemos que $\langle r \rangle$ es cíclico de orden $n$, luego $\langle r \rangle \cong \bb{Z}_n$. Como la longitud y los factores se mantienen bajo isomorfismos, y usando el Ejercicio~\ref{ej:5.9} con $n = p_1^{e_1} p_2^{e_2} \cdots p_r^{e_r}$ y $2$ primo, se tiene que:
    \begin{align*}
        l(D_n) &= l\left(\dfrac{D_n}{\langle r \rangle}\right) + l\left(\dfrac{\langle r \rangle}{\{1\}}\right) 
        = l(\bb{Z}_2) + l(\bb{Z}_n) 
        = 1 + \left(e_1 + e_2 + \cdots + e_r\right) \\
        &= e_1 + e_2 + \cdots + e_r + 1\\
        \fact(D_n) &= \fact\left(\dfrac{D_n}{\langle r \rangle}\right) \cup \fact\left(\dfrac{\langle r \rangle}{\{1\}}\right)
        = \fact(\bb{Z}_2) \cup \fact(\bb{Z}_n) \\
        &= (\bb{Z}_2) \cup \left(\bb{Z}_{p_1}, \stackrel{(e_1)}{\ldots}\bb{Z}_{p_1}, \dots, \bb{Z}_{p_r},\stackrel{(e_r)}{\ldots}, \bb{Z}_{p_r}\right)\\
        &= (\bb{Z}_{p_1}, \stackrel{(e_1)}{\ldots}\bb{Z}_{p_1}, \dots, \bb{Z}_{p_r},\stackrel{(e_r)}{\ldots}, \bb{Z}_{p_r}, \bb{Z}_2).
    \end{align*}
\end{ejercicio}

\begin{ejercicio}
    Demostrar que $D_n$, $S_2$, $S_3$ y $S_4$ son grupos resolubles.
    \begin{enumerate}
        \item $D_n$.
        
        Una serie normal de $D_n$ es la siguiente:
        \begin{align*}
            D_n &\rhd \langle r \rangle \rhd \{1\}
        \end{align*}

        Sus factores son:
        \begin{align*}
            \dfrac{D_n}{\langle r \rangle} &\cong \bb{Z}_2 \\
            \dfrac{\langle r \rangle}{\{1\}} &\cong \langle r \rangle \cong \bb{Z}_n
        \end{align*}

        Por tanto, todos sus factores son abelianos, luego $D_n$ es resoluble.

        \item $S_2$.
        
        La serie derivada de $S_2$ es la siguiente:
        \begin{align*}
            S_2 &\rhd \{1\}
        \end{align*}

        Donde he empleado que $S_2\cong C_2$ es abeliano, luego $[S_2, S_2]=\{1\}$.
        Por tanto, $S_2$ es resoluble.

        \item $S_3$.
        
        Sabemos que $S_3'=[S_3, S_3]=A_3\cong C_3$ abeliano, luego la serie derivada de $S_3$ es la siguiente:
        \begin{align*}
            S_3 &\rhd A_3 \rhd \{1\}
        \end{align*}
        Por tanto, $S_3$ es resoluble.

        \item $S_4$.
        
        Una serie normal de $S_4$ es la siguiente:
        \begin{align*}
            S_4 &\rhd A_4 \rhd V \rhd \{1\}
        \end{align*}

        Sus factores son:
        \begin{align*}
            \dfrac{S_4}{A_4} &\cong \bb{Z}_2 \\
            \dfrac{A_4}{V} &\cong \bb{Z}_3 \\
            \dfrac{V}{\{1\}} &\cong V
        \end{align*}
        Donde $V$ es el grupo de Klein, que es abeliano. Por tanto, todos sus factores son abelianos, luego $S_4$ es resoluble.        
    \end{enumerate}
\end{ejercicio}

\begin{ejercicio}
    Sean $H$ y $K$ subgrupos normales de un grupo $G$ tales que $G/H$ y $G/K$ son ambos resolubles. Demostrar que $G/(H \cap K)$ también es resoluble.\\


    Por el Segundo Teorema de Isomorfía, como $H\lhd G$, tenemos $(H\cap K)\lhd K$ y:
    \begin{align*}
        \dfrac{K}{H\cap K} &\cong \dfrac{KH}{H}
    \end{align*}

    Este Teorema también afirma que $KH<G$, luego $KH/H<G/H$. Como $G/H$ es resoluble, se tiene que $KH/H$ es resoluble. Por tanto, $K/(H\cap K)$ es resoluble.\\

    Por otro lado, como $K,H\lhd G$, se tiene que $(H\cap K)\lhd G$. Como $(H\cap K)\subset K$ y $K\lhd G$, por el Tercer Teorema de Isomorfía se tiene que $H/(H\cap K)\lhd G/(H\cap K)$ y:
    \begin{align*}
        \dfrac{G/(H\cap K)}{K/(H\cap K)} &\cong \dfrac{G}{K}
    \end{align*}

    Como $G/K$ es resoluble, se tiene que $G/(H\cap K)/K/(H\cap K)$ es resoluble (puesto que esta propiedad se mantiene por isomorfismo).\\

    Como $\dfrac{G/(H\cap K)}{K/(H\cap K)}$ y $K/(H\cap K)$ son ambos resolubles, entonces $G/(H\cap K)$ es resoluble.
\end{ejercicio}

\begin{ejercicio}
    Sea $G$ un grupo resoluble y sea $H$ un subgrupo normal no trivial de $G$. Demostrar que existe un subgrupo no trivial $A$ de $H$ que es abeliano y normal en $G$.\\

    Como $H<G$, entonces $H$ es resoluble. Consideramos su serie derivada:
    \begin{align*}
        H &\rhd H' \rhd H'' \rhd \cdots \rhd H^{(n)} = \{1\}
    \end{align*}
    Como $H\neq \{1\}$, $n\neq 0$. Sea ahora $A=H^{(n-1)}$ (que podemos considerarlo puesto que $n\neq 0$). Como $[A,A]=[H^{(n-1)},H^{(n-1)}]=H^{(n)}=\{1\}$, se tiene que $A$ es abeliano. Nos falta por ver que $A\lhd G$.\\


    Consideramos la siguiente serie normal de $G$:
    \begin{align*}
        G &\rhd H\rhd H' \rhd H'' \rhd \cdots \rhd H^{(n)} = \{1\}
    \end{align*}
    
    Veamos que $H^{(i)}\lhd G$ para todo $i\in\{0,\ldots,n\}$.
    \begin{itemize}
        \item Para $i=0$, $G\lhd H$, luego se tiene que $H^0\lhd G$.
        \item Supuesto cierto para $i$, veamos que se cumple para $i+1$.
        
        Sabemos que $H^{(i)}\lhd G$, y queremos ver que $[H^{(i)},H^{(i)}]\lhd G$. Como se tiene que $[H^{(i)},H^{(i)}]=\langle [x,y] \mid x,y\in H^{(i)}\rangle$ y $[x,y]^{-1}=[y,x]$, tan solo es necesario comprobarlo sobre los generadores. Por tanto, sea $x,y\in H^{(i)}$, $g\in G$. Entonces:
        \begin{equation*}
            g[x,y]g^{-1} = [gxg^{-1},gyg^{-1}]
        \end{equation*}
        Como $H^{(i)}\lhd G$, se tiene que $gxg^{-1},gyg^{-1}\in H^{(i)}$, luego concluimos que $[gxg^{-1},gyg^{-1}]\in [H^{(i)},H^{(i)}]$. Por tanto, $H^{(i+1)}= [H^{(i)},H^{(i)}]\lhd G$.
    \end{itemize}

    Por tanto, $H^{(i)}\lhd G$ para todo $i\in\{0,\ldots,n\}$. En particular, $A=H^{(n-1)}\lhd G$.
\end{ejercicio}

\begin{ejercicio}
    Demuestra que todo $p$-grupo finito es resoluble.


    % // TODO: Hacer
\end{ejercicio}

\begin{ejercicio}
    Demuestra que todo grupo de orden $pq$, con $p$ y $q$ primos, es un grupo resoluble.

    % // TODO: Hacer
\end{ejercicio}

\begin{ejercicio}
    Demuestra que todo grupo de orden $p^2q$, con $p$ y $q$ primos, es un grupo resoluble.


    % // TODO: Hacer
\end{ejercicio}

\begin{ejercicio}
    Demuestra que si $p_1$, $p_2$, $p_3$ son tres primos tales que $p_3 > p_1 p_2$ entonces cualquier grupo de orden $p_1 p_2 p_3$ es resoluble.


    % // TODO: Hacer
\end{ejercicio}

\begin{ejercicio}~
    \begin{enumerate}
        \item Demuestra que todo grupo de orden $70$ es resoluble.
        \item Demuestra que todo grupo de orden $24$ es resoluble.
        \item Demuestra que todo grupo de orden $100$ es resoluble.
        \item Demuestra que todo grupo de orden $48$ es resoluble.
        \item Sea $G$ un grupo de orden $200$. Demuestra que $G \times D_{41}$ es resoluble.
        \item Demuestra que todo grupo de orden $63$ es soluble (sin usar que es un caso particular de un grupo de orden $p^2q$ con $p$ y $q$ primos).
    \end{enumerate}

    % // TODO: Hacer
\end{ejercicio}
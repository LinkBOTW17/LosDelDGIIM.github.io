\section{Clasificación de grupos abelianos finitos}

\begin{ejercicio}\label{ej:7.1}
    Calcular los órdenes de todos los elementos de los distintos grupos abelianos de orden $8$, $12$, $16$ y $24$.
\end{ejercicio}

\begin{ejercicio}\label{ej:7.2}
    Para los siguientes grupos calcular sus descomposiciones cíclicas.
    \begin{enumerate}
        \item $G_1 = \{1, 8, 12, 14, 18, 21, 27, 31, 34, 38, 44, 47, 51, 53, 57, 64\}$ con operación dada por multiplicación módulo $65$.
        \item $G_2 = \{1, 8, 17, 19, 26, 28, 37, 44, 46, 53, 62, 64, 71, 73, 82, 89, 91, 98, 107,109, 116, 118, 127, 134\}$ con operación dada por multiplicación módulo $135$.
        \item $G_3 = \{1, 7, 17, 23, 49, 55, 65, 71\}$ con operación dada por multiplicación módulo $96$.
        \item $G_4 = \{1, 4, 11, 14, 16, 19, 26, 29, 31, 34, 41, 44\}$ con operación dada por multiplicación módulo $45$.
    \end{enumerate}
\end{ejercicio}

\begin{ejercicio}\label{ej:7.3}
    Calcular la descomposición cíclica y cíclica primaria de los grupos abelianos $C_{24} \times C_{40} \times C_{35}$ y $C_{14} \times C_{100} \times C_{40}$. ¿Son isomorfos?
\end{ejercicio}

\begin{ejercicio}\label{ej:7.4}
    Sea $G$ el grupo de las simetrías de un rectángulo (no cuadrado). Probar que $G$ es un grupo abeliano. Calcular sus descomposiciones cíclica y cíclica primaria.
\end{ejercicio}

\begin{ejercicio}\label{ej:7.5}
    Sea $G$ un grupo abeliano de orden $n$ y $l(G)$ su longitud. Si la descomposición de $n$ en factores primos es $n = p_1^{e_1} \cdots p_r^{e_r}$, demostrar que
    \begin{equation*}
        l(G) = e_1 + \cdots + e_r,
    \end{equation*}
    y que
    \begin{equation*}
        \fact(G) = (C_{p_1},\stackrel{(e_1)}{\ldots}, C_{p_1}, \ldots, C_{p_r},\stackrel{(e_r)}{\ldots}, C_{p_r}).
    \end{equation*}
    En particular, todos los grupos abelianos del mismo orden tienen la misma longitud y la misma lista de factores de composición.
\end{ejercicio}

\begin{ejercicio}\label{ej:7.6}
    Listar todos los grupos abelianos no isomorfos de orden $10$, $16$, $20$, $30$, $40$, $108$ y $360$, dando sus factores invariantes, divisores elementales y descomposiciones cíclicas y cíclicas primarias.
\end{ejercicio}

\begin{ejercicio}\label{ej:7.7}
    Calcular la forma normal, los factores invariantes y los divisores elementales de las siguientes matrices:
    \begin{align*}
        &A_1 = \begin{pmatrix}
            0 & 2 & 0 \\
            -6 & -4 & -6 \\
            6 & 6 & 6 \\
            7 & 10 & 6
        \end{pmatrix},
        &A_2 = \begin{pmatrix}
            -22 & -48 & -267 \\
            -4 & -4 & 31 \\
            -4 & -24 & 105 \\
            4 & -6 & -6
        \end{pmatrix},\\
        &A_3 = \begin{pmatrix}
            9 & 4 & 5 \\
            -4 & 0 & -3 \\
            -6 & -4 & -2
        \end{pmatrix},
        &A_4 = \begin{pmatrix}
            4 & 0 & 0 \\
            0 & 6 & 0 \\
            0 & 0 & 8
        \end{pmatrix}.
    \end{align*}
\end{ejercicio}

\begin{ejercicio}\label{ej:7.8}
    Para los siguientes grupos abelianos calcular sus rangos y sus descomposiciones cíclicas y cíclicas primarias. ¿Son algunos de estos grupos isomorfos?
    \begin{enumerate}
        \item $G_1 = \left\langle a, b, c \left|
            \begin{array}{rcl}
                3a + 9b + 9c &=& 0 \\
                9a - 3b + 9c &=& 0
            \end{array}
        \right.\right\rangle$.
        \item $G_2 = \left\langle a, b, c \left|
            \begin{array}{rcl}
                2a + 2b + 3c &=& 0 \\
                5a + 2b - 3c &=& 0
            \end{array}
        \right.\right\rangle$.
        \item $G_3 = \left\langle a, b, c, d \left|
            \begin{array}{rcl}
                a + 3b + 2c &=& 0 \\
                5a + 17b + 12c &=& 0 \\
                6a + 4c &=& 0
            \end{array}
        \right.\right\rangle$.
        \item $G_4 = \left\langle a, b, c \left|
            \begin{array}{rcl}
                12a + 4b + 6c &=& 0 \\
                -4a + 2b + 8c &=& 0 \\
                -2a + 16b + 34c &=& 0
            \end{array}
        \right.\right\rangle$.
        \item $G_5 = \bb{Z}_{24} \oplus \bb{Z}_{40} \oplus \bb{Z}_{35}$.
    \end{enumerate}
\end{ejercicio}

\begin{ejercicio}\label{ej:7.9}
    Dados los grupos abelianos:
    \begin{align*}
        G &= \left\langle a, b, c, d \left|
            \begin{array}{rcl}
                a + 2c - d &=& 0 \\
                a + 5c + 5d &=& 0 \\
                2a + 4c + 2d &=& 0
            \end{array}
        \right.\right\rangle,\\
        H &= \bb{Z}^3/K,
    \end{align*}
    donde $K$ es el subgrupo con generadores $\{(1, 2, 7),(1, 4, 7),(-1, 0, 2)\}$. Calcular:
    \begin{enumerate}
        \item El rango, los factores invariantes y los divisores elementales de cada uno de ellos.
        \item Sus descomposiciones cíclicas y cíclicas primarias.
        \item Las descomposiciones cíclica y cíclica primaria de $G \oplus H$.
    \end{enumerate}
\end{ejercicio}

\begin{ejercicio}\label{ej:7.10}~
    \begin{enumerate}
        \item Encuentra todos los grupos abelianos distintos, salvo isomorfismo, de orden $500$. Da para cada uno de ellos sus descomposiciones cíclica y cíclica primaria.
        \item Calcula las descomposiciones cíclica y cíclica primaria de
        \begin{equation*}
            G = \left\langle a, b, c \left|
                \begin{array}{rcl}
                    3a - 3b + 9c &=& 0 \\
                    6a + 12b - 9c &=& 0 \\
                    12b + 9c &=& 0
                \end{array}
            \right.\right\rangle.
        \end{equation*}
        ¿Cuántos elementos tiene $G$? ¿Tiene algún elemento de orden $6$?
    \end{enumerate}
\end{ejercicio}

\begin{ejercicio}\label{ej:7.11}
    Dados los grupos abelianos
    \begin{align*}
        G &= \left\langle a, b, c \left|
            \begin{array}{rcl}
                2a - 6b + 18c &=& 0 \\
                6a + 6c &=& 0
            \end{array}
        \right.\right\rangle,\\
        H &= \bb{Z}^3/\langle(1, -9, 3),(1, -7, 1),(1, -1, 1)\rangle.
    \end{align*}
    \begin{enumerate}
        \item Calcula sus rangos, descomposiciones cíclicas y cíclicas primarias.
        \item ¿Son isomorfos? ¿Lo son sus subgrupos de torsión?
        \item ¿Cuántos elementos de orden $6$ tiene $H$? ¿Y $G$?
        \item ¿Cuántos grupos hay, salvo isomorfismos, con los mismos elementos que $H$?
    \end{enumerate}
\end{ejercicio}

\begin{ejercicio}\label{ej:7.12}~
    \begin{enumerate}
        \item Calcula la descomposición cíclica y cíclica primaria de todos los grupos abelianos no isomorfos de orden $484$.
        \item Sea
        \begin{equation*}
            G = \left\langle a, b, c \left|
                \begin{array}{rcl}
                    2a + b + 4c &=& 0 \\
                    2a + 2b + 6c &=& 0
                \end{array}
            \right.\right\rangle,
        \end{equation*}
        y $H = \bb{Z}^2/K$, con $K$ el subgrupo de $\bb{Z}^2$ generado por los pares $(2, 3)$ y $(6, 3)$. Razona, calculando las descomposiciones cíclica y cíclica primaria de ambos, que no son isomorfos.
    \end{enumerate}
\end{ejercicio}

\begin{ejercicio}\label{ej:7.13}~
    \begin{enumerate}
        \item Encuentra todos los grupos abelianos distintos, salvo isomorfismo, de orden $1176$. Da para cada uno de ellos sus descomposiciones cíclica y cíclica primaria.
        \item Calcula las descomposiciones cíclica y cíclica primaria del grupo abeliano dado en términos de generadores y relaciones siguiente:
        \begin{equation*}
            G = \left\langle x, y, z \left|
                \begin{array}{rcl}
                    2x &=& 5y \\
                    2y &=& 5z \\
                    2z &=& 5x
                \end{array}
            \right.\right\rangle.
        \end{equation*}
        ¿Qué tipo de órdenes tienen sus elementos?
    \end{enumerate}
\end{ejercicio}

\begin{ejercicio}\label{ej:7.14}
    Calcular las descomposiciones cíclica y cíclica primaria del siguiente grupo abeliano dados en términos de generadores y relaciones:
    \begin{equation*}
        G = \left\langle a, b, c, d \left|
            \begin{array}{rcl}
                9a + 9b + c + 8d &=& 0 \\
                63a - b + 63c + 64d &=& 0 \\
                56a - 8b + 64c + 56d &=& 0
            \end{array}
        \right.\right\rangle.
    \end{equation*}
    ¿Tiene $G$ elementos de orden infinito? ¿Y de orden finito? Calcular cuántos grupos abelianos no isomorfos hay con el mismo orden que la torsión de $G$.
\end{ejercicio}

\begin{ejercicio}\label{ej:7.15}
    Calcular las descomposiciones cíclica y cíclica primaria de todos los grupos abelianos no isomorfos de orden $13916$. Identifica la componente $3-$primaria de cualquiera de esos grupos.
\end{ejercicio}
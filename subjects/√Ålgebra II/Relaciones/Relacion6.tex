\chapter{$G-$conjuntos y $p$-grupos}

\begin{ejercicio}\label{ej:6.1}
    Si $X$ es un $G-$conjunto, demostrar que $x^g = \prescript{g^{-1}}{}{x},~ x \in X, g \in G$, define una acción por la derecha de $G$ sobre $X$.
\end{ejercicio}

\begin{ejercicio}\label{ej:6.2}
    Sea $G$ un grupo y $N$ un subgrupo normal abeliano de $G$. Demostrar que $G/N$ actúa sobre $N$ por conjugación y obtener un homomorfismo $\varphi: G/N \to \Aut(N)$.
\end{ejercicio}

\begin{ejercicio}\label{ej:6.3}
    Sean $S$ y $T$ dos $G-$conjuntos. Se define la \emph{acción diagonal} de $G$ sobre el producto cartesiano $S \times T$ mediante $\prescript{x}{}{(s,t)} = (\prescript{x}{}{s},\prescript{x}{}{t})$. Demostrar que, para la acción diagonal, el estabilizador de $(s, t)$ es la intersección de los estabilizadores de $s$ y $t$ en las acciones dadas.
\end{ejercicio}

\begin{ejercicio}\label{ej:6.4}
    Demostrar que si $G$ contiene un elemento $x$ que tiene exactamente dos conjugados, entonces $G$ tiene un subgrupo normal propio.
    \begin{observacion}
        Considerar el centralizador de $x$.
    \end{observacion}
\end{ejercicio}

\begin{ejercicio}\label{ej:6.5}
    Encontrar todos los grupos finitos que tienen exactamente dos clases de conjugación.
\end{ejercicio}

\begin{ejercicio}\label{ej:6.6}
    Describir explícitamente las clases de conjugación del grupo $D_4$.
\end{ejercicio}

\begin{ejercicio}\label{ej:6.7}
    Se dice que la acción de un grupo finito $G$ sobre un conjunto $X$ es \emph{transitiva} si hay una sola órbita para esta acción (es decir, si para cada $x, y \in X$ existe algún $g \in G$ tal que $\prescript{g}{}{x} = y$). Demostrar que si $G$ actúa transitivamente sobre un conjunto $X$ con $n$ elementos, entonces $|G|$ es un múltiplo de $n$.
\end{ejercicio}

\begin{ejercicio}\label{ej:6.8}
    Un subgrupo $G \leq S_n$ se dice \emph{transitivo} si la acción de $G$ sobre $\{1, 2, \ldots, n\}$ es transitiva. Encontrar todos los subgrupos transitivos de $S_3$ y $S_4$.
\end{ejercicio}

\begin{ejercicio}\label{ej:6.9}
    Sea $n\in \bb{N}$. Una \emph{partición} de $n$ es una sucesión no decreciente de enteros positivos cuya suma es $n$. Dada una permutación $\sigma \in S_n$, la descomposición en ciclos disjuntos (incluyendo los ciclos de longitud 1) de $\sigma = \gamma_1 \gamma_2 \cdots \gamma_r$ determina una partición $n_1, n_2, \ldots, n_r$ de $n$ donde cada $n_i$ es la longitud del ciclo $\gamma_i$. Dos permutaciones en $S_n$ se dice que son del mismo tipo si determinan la misma partición de $n$. Demostrar:
    \begin{enumerate}
        \item Dos elementos de $S_n$ son conjugados si y solo si son del mismo tipo.
        \item El número de clases de conjugación de $S_n$ es igual al número de particiones de $n$.
    \end{enumerate}
\end{ejercicio}

\begin{ejercicio}\label{ej:6.10}
    Calcular el número de clases de conjugación de $S_5$. Dar un representante de cada una y encontrar el orden de cada clase. Calcular el estabilizador de $(1\ 2\ 3)$ bajo la acción de conjugación de $S_5$ sobre sí mismo.
\end{ejercicio}

\begin{ejercicio}
    Sea $G$ un grupo finito y $\Phi: G \to \Perm(G)$ la representación regular izquierda (que corresponde a la acción de $G$ sobre sí mismo por traslación por la izquierda).
    \begin{enumerate}
        \item Demostrar que si $x$ es un elemento de $G$ de orden $n$ y $|G| = nm$, entonces $\Phi(x)$ es un producto de $n-$ciclos. Deducir que $\Phi(x)$ es una permutación impar si y solo si el orden de $x$ es par y el cociente del orden de $G$ y el de $x$ es impar.
        \item Demostrar que si $Im(\Phi)$ contiene una permutación impar entonces $G$ tiene un subgrupo de índice 2.
        \item Demostrar que si $|G| = 2^k$ con $k$ impar, entonces $G$ tiene un subgrupo de índice 2.
        \begin{observacion}
            Usar el Teorema de Cauchy para obtener un elemento de orden 2 y entonces usar los dos apartados anteriores.
        \end{observacion}
    \end{enumerate}
\end{ejercicio}

\begin{ejercicio}\label{ej:6.12}
    Sea $G$ un $p-$grupo actuando sobre un conjunto finito $X$. Demostrar que
    \[
        |X| \equiv |{Fix}_G(X)| \mod p.
    \]
\end{ejercicio}

\begin{ejercicio}
    Sea $G$ un $2-$grupo finito que actúa sobre un conjunto finito $X$ cuya cardinalidad es un número impar. ¿Podemos afirmar que existe al menos un punto de $X$ que queda fijo bajo la acción de $G$? ¿Podemos decir lo mismo si $|X|$ es par?
\end{ejercicio}

\begin{ejercicio}\label{ej:6.14}
    Sea $C_n = \langle a \mid a^n = 1 \rangle$ un grupo cíclico de orden $n$. Describir sus subgrupos de Sylow.
\end{ejercicio}

\begin{ejercicio}\label{ej:6.15}
    Sea $G$ un grupo finito y $|G| = pn$ con $p$ primo y $p > n$. Demostrar que $G$ contiene un subgrupo normal de orden $p$ y que todo subgrupo de $G$ de orden $p$ es normal en $G$.
\end{ejercicio}

\begin{ejercicio}\label{ej:6.16}
    Sea $H$ un subgrupo de un grupo finito $G$ con $[G : H] = p$ primo y $p$ el menor primo que divide a $|G|$. Demostrar que entonces $H$ es normal en $G$.
\end{ejercicio}

\begin{ejercicio}\label{ej:6.17}
    Sea $p$ un número primo. Demostrar:
    \begin{enumerate}
        \item Todo grupo no abeliano de orden $p^3$ tiene un centro de orden $p$.
        \item Existen únicamente dos grupos no isomorfos de orden $p^2$.
        \item Todo subgrupo normal de orden $p$ de un $p-$grupo finito está contenido en el centro.
    \end{enumerate}
\end{ejercicio}

\begin{ejercicio}\label{ej:6.18}
    Demostrar que si $N\lhd G$ y $N$ y $G/N$ son $p-$grupos entonces $G$ es un $p-$grupo.
\end{ejercicio}

\begin{ejercicio}\label{ej:6.19}
    Si $G$ es un grupo de orden $p^n$, $p$ primo, demostrar que para todo $k$, $0 \leq k \leq n$, existe un subgrupo normal de $G$ de orden $p^k$.
\end{ejercicio}

\begin{ejercicio}\label{ej:6.20}
    Hallar todos los subgrupos de Sylow de los grupos $S_3$ y $S_4$.
    \begin{observacion}
        Para los $2-$subgrupos de Sylow de $S_4$, observar primero que todos deben contener al subgrupo de Klein $V$, y, al menos, una trasposición $\tau$, y que como consecuencia se pueden obtener como producto de $V$ por el grupo cíclico generado por $\tau$.
    \end{observacion}
\end{ejercicio}

\begin{ejercicio}\label{ej:6.21}
    Hallar todos los subgrupos de Sylow de los grupos $\bb{Z}_{600}$, $Q_2$, $D_5$, $D_6$, $A_4$, $A_5$, $S_5$.
\end{ejercicio}

\begin{ejercicio}\label{ej:6.22}
    Demostrar que $D_4$ es isomorfo a los $2-$subgrupos de Sylow de $S_4$.
    \begin{observacion}
        Considerar la representación asociada a la acción de $D_4$ sobre los vértices del cuadrado.
    \end{observacion}
\end{ejercicio}

\begin{ejercicio}\label{ej:6.23}
    Demostrar que todo grupo de orden $12$ con más de un $3-$subgrupo de Sylow es isomorfo al grupo alternado $A_4$.
    \begin{observacion}
        Considerar la acción por traslación de un tal grupo sobre el conjunto de clases módulo $P$, siendo $P$ un $3-$subgrupo de Sylow. Probar que dicha acción es fiel.
    \end{observacion}
\end{ejercicio}

\begin{ejercicio}\label{ej:6.24}~
    \begin{enumerate}
        \item Demostrar que no existen grupos simples de orden $12$. Más concretamente, demostrar que todo grupo de orden $12$ admite un subgrupo normal de orden $3$ o de orden $4$.
        \item Demostrar que no existen grupos simples de orden $28$. Más concretamente, probar que todo grupo de orden $28$ contiene un subgrupo normal de orden $7$.
        \item Demostrar que no existen grupos simples de orden $56$. Más concretamente, probar que todo grupo de orden $56$ contiene un subgrupo normal de orden $7$ o de orden $8$.
        \item Demostrar que no existen grupos simples de orden $148$ ni de orden $200$ ni de orden $351$.
    \end{enumerate}
\end{ejercicio}

\begin{ejercicio}\label{ej:6.25}
    Calcular el número de elementos de orden $7$ que tiene un grupo simple de orden $168$.
\end{ejercicio}
\section{$G-$conjuntos y $p$-grupos}

\begin{ejercicio}\label{ej:6.1}
    Si $X$ es un $G-$conjunto, demostrar que $x^g = \prescript{g^{-1}}{}{x},~ x \in X, g \in G$, define una acción por la derecha de $G$ sobre $X$.\\

    En primer lugar, vemos que se trata de una aplicación de $G \times X$ en $X$. Veamos ahora que cumple las condiciones necesarias para ser una acción por la derecha:
    \begin{itemize}
        \item $x^1 = x$ para todo $x \in X$.
        \begin{equation*}
            x^1 = \prescript{1^{-1}}{}{x} = \prescript{1}{}{x} = x
        \end{equation*}

        \item $(x^g)^h = x^{gh}$ para todo $x \in X$ y $g, h \in G$.
        \begin{equation*}
            (x^g)^h = \prescript{h^{-1}}{}{(x^g)} = \prescript{h^{-1}}{}{(\prescript{g^{-1}}{}{x})} = \prescript{h^{-1}g^{-1}}{}{x} =  \prescript{(gh)^{-1}}{}{x} = x^{gh}
        \end{equation*}
    \end{itemize}

    Por tanto, se trata de una acción por la derecha de $G$ sobre $X$.
\end{ejercicio}

\begin{ejercicio}\label{ej:6.2}
    Sea $G$ un grupo y $N$ un subgrupo normal abeliano de $G$. Demostrar que $G/N$ actúa sobre $N$ por conjugación y obtener entonces un homomorfismo $\varphi: G/N \to \Aut(N)$.\\

    Veamos en primer lugar que $G/N$ actúa sobre $N$ por conjugación. Es decir, que la siguiente aplicación es una acción de $G/N$ sobre $N$:
    \Func{ac}{G/N \times N}{N}{(gN, n)}{\prescript{gN}{}{n} = gng^{-1}}

    Veamos en primer lugar que está bien definida. Sean $g_1, g_2 \in G$ de forma que $g_1N = g_2N$. Entonces $\exists n'\in N$ tal que $g_1 = g_2n'$. Entonces:
    \begin{align*}
        \prescript{g_1N}{}{n} &= g_1ng_1^{-1} = g_2n' n (g_2n')^{-1} = g_2n' n (n')^{-1}g_2^{-1} \AstIg g_2 n'(n')^{-1} n g_2^{-1} = g_2 n g_2^{-1}
        = \prescript{g_2N}{}{n}
    \end{align*}
    donde en $(\ast)$ hemos usado que $N$ es abeliano. Por tanto, la acción está bien definida. Veamos ahora que se trata de una acción.
    \begin{itemize}
        \item $\prescript{1N}{}{n} = 1n1^{-1} = n$ para todo $n \in N$.
        \item Comprobemos la segunda propiedad:
        \begin{align*}
            \prescript{(g_1N)(g_2N)}{}{n} &= \prescript{g_1g_2N}{}{n} = g_1g_2ng_2^{-1}g_1^{-1} = g_1 \left(\prescript{g_2N}{}{n}\right) g_1^{-1}
            = \prescript{g_1N}{}{\left(\prescript{g_2N}{}{n}\right)}.
        \end{align*}
    \end{itemize}

    Buscamos ahora el homomorfismo $\varphi: G/N \to \Aut(N)$. En primer lugar, consideramos el siguiente homomorfismo:
    \Func{\Phi}{G/N}{\Perm(N)}{gN}{\prescript{gN}{}{(\cdot)}=ac(gN, \cdot)}


    Es necesario ver que, fijado $gN \in G/N$, la aplicación siguiente, además de pertenecer a $\Perm(N)$, pertenece a $\Aut(N)$:
    \Func{f}{N}{N}{n}{\prescript{gN}{}{n} = gng^{-1}}

    Sabemos que es biyectiva, por lo que tan solo nos queda probar que es un homomorfismo. Sean $n_1, n_2 \in N$:
    \begin{align*}
        f(n_1n_2) &= \prescript{gN}{}{(n_1n_2)} = g(n_1n_2)g^{-1} = g n_1 g^{-1} g n_2 g^{-1} = f(n_1)f(n_2).
    \end{align*}

    Por tanto, $f$ es un homomorfismo. La aplicación $\varphi$ pedida entonces es:
    \Func{\varphi}{G/N}{\Aut(N)}{gN}{f = \prescript{gN}{}{(\cdot)}}
\end{ejercicio}

\begin{ejercicio}\label{ej:6.3}
    Sean $S$ y $T$ dos $G-$conjuntos. Se define la \emph{acción diagonal} de $G$ sobre el producto cartesiano $S \times T$ mediante $\prescript{x}{}{(s,t)} = (\prescript{x}{}{s},\prescript{x}{}{t})$. Demostrar que, para la acción diagonal, el estabilizador de $(s, t)$ es la intersección de los estabilizadores de $s$ y $t$ en las acciones dadas.\\

    Fijados $s \in S$ y $t \in T$, el estabilizador de $(s,t)$ es:
    \begin{align*}
        \Stab_{G}(s,t) &= \{g \in G \mid \prescript{g}{}{(s,t)} = (s,t)\} = \{g \in G \mid (\prescript{g}{}{s},\prescript{g}{}{t}) = (s,t)\}\\
        &= \{g \in G \mid \prescript{g}{}{s} = s \land \prescript{g}{}{t} = t\} = \{g \in G \mid \prescript{g}{}{s} = s\} \cap \{g \in G \mid \prescript{g}{}{t} = t\}\\
        &= \Stab_{G}(s) \cap \Stab_{G}(t).
    \end{align*}
\end{ejercicio}

\begin{ejercicio}\label{ej:6.4}
    Demostrar que si $G$ contiene un elemento $x$ que tiene exactamente dos conjugados, entonces $G$ tiene un subgrupo normal propio.
    \begin{observacion}
        Considerar el centralizador de $x$.
    \end{observacion}

    % // TODO: Sin suponer que G es finito??

    Consideramos la acción por conjugación de $G$ sobre sí mismo:
    \Func{ac}{G \times G}{G}{(g,h)}{\prescript{g}{}{h} = ghg^{-1}}

    Calculamos el centralizador de $x$:
    \begin{align*}
        C_G(\{x\}) &= \{g \in G \mid gx = xg\} = \{g \in G \mid gxg^{-1} = x\} = \{g \in G \mid \prescript{g}{}{x} = x\} = \Stab_G(x)
    \end{align*}

    Por tanto, $C_G(\{x\}) = \Stab_G(x)<G$. Veamos ahora que es normal en $G$. Para ello, tenemos que:
    \begin{equation*}
        [G:C_G(\{x\})] = [G:\Stab_G(x)] = |\Orb(x)|
    \end{equation*}

    Calculemos la órbita de $x$:
    \begin{align*}
        \Orb(x) &= \{y\in G \mid \exists g \in G \text{ tal que } y = \prescript{g}{}{x}\} = \{y\in G \mid \exists g \in G \text{ tal que } y = gxg^{-1}\}
        = \Cl_G(x)
    \end{align*}

    Como $x$ tiene exactamente dos conjugados (él mismo y otro elemento $y\in G$), tenemos que $|\Orb(x)| = 2$. Por tanto:
    \begin{align*}
        [G:C_G(\{x\})] &= |\Orb(x)| = 2 \implies C_G(\{x\})\lhd G
    \end{align*}

    Por tanto, $C_G(\{x\})$ es un subgrupo normal de $G$. Tan solo falta por comprobar que es propio.
    \begin{itemize}
        \item Si $C_G(\{x\}) = G$, entonces:
        \begin{equation*}
            2 = |\Orb(x)| = [G:\Stab_G(x)] = [G:C_G(\{x\})] = 1 \implies \text{Contradicción.}
        \end{equation*}

        \item Si $C_G(\{x\}) = \{1\}$, entonces:
        \begin{equation*}
            2 = |\Orb(x)| = [G:\Stab_G(x)] = [G:C_G(\{x\})] = |G|
        \end{equation*}
        Por tanto, $G=\{1,x\}$. Calculemos el número de conjugados de $1$ y de $x$:
        \begin{align*}
            \Cl_G(1) &= \{g1g^{-1} \mid g \in G\} = \{1\} \\
            \Cl_G(x) &= \{gxg^{-1} \mid g \in G\} = \{1x1, xxx^{-1}\} = \{x\}
        \end{align*}
        Por tanto, ambos tienen un único conjugado. Por tanto, no se puede dar este caso.
    \end{itemize}
\end{ejercicio}

\begin{ejercicio}\label{ej:6.5}
    Encontrar todos los grupos finitos que tienen exactamente dos clases de conjugación.\\

    Sea $G$ un grupo finito con $|G| = n$ que tiene exactamente dos clases de conjugación; a saber, $\exists x_1, x_2 \in G$ tales que $\Cl_G(x_1) \neq \Cl_G(x_2)$. Considerando la acción de $G$ sobre sí mismo por conjugación, tenemos que:
    \begin{equation*}
        \Orb(x) = \Cl_G(x) \qquad \forall x \in G
    \end{equation*}

    Como las órbitas forman una partición de $G$, tenemos que:
    \begin{equation*}
        |G| = |\Orb(x_1)| + |\Orb(x_2)| = |\Cl_G(x_1)| + |\Cl_G(x_2)|
    \end{equation*}

    Calculamos no obstante la clase de conjugación del $1\in G$:
    \begin{align*}
        \Cl_G(1) &= \{g1g^{-1} \mid g \in G\} = \{g g^{-1} \mid g \in G\} = \{1\}
    \end{align*}
    Por tanto, $|\Cl_G(1)| = 1$. Supongamos sin pérdida de generalidad que $1\in \Cl_G(x_1)$. Entonces:
    \begin{equation*}
        n = |\Cl_G(x_1)| + |\Cl_G(x_2)| = 1 + |\Cl_G(x_2)|
        \Longrightarrow |\Cl_G(x_2)| = n - 1
    \end{equation*}

    Por otro lado, como $|\Cl_G(x_2)| = [G:\Stab_G(x_2)]$, tenemos que $|\Cl_G(x_2)|$ divide a $|G|$; es decir, $(n-1) \mid n$. Por tanto, $n=2$, y tenemos por tanto que:
    \begin{equation*}
        G\cong \bb{Z}_2
    \end{equation*}
\end{ejercicio}

\begin{ejercicio}\label{ej:6.6}
    Describir explícitamente las clases de conjugación del grupo $D_4$.\\

    Consideramos el grupo $D_4$:
    \begin{align*}
        D_4 &= \{1, r, r^2, r^3, s, sr, sr^2, sr^3\} \\
        &= \{s^ir^j \mid i = 0, 1, j = 0, 1, 2, 3\}
    \end{align*}

    Tenemos que:
    \begin{align*}
        \Cl_{D_4}(1) &= \{(s^i r^j)1(s^i r^j)^{-1} \mid i = 0, 1, j = 0, 1, 2, 3\} = \{1\} \\
        \Cl_{D_4}(r) &= \{(s^i r^j)r(s^i r^j)^{-1} \mid i = 0, 1, j = 0, 1, 2, 3\} = \{s^ir^j\ r\ r^{-j}s^{-i} \mid i = 0, 1, j = 0, 1, 2, 3\} \\
        &= \{s^i r s^i \mid i = 0, 1\} = \{r, r^3\} = \Cl_{D_4}(r^3) \\
        \Cl_{D_4}(r^2) &= \{(s^i r^j)r^2(s^i r^j)^{-1} \mid i = 0, 1, j = 0, 1, 2, 3\} = \{s^i r^j\ r^2\ r^{-j}s^{-i} \mid i = 0, 1, j = 0, 1, 2, 3\} \\
        &= \{s^i r^2 s^{-i} \mid i = 0, 1\} = \{r^2\}\\
        \Cl_{D_4}(s) &= \{(s^i r^j)s(s^i r^j)^{-1} \mid i = 0, 1, j = 0, 1, 2, 3\}
        = \{s^ir^j\ s\ r^{-j}s^{-i} \mid i = 0, 1, j = 0, 1, 2, 3\} 
    \end{align*}

    Este último no es tan sencillo, puesto que $r$ y $s$ no conmutan. Calculamos en primer lugar para $s=0$, sabiendo que las clases de conjugación son cerradas para inversos.
    \begin{align*}
        r\ s\ r^{-1} &= r\ s\ r^3 = sr^6 = sr^2 \in \Cl_{D_4}(s) \\
        r^2\ s\ r^{-2} &= r^2\ s\ r^2 = sr^6r^2=s\in \Cl_{D_4}(s) \\
        r^3\ s\ r^{-3} &= r^3\ s\ r = sr^9r = sr^2 \in \Cl_{D_4}(s)
    \end{align*}

    Por otro lado, para $s=1$, tenemos que:
    \begin{equation*}
        s\ s\ s = s\qquad \text{ y } s\ sr^2\ s = r^2s = sr^6 = sr^2
    \end{equation*}

    Por tanto, $\Cl_{D_4}(s) = \{s, sr^2\} = \Cl_{D_4}(sr^2)$. Tan solo queda por tanto calcular la clase de conjugación de $sr$ y de $sr^3$.
    \begin{equation*}
        r\ sr\ r^{-1} = r\ sr\ r^3 = rs = sr^3 \in \Cl_{D_4}(sr)
    \end{equation*}

    Por tanto, tenemos que $\Cl_{D_4}(sr) = \Cl_{D_4}(sr^3)$. Como las clases de conjugación forman una partición de $D_4$, tenemos que:
    \begin{align*}
        \Cl_{D_4}(1) &= \{1\} \\
        \Cl_{D_4}(r) &= \{r, r^3\} \\
        \Cl_{D_4}(r^2) &= \{r^2\} \\
        \Cl_{D_4}(s) &= \{s, sr^2\} \\
        \Cl_{D_4}(sr) &= \{sr, sr^3\}
    \end{align*}
\end{ejercicio}

\begin{ejercicio}\label{ej:6.7}
    Se dice que la acción de un grupo finito $G$ sobre un conjunto $X$ es \emph{transitiva} si hay una sola órbita para esta acción (es decir, si para cada $x, y \in X$ existe algún $g \in G$ tal que $\prescript{g}{}{x} = y$). Demostrar que si $G$ actúa transitivamente sobre un conjunto $X$ con $n$ elementos, entonces $|G|$ es un múltiplo de $n$.
\end{ejercicio}

\begin{ejercicio}\label{ej:6.8}
    Un subgrupo $G \leq S_n$ se dice \emph{transitivo} si la acción de $G$ sobre $\{1, 2, \ldots, n\}$ es transitiva. Encontrar todos los subgrupos transitivos de $S_3$ y $S_4$.
\end{ejercicio}

\begin{ejercicio}\label{ej:6.9}
    Sea $n\in \bb{N}$. Una \emph{partición} de $n$ es una sucesión no decreciente de enteros positivos cuya suma es $n$. Dada una permutación $\sigma \in S_n$, la descomposición en ciclos disjuntos (incluyendo los ciclos de longitud 1) de $\sigma = \gamma_1 \gamma_2 \cdots \gamma_r$ determina una partición $n_1, n_2, \ldots, n_r$ de $n$ donde cada $n_i$ es la longitud del ciclo $\gamma_i$. Dos permutaciones en $S_n$ se dice que son del mismo tipo si determinan la misma partición de $n$. Demostrar:
    \begin{enumerate}
        \item Dos elementos de $S_n$ son conjugados si y solo si son del mismo tipo.
        \item El número de clases de conjugación de $S_n$ es igual al número de particiones de $n$.
    \end{enumerate}
\end{ejercicio}

\begin{ejercicio}\label{ej:6.10}
    Calcular el número de clases de conjugación de $S_5$. Dar un representante de cada una y encontrar el orden de cada clase. Calcular el estabilizador de $(1\ 2\ 3)$ bajo la acción de conjugación de $S_5$ sobre sí mismo.
\end{ejercicio}

\begin{ejercicio}
    Sea $G$ un grupo finito y $\Phi: G \to \Perm(G)$ la representación regular izquierda (que corresponde a la acción de $G$ sobre sí mismo por traslación por la izquierda).
    \begin{enumerate}
        \item Demostrar que si $x$ es un elemento de $G$ de orden $n$ y $|G| = nm$, entonces $\Phi(x)$ es un producto de $n-$ciclos. Deducir que $\Phi(x)$ es una permutación impar si y solo si el orden de $x$ es par y el cociente del orden de $G$ y el de $x$ es impar.
        \item Demostrar que si $Im(\Phi)$ contiene una permutación impar entonces $G$ tiene un subgrupo de índice 2.
        \item Demostrar que si $|G| = 2^k$ con $k$ impar, entonces $G$ tiene un subgrupo de índice 2.
        \begin{observacion}
            Usar el Teorema de Cauchy para obtener un elemento de orden 2 y entonces usar los dos apartados anteriores.
        \end{observacion}
    \end{enumerate}
\end{ejercicio}

\begin{ejercicio}\label{ej:6.12}
    Sea $G$ un $p-$grupo actuando sobre un conjunto finito $X$. Demostrar que
    \[
        |X| \equiv |{Fix}_G(X)| \mod p.
    \]
\end{ejercicio}

\begin{ejercicio}
    Sea $G$ un $2-$grupo finito que actúa sobre un conjunto finito $X$ cuya cardinalidad es un número impar. ¿Podemos afirmar que existe al menos un punto de $X$ que queda fijo bajo la acción de $G$? ¿Podemos decir lo mismo si $|X|$ es par?
\end{ejercicio}

\begin{ejercicio}\label{ej:6.14}
    Sea $C_n = \langle a \mid a^n = 1 \rangle$ un grupo cíclico de orden $n$. Describir sus subgrupos de Sylow.
\end{ejercicio}

\begin{ejercicio}\label{ej:6.15}
    Sea $G$ un grupo finito y $|G| = pn$ con $p$ primo y $p > n$. Demostrar que $G$ contiene un subgrupo normal de orden $p$ y que todo subgrupo de $G$ de orden $p$ es normal en $G$.
\end{ejercicio}

\begin{ejercicio}\label{ej:6.16}
    Sea $H$ un subgrupo de un grupo finito $G$ con $[G : H] = p$ primo y $p$ el menor primo que divide a $|G|$. Demostrar que entonces $H$ es normal en $G$.
\end{ejercicio}

\begin{ejercicio}\label{ej:6.17}
    Sea $p$ un número primo. Demostrar:
    \begin{enumerate}
        \item Todo grupo no abeliano de orden $p^3$ tiene un centro de orden $p$.
        \item Existen únicamente dos grupos no isomorfos de orden $p^2$.
        \item Todo subgrupo normal de orden $p$ de un $p-$grupo finito está contenido en el centro.
    \end{enumerate}
\end{ejercicio}

\begin{ejercicio}\label{ej:6.18}
    Demostrar que si $N\lhd G$ y $N$ y $G/N$ son $p-$grupos entonces $G$ es un $p-$grupo.
\end{ejercicio}

\begin{ejercicio}\label{ej:6.19}
    Si $G$ es un grupo de orden $p^n$, $p$ primo, demostrar que para todo $k$, $0 \leq k \leq n$, existe un subgrupo normal de $G$ de orden $p^k$.
\end{ejercicio}

\begin{ejercicio}\label{ej:6.20}
    Hallar todos los subgrupos de Sylow de los grupos $S_3$ y $S_4$.
    \begin{observacion}
        Para los $2-$subgrupos de Sylow de $S_4$, observar primero que todos deben contener al subgrupo de Klein $V$, y, al menos, una trasposición $\tau$, y que como consecuencia se pueden obtener como producto de $V$ por el grupo cíclico generado por $\tau$.
    \end{observacion}
\end{ejercicio}

\begin{ejercicio}\label{ej:6.21}
    Hallar todos los subgrupos de Sylow de los grupos $\bb{Z}_{600}$, $Q_2$, $D_5$, $D_6$, $A_4$, $A_5$, $S_5$.
\end{ejercicio}

\begin{ejercicio}\label{ej:6.22}
    Demostrar que $D_4$ es isomorfo a los $2-$subgrupos de Sylow de $S_4$.
    \begin{observacion}
        Considerar la representación asociada a la acción de $D_4$ sobre los vértices del cuadrado.
    \end{observacion}
\end{ejercicio}

\begin{ejercicio}\label{ej:6.23}
    Demostrar que todo grupo de orden $12$ con más de un $3-$subgrupo de Sylow es isomorfo al grupo alternado $A_4$.
    \begin{observacion}
        Considerar la acción por traslación de un tal grupo sobre el conjunto de clases módulo $P$, siendo $P$ un $3-$subgrupo de Sylow. Probar que dicha acción es fiel.
    \end{observacion}
\end{ejercicio}

\begin{ejercicio}\label{ej:6.24}~
    \begin{enumerate}
        \item Demostrar que no existen grupos simples de orden $12$. Más concretamente, demostrar que todo grupo de orden $12$ admite un subgrupo normal de orden $3$ o de orden $4$.
        \item Demostrar que no existen grupos simples de orden $28$. Más concretamente, probar que todo grupo de orden $28$ contiene un subgrupo normal de orden $7$.
        \item Demostrar que no existen grupos simples de orden $56$. Más concretamente, probar que todo grupo de orden $56$ contiene un subgrupo normal de orden $7$ o de orden $8$.
        \item Demostrar que no existen grupos simples de orden $148$ ni de orden $200$ ni de orden $351$.
    \end{enumerate}
\end{ejercicio}

\begin{ejercicio}\label{ej:6.25}
    Calcular el número de elementos de orden $7$ que tiene un grupo simple de orden $168$.
\end{ejercicio}
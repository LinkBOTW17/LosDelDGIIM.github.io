\section{Problemas de sesiones prácticas}

\begin{ejercicio}\label{ej:1.1}
    Para todo $n\in \bb{N}$, demostrar que es cierta la siguiente igualdad:
    \begin{equation*}
        \sum_{i=0}^n i = \dfrac{n(n+1)}{2}
    \end{equation*}
    \begin{proof}
        La demostración es por inducción según el principio de inducción matemática y predicado $P(n)$ del contenido literal (tenor):
        \begin{equation*}
            \sum_{i=0}^n i = \dfrac{n(n+1)}{2}
        \end{equation*}

        \begin{itemize}
            \item \ul{Para $n=0$}:
            \begin{equation*}
                \sum_{i=0}^0 i = 0 = \frac{0}{2} = \frac{0\cdot 1}{2}
                 = \dfrac{0(0+1)}{2}
            \end{equation*}
            Por tanto, se tiene $P(0)$.

            \item Como hipótesis de inducción supondremos que $n\in \bb{N}$ y que $P(n)$ ees cierto, es decir, que:
            \begin{equation*}
                \sum_{i=0}^n i = \dfrac{n(n+1)}{2}
            \end{equation*}
            y en el paso de inducción demostraremos que $P(n+1)$ es cierto.
            \begin{align*}
                \sum_{i=0}^{n+1} i &= \sum_{i=0}^n i + (n+1)
                \AstIg \dfrac{n(n+1)}{2} + (n+1)
                =\\&= \dfrac{n(n+1) + 2(n+1)}{2}
                = \dfrac{(n+1)(n+2)}{2}
                = \dfrac{(n+1)((n+1)+1)}{2}
            \end{align*}
            donde en $(\ast)$ he utilizado la hipótesis de inducción. Por tanto, $P(n+1)$ es cierto.
        \end{itemize}

        Por el principio de inducción matemática, sabemos que para todo $n\in \bb{N}$, $P(n)$ es cierto, por lo que se tiene lo que se pedía.
    \end{proof}
\end{ejercicio}

\begin{ejercicio}\label{ej:1.2}
    Demustre que para todo número natural $n$:
    $$\left(\sum_{k=0}^n k\right)^2 = \left(\sum_{k=0}^{n-1}\right)^2 + n^3$$
\begin{proof}~
    En este caso, no se usa la demostración mediante inducción, sino la demostración por casos:
    \begin{itemize}
        \item \ul{$n = 0$:}
            $$\left(\sum_{n=0}^0 k\right)^2 = 0^2 = 0 = 0+0 = \left(\sum_{k=0}^{-1}k\right)^2 + 0^3$$
        \item \ul{$n = 1$:}
            $$\left(\sum_{k=0}^1 k\right)^2 = 0+1 = \left(\sum_{k=0}^0 k\right)^2 + 1^3 = \left(\sum_{k=0}^{n-1} k\right)^2 + n^3$$
        \item \ul{$n > 1$:}
        \begin{align*}
            \left(\sum_{k=0}^n k\right)^2 &= \left[\left(\sum_{k=0}^{n-1} k\right) + n\right]^2 = \left(\sum_{k=0}^{n-1}k\right)^2 + n^2 + 2\left(\sum_{k=0}^{n-1} k\right)n \AstIg\\
            &\AstIg \left(\sum_{k=0}^{n-1}k\right)^2 + n^2 + \cancel{2} \cdot \dfrac{(n-1)n}{\cancel{2}}\cdot n = \left(\sum_{k=0}^{n-1}k\right)^2 + n^2 + (n-1)n^2 =\\
            &= \left(\sum_{k=0}^{n-1}k\right)^2 + n^2 (1+n-1) = \left(\sum_{k=0}^{n-1}k\right)^2 + n^3
        \end{align*}
        donde en $(\ast)$ he utilizado el Ejercicio \ref{ej:1.1}.
    \end{itemize}
\end{proof}
\end{ejercicio}

\begin{ejercicio}[Teorema de Nicomachus]
    Demuestre que para todo número natural $n$ vale la siguiente igualdad:
    $$\sum_{k=0}^n k^3 = \left(\sum_{k=0}^n k\right)^2$$
\begin{proof}
    La demostración es por inducción según el principio de inducción matemática y predicado $P(n)$ del contenido literal (tenor):
    $$\sum_{k=0}^n k^3 = \left(\sum_{k=0}^n k\right)^2$$
    \begin{itemize}
        \item \ul{En el caso base $n = 0$}:
        $$\sum_{k=0}^0 k^3 = 0^3 = 0 = 0^2 = \left(\sum_{k=0}^0 k\right)^2$$
        Y por tanto, $P(0)$ es correcto.
        
        \item Como hipótesis de inducción, supondremos que $n$ es un número natural y que $P(n)$ es cierto; es decir, 
        $$\sum_{k=0}^{n} k^3 = \left(\sum_{k=0}^n k\right)^2$$
        En el paso de inducción, demostraremos que $P(n+1)$ se cumple.
        $$\sum_{k=0}^{n+1} k^3 = \left(\sum_{k=0}^n k^3\right) + (n+1)^3 \AstIg \left(\sum_{k=0}^n k\right)^2 + (n+1)^3 \stackrel{(\ast\ast)}{=} \left(\sum_{k=0}^{n+1} k\right)^2  $$

        donde en $(\ast)$ he utilizado la hipótesis de inducción y en $(\ast\ast)$ he utilizado el Ejercicio \ref{ej:1.2}. Por tanto, $P(n+1)$ es cierto.
        Luego $P(n+1)$ es cierto.
    \end{itemize}
    Por el principio de inducción matemática para todo número natural $n$, $P(n)$ se tiene, como se pedía.
\end{proof}
\end{ejercicio}

\begin{observacion}
    El segundo principio de inducción matemática se utiliza cuando, en vez de usar como hipótesis una verdad sobre $n$, usar una verdad como $n-k$ con $k >1$, cuando estemos demostrado que el predicado vale para $n+1$.
\end{observacion}

% // TODO: Arturo, modifica esto para que salga k esto es un ejemplo de 2o principio de inducción
\begin{ejercicio}[Ejemplo de segundo principio de inducción]
    Todo número natural mayor que 1 tiene al menos un factor primo.
\begin{proof}
    El razonamiento es por el segundo principio de inducción según el predicado (o fórmula) $P(n)$ del tenor:
    $$\mbox{''}n \mbox{ tiene un factor primo}\mbox{''}$$
    donde $n \in \omega \setminus \{ 0, 1\}$ (tenemos que $i_0 = 2$).\\

    \noindent
    Como hipótesis de inducción, supongamos que $n$ es un número natural superior a 1 y que $P(k)$ vale para todo $1<k < n$.\\

    \noindent
    En el paso de inducción distinguimos dos casos:
    
    \begin{itemize}
        \item \ul{$n$ es primo:} 

            En este caso, $n$ es un factor primo de $n$ (note que 2 es un ejemplo de los números en este caso).
        \item \ul{$n$ no es primo:}
            Si $n$ no es primo, existen números naturales $u$ y $v$ tales que $n = uv$ y $1<u,v$. Claro está entonces, que $1<u,v<n$.\newline
            Por la hipótesis de inducción, $P(n)$ vale, luego $u$ tendrá al menos un factor primo, al que podemos llamar $p$. Así pues, $p \mid u$ y por tanto:
            $$p \mid n$$
            Luego $P(n)$ vale y por el segundo principio de inducción, para todo número natural $n$ vale $P(n)$.

    \end{itemize}
\end{proof}
\end{ejercicio}

\begin{observacion}
    Siempre tiene que ocurrir que el caso base ($i_0$) esté incluido en uno de los casos, por eso lo hemos destacado anteriormente con $i_0 = 2$.
\end{observacion}

\begin{observacion}
    No conviene usar en el metalenguaje símbolos del lenguaje.
\end{observacion}

\noindent
% // TODO: Arturo, modifica esto
Ejemplo de buena ordenación
\begin{ejercicio}[Multiplicación por el Método del Campesino Ruso]
Sea $p$ la función dada por:
\begin{align*}
    p(a,0) &= 0,\\
    p(a,b) &= \left\{ \begin{array}{ll}
        p\left(2a,\dfrac{b}{2}\right) & \mbox{si } b \mbox{ es par, }\\
                                      & \\
        p\left(2a,\dfrac{b-1}{2}\right)+a & \mbox{si } b \mbox{ es impar, }
\right.\end{array}
\end{align*}
Demuestre por inducción que para cualesquiera números naturales $a$ y $b$, $p(a,b) = ab$.
\begin{observacion}
    Notemos que esta función pasa a binario $b$ y almacena en $a$ tantos ''$2$'' como el popcount de $b$.
\end{observacion}
\begin{proof}
\end{proof}
\end{ejercicio}

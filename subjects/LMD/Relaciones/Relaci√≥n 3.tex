\section{Lógica Proposicional}

\begin{ejemplo}
    $a\to b$ (siempre que $b\not\equiv a$) es refutable
\end{ejemplo}
\begin{proof}
    Basta tomar una valoración $v$ tal que $v(a)=1$ y $v(b)=0$.

    De hecho, $a\to b$ es satisfacible, luego es una fórmula contingente.
\end{proof}

% // TODO: formatear bien
\begin{ejemplo}
    Todas las siguientes fórmulas proposicionales son tautologías (siempre verdaderas):
    \begin{enumerate}
        \item $(a\rightarrow a)$
        \item $(\alpha\lor \lnot \alpha)$
    \end{enumerate}
\end{ejemplo}
\begin{proof}
    \ 
    \begin{enumerate}
        \item 
            \begin{align*}
                v(a\rightarrow a) &= v(a)v(a)+v(a)+1 \\
                         &= v(a) + v(a) + 1 \\
                         &= 0 + 1 = 1
            \end{align*}
        \item 
            \begin{align*}
                v(\alpha\lor \lnot \alpha) &= v(\alpha)v(\lnot \alpha) + v(\alpha) + v(\lnot \alpha)\\
                                          &= v(\alpha)(v(\alpha)+1) + v(\alpha) + v(\alpha) + 1 \\
                                          &= v(\alpha) + v(\alpha) + v(\alpha) + v(\alpha) + 1\\
                                          &= 0 + 0 + 1 = 1
            \end{align*}
    \end{enumerate}
\end{proof}

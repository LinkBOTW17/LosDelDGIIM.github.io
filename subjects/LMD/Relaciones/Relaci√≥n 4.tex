\section{Álgebra de Boole}

\begin{ejercicio}
    Sea $\bf{B}=\langle B,+,\cdot,\ol{\phantom{a}}, 0,1\rangle$ un álgebra de Boole. Demuestra que, para todo $a,b\in B$ son equivalentes:
    \begin{enumerate}
        \item $a^b=1$.
        \item $a=b$.
    \end{enumerate}
    \begin{proof}
        Demostramos mediante una doble implicación.
        \begin{description}
            \item[$\Longrightarrow)$]
            
            Supongamos que $a^b=1$. Entonces, tenemos que:
            \begin{align*}
                a &= a\cdot 1\\
                &= a\cdot a^b\\
                &= a\cdot (ab + \ol{a}\ol{b})\\
                &= a\cdot ab + a\cdot \ol{a}\ol{b}\\
                &= ab + 0\cdot \ol{b}\\
                &= ab + 0\\
                &= ab\cdot b + \ol{a}\ol{b}\cdot b\\
                &= (ab + \ol{a}\ol{b})\cdot b\\
                &= a^b\cdot b\\
                &= 1\cdot b\\
                &= b.
            \end{align*}

            \item[$\Longleftarrow)$]
            
            Como $a=b$, tenemos que:
            \begin{align*}
                a^b &= a^a\\
                &= a\cdot a + \ol{a}\cdot \ol{a}\\
                &= a + \ol{a}\\
                &= 1.
            \end{align*}
        \end{description}
    \end{proof}
\end{ejercicio}



\begin{ejercicio}
    Sea $\bf{B}=\langle B,+,\cdot,\ol{\phantom{a}}, 0,1\rangle$ un álgebra de Boole. Demuestra que,
    para todo $a,b,c\in B$ son equivalentes:
    \begin{enumerate}
        \item $a+b=a+c$ y $ab=ac$.
        \item $b=c$.
    \end{enumerate}
    \begin{proof}
        Demostramos mediante una doble implicación.
        \begin{description}
            \item[$\Longrightarrow)$]
            
            Supongamos que $a+b=a+c$ y $ab=ac$. Entonces, tenemos que:
            \begin{figure}[H]
                \centering
                \begin{subfigure}[c]{0.4\linewidth}
                    \centering
                    \begin{align*}
                        b &= b + 0\\
                        &= b + (a\ol{a})\\
                        &= (b+a) \cdot (b+\ol{a})\\
                        &= (a+c) \cdot (\ol{a}+b)\\
                        &= (a+c)\ol{a} + (a+c)b\\
                        &= a\ol{a} + c\ol{a} + ab + cb\\
                        &= 0 + c\ol{a} + ac + cb\\
                        &= c\ol{a} + c(a+b)\\
                        &= c(\ol{a} + a + b)\\
                        &= c(1+b)\\
                        &= c\cdot 1\\
                        &= c.
                    \end{align*}
                    \caption{Opción 1.}
                \end{subfigure}\hfill
                \begin{subfigure}[c]{0.4\linewidth}
                    \centering
                    \begin{align*}
                        b &= b\cdot 1\\
                        &= b\cdot (1+a) \\
                        &= b+ab\\
                        &= b+ac\\
                        &= (b+a)(b+c)\\
                        &= (a+c)(b+c)\\
                        &= c+ab\\
                        &= c+ac\\
                        &= c(1+a)\\
                        &= c\cdot 1\\
                        &= c.
                    \end{align*}
                    \caption{Opción 2.}
                \end{subfigure}
            \end{figure}

            \item[$\Longleftarrow)$]
            
            Como $a=a$ y $b=c$, entonces trivialmente $a+b=a+c$ y $ab=ac$.
        \end{description}
    \end{proof}
\end{ejercicio}

\begin{ejercicio}
    Sean $m,n\in \bb{N}$ tales que $\bf{D}(m)$ y $\bf{D}(n)$ son álgebras de Boole. Demuestra que son equivalentes las siguientes afirmaciones:
    \begin{enumerate}
        \item $\bf{D}(m)\times \bf{D}(n)$ es isomorfa a $\bf{D}(mn)$.
        \item $\mcd(m,n)=1$.
    \end{enumerate}
    \begin{proof}
        Demostramos mediante una doble implicación.
        \begin{description}
            \item[$\Longrightarrow)$]
            
            Como $m,n,mn\in \bb{N}$, tenemos que $D(m),D(n),D(mn)$ son conjuntos finitos.
            Supongamos que $\bf{D}(m)\times \bf{D}(n)$ es isomorfa a $\bf{D}(mn)$. Entonces, tenemos que:
            
            \item[$\Longleftarrow)$]
        \end{description}
        % // TODO: Completar demostración.
    \end{proof}
\end{ejercicio}


\begin{ejercicio}
    Calcule el número natural $n$ sabiendo que $\bf{D}(n)$ es un álgebra de Boole,
    que $42$ y $66$ son elementos de $\bf{D}(n)$ y que $42$ es un coátomo.
    Encuentre todos los elementos de $\bf{D}(n)$ tal que $42\cdot \ol{x}=6$.\\

    Como $42$ es un coátomo, se tiene que:
    \begin{equation*}
        42 + x = \mcm(42,x) = \left\{
            \begin{array}{c}
                n \\
                \text{ó} \\
                42
            \end{array}
        \right.
        \hspace{1cm} \text{para todo } x\in D(n)
    \end{equation*}
    Usemos $x=66\in D(n)$ para calcular $n$:
    \begin{equation*}
        \left.
            \begin{array}{rl}
                42 &= 2\cdot 3\cdot 7\\
                66 &= 2\cdot 3\cdot 11
            \end{array}
        \right\} \Longrightarrow \mcm(42,66)=2\cdot 3\cdot 7\cdot 11=462
    \end{equation*}
    Por tanto, deducimos que $n=2\cdot 3\cdot 7\cdot 11=462$.\\
    
    Buscamos ahora los elementos $x\in D(n)$ tales que cumplen:
    \begin{equation*}
        6 = 42\cdot \ol{x} = 42\cdot \frac{n}{x} = \mcd\left(42,\frac{n}{x}\right)
    \end{equation*}
    Busquemos en primer lugar los valores $y\in \bb{N}$ tales que:
    \begin{equation*}
        6 = 3\cdot 2 = \mcd(42,y) = \mcd(2\cdot 3\cdot 7,y) \Longrightarrow
        y = 3\cdot 2 \cdot k
    \end{equation*}
    donde $k\in \bb{N}$ es $1$ o un número primero distinto de $2,~3$ y $7$. Además, como $x\in D(n)$,
    sus factores primos tienen que ser factores primos también de $n$, por lo que $k=1$ o $k=11$. Es decir:
    \begin{align*}
        y = 3\cdot 2 &\Longrightarrow x = \frac{n}{y} = \frac{2\cdot 3\cdot 7\cdot 11}{3\cdot 2} = 7\cdot 11 = 77\\
        y = 3\cdot 2\cdot 11 &\Longrightarrow x = \frac{n}{y} = \frac{2\cdot 3\cdot 7\cdot 11}{3\cdot 2\cdot 11} = 7
    \end{align*}

    Por tanto, los elementos de $D(462)$ tales que $42\cdot \ol{x}=6$ son $x=7$ y $x=77$.
\end{ejercicio}


\begin{ejercicio}
    Sea $n\in \bb{N}$ un número natural tal que $\bf{D}(n)$ es un álgebra de Boole. Demuestra que
    los átomos de $\bf{D}(n)$ son los factores primos de $n$.
    \begin{proof}
        Supongamos que $a\in D(n)$ es un átomo, por lo que $a\neq 1$. Entonces, para todo $x\in D(n)$ se tiene que:
        \begin{equation*}
            ax = \mcd(a,x) = \left\{
                \begin{array}{c}
                    a \\
                    \text{ó} \\
                    1
                \end{array}
            \right.
        \end{equation*}

        Por contrarrecíproco, supongamos que $a\neq 1$ no primo, por lo que $\exists c\in \bb{N}$ tal que $c\mid a$ y $c\neq 1,~c\neq a$,
        y sabemos que $\mcd(a,c)=c$. Usando $x=c\in D(n)$, tenemos que:
        \begin{equation*}
            c = \mcd(a,c) = \left\{
                \begin{array}{c}
                    a \\
                    \text{ó} \\
                    1
                \end{array}
            \right.
        \end{equation*}

        Por tanto, hemos llegado a una contradicción, por lo que $a\in D(n)$ tiene que ser un número primo, es decir, un factor primo de $n$.
    \end{proof}
\end{ejercicio}


\begin{ejercicio}
    Encuentre el polinomio de Zhegalkine de las siguientes expresiones booleanas:
    \begin{enumerate}
        \item $x\subset y$:
        \begin{align*}
            p_Z(x\subset y) &= p_Z(\ol{x} + y)\\
            &= p_Z(\ol{x})p_Z(y) + p_Z(\ol{x}) + p_Z(y)\\
            &= (x+1)y + (x+1) + y\\
            &= xy + y + x + 1 + y\\
            &= xy + x + 1.
        \end{align*}

        \item $x\uparrow y$:
        \begin{align*}
            p_Z(x\uparrow y) &= p_Z(\ol{xy})\\
            &= p_Z(\ol{x} + \ol{y})\\
            &= p_Z(\ol{x})p_Z(\ol{y}) + p_Z(\ol{x}) + p_Z(\ol{y})\\
            &= (x+1)(y+1) + (x+1) + (y+1)\\
            &= xy + x + y + 1 + x + 1 + y + 1\\
            &= xy + 1.
        \end{align*}

        \item $x\downarrow y$:
        \begin{align*}
            p_Z(x\downarrow y) &= p_Z(\ol{x+y})\\
            &= p_Z(\ol{x}~\ol{y})\\
            &= p_Z(\ol{x})p_Z(\ol{y})\\
            &= (x+1)(y+1)\\
            &= xy + x + y + 1.
        \end{align*}

        \item $x\equiv y$:
        \begin{align*}
            p_Z(x\equiv y) &= p_Z(xy + \ol{x}~\ol{y})\\
            &= p_Z(xy) + p_Z(\ol{x}~\ol{y})\\
            &= p_Z(x)p_Z(y) + p_Z(\ol{x})p_Z(\ol{y})\\
            &= xy + (x+1)(y+1)\\
            &= xy + xy + x + y + 1\\
            &= x + y + 1.
        \end{align*}

        \item $x\oplus y$:
        \begin{align*}
            p_Z(x\oplus y) &= p_Z(x\ol{y} + \ol{x}y)\\
            &= p_Z(x\ol{y}) + p_Z(\ol{x}y)\\
            &= p_Z(x)p_Z(\ol{y}) + p_Z(\ol{x})p_Z(y)\\
            &= x(y+1) + (x+1)y\\
            &= xy + x + xy + y\\
            &= x + y.
        \end{align*}
    \end{enumerate}
\end{ejercicio}



\begin{ejercicio} \label{ej:4.4}
    Dadas las funciones de transmisión $s,a:B_2^3\to B_2$ por la Tabla~\ref{tab:4.4} obtener la expresión normal disyuntiva de $s$
    y la expresión normal conjuntiva de $a$.
    \begin{table}[H]
        \centering
        \begin{tabular}{ccc|cc}
            $x$ & $y$ & $z$ & $s$ & $a$\\
            \hline
            $0$ & $0$ & $0$ & $0$ & $0$\\
            $0$ & $0$ & $1$ & $1$ & $0$\\
            $0$ & $1$ & $0$ & $1$ & $0$\\
            $0$ & $1$ & $1$ & $0$ & $1$\\
            $1$ & $0$ & $0$ & $1$ & $0$\\
            $1$ & $0$ & $1$ & $0$ & $1$\\
            $1$ & $1$ & $0$ & $0$ & $1$\\
            $1$ & $1$ & $1$ & $1$ & $1$
        \end{tabular}
        \caption{Tabla de valores de $s$ y $a$ del ejercicio~\ref{ej:4.4}.}
        \label{tab:4.4}
    \end{table}

    Las filas en las que $s(x,y,z)=1$ son las filas:
    \begin{equation*}
        \langle 0,0,1\rangle,~\langle 0,1,0\rangle,~\langle 1,0,0\rangle,~\langle 1,1,1\rangle
    \end{equation*}

    Por tanto, tenemos que la forma normal disyuntiva de $s$ es:
    \begin{equation*}
        s(x,y,z)
        = \sum m(1,2,4,7)
        = \ol{x}~\ol{y}z + \ol{x}y\ol{z} + x\ol{y}~\ol{z} + xyz
    \end{equation*}

    Las filas en las que $a(x,y,z)=0$ son las filas:
    \begin{equation*}
        \langle 0,0,0\rangle,~\langle 0,0,1\rangle,~\langle 0,1,0\rangle,~\langle 1,0,0\rangle
    \end{equation*}

    Por tanto, tenemos que la forma normal conjuntiva de $a$ es:
    \begin{equation*}
        a(x,y,z)
        = \prod M(0,1,2,4)
        = (x+y+z)\cdot (x+y+\ol{z})\cdot (x+\ol{y}+z)\cdot (\ol{x}+y+z)
    \end{equation*}
\end{ejercicio}



\begin{ejercicio}
    Minimice la siguiente función de conmutación $f:B_2^4\to B_2$ usando mapas de Karnaugh:
    \begin{equation*}
        f(a,b,c,d)=\sum m(0,1,2,4, 5, 6, 8, 9, 15)
    \end{equation*}

    \begin{figure}[H]
        \centering
        \begin{karnaugh-map}[4][4][1][$d$][$c$][$b$][$a$]
            \minterms{0,1,2,4,5,6,8,9,15}
            \autoterms[0]
            \implicant{0}{5}
            \implicant{15}{15}
            \implicantedge{0}{4}{2}{6}
            \implicantedge{0}{1}{8}{9}
        \end{karnaugh-map}
    \end{figure}

    Tenemos por tanto que la expresión mínima de $f$ es:
    \begin{equation*}
        f(a, b, c, d) = \ol{a}~\ol{c} + \ol{b}~\ol{c} + \ol{a}~\ol{d} + abcd
    \end{equation*}

\end{ejercicio}



\begin{ejercicio}
    Encuentre una expresión minimal para la función de conmutación $f:B_2^4\to B_2$ dada por:
    \begin{equation*}
        f(a,b,c,d)=\sum m(2,3,7,9,11,13) + \sum d(1,10,15)
    \end{equation*}
    a condición de que sea del tipo SOP.
    \begin{description}
        \item[Opción 1.] Usando mapas de Karnaugh:
        \begin{figure}[H]
            \centering
            \begin{karnaugh-map}[4][4][1][$d$][$c$][$b$][$a$]
                \minterms{2,3,7,9,11,13}
                \indeterminants{1,10,15}
                \autoterms[0]
                \implicant{13}{11}
                \implicant{3}{11}
                \implicantedge{3}{2}{11}{10}
            \end{karnaugh-map}
        \end{figure}
    
        Por tanto, la expresión minimal de $f$ es:
        \begin{equation*}
            f(a,b,c,d) = ad + cd + \ol{b}~c
        \end{equation*}

        \item[Opción 2.] Usando el algoritmo de Quine-McCluskey:
        
        En primer lugar, generamos los implicantes primos:
        \begin{table}[H]
            \centering
            \begin{tabular}{rcc|rcc|rcc}
                \multicolumn{3}{c}{Columna 1} & \multicolumn{3}{|c|}{Columna 2} & \multicolumn{3}{c}{Columna 3} \\ \hline
                1 & 0001 & \checkmark & \{1,3\} & 00\_1 & \checkmark & \{1,3,9,11\} & \_0\_1 & $\ast$
                \\
                2 & 0010 & \checkmark & \{1,9\} & \_001 & \checkmark & \{2,3,10,11\} & \_01\_ & $\ast$
                \\ \cline{1-3} \cline{7-9}
                3 & 0011 & \checkmark & \{2,3\} & 001\_ & \checkmark & \{3,7,11,15\} & \_\_11 & $\ast$
                \\
                9 & 1001 & \checkmark & \{2,10\} & \_010 & \checkmark & \{9,11,13,15\} & 1\_\_1 & $\ast$
                \\ \cline{4-6} \cline{7-9}
                10 & 1010& \checkmark & \{3,7\} & 0\_11 & \checkmark
                \\ \cline{1-3}
                7 & 0111 & \checkmark & \{3,11\} & \_011 & \checkmark
                \\
                11 & 1011 & \checkmark& \{9,11\} & 10\_1 & \checkmark
                \\
                13 & 1101 & \checkmark& \{9,13\} & 1\_01 & \checkmark
                \\ \cline{1-3}
                15 & 1111 & \checkmark& \{10,11\} & 101\_ & \checkmark
                \\ \cline{1-3} \cline{4-6}
                & & & \{7,15\} & \_111 & \checkmark
                \\
                & & & \{11,15\} & 1\_11 & \checkmark
                \\
                & & & \{13,15\} & 11\_1 & \checkmark
                \\ \cline{4-6}
            \end{tabular}
        \end{table}
        
        Los implicantes primos son, por tanto, los que se han marcado con $\ast$.
        Ahora, reducimos la tabla de implicantes primos:
        \begin{table}[H]
            \centering
            \begin{tabular}{c|ll|cccccc}
                && & \tikzmark{col2Start}2 & \tikzmark{col3Start}3 & 7 & \tikzmark{col9Start}9 & \tikzmark{col11Start}11 & 13 \\ \hline
                &\{1,3,9,11\} & \_0\_1 & & $\circ$ && $\circ$ & $\circ$ \\
                {\color{red}$\ast$}&\{2,3,10,11\} & \_01\_ &\tikzmark{fil2Start}$\circ$ & $\circ$ &&&$\circ$ & \tikzmark{fil2End} \\
                {\color{teal}$\ast$}&\{3,7,11,15\} & \_\_11 &\tikzmark{fil7Start}& $\circ$ & $\circ$ & & $\circ$ & \tikzmark{fil7End}\\
                {\color{blue}$\ast$}&\{9,11,13,15\} & 1\_\_1 &\tikzmark{col2End}\tikzmark{fil13Start}&\tikzmark{col3End}&& \tikzmark{col9End}$\circ$ &\tikzmark{col11End}$\circ$ & $\circ$\tikzmark{fil13End} \\
            \end{tabular}
            \tikz[remember picture] \draw[overlay, red] ([yshift=.25em]pic cs:fil2Start) -- ([yshift=.25em]pic cs:fil2End);
            \tikz[remember picture] \draw[overlay, red] ([xshift=.25em]pic cs:col2Start) -- (pic cs:col2End);
            \tikz[remember picture] \draw[overlay, red] ([xshift=.25em]pic cs:col3Start) -- (pic cs:col3End);
            \tikz[remember picture] \draw[overlay, red] ([xshift=.45em]pic cs:col11Start) -- ([xshift=.30em]pic cs:col11End);
            \tikz[remember picture] \draw[overlay, teal] ([yshift=.25em]pic cs:fil7Start) -- ([yshift=.25em]pic cs:fil7End);
            \tikz[remember picture] \draw[overlay, blue] ([yshift=.25em]pic cs:fil13Start) -- ([yshift=.25em]pic cs:fil13End);
            \tikz[remember picture] \draw[overlay, blue] ([xshift=.25em]pic cs:col9Start) -- ([xshift=.25em]pic cs:col9End);
        \end{table}

        Por tanto, la expresión minimal de $f$ es:
        \begin{equation*}
            f(a,b,c,d) = \ol{b}~c + cd + ad
        \end{equation*}
    \end{description}

    Calculemos ahora cuál es el coste de la función $f$:
    \begin{figure}[H]
        \centering
        \begin{forest}
            [$+$
                [$\cdot$
                    [$\ol{b}$]
                    [$c$]
                ]
                [$\cdot$
                    [$c$]
                    [$d$]
                ]
                [$\cdot$
                    [$a$]
                    [$d$]
                ]
            ]
        \end{forest}
    \end{figure}

    Tenemos que hay:
    \begin{itemize}
        \item $1$ suma.
        \item $3$ productos.
        \item $9$ ejes.
    \end{itemize}

    Por tanto, el coste es $1+3+9=13$.
\end{ejercicio}


% // Autoduales
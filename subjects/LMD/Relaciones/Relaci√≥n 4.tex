\section{Álgebra de Boole}

\begin{ejercicio}
    Sea $\bf{B}=\langle B,+,\cdot,\ol{\phantom{a}}, 0,1\rangle$ un álgebra de Boole. Demuestra que,
    para todo $a,b,c\in B$ son equivalentes:
    \begin{enumerate}
        \item $a+b=a+c$ y $ab=ac$.
        \item $b=c$.
    \end{enumerate}
    \begin{proof}
        Demostramos mediante una doble implicación.
        \begin{description}
            \item[$\Longrightarrow)$]
            
            Supongamos que $a+b=a+c$ y $ab=ac$. Entonces, tenemos que:
            \begin{align*}
                b &= b + 0\\
                &= b + (a\ol{a})\\
                &= (b+a) \cdot (b+\ol{a})\\
                &= (a+c) \cdot (\ol{a}+b)\\
                &= (a+c)\ol{a} + (a+c)b\\
                &= a\ol{a} + c\ol{a} + ab + cb\\
                &= 0 + c\ol{a} + ac + cb\\
                &= c\ol{a} + c(a+b)\\
                &= c(\ol{a} + a + b)\\
                &= c(1+b)\\
                &= c\cdot 1\\
                &= c.
            \end{align*}

            \item[$\Longleftarrow)$]
            
            Como $a=a$ y $b=c$, entonces $a+b=a+c$ y $ab=ac$.
        \end{description}
    \end{proof}
\end{ejercicio}


\begin{ejercicio}
    Sean $m,n\in \bb{N}$ tales que $\bf{D}(m)$ y $\bf{D}(n)$ son álgebras de Boole. Demuestra que son equivalentes las siguientes afirmaciones:
    \begin{enumerate}
        \item $\bf{D}(m)\times \bf{D}(n)$ es isomorfa a $\bf{D}(mn)$.
        \item $\mcd(m,n)=1$.
    \end{enumerate}
    \begin{proof}
        Demostramos mediante una doble implicación.
        \begin{description}
            \item[$\Longrightarrow)$]
            
            Como $m,n,mn\in \bb{N}$, tenemos que $D(m),D(n),D(mn)$ son conjuntos finitos.
            Supongamos que $\bf{D}(m)\times \bf{D}(n)$ es isomorfa a $\bf{D}(mn)$. Entonces, tenemos que:
            
            \item[$\Longleftarrow)$]
        \end{description}
        % // TODO: Completar demostración.
    \end{proof}
\end{ejercicio}


\begin{ejercicio}
    Calcule el número natural $n$ sabiendo que $\bf{D}(n)$ es un álgebra de Boole,
    que $42$ y $66$ son elementos de $\bf{D}(n)$ y que $42$ es un coátomo.
    Encuentre todos los elementos de $\bf{D}(n)$ tal que $42\cdot \ol{x}=6$.\\

    Como $42$ es un coátomo, se tiene que:
    \begin{equation*}
        42 + x = \mcm(42,x) = \left\{
            \begin{array}{c}
                n \\
                \text{ó} \\
                42
            \end{array}
        \right.
        \hspace{1cm} \text{para todo } x\in D(n)
    \end{equation*}
    Usemos $x=66\in D(n)$ para calcular $n$:
    \begin{equation*}
        \left.
            \begin{array}{rl}
                42 &= 2\cdot 3\cdot 7\\
                66 &= 2\cdot 3\cdot 11
            \end{array}
        \right\} \Longrightarrow \mcm(42,66)=2\cdot 3\cdot 7\cdot 11=462
    \end{equation*}
    Por tanto, deducimos que $n=2\cdot 3\cdot 7\cdot 11=462$.\\
    
    Buscamos ahora los elementos $x\in D(n)$ tales que cumplen:
    \begin{equation*}
        6 = 42\cdot \ol{x} = 42\cdot \frac{n}{x} = \mcd\left(42,\frac{n}{x}\right)
    \end{equation*}
    Busquemos en primer lugar los valores $y\in \bb{N}$ tales que:
    \begin{equation*}
        6 = 3\cdot 2 = \mcd(42,y) = \mcd(2\cdot 3\cdot 7,y) \Longrightarrow
        y = 3\cdot 2 \cdot k
    \end{equation*}
    donde $k\in \bb{N}$ es $1$ o un número primero distinto de $2,~3$ y $7$. Además, como $x\in D(n)$,
    sus factores primos tienen que ser factores primos también de $n$, por lo que $k=1$ o $k=11$. Es decir:
    \begin{align*}
        y = 3\cdot 2 &\Longrightarrow x = \frac{n}{y} = \frac{2\cdot 3\cdot 7\cdot 11}{3\cdot 2} = 7\cdot 11 = 77\\
        y = 3\cdot 2\cdot 11 &\Longrightarrow x = \frac{n}{y} = \frac{2\cdot 3\cdot 7\cdot 11}{3\cdot 2\cdot 11} = 7
    \end{align*}

    Por tanto, los elementos de $D(462)$ tales que $42\cdot \ol{x}=6$ son $x=7$ y $x=77$.
\end{ejercicio}
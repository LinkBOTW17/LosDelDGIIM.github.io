\section{Álgebra de Boole}

\begin{ejercicio}
    Sea $\bf{B}=\langle B,+,\cdot,\ol{\phantom{a}}, 0,1\rangle$ un álgebra de Boole. Demuestra que, para todo $a,b\in B$ son equivalentes:
    \begin{enumerate}
        \item $a^b=1$.
        \item $a=b$.
    \end{enumerate}
    \begin{proof}
        Demostramos mediante una doble implicación.
        \begin{description}
            \item[$\Longrightarrow)$]
            
            Supongamos que $a^b=1$. Entonces, tenemos que:
            \begin{align*}
                a &= a\cdot 1\\
                &= a\cdot a^b\\
                &= a\cdot (ab + \ol{a}\ol{b})\\
                &= a\cdot ab + a\cdot \ol{a}\ol{b}\\
                &= ab + 0\cdot \ol{b}\\
                &= ab + 0\\
                &= ab\cdot b + \ol{a}\ol{b}\cdot b\\
                &= (ab + \ol{a}\ol{b})\cdot b\\
                &= a^b\cdot b\\
                &= 1\cdot b\\
                &= b.
            \end{align*}

            \item[$\Longleftarrow)$]
            
            Como $a=b$, tenemos que:
            \begin{align*}
                a^b &= a^a\\
                &= a\cdot a + \ol{a}\cdot \ol{a}\\
                &= a + \ol{a}\\
                &= 1.
            \end{align*}
        \end{description}
    \end{proof}
\end{ejercicio}



\begin{ejercicio}
    Sea $\bf{B}=\langle B,+,\cdot,\ol{\phantom{a}}, 0,1\rangle$ un álgebra de Boole. Demuestra que,
    para todo $a,b,c\in B$ son equivalentes:
    \begin{enumerate}
        \item $a+b=a+c$ y $ab=ac$.
        \item $b=c$.
    \end{enumerate}
    \begin{proof}
        Demostramos mediante una doble implicación.
        \begin{description}
            \item[$\Longrightarrow)$]
            
            Supongamos que $a+b=a+c$ y $ab=ac$. Entonces, tenemos que:
            \begin{figure}[H]
                \centering
                \begin{subfigure}[c]{0.4\linewidth}
                    \centering
                    \begin{align*}
                        b &= b + 0\\
                        &= b + (a\ol{a})\\
                        &= (b+a) \cdot (b+\ol{a})\\
                        &= (a+c) \cdot (\ol{a}+b)\\
                        &= (a+c)\ol{a} + (a+c)b\\
                        &= a\ol{a} + c\ol{a} + ab + cb\\
                        &= 0 + c\ol{a} + ac + cb\\
                        &= c\ol{a} + c(a+b)\\
                        &= c(\ol{a} + a + b)\\
                        &= c(1+b)\\
                        &= c\cdot 1\\
                        &= c.
                    \end{align*}
                    \caption{Opción 1.}
                \end{subfigure}\hfill
                \begin{subfigure}[c]{0.4\linewidth}
                    \centering
                    \begin{align*}
                        b &= b\cdot 1\\
                        &= b\cdot (1+a) \\
                        &= b+ab\\
                        &= b+ac\\
                        &= (b+a)(b+c)\\
                        &= (a+c)(b+c)\\
                        &= c+ab\\
                        &= c+ac\\
                        &= c(1+a)\\
                        &= c\cdot 1\\
                        &= c.
                    \end{align*}
                    \caption{Opción 2.}
                \end{subfigure}
            \end{figure}

            \item[$\Longleftarrow)$]
            
            Como $a=a$ y $b=c$, entonces trivialmente $a+b=a+c$ y $ab=ac$.
        \end{description}
    \end{proof}
\end{ejercicio}



\begin{ejercicio}
    Sea $n\in \bb{N}$ un número natural tal que $\bf{D}(n)$ es un álgebra de Boole. Demuestra que
    los átomos de $\bf{D}(n)$ son los factores primos de $n$.
    \begin{proof}
        Supongamos que $a\in D(n)$ es un átomo, por lo que $a\neq 1$. Entonces, para todo $x\in D(n)$ se tiene que:
        \begin{equation*}
            ax = \mcd(a,x) = \left\{
                \begin{array}{c}
                    a \\
                    \text{ó} \\
                    1
                \end{array}
            \right.
        \end{equation*}

        Por contrarrecíproco, supongamos que $a\neq 1$ no primo, por lo que $\exists c\in \bb{N}$ tal que $c\mid a$ y $c\neq 1,~c\neq a$,
        y sabemos que $\mcd(a,c)=c$. Usando $x=c\in D(n)$, tenemos que:
        \begin{equation*}
            c = \mcd(a,c) = \left\{
                \begin{array}{c}
                    a \\
                    \text{ó} \\
                    1
                \end{array}
            \right.
        \end{equation*}

        Por tanto, hemos llegado a una contradicción, por lo que $a\in D(n)$ tiene que ser un número primo, es decir, un factor primo de $n$.
    \end{proof}
\end{ejercicio}




\begin{ejercicio}
    Calcule el número natural $n$ sabiendo que $\bf{D}(n)$ es un álgebra de Boole,
    que $42$ y $66$ son elementos de $\bf{D}(n)$ y que $42$ es un coátomo.
    Encuentre todos los elementos de $\bf{D}(n)$ tal que $42\cdot \ol{x}=6$.\\

    Como $42$ es un coátomo, se tiene que:
    \begin{equation*}
        42 + x = \mcm(42,x) = \left\{
            \begin{array}{c}
                n \\
                \text{ó} \\
                42
            \end{array}
        \right.
        \hspace{1cm} \text{para todo } x\in D(n)
    \end{equation*}
    Usemos $x=66\in D(n)$ para calcular $n$:
    \begin{equation*}
        \left.
            \begin{array}{rl}
                42 &= 2\cdot 3\cdot 7\\
                66 &= 2\cdot 3\cdot 11
            \end{array}
        \right\} \Longrightarrow \mcm(42,66)=2\cdot 3\cdot 7\cdot 11=462
    \end{equation*}
    Por tanto, deducimos que $n=2\cdot 3\cdot 7\cdot 11=462$.\\
    
    Buscamos ahora los elementos $x\in D(n)$ tales que cumplen:
    \begin{equation*}
        6 = 42\cdot \ol{x} = 42\cdot \frac{n}{x} = \mcd\left(42,\frac{n}{x}\right)
    \end{equation*}
    Busquemos en primer lugar los valores $y\in \bb{N}$ tales que:
    \begin{equation*}
        6 = 3\cdot 2 = \mcd(42,y) = \mcd(2\cdot 3\cdot 7,y) \Longrightarrow
        y = 3\cdot 2 \cdot k
    \end{equation*}
    donde $k\in \bb{N}$ es $1$ o un número primero distinto de $2,~3$ y $7$. Además, como $x\in D(n)$,
    sus factores primos tienen que ser factores primos también de $n$, por lo que $k=1$ o $k=11$. Es decir:
    \begin{align*}
        y = 3\cdot 2 &\Longrightarrow x = \frac{n}{y} = \frac{2\cdot 3\cdot 7\cdot 11}{3\cdot 2} = 7\cdot 11 = 77\\
        y = 3\cdot 2\cdot 11 &\Longrightarrow x = \frac{n}{y} = \frac{2\cdot 3\cdot 7\cdot 11}{3\cdot 2\cdot 11} = 7
    \end{align*}

    Por tanto, los elementos de $D(462)$ tales que $42\cdot \ol{x}=6$ son $x=7$ y $x=77$.
\end{ejercicio}


\begin{ejercicio}
    Sean $m,n\in \bb{N}$ tales que $\bf{D}(m)$ y $\bf{D}(n)$ son álgebras de Boole. Demuestra que son equivalentes las siguientes afirmaciones:
    \begin{enumerate}
        \item $\bf{D}(mn)$ es un álgebra de Boole.
        \item $\mcd(m,n)=1$.
    \end{enumerate}
    En el caso de que se tenga $1)$ y $2)$, demuestra que $\bf{D}(m)\times \bf{D}(n)$ es isomorfa a $\bf{D}(mn)$.
    \begin{proof}
        En primer lugar, consideramos la descomposición en factores primos de $m$ y $n$:
        \begin{equation*}
            m = \prod_{i=1}^{k_m} m_i \hspace{1cm} n = \prod_{j=1}^{k_n} n_j
        \end{equation*}
        donde $k_m,k_n\in \bb{N}$ denotan el número de factores primos de $m$ y $n$ respectivamente,
        y $m_i,n_j\in \bb{N}$ son los factores primos de $m$ y $n$ respectivamente. Como $\bf{D}(m)$ y $\bf{D}(n)$
        son álgebras de Boole, entonces $m_i\neq m_j$ para todo $i\neq j$ y $n_i\neq n_j$ para todo $i\neq j$.
        Tenemos entonces que:
        \begin{equation*}
            m \cdot n = m_1\cdot m_2\cdots m_{k_m}\cdot n_1\cdot n_2\cdots n_{k_n}
        \end{equation*}
        siendo todos los factores primos.

        Demostramos ahora lo buscado mediante una doble implicación.
        \begin{description}
            \item[$\Longrightarrow)$]
            
            Supongamos que $\bf{D}(mn)$ es un álgebra de Boole. Entonces, todos los factores primos de $mn$ son distintos, por lo que $m_i\neq n_j$ para todo $i,j$. Por tanto,
            tenemos que $m,n$ no tienen factores primos en común, es decir, $\mcd(m,n)=1$.
            
            
            
            \item[$\Longleftarrow)$]
            
            Como $\mcd(m,n)=1$, entonces $m_i\neq n_j$ para todo $i,j$. Por tanto, todos los factores primos de $mn$ son distintos, por lo que $\bf{D}(mn)$ es un álgebra de Boole.
        \end{description}
        
        Una vez demostrada la equivalencia, buscamos ahora demostrar que $\bf{D}(m)\times \bf{D}(n)$ es isomorfa a $\bf{D}(mn)$ en el caso de que la segunda sea un álgebra de Boole.
        Tenemos:
        \begin{align*}
            \Atm(\bf{D}(m)) &= \{m_1,m_2,\ldots,m_{k_m}\} \\
            \Atm(\bf{D}(n)) &= \{n_1,n_2,\ldots,n_{k_n}\} \\
            \Atm(\bf{D}(mn)) &= \{m_1,m_2,\ldots,m_{k_m},n_1,n_2,\ldots,n_{k_n}\}
        \end{align*}

        Por tanto, teniendo en cuenta que el $0$ de este tipo de álgebras es el $1$ de los enteros, tenemos que los átomos de $\bf{D}(m)\times \bf{D}(n)$ son:
        \begin{align*}
            \Atm(\bf{D}(m)\times \bf{D}(n)) &= \left\{\langle m_i,1\rangle, \langle n_j,1\rangle\mid m_i\in \Atm(\bf{D}(m)),~n_j\in \Atm(\bf{D}(n))\right\}
        \end{align*}

        Sabemos entonces que:
        \begin{align*}
            \Car(\Atm(\bf{D}(mn))) &= k_m + k_n\\
            \Car(\Atm(\bf{D}(m)\times \bf{D}(n))) &= \Car(\Atm(\bf{D}(m))) + \Car(\Atm(\bf{D}(n))) = k_m + k_n
        \end{align*}
        
        Como tenemos que ambos cardinales coinciden, tenemos que tienen el mismo número de átomos, por lo que su cardinal es el mismo y, por tanto, son isomorfas.
    \end{proof}
\end{ejercicio}


\begin{ejercicio}
    Encuentre el polinomio de Zhegalkine de las siguientes expresiones booleanas:
    \begin{enumerate}
        \item $x\supset y$:
        \begin{align*}
            p_Z(x\supset y) &= p_Z(\ol{x} + y)\\
            &= p_Z(\ol{x})p_Z(y) + p_Z(\ol{x}) + p_Z(y)\\
            &= (x+1)y + (x+1) + y\\
            &= xy + y + x + 1 + y\\
            &= xy + x + 1.
        \end{align*}

        \item $x\uparrow y$:
        \begin{align*}
            p_Z(x\uparrow y) &= p_Z(\ol{xy})\\
            &= p_Z(\ol{x} + \ol{y})\\
            &= p_Z(\ol{x})p_Z(\ol{y}) + p_Z(\ol{x}) + p_Z(\ol{y})\\
            &= (x+1)(y+1) + (x+1) + (y+1)\\
            &= xy + x + y + 1 + x + 1 + y + 1\\
            &= xy + 1.
        \end{align*}

        \item $x\downarrow y$:
        \begin{align*}
            p_Z(x\downarrow y) &= p_Z(\ol{x+y})\\
            &= p_Z(\ol{x}~\ol{y})\\
            &= p_Z(\ol{x})p_Z(\ol{y})\\
            &= (x+1)(y+1)\\
            &= xy + x + y + 1.
        \end{align*}

        \item $x\equiv y$:
        \begin{align*}
            p_Z(x\equiv y) &= p_Z(xy + \ol{x}~\ol{y})\\
            &= p_Z(xy) + p_Z(\ol{x}~\ol{y})\\
            &= p_Z(x)p_Z(y) + p_Z(\ol{x})p_Z(\ol{y})\\
            &= xy + (x+1)(y+1)\\
            &= xy + xy + x + y + 1\\
            &= x + y + 1.
        \end{align*}

        \item $x\oplus y$:
        \begin{align*}
            p_Z(x\oplus y) &= p_Z(x\ol{y} + \ol{x}y)\\
            &= p_Z(x\ol{y}) + p_Z(\ol{x}y)\\
            &= p_Z(x)p_Z(\ol{y}) + p_Z(\ol{x})p_Z(y)\\
            &= x(y+1) + (x+1)y\\
            &= xy + x + xy + y\\
            &= x + y.
        \end{align*}
    \end{enumerate}
\end{ejercicio}


\begin{ejercicio}
    Sea $\bf{B}=\langle B,+,\cdot,\ol{\phantom{a}}, 0,1\rangle$ un álgebra de Boole. Demuestra que, para todo $x,y\in B$ son equivalentes:
    \begin{enumerate}
        \item $x\leq y$
        \item $x+y=y$
        \item $x\supset y=1$
    \end{enumerate}
    \begin{proof}
        Demostramos mediante implicaciones sucesivas.
        \begin{description}
            \item[$1\Longrightarrow 2)$]
            
            Supongamos que $x\leq y$, es decir, $xy=x$. Entonces, tenemos que:
            \begin{align*}
                x+y &= xy +y = y(x+1) = y
            \end{align*}

            \item[$2\Longrightarrow 3)$]
            
            Supongamos que $x+y=y$. Entonces, tenemos que:
            \begin{align*}
                x\supset y &= \ol{x} + y\\
                &= \ol{x} + (x+y)\\
                &= (\ol{x} + x) + y\\
                &= 1 + y\\
                &= 1.
            \end{align*}

            \item[$3\Longrightarrow 1)$]
            
            Supongamos que $x\supset y=1$, es decir, $\ol{x}+y=1$. Entonces, tenemos que:
            \begin{align*}
                x &= x1\\
                &= x(\ol{x}+y)\\
                &= x~\ol{x} + xy\\
                &= 0 + xy\\
                &= xy
            \end{align*}
            Por tanto, $x\leq y$.
        \end{description}
    \end{proof}
\end{ejercicio}



\begin{ejercicio} \label{ej:4.4}
    Dadas las funciones de transmisión $s,a:B_2^3\to B_2$ por la Tabla~\ref{tab:4.4} obtener la expresión normal disyuntiva de $s$
    y la expresión normal conjuntiva de $a$.
    \begin{table}[H]
        \centering
        \begin{tabular}{ccc|cc}
            $x$ & $y$ & $z$ & $s$ & $a$\\
            \hline
            $0$ & $0$ & $0$ & $0$ & $0$\\
            $0$ & $0$ & $1$ & $1$ & $0$\\
            $0$ & $1$ & $0$ & $1$ & $0$\\
            $0$ & $1$ & $1$ & $0$ & $1$\\
            $1$ & $0$ & $0$ & $1$ & $0$\\
            $1$ & $0$ & $1$ & $0$ & $1$\\
            $1$ & $1$ & $0$ & $0$ & $1$\\
            $1$ & $1$ & $1$ & $1$ & $1$
        \end{tabular}
        \caption{Tabla de valores de $s$ y $a$ del ejercicio~\ref{ej:4.4}.}
        \label{tab:4.4}
    \end{table}

    Las filas en las que $s(x,y,z)=1$ son las filas:
    \begin{equation*}
        \langle 0,0,1\rangle,~\langle 0,1,0\rangle,~\langle 1,0,0\rangle,~\langle 1,1,1\rangle
    \end{equation*}

    Por tanto, tenemos que la forma normal disyuntiva de $s$ es:
    \begin{equation*}
        s(x,y,z)
        = \sum m(1,2,4,7)
        = \ol{x}~\ol{y}z + \ol{x}y\ol{z} + x\ol{y}~\ol{z} + xyz
    \end{equation*}

    Las filas en las que $a(x,y,z)=0$ son las filas:
    \begin{equation*}
        \langle 0,0,0\rangle,~\langle 0,0,1\rangle,~\langle 0,1,0\rangle,~\langle 1,0,0\rangle
    \end{equation*}

    Por tanto, tenemos que la forma normal conjuntiva de $a$ es:
    \begin{equation*}
        a(x,y,z)
        = \prod M(0,1,2,4)
        = (x+y+z)\cdot (x+y+\ol{z})\cdot (x+\ol{y}+z)\cdot (\ol{x}+y+z)
    \end{equation*}
\end{ejercicio}



\begin{ejercicio}
    Minimice la siguiente función de conmutación $f:B_2^4\to B_2$ usando mapas de Karnaugh:
    \begin{equation*}
        f(a,b,c,d)=\sum m(0,1,2,4, 5, 6, 8, 9, 15)
    \end{equation*}

    \begin{figure}[H]
        \centering
        \begin{karnaugh-map}[4][4][1][$d$][$c$][$b$][$a$]
            \minterms{0,1,2,4,5,6,8,9,15}
            \autoterms[0]
            \implicant{0}{5}
            \implicant{15}{15}
            \implicantedge{0}{4}{2}{6}
            \implicantedge{0}{1}{8}{9}
        \end{karnaugh-map}
    \end{figure}

    Tenemos por tanto que la expresión mínima de $f$ es:
    \begin{equation*}
        f(a, b, c, d) = \ol{a}~\ol{c} + \ol{b}~\ol{c} + \ol{a}~\ol{d} + abcd
    \end{equation*}

\end{ejercicio}



\begin{ejercicio}
    Encuentre una expresión minimal para la función de conmutación $f:B_2^4\to B_2$ dada por:
    \begin{equation*}
        f(a,b,c,d)=\sum m(2,3,7,9,11,13) + \sum d(1,10,15)
    \end{equation*}
    a condición de que sea del tipo SOP.
    \begin{description}
        \item[Opción 1.] Usando mapas de Karnaugh:
        \begin{figure}[H]
            \centering
            \begin{karnaugh-map}[4][4][1][$d$][$c$][$b$][$a$]
                \minterms{2,3,7,9,11,13}
                \indeterminants{1,10,15}
                \autoterms[0]
                \implicant{13}{11}
                \implicant{3}{11}
                \implicantedge{3}{2}{11}{10}
            \end{karnaugh-map}
        \end{figure}
    
        Por tanto, la expresión minimal de $f$ es:
        \begin{equation*}
            f(a,b,c,d) = ad + cd + \ol{b}~c
        \end{equation*}

        \item[Opción 2.] Usando el algoritmo de Quine-McCluskey:
        
        En primer lugar, generamos los implicantes primos:
        \begin{table}[H]
            \centering
            \begin{tabular}{rcc|rcc|rcc}
                \multicolumn{3}{c}{Columna 1} & \multicolumn{3}{|c|}{Columna 2} & \multicolumn{3}{c}{Columna 3} \\ \hline
                1 & 0001 & \checkmark & \{1,3\} & 00\_1 & \checkmark & \{1,3,9,11\} & \_0\_1 & $\ast$
                \\
                2 & 0010 & \checkmark & \{1,9\} & \_001 & \checkmark & \{2,3,10,11\} & \_01\_ & $\ast$
                \\ \cline{1-3} \cline{7-9}
                3 & 0011 & \checkmark & \{2,3\} & 001\_ & \checkmark & \{3,7,11,15\} & \_\_11 & $\ast$
                \\
                9 & 1001 & \checkmark & \{2,10\} & \_010 & \checkmark & \{9,11,13,15\} & 1\_\_1 & $\ast$
                \\ \cline{4-6} \cline{7-9}
                10 & 1010& \checkmark & \{3,7\} & 0\_11 & \checkmark
                \\ \cline{1-3}
                7 & 0111 & \checkmark & \{3,11\} & \_011 & \checkmark
                \\
                11 & 1011 & \checkmark& \{9,11\} & 10\_1 & \checkmark
                \\
                13 & 1101 & \checkmark& \{9,13\} & 1\_01 & \checkmark
                \\ \cline{1-3}
                15 & 1111 & \checkmark& \{10,11\} & 101\_ & \checkmark
                \\ \cline{1-3} \cline{4-6}
                & & & \{7,15\} & \_111 & \checkmark
                \\
                & & & \{11,15\} & 1\_11 & \checkmark
                \\
                & & & \{13,15\} & 11\_1 & \checkmark
                \\ \cline{4-6}
            \end{tabular}
        \end{table}
        
        Los implicantes primos son, por tanto, los que se han marcado con $\ast$.
        Ahora, reducimos la tabla de implicantes primos:
        \begin{table}[H]
            \centering
            \begin{tabular}{c|ll|cccccc}
                && & \tikzmark{Ej1col2Start}2 & \tikzmark{Ej1col3Start}3 & 7 & \tikzmark{Ej1col9Start}9 & \tikzmark{Ej1col11Start}11 & 13 \\ \hline
                &\{1,3,9,11\} & \_0\_1 & & $\circ$ && $\circ$ & $\circ$ \\
                {\color{red}$\ast$}&\{2,3,10,11\} & \_01\_ &\tikzmark{Ej1fil2Start}$\circ$ & $\circ$ &&&$\circ$ & \tikzmark{Ej1fil2End} \\
                {\color{teal}$\ast$}&\{3,7,11,15\} & \_\_11 &\tikzmark{Ej1fil7Start}& $\circ$ & $\circ$ & & $\circ$ & \tikzmark{Ej1fil7End}\\
                {\color{blue}$\ast$}&\{9,11,13,15\} & 1\_\_1 &\tikzmark{Ej1col2End}\tikzmark{Ej1fil13Start}&\tikzmark{Ej1col3End}&& \tikzmark{Ej1col9End}$\circ$ &\tikzmark{Ej1col11End}$\circ$ & $\circ$\tikzmark{Ej1fil13End} \\
            \end{tabular}
            \tikz[remember picture] \draw[overlay, red] ([yshift=.25em]pic cs:Ej1fil2Start) -- ([yshift=.25em]pic cs:Ej1fil2End);
            \tikz[remember picture] \draw[overlay, red] ([xshift=0.4em]pic cs:Ej1col2Start) -- (pic cs:Ej1col2End);
            \tikz[remember picture] \draw[overlay, red] ([xshift=0.4em]pic cs:Ej1col3Start) -- (pic cs:Ej1col3End);
            \tikz[remember picture] \draw[overlay, red] ([xshift=.47em]pic cs:Ej1col11Start) -- ([xshift=.30em]pic cs:Ej1col11End);
            \tikz[remember picture] \draw[overlay, teal] ([yshift=.25em]pic cs:Ej1fil7Start) -- ([yshift=.25em]pic cs:Ej1fil7End);
            \tikz[remember picture] \draw[overlay, blue] ([yshift=.25em]pic cs:Ej1fil13Start) -- ([yshift=.25em]pic cs:Ej1fil13End);
            \tikz[remember picture] \draw[overlay, blue] ([xshift=.25em]pic cs:Ej1col9Start) -- ([xshift=.25em]pic cs:Ej1col9End);
        \end{table}

        Por tanto, la expresión minimal de $f$ como SOP es:
        \begin{equation*}
            f(a,b,c,d) = \ol{b}~c + cd + ad
        \end{equation*}
    \end{description}

    Calculemos ahora cuál es el coste de la expresión minimal de $f$ como SOP:
    \begin{figure}[H]
        \centering
        \begin{forest}
            [$+$
                [$\cdot$
                    [$\ol{b}$]
                    [$c$]
                ]
                [$\cdot$
                    [$c$]
                    [$d$]
                ]
                [$\cdot$
                    [$a$]
                    [$d$]
                ]
            ]
        \end{forest}
    \end{figure}

    Tenemos que hay:
    \begin{itemize}
        \item $1$ suma.
        \item $3$ productos.
        \item $9$ ejes.
    \end{itemize}

    Por tanto, el coste es $1+3+9=13$.

    \begin{observacion}
        Notemos que este coste no es el mínimo. Notemos que podríamos expresar $f$ de la siguiente forma:
        \begin{equation*}
            f(a,b,c,d) = \ol{b}~c + cd + ad = \ol{b}~c + d(a+c)
        \end{equation*}

        En este caso, tendríamos:
        \begin{figure}[H]
            \centering
            \begin{forest}
                [$+$
                    [$\cdot$
                        [$\ol{b}$]
                        [$c$]
                    ]
                    [$\cdot$
                        [$d$]
                        [$+$
                            [$a$]
                            [$c$]
                        ]
                    ]
                ]
            \end{forest}
        \end{figure}

        Tenemos que hay:
        \begin{itemize}
            \item $2$ suma.
            \item $2$ productos.
            \item $8$ ejes.
        \end{itemize}

        El coste es $2+2+8=12$, llegando por tanto a un coste menor. No obstante,
        no se trata de una expresión como suma de productos, por lo que no es válida para el ejercicio. La implementación
        en un circuito lógico sería más compleja en este caso.
    \end{observacion}
\end{ejercicio}


\begin{ejercicio}
    Encuentre una expresión minimal para la función de conmutación $f:B_2^4\to B_2$ dada por:
    \begin{equation*}
        f(a,b,c,d)=\sum m(0,2,3,6,7,8,9,10,13)
    \end{equation*}
    que venga expresada como suma de productos (SOP).

    \begin{description}
        \item[Opción 1.] Usando mapas de Karnaugh:
        \begin{figure}[H]
            \centering
            \begin{karnaugh-map}[4][4][1][$d$][$c$][$b$][$a$]
                \minterms{0,2,3,6,7,8,9,10,13}
                \autoterms[0]
                \implicantcorner
                \implicant{3}{6}
                \implicant{13}{9}
            \end{karnaugh-map}
        \end{figure}
    
        Por tanto, la expresión minimal de $f$ es:
        \begin{equation*}
            f(a,b,c,d) = \ol{a}~c + \ol{b}~\ol{d} + a~\ol{c}~d
        \end{equation*}

        \item[Opción 2.] Usando el algoritmo de Quine-McCluskey:
        
        En primer lugar, generamos los implicantes primos:
        \begin{table}[H]
            \centering
            \begin{tabular}{rcc|rcc|rcc}
                \multicolumn{3}{c}{Columna 1} & \multicolumn{3}{|c|}{Columna 2} & \multicolumn{3}{c}{Columna 3} \\ \hline
                0 & 0000 & \checkmark & \{0,2\} & 00\_0 & \checkmark & \{0,2,8,10\} & \_0\_0 & $\ast$
                \\ \cline{1-3} \cline{7-9}
                2 & 0010 & \checkmark & \{0,8\} & \_000 & \checkmark & \{2,3,6,7\} & 0\_1\_ & $\ast$
                \\  \cline{4-6} \cline{7-9}
                8 & 1000 & \checkmark & \{2,3\} & 001\_ & \checkmark &
                \\ \cline{1-3}
                3 & 0011 & \checkmark & \{2,6\} & 0\_10 & \checkmark &
                \\
                6 & 0110 & \checkmark & \{2,10\} & \_010 & \checkmark &
                \\
                9 & 1001 & \checkmark & \{8,9\} & 100\_ & $\ast$ &
                \\
                10 & 1010 & \checkmark & \{8,10\} & 10\_0 & \checkmark &
                \\ \cline{1-3} \cline{4-6}
                7 & 0111 & \checkmark & \{3,7\} & 0\_11 & \checkmark &
                \\
                13 & 1101 & \checkmark & \{6,7\} & 011\_ & \checkmark &
                \\ \cline{1-3}
                &&& \{9,13\} & 1\_01 & $\ast$ &
                \\ \cline{4-6}
            \end{tabular}
        \end{table}

        Los implicantes primos son, por tanto, los que se han marcado con $\ast$. Reducimos la tabla de implicantes primos:
        \begin{table}[H]
            \centering
            \begin{tabular}{c|ll|ccccccccc}
                && & 0 & \tikzmark{col2Start}2 & \tikzmark{col3Start}3 & 6 & 7 & \tikzmark{col8Start}8 & \tikzmark{col9Start}9 & \tikzmark{col10Start}10 & 13 \\ \hline
                {\color{red}$\ast$} &\{0,2,8,10\} & \_0\_0 & \tikzmark{fil1Start}$\circ$ & $\circ$ & & & & $\circ$ & & $\circ$ &\tikzmark{fil1End}
                \\
                {\color{blue}$\ast$}&\{2,3,6,7\} & 0\_1\_ & \tikzmark{fil2Start}& $\circ$ & $\circ$ & $\circ$ & $\circ$ & & & $\circ$ & \tikzmark{fil2End}
                \\
                &\{8,9\} & 100\_ & & & & & & $\circ$ & $\circ$
                \\
                {\color{teal}$\ast$}& \{9,13\} & 1\_01 & \tikzmark{fil4Start}&\tikzmark{col2End}& & & &\tikzmark{col8End}& \tikzmark{col9End}$\circ$ &\tikzmark{col10End}& $\circ$\tikzmark{fil4End}
            \end{tabular}
            \tikz[remember picture] \draw[overlay, red] ([yshift=.25em]pic cs:fil1Start) -- ([yshift=.25em]pic cs:fil1End);
            \tikz[remember picture] \draw[overlay, red] ([xshift=0.25em]pic cs:col2Start) -- (pic cs:col2End);
            \tikz[remember picture] \draw[overlay, red] ([xshift=0.25em]pic cs:col8Start) -- (pic cs:col8End);
            \tikz[remember picture] \draw[overlay, red] ([xshift=0.5em]pic cs:col10Start) -- (pic cs:col10End);
            \tikz[remember picture] \draw[overlay, teal] ([yshift=.25em]pic cs:fil4Start) -- ([yshift=.25em]pic cs:fil4End);
            \tikz[remember picture] \draw[overlay, teal] ([xshift=0.25em]pic cs:col9Start) -- ([xshift=0.25em]pic cs:col9End);
            \tikz[remember picture] \draw[overlay, blue] ([yshift=.25em]pic cs:fil2Start) -- ([yshift=.25em]pic cs:fil2End);
        \end{table}

        Por tanto, la expresión minimal de $f$ como suma de productos es:
        \begin{equation*}
            f(a,b,c,d) = \ol{a}~c + \ol{b}~\ol{d} + a~\ol{c}~d
        \end{equation*}
    \end{description}
\end{ejercicio}


\begin{ejercicio}
    Encuentre una expresión minimal para la función de conmutación $f:B_2^4\to B_2$ dada por:
    \begin{equation*}
        f(a,b,c,d) = \sum m(0,2,5,6,7,8,10,12,13,14,15)
    \end{equation*}
    que venga expresada como suma de productos (SOP).

    \begin{description}
        \item[Opción 1.] Usando mapas de Karnaugh:
        \begin{figure}[H]
            \centering
            \begin{karnaugh-map}[4][4][1][$d$][$c$][$b$][$a$]
                \minterms{0,2,5,6,7,8,10,12,13,14,15}
                \autoterms[0]
                \implicantcorner
                \implicant{12}{14}
                \implicant{2}{10}
                \implicant{5}{15}
            \end{karnaugh-map}
        \end{figure}
    
        Por tanto, una expresión minimal de $f$ (aunque no es única) es:
        \begin{equation*}
            f(a,b,c,d) = \ol{b}~\ol{d} + b~d + a~b + c~\ol{d}
        \end{equation*}

        \item[Opción 2.] Usando el algoritmo de Quine-McCluskey:
        
        En primer lugar, generamos los implicantes primos:
        \begin{table}[H]
            \centering
            \begin{tabular}{rcc|rcc|rcc}
                \multicolumn{3}{c}{Columna 1} & \multicolumn{3}{|c|}{Columna 2} & \multicolumn{3}{c}{Columna 3} \\ \hline
                0 & 0000 & \checkmark & \{0,2\} & 00\_0 & \checkmark & \{0,2,8,10\} & \_0\_0 & $\ast$
                \\ \cline{1-3} \cline{7-9}
                2 & 0010 & \checkmark & \{0,8\} & \_000 & \checkmark & \{2,6,10,14\} & \_\_10 & $\ast$
                \\ \cline{4-6}
                8 & 1000 & \checkmark & \{2,6\} & 0\_10 & \checkmark & \{8,10,12,14\} & 1\_\_0 & $\ast$
                \\ \cline{1-3} \cline{7-9}
                5 & 0101 & \checkmark & \{2,10\} & \_010 & \checkmark & \{5,7,13,15\} & \_1\_1 & $\ast$
                \\
                6 & 0110 & \checkmark & \{8,10\} & 10\_0 & \checkmark & \{6,7,14,15\} & \_11\_ & $\ast$
                \\
                10 & 1010 & \checkmark & \{8,12\} & 1\_00 & \checkmark & \{12,13,14,15\} & 11\_\_ & $\ast$
                \\ \cline{4-6} \cline{7-9}
                12 & 1100 & \checkmark & \{5,7\} & 01\_1 & \checkmark &
                \\ \cline{1-3}
                7 & 0111 & \checkmark & \{5,13\} & \_101 & \checkmark &
                \\
                13 & 1101 & \checkmark & \{6,7\} & 011\_ & \checkmark &
                \\
                14 & 1110 & \checkmark & \{6,14\} & \_110 & \checkmark &
                \\ \cline{1-3}
                15 & 1111 & \checkmark & \{10,14\} & 1\_10 & \checkmark &
                \\ \cline{1-3}
                &&& \{12,13\} & 110\_ & \checkmark &
                \\
                &&& \{12,14\} & 11\_0 & \checkmark &
                \\ \cline{4-6}
                &&& \{7,15\} & \_111 & \checkmark &
                \\
                &&& \{13,15\} & 11\_1 & \checkmark &
                \\
                &&& \{14,15\} & 111\_ & \checkmark &
                \\ \cline{4-6}
            \end{tabular}
        \end{table}

        Los implicantes primos son, por tanto, los que se han marcado con $\ast$. Reducimos la tabla de implicantes primos,
        donde hemos de destacar que la columna $14$ domina a la del $12$, por lo que se ha descartado también.
        \begin{table}[H]
            \centering
            \begin{tabular}{c|ll|ccccccccccc}
                && & 0 & \tikzmark{Ej3col2Start}2 & \tikzmark{Ej3col3Start}5 & 6 & \tikzmark{Ej3col7Start}7 & \tikzmark{Ej3col8Start}8 & \tikzmark{Ej3col10Start}10 & \tikzmark{Ej3col12Start}12 & \tikzmark{Ej3col13Start}13 & \tikzmark{Ej3col14Start}14 & \tikzmark{Ej3col15Start}15 \\ \hline
                {\color{red}$\ast$} &\{0,2,8,10\} & \_0\_0 & \tikzmark{Ej3fil1Start}$\circ$ & $\circ$ & & & & $\circ$ & $\circ$ & & & & \tikzmark{Ej3fil1End}
                \\
                &\{2,6,10,14\} & \_\_10 & & $\circ$ & & $\circ$ & & & $\circ$ & & & $\circ$ &
                \\
                &\{8,10,12,14\} & 1\_\_0 & & & & & & $\circ$ & $\circ$ & $\circ$ & & $\circ$ &
                \\
                {\color{blue}$\ast$}&\{5,7,13,15\} & \_1\_1 &\tikzmark{Ej3fil4Start}& & $\circ$ & & $\circ$ & & & & $\circ$ & & $\circ$\tikzmark{Ej3fil4End}
                \\
                &\{6,7,14,15\} & \_11\_ & & & & $\circ$ & $\circ$ & & & & & $\circ$ & $\circ$
                \\
                &\{12,13,14,15\} & 11\_\_ & &\tikzmark{Ej3col2End}& & & \tikzmark{Ej3col7End}& \tikzmark{Ej3col8End} & \tikzmark{Ej3col10End}& $\circ$ & \tikzmark{Ej3col13End}$\circ$ & \tikzmark{Ej3col14End}$\circ$ & \tikzmark{Ej3col15End}$\circ$
            \end{tabular}
            \tikz[remember picture] \draw[overlay, red] ([yshift=.25em]pic cs:Ej3fil1Start) -- ([yshift=.25em]pic cs:Ej3fil1End);
            \tikz[remember picture] \draw[overlay, red] ([xshift=0.25em]pic cs:Ej3col2Start) -- (pic cs:Ej3col2End);
            \tikz[remember picture] \draw[overlay, red] ([xshift=0.25em]pic cs:Ej3col8Start) -- (pic cs:Ej3col8End);
            \tikz[remember picture] \draw[overlay, red] ([xshift=0.5em]pic cs:Ej3col10Start) -- (pic cs:Ej3col10End);

            \tikz[remember picture] \draw[overlay, blue] ([yshift=.25em]pic cs:Ej3fil4Start) -- ([yshift=.25em]pic cs:Ej3fil4End);
            \tikz[remember picture] \draw[overlay, blue] ([xshift=0.25em]pic cs:Ej3col7Start) -- (pic cs:Ej3col7End);
            \tikz[remember picture] \draw[overlay, blue] ([xshift=0.5em]pic cs:Ej3col13Start) -- ([xshift=0.25em]pic cs:Ej3col13End);
            \tikz[remember picture] \draw[overlay, blue] ([xshift=0.5em]pic cs:Ej3col15Start) -- ([xshift=0.25em]pic cs:Ej3col15End);

            \tikz[remember picture] \draw[overlay, purple] ([xshift=0.5em]pic cs:Ej3col14Start) -- ([xshift=0.25em]pic cs:Ej3col14End);
        \end{table}

        Tras haber llegado a este paso, hemos detectado ya dos implicantes primos esenciales. No obstante, para cubrir el minterm $6$ tenemos dos opciones,
        y para cubrir el minterm $12$ también tenemos dos opciones. Por tanto, las 4 expresiones minimales dadas en forma SOP son:
        \begin{align*}
            f(a,b,c,d) &= \ol{b}~\ol{d} + b~d + c~\ol{d} + a~\ol{d}\\
            f(a,b,c,d) &= \ol{b}~\ol{d} + b~d + c~\ol{d} + a~b\\
            f(a,b,c,d) &= \ol{b}~\ol{d} + b~d + b~c + a~\ol{d}\\
            f(a,b,c,d) &= \ol{b}~\ol{d} + b~d + b~c + a~b
        \end{align*}
    \end{description}
\end{ejercicio}











































































\begin{ejercicio}
    Compruebe si la expresión booleana $\varphi=xy+yz+zx$ es autodual.

    Tenemos que:
    \begin{align*}
        \varphi^d &= (x+y)(y+z)(z+x)\\
        &= (xy + xz + y + yz)(z + x)\\
        &= (xy + xz + y(1+z))(z + x)\\
        &= (xy + xz + y)(z + x)\\
        &= (y(x+1) + xz)(z + x)\\
        &= (y + xz)(z + x)\\
        &= yz + yx + xz + xz\\
        &= xy + yz + zx = \varphi.
    \end{align*}

    Por tanto, la expresión $\varphi$ es autodual.
\end{ejercicio}
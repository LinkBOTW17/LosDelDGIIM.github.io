\section{Inducción}

\begin{ejercicio}\label{ej:1.1}
    Para todo $n\in \bb{N}$, demostrar que es cierta la siguiente igualdad:
    \begin{equation*}
        \sum_{i=0}^n i = \dfrac{n(n+1)}{2}
    \end{equation*}
    \begin{proof}
        La demostración es por inducción según el principio de inducción matemática y predicado $P(n)$ del contenido literal (tenor):
        \begin{equation*}
            \sum_{i=0}^n i = \dfrac{n(n+1)}{2}
        \end{equation*}

        \begin{itemize}
            \item \ul{Para $n=0$}:
            \begin{equation*}
                \sum_{i=0}^0 i = 0 = \frac{0}{2} = \frac{0\cdot 1}{2}
                 = \dfrac{0(0+1)}{2}
            \end{equation*}
            Por tanto, se tiene $P(0)$.

            \item Como hipótesis de inducción supondremos que $n\in \bb{N}$ y que $P(n)$ ees cierto, es decir, que:
            \begin{equation*}
                \sum_{i=0}^n i = \dfrac{n(n+1)}{2}
            \end{equation*}
            y en el paso de inducción demostraremos que $P(n+1)$ es cierto.
            \begin{align*}
                \sum_{i=0}^{n+1} i &= \sum_{i=0}^n i + (n+1)
                \AstIg \dfrac{n(n+1)}{2} + (n+1)
                =\\&= \dfrac{n(n+1) + 2(n+1)}{2}
                = \dfrac{(n+1)(n+2)}{2}
                = \dfrac{(n+1)((n+1)+1)}{2}
            \end{align*}
            donde en $(\ast)$ he utilizado la hipótesis de inducción. Por tanto, $P(n+1)$ es cierto.
        \end{itemize}

        Por el principio de inducción matemática, sabemos que para todo $n\in \bb{N}$, $P(n)$ es cierto, por lo que se tiene lo que se pedía.
    \end{proof}
\end{ejercicio}

\begin{ejercicio}\label{ej:1.2}
    Demustre que para todo número natural $n$:
    $$\left(\sum_{k=0}^n k\right)^2 = \left(\sum_{k=0}^{n-1}\right)^2 + n^3$$
\begin{proof}~
    En este caso, no se usa la demostración mediante inducción, sino la demostración por casos:
    \begin{itemize}
        \item \ul{$n = 0$:}
            $$\left(\sum_{n=0}^0 k\right)^2 = 0^2 = 0 = 0+0 = \left(\sum_{k=0}^{-1}k\right)^2 + 0^3$$
        \item \ul{$n = 1$:}
            $$\left(\sum_{k=0}^1 k\right)^2 = 0+1 = \left(\sum_{k=0}^0 k\right)^2 + 1^3 = \left(\sum_{k=0}^{n-1} k\right)^2 + n^3$$
        \item \ul{$n > 1$:}
        \begin{align*}
            \left(\sum_{k=0}^n k\right)^2 &= \left[\left(\sum_{k=0}^{n-1} k\right) + n\right]^2 = \left(\sum_{k=0}^{n-1}k\right)^2 + n^2 + 2\left(\sum_{k=0}^{n-1} k\right)n \AstIg\\
            &\AstIg \left(\sum_{k=0}^{n-1}k\right)^2 + n^2 + \cancel{2} \cdot \dfrac{(n-1)n}{\cancel{2}}\cdot n = \left(\sum_{k=0}^{n-1}k\right)^2 + n^2 + (n-1)n^2 =\\
            &= \left(\sum_{k=0}^{n-1}k\right)^2 + n^2 (1+n-1) = \left(\sum_{k=0}^{n-1}k\right)^2 + n^3
        \end{align*}
        donde en $(\ast)$ he utilizado el Ejercicio \ref{ej:1.1}.
    \end{itemize}
\end{proof}
\end{ejercicio}

\begin{ejercicio}[Teorema de Nicomachus]
    Demuestre que para todo número natural $n$ vale la siguiente igualdad:
    $$\sum_{k=0}^n k^3 = \left(\sum_{k=0}^n k\right)^2$$
\begin{proof}
    La demostración es por inducción según el principio de inducción matemática y predicado $P(n)$ del contenido literal (tenor):
    $$\sum_{k=0}^n k^3 = \left(\sum_{k=0}^n k\right)^2$$
    \begin{itemize}
        \item \ul{En el caso base $n = 0$}:
        $$\sum_{k=0}^0 k^3 = 0^3 = 0 = 0^2 = \left(\sum_{k=0}^0 k\right)^2$$
        Y por tanto, $P(0)$ es correcto.
        
        \item Como hipótesis de inducción, supondremos que $n$ es un número natural y que $P(n)$ es cierto; es decir, 
        $$\sum_{k=0}^{n} k^3 = \left(\sum_{k=0}^n k\right)^2$$
        En el paso de inducción, demostraremos que $P(n+1)$ se cumple.
        $$\sum_{k=0}^{n+1} k^3 = \left(\sum_{k=0}^n k^3\right) + (n+1)^3 \AstIg \left(\sum_{k=0}^n k\right)^2 + (n+1)^3 \stackrel{(\ast\ast)}{=} \left(\sum_{k=0}^{n+1} k\right)^2  $$

        donde en $(\ast)$ he utilizado la hipótesis de inducción y en $(\ast\ast)$ he utilizado el Ejercicio \ref{ej:1.2}. Por tanto, $P(n+1)$ es cierto.
        Luego $P(n+1)$ es cierto.
    \end{itemize}
    Por el principio de inducción matemática para todo número natural $n$, $P(n)$ se tiene, como se pedía.
\end{proof}
\end{ejercicio}

\begin{observacion}
    El segundo principio de inducción matemática se utiliza cuando, en vez de usar como hipótesis una verdad sobre $n$, se quiere usar una verdad sobre $n-k$ con $k >1$, cuando estemos demostrado que el predicado vale para $n+1$.
\end{observacion}

Veamos un ejemplo de uso del segundo principio de inducción matemática.
\begin{ejercicio}
    Todo número natural mayor que 1 tiene al menos un factor primo.
\begin{proof}
    El razonamiento es por el segundo principio de inducción según el predicado (o fórmula) $P(n)$ del tenor:
    $$\text{``}n \text{ tiene un factor primo}\text{''}$$
    donde $n \in \omega \setminus \{ 0, 1\}$ (tenemos que $i_0 = 2$).\\

    \noindent
    Como hipótesis de inducción, supongamos que $n$ es un número natural superior a 1 y que $P(k)$ vale para todo $1<k < n$.\\

    \noindent
    En el paso de inducción distinguimos dos casos:
    
    \begin{itemize}
        \item \ul{$n$ es primo:} 

            En este caso, $n$ es un factor primo de $n$ (note que 2 es un ejemplo de los números en este caso).
        \item \ul{$n$ no es primo:}
            Si $n$ no es primo, existen números naturales $u$ y $v$ tales que $n = uv$ y $1<u,v$. Claro está entonces, que $1<u,v<n$.\newline
            Por la hipótesis de inducción, $P(u)$ vale, luego $u$ tendrá al menos un factor primo, al que podemos llamar $p$. Así pues, $p \mid u$ y por tanto:
            $$p \mid n$$
            Luego $P(n)$ vale y por el segundo principio de inducción, para todo número natural $n$ vale $P(n)$.
    \end{itemize}
\end{proof}

Notemos que siempre tiene que ocurrir que el caso base ($i_0$) esté incluido en uno de los casos, por eso lo hemos destacado anteriormente con $i_0 = 2$.
\end{ejercicio}


\begin{ejercicio}[Multiplicación por el Método del Campesino Ruso]
Sea $p$ la función dada por:
    \begin{align*}
        p(a,0) &= 0,\\
        p(a,b) &= \left\{ \begin{array}{ll}
            p\left(2a,\dfrac{b}{2}\right) & \text{si } b \text{ es par, }\\
                                        & \\
            p\left(2a,\dfrac{b-1}{2}\right)+a & \text{si } b \text{ es impar, }
    \end{array}\right.
    \end{align*}
    Demuestre por inducción que para cualesquiera números naturales $a$ y $b$, $p(a,b) = ab$.
    \begin{comment}
        Notemos que esta función pasa a binario $b$ y almacena en $a$ tantos ''$2$'' como el popcount de $b$.
    \end{comment}
    \begin{proof}
       La demostración es por inducción según el segundo principio de inducción y el predicado del tenor: 
        \begin{center}
            ``Para todo número natural $m$, $p(m,n) = mn$.''
        \end{center}
        Supongamos como hipótesis de inducción que $k$ es un número natural y que $P(k)$ vale para todo $0 \leq k < n$. Distinguimos los siguientes casos:
        \begin{itemize}
            \item \ul{$n=0$, (sea cual sea $m$)}:
                $$p(m,0) = 0 = m \cdot 0$$
                Luego $P(0)$ vale.
            \item \ul{En el paso de inducción, demostraremos que $P(n)$ vale}:
            \item 
                Suponemos aquí que $n>0$. Caben dos casos:
                \begin{enumerate}
                    \item \ul{$n\equiv 0\mod 2$} (es par):
                        $$p(m,n) = p\left(2m, \dfrac{n}{2}\right) \AstIg 2m \cdot \dfrac{n}{2}=mn$$
                        Donde en $(\ast)$ he usado la hipótesis de inducción, ya que $\dfrac{n}{2} < n$.
                    \item \ul{$n\equiv 1\mod 2$} (es impar):
                        \begin{align*}
                            p(m,n) &= p\left(2m, \dfrac{n-1}{2}\right)+m \AstIg \left( \cancel{2}m \cdot \dfrac{n-1}{\cancel{2}}\right) +m =\\
                                    &= m(n-1)+m = mn-m+m = mn
                        \end{align*}
                        donde en $(\ast)$ he usado la hipótesis de inducción, ya que $\dfrac{n-1}{2} < n$.
                \end{enumerate}
        \end{itemize}
        Por el segundo principio de inducción, para todo número natural $n$, vale $P(n)$.
    \end{proof}
\end{ejercicio}



\begin{ejercicio}\label{ej:2dividea}
    Para todo número natural $n$ no nulo, demostrar que:
    $$2\mid \left(5^n + 3^{n-1}\right)$$

    \begin{proof}
        La demostración es por inducción según el principio de inducción matemática y el predicado $P(n)$ del tenor:
        $$\text{``}2\mid \left(5^n + 3^{n-1}\right)\text{''}$$

        \begin{itemize}
            \item En el caso base, $n=1$:
                $$5^1 + 3^{1-1} = 5+1 = 6$$
                Como $2\mid 6$, $P(1)$ es cierto.

            \item Como hipótesis de inducción, supondremos que $n$ es un número natural no nulo y que $P(n)$ es cierto, es decir, que:
                $$2\mid \left(5^n + 3^{n-1}\right)$$
                En el paso de inducción, demostraremos que $P(n+1)$ es cierto.

                Para demostrarlo, antes tenemos en cuenta que, dados $a,b,n\in \bb{Z}$, se tiene que:
                \begin{equation*}
                    \left.\begin{array}{l}
                        n\mid (a-b) \\
                        n\mid b
                    \end{array} \right\} \Longrightarrow n\mid a
                \end{equation*}
                Esto se debe a que:
                \begin{equation*}
                    n\cdot k = a-b = a-nk_1 \Longrightarrow a = n(k+k_1) = n\cdot k_2 \Longrightarrow n\mid a
                \end{equation*}

                Por tanto, haciendo uso de esto, tenemos que:
                \begin{align*}
                    \left(5^{n+1} + 3^{(n+1)-1}\right) - \left(5^n + 3^{n-1}\right) &=
                     4 \cdot 5^n + 3^{n-1}\cdot 2 =\\
                    &= 2 \cdot \left(2\cdot 5^n + 3^{n-1}\right)
                \end{align*}
                Como $2\mid \left(5^{n+1} + 3^{(n+1)-1} - \left(5^n + 3^{n-1}\right)\right)$ y, por hipótesis de inducción, se tiene que  $2\mid 5^n + 3^{n-1}$, hemos visto que $2\mid 5^{n+1} + 3^{(n+1)-1}$.
                Por tanto, $P(n+1)$ es cierto. 
                
            \end{itemize}
            Por tanto, por el principio de inducción matemática, para todo número natural $n$ no nulo, se tiene que $2\mid \left(5^n + 3^{n-1}\right)$.
    \end{proof}
\end{ejercicio}


\begin{ejercicio}
    Para todo número natural $n$ no nulo, demostrar que:
    $$8\mid \left(5^n + 2\cdot 3^{n-1}+1\right)$$

    \begin{proof}
        La demostración es por inducción según el principio de inducción matemática y el predicado $P(n)$ del tenor:
        $$\text{``}8\mid \left(5^n + 2\cdot 3^{n-1}+1\right)\text{''}$$

        \begin{itemize}
            \item En el caso base, $n=1$:
                $$5^1 + 2\cdot 3^{1-1}+1 = 5+2+1 = 8$$
                Como $8\mid 8$, $P(1)$ es cierto.

            \item Como hipótesis de inducción, supondremos que $n$ es un número natural no nulo y que $P(n)$ es cierto, es decir, que:
                $$8\mid \left(5^n + 2\cdot 3^{n-1}+1\right)$$
                En el paso de inducción, demostraremos que $P(n+1)$ es cierto.

                Para demostrarlo, antes tenemos en cuenta que, dados $a,b,n\in \bb{Z}$, se tiene que:
                \begin{equation*}
                    \left.\begin{array}{l}
                        n\mid (a-b) \\
                        n\mid b
                    \end{array} \right\} \Longrightarrow n\mid a
                \end{equation*}
                Esto se debe a que:
                \begin{equation*}
                    n\cdot k = a-b = a-nk_1 \Longrightarrow a = n(k+k_1) = n\cdot k_2 \Longrightarrow n\mid a
                \end{equation*}

                Por tanto, haciendo uso de esto, tenemos que:
                \begin{align*}
                    \left(5^{n+1} + 2\cdot 3^{(n+1)-1}+1\right) &- \left(5^n + 2\cdot 3^{n-1}+1\right) =\\
                    &= 5^n\cdot 5 + 2\cdot 3^{n}+\cancel{1} - 5^n - 2\cdot 3^{n-1} - \cancel{1} =\\
                    &= 5^n(5-1)+2\cdot 3^{n-1}(3-1) =\\
                    &= 4\cdot 5^n + 2\cdot 3^{n-1}\cdot 2 =\\
                    &= 4\left(5^n + 3^{n-1}\right)
                    \AstIg 4\cdot 2k = 8k
                \end{align*}
                donde en $(\ast)$ he usado que el Ejercicio \ref{ej:2dividea}.
                Por tanto, como hemos visto que $8\mid \left[\left(5^{n+1} + 2\cdot 3^{(n+1)-1}+1\right) - \left(5^n + 2\cdot 3^{n-1}+1\right)\right]$ y, por hipótesis de inducción, $8\mid 5^n + 2\cdot 3^{n-1}+1$, se tiene que $8\mid 5^{n+1} + 2\cdot 3^{(n+1)-1}+1$.
                Por tanto, $P(n+1)$ es cierto.
            \end{itemize}
            Por tanto, por el principio de inducción matemática, para todo número natural $n$ no nulo, se tiene que $8\mid \left(5^n + 2\cdot 3^{n-1}+1\right)$, como se pedía.
    \end{proof}
\end{ejercicio}

\begin{ejercicio}
    Demouestre que para todo número natural $n$ no nulo, se tiene que:
    \begin{equation*}
        \prod_{k=1}^n \frac{2k-1}{2k} \leq \frac{1}{\sqrt{n+1}}
    \end{equation*}
    \begin{proof}
        La demostración es por inducción según el principio de inducción matemática y el predicado $P(n)$ del tenor:
        \begin{equation*}
            \text{``}\prod_{k=1}^n \frac{2k-1}{2k} \leq \frac{1}{\sqrt{n+1}}\text{''}
        \end{equation*}
        \begin{itemize}
            \item En el caso base, $n=1$:
                \begin{equation*}
                    \prod_{k=1}^1 \frac{2k-1}{2k} = \frac{1}{2} \leq \frac{1}{\sqrt{2}} \Longleftrightarrow 2 \geq \sqrt{2}
                \end{equation*}
                Como $2 \geq \sqrt{2}$, $P(1)$ es cierto.

            \item Como hipótesis de inducción, supondremos que $n$ es un número natural no nulo y que $P(n)$ es cierto, es decir, que:
                \begin{equation*}
                    \prod_{k=1}^n \frac{2k-1}{2k} \leq \frac{1}{\sqrt{n+1}}
                \end{equation*}
                En el paso de inducción, demostraremos que $P(n+1)$ es cierto. Tenemos que:
                \begin{align*}
                    \prod_{k=1}^{n+1} \frac{2k-1}{2k} &= \left(\prod_{k=1}^n \frac{2k-1}{2k}\right) \cdot \frac{2(n+1)-1}{2(n+1)} \stackrel{(\ast)}{\leq} \frac{1}{\sqrt{n+1}} \cdot \frac{2(n+1)-1}{2(n+1)} =\\
                    &= \frac{1}{\sqrt{n+1}} \cdot \frac{2n+1}{2(n+1)}
                \end{align*}
                donde en $(\ast)$ hemos utilizado la hipótesis de inducción. Veamos ahora que $P(n+1)$ se tiene:
                \begin{align*}
                    \frac{1}{\sqrt{n+1}} &\cdot \frac{2n+1}{2(n+1)}\leq \frac{1}{\sqrt{n+2}} \Longleftrightarrow \frac{2n+1}{2(n+1)}\leq \frac{\sqrt{n+1}}{\sqrt{n+2}} \Longleftrightarrow\\
                    &\Longleftrightarrow \frac{2n+1}{2n+2}\leq \sqrt{\frac{n+1}{n+2}}
                    \Longleftrightarrow \frac{(2n+1)^2}{(2n+2)^2}\leq \frac{n+1}{n+2} \Longleftrightarrow \\
                    &\Longleftrightarrow (2n+1)^2(n+2)\leq (2n+2)^2(n+1) \Longleftrightarrow\\
                    &\Longleftrightarrow (n+2)(4n^2+4n+1)\leq (n+1)(4n^2+8n+4) \Longleftrightarrow\\
                    &\Longleftrightarrow \cancel{4n^3}+\cancel{4n^2}+n+\cancel{8n^2}+\cancel{8n}+2\leq \cancel{4n^3}+\cancel{8n^2}+4n+\cancel{4n^2}+\cancel{8n}+4 \Longleftrightarrow\\
                    &\Longleftrightarrow 2+n \leq 4+4n \Longleftrightarrow 0 \leq 2+3n
                \end{align*}
                Como $0 \leq 2+3n$, $P(n+1)$ es cierto.
        \end{itemize}        
        Por tanto, por el principio de inducción matemática, para todo número natural $n$, $P(n)$ es cierto, como se pedía.
    \end{proof}
\end{ejercicio}


\begin{ejercicio}\label{ej:SigCuadradoMenorCubo}
    Demuestra que, para todo número natural $n$ mayor que 2, se tiene que:
    $$(n+1)^2 < n^3$$
    
    \begin{proof}
        La demostración es por inducción según el principio de inducción matemática y el predicado $P(n)$ del tenor:
        \begin{equation*}
            \text{``}(n+1)^2 < n^3\text{''}
        \end{equation*}
        \begin{itemize}
            \item En el caso base, $n=3$:
                \begin{equation*}
                    (3+1)^2 = 16 < 27 = 3^3
                \end{equation*}
                Como $16 < 27$, $P(3)$ es cierto.

            \item Como hipótesis de inducción, supondremos que $n$ es un número natural mayor que 2 y que $P(n)$ es cierto, es decir, que:
                \begin{equation*}
                    (n+1)^2 < n^3
                \end{equation*}
                En el paso de inducción, demostraremos que $P(n+1)$ es cierto. Tenemos que:
                \begin{align*}
                    (n+1+1)^2 &= (n+1)^2 + 2(n+1) + 1 \stackrel{(\ast)}{\leq} \\ &\stackrel{(\ast)}{\leq}
                    n^3 + 2n + 1 \leq n^3 +3n^2 + 3n + 1 = (n+1)^3
                \end{align*}
                donde en $(\ast)$ he utilizado la hipótesis de inducción. Por tanto, $P(n+1)$ es cierto.
        \end{itemize}
        Por tanto, por el principio de inducción matemática, para todo número natural $n$ mayor que 2, se tiene que $(n+1)^2 < n^3$, como se pedía.
    \end{proof}
\end{ejercicio}

\begin{ejercicio}
    Demuestre que para todo número natural $n$ superior a $5$, se tiene que $n^3<n!$.

    \begin{proof}
        La demostración es por inducción según el principio de inducción matemática y el predicado $P(n)$ del tenor:
    \begin{equation*}
        \text{``}n^3<n!\text{''}
    \end{equation*}

    \begin{itemize}
        \item En el caso base, $n=6$:
            \begin{equation*}
                6^3 \leq 6! \Longleftrightarrow 6^2\leq 5!=2\cdot 3\cdot 4\cdot 5 \Longleftrightarrow 6 \leq 4\cdot 5
            \end{equation*}
            Como $6 < 20$, $P(6)$ es cierto.

        \item Como hipótesis de inducción, supondremos que $n$ es un número natural superior a 5 y que $P(n)$ es cierto, es decir, que:
            \begin{equation*}
                n^3<n!
            \end{equation*}
            En el paso de inducción, demostraremos que $P(n+1)$ es cierto. Tenemos que:
            \begin{equation*}
                (n+1)^3 = (n+1)^2(n+1) \stackrel{(\ast)}{<{}} n^3(n+1) \stackrel{(\ast\ast)}{<{}} n!(n+1) = (n+1)!
            \end{equation*}
            donde en $(\ast)$ he utilizado el Ejercicio \ref{ej:SigCuadradoMenorCubo} y en $(\ast\ast)$ he empleado la hipótesis de inducción.
            Por tanto, $P(n+1)$ es cierto.
    \end{itemize}
    Por tanto, por el principio de inducción matemática, para todo número natural $n$ superior a 5, se tiene que $n^3<n!$, como se pedía.
    \end{proof}
\end{ejercicio}

\begin{ejercicio}
    Demuestre que, para todo número natural $n$, $8^n - 3^n$ es múltiplo de $5$.

    \begin{proof}
        La demostración es por inducción según el principio de inducción matemática y el predicado $P(n)$ del tenor:
        \begin{equation*}
            \text{``}5 \mid 8^n - 3^n\text{''}
        \end{equation*}

        \begin{itemize}
            \item En el caso base, $n=0$:
                \begin{equation*}
                    8^0 - 3^0 = 1-1 = 0
                \end{equation*}
                Como $5\mid 0$, $P(0)$ es cierto.

            \item Como hipótesis de inducción, supondremos que $n$ es un número natural y que $P(n)$ es cierto, es decir, que:
                \begin{equation*}
                    5\mid 8^n - 3^n
                \end{equation*}
                En el paso de inducción, demostraremos que $P(n+1)$ es cierto. Tenemos que:
                
                \begin{align*}
                    8^{n+1}-3^{n+1} - 8^n + 3^n &= 8^n\cdot (8-1) -3^n\cdot (3-1) = 7\cdot 8^n - 2\cdot 3^n =\\
                    &= 5\cdot 8^n + 2\cdot 8^n - 2\cdot 3^n = 5\cdot 8^n + 2\cdot (8^n - 3^n) \AstIg\\
                    &\AstIg 5\cdot 8^n + 2\cdot 5k = 5(8^n + 2k)
                \end{align*}
                donde en $(\ast)$ he utilizado la hipótesis de inducción.
                Como por hipótesis de inducción se tiene también que $5\mid 8^n - 3^n$, se tiene que $5\mid 8^{n+1} - 3^{n+1}$.
                Por tanto, $P(n+1)$ es cierto.
        \end{itemize}
        Por tanto, por el principio de inducción matemática, para todo número natural $n$, $8^n - 3^n$ es múltiplo de $5$, como se pedía.
    \end{proof}
\end{ejercicio}


Veamos ahora un ejemplo de uso del principio del buen orden de los números naturales.
\begin{ejercicio}
    Demuestra que, para cualesquiera números naturales $a$ y $b$, existe un mínimo
    común múltiplo de ellos.
    \begin{proof}
        Distinguimos casos según el valor de $a$ y $b$:
        \begin{itemize}
            \item \ul{$a = 0$ o $b = 0$:}
            
            Tenemos que $0$ es un múltiplo común de $a$ y de $b$. Además, es el mínimo múltiplo común de $a$ y $b$, ya que cualquier otro múltiplo común de $a$ y de $b$ es mayor que $0$.

            \item \ul{$a,b > 0$:}
            
            Sea $M_{a,b}$ el conjunto de los múltiplos comunes de $a$ y de $b$.
            Es claro que $0\in M_{a,b}$, ya que $0$ es múltiplo de cualquier número natural. Además, tenemos que
            $ab\in M_{a,b}$, ya que $ab$ es múltiplo de $a$ y de $b$. Como $a,b>0$, se tiene que $ab>0$.
            Consideramos ahora el siguiente conjunto:
            \begin{equation*}
                v_{a,b} = M_{a,b}\setminus \{0\} \subsetneq M_{a,b} \subset \bb{N}
            \end{equation*}
            Como hemos visto, $v_{a,b}$ es un subconjunto no vacío de $\bb{N}$, por lo que, por el principio del buen orden de los números naturales, $v_{a,b}$ tiene un mínimo, al que llamaremos $m_{a,b}$.
            Así pues, $m_{a,b}$ es el mínimo común múltiplo de $a$ y de $b$.
        \end{itemize}
    \end{proof}
\end{ejercicio}


\begin{ejercicio}
    Estime un valor de $n$ para el que
    $$100^n < n!$$
\end{ejercicio}

\begin{ejercicio}[Ejemplo de principio del buen orden]
Sea $n$ un número natural y sea $S$ un conjunto de números naturales menores que $n$. Demuestre que $S$ es vacío o tiene máximo.
\begin{proof}
Sea $S$ un conjunto en las condiciones del enunciado y supongamos que $S$ es no vacío. Pueden darse dos casos
\begin{enumerate}
    \item $S = \{0\}$; en este caso, $S$ tiene máximo y es $0$.
    \item $S\setminus\{0\} \neq \emptyset$ (o que $S$ tiene elementos distintos de $0$); en este caso, sea
    \begin{equation*}
        M(S) = \{ m \in \omega \mid \text{\ para todo\ } x \in S, x \leq m\}
    \end{equation*}
    Se cumple lo siguiente:
    \begin{itemize}
        \item $n \in M(S)$
        \item $M(S) \neq \emptyset$
        \item Por el principio del buen orden, existe el mínimo de $M(S)$, denotado como $m_0$.
        \item $m_0 \neq 0$, porque $m_0$ es un mayorante de $S\setminus \{0\}$ que es no vacío.
        \item Si para todo $x \in S$ se cumpliera que
        \begin{equation*}
            x < m_0
        \end{equation*}
        entonces, para todo $x \in S$ se cumpliría que 
        \begin{equation*}
            x \leq m_0 -1
        \end{equation*}
        y por tanto $m_0 - 1 \in M(S)$ y además
        \begin{equation*}
            0 \leq m_0 -1 < m_0
        \end{equation*}
        de donde $m_0$ no sería el mínimo de $M(S)$, en contra de lo supuesto.

        \item Existe $x_0 \in S$ tal que
        \begin{equation*}
            m_0 \leq x_0
        \end{equation*}
        \item Como $m_0$ es un un mayorante de $S$ se cumplirá que 
        \begin{equation*}
            x_0 \leq m_0
        \end{equation*}
        \item Por las dos desigualdados anteriores, $m_0 = x_0$
        \item $m_0 \in S \cap M(S)$
        \item El mínimo de $M(S)$ es un elemento de $S$, luego es el máximo de $S$.
    \end{itemize}
\end{enumerate}
\end{proof}
\end{ejercicio}

\begin{ejercicio}
Demuestre mediante inducción que para todo número natural $n$ tal que $2 \leq n$ se cumple:
\begin{equation*}
    \sqrt{n} < \sum_{k=1}^n \dfrac{1}{\sqrt{k}}
\end{equation*}
\begin{proof}
    A continuación, daremos las ideas de la demostración para que el lector termine el desarrollo de la misma.

    El razonamiento es por el principio de inducción matemática 
    \begin{equation*}% // TODO: Arreglar comillas
        P(n) \text{\ es\ } ``2\leq n \text{\ y \ } \sqrt{n} < \sum_{k=1}^n \dfrac{1}{\sqrt{k}} ''
    \end{equation*}
    \begin{itemize}
        \item Caso base: $n = 2$
        \begin{align*}
            1 < 2 &\Rightarrow 1 = \sqrt{1} < \sqrt{2}\\
            &\Rightarrow 2 = 1+1 < \sqrt{2} +1\\
            &\Rightarrow (\sqrt{2})^2 < \sqrt{2} + 1\\
            &\Rightarrow \sqrt{2} < \dfrac{\sqrt{2}+1}{\sqrt{2}}
        \end{align*}
        \begin{align*}
            \dfrac{\sqrt{2}+1}{\sqrt{2}}&= \dfrac{\sqrt{2}}{\sqrt{2}}+\dfrac{1}{\sqrt{2}}\\
            &= 1 + \dfrac{1}{\sqrt{2}}\\
            &= \dfrac{1}{\sqrt{1}} + \dfrac{1}{\sqrt{2}}\\
            &= \sum_{k=1}^2 \dfrac{1}{\sqrt{k}}
        \end{align*}
        Luego $P(2)$ es cierto.

        \item Hipótesis de inducción: supongamos que $n \in \omega$ y que vale $P(n)$, es decir
        \begin{equation*}
            2\leq n \text{\ y \ } \sqrt{n} < \sum_{k=1}^n \dfrac{1}{\sqrt{k}}
        \end{equation*}
        \item paso de inducción:
        \begin{align*}
            n = (\sqrt{n})^2 = \sqrt{n}\sqrt{n} < \sqrt{n}\sqrt{n+1}
        \end{align*}
        Entonces:
        \begin{align*}
            n+1 &< \sqrt{n}\sqrt{n+1} +1\\
            (\sqrt{n+1})^2 &< \sqrt{n}\sqrt{n+1}+1\\
            \sqrt{n+1} &< \dfrac{\sqrt{n}\sqrt{n+1}}{\sqrt{n+1}}+\dfrac{1}{\sqrt{n+1}}\\
                           &=\sqrt{n}+\dfrac{1}{\sqrt{n+1}}\\
                           &<\left( \sum_{k=1}^n \dfrac{1}{\sqrt{k}}\right)+\dfrac{1}{\sqrt{n+1}}\\
                           &=\sum_{k=1}^{n+1} \dfrac{1}{\sqrt{n+1}}
        \end{align*}
        Luego $P(n+1)$ vale.

        Por el principio de inducción matemática, para todo $n \in \omega$ con $n \leq 2$, $P(n)$ vale.
        

    \end{itemize}
    
\end{proof}
\end{ejercicio}

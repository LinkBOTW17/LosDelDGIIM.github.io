\section{Formas bilineales simétricas y formas cuadráticas.}
\label{sec:EjerciciosTema2}
\begin{ejercicio}\label{Ej1}
    Sea $g$ una métrica en un espacio vectorial $V^2(\bb{R})$ que respecto de la base $\mathcal{B} = \{e_1, e_2\}$ viene dada por la matriz $M(g_i, \mathcal{B}) = A_i$. Clasificar la métrica en cada caso ($=$ calcular nulidad e índice).
    \begin{equation*}
        A_1=\left(\begin{array}{cc}
            0 & 1 \\
            1 & 0
        \end{array} \right) \qquad
        A_2=\left(\begin{array}{cc}
            0 & 1 \\
            1 & 1
        \end{array} \right) \qquad
        A_3=\left(\begin{array}{cc}
            1 & 1 \\
            1 & 1
        \end{array} \right)
    \end{equation*}
    \begin{equation*}
        A_4=\left(\begin{array}{cc}
            1 & 1 \\
            1 & -1
        \end{array} \right) \qquad
        A_5=\left(\begin{array}{cc}
            1 & 2 \\
            2 & 3
        \end{array} \right) \qquad
        A_6=\left(\begin{array}{cc}
            1 & -2 \\
            -2 & 3
        \end{array} \right) \qquad
        A_7=\left(\begin{array}{cc}
            1 & -2 \\
            -2 & -3
        \end{array} \right)
    \end{equation*}

    \begin{enumerate}
        \item $A_1=\left(\begin{array}{cc}
            0 & 1 \\
            1 & 0
        \end{array} \right)$
        
        Tenemos que $rg(A_1) = 2 \Longrightarrow Nul(g_1)=0$. Por tanto, por el Teorema de Sylvester,
        \begin{equation*}
            A_1 \sim_c \left(\begin{array}{cc}
                1 &  \\
                & -1
            \end{array} \right)
            \qquad \text{o} \qquad
            A_1 \sim_c \left(\begin{array}{cc}
                1 &  \\
                & 1
            \end{array} \right)
            \qquad \text{o} \qquad
            A_1 \sim_c \left(\begin{array}{cc}
                -1 &  \\
                & -1
            \end{array} \right)
        \end{equation*}

        Tenemos también que $|A_1| = -1$. Como el signo del determinante es un invariante, concluimos que:
        \begin{equation*}
            A_1 \sim_c \left(\begin{array}{cc}
                1 &  \\
                & -1
            \end{array} \right)
        \end{equation*}
        
        Por tanto, concluimos que $Nul(g_1)=0$ y $Ind(g_1)=1$.

        \item $A_2=\left(\begin{array}{cc}
            0 & 1 \\
            1 & 1
        \end{array} \right)$
        
        Tenemos que $rg(A_2) = 2 \Longrightarrow Nul(g_2)=0$. Por tanto, por el Teorema de Sylvester,
        \begin{equation*}
            A_2 \sim_c \left(\begin{array}{cc}
                1 &  \\
                & -1
            \end{array} \right)
            \qquad \text{o} \qquad
            A_2 \sim_c \left(\begin{array}{cc}
                1 &  \\
                & 1
            \end{array} \right)
            \qquad \text{o} \qquad
            A_2 \sim_c \left(\begin{array}{cc}
                -1 &  \\
                & -1
            \end{array} \right)
        \end{equation*}

        Tenemos también que $|A_2| = -1$. Como el signo del determinante es un invariante, concluimos que:
        \begin{equation*}
            A_2 \sim_c \left(\begin{array}{cc}
                1 &  \\
                & -1
            \end{array} \right)
        \end{equation*}
        
        Por tanto, concluimos que $Nul(g_2)=0$ y $Ind(g_2)=1$.

        \item $A_3=\left(\begin{array}{cc}
            1 & 1 \\
            1 & 1
        \end{array} \right)$
        
        Tenemos que $rg(A_3) = 1 \Longrightarrow Nul(g_3)=1$. Por tanto, por el Teorema de Sylvester,
        \begin{equation*}
            A_3 \sim_c \left(\begin{array}{cc}
                1 &  \\
                & 0
            \end{array} \right)
            \qquad \text{o} \qquad
            A_3 \sim_c \left(\begin{array}{cc}
                -1 &  \\
                & 0
            \end{array} \right)
        \end{equation*}

        Considerando $U=\cc{L}\{e_1\}$, tenemos que $g_{\left|U \right.}$ es definida positiva. Por tanto, como $\dim U = 1$, tenemos que $k\geq 1$, siendo $k$ el número de $1$ de la matriz asociada a la base de Sylvester.
        Por tanto,
        \begin{equation*}
            A_3 \sim_c \left(\begin{array}{cc}
                1 &  \\
                & 0
            \end{array} \right)
        \end{equation*}
        
        Por tanto, concluimos que $Nul(g_3)=1$ y $Ind(g_3)=0$.

        \item $A_4=\left(\begin{array}{cc}
            1 & 1 \\
            1 & -1
        \end{array} \right)$
        
        Tenemos que $rg(A_4) = 2 \Longrightarrow Nul(g_4)=0$. Por tanto, por el Teorema de Sylvester,
        \begin{equation*}
            A_4 \sim_c \left(\begin{array}{cc}
                1 &  \\
                & -1
            \end{array} \right)
            \qquad \text{o} \qquad
            A_4 \sim_c \left(\begin{array}{cc}
                1 &  \\
                & 1
            \end{array} \right)
            \qquad \text{o} \qquad
            A_4 \sim_c \left(\begin{array}{cc}
                -1 &  \\
                & -1
            \end{array} \right)
        \end{equation*}

        Tenemos también que $|A_4| = -2$. Como el signo del determinante es un invariante, concluimos que:
        \begin{equation*}
            A_4 \sim_c \left(\begin{array}{cc}
                1 &  \\
                & -1
            \end{array} \right)
        \end{equation*}
        
        Por tanto, concluimos que $Nul(g_4)=0$ y $Ind(g_4)=1$.

        \item $A_5=\left(\begin{array}{cc}
            1 & 2 \\
            2 & 3
        \end{array} \right)$
        
        Tenemos que $rg(A_5) = 2 \Longrightarrow Nul(g_5)=0$. Por tanto, por el Teorema de Sylvester,
        \begin{equation*}
            A_5 \sim_c \left(\begin{array}{cc}
                1 &  \\
                & -1
            \end{array} \right)
            \qquad \text{o} \qquad
            A_5 \sim_c \left(\begin{array}{cc}
                1 &  \\
                & 1
            \end{array} \right)
            \qquad \text{o} \qquad
            A_5 \sim_c \left(\begin{array}{cc}
                -1 &  \\
                & -1
            \end{array} \right)
        \end{equation*}

        Tenemos también que $|A_5| = -1$. Como el signo del determinante es un invariante, concluimos que:
        \begin{equation*}
            A_5 \sim_c \left(\begin{array}{cc}
                1 &  \\
                & -1
            \end{array} \right)
        \end{equation*}
        
        Por tanto, concluimos que $Nul(g_5)=0$ y $Ind(g_5)=1$.

        \item $A_6=\left(\begin{array}{cc}
            1 & -2 \\
            -2 & 3
        \end{array} \right)$
        
        Tenemos que $rg(A_6) = 2 \Longrightarrow Nul(g_6)=0$. Por tanto, por el Teorema de Sylvester,
        \begin{equation*}
            A_6 \sim_c \left(\begin{array}{cc}
                1 &  \\
                & -1
            \end{array} \right)
            \qquad \text{o} \qquad
            A_6 \sim_c \left(\begin{array}{cc}
                1 &  \\
                & 1
            \end{array} \right)
            \qquad \text{o} \qquad
            A_6 \sim_c \left(\begin{array}{cc}
                -1 &  \\
                & -1
            \end{array} \right)
        \end{equation*}

        Tenemos también que $|A_6| = -1$. Como el signo del determinante es un invariante, concluimos que:
        \begin{equation*}
            A_6 \sim_c \left(\begin{array}{cc}
                1 &  \\
                & -1
            \end{array} \right)
        \end{equation*}
        
        Por tanto, concluimos que $Nul(g_6)=0$ y $Ind(g_6)=1$.

        \item $A_7=\left(\begin{array}{cc}
            1 & -2 \\
            -2 & -3
        \end{array} \right)$
        
        Tenemos que $rg(A_7) = 2 \Longrightarrow Nul(g_7)=0$. Por tanto, por el Teorema de Sylvester,
        \begin{equation*}
            A_7 \sim_c \left(\begin{array}{cc}
                1 &  \\
                & -1
            \end{array} \right)
            \qquad \text{o} \qquad
            A_7 \sim_c \left(\begin{array}{cc}
                1 &  \\
                & 1
            \end{array} \right)
            \qquad \text{o} \qquad
            A_7 \sim_c \left(\begin{array}{cc}
                -1 &  \\
                & -1
            \end{array} \right)
        \end{equation*}

        Tenemos también que $|A_7| = -7$. Como el signo del determinante es un invariante, concluimos que:
        \begin{equation*}
            A_7 \sim_c \left(\begin{array}{cc}
                1 &  \\
                & -1
            \end{array} \right)
        \end{equation*}
        
        Por tanto, concluimos que $Nul(g_7)=0$ y $Ind(g_7)=1$.
    \end{enumerate}

    Por tanto, tenemos que:
    \begin{equation}
        \begin{array}{c|c|c}
            A_i=M(g_i;\cc{B}) & Nul(g_i) & Ind(g_i)\\ \hline
            g_1 & 0 & 1 \\
            g_2 & 0 & 1 \\
            g_3 & 1 & 0 \\
            g_4 & 0 & 1 \\
            g_5 & 0 & 1 \\
            g_6 & 0 & 1 \\
            g_7 & 0 & 1 \\
        \end{array}
    \end{equation}
\end{ejercicio}

\begin{ejercicio} Referidas a las matrices del ejercicio anterior,
    \begin{enumerate}
        \item Estudiar cuáles de estas matrices son congruentes.

        Haciendo uso de que $\sim_c$ es una relación de equivalencia, tenemos que:
        \begin{equation*}
            A_i \sim_c A_j \;\forall i,j\neq 3
            \qquad \qquad
            A_3 \nsim_c A_i \; \forall i
        \end{equation*}

        \item Estudiar cuáles de las métricas anteriores son isométricas.

        Como dos matrices isométricas equivale a que sean congruentes, tenemos el mismo resultado que en el apartado anterior.
    \end{enumerate}
\end{ejercicio}

\begin{ejercicio}
    Para cada una de las métricas anteriores construir, a partir de la base $\mathcal{B}=\{e_1,e_2\}$, una base de Sylvester $\cc{B}_{i}^S$.
    \begin{enumerate}
        \item $A_1=\left(\begin{array}{cc}
            0 & 1 \\
            1 & 0
        \end{array} \right)$
        
        Tenemos que:
        \begin{equation*}
            A_1 \sim_c \left(\begin{array}{cc}
                1 &  \\
                & -1
            \end{array} \right) = M(g_1, \cc{B}_{1}^S)
        \end{equation*}
        
        Busco ahora un vector $\bar{e_1}$ de cuadrado no nulo. Sea $\bar{e_1} = (1,1)^t_{\cc{B}} = e_1 + e_2$.
        \begin{equation*}
            g_1(\bar{e_1}, \bar{e_1}) = g_1(e_1+e_2, e_1+e_2) = g(e_1,e_1) + g(e_2,e_2) + 2g(e_1, e_2) = 2 \neq 0
        \end{equation*}

        Busco ahora $\bar{e_2} \perp \bar{e_1}$.
        \begin{equation*}\begin{split}
            <\bar{e_1}>^\perp &= \{v \in V \mid g_1(\bar{e_1},v) = 0\} 
            = \left\{ \left(\begin{array}{c}
                 x_1 \\ x_2
            \end{array} \right) \in V \mid \bar{e_1}^t A_1
            \left(\begin{array}{c}
                 x_1 \\ x_2
            \end{array} \right) = 0\right\} \\
            &= \left\{ \left(\begin{array}{c}
                 x_1 \\ x_2
            \end{array} \right) \in V \mid (1,1)\left(\begin{array}{cc}
                0 & 1 \\
                1 & 0
            \end{array} \right) 
            \left(\begin{array}{c}
                 x_1 \\ x_2
            \end{array} \right) = 0\right\} \\
            &= \left\{ \left(\begin{array}{c}
                 x_1 \\ x_2
            \end{array} \right) \in V \mid (1,1)
            \left(\begin{array}{c}
                 x_1 \\ x_2
            \end{array} \right) = 0\right\} \\
            &= \left\{ \left(\begin{array}{c}
                 x_1 \\ x_2
            \end{array} \right) \in V \mid x_1+x_2 = 0\right\}
            = \cc{L} \left\{ \left(\begin{array}{c}
                 -1 \\ 1
            \end{array} \right) \right\}
        \end{split}\end{equation*}
        Por tanto, $\bar{e_2} = (-1,1)^t_\cc{B} = -e_1+e_2$.
        \begin{equation*}
            g_1(\bar{e_2}, \bar{e_2}) = g_1(-e_1+e_2, -e_1+e_2) = g(e_1,e_1) + g(e_2,e_2) - 2g(e_1, e_2) = -2 \neq 0
        \end{equation*}

        Por tanto, tenemos que, dado $\bar{\cc{B}} = \{\bar{e_1},\bar{e_2}\} = \{e_1+e_2, -e_1+e_2\}$,
        \begin{equation*}
            M(g_1, \bar{\cc{B}}) = \left(\begin{array}{cc}
            2 &  \\
             & -2
        \end{array} \right)
        \end{equation*}

        Por tanto, la base de Sylvester es:
        \begin{equation*}
            \cc{B}_1^S = \left\{\frac{\bar{e_1}}{\sqrt{2}},\frac{\bar{e_2}}{\sqrt{2}} \right\} = \left\{\frac{e_1+e_2}{\sqrt{2}},\frac{-e_1 + e_2}{\sqrt{2}} \right\}
        \end{equation*}

        Calculamos sus imágenes:
        \begin{equation*}
            g_1 \left(\frac{\bar{e_1}}{\sqrt{2}},\frac{\bar{e_1}}{\sqrt{2}}\right) = \frac{1}{\sqrt{2}^2} \cdot g_1(\bar{e_1}, \bar{e_1}) = \frac{1}{2}\cdot 2 = 1
        \end{equation*}
        \begin{equation*}
            g_1 \left(\frac{\bar{e_2}}{\sqrt{2}},\frac{\bar{e_2}}{\sqrt{2}}\right) = \frac{1}{\sqrt{2}^2} \cdot g_1(\bar{e_2}, \bar{e_2}) = \frac{1}{2}\cdot (-2) = -1
        \end{equation*}

        Efectivamente, tenemos que:
        \begin{equation*}
            M(g_1, \cc{B}_{1}^S) = \left(\begin{array}{cc}
                1 &  \\
                & -1
            \end{array} \right)
        \end{equation*}

        \item $A_2=\left(\begin{array}{cc}
            0 & 1 \\
            1 & 1
        \end{array} \right)$
        
        Tenemos que:
        \begin{equation*}
            A_2 \sim_c \left(\begin{array}{cc}
                1 &  \\
                & -1
            \end{array} \right) = M(g_2, \cc{B}_{2}^S)
        \end{equation*}
        
        Busco ahora un vector $\bar{e_1}$ de cuadrado no nulo. Sea $\bar{e_1} = (0,1)^t_{\cc{B}} = e_2$.
        \begin{equation*}
            g_2(\bar{e_1}, \bar{e_1}) =
            g_2(e_2, e_2) = 1 \neq 0
        \end{equation*}

        Busco ahora $\bar{e_2} \perp \bar{e_1}$.
        \begin{equation*}\begin{split}
            <\bar{e_1}>^\perp &= \{v \in V \mid g_2(\bar{e_1},v) = 0\} 
            = \left\{ \left(\begin{array}{c}
                 x_1 \\ x_2
            \end{array} \right) \in V \mid \bar{e_1}^t A_2
            \left(\begin{array}{c}
                 x_1 \\ x_2
            \end{array} \right) = 0\right\} \\
            &= \left\{ \left(\begin{array}{c}
                 x_1 \\ x_2
            \end{array} \right) \in V \mid (0,1)\left(\begin{array}{cc}
                0 & 1 \\
                1 & 1
            \end{array} \right) 
            \left(\begin{array}{c}
                 x_1 \\ x_2
            \end{array} \right) = 0\right\} \\
            &= \left\{ \left(\begin{array}{c}
                 x_1 \\ x_2
            \end{array} \right) \in V \mid (1,1)
            \left(\begin{array}{c}
                 x_1 \\ x_2
            \end{array} \right) = 0\right\} \\
            &= \left\{ \left(\begin{array}{c}
                 x_1 \\ x_2
            \end{array} \right) \in V \mid x_1+x_2 = 0\right\}
            = \cc{L} \left\{ \left(\begin{array}{c}
                 -1 \\ 1
            \end{array} \right) \right\}
        \end{split}\end{equation*}
        Por tanto, $\bar{e_2} = (-1,1)^t_\cc{B} = -e_1+e_2$.
        \begin{equation*}
            g_2(\bar{e_2}, \bar{e_2}) = g_2(-e_1+e_2, -e_1+e_2) = g_2(e_1,e_1) + g_2(e_2,e_2) - 2g_2(e_1, e_2) = 0+1-2=-1 \neq 0
        \end{equation*}

        Por tanto, tenemos que, dado $\bar{\cc{B}} = \{\bar{e_1},\bar{e_2}\} = \{e_2, -e_1+e_2\}$,
        \begin{equation*}
            M(g_2, \bar{\cc{B}}) = \left(\begin{array}{cc}
            1 &  \\
             & -1
        \end{array} \right)
        \end{equation*}

        Por tanto, la base de Sylvester es:
        \begin{equation*}
            \cc{B}_2^S = \bar{\cc{B}} = \{\bar{e_1},\bar{e_2}\} = \{e_2, -e_1+e_2\}
        \end{equation*}

        \item $A_3=\left(\begin{array}{cc}
            1 & 1 \\
            1 & 1
        \end{array} \right)$
        
        Tenemos que:
        \begin{equation*}
            A_3 \sim_c \left(\begin{array}{cc}
                1 &  \\
                & 0
            \end{array} \right) = M(g_3, \cc{B}_{3}^S)
        \end{equation*}
        
        Busco ahora un vector $\bar{e_1}$ de cuadrado no nulo. Sea $\bar{e_1} = (1,0)^t_{\cc{B}} = e_1$.
        \begin{equation*}
            g_3(\bar{e_1}, \bar{e_1}) =
            g_3(e_1, e_1) = 1 \neq 0
        \end{equation*}

        Busco ahora $\bar{e_2} \in Ker(g_3)$.
        \begin{equation*}\begin{split}
            Ker(g_3) &= \{v \in V \mid g_3(u,v) = 0 \qquad \forall u\in V\} 
            =\\
            &= \left\{ \left(\begin{array}{c}
                 x_1 \\ x_2
            \end{array} \right) \in V \mid
            A_3
            \left(\begin{array}{c}
                 x_1 \\ x_2
            \end{array} \right) = 0\right\} \\
            &= \left\{ \left(\begin{array}{c}
                 x_1 \\ x_2
            \end{array} \right) \in V \mid \left(\begin{array}{cc}
                1 & 1 \\
                1 & 1
            \end{array} \right) 
            \left(\begin{array}{c}
                 x_1 \\ x_2
            \end{array} \right) = 0\right\} \\
            &= \left\{ \left(\begin{array}{c}
                 x_1 \\ x_2
            \end{array} \right) \in V \mid x_1+x_2 = 0\right\}
            = \cc{L} \left\{ \left(\begin{array}{c}
                 -1 \\ 1
            \end{array} \right) \right\}
        \end{split}\end{equation*}
        Por tanto, $\bar{e_2} = (-1,1)^t_\cc{B} = -e_1+e_2$.
        \begin{equation*}
            g_3(\bar{e_2}, \bar{e_2}) = 0
        \end{equation*}

        Por tanto, tenemos que, dado $\bar{\cc{B}} = \{\bar{e_1},\bar{e_2}\} = \{e_1, -e_1+e_2\}$,
        \begin{equation*}
            M(g_3, \bar{\cc{B}}) = \left(\begin{array}{cc}
            1 &  \\
             & 0
        \end{array} \right)
        \end{equation*}

        Por tanto, la base de Sylvester es:
        \begin{equation*}
            \cc{B}_3^S = \bar{\cc{B}} = \{\bar{e_1},\bar{e_2}\} = \{e_1, -e_1+e_2\}
        \end{equation*}

        \item $A_4=\left(\begin{array}{cc}
            1 & 1 \\
            1 & -1
        \end{array} \right)$
        
        Tenemos que:
        \begin{equation*}
            A_4 \sim_c \left(\begin{array}{cc}
                1 &  \\
                & -1
            \end{array} \right) = M(g_4, \cc{B}_{4}^S)
        \end{equation*}
        
        Busco ahora un vector $\bar{e_1}$ de cuadrado no nulo. Sea $\bar{e_1} = (1,0)^t_{\cc{B}} = e_1$.
        \begin{equation*}
            g_4(\bar{e_1}, \bar{e_1}) =
            g_4(e_1, e_1) = 1 \neq 0
        \end{equation*}

        Busco ahora $\bar{e_2} \perp \bar{e_1}$.
        \begin{equation*}\begin{split}
            <\bar{e_1}>^\perp &= \{v \in V \mid g_4(\bar{e_1},v) = 0\} 
            = \left\{ \left(\begin{array}{c}
                 x_1 \\ x_2
            \end{array} \right) \in V \mid \bar{e_1}^t A_4
            \left(\begin{array}{c}
                 x_1 \\ x_2
            \end{array} \right) = 0\right\} \\
            &= \left\{ \left(\begin{array}{c}
                 x_1 \\ x_2
            \end{array} \right) \in V \mid (1,0)\left(\begin{array}{cc}
                1 & 1 \\
                1 & -1
            \end{array} \right) 
            \left(\begin{array}{c}
                 x_1 \\ x_2
            \end{array} \right) = 0\right\} \\
            &= \left\{ \left(\begin{array}{c}
                 x_1 \\ x_2
            \end{array} \right) \in V \mid (1,1)
            \left(\begin{array}{c}
                 x_1 \\ x_2
            \end{array} \right) = 0\right\} \\
            &= \left\{ \left(\begin{array}{c}
                 x_1 \\ x_2
            \end{array} \right) \in V \mid x_1+x_2 = 0\right\}
            = \cc{L} \left\{ \left(\begin{array}{c}
                 -1 \\ 1
            \end{array} \right) \right\}
        \end{split}\end{equation*}
        Por tanto, $\bar{e_2} = (-1,1)^t_\cc{B} = -e_1+e_2$.
        \begin{equation*}
            g_4(\bar{e_2}, \bar{e_2}) = g_4(-e_1+e_2, -e_1+e_2) = g_4(e_1,e_1) + g_4(e_2,e_2) - 2g_4(e_1, e_2) = 1-1-2=-2 \neq 0
        \end{equation*}

        Por tanto, tenemos que, dado $\bar{\cc{B}} = \{\bar{e_1},\bar{e_2}\} = \{e_1, -e_1+e_2\}$,
        \begin{equation*}
            M(g_4, \bar{\cc{B}}) = \left(\begin{array}{cc}
            1 &  \\
             & -2
        \end{array} \right)
        \end{equation*}

        Por tanto, la base de Sylvester es:
        \begin{equation*}
            \cc{B}_4^S = \left\{\bar{e_1},\frac{\bar{e_2}}{\sqrt{2}}\right\} = \left\{e_1, \frac{-e_1+e_2}{\sqrt{2}}\right\}
        \end{equation*}

        \item $A_5=\left(\begin{array}{cc}
            1 & 2 \\
            2 & 3
        \end{array} \right)$
        
        Tenemos que:
        \begin{equation*}
            A_5 \sim_c \left(\begin{array}{cc}
                1 &  \\
                & -1
            \end{array} \right) = M(g_5, \cc{B}_{5}^S)
        \end{equation*}
        
        Busco ahora un vector $\bar{e_1}$ de cuadrado no nulo. Sea $\bar{e_1} = (1,0)^t_{\cc{B}} = e_1$.
        \begin{equation*}
            g_5(\bar{e_1}, \bar{e_1}) =
            g_5(e_1, e_1) = 1 \neq 0
        \end{equation*}

        Busco ahora $\bar{e_2} \perp \bar{e_1}$.
        \begin{equation*}\begin{split}
            <\bar{e_1}>^\perp &= \{v \in V \mid g_5(\bar{e_1},v) = 0\} 
            = \left\{ \left(\begin{array}{c}
                 x_1 \\ x_2
            \end{array} \right) \in V \mid \bar{e_1}^t A_5
            \left(\begin{array}{c}
                 x_1 \\ x_2
            \end{array} \right) = 0\right\} \\
            &= \left\{ \left(\begin{array}{c}
                 x_1 \\ x_2
            \end{array} \right) \in V \mid (1,0)\left(\begin{array}{cc}
                1 & 2 \\
                2 & 3
            \end{array} \right) 
            \left(\begin{array}{c}
                 x_1 \\ x_2
            \end{array} \right) = 0\right\} \\
            &= \left\{ \left(\begin{array}{c}
                 x_1 \\ x_2
            \end{array} \right) \in V \mid (1,2)
            \left(\begin{array}{c}
                 x_1 \\ x_2
            \end{array} \right) = 0\right\} \\
            &= \left\{ \left(\begin{array}{c}
                 x_1 \\ x_2
            \end{array} \right) \in V \mid x_1+2x_2 = 0\right\}
            = \cc{L} \left\{ \left(\begin{array}{c}
                 2 \\ -1
            \end{array} \right) \right\}
        \end{split}\end{equation*}
        Por tanto, $\bar{e_2} = (2,-1)^t_\cc{B} = 2e_1-e_2$.
        \begin{multline*}
            g_5(\bar{e_2}, \bar{e_2}) = g_5(2e_1-e_2, 2e_1-e_2) = g_5(2e_1,2e_1) + g_5(e_2,e_2) - 2g_5(2e_1, e_2)
            =\\=
            2^2g_5(e_1, e_1) + g_5(e_2,e_2)-2^2g_5(e_1, e_2) = 2^2 + 3-2^3 = -1
        \end{multline*}

        Por tanto, tenemos que, dado $\bar{\cc{B}} = \{\bar{e_1},\bar{e_2}\} = \{e_1, 2e_1-e_2\}$,
        \begin{equation*}
            M(g_5, \bar{\cc{B}}) = \left(\begin{array}{cc}
            1 &  \\
             & -1
        \end{array} \right)
        \end{equation*}

        Por tanto, la base de Sylvester es:
        \begin{equation*}
            \cc{B}_5^S = \bar{\cc{B}} = \left\{\bar{e_1},\bar{e_2}\right\} = \left\{e_1, 2e_1-e_2\right\}
        \end{equation*}

        \item $A_6=\left(\begin{array}{cc}
            1 & -2 \\
            -2 & 3
        \end{array} \right)$
        
        Tenemos que:
        \begin{equation*}
            A_6 \sim_c \left(\begin{array}{cc}
                1 &  \\
                & -1
            \end{array} \right) = M(g_6, \cc{B}_{6}^S)
        \end{equation*}
        
        Busco ahora un vector $\bar{e_1}$ de cuadrado no nulo. Sea $\bar{e_1} = (1,0)^t_{\cc{B}} = e_1$.
        \begin{equation*}
            g_6(\bar{e_1}, \bar{e_1}) =
            g_6(e_1, e_1) = 1 \neq 0
        \end{equation*}

        Busco ahora $\bar{e_2} \perp \bar{e_1}$.
        \begin{equation*}\begin{split}
            <\bar{e_1}>^\perp &= \{v \in V \mid g_6(\bar{e_1},v) = 0\} 
            = \left\{ \left(\begin{array}{c}
                 x_1 \\ x_2
            \end{array} \right) \in V \mid \bar{e_1}^t A_6
            \left(\begin{array}{c}
                 x_1 \\ x_2
            \end{array} \right) = 0\right\} \\
            &= \left\{ \left(\begin{array}{c}
                 x_1 \\ x_2
            \end{array} \right) \in V \mid (1,0)\left(\begin{array}{cc}
                1 & -2 \\
                -2 & 3
            \end{array} \right) 
            \left(\begin{array}{c}
                 x_1 \\ x_2
            \end{array} \right) = 0\right\} \\
            &= \left\{ \left(\begin{array}{c}
                 x_1 \\ x_2
            \end{array} \right) \in V \mid (1,-2)
            \left(\begin{array}{c}
                 x_1 \\ x_2
            \end{array} \right) = 0\right\} \\
            &= \left\{ \left(\begin{array}{c}
                 x_1 \\ x_2
            \end{array} \right) \in V \mid x_1-2x_2 = 0\right\}
            = \cc{L} \left\{ \left(\begin{array}{c}
                 2 \\ 1
            \end{array} \right) \right\}
        \end{split}\end{equation*}
        Por tanto, $\bar{e_2} = (2,1)^t_\cc{B} = 2e_1+e_2$.
        \begin{multline*}
            g_6(\bar{e_2}, \bar{e_2}) = g_6(2e_1+e_2, 2e_1+e_2) = g_6(2e_1,2e_1) + g_6(e_2,e_2) + 2g_6(2e_1, e_2)
            =\\=
            2^2g_6(e_1, e_1) + g_6(e_2,e_2)+2^2g_6(e_1, e_2) = 2^2 + 3+(-2)^3 = -1
        \end{multline*}

        Por tanto, tenemos que, dado $\bar{\cc{B}} = \{\bar{e_1},\bar{e_2}\} = \{e_1, 2e_1+e_2\}$,
        \begin{equation*}
            M(g_6, \bar{\cc{B}}) = \left(\begin{array}{cc}
            1 &  \\
             & -1
        \end{array} \right)
        \end{equation*}

        Por tanto, la base de Sylvester es:
        \begin{equation*}
            \cc{B}_6^S = \bar{\cc{B}} = \left\{\bar{e_1},\bar{e_2}\right\} = \left\{e_1, 2e_1+e_2\right\}
        \end{equation*}

        \item $A_7=\left(\begin{array}{cc}
            1 & -2 \\
            -2 & -3
        \end{array} \right)$
        
        Tenemos que:
        \begin{equation*}
            A_7 \sim_c \left(\begin{array}{cc}
                1 &  \\
                & -1
            \end{array} \right) = M(g_7, \cc{B}_{7}^S)
        \end{equation*}
        
        Busco ahora un vector $\bar{e_1}$ de cuadrado no nulo. Sea $\bar{e_1} = (1,0)^t_{\cc{B}} = e_1$.
        \begin{equation*}
            g_7(\bar{e_1}, \bar{e_1}) =
            g_7(e_1, e_1) = 1 \neq 0
        \end{equation*}

        Busco ahora $\bar{e_2} \perp \bar{e_1}$.
        \begin{equation*}\begin{split}
            <\bar{e_1}>^\perp &= \{v \in V \mid g_7(\bar{e_1},v) = 0\} 
            = \left\{ \left(\begin{array}{c}
                 x_1 \\ x_2
            \end{array} \right) \in V \mid \bar{e_1}^t A_7
            \left(\begin{array}{c}
                 x_1 \\ x_2
            \end{array} \right) = 0\right\} \\
            &= \left\{ \left(\begin{array}{c}
                 x_1 \\ x_2
            \end{array} \right) \in V \mid (1,0)\left(\begin{array}{cc}
                1 & -2 \\
                -2 & -3
            \end{array} \right) 
            \left(\begin{array}{c}
                 x_1 \\ x_2
            \end{array} \right) = 0\right\} \\
            &= \left\{ \left(\begin{array}{c}
                 x_1 \\ x_2
            \end{array} \right) \in V \mid (1,-2)
            \left(\begin{array}{c}
                 x_1 \\ x_2
            \end{array} \right) = 0\right\} \\
            &= \left\{ \left(\begin{array}{c}
                 x_1 \\ x_2
            \end{array} \right) \in V \mid x_1-2x_2 = 0\right\}
            = \cc{L} \left\{ \left(\begin{array}{c}
                 2 \\ 1
            \end{array} \right) \right\}
        \end{split}\end{equation*}
        Por tanto, $\bar{e_2} = (2,1)^t_\cc{B} = 2e_1+e_2$.
        \begin{multline*}
            g_7(\bar{e_2}, \bar{e_2}) = g_7(2e_1+e_2, 2e_1+e_2) = g_7(2e_1,2e_1) + g_7(e_2,e_2) + 2g_7(2e_1, e_2)
            =\\=
            2^2g_7(e_1, e_1) + g_7(e_2,e_2)+2^2g_7(e_1, e_2) = 2^2 - 3-2^3 = -7
        \end{multline*}

        Por tanto, tenemos que, dado $\bar{\cc{B}} = \{\bar{e_1},\bar{e_2}\} = \{e_1, 2e_1+e_2\}$,
        \begin{equation*}
            M(g_7, \bar{\cc{B}}) = \left(\begin{array}{cc}
            1 &  \\
             & -7
        \end{array} \right)
        \end{equation*}

        Por tanto, la base de Sylvester es:
        \begin{equation*}
            \cc{B}_7^S = \left\{\bar{e_1},\frac{\bar{e_2}}{\sqrt{7}}\right\} = \left\{e_1, \frac{2e_1+e_2}{\sqrt{7}}\right\}
        \end{equation*}
    \end{enumerate}
\end{ejercicio}

\begin{ejercicio}
    Sea $g$ una métrica en un espacio vectorial $V^3(\bb{K})$, $\bb{K}=\bb{R}$ o $\bb{K} = \bb{C}$, que respecto de la base $\mathcal{B}=\{e_1, e_2, e_3\}$ viene dada por la matriz $M(g_i; \mathcal{B})=A_i$.

    \begin{equation*}
        A_1 = \left(\begin{array}{ccc}
            0 & 0 & 1 \\
            0 & 1 & 0 \\
            1 & 0 & 0 \\
        \end{array} \right) \qquad
        A_2 = \left(\begin{array}{ccc}
            0 & 1 & 1 \\
            1 & 1 & 1 \\
            1 & 1 & 0 \\
        \end{array} \right) \qquad
        A_3 = \left(\begin{array}{ccc}
            1 & 1 & 1 \\
            1 & 1 & 0 \\
            1 & 0 & 0 \\
        \end{array} \right)
    \end{equation*}
    \begin{equation*}
        A_4 = \left(\begin{array}{ccc}
            0 & 1 & 1 \\
            1 & 0 & 1 \\
            1 & 1 & 0 \\
        \end{array} \right) \qquad
        A_5 = \left(\begin{array}{ccc}
            1 & 1 & 1 \\
            1 & 1 & 1 \\
            1 & 1 & 1 \\
        \end{array} \right) \qquad
        A_6 = \left(\begin{array}{ccc}
            1 & 0 & 1 \\
            0 & 1 & 0 \\
            1 & 0 & 1 \\
        \end{array} \right)
    \end{equation*}

    \begin{enumerate}
        \item Clasificar la métrica en cada caso.
        \begin{enumerate}
            \item $A_1 = \left(\begin{array}{ccc}
                    0 & 0 & 1 \\
                    0 & 1 & 0 \\
                    1 & 0 & 0 \\
                \end{array} \right)$

            Tenemos que $rg(A_1) = 3 \Longrightarrow Nul(g_1)=0$. Además, tenemos que \\ $|A_1|=-1$. Por tanto, por el Teorema de Sylvester,
            \begin{equation*}
                A_1 \sim_c \left(\begin{array}{ccc}
                    1 &  \\
                    & 1 \\
                    && -1 
                \end{array} \right)
                \qquad \text{o} \qquad
                A_1 \sim_c \left(\begin{array}{ccc}
                    -1 &  \\
                    & -1 \\
                    && -1 
                \end{array} \right)
            \end{equation*}

            Si fuese el segundo caso, tendríamos que $g_1$ sería definida negativa. No obstante, vemos que $g(e_2, e_2)=1>0$. Por tanto, no puede ser definida negativa, por lo que nos encontramos en el primer caso.
            \begin{equation*}
                A_1 \sim_c \left(\begin{array}{ccc}
                    1 &  \\
                    & 1 \\
                    && -1 
                \end{array} \right)
            \end{equation*}
            
            Por tanto, concluimos que $Nul(g_1)=0$ y $Ind(g_1)=1$.

            \item $A_2 = \left(\begin{array}{ccc}
                    0 & 1 & 1 \\
                    1 & 1 & 1 \\
                    1 & 1 & 0 \\
                \end{array} \right)$

            Tenemos que $rg(A_2) = 3 \Longrightarrow Nul(g_2)=0$. Además, tenemos que $|A_2| =~1$. Por tanto, por el Teorema de Sylvester,
            \begin{equation*}
                A_2 \sim_c \left(\begin{array}{ccc}
                    1 &  \\
                    & 1 \\
                    && 1 
                \end{array} \right)
                \qquad \text{o} \qquad
                A_2 \sim_c \left(\begin{array}{ccc}
                    1 &  \\
                    & -1 \\
                    && -1 
                \end{array} \right)
            \end{equation*}

            Si fuese el primer caso, tendríamos que $g_2$ sería definida positiva. No obstante, vemos que $g(e_1, e_1)=0 \ngtr 0$. Por tanto, no puede ser definida positiva, por lo que nos encontramos en el segundo caso.
            \begin{equation*}
                A_2 \sim_c \left(\begin{array}{ccc}
                    1 &  \\
                    & -1 \\
                    && -1 
                \end{array} \right)
            \end{equation*}
            
            Por tanto, concluimos que $Nul(g_2)=0$ y $Ind(g_2)=2$.

            \item $A_3 = \left(\begin{array}{ccc}
                    1 & 1 & 1 \\
                    1 & 1 & 0 \\
                    1 & 0 & 0 \\
                \end{array} \right)$

            Tenemos que $rg(A_3) = 3 \Longrightarrow Nul(g_3)=0$. Además, tenemos que\\$|A_3| =-1$. Por tanto, por el Teorema de Sylvester,
            \begin{equation*}
                A_3 \sim_c \left(\begin{array}{ccc}
                    1 &  \\
                    & 1 \\
                    && -1 
                \end{array} \right)
                \qquad \text{o} \qquad
                A_3 \sim_c \left(\begin{array}{ccc}
                    -1 &  \\
                    & -1 \\
                    && -1 
                \end{array} \right)
            \end{equation*}

            Si fuese el segundo caso, tendríamos que $g_3$ sería definida negativa. No obstante, vemos que $g(e_1, e_1)=1 \nless 0$. Por tanto, no puede ser definida negativa, por lo que nos encontramos en el primer caso.
            \begin{equation*}
                A_3 \sim_c \left(\begin{array}{ccc}
                    1 &  \\
                    & 1 \\
                    && -1 
                \end{array} \right)
            \end{equation*}
            
            Por tanto, concluimos que $Nul(g_3)=0$ y $Ind(g_3)=1$.

            \item $A_4 = \left(\begin{array}{ccc}
                    0 & 1 & 1 \\
                    1 & 0 & 1 \\
                    1 & 1 & 0 \\
                \end{array} \right)$

            Tenemos que $rg(A_4) = 3 \Longrightarrow Nul(g_4)=0$. Además, tenemos que $|A_4| =~2$. Por tanto, por el Teorema de Sylvester,
            \begin{equation*}
                A_4 \sim_c \left(\begin{array}{ccc}
                    1 &  \\
                    & -1 \\
                    && -1 
                \end{array} \right)
                \qquad \text{o} \qquad
                A_4 \sim_c \left(\begin{array}{ccc}
                    1 &  \\
                    & 1 \\
                    && 1 
                \end{array} \right)
            \end{equation*}

            Si fuese el segundo caso, tendríamos que $g_4$ sería definida positiva. No obstante, vemos que $g(e_2, e_2)=0 \ngtr 0$. Por tanto, no puede ser definida positiva, por lo que nos encontramos en el primer caso.
            \begin{equation*}
                A_4 \sim_c \left(\begin{array}{ccc}
                    1 &  \\
                    & -1 \\
                    && -1 
                \end{array} \right)
            \end{equation*}
            
            Por tanto, concluimos que $Nul(g_4)=0$ y $Ind(g_4)=2$.

            \item $A_5 = \left(\begin{array}{ccc}
                    1 & 1 & 1 \\
                    1 & 1 & 1 \\
                    1 & 1 & 1 \\
                \end{array} \right)$

            Tenemos que $rg(A_5) = 1 \Longrightarrow Nul(g_5)=2$. Por tanto, por el Teorema de Sylvester,
            \begin{equation*}
                A_5 \sim_c \left(\begin{array}{ccc}
                    1 &  \\
                    & 0 \\
                    && 0 
                \end{array} \right)
                \qquad \text{o} \qquad
                A_5 \sim_c \left(\begin{array}{ccc}
                    -1 &  \\
                    & 0 \\
                    && 0 
                \end{array} \right)
            \end{equation*}

            Sea $U=\cc{L}\{e_1\}$. Como ${g_5}_{\left|U\right.}$ es definida positiva, entonces tenemos que $k\geq 1$, con $k$ el número de 1 en la matriz asociada a la Base de Sylvester. Por tanto, nos encontramos en el primer caso.
            \begin{equation*}
                A_5 \sim_c \left(\begin{array}{ccc}
                    1 &  \\
                    & 0 \\
                    && 0 
                \end{array} \right)
            \end{equation*}
            
            Por tanto, concluimos que $Nul(g_5)=2$ y $Ind(g_5)=0$.

            \item $A_6 = \left(\begin{array}{ccc}
                    1 & 0 & 1 \\
                    0 & 1 & 0 \\
                    1 & 0 & 1 \\
                \end{array} \right)$

            Tenemos que $rg(A_6) = 2 \Longrightarrow Nul(g_6)=1$. Por tanto, por el Teorema de Sylvester,
            \begin{equation*}
                A_6 \sim_c \left(\begin{array}{ccc}
                    1 &  \\
                    & 1 \\
                    && 0 
                \end{array} \right)
                \quad \text{o} \quad
                A_6 \sim_c \left(\begin{array}{ccc}
                    -1 &  \\
                    & -1 \\
                    && 0 
                \end{array} \right)
                \quad \text{o} \quad
                A_6 \sim_c \left(\begin{array}{ccc}
                    1 &  \\
                    & -1 \\
                    && 0 
                \end{array} \right)
            \end{equation*}

            Sea $U=\cc{L}\{e_1, e_2\}$. Como ${g_6}_{\left|U\right.}$ es definida positiva, entonces tenemos que $k\geq 2$, con $k$ el número de 1 en la matriz asociada a la Base de Sylvester. Por tanto, nos encontramos en el primer caso.
            \begin{equation*}
                A_6 \sim_c \left(\begin{array}{ccc}
                    1 &  \\
                    & 1 \\
                    && 0 
                \end{array} \right)
            \end{equation*}
            
            Por tanto, concluimos que $Nul(g_6)=1$ y $Ind(g_6)=0$.
            
        \end{enumerate}
        Por tanto, tenemos que:
        \begin{equation}
            \begin{array}{c|c|c|c}
                g_i & Nul(g_i) & Ind(g_i) \;si\;\bb{K}=\bb{R} & Ind(g_i) \;si\;\bb{K}=\bb{C}\\ \hline
                g_1 & 0 & 1 & 0 \\
                g_2 & 0 & 2 & 0\\
                g_3 & 0 & 1 & 0\\
                g_4 & 0 & 2 & 0\\
                g_5 & 2 & 0 & 0\\
                g_6 & 1 & 0 & 0\\
            \end{array}
        \end{equation}

        \item Estudiar cuales de las métricas son congruentes entre sí.

        Haciendo uso de que $\sim_c$ es una relación de equivalencia, tenemos que:
        \begin{itemize}
            \item Para $\bb{K}=\bb{R}$: $g_1 \sim_c g_3 \qquad g_2\sim_c g_4$
            \item Para $\bb{K}=\bb{C}$: $g_i \sim_c g_j \qquad \forall i,j=1,\dots,4$
        \end{itemize}

        \item Construir, a partir de la base $\mathcal{B}$, una base de Sylvester en cada caso.

        \begin{enumerate}
            \item $A_1 = \left(\begin{array}{ccc}
                    0 & 0 & 1 \\
                    0 & 1 & 0 \\
                    1 & 0 & 0 \\
                \end{array} \right)$

            Tenemos que $Nul(A_1)=0$. Buscamos $\Bar{e_1}\in V^3 \mid g_1(\Bar{e_1},\Bar{e_1})=1$. Sea $\Bar{e_1}=(0,1,0)^t=e_2$.
            \begin{equation*}\begin{split}
                <\bar{e_1}>^\perp &= \{v \in V \mid g_1(\bar{e_1},v) = 0\} 
                = \left\{ \left(\begin{array}{c}
                     x_1 \\ x_2 \\ x_3
                \end{array} \right) \in V \mid \bar{e_1}^t A_1
                \left(\begin{array}{c}
                     x_1 \\ x_2 \\ x_3
                \end{array} \right) = 0\right\} \\
                &= \left\{ \left(\begin{array}{c}
                     x_1 \\ x_2 \\ x_3
                \end{array} \right) \in V \mid (0,1,0)\left(\begin{array}{ccc}
                    0 & 0 & 1 \\
                    0 & 1 & 0 \\
                    1 & 0 & 0 \\
                \end{array} \right) 
                \left(\begin{array}{c}
                     x_1 \\ x_2 \\ x_3
                \end{array} \right) = 0\right\} \\
                &= \left\{ \left(\begin{array}{c}
                     x_1 \\ x_2 \\ x_3
                \end{array} \right) \in V \mid (0,1,0)
                \left(\begin{array}{c}
                     x_1 \\ x_2 \\ x_3
                \end{array} \right) = 0\right\} \\
                &= \left\{ \left(\begin{array}{c}
                     x_1 \\ x_2 \\ x_3
                \end{array} \right) \in V \mid x_2 = 0\right\}
                = \cc{L} \left\{ \left(\begin{array}{c}
                     1 \\ 0 \\ 0
                \end{array} \right),
                \left(\begin{array}{c}
                     0 \\ 0 \\ 1
                \end{array} \right)\right\}
            \end{split}\end{equation*}
            Como necesito cuadrado no nulo, sea ahora $\bar{e_2}=(1,0,1)^t = e_1+e_3$.
            \begin{equation*}
                g_1(\bar{e_2},\bar{e_2}) = g(e_1,e_1) + g(e_3,e_3) + 2g(e_1,e_3) = 2
            \end{equation*}
            \begin{equation*}\begin{split}
                <\bar{e_2}>^\perp &= \{v \in V \mid g_1(\bar{e_2},v) = 0\} 
                = \left\{ \left(\begin{array}{c}
                     x_1 \\ x_2 \\ x_3
                \end{array} \right) \in V \mid \bar{e_2}^t A_1
                \left(\begin{array}{c}
                     x_1 \\ x_2 \\ x_3
                \end{array} \right) = 0\right\} \\
                &= \left\{ \left(\begin{array}{c}
                     x_1 \\ x_2 \\ x_3
                \end{array} \right) \in V \mid (1,0,1)\left(\begin{array}{ccc}
                    0 & 0 & 1 \\
                    0 & 1 & 0 \\
                    1 & 0 & 0 \\
                \end{array} \right) 
                \left(\begin{array}{c}
                     x_1 \\ x_2 \\ x_3
                \end{array} \right) = 0\right\} \\
                &= \left\{ \left(\begin{array}{c}
                     x_1 \\ x_2 \\ x_3
                \end{array} \right) \in V \mid (1,0,1)
                \left(\begin{array}{c}
                     x_1 \\ x_2 \\ x_3
                \end{array} \right) = 0\right\} \\
                &= \left\{ \left(\begin{array}{c}
                     x_1 \\ x_2 \\ x_3
                \end{array} \right) \in V \mid x_1+x_3 = 0\right\}
                = \cc{L} \left\{ \left(\begin{array}{c}
                     1 \\ 0 \\ -1
                \end{array} \right),
                \left(\begin{array}{c}
                     0 \\ 1 \\ 0
                \end{array} \right)\right\}
            \end{split}\end{equation*}
            Como necesito cuadrado no nulo, sea ahora $\bar{e_3}=(1,0,-1)^t = e_1-e_3$.
            \begin{equation*}
                g_1(\bar{e_3},\bar{e_3}) = g(e_1,e_1) + g(e_3,e_3) - 2g(e_1,e_3) = -2
            \end{equation*}
            Por tanto, dado $\Bar{\cc{B}}=\{\bar{e_1},\bar{e_2},\bar{e_3}\} = \{e_2, e_1+e_3, e_1-e_3\}$, tenemos:
            \begin{equation*}
                M(g_1, \Bar{\cc{B}}) = \left(\begin{array}{ccc}
                    1 &  &  \\
                     & 2 &  \\
                     &  & -2 \\
                \end{array} \right)
            \end{equation*}

            Por tanto, la base de Sylvester es:
            \begin{itemize}
                \item Para $\bb{K}=\bb{R}$: $\cc{B}_S = \left\{ e_2, \frac{e_1+e_3}{\sqrt{2}}, \frac{e_1-e_3}{\sqrt{2}}\right\}$.
                \item Para $\bb{K}=\bb{C}$: $\cc{B}_S = \left\{ e_2, \frac{e_1+e_3}{\sqrt{2}}, \frac{e_1-e_3}{\sqrt{-2}}\right\}$
            \end{itemize}

            \item $A_2 = \left(\begin{array}{ccc}
                    0 & 1 & 1 \\
                    1 & 1 & 1 \\
                    1 & 1 & 0 \\
                \end{array} \right)$

            Tenemos que $Nul(A_2)=0$. Buscamos $\Bar{e_1}\in V^3 \mid g_2(\Bar{e_1},\Bar{e_1})=1$. Sea $\Bar{e_1}=(0,1,0)^t=e_2$.
            \begin{equation*}\begin{split}
                <\bar{e_1}>^\perp &= \{v \in V \mid g_2(\bar{e_1},v) = 0\} 
                = \left\{ \left(\begin{array}{c}
                     x_1 \\ x_2 \\ x_3
                \end{array} \right) \in V \mid \bar{e_1}^t A_2
                \left(\begin{array}{c}
                     x_1 \\ x_2 \\ x_3
                \end{array} \right) = 0\right\} \\
                &= \left\{ \left(\begin{array}{c}
                     x_1 \\ x_2 \\ x_3
                \end{array} \right) \in V \mid (0,1,0)\left(\begin{array}{ccc}
                    0 & 1 & 1 \\
                    1 & 1 & 1 \\
                    1 & 1 & 0 \\
                \end{array} \right) 
                \left(\begin{array}{c}
                     x_1 \\ x_2 \\ x_3
                \end{array} \right) = 0\right\} \\
                &= \left\{ \left(\begin{array}{c}
                     x_1 \\ x_2 \\ x_3
                \end{array} \right) \in V \mid (1,1,1)
                \left(\begin{array}{c}
                     x_1 \\ x_2 \\ x_3
                \end{array} \right) = 0\right\} \\
                &= \left\{ \left(\begin{array}{c}
                     x_1 \\ x_2 \\ x_3
                \end{array} \right) \in V \mid x_1+x_2+x_3 = 0\right\}
                = \cc{L} \left\{ \left(\begin{array}{c}
                     1 \\ 0 \\ -1
                \end{array} \right),
                \left(\begin{array}{c}
                     0 \\ 1 \\ -1
                \end{array} \right)\right\}
            \end{split}\end{equation*}
            Como necesito cuadrado no nulo, sea ahora $\bar{e_2}=(1,0,-1)^t = e_1-e_3$.
            \begin{equation*}
                g_2(\bar{e_2},\bar{e_2}) = g(e_1,e_1) + g(e_3,e_3) - 2g(e_1,e_3) = -2
            \end{equation*}
            \begin{equation*}\begin{split}
                <\bar{e_2}>^\perp &= \{v \in V \mid g_2(\bar{e_2},v) = 0\} 
                = \left\{ \left(\begin{array}{c}
                     x_1 \\ x_2 \\ x_3
                \end{array} \right) \in V \mid \bar{e_2}^t A_2
                \left(\begin{array}{c}
                     x_1 \\ x_2 \\ x_3
                \end{array} \right) = 0\right\} \\
                &= \left\{ \left(\begin{array}{c}
                     x_1 \\ x_2 \\ x_3
                \end{array} \right) \in V \mid (1,0,-1)\left(\begin{array}{ccc}
                    0 & 1 & 1 \\
                    1 & 1 & 1 \\
                    1 & 1 & 0 \\
                \end{array} \right) 
                \left(\begin{array}{c}
                     x_1 \\ x_2 \\ x_3
                \end{array} \right) = 0\right\} \\
                &= \left\{ \left(\begin{array}{c}
                     x_1 \\ x_2 \\ x_3
                \end{array} \right) \in V \mid (-1,0,1)
                \left(\begin{array}{c}
                     x_1 \\ x_2 \\ x_3
                \end{array} \right) = 0\right\} \\
                &= \left\{ \left(\begin{array}{c}
                     x_1 \\ x_2 \\ x_3
                \end{array} \right) \in V \mid -x_1+x_3 = 0\right\}
                = \cc{L} \left\{ \left(\begin{array}{c}
                     1 \\ 0 \\ 1
                \end{array} \right),
                \left(\begin{array}{c}
                     0 \\ 1 \\ 0
                \end{array} \right)\right\}
            \end{split}\end{equation*}
            Por tanto, necesito que $\bar{e_3}\in (<\bar{e_1}>^\perp \cap <\bar{e_2}>^\perp)$. Sea $\bar{e_3}=(1,-2,1)^t = e_1-2e_2+e_3$.
            \begin{equation*}
                g_2(\bar{e_3},\bar{e_3}) = (1,-2,1)\left(\begin{array}{ccc}
                    0 & 1 & 1 \\
                    1 & 1 & 1 \\
                    1 & 1 & 0 \\
                \end{array} \right) \left(\begin{array}{c}
                     1 \\ -2 \\ 1
                \end{array} \right) = (-1,0,-1)\left(\begin{array}{c}
                     1 \\ -2 \\ 1
                \end{array} \right) = -2
            \end{equation*}
            Por tanto, dado $\Bar{\cc{B}}=\{\bar{e_1},\bar{e_2},\bar{e_3}\} = \{e_2, e_1-e_3, e_1-2e_2+e_3\}$, tenemos:
            \begin{equation*}
                M(g_2, \Bar{\cc{B}}) = \left(\begin{array}{ccc}
                    1 &  &  \\
                     & -2 &  \\
                     &  & -2 \\
                \end{array} \right)
            \end{equation*}

            Por tanto, la base de Sylvester es:
            \begin{itemize}
                \item Para $\bb{K}=\bb{R}$: $\cc{B}_S = \left\{ e_2, \frac{e_1-e_3}{\sqrt{2}}, \frac{e_1-2e_2+e_3}{\sqrt{2}}\right\}$.
                \item Para $\bb{K}=\bb{C}$: $\cc{B}_S = \left\{ e_2, \frac{e_1-e_3}{\sqrt{-2}}, \frac{e_1-2e_2+e_3}{\sqrt{-2}}\right\}$.
            \end{itemize}

            \item $A_3 = \left(\begin{array}{ccc}
                    1 & 1 & 1 \\
                    1 & 1 & 0 \\
                    1 & 0 & 0 \\
                \end{array} \right)$

            Tenemos que $Nul(A_3)=0$. Buscamos $\Bar{e_1}\in V^3 \mid g_3(\Bar{e_1},\Bar{e_1})=1$. Sea $\Bar{e_1}=(1,0,0)^t=e_1$.
            \begin{equation*}\begin{split}
                <\bar{e_1}>^\perp &= \{v \in V \mid g_3(\bar{e_1},v) = 0\} 
                = \left\{ \left(\begin{array}{c}
                     x_1 \\ x_2 \\ x_3
                \end{array} \right) \in V \mid \bar{e_1}^t A_3
                \left(\begin{array}{c}
                     x_1 \\ x_2 \\ x_3
                \end{array} \right) = 0\right\} \\
                &= \left\{ \left(\begin{array}{c}
                     x_1 \\ x_2 \\ x_3
                \end{array} \right) \in V \mid (1,0,0)\left(\begin{array}{ccc}
                    1 & 1 & 1 \\
                    1 & 1 & 0 \\
                    1 & 0 & 0 \\
                \end{array} \right) 
                \left(\begin{array}{c}
                     x_1 \\ x_2 \\ x_3
                \end{array} \right) = 0\right\} \\
                &= \left\{ \left(\begin{array}{c}
                     x_1 \\ x_2 \\ x_3
                \end{array} \right) \in V \mid (1,1,1)
                \left(\begin{array}{c}
                     x_1 \\ x_2 \\ x_3
                \end{array} \right) = 0\right\} \\
                &= \left\{ \left(\begin{array}{c}
                     x_1 \\ x_2 \\ x_3
                \end{array} \right) \in V \mid x_1+x_2+x_3 = 0\right\}
                = \cc{L} \left\{ \left(\begin{array}{c}
                     1 \\ 0 \\ -1
                \end{array} \right),
                \left(\begin{array}{c}
                     0 \\ 1 \\ -1
                \end{array} \right)\right\}
            \end{split}\end{equation*}
            Como necesito cuadrado no nulo, sea ahora $\bar{e_2}=(1,0,-1)^t = e_1-e_3$.
            \begin{equation*}
                g_3(\bar{e_2},\bar{e_2}) = g(e_1,e_1) + g(e_3,e_3) - 2g(e_1,e_3) = 1+0-2 = -1
            \end{equation*}
            \begin{equation*}\begin{split}
                <\bar{e_2}>^\perp &= \{v \in V \mid g_3(\bar{e_2},v) = 0\} 
                = \left\{ \left(\begin{array}{c}
                     x_1 \\ x_2 \\ x_3
                \end{array} \right) \in V \mid \bar{e_2}^t A_3
                \left(\begin{array}{c}
                     x_1 \\ x_2 \\ x_3
                \end{array} \right) = 0\right\} \\
                &= \left\{ \left(\begin{array}{c}
                     x_1 \\ x_2 \\ x_3
                \end{array} \right) \in V \mid (1,0,-1)\left(\begin{array}{ccc}
                    1 & 1 & 1 \\
                    1 & 1 & 0 \\
                    1 & 0 & 0 \\
                \end{array} \right) 
                \left(\begin{array}{c}
                     x_1 \\ x_2 \\ x_3
                \end{array} \right) = 0\right\} \\
                &= \left\{ \left(\begin{array}{c}
                     x_1 \\ x_2 \\ x_3
                \end{array} \right) \in V \mid (0,1,1)
                \left(\begin{array}{c}
                     x_1 \\ x_2 \\ x_3
                \end{array} \right) = 0\right\} \\
                &= \left\{ \left(\begin{array}{c}
                     x_1 \\ x_2 \\ x_3
                \end{array} \right) \in V \mid x_2+x_3 = 0\right\}
                = \cc{L} \left\{ \left(\begin{array}{c}
                     1 \\ 0 \\ 0
                \end{array} \right),
                \left(\begin{array}{c}
                     0 \\ 1 \\ -1
                \end{array} \right)\right\}
            \end{split}\end{equation*}
            Por tanto, necesito que $\bar{e_3}\in (<\bar{e_1}>^\perp \cap <\bar{e_2}>^\perp)$. Sea $\bar{e_3}=(0,1,-1)^t = e_2-e_3$.
            \begin{equation*}
                g_3(\bar{e_3},\bar{e_3}) = g(e_2,e_2)+g(e_3,e_3)-2g(e_2,e_3) = 1+0-0=1
            \end{equation*}
            Por tanto, dado $\Bar{\cc{B}}=\{\bar{e_1},\bar{e_2},\bar{e_3}\} = \{e_1, e_1-e_3, e_2-e_3\}$, tenemos:
            \begin{equation*}
                M(g_3, \Bar{\cc{B}}) = \left(\begin{array}{ccc}
                    1 &  &  \\
                     & -1 &  \\
                     &  & 1 \\
                \end{array} \right)
            \end{equation*}

            Por tanto, la base de Sylvester es:
            \begin{itemize}
                \item Para $\bb{K}=\bb{R}$: $\cc{B}_S = \left\{ e_1,e_2-e_3,e_1-e_3 \right\}$.
                \item Para $\bb{K}=\bb{C}$: $\cc{B}_S = \left\{ e_1,e_2-e_3,\frac{e_1-e_3}{i}\right\}$.
            \end{itemize}

            \item $A_4 = \left(\begin{array}{ccc}
                    0 & 1 & 1 \\
                    1 & 0 & 1 \\
                    1 & 1 & 0 \\
                \end{array} \right)$

            Tenemos que $Nul(A_4)=0$. Buscamos $\Bar{e_1}\in V^3 \mid g_4(\Bar{e_1},\Bar{e_1})\neq 0$. Sea $\Bar{e_1}=(1,1,0)^t=e_1+e_2$.
            \begin{equation*}
                g_4(\bar{e_1}, \bar{e_1}) = g(e_1,e_1) + g(e_2,e_2) + 2g(e_1,e_2) = 2
            \end{equation*}
            \begin{equation*}\begin{split}
                <\bar{e_1}>^\perp &= \{v \in V \mid g_4(\bar{e_1},v) = 0\} 
                = \left\{ \left(\begin{array}{c}
                     x_1 \\ x_2 \\ x_3
                \end{array} \right) \in V \mid \bar{e_1}^t A_4
                \left(\begin{array}{c}
                     x_1 \\ x_2 \\ x_3
                \end{array} \right) = 0\right\} \\
                &= \left\{ \left(\begin{array}{c}
                     x_1 \\ x_2 \\ x_3
                \end{array} \right) \in V \mid (1,1,0)\left(\begin{array}{ccc}
                    0 & 1 & 1 \\
                    1 & 0 & 1 \\
                    1 & 1 & 0 \\
                \end{array} \right) 
                \left(\begin{array}{c}
                     x_1 \\ x_2 \\ x_3
                \end{array} \right) = 0\right\} \\
                &= \left\{ \left(\begin{array}{c}
                     x_1 \\ x_2 \\ x_3
                \end{array} \right) \in V \mid (1,1,2)
                \left(\begin{array}{c}
                     x_1 \\ x_2 \\ x_3
                \end{array} \right) = 0\right\} \\
                &= \left\{ \left(\begin{array}{c}
                     x_1 \\ x_2 \\ x_3
                \end{array} \right) \in V \mid x_1+x_2+2x_3 = 0\right\}
                = \cc{L} \left\{ \left(\begin{array}{c}
                     1 \\  1 \\ -1
                \end{array} \right),
                \left(\begin{array}{c}
                     0 \\ 2 \\ -1
                \end{array} \right)\right\}
            \end{split}\end{equation*}
            Como necesito cuadrado no nulo, sea ahora $\bar{e_2}=(0,2,-1)^t = 2e_2-e_3$.
            \begin{equation*}
                g_3(\bar{e_2},\bar{e_2}) = 2^2g(e_2,e_2) + g(e_3,e_3) - 4g(e_2,e_3) = -4
            \end{equation*}
            \begin{equation*}\begin{split}
                <\bar{e_2}>^\perp &= \{v \in V \mid g_4(\bar{e_2},v) = 0\} 
                = \left\{ \left(\begin{array}{c}
                     x_1 \\ x_2 \\ x_3
                \end{array} \right) \in V \mid \bar{e_2}^t A_4
                \left(\begin{array}{c}
                     x_1 \\ x_2 \\ x_3
                \end{array} \right) = 0\right\} \\
                &= \left\{ \left(\begin{array}{c}
                     x_1 \\ x_2 \\ x_3
                \end{array} \right) \in V \mid (0,2,-1)\left(\begin{array}{ccc}
                    0 & 1 & 1 \\
                    1 & 0 & 1 \\
                    1 & 1 & 0 \\
                \end{array} \right) 
                \left(\begin{array}{c}
                     x_1 \\ x_2 \\ x_3
                \end{array} \right) = 0\right\} \\
                &= \left\{ \left(\begin{array}{c}
                     x_1 \\ x_2 \\ x_3
                \end{array} \right) \in V \mid (1,-1,2)
                \left(\begin{array}{c}
                     x_1 \\ x_2 \\ x_3
                \end{array} \right) = 0\right\} \\
                &= \left\{ \left(\begin{array}{c}
                     x_1 \\ x_2 \\ x_3
                \end{array} \right) \in V \mid x_1-x_2+2x_3 = 0\right\}
                = \cc{L} \left\{ \left(\begin{array}{c}
                     1 \\ 1 \\ 0
                \end{array} \right),
                \left(\begin{array}{c}
                     0 \\ 2 \\ 1
                \end{array} \right)\right\}
            \end{split}\end{equation*}
            Por tanto, necesito que $\bar{e_3}\in (<\bar{e_1}>^\perp \cap <\bar{e_2}>^\perp)$. Sea $\bar{e_3}=(2,0,-1)^t = 2e_1-e_3$.
            \begin{equation*}
                g_4(\bar{e_3},\bar{e_3}) = 2^2g(e_1,e_1)+g(e_3,e_3)-4g(e_1,e_3) = -4
            \end{equation*}
            Por tanto, dado $\Bar{\cc{B}}=\{\bar{e_1},\bar{e_2},\bar{e_3}\} = \{e_1+e_2, 2e_2-e_3, 2e_1-e_3\}$, tenemos:
            \begin{equation*}
                M(g_4, \Bar{\cc{B}}) = \left(\begin{array}{ccc}
                    2 &  &  \\
                     & -4 &  \\
                     &  & -4 \\
                \end{array} \right)
            \end{equation*}

            Por tanto, la base de Sylvester es:
            \begin{itemize}
                \item Para $\bb{K}=\bb{R}$: $\cc{B}_S = \left\{ \frac{e_1+e_2}{\sqrt{2}}, \frac{2e_2-e_3}{2}, \frac{2e_1-e_3}{2}\right\}$.
                \item Para $\bb{K}=\bb{C}$: $\cc{B}_S = \left\{ \frac{e_1+e_2}{\sqrt{2}}, \frac{2e_2-e_3}{2i}, \frac{2e_1-e_3}{2i}\right\}$.
            \end{itemize}

            \item $A_5 = \left(\begin{array}{ccc}
                    1 & 1 & 1 \\
                    1 & 1 & 1 \\
                    1 & 1 & 1 \\
                \end{array} \right)$

            Tenemos que $Nul(A_5)=2$. Calculamos por tanto $Ker(g_5)$:
            \begin{equation*}\begin{split}
                Ker(g_5) &= \{v \in V \mid g_5(v,u) = 0 \quad \forall u\in V\} 
                = \left\{ \left(\begin{array}{c}
                     x_1 \\ x_2 \\ x_3
                \end{array} \right) \in V \mid A_5
                \left(\begin{array}{c}
                     x_1 \\ x_2 \\ x_3
                \end{array} \right) = 0\right\} \\
                &= \left\{ \left(\begin{array}{c}
                     x_1 \\ x_2 \\ x_3
                \end{array} \right) \in V \mid
                \left(\begin{array}{ccc}
                    1 & 1 & 1 \\
                    1 & 1 & 1 \\
                    1 & 1 & 1 \\
                \end{array} \right) 
                \left(\begin{array}{c}
                     x_1 \\ x_2 \\ x_3
                \end{array} \right) = 0\right\} \\
                &= \left\{ \left(\begin{array}{c}
                     x_1 \\ x_2 \\ x_3
                \end{array} \right) \in V \mid x_1+x_2+x_3 = 0\right\}
                = \cc{L} \left\{ \left(\begin{array}{c}
                     1 \\  0 \\ -1
                \end{array} \right),
                \left(\begin{array}{c}
                     0 \\ 1 \\ -1
                \end{array} \right)\right\}
            \end{split}\end{equation*}

            Por tanto, sean $\bar{e_2}=(1,0,-1)^t$ y $\bar{e_3}=(0,1,-1)^t$. Como pertenecen al núcleo, son ortogonales a todos los demás vectores. Considero también $\bar{e_1}=e_1$, teniendo que $g(\bar{e_1},\bar{e_1}) = 1$.
            
            Por tanto, dado $\Bar{\cc{B}}=\{\bar{e_1},\bar{e_2},\bar{e_3}\} = \{e_1, e_1-e_3, e_2-e_3\}$, tenemos:
            \begin{equation*}
                M(g_5, \Bar{\cc{B}}) = \left(\begin{array}{ccc}
                    1 &  &  \\
                     & 0 &  \\
                     &  & 0 \\
                \end{array} \right)
            \end{equation*}

            $\cc{B}$ es la Base de Sylvester para $\bb{K}=\bb{C},\bb{R}$.

            \item $A_6 = \left(\begin{array}{ccc}
                    1 & 0 & 1 \\
                    0 & 1 & 0 \\
                    1 & 0 & 1 \\
                \end{array} \right)$

            Tenemos que $Nul(A_6)=1$. Calculamos por tanto $Ker(g_6)$:
            \begin{equation*}\begin{split}
                Ker(g_6) &= \{v \in V \mid g_6(v,u) = 0 \quad \forall u\in V\} 
                = \left\{ \left(\begin{array}{c}
                     x_1 \\ x_2 \\ x_3
                \end{array} \right) \in V \mid A_6
                \left(\begin{array}{c}
                     x_1 \\ x_2 \\ x_3
                \end{array} \right) = 0\right\} \\
                &= \left\{ \left(\begin{array}{c}
                     x_1 \\ x_2 \\ x_3
                \end{array} \right) \in V \mid
                \left(\begin{array}{ccc}
                    1 & 0 & 1 \\
                    0 & 1 & 0 \\
                    1 & 0 & 1 \\
                \end{array} \right) 
                \left(\begin{array}{c}
                     x_1 \\ x_2 \\ x_3
                \end{array} \right) = 0\right\} \\
                &= \left\{ \left(\begin{array}{c}
                     x_1 \\ x_2 \\ x_3
                \end{array} \right) \in V \left|\begin{array}{c}
                    x_1+x_3 = 0 \\
                    x_2=0
                \end{array}\right.\right\}
                = \cc{L} \left\{ \left(\begin{array}{c}
                     1 \\  0 \\ -1
                \end{array} \right)\right\}
            \end{split}\end{equation*}

            Por tanto, sea $\bar{e_3}=(1,0,-1)^t$. Como pertenece al núcleo, es ortogonales a todos los demás vectores.
            
            Considero también $\bar{e_1}=e_1$,$\bar{e_2}=e_2$, teniendo que $g(\bar{e_1},\bar{e_1}) = 1 = g(\bar{e_2},\bar{e_2})$. Además, podemos ver por la matriz $A_6$ que $e_1\perp e_2$.
            
            Por tanto, dado $\Bar{\cc{B}}=\{\bar{e_1},\bar{e_2},\bar{e_3}\} = \{e_1, e_2, e_1-e_3\}$, tenemos:
            \begin{equation*}
                M(g_6, \Bar{\cc{B}}) = \left(\begin{array}{ccc}
                    1 &  &  \\
                     & 1 &  \\
                     &  & 0 \\
                \end{array} \right)
            \end{equation*}

            $\cc{B}$ es la Base de Sylvester para $\bb{K}=\bb{C},\bb{R}$.
            
        \end{enumerate}
    \end{enumerate}
\end{ejercicio}

\begin{ejercicio}
    Dado $a\in \bb{R}$, consideramos la matriz simétrica real
    \begin{equation*}
        A_a = \left( \begin{array}{cccc}
            0 & 0 & 0 & 1 \\
            0 & a & 1 & 0 \\
            0 & 1 & a & 1 \\
            1 & 0 & 1 & 1 \\
        \end{array} \right)
    \end{equation*}

    \begin{enumerate}
        \item Clasificar la clase de congruencia de la matriz en función de los valores de $a$.

        Tomamos $A_a=M(g, \cc{B})$, para determinado $\cc{B}=\{e_1, e_2, e_3, e_4\}$.

        \begin{equation*}
            |A_a| = \left| \begin{array}{cccc}
                0 & 0 & 0 & 1 \\
                0 & a & 1 & 0 \\
                0 & 1 & a & 1 \\
                1 & 0 & 1 & 1 \\
            \end{array} \right|
            = -\left| \begin{array}{ccc}
                0 & 0 & 1 \\
                a & 1 & 0 \\
                1 & a & 1 \\
            \end{array} \right|
            = -\left| \begin{array}{cc}
                a & 1 \\
                1 & a \\
            \end{array} \right| = -a^2 + 1 = 0\Longleftrightarrow{a=\pm 1}
        \end{equation*}
        \begin{enumerate}
            \item \underline{Para $a>1$}:

            Tengo que $|A_a|<0$. Por tanto,
            \begin{equation*}
                A_a \sim_c \left( \begin{array}{cccc}
                1 &&& \\
                &1&& \\
                &&1&\\
                &&&-1
            \end{array} \right)
            \qquad \text{o} \qquad
            A_a \sim_c \left( \begin{array}{cccc}
                1 &&& \\
                &-1&& \\
                &&-1&\\
                &&&-1
            \end{array} \right)
            \end{equation*}

            No obstante, dado $U=\cc{L}\{e_3, e_4\}$, la restricción de $g$ a $U$ es definida positiva, por lo que en la diagonal de la matriz asociada a la base de Sylvester habrá, como mínimo, 2 unos. Por tanto, nos encontramos en el primer caso.
            \begin{equation*}
                A_a \sim_c \left( \begin{array}{cccc}
                1 &&& \\
                &1&& \\
                &&1&\\
                &&&-1
            \end{array} \right)
            \end{equation*}



            \item \underline{Para $a=1$}:

            Tengo que $|A_a|=0$. Además, 
            \begin{equation*}
                \left| \begin{array}{ccc}
                    1 & 1 & 0 \\
                    1 & 1 & 1 \\
                    0 & 1 & 1 \\
                \end{array} \right|
                = 1-1-1 = -1
            \end{equation*}

            Por tanto, tengo $rg(A_1)=3$, es decir, $Nul(A_1) = 1$.

            Además, dado $U=\cc{L}\{e_2, e_4\}$, tenemos que:
            \begin{equation*}
                M(g_{|U}, \cc{B}) = \left( \begin{array}{cc}
                    1 & 0 \\
                    0 & 1 \\
                \end{array} \right)
            \end{equation*}

            Por tanto, $g_{|U}$ es definida positiva, por lo que en la diagonal principal de la matriz asociada a la base de Sylvester habrá, como mínimo dos 1.
        
            Por tanto,
            \begin{equation*}
                A_1 \sim_c \left( \begin{array}{cccc}
                1 &&& \\
                &1&& \\
                &&1&\\
                &&&0
            \end{array} \right)
            \qquad \text{o} \qquad
            A_1 \sim_c \left( \begin{array}{cccc}
                1 &&& \\
                &1&& \\
                &&-1&\\
                &&&0
            \end{array} \right)
            \end{equation*}

            No obstante, tenemos lo siguiente:
            \begin{equation*}
                g(e_1 - e_4, e_1 - e_4) = 0 + 1 -2\cdot 1 = -1
            \end{equation*}
            Por tanto, vemos que $g$ no es semifidefinida positiva, por lo que $A_1$ tampoco. Por tanto, tenemos que:
            \begin{equation*}
                A_1 \sim_c \left( \begin{array}{cccc}
                    1 &&& \\
                    &1&& \\
                    &&-1&\\
                    &&&0
                \end{array} \right)
            \end{equation*}



            \item \underline{Para $-1<a<1$}:

            Tengo que $|A_a|>0$. Por tanto,
            \begin{gather*}
                A_a \sim_c \left( \begin{array}{cccc}
                1 &&& \\
                &1&& \\
                &&1&\\
                &&&1
            \end{array} \right)
            \quad \text{o} \quad
            A_a \sim_c \left( \begin{array}{cccc}
                -1 &&& \\
                &-1&& \\
                &&-1&\\
                &&&-1
            \end{array} \right)
            \quad \text{o} \\ \text{o} \quad
            A_a \sim_c \left( \begin{array}{cccc}
                1 &&& \\
                &1&& \\
                &&-1&\\
                &&&-1
            \end{array} \right)
            \end{gather*}
            
            Como $g(e_1, e_1=0)$, $g$ no es definida positiva ni negativa. Por tanto, estamos en la tercera opción:
            \begin{equation*}
                A_a \sim_c \left( \begin{array}{cccc}
                    1 &&& \\
                    &1&& \\
                    &&-1&\\
                    &&&-1
                \end{array} \right)
            \end{equation*}



            \item \underline{Para $a=-1$}:
            Tenemos que:
            \begin{equation*}
                A_{-1} = \left( \begin{array}{cccc}
                    0 & 0 & 0 & 1 \\
                    0 & -1 & 1 & 0 \\
                    0 & 1 & -1 & 1 \\
                    1 & 0 & 1 & 1 \\
                \end{array} \right)
            \end{equation*}
            
            Tengo que $|A_{-1}|=0$. Además, 
            \begin{equation*}
                \left| \begin{array}{ccc}
                    -1 & 1 & 0 \\
                    1 & -1 & 1 \\
                    0 & 1 & 1 \\
                \end{array} \right|
                = 1+1-1 = 1
            \end{equation*}

            Por tanto, tengo $rg(A_{-1})=3$, es decir, $Nul(A_{-1}) = 1$.

            Como $g(e_4, e_4)=1$, tenemos que al menos hay un $1$ en la matriz asociada a la base de Sylvester. Consideramos ahora $U=\cc{L}\{e_2, e_3, e_4\}$.
            \begin{equation*}
                A_U=M(g_{|U}; \cc{B}) = \left( \begin{array}{cccc}
                    -1 & 1 & 0 \\
                    1 & -1 & 1 \\
                    0 & 1 & 1 \\
                \end{array} \right)
            \end{equation*}

            Como $|A_U|=1$, y sabemos que $g$ es no degenerada e indefinida, tenemos que:
            \begin{equation*}
                A_U\sim_C \left( \begin{array}{cccc}
                    1 &  &  \\
                     & -1 &  \\
                     &  & -1 \\
                \end{array} \right)
            \end{equation*}
            Por tanto, $Ind(A_U)=2$, por lo que existe un plano $W\subset U$ subespacio de $U=~\cc{L}\{e_2, e_3, e_4\}$ tal que la restricción a ese plano es definida negativa.

            Por tanto, como $U\subset V$, tenemos que existe un plano $W\subset V$ subespacio de $V$ tal que la restricción a ese plano es definida negativa.

            Por tanto, tenemos que $Ind(g)=2$ y:
            \begin{equation*}
                A_{-1} \sim_c \left( \begin{array}{cccc}
                    1 &  &  \\
                     & -1 &  \\
                     &  & -1 \\
                     &&&0
                \end{array} \right)
            \end{equation*}

            \begin{comment}
            Sea ahora $U=\cc{L}\{2e_2-e_3, e_4\}$.
            \begin{equation*}
                g(2e_2-e_3, 2e_2-e_3) = 4a + a -2\cdot 2 = -4 \qquad g(e_4, e_4)=1 \qquad g(e_2-e_3, e_4) = g(e_2, e_4) - g(e_3, e_4) = 0-1 = -1
            \end{equation*}
            Por tanto,
            \begin{equation*}
                M(g, \cc{B}') = \left(\begin{array}{cc}
                    -4 & -1 \\
                    -1 & 1
                \end{array} \right)
            \end{equation*}
            \end{comment}

            \item \underline{Para $a<-1$}:

            Tengo que $|A_{a}|<0$. Como el determinante es un invariante, tenemos que:
            \begin{equation*}
                A_a \sim_c \left( \begin{array}{cccc}
                1 &&& \\
                &-1&& \\
                &&-1&\\
                &&&-1
            \end{array} \right)
            \quad \text{o} \quad
            A_a \sim_c \left( \begin{array}{cccc}
                1 &&& \\
                &1&& \\
                &&1&\\
                &&&-1
            \end{array} \right)
            \end{equation*}            
            
            Además, sea $U=\cc{L}\{e_2, e_3\}$.
            \begin{equation*}
                |a|=a < 0 \qquad
                \left|\begin{array}{cc}
                    a & 1 \\
                    1 & a
                \end{array} \right| = a^2 - 1 > 0 \Longleftrightarrow a^2 > 1 \Longleftrightarrow |a|>1
            \end{equation*}
            Por tanto, tenemos que la restricción de $g$ a $U$ es definida negativa, por lo que $Ind(g)\geq 2$. Por tanto,
            \begin{equation*}
                A_a \sim_c \left( \begin{array}{cccc}
                1 &&& \\
                &-1&& \\
                &&-1&\\
                &&&-1
            \end{array} \right)
            \end{equation*}

            Por tanto,
            \begin{equation}
                \begin{array}{c|c|c}
                    \text{Valor de } a& Ind(g) & Nul(g) \\ \hline
                    a>1 & 1 & 0 \\
                    a=1 & 1 & 1 \\
                    -1<a<1 & 2 & 0 \\
                    a=-1 & 2 & 1 \\
                    a<-1 & 3 & 0
                \end{array}
            \end{equation}
            
        \end{enumerate}

        \item Para cada $a$ encontrar una matriz regular $P$ tal que $P^tAP$ sea diagonal.

        Nos están pidiendo $M(\cc{B}_D; \cc{B})$, con
        $\cc{B}_D=\{\bar{e}_1, \bar{e}_2, \bar{e}_3, \bar{e}_4\}$ base tal que $M(g;\cc{B}_D)$ es diagonal; es decir, $\bar{e_i}\perp \bar{e_j} \quad\forall i,j$.
        \begin{itemize}
            \item \underline{Para $a\neq 0$}:\footnote{Empecé haciendo el razonamiento en función del apartado anterior, con $a>1$. No obstante, vi que este razonamiento es válido $\forall a\neq 0$.}

            Sea $\bar{e}_1 = e_2$, $\bar{e}_2=e_4$. Tenemos que $\bar{e}_1\perp \bar{e_2}$.
            
            
            Necesitamos calcular ahora $\bar{e}_3\in V \mid \bar{e}_3\perp \bar{e}_1\land \bar{e}_3\perp \bar{e}_2$.
            \begin{equation*}\begin{split}
                <\bar{e_1}>^\perp &= \{v \in V \mid g(\bar{e_1},v) = 0\} \\
                &= \left\{ \left(\begin{array}{c}
                     x_1 \\ x_2 \\ x_3 \\ x_4
                \end{array} \right) \in V \mid (0, 1, 0, 0) \left(\begin{array}{cccc}
                    0 & 0 & 0 & 1 \\
                    0 & a & 1 & 0 \\
                    0 & 1 & a & 1 \\
                    1 & 0 & 1 & 1
                \end{array} \right) 
                \left(\begin{array}{c}
                     x_1 \\ x_2 \\ x_3 \\x_4
                \end{array} \right) = 0\right\} \\
                &= \left\{ \left(\begin{array}{c}
                     x_1 \\ x_2 \\ x_3 \\x_4
                \end{array} \right) \in V \mid (0, a, 1, 0)
                \left(\begin{array}{c}
                     x_1 \\ x_2 \\ x_3 \\ x_4
                \end{array} \right) = 0\right\} \\
                &= \left\{ \left(\begin{array}{c}
                     x_1 \\ x_2 \\ x_3 \\ x_4
                \end{array} \right) \in V \mid ax_2+x_3 = 0\right\}
            \end{split}\end{equation*}

            \begin{equation*}\begin{split}
                <\bar{e_2}>^\perp &= \{v \in V \mid g(\bar{e_2},v) = 0\} 
                 \\
                &= \left\{ \left(\begin{array}{c}
                     x_1 \\ x_2 \\ x_3 \\ x_4
                \end{array} \right) \in V \mid (0, 0, 0, 1) \left(\begin{array}{cccc}
                    0 & 0 & 0 & 1 \\
                    0 & a & 1 & 0 \\
                    0 & 1 & a & 1 \\
                    1 & 0 & 1 & 1
                \end{array} \right) 
                \left(\begin{array}{c}
                     x_1 \\ x_2 \\ x_3 \\x_4
                \end{array} \right) = 0\right\} \\
                &= \left\{ \left(\begin{array}{c}
                     x_1 \\ x_2 \\ x_3 \\x_4
                \end{array} \right) \in V \mid (1, 0, 1, 1)
                \left(\begin{array}{c}
                     x_1 \\ x_2 \\ x_3 \\ x_4
                \end{array} \right) = 0\right\} \\
                &= \left\{ \left(\begin{array}{c}
                     x_1 \\ x_2 \\ x_3 \\ x_4
                \end{array} \right) \in V \mid x_1+x_3+x_4 = 0\right\}
            \end{split}\end{equation*}

            Por tanto, sea $\bar{e_3} = (1, 0, 0, -1)^t = e_1 - e_4$.

            Necesitamos calcular ahora $\bar{e}_4$ tal que $ \bar{e}_4\perp \bar{e}_1 \land \bar{e}_4\perp \bar{e}_2 \land \bar{e}_4\perp \bar{e}_3$.
            \begin{equation*}\begin{split}
                <\bar{e_3}>^\perp &= \{v \in V \mid g(\bar{e_3},v) = 0\} 
                = \left\{ \left(\begin{array}{c}
                     x_1 \\ x_2 \\ x_3 \\ x_4
                \end{array} \right) \in V \mid \bar{e_3}^t A_a
                \left(\begin{array}{c}
                     x_1 \\ x_2 \\ x_3 \\x_4
                \end{array} \right) = 0\right\} \\
                &= \left\{ \left(\begin{array}{c}
                     x_1 \\ x_2 \\ x_3 \\ x_4
                \end{array} \right) \in V \mid (1, 0, 0, -1) \left(\begin{array}{cccc}
                    0 & 0 & 0 & 1 \\
                    0 & a & 1 & 0 \\
                    0 & 1 & a & 1 \\
                    1 & 0 & 1 & 1
                \end{array} \right) 
                \left(\begin{array}{c}
                     x_1 \\ x_2 \\ x_3 \\x_4
                \end{array} \right) = 0\right\} \\
                &= \left\{ \left(\begin{array}{c}
                     x_1 \\ x_2 \\ x_3 \\x_4
                \end{array} \right) \in V \mid (-1, 0, -1, 0)
                \left(\begin{array}{c}
                     x_1 \\ x_2 \\ x_3 \\ x_4
                \end{array} \right) = 0\right\} \\
                &= \left\{ \left(\begin{array}{c}
                     x_1 \\ x_2 \\ x_3 \\ x_4
                \end{array} \right) \in V \mid x_1+x_3 = 0\right\}
            \end{split}\end{equation*}

            Por tanto, $v=\left(\begin{array}{c}
                     x_1 \\ x_2 \\ x_3 \\ x_4
                \end{array} \right)$ cumple que:
            $\left\{\begin{array}{l}
                ax_2+x_3=0\\
                x_1+x_3+x_4=0 \\
                x_1+x_3=0
            \end{array} \right.$

            Por tanto, sea $\bar{e_4}=(a, 1, -a, 0)^t$.
            Comprobamos que $\cc{B}_D$ forma base:
            \begin{equation*}
                \left|\begin{array}{cccc}
                    0 & 0 & 1 & a \\
                    1 & 0 & 0 & 1 \\
                    0 & 0 & 0 & -a \\
                    0 & 1 & -1 & 0
                \end{array}\right|
                = -\left|\begin{array}{ccc}
                    0 & 1 & a \\
                    0 & 0 & -a \\
                    1 & -1 & 0
                \end{array}\right| = a \neq 0
            \end{equation*}
            
            \begin{equation*}
                P=M(\cc{B}_D; \cc{B})=\left(\begin{array}{cccc}
                    0 & 0 & 1 & a \\
                    1 & 0 & 0 & 1 \\
                    0 & 0 & 0 & -a \\
                    0 & 1 & -1 & 0
                \end{array} \right)
            \end{equation*}
            
            \item \underline{Para $a=0$}:
            Tenemos que:
            \begin{equation*}
                A_0 = \left(\begin{array}{cccc}
                    0 & 0 & 0 & 1 \\
                    0 & 0 & 1 & 0 \\
                    0 & 1 & 0 & 1 \\
                    1 & 0 & 1 & 1
                \end{array} \right) \sim_c \left( \begin{array}{cccc}
                    1 &&& \\
                    &1&& \\
                    &&-1&\\
                    &&&-1
                \end{array} \right)
            \end{equation*}

            
            Sea $\bar{e}_1 = e_4$. Buscamos $\bar{e_2}\in V \mid \bar{e}_1\perp \bar{e_2}$.
            \begin{equation*}\begin{split}
                <\bar{e_1}>^\perp &= \{v \in V \mid g(\bar{e_1},v) = 0\} = \left\{ \left(\begin{array}{c}
                     x_1 \\ x_2 \\ x_3 \\ x_4
                \end{array} \right) \in V \mid \bar{e_1}^t A_0
                \left(\begin{array}{c}
                     x_1 \\ x_2 \\ x_3 \\x_4
                \end{array} \right) = 0\right\} \\
                &= \left\{ \left(\begin{array}{c}
                     x_1 \\ x_2 \\ x_3 \\ x_4
                \end{array} \right) \in V \mid (0, 0, 0, 1) \left(\begin{array}{cccc}
                    0 & 0 & 0 & 1 \\
                    0 & 0 & 1 & 0 \\
                    0 & 1 & 0 & 1 \\
                    1 & 0 & 1 & 1
                \end{array} \right) 
                \left(\begin{array}{c}
                     x_1 \\ x_2 \\ x_3 \\x_4
                \end{array} \right) = 0\right\} \\
                &= \left\{ \left(\begin{array}{c}
                     x_1 \\ x_2 \\ x_3 \\x_4
                \end{array} \right) \in V \mid (1, 0, 1, 1)
                \left(\begin{array}{c}
                     x_1 \\ x_2 \\ x_3 \\ x_4
                \end{array} \right) = 0\right\} \\
                &= \left\{ \left(\begin{array}{c}
                     x_1 \\ x_2 \\ x_3 \\ x_4
                \end{array} \right) \in V \mid x_1+x_3+x_4 = 0\right\}
            \end{split}\end{equation*}

            Sea $\bar{e}_2 = (1, 1, -1, 0)^t$.
            
            Necesitamos calcular ahora $\bar{e}_3\in V \mid \bar{e}_3\perp \bar{e}_1\land \bar{e}_3\perp \bar{e}_2$.           \begin{equation*}\begin{split}
                <\bar{e_2}>^\perp &= \{v \in V \mid g(\bar{e_2},v) = 0\} = \left\{ \left(\begin{array}{c}
                     x_1 \\ x_2 \\ x_3 \\ x_4
                \end{array} \right) \in V \mid \bar{e_2}^t A_0
                \left(\begin{array}{c}
                     x_1 \\ x_2 \\ x_3 \\x_4
                \end{array} \right) = 0\right\} 
                 \\
                &= \left\{ \left(\begin{array}{c}
                     x_1 \\ x_2 \\ x_3 \\ x_4
                \end{array} \right) \in V \mid (1, 1, -1, 0) \left(\begin{array}{cccc}
                    0 & 0 & 0 & 1 \\
                    0 & 0 & 1 & 0 \\
                    0 & 1 & 0 & 1 \\
                    1 & 0 & 1 & 1
                \end{array} \right) 
                \left(\begin{array}{c}
                     x_1 \\ x_2 \\ x_3 \\x_4
                \end{array} \right) = 0\right\} \\
                &= \left\{ \left(\begin{array}{c}
                     x_1 \\ x_2 \\ x_3 \\x_4
                \end{array} \right) \in V \mid (0, -1, 1, 0)
                \left(\begin{array}{c}
                     x_1 \\ x_2 \\ x_3 \\ x_4
                \end{array} \right) = 0\right\} \\
                &= \left\{ \left(\begin{array}{c}
                     x_1 \\ x_2 \\ x_3 \\ x_4
                \end{array} \right) \in V \mid -x_2+x_3 = 0\right\}
            \end{split}\end{equation*}

            Sea $\bar{e}_3 = (1, 0, 0, -1)^t$.

            Necesitamos calcular ahora $\bar{e}_4$ tal que $ \bar{e}_4\perp \bar{e}_1 \land \bar{e}_4\perp \bar{e}_2 \land \bar{e}_4\perp \bar{e}_3$.
            \begin{equation*}\begin{split}
                <\bar{e_3}>^\perp &= \{v \in V \mid g(\bar{e_3},v) = 0\} 
                = \left\{ \left(\begin{array}{c}
                     x_1 \\ x_2 \\ x_3 \\ x_4
                \end{array} \right) \in V \mid \bar{e_3}^t A_0
                \left(\begin{array}{c}
                     x_1 \\ x_2 \\ x_3 \\x_4
                \end{array} \right) = 0\right\} \\
                &= \left\{ \left(\begin{array}{c}
                     x_1 \\ x_2 \\ x_3 \\ x_4
                \end{array} \right) \in V \mid (1, 0, 0, -1) \left(\begin{array}{cccc}
                    0 & 0 & 0 & 1 \\
                    0 & 0 & 1 & 0 \\
                    0 & 1 & 0 & 1 \\
                    1 & 0 & 1 & 1
                \end{array} \right) 
                \left(\begin{array}{c}
                     x_1 \\ x_2 \\ x_3 \\x_4
                \end{array} \right) = 0\right\} \\
                &= \left\{ \left(\begin{array}{c}
                     x_1 \\ x_2 \\ x_3 \\x_4
                \end{array} \right) \in V \mid (-1, 0, -1, 0)
                \left(\begin{array}{c}
                     x_1 \\ x_2 \\ x_3 \\ x_4
                \end{array} \right) = 0\right\} \\
                &= \left\{ \left(\begin{array}{c}
                     x_1 \\ x_2 \\ x_3 \\ x_4
                \end{array} \right) \in V \mid x_1+x_3 = 0\right\}
            \end{split}\end{equation*}

            Por tanto, $v=\left(\begin{array}{c}
                     x_1 \\ x_2 \\ x_3 \\ x_4
                \end{array} \right)$ cumple que:
            $\left\{\begin{array}{l}
                x_1+x_3+x_4=0 \\
                x_3-x_2=0\\
                x_1+x_3=0
            \end{array} \right.$

            Por tanto, sea $\bar{e_4}=(-1, 1, 1, 0)^t$.
            Comprobamos que $\cc{B}_D$ forma base:
            \begin{equation*}
                \left|\begin{array}{cccc}
                    0 & 1 & 1 & -1 \\
                    0 & 1 & 0 & 1 \\
                    0 & -1 & 0 & 1 \\
                    1 & 0 & -1 & 0
                \end{array}\right|
                = -\left|\begin{array}{ccc}
                    1 & 1 & -1 \\
                    1 & 0 & 1 \\
                    -1 & 0 & 1 \\
                \end{array}\right|
                = \left|\begin{array}{cc}
                    1 & 1 \\
                    -1 & 1 \\
                \end{array}\right| = 2 \neq 0
            \end{equation*}
            
            \begin{equation*}
                P=M(\cc{B}_D; \cc{B})=\left(\begin{array}{cccc}
                    0 & 1 & 1 & -1 \\
                    0 & 1 & 0 & 1 \\
                    0 & -1 & 0 & 1 \\
                    1 & 0 & -1 & 0
                \end{array} \right)
            \end{equation*}
        \end{itemize}

        \begin{observacion}
            También sería válido obtener la base de Sylvester en cada uno de los 5 casos, ya que su matriz asociada es la del teorema del Sylvester, que es diagonal. No obstante, esto no es necesario; ya que el valor de los cuadrados en este caso no nos es relevante.
        \end{observacion}

        

        
    \end{enumerate}
\end{ejercicio}

\begin{ejercicio}
    Responde de forma razonada si las siguientes afirmaciones son verdaderas o falsas:
    \begin{enumerate}
    \renewcommand{\theenumi}{\alph{enumi}}

        \item ¿Es bilineal la aplicación $g : \bb{R}^2 \times \bb{R}^2 \longrightarrow \bb{R}$ dada por $g((x,y), (x',y')) = xy$?

        Veamos si cumple la propiedad del producto por escalares. Sean $k,k'\in \bb{R}$.
        \begin{equation*}
           g(k(x,y), k'(x',y')) = g((kx,ky), (k'x',k'y')) = k(x,y) 
        \end{equation*}

        Por tanto, no es lineal en la segunda variable, por lo que no es bilineal.

        \item Toda forma bilineal $g : \bb{R} \times \bb{R} \longrightarrow \bb{R}$ es de la forma $g(x,y) = axy$ con $a\in \bb{R}$.

        Sabemos que toda forma bilineal $T:V^n(\bb{K}) \times V^n(\bb{K}) \Longrightarrow \bb{K}$ es de la forma:
        \begin{equation*}
            g(x,y)=x^tAy \qquad x,y\in V^n(\bb{K})\quad A\in \cc{M}_n(\bb{K})
        \end{equation*}

        Por tanto, en el caso que estamos tratando, como sabemos que $x^t=x$ por ser $x\in \bb{R}$, y sabiendo que una matriz de orden 1 cuadrada real es un escalar, tenemos que:
        \begin{equation*}
            g(x,y)=xay = axy \qquad a,x,y\in \bb{R}
        \end{equation*}

        Por tanto, el enunciado es \textbf{verdadero}.
        
        \item Si dos matrices cuadradas son congruentes, entonces tienen la misma traza. ¿Deben tener el mismo determinante?

        No, tan solo debe mantenerse el signo del determinante.

        En el ejercicio \ref{Ej1}, vimos que $A_5=\left(\begin{array}{cc}
            1 & 2 \\
            2 & 3
        \end{array}\right)$ era congruente a $A_7=\left(\begin{array}{cc}
            1 & -2 \\
            -2 & -3
        \end{array}\right)$. Sin embargo,
        \begin{equation*}
            |A_5| = 3-4=-1\qquad |A_7|=-3-4=-7
        \end{equation*}
        \begin{equation*}
            tr(A_5)=5 \qquad tr(A_7)=-2
        \end{equation*}
        Por tanto, vemos que no tienen el mismo determinante ni la misma traza, por lo que es \textbf{falso}.

        El determinante, no obstante, debe mantener el signo, ya que si $A_1\sim_c A_2$, es decir, $A_2=P^tA_1P$,
        \begin{equation*}
            |A_2| = |P^tA_1P| = |P||P^t||A_1|=|P|^2|A_1|
        \end{equation*}

        \item En $\bb{R}^2$ existe una métrica tal que $(\mathcal{L}(\{(2, 1)^t\}))^\perp = \mathcal{L}(\{(2,1)^t\})$.

        Sea $\cc{B}=\{e_1,e_2\}$ base $\bb{R}^2$. Supuesto que existe esa métrica, tendrá la forma de:
        \begin{equation*}
            M(g, \cc{B}) = \left( \begin{array}{cc}
                a & b \\
                b & c
            \end{array} \right)
        \end{equation*}

        Como $(2,1)^t \in (\mathcal{L}(\{(2, 1)^t\}))^\perp$, tenemos que:
        \begin{equation*}
            g\left[\left(
            \begin{array}{c}
                2 \\ 1
            \end{array}\right),
            \left(
            \begin{array}{c}
                2 \\ 1
            \end{array}\right)\right] = 0
        \end{equation*}

        Por tanto,
        \begin{multline*}
            0 = g\left[\left(
            \begin{array}{c}
                2 \\ 1
            \end{array}\right),
            \left(
            \begin{array}{c}
                2 \\ 1
            \end{array}\right)\right] =
            (2,1)
            \left( \begin{array}{cc}
                a & b \\
                b & c
            \end{array} \right)
            \left(
            \begin{array}{c}
                2 \\ 1
            \end{array}\right) =
            \left( \begin{array}{cc}
                2a+b & 2b+c \\
            \end{array} \right)
            \left(
            \begin{array}{c}
                2 \\ 1
            \end{array}\right)
            =\\= 4a+2b + 2b + c = 4a + 4b + c= 0
        \end{multline*}

        Supongamos $b=0$ y $a=1$, $c=-4$:
        \begin{equation*}
            M(g, \cc{B}) = \left( \begin{array}{cc}
                1 & 0 \\
                0 & -4
            \end{array} \right)
        \end{equation*}

        Comprobamos este resultado:
        \begin{equation*}\begin{split}
            \left<\left( \begin{array}{c}
           2 \\ 1
        \end{array}\right)\right>^\perp & = \left\{x\in \bb{R}^2 \left|
            (2,1)\left( \begin{array}{cc}
                1 & 0 \\
                0 & -4 \\
            \end{array}\right)
            \left( \begin{array}{c}
                x_1 \\ x_2
            \end{array}\right)
            = 0
            \right.\right\}\\
            & = \left\{x\in \bb{R}^2 \left|
            (2,-4)
            \left( \begin{array}{c}
                x_1 \\ x_2
            \end{array}\right)
            = 0
            \right.\right\} \\
            &= \left\{x\in \bb{R}^2 \left|
            \begin{array}{c}
                x_1 -2 x_2= 0
            \end{array}
            \right.\right\} = \mathcal{L} \left( \left\{ \left(
            \begin{array}{c}
                2 \\ 1
            \end{array}
            \right) \right\}\right)
        \end{split}\end{equation*}

        Por tanto, hemos visto que esa métrica existe, por lo que es \textbf{cierto}.
        

        \item Sea $V$ un espacio vectorial real y $g$ una métrica en $V$. Supongamos que en la diagonal de $M(g,\mathcal{B})$ respecto de una cierta base $\mathcal{B}$ existen dos números $a$ y $b$ con $ab < 0$. Entonces $g$ es indefinida.

        Sea $\cc{B}=\{e_1, \dots, e_n\}$. Si en la diagonal de la matriz asociada a dicha base se encuentran los números $a.b\in \bb{R}$, implica que:
        \begin{equation*}
            g(e_i, e_i) = a \qquad g(e_j, e_j) = b
        \end{equation*}
        Para algún $i,j \mid 1\leq i,j,\leq n,\; i\neq j$.

        Además, como $a,b\in \bb{R}$, $ab < 0$ implica que tienen signo distinto; es decir:
        \begin{equation*}
            a > 0 \land b<0 \qquad \lor \qquad a<0 \land b>0
        \end{equation*}

        Por tanto, $\exists i,j$ tal que:
        \begin{equation*}
            g(e_i, e_i) > 0 \land g(e_j, e_j)<0 \qquad \lor \qquad g(e_i, e_i)<0 \land g(e_j, e_j)>0
        \end{equation*}

        Por tanto, efectivamente, la métrica es indefinida. Por tanto es \textbf{cierto}.

        \item Si una métrica sobre $\bb{R}^2$ está representada en una cierta base por una matriz con determinante negativo entonces se trata de una métrica definida negativa.

        Sea $\cc{B}=\{e_1, e_2\}$ base de $\bb{R}^2$, y sea la matriz asociada a una métrica $g$ en dicha base:
        \begin{equation*}
            A = M(g, \cc{B}) = \left( \begin{array}{cc}
                1 & 0 \\
                0 & -1
            \end{array} \right)
        \end{equation*}

        Tenemos que su determinante es $|A|=-1 < 0$, pero no es definida negativa, ya que $g(e_1, e_1) = 1 \nless 0$.

        Por tanto, es \textbf{falso}.

        \item Sea $V$ un espacio vectorial real de dimensión 3 y $g$ una métrica. Supongamos que existen vectores $u,v \in V$ linealmente independientes, ortogonales entre sí y tales que $g(u,u),g(v,v) < 0$. Entonces, $g$ es degenerada.

        Sea $\cc{B}=\{u, v, e_3\}$ base de $\bb{R}^3$, y sea la matriz asociada a una métrica $g$ en dicha base:
        \begin{equation*}
            A = M(g, \cc{B}) = \left( \begin{array}{ccc}
                -1 & 0 & 0\\
                0 & -1 & 0 \\
                0 & 0 & 1    
            \end{array} \right)
        \end{equation*}

        Podemos ver que $u,v$ son linealmente independientes por formar base, y además son ortogonales ya que $g(u,v)=g(v,0)=0$. Además también tenemos que:
        \begin{equation*}
            g(u,u)= g(v,v) = -1 < 0
        \end{equation*}

        Por tanto, nos encontramos en la situación del enunciado. No obstante, $|A|=-1 \neq 0$, por lo que tenemos que $Ker(g)= \{0\}$, por lo que es no degenerada.

        Por tanto, es \textbf{falso}.

        \item Sea $V$ un espacio vectorial real y $g$ una métrica. Si todos los elementos diagonales de la matriz de $g$ en una cierta base $\mathcal{B}$ son negativos, entonces $g$ es definida negativa.

        Es falso, y ponemos un contraejemplo para $V=\bb{R}^2$.

        Sea $\cc{B}=\{e_1, e_2\}$ base de $\bb{R}^2$, y sea la matriz asociada a una métrica $g$ en dicha base:
        \begin{equation*}
            A = M(g, \cc{B}) = \left( \begin{array}{cc}
                -1 & -1\\
                -1 & -1\\
            \end{array} \right)
        \end{equation*}
        
        Vemos que todos los elementos de la diagonal de $A$ son negativos. No obstante, tenemos que:
        \begin{equation*}
            g(e_1-e_2, e_1-e_2) = g(e_1, e_1) + g(e_2, e_2) -2g(e_1, e_2) = -1 -1 -2(-1) = 0 \nless 0
        \end{equation*}

        Por tanto, tenemos que $g$ no es definida negativa. Es decir, es \textbf{falso}.

        En el caso de que se hubiese dicho que $A$ es una matriz diagonal, entonces sí sería cierto.

        \item Toda métrica indefinida sobre un espacio vectorial real tiene vectores no nulos $u$ tales que $g(u,u) < 0$.

        Sí, ya que esa es la definición. Decimos que $g$ es indefinida si:
        \begin{equation*}
            \exists v,w \in V \mid g(v,v)<0 \quad \land \quad g(w,w)>0
        \end{equation*}

        Por tanto, es \textbf{cierto} por definición.

        \item Dos vectores perpendiculares y no nulos de una métrica $g$ son linealmente independientes. ¿Y si alguno de los dos vectores $u$ verifica $g(u,u) > 0$?

        Supongamos que es falso; es decir, $\exists u,v \in V-\{0\} \mid g(u,v)=0$ con $\{u,v\}$ linealmente dependientes; y busquemos el contraejemplo.
        
        Trabajamos en $\bb{R}^2$ con la base usual $\cc{B}_u=\{e_1, e_2\}$.
        \begin{equation*}
            M(g,\cc{B}_u) = \left(\begin{array}{cc}
                a & b \\
                b & c
            \end{array}\right)
        \end{equation*}
        
        Sea $u=xe_1 + ye_2$. Como son linealmente dependientes, sea $v=ku$ para algún $k\in\bb{R}^\ast$. Forzamos ahora que $g(u,v)=0$:
        \begin{multline*}
            0 = g(u,v) = g(u, ku) = kg(u,u) = kg(xe_1 + ye_2, xe_1 + ye_2) =\\=
            k[g(xe_1, xe_1) +g(ye_2, ye_2) +2g(xe_1, ye_2)]
            = k[x^2a+y^2c+2xyb]
        \end{multline*}

        Por tanto, para el contraejemplo, ponemos $k=2$, $a=y=1$, $x=c=b=0$, es decir,
        \begin{gather*}
            u=ye_2 = e_2 \qquad v=2u=2e_2
            \\
            M(g,\cc{B}_u) = \left(\begin{array}{cc}
                1 & 0 \\
                0 & 0
            \end{array}\right)
        \end{gather*}
        Tenemos que $g(u,v)=g(e_2, 2e_2)=2\cdot 0 = 0$, por lo que son perpendiculares. No obstante, son linealmente dependientes, por lo que el enunciado es \textbf{falso}.


        Respecto al caso de $g(u,u)>0$, es \textbf{cierto} y demostramos mediante reducción al absurdo. Supongamos que $\exists u,v \in V-\{0\} \mid g(u,v)=0$ y $g(u,u)>0\;\lor\;g(v,v)>0$; con $\{u,v\}$ linealmente dependientes.

        Como son linealmente dependientes, sea $v=ku$ para algún $k\in \bb{R}^\ast$.
        
        Como $g(u,u)>0 \;\lor g(v,v)>0$, tenemos que:
        \begin{gather*}
            g(u,u) > 0
            \\ \lor \\
            g(v,v)=g(ku,ku) = k^2g(u,u) >0 \Longleftrightarrow g(u,u) > 0
        \end{gather*}
        Por tanto, tenemos que $g(u,u)>0$. Como son ortogonales, tenemos que:
        \begin{equation*}
            0=g(u,v)=g(u,ku) = kg(u,u) = 0 \Longleftrightarrow g(u,u)=0
        \end{equation*}

        Por tanto, tenemos simultáneamente $g(u,u)=0$ y $g(u,u)>0$, por lo que llegamos a una contradicción y nuestra hipótesis era falsa. Los vectores son linealmente independientes y, si $g(u,u)>0$, es \textbf{cierto}.


        \item Si $g$ es una métrica semidefinida negativa, entonces un vector $v\in V$ verifica $g(v,v)=0$ si y sólo si está en el núcleo de $g$. ¿Y si $g$ es no degenerada pero no semidefinida?

        Supongamos $g$ semidefinida negativa. Queremos demostrar que:
        \begin{equation*}
            g(v,v)=0\Longleftrightarrow v\in Ker(g)
        \end{equation*}
        \begin{description}
            \item [$\Longleftarrow$)] Como $v\in Ker(g)$, tenemos que $g(v,u)=0\;\forall u\in V$, por lo que, en concreto, también se da para $v$. Por tanto, $g(v,v)=0$.

            \item [$\Longrightarrow$)] Supongemos $g(v,v)=0$. Como $g$ es semidefinida negativa, dada la base de Sylvester $\cc{B}_S=\{e_1, \dots, e_s,e_{s+1},\dots, e_n\}$, donde $\{e_{s+1},\dots, e_n\}\subseteq~Ker(g)$,  tenemos:
            \begin{equation*}
                M(g;\cc{B}_S) = \left(\begin{array}{c|c}
                    -I_s & 0 \\ \hline
                    0 & 0_r
                \end{array}\right)
            \end{equation*}
            donde $s=Ind(g)$ y $r=Nul(g)$.

            Como $\cc{B}_S$ es una base, sea $v=a_1e_1 + \dots +a_se_s + a_{s+1}e_{s+1} + \dots + a_ne_n$. Por tanto,
            \begin{multline*}
                0=g(v,v)=(a_1,\dots,a_n)\left(\begin{array}{c|c}
                    -I_s & 0 \\ \hline
                    0 & 0_r
                \end{array}\right)
                \left(\begin{array}{c}
                    a_1 \\ \vdots \\ a_n
                \end{array}\right) =\\= (-a_1,\dots, -a_s,0\dots 0)\left(\begin{array}{c}
                    a_1 \\ \vdots \\ a_n
                \end{array}\right)
                = -a_1^2 - \dots -a_s^2
            \end{multline*}
            Por tanto, tenemos que $0=-a_1^2-\dots -a_s^2$, por lo que $a_1=\dots=a_s=0$; es decir, $v$ es una combinación lineal de los elementos $\{e_{s+1},\dots, e_n\}\subseteq~Ker(g)$, por lo que $v\in Ker(g)$.
        \end{description}

        Por tanto, tenemos que si $g$ es semidefinida negativa, el resultado es \textbf{cierto}. En el caso de que $g$ fuese no degenerada, es \textbf{falso}. Como contraejemplo, dada una base $\cc{B}$, sea la métrica no degenerada
        \begin{equation*}
            M(g;\cc{B}) = \left(\begin{array}{cc}
                1 & 0 \\
                0 & -1 \\
            \end{array} \right)
        \end{equation*}
    
        Tomemos $v=(1,1)^t = e_1 + e_2$.
        \begin{equation*}
            g(v,v) = g(e_1+e_2,e_1+e_2) = g(e_1,e_1) + g(e_2,e_2)+2g(e_1,e_2)=1-1+0 = 0
        \end{equation*}
    
        No obstante, $v\notin Ker(g)=\{0\}$ por ser la matriz asociada regular.


        \item Existe una métrica no degenerada $g$ sobre $\bb{R}^3$ tal que $g_{\left|U\right.}=0$ para cierta recta vectorial $U\subset \bb{R}^3$. ¿Y si $U$ es un plano vectorial?
        
        Sea la métrica representada por la matriz asociada a la base $\cc{B}=\{e_1,e_2,e_3\}$ de la siguiente forma:
        \begin{equation*}
            A=M(g;\cc{B}) = \left( \begin{array}{ccc}
                1 & \\
                 & 1 \\
                 & & -1 \\
            \end{array} \right)
        \end{equation*}

        Sea $\bar{e}=(1,0,1)^t = e_1+e_3$. Tenemos que:
        \begin{equation*}
            g(\bar{e},\bar{e}) = g(e_1+e_3, e_1+e_3) = 1-1+0 = 0
        \end{equation*}

        Consideremos ahora la recta $U=\cc{L}\{\bar{e}\} = \{k\bar{e}\}$. Por ser $g$ una forma bilineal:
        \begin{equation*}
            g(k\bar{e},k'\bar{e}) = kk'\cdot g(\bar{e},\bar{e})=kk' \cdot 0 = 0
        \end{equation*}
        Es decir, sí existe dicha recta $U$, ya que este es un ejemplo de una métrica no degenerada que anula a toda una recta.

        No obstante, si $U$ fuese un plano vectorial, no existiría una métrica no degenerada que cumpliese esas condiciones. Demostramos por reducción al absurdo.

        Supongamos que sí existe. Sea el plano $U=\cc{L}\{e_1, e_2\}$. Ampliamos a $\cc{B}=\{e_1, e_2, e_3\}$ base de $V$. Por tanto, como $g$ anula a todos los elementos de $U\times U$,
        \begin{equation*}
            g(e_1,e_1) = g(e_2,e_2) = g(e_1,e_2) = 0
        \end{equation*}
        
        Por tanto,
        \begin{equation*}
            M(g;\cc{B})= \left(\begin{array}{ccc}
                0 & 0 & a \\
                0 & 0 & b \\
                a & b & c
            \end{array}\right)
        \end{equation*}

        No obstante, $|M(g;\cc{B})| = 0 \Longrightarrow g$ es degenerada. Por lo que hemos llegado a una contradicción. Por tanto, para un plano es \textbf{falso}.

        \item Existe en $\bb{R}^4$ una métrica no degenerada tal que $g_{\left|U\right.}=0$ y $g_{\left|V\right.}=0$ donde:
        \begin{equation*}
            U=\left\{(x,y,z,t)\in \bb{R}^4 \mid x+y=0, \quad z+t=0 \right\} = \cc{L} \left\{
            \left(\begin{array}{r}
                 1 \\ -1 \\ 0 \\ 0
            \end{array} \right),
            \left(\begin{array}{r}
                  0 \\ 0 \\ 1 \\ -1
            \end{array} \right)
            \right\}
        \end{equation*}
        \begin{equation*}
            V=\left\{(x,y,z,t)\in \bb{R}^4 \mid x-y=0, \quad z-t=0 \right\} = \cc{L} \left\{
            \left(\begin{array}{r}
                 1 \\ 1 \\ 0 \\ 0
            \end{array} \right),
            \left(\begin{array}{r}
                  0 \\ 0 \\ 1 \\ 1
            \end{array} \right)
            \right\}
        \end{equation*}

        veamos que $\left\{
            \left(\begin{array}{r}
                 1 \\ -1 \\ 0 \\ 0
            \end{array} \right),
            \left(\begin{array}{r}
                  0 \\ 0 \\ 1 \\ -1
            \end{array} \right),
            \left(\begin{array}{r}
                 1 \\ 1 \\ 0 \\ 0
            \end{array} \right),
            \left(\begin{array}{r}
                  0 \\ 0 \\ 1 \\ 1
            \end{array} \right)
            \right\}$ son linealmente independientes.

        \begin{equation*}
            \left|\begin{array}{cccc}
                1 & 1 & 0 & 0 \\
                -1 & 1 & 0 & 0 \\
                0 & 0 & 1 & 1 \\
                0 & 0 & -1 & 1
            \end{array} \right|=
            \left|\begin{array}{cc}
                1 & 1  \\
                -1 & 1 \\
            \end{array} \right|
            \left|\begin{array}{cc}
                1 & 1  \\
                -1 & 1 \\
            \end{array} \right| = 2\cdot 2 = 4 \neq 0
        \end{equation*}

        Por tanto, 4 vectores linealmente independientes en $\bb{R}^4$ forman base. Por tanto, sea la base $$\cc{B} = \{e_1,e_2,e_3,e_4\} = \left\{
            \left(\begin{array}{r}
                 1 \\ -1 \\ 0 \\ 0
            \end{array} \right),
            \left(\begin{array}{r}
                  0 \\ 0 \\ 1 \\ -1
            \end{array} \right),
            \left(\begin{array}{r}
                 1 \\ 1 \\ 0 \\ 0
            \end{array} \right),
            \left(\begin{array}{r}
                  0 \\ 0 \\ 1 \\ 1
            \end{array} \right)
            \right\}$$

        Supongamos ahora que sí existe esa métrica $g$. Entonces,
        \begin{equation*}
            A = M(g, \cc{B}) = \left(\begin{array}{cccc}
                0 & 0 & a & b \\
                0 & 0 & c & d \\
                a & c & 0 & 0 \\
                b & d & 0 & 0
            \end{array} \right)
        \end{equation*}

        Nos falta comprobar que $g$ es no degenerada.
        \begin{equation*}
            |A| = \left|\begin{array}{cccc}
                0 & 0 & a & b \\
                0 & 0 & c & d \\
                a & c & 0 & 0 \\
                b & d & 0 & 0
            \end{array} \right|
            = \left|\begin{array}{cc}
                a & b  \\
                c & d \\
            \end{array} \right|
            \left|\begin{array}{cc}
                a & c  \\
                b & d \\
            \end{array} \right| = (ad -bc)(ad -bc) = (ad-bc)^2 \neq 0 \Longleftrightarrow ad \neq bc
        \end{equation*}

        Sea, por ejemplo, $a=d=0$, $b=c=1$.
        \begin{equation*}
            A = M(g, \cc{B}) = \left(\begin{array}{cccc}
                0 & 0 & 0 & 1 \\
                0 & 0 & 1 & 0 \\
                0 & 1 & 0 & 0 \\
                1 & 0 & 0 & 0
            \end{array} \right)
        \end{equation*}
        Podemos ver que $g$ es no degenerada, y que, dados $\cc{L}\{e_1, e_2\} = U$ y $\cc{L}\{e_3, e_4\} =~V$ se cumple que $g_{\left|U\right.}=0$ y $g_{\left|V\right.}=0$. Por tanto, es \textbf{cierto}.

        \item Dadas $M$ y $N$ dos matrices simétricas de orden $20$ y de rango $1$ se tiene que $M$ y $N$ son congruentes si y sólo si $tr(M)\cdot tr(N)>0$.

        Como el rango es 1, sabemos que:
        \begin{equation*}
            M \sim_c \left(\begin{array}{cccc}
                a \\
                &0 \\
                && \ddots \\
                &&&0
            \end{array} \right)
            \quad
            N \sim_c \left(\begin{array}{cccc}
                b \\
                &0 \\
                && \ddots \\
                &&&0
            \end{array} \right) \qquad a,b\neq 0
        \end{equation*}
        donde $a,b\neq 0$. Por tanto, tenemos que $M,N$ son semidefinidas positivas o negativas.

        Tenemos que $A=(a_{ij})_{i,j}$ matriz asociada a la métrica $g_a$ sea semidefinida positiva implica que $a_{ii}\geq 0\;\forall i,j$ y $a_{ii}=0=g_a(e_i,e_i)\Longleftrightarrow e_i\in Ker(g_a)$. Como el rango es 1, tenemos al menos uno de los elementos de la diagonal es no nulo, por lo que$tr(A)\neq 0$. Por tanto,
        \begin{gather*}
            A \text{ semidefinida positiva} \Longrightarrow tr(A)> 0 \\
            B \text{ semidefinida negativa} \Longrightarrow tr(B)< 0
        \end{gather*}

        Considerando el recíproco de cada una, tenemos que la implicación va en ambos sentidos, no en uno solo. Por tanto,
        \begin{gather*}
            A \text{ semidefinida positiva} \Longleftrightarrow tr(A)> 0 \\
            B \text{ semidefinida negativa} \Longleftrightarrow tr(B)< 0
        \end{gather*}

        Además,
        \begin{gather*}
            M \text{ semidefinida positiva} \Longleftrightarrow a>0 \qquad
            M \text{ semidefinida negativa} \Longleftrightarrow a<0 \\
            N \text{ semidefinida positiva} \Longleftrightarrow b>0 \qquad
            N \text{ semidefinida negativa} \Longleftrightarrow b<0 \\
        \end{gather*}
        
        \begin{description}
            \item [$\Longrightarrow$)] Supongamos $M\sim_c N$. Como ser congruente es una relación de equivalencia, tenemos que $ab>0$.

            Por tanto, tenemos que:
            \begin{equation*}
                \begin{array}{c}
                    a,b>0 \Longrightarrow 
                    M,N\text{ semidef. positivas}
                    \Longrightarrow
                    tr(M),tr(N)> 0 
                    \Longrightarrow tr(M)tr(N)> 0
                    \\
                    \lor \\
                    a,b<0 \Longrightarrow 
                    M,N\text{ semidef. negativas}
                    \Longrightarrow
                    tr(M),tr(N)< 0 
                    \Longrightarrow tr(M)tr(N)> 0
                \end{array}
            \end{equation*}

            En ambos casos, tenemos que $tr(M)tr(N)> 0$.



            \item [$\Longleftarrow$)] Supongamos $tr(M)tr(N)>0$. Entonces, tenemos dos opciones:
            \begin{itemize}
                \item \underline{$tr(M),tr(N)>0$}: Entonces tenemos que $M,N$ semidefinidas positivas, por lo que $a,b>0$.

                \item \underline{$tr(M),tr(N)<0$}: Entonces tenemos que $M,N$ semidefinidas negativas, por lo que $a,b<0$
            \end{itemize}

            Por tanto, tenemos que $a,b$ tienn el mismo signo. Como la congruencia es una relación de equivalencia, tenemos que $M\sim_cN$.
        \end{description}

        \item Dado $(V,g)$ un espacio vectorial métrico tal que para todo subespacio $U$ de $V$ se tiene $\dim(V) = \dim(U) + \dim(U^\perp)$ entonces $g$ es no degenerada. \label{Ej6.ñ}

        Por la condición del enunciado, tenemos que:
        \begin{equation}\label{Ej6.ñ.Dim}
            \dim(V) = \dim(U) + \dim(U^\perp) \qquad \forall U\subseteq V \text{ subesp. vectorial}.
        \end{equation}

        El núcleo de $g$ está definido como;
        \begin{equation*}
            Ker(g)=\{v\in V \mid g(u,v)=0 \qquad \forall u\in V\}
        \end{equation*}

        Además, dado el subespacio $U$, su ortogonal se define como:
        \begin{equation*}
            U^\perp=\{v\in V \mid g(u,v)=0 \qquad \forall u\in U\}
        \end{equation*}

        De las respectivas definiciones, se tiene fácilmente que:
        \begin{equation} \label{Ej6.ñ.Ker}
            Ker(g)\subseteq U^\perp \qquad \forall U\subseteq V\text{ subesp. vectorial}.
        \end{equation}

        Como se tiene que es válido $\forall U\subseteq V$ subespacio vectorial, consideramos $U=~V$. Por la ecuación \ref{Ej6.ñ.Dim}, tenemos que $\dim U^\perp = 0$, por lo que $U^t=\{0\}$. Por la ecuación \ref{Ej6.ñ.Ker}, tenemos que $Ker(g)\subseteq U^\perp=\{0\}$, por lo que $Ker(g)=\{0\}$, teniendo así que $g$ es no degenerada. Por tanto, es \textbf{cierto}.

        \item Una matriz simétrica $A$ es semidefinida positiva si y sólo si existe una matriz cuadrada $Q$ tal que $A=Q^t\cdot Q$.

        \begin{description}
            \item [$\Longrightarrow$)] Realizamos distinción de casos:
            \begin{itemize}
                \item Supongamos $A$ es la matriz dada por el Teorema de Sylvester.
                
                Definimos $Q=A$. Por ser $g$ semidefinida positiva, tenemos que:
                \begin{equation*}
                    Q=A=\left(\begin{array}{c|c}
                        I & 0 \\ \hline
                        0 & 0
                    \end{array}\right)
                \end{equation*}
    
                Por tanto,
                \begin{equation*}
                    QQ^t=AA^t=A^2=\left(\begin{array}{c|c}
                        I & 0 \\ \hline
                        0 & 0
                    \end{array}\right)
                    \left(\begin{array}{c|c}
                        I & 0 \\ \hline
                        0 & 0
                    \end{array}\right) = \left(\begin{array}{c|c}
                        I & 0 \\ \hline
                        0 & 0
                    \end{array}\right) = A
                \end{equation*}
    
                Por tanto, si $A$ es la matriz dada por el Teorema de Sylvester, se cumple.
    
                \item Supongamos ahora que $A$ no es la matriz dada por el Teorema de Sylvester.
                
                Por el Teorema de Sylvester, tenemos que:
                \begin{equation*}
                    A\sim_c \left(\begin{array}{c|c}
                        I & 0 \\ \hline
                        0 & 0
                    \end{array}\right) = S \Longrightarrow \exists P\mid A=P^tSP 
                \end{equation*}

                Por lo anterior demostrado, tenemos que $S=(Q')^tQ'$. Por tanto,
                \begin{equation*}
                    A=P^t(Q')^tQ'P
                \end{equation*}

                Definiendo $Q=Q'P$, tenemos que $A=Q^tQ$, por lo que se cumple.
            \end{itemize}

            \item [$\Longleftarrow$)] Suponemos que $\exists Q\mid A=Q^tQ$.

            Sea $v\in V$ cuyas coordenadas en cierta base son $x=(x_1, \dots, x_n)^t\in \bb{K}^n$
            \begin{equation*}
                g(v,v) = x^tAx = x^tQ^tQx = (Qx)^t(Qx)
            \end{equation*}

            Como $x\in \bb{K}^n$, tengo que $Qx\in \bb{K}^n$. Por tanto, tengo que:
            \begin{equation*}
                g(v,v) = (Qx)^t(Qx) = \left<Qx,Qx\right> \geq 0
            \end{equation*}
            donde he usado que $<;>$ es definida positiva y que $M(<;>\;;\cc{B}_u)=I$.

            Por tanto, tengo que $g(v,v)\geq 0\;\forall v\in V$, por lo que $g$ es semidefinida positiva. No podemos asegurar que sea definida positiva, ya que al no ser $Q$ regular cabe la posibilidad de que $Qx=0$.
            
            
        \end{description}

        \item Si $(V,g)$ es un espacio vectorial métrico tal que para todo subespacio $U$ de $V$ de dimensión mayor o igual que $1$ se tiene $\dim(V) = \dim(U) + \dim(U^\perp)$, entonces $g$ es no degenerada.

        Se ha demostrado \textbf{cierto} en el apartado \ref{Ej6.ñ}.

        \item Existe una métrica degenerada en $\bb{R}^4$ tal que el ortogonal a la recta generada por el vector $v = (1,1,1,1)$ es la propia recta.

        Supongamos que sí existe. Dada $\cc{B}$ una base de $\bb{R}^4$, tenemos que:
        \begin{equation*}
            A = M(g, \cc{B}) = \left( \begin{array}{cccc}
                 a_{11} & a_{12} & a_{13} & a_{14} \\
                 a_{12} & a_{22} & a_{23} & a_{24} \\
                 a_{13} & a_{23} & a_{33} & a_{34} \\
                 a_{14} & a_{24} & a_{34} & a_{44}
            \end{array}\right)
        \end{equation*}

        Calculamos por tanto el espacio ortogonal a la recta.
        
        \begin{equation*}\begin{split}
            \left<\left( \begin{array}{c}
           1 \\ 1 \\ 1 \\ 1
        \end{array}\right)\right>^\perp & = \left\{x\in \bb{R}^4 \left|
            (1,1,1,1)\left( \begin{array}{cccc}
                a_{11} & a_{12} & a_{13} & a_{14} \\
                a_{12} & a_{22} & a_{23} & a_{24} \\
                a_{13} & a_{23} & a_{33} & a_{34} \\
                a_{14} & a_{24} & a_{34} & a_{44}
            \end{array}\right)
            \left( \begin{array}{c}
                x_1 \\ x_2 \\ x_3 \\ x_4
            \end{array}\right)
            = 0
            \right.\right\}\\
            & = \left\{x\in \bb{R}^4 \left|
            (b_1, b_2, b_3, b_4)
            \left( \begin{array}{c}
                x_1 \\ x_2 \\ x_3 \\ x_4
            \end{array}\right)
            = 0
            \right.\right\} \\
            &= \left\{x\in \bb{R}^4 \left|
            \begin{array}{c}
                b_1x_1 + b_2x_2 + b_3x_3 + b_4x_4= 0
            \end{array}
            \right.\right\}
        \end{split}\end{equation*}

        Por tanto, podemos ver que, de existir la métrica, el ortogonal a la recta es un hiperplano o el mismo $\bb{R}^4$, en contradicción con que el ortogonal a la recta sea una recta. Por tanto, no existe dicha métrica.

        Es, por tanto, \textbf{falso}.
    \end{enumerate}
\end{ejercicio}

\begin{ejercicio}
    Sea $\bb{R}_2[x]$ espacio vectorial y $g$ una aplicación definida como:
    $$g(p,q) = p(0)q(0) + p(1)q(1) \qquad \forall p,q \in \bb{R}_2[x]$$

    \begin{enumerate}
        \item \underline{¿Es $g$ una métrica?}\\
        Veamos en primer lugar que es bilineal.

        En primer lugar, verifico que cumple la condición del producto por escalares en las dos variables.
        $$g(kp, k'q) = kk'p(0)q(0) + kk'p(1)q(1) = kk' g(p,q)$$

        Ahora verificamos la suma de vectores en cada una de las variables.
        $$g(p+p', q) = (p(0)+p'(0))q(0) + (p(1) + p'(1))q(1) = g(p,q) + g(p', q)$$
        $$g(p,q+q') = p(0)(q(0)+q'(0)) + p(1)(q(1)+q'(1)) = g(p,q) + g(p, q')$$

        Veamos ahora que es simétrica. Para ello, sea $\mathcal{B}=\{1,x,x^2\}$ base de $\bb{R}_2[x]$:
        \begin{equation*}
            \begin{array}{l}
                g(1,1)=1+1=2 \\
                g(1,x)=1 \\
                g(1,x^2)= 1\\
            \end{array}
            \quad
            \begin{array}{l}
                g(x,1)=1 \\
                g(x,x)=1 \\
                g(x,x^2)= 1\\
            \end{array}
            \quad
            \begin{array}{l}
                g(x^2,1)=1 \\
                g(x^2,x)=1 \\
                g(x^2,x^2)= 1\\
            \end{array}
        \end{equation*}
        Por tanto, como podemos ver, la matriz asociada a $g$ sí es simétrica.
        \begin{equation*}
            A = M(g, \mathcal{B}) = \left( \begin{array}{ccc}
                2 & 1 & 1 \\
                1 & 1 & 1 \\
                1 & 1 & 1 \\
            \end{array}\right)
        \end{equation*}
        Por tanto, $g$ sí es una métrica.

        Alternativamente, para razonar que es simétrica, se podría haber hecho lo siguiente:
        \begin{equation*}
            g(p,q) = p(0)q(0) + p(1)q(1) = q(0)p(0) + q(1)p(1) = g(q,p)
        \end{equation*}
        donde he aplicado la conmutatividad del producto en $\bb{R}$. No obstante, he optado por la primera opción ya que es necesario para el apartado siguiente.

        \item \underline{Calcular la base de sylvester.}
        
        Habiendo definido $A=M(g,\cc{B})$, vemos que $rg(A)=2 \Longrightarrow Nul(g)=1$. Por tanto, busco $e_1\in Ker(g)$.
        
        \begin{equation*}\begin{split}
            Ker(g) & = \left\{x\in \bb{R}_2[x] \left|
            \left( \begin{array}{ccc}
                2 & 1 & 1 \\
                1 & 1 & 1 \\
                1 & 1 & 1 \\
            \end{array}\right)
            \left( \begin{array}{c}
                x_1 \\ x_2 \\ x_3
            \end{array}\right)
            = 0
            \right.\right\}\\
            &= \left\{x\in \bb{R}_2[x] \left|
            \begin{array}{c}
                2x_1 + x_2 + x_3 = 0 \\
                x_1 + x_2 + x_3 = 0
            \end{array}
            \right.\right\} = \mathcal{L} \left( \left\{ \left(
            \begin{array}{c}
                0 \\ 1 \\ -1
            \end{array}
            \right) \right\}\right)
        \end{split}\end{equation*}

        Sea $e_1 = (0,1,-1)^T = x-x^2$.
        
        Elijo ahora $e_2\in \bb{R}_2[x]$ de cuadrado no nulo. Sea $e_2 = x = (0,1,0)^t$.

        Busco ahora $e_3 \in <e_2>^\perp$ de cuadrado no nulo:
        \begin{equation*}\begin{split}
            <e_2>^\perp & = \left\{x\in \bb{R}_2[x] \left|
            (0, 1, 0)\left( \begin{array}{ccc}
                2 & 1 & 1 \\
                1 & 1 & 1 \\
                1 & 1 & 1 \\
            \end{array}\right)
            \left( \begin{array}{c}
                x_1 \\ x_2 \\ x_3
            \end{array}\right)
            = 0
            \right.\right\}\\
            & = \left\{x\in \bb{R}_2[x] \left|
            (1, 1, 1)
            \left( \begin{array}{c}
                x_1 \\ x_2 \\ x_3
            \end{array}\right)
            = 0
            \right.\right\} \\
            &= \left\{x\in \bb{R}_2[x] \left|
            \begin{array}{c}
                x_1 + x_2 + x_3 = 0
            \end{array}
            \right.\right\} = \mathcal{L} \left( \left\{ \left(
            \begin{array}{c}
                -1 \\ 1 \\ 0
            \end{array}
            \right) \right\}\right)
        \end{split}\end{equation*}

        Tomo $e_3 = (-1,1,0)^T = -1 + x$.
        \begin{equation*}
            g(e_3, e_3) = g(-1+x, -1+x) = g(1,1) + g(x,x) -2g(1,x) = 2+1-2 = 1
        \end{equation*}

        Por tanto, dado $\mathcal{B}_S = \{e_1, e_2, e_3\} = \{x-x^2, x, -1+x\}$ base de Sylvester, tenemos:
        \begin{equation*}
            M(g, \mathcal{B}_S) = \left( \begin{array}{ccc}
                0 &  &  \\
                 & 1 &  \\
                 &  & 1 \\
            \end{array}\right)
        \end{equation*}

        \item Supongamos $g(p,q) = ap(0)q(0) + p(1)q(1)$, con $a\in \bb{R}$. Calcular la nulidad y el índice en función de $a$.

        Calculamos en primer lugar $M(g; \bb{B})$, con $\mathcal{B}=\{1,x,x^2\}$ base de $\bb{R}_2[x]$:
        \begin{equation*}
            \begin{array}{l}
                g(1,1)=a+1 \\
                g(1,x)=1 \\
                g(1,x^2)= 1\\
            \end{array}
            \quad
            \begin{array}{l}
                g(x,1)=1 \\
                g(x,x)=1 \\
                g(x,x^2)= 1\\
            \end{array}
            \quad
            \begin{array}{l}
                g(x^2,1)=1 \\
                g(x^2,x)=1 \\
                g(x^2,x^2)= 1\\
            \end{array}
        \end{equation*}
        Por tanto,
        \begin{equation*}
            A = M(g, \mathcal{B}) = \left( \begin{array}{ccc}
                a+1 & 1 & 1 \\
                1 & 1 & 1 \\
                1 & 1 & 1 \\
            \end{array}\right)
        \end{equation*}

        Calculamos, en primer lugar, el rango de $A$:
        \begin{equation*}
            rg(A) = \left\{ \begin{array}{ccc}
            1 & \text{si} & a=0 \\
            2 & \text{si} & a\neq 0 \\
            \end{array} \right.
        \end{equation*}

        \begin{itemize}
            \item \underline{Para $a=0$}:

            Tenemos que $Nul(g)=2$. Por tanto,
            \begin{equation*}
                A \sim_c \left(\begin{array}{ccc}
                    1 &  \\
                    & 0 \\
                    && 0 
                \end{array} \right)
                \qquad \text{o} \qquad
                A \sim_c \left(\begin{array}{ccc}
                    -1 &  \\
                    & 0 \\
                    && 0 
                \end{array} \right)
            \end{equation*}

            Sea ahora $U=\cc{L}\{e_3\}$. Tenemos que $g_{\left|U \right.}$ es definida positiva, por lo que $k\geq 1$, con $k$ el número de 1 de la matriz asociada a la base de Sylvester. Por tanto, estamos en el primer caso.
            \begin{equation*}
                A \sim_c \left(\begin{array}{ccc}
                    1 &  \\
                    & 0 \\
                    && 0 
                \end{array} \right)
            \end{equation*}

            Por tanto, para $a=0$, tenemos $Nul(g)=2$, $Ind(g)=0$.

            \item \underline{Para $a\neq0$}:

            Tenemos que $Nul(g)=1$. Además, dado ea $U=\cc{L}\{e_3\}$, tenemos que $g_{\left|U \right.}$ es definida positiva, por lo que $k\geq 1$, con $k$ el número de 1 de la matriz asociada a la base de Sylvester. Por tanto,
            \begin{equation*}
                A \sim_c \left(\begin{array}{ccc}
                    1 &  \\
                    & 1 \\
                    && 0 
                \end{array} \right)
                \qquad \text{o} \qquad
                A \sim_c \left(\begin{array}{ccc}
                    1 &  \\
                    & -1 \\
                    && 0 
                \end{array} \right)
            \end{equation*}

            \begin{itemize}
                \item \underline{Para $a>0$}:
                Sea $U'=\cc{L}\{e_1, e_2\}$. Veamos que $M(g_{\left|U \right.}, \cc{B})$ es definida positiva:
                \begin{gather*}
                    M(g_{\left|U \right.}, \cc{B}) = \left( \begin{array}{cc}
                        a+1 & 1 \\
                        1 & 1
                    \end{array} \right) \\
                    |a+1|=a+1 > 0 \qquad \left| \begin{array}{cc}
                        a+1 & 1 \\
                        1 & 1
                    \end{array} \right| = a+1 -1 = a>0
                \end{gather*}
                
                Por tanto, $g_{\left|U \right.}$ es definida positiva, por lo que $k \geq 2$, siendo $k$ el número de 1 en la matriz asociada a la Base de Sylvester. Por tanto, estamos en el primer caso.
                \begin{equation*}
                    A \sim_c \left(\begin{array}{ccc}
                        1 &  \\
                        & 1 \\
                        && 0 
                    \end{array} \right)
                \end{equation*}
    
                Por tanto, para $a>0$, tenemos $Nul(g)=1$, $Ind(g)=0$.

                \item \underline{Para $a<0$}:

                Tenemos que:
                \begin{equation*}
                    g(e_1-e_2, e_1-e_2) = g(e_1,e_1) + g(e_2,e_2) -2g(e_1, e_2) = a+1 +1 -2 = a <0
                \end{equation*}

                Por tanto, $g$ no está definida positiva para $a<0$. Por tanto, estamos en el caso de la derecha:
                \begin{equation*}
                        A \sim_c \left(\begin{array}{ccc}
                            1 &  \\
                            & -1 \\
                            && 0 
                        \end{array} \right)
                    \end{equation*}
                
            \end{itemize}

            
        \end{itemize}

        Por tanto, tenemos que:
        \begin{equation*}
            \begin{array}{c|c|c}
                \text{Caso} & Ind(g) & Nul(g) \\ \hline
                a < 0 & 1 & 1 \\
                a=0 & 0 & 2 \\
                a > 0 & 0 & 1 \\           
            \end{array}
        \end{equation*}
        
    \end{enumerate}
\end{ejercicio}

\begin{ejercicio}
    Sea $V^2(\bb{K})$ e.v. y sea $\mathcal{B} = \{e_1, e_2\}$ base de $V$. Sea $T$ una forma bilineal simétrica donde
    \begin{equation*}
        A=M(T;\mathcal{B}) = \left( \begin{array}{cc}
            1 & a \\
            a & 1
        \end{array} \right),\quad a\in \bb{K}
    \end{equation*}
    Obtener la base de Sylvester.\\

    En primer lugar, estudio la nulidad de la forma bilineal:
    $$det(A)=1-a^2 = 0 \Longleftrightarrow a = \pm 1$$
    \begin{itemize}
        \item \underline{Caso complejo:}
        \begin{itemize}
            \item $a\neq \pm 1 \Longrightarrow Nul(T)=0$

            Buscamos $\bar{\mathcal{B}} = \{\bar{e_1}, \bar{e_2}\}$. Tomamos $\bar{e_1} = \left( \begin{array}{c}
                1 \\ 0
            \end{array}\right) = e_1$ y $\bar{e_2} \in <\bar{e_1}>^\perp$.
            \begin{equation*}
                <\bar{e_1}>^\perp = \left\{ x\in V^2 \mid (1,0) \left( \begin{array}{cc}
                    1 & a \\
                    a & 1
                \end{array} \right) \left( \begin{array}{c}
                    x_1 \\ x_2
                \end{array}\right) = 0\right\} = \left\{ x \in V^2 \mid x_1 + ax_2 = 0\right\}
            \end{equation*}
            Por tanto, $\bar{e_2}=\left( \begin{array}{c}
                a \\ -1
            \end{array}\right) = ae_1-e_2$.

            Como $\bar{e_2} \in <\bar{e_1}>^\perp \Longrightarrow T(\bar{e_1}, \bar{e_2}) = T(\bar{e_2}, \bar{e_1}) = 0$. Veamos ahora el valor de los cuadrados.
            \begin{equation*}
                T(\bar{e_1}, \bar{e_1}) = (1,0) \left( \begin{array}{cc}
                    1 & a \\
                    a & 1
                \end{array} \right) \left( \begin{array}{c}
                    1 \\ 0
                \end{array}\right) = (1,a)\left( \begin{array}{c}
                    1 \\ 0
                \end{array}\right) = 1
            \end{equation*}
            \begin{equation*}
                T(\bar{e_2}, \bar{e_2}) = (a, -1) \left( \begin{array}{cc}
                    1 & a \\
                    a & 1
                \end{array} \right) \left( \begin{array}{c}
                    a \\ -1
                \end{array}\right) = (0,a^2-1)\left( \begin{array}{c}
                    a \\ -1
                \end{array}\right) = 1-a^2
            \end{equation*}
            \begin{equation*}
                M(T;\bar{\mathcal{B}}) = \left( \begin{array}{cc}
                    1 & 0 \\
                    0 & 1-a^2
                \end{array} \right)
            \end{equation*}

            Considero ahora $\lambda = \frac{1}{\sqrt{1-a^2}}$. Sea $\bar{\bar{B}} = \{\bar{e_1}, \bar{\bar{e_2}}\}$, con $\bar{\bar{e_2}} = \lambda \bar{e_2}$.
            \begin{equation*}
                T(\bar{\bar{e_2}}, \bar{\bar{e_2}}) = T(\lambda \bar{e_2}, \lambda \bar{e_2}) = \lambda^2 T(\bar{e_2}, \bar{e_2}) = \lambda^2 (1-a^2) = 1
            \end{equation*}
            \begin{equation*}
                M(T;\bar{\bar{\mathcal{B}}}) = \left( \begin{array}{cc}
                    1 & 0 \\
                    0 & 1
                \end{array} \right) \qquad
                P = \left( \begin{array}{cc}
                    1 & \frac{a}{\sqrt{1-a^2}} \\
                    0 & \frac{-1}{\sqrt{1-a^2}}
                \end{array} \right)
            \end{equation*}

            \item $a=\pm 1 \Longrightarrow Nul(T)=1$ \\

            Tomamos $\bar{\mathcal{B}} = \{\bar{e_1}, \bar{e_2}\}$, con $\bar{e_1} = (1,0)^T$ y $\bar{e_2} = (a, -1)^T$.
            \begin{equation*}
                M(T;\bar{\mathcal{B}}) = \left( \begin{array}{cc}
                    1 & 0 \\
                    0 & 0
                \end{array} \right) \qquad
                P = \left( \begin{array}{cc}
                    1 & a \\
                    0 & -1
                \end{array} \right)
            \end{equation*}
        \end{itemize}

        \item \underline{Caso real:}
        \begin{itemize}
            \item $a\neq \pm 1 \Longrightarrow Nul(T)=0$

            Buscamos $\bar{\mathcal{B}} = \{\bar{e_1}, \bar{e_2}\}$. Tomamos $\bar{e_1} = \left( \begin{array}{c}
                1 \\ 0
            \end{array}\right)$ y $\bar{e_2} \in <\bar{e_1}>^\perp$.
            \begin{equation*}
                <\bar{e_1}>^\perp = \left\{ x\in V^2 \mid (1,0) \left( \begin{array}{cc}
                    1 & a \\
                    a & 1
                \end{array} \right) \left( \begin{array}{c}
                    x_1 \\ x_2
                \end{array}\right) = 0\right\} = \left\{ x \in V^2 \mid x_1 + ax_2 = 0\right\}
            \end{equation*}
            Por tanto, $\bar{e_2}=\left( \begin{array}{c}
                a \\ -1
            \end{array}\right)$.

            Como $\bar{e_2} \in <\bar{e_1}>^\perp \Longrightarrow T(\bar{e_1}, \bar{e_2}) = T(\bar{e_2}, \bar{e_1}) = 0$. Veamos ahora el valor de los cuadrados.
            \begin{equation*}
                T(\bar{e_1}, \bar{e_1}) = (1,0) \left( \begin{array}{cc}
                    1 & a \\
                    a & 1
                \end{array} \right) \left( \begin{array}{c}
                    1 \\ 0
                \end{array}\right) = (1,a)\left( \begin{array}{c}
                    1 \\ 0
                \end{array}\right) = 1
            \end{equation*}
            \begin{equation*}
                T(\bar{e_2}, \bar{e_2}) = (a, -1) \left( \begin{array}{cc}
                    1 & a \\
                    a & 1
                \end{array} \right) \left( \begin{array}{c}
                    a \\ -1
                \end{array}\right) = (0,a^2-1)\left( \begin{array}{c}
                    a \\ -1
                \end{array}\right) = 1-a^2
            \end{equation*}
            \begin{equation*}
                M(T;\bar{\mathcal{B}}) = \left( \begin{array}{cc}
                    1 & 0 \\
                    0 & 1-a^2
                \end{array} \right)
            \end{equation*}

            \begin{itemize}
                \item $a^2 < 1$\\
                Considero ahora $\lambda = \frac{1}{\sqrt{1-a^2}}$. Sea $\bar{\bar{B}} = \{\bar{e_1}, \bar{\bar{e_2}}\}$, con $\bar{\bar{e_2}} = \lambda \bar{e_2}$.
                \begin{equation*}
                    T(\bar{\bar{e_2}}, \bar{\bar{e_2}}) = T(\lambda \bar{e_2}, \lambda \bar{e_2}) = \lambda^2 T(\bar{e_2}, \bar{e_2}) = \lambda^2 (1-a^2) = 1
                \end{equation*}
                \begin{equation*}
                    M(T;\bar{\bar{\mathcal{B}}}) = \left( \begin{array}{cc}
                        1 & 0 \\
                        0 & 1
                    \end{array} \right) \qquad
                    P = \left( \begin{array}{cc}
                        1 & \frac{a}{\sqrt{1-a^2}} \\
                        0 & \frac{-1}{\sqrt{1-a^2}}
                    \end{array} \right)
                \end{equation*}

                \item $a^2 > 1$\\
                Considero en este caso $\lambda = \frac{1}{\sqrt{a^2-1}}$. Sea $\bar{\bar{B}} = \{\bar{e_1}, \bar{\bar{e_2}}\}$, con $\bar{\bar{e_2}} = \lambda \bar{e_2}$.
                \begin{equation*}
                    T(\bar{\bar{e_2}}, \bar{\bar{e_2}}) = T(\lambda \bar{e_2}, \lambda \bar{e_2}) = \lambda^2 T(\bar{e_2}, \bar{e_2}) = \lambda^2 (1-a^2) = -1
                \end{equation*}
                \begin{equation*}
                    M(T;\bar{\bar{\mathcal{B}}}) = \left( \begin{array}{cc}
                        1 & 0 \\
                        0 & -1
                    \end{array} \right) \qquad
                    P = \left( \begin{array}{cc}
                        1 & \frac{a}{\sqrt{a^2-1}} \\
                        0 & \frac{-1}{\sqrt{a^2-1}}
                    \end{array} \right)
                \end{equation*}
            \end{itemize}

            

            \item $a=\pm 1 \Longrightarrow Nul(T)=1$ \\

            Tomamos $\bar{\mathcal{B}} = \{\bar{e_1}, \bar{e_2}\}$, con $\bar{e_1} = (1,0)^T$ y $\bar{e_2} = (a, -1)^T$.
            \begin{equation*}
                M(T;\bar{\mathcal{B}}) = \left( \begin{array}{cc}
                    1 & 0 \\
                    0 & 0
                \end{array} \right) \qquad
                P = \left( \begin{array}{cc}
                    1 & a \\
                    0 & -1
                \end{array} \right)
            \end{equation*}
        \end{itemize} 
    \end{itemize}
\end{ejercicio}

\begin{ejercicio}
    Sea $(V^n(\bb{K}),g)$ e.v. métrico indefinido. Demostrar que, dado $v\in V$,
    $$g(v,v)=0\nRightarrow v\in Ker(g)$$

    Dada una base $\cc{B}$, sea la métrica indefinida
    \begin{equation*}
        M(g;\cc{B}) = \left(\begin{array}{cc}
            1 & 0 \\
            0 & -1 \\
        \end{array} \right)
    \end{equation*}

    Tomemos $v=(1,1)^t = e_1 + e_2$.
    \begin{equation*}
        g(v,v) = g(e_1+e_2,e_1+e_2) = g(e_1,e_1) + g(e_2,e_2)+2g(e_1,e_2)=1-1+0 = 0
    \end{equation*}

    No obstante, $v\notin Ker(g)=\{0\}$ por ser la matriz asociada regular.
\end{ejercicio}

\begin{ejercicio}
    En $\bb{R}^4$, y dado $a\in \bb{R}$, estudiar la métrica:
    \begin{equation*}
        M(g,\cc{B}_u) = A = \left(\begin{array}{cccc}
            a & 1 & 0 & 0 \\
            1 & 2 & 1 & 0 \\
            0 & 1 & 1 & 0 \\
            0 & 0 & 0 & a \\
        \end{array} \right)
    \end{equation*}

    En primer lugar, calculo su determinante:
    \begin{equation*}
        |A|=a\left|\begin{array}{ccc}
            a & 1 & 0 \\
            1 & 2 & 1 \\
            0 & 1 & 1 \\
        \end{array} \right| a(2a-1-a) = a(a-1)
    \end{equation*}
    Por tanto, ya que el signo del determinante es un invariante, trabajo con los intervalos:
    \begin{equation*}
        a\in ]-\infty,0[\Longrightarrow |A|>0
        \qquad
        a\in ]0,1[ \Longrightarrow |A|<0
        \qquad
        a\in ]1,+\infty[\Longrightarrow |A|>0
    \end{equation*}

    \begin{itemize}
        \item \underline{Para $a>1$}:

        Veamos los menores principales:
        \begin{equation*}
            |a|=a
            \qquad
            \left|\begin{array}{cc}
                a & 1 \\
                1 & 2 \\
            \end{array} \right| = 2a - 1
            \qquad 
            \left|\begin{array}{ccc}
                a & 1 & 0 \\
                1 & 2 & 1 \\
                0 & 1 & 1 \\
            \end{array} \right| = a-1
            \qquad |A|=a(a-1)
        \end{equation*}
        
        $g$ es definida positiva si $a>1$, por lo que:
        \begin{equation*}
            A\sim_c\left(\begin{array}{cccc}
                 1&&&  \\
                 &1&& \\
                 &&1& \\
                 &&&1 \\
            \end{array}\right)
        \end{equation*}
    
        \item \underline{Para $0<a<1$}:
        
        Sabemos que $|A|<0$. Por tanto, $Nul(g)=0$ y:
        \begin{equation*}
            A\sim_c\left(\begin{array}{cccc}
                 1&&&  \\
                 &1&& \\
                 &&1& \\
                 &&&-1 \\
            \end{array}\right)
            \qquad o \qquad
            A\sim_c\left(\begin{array}{cccc}
                 1&&&  \\
                 &-1&& \\
                 &&-1& \\
                 &&&-1 \\
            \end{array}\right)
        \end{equation*}
    
        Como la restricción a  $U=\cc{L}\{e_3,e_4\}$ es definida positiva, entonces hay un plano que es definido positivo. Como en el segundo caso no puede haber un plano definido positivo, entonces estamos en el primer caso.
        \begin{equation*}
            A\sim_c\left(\begin{array}{cccc}
                 1&&&  \\
                 &1&& \\
                 &&1& \\
                 &&&-1 \\
            \end{array}\right)
        \end{equation*}

        \item \underline{Para $a<0$}:
        
        Sabemos que $|A|>0$. Por tanto, $Nul(g)=0$ y:
        \begin{equation*}
            A\sim_c I_4
            \qquad o \qquad
            A\sim_c\left(\begin{array}{cccc}
                 1&&&  \\
                 &1&& \\
                 &&-1& \\
                 &&&-1 \\
            \end{array}\right)
            \qquad o \qquad
            A\sim_c -I_4
        \end{equation*}
    
        Como la restricción a $U=\cc{L}\{e_3,e_4\}$ es definida positiva, entonces $k\geq 2$, con $k$ el número de unos en la matriz asociada a la Base de Sylvester.
        Como la restricción a $U'=\cc{L}\{e_1\}$ es definida negativa, entonces $Ind(g)\geq 1$.
        
        Por tanto, la única opción que reúne ambas condiciones es la central.
        \begin{equation*}
            A\sim_c\left(\begin{array}{cccc}
                 1&&&  \\
                 &1&& \\
                 &&-1& \\
                 &&&-1 \\
            \end{array}\right)
        \end{equation*}

        \item \underline{Para $a=1$}:
        
        En este caso, $Ker(g)\neq 0$. Veamos el valor del rango de $A$. Como
        \begin{equation*}
            |A|=a(a-1) = 0
            \qquad
            \left|\begin{array}{ccc}
                2 & 1 & 0 \\
                1 & 1 & 0 \\
                0 & 0 & a \\
            \end{array} \right|=a(2-1)=a \neq 0\Longrightarrow rg(A)=3
        \end{equation*}
    
        Por tanto, $Nul(g)=1$. Sea $U=\cc{L}\{e_2, e_3,e_4\}$. Como la restricción a $U$ es definida positiva, al menos hay tres unos en la matriz asociada a la base de Sylvester. Por tanto,
        \begin{equation*}
            A\sim_c\left(\begin{array}{cccc}
                 1&&&  \\
                 &1&& \\
                 &&1& \\
                 &&&0 \\
            \end{array}\right)
        \end{equation*}

        \item \underline{Para $a=0$}:
        En este caso, $Ker(g)\neq 0$. Veamos el valor del rango de $A$. Como
        \begin{equation*}
            |A|=a(a-1) = 0
            \qquad
            \left|\begin{array}{ccc}
                0 & 1 & 0 \\
                1 & 2 & 1 \\
                0 & 1 & 1 \\
            \end{array} \right|=a-1=-1 \neq 0\Longrightarrow rg(A)=3
        \end{equation*}
        Por tanto, $Nul(g)=1$. Sea $U=\cc{L}\{e_2, e_3\}$.
        Como la restricción a $U$ es definida positiva, al menos hay 2 unos en la matriz asociada a la base de Sylvester.
        \begin{equation*}
            A\sim_c\left(\begin{array}{cccc}
                 1&&&  \\
                 &1&& \\
                 &&1& \\
                 &&&0 \\
            \end{array}\right)
            \qquad \text{o} \qquad
            A\sim_c\left(\begin{array}{cccc}
                 1&&&  \\
                 &1&& \\
                 &&-1& \\
                 &&&0 \\
            \end{array}\right)
        \end{equation*}

        No obstante, tenemos el siguiente resultado:
        \begin{multline*}
            g(3e_1 - e_2, 3e_1 - e_2) = 3^2g(e_1, e_1) + g(e_2, e_2) -2g(3e_1, e_2) =\\= 0 + 2 -2\cdot 3g(e_1, e_2) = 2-2\cdot 3\cdot 1 = -4
        \end{multline*}

        Por tanto, tenemos que $g$ no puede ser definida positiva. Por tanto, estamos en el segundo caso:
        \begin{equation*}
            A\sim_c\left(\begin{array}{cccc}
                 1&&&  \\
                 &1&& \\
                 &&-1& \\
                 &&&0 \\
            \end{array}\right)
        \end{equation*}
        
    \end{itemize}
    
    Por tanto, tenemos que:
    \begin{equation}\label{Ej12:Resultados}
        \begin{array}{c|c|c}
            \text{Caso} & Ind(g) & Nul(g) \\ \hline
            a < 0 & 2 & 0 \\
            a=0 & 1 & 1\\
            0 < a < 1 & 1 & 0\\
            a=1 & 0 & 1 \\
            a > 1 & 0 & 0 \\           
        \end{array}
    \end{equation}    
\end{ejercicio}

\begin{ejercicio}
    Sea la forma cuadrática $F:\bb{R}^4\to \bb{R}$ expresada en las coordenadas de la base usual como:
    \begin{equation*}
        F(x_1,\dots,x_4) = ax_1^2+2x_2^2+x_3^2+ax_4^2 +2x_1x_2 + 2x_2x_3 \qquad a\in \bb{R}
    \end{equation*}

    \begin{enumerate}
        \item Encontrar la expresión reducida de $F$.

        Buscamos en primer lugar su matriz asociada a la base $\bb{B}=\{e_1,\dots,e_4\}$:
        \begin{equation*}
            A=M(F;\cc{B}) = \left( \begin{array}{cccc}
                a & 1 & 0 & 0 \\
                1 & 2 & 1 & 0 \\
                0 & 1 & 1 & 0 \\
                0 & 0 & 0 & a
            \end{array} \right)
        \end{equation*}
    
        Necesito clasificar la matriz para poder hallar la forma de $F$ simplificada. Esto se ha hecho en el ejercicio anterior, donde se pueden ver los resultados en la Ecuación \ref{Ej12:Resultados}. Por tanto,
        \begin{equation*}
            \begin{array}{c|c|c||l}
                \text{Caso} & Ind(g) & Nul(g) & F \text{ reducida} \\ \hline
                a < 0 & 2 & 0 & F(x_1,\dots,x_4)=x_1^2+x_2^2 -x_3^2-x_4^2\\
                a=0 & 1 & 1 & F(x_1, \dots, x_4) = x_1^2 + x_2^2 - x_3^2\\
                0 < a < 1 & 1 & 0 & F(x_1,\dots,x_4)=x_1^2+x_2^2 +x_3^2-x_4^2\\
                a=1 & 0 & 1 & F(x_1,\dots,x_4)=x_1^2+x_2^2 +x_3^2\\
                a > 1 & 0 & 0 & F(x_1,\dots,x_4)=x_1^2+x_2^2 +x_3^2+x_4^2\\           
            \end{array}
        \end{equation*} 

        \item Dar la expresión reducida de la forma cuadrática para $a=1$.

        En este caso, se pide dar la matriz que proporciona el Teorema de Sylvester, ya que será la de la forma reducida.

        Busco en primer un vector $\bar{e_1}$ de cuadrado no nulo. Sea $\bar{e_1}=e_3$, ya que $g(e_3,e_3) = 1$.

        Busco ahora $\bar{e_2} \in <e_1>^\perp$.
        \begin{equation*}\begin{split}
            <e_1>^\perp & = \left\{x\in \bb{R}^4 \left|
            (0, 0, 1, 0)\left( \begin{array}{cccc}
                a & 1 & 0 & 0 \\
                1 & 2 & 1 & 0 \\
                0 & 1 & 1 & 0 \\
                0 & 0 & 0 & a
            \end{array}\right)
            \left( \begin{array}{c}
                x_1 \\ x_2 \\ x_3 \\ x_4
            \end{array}\right)
            = 0
            \right.\right\}\\
            & = \left\{x\in \bb{R}^4 \left|
            (0, 1, 1, 0)
            \left( \begin{array}{c}
                x_1 \\ x_2 \\ x_3 \\ x_4
            \end{array}\right)
            = 0
            \right.\right\} \\
            &= \left\{x\in \bb{R}^4 \left|
            \begin{array}{c}
                x_2 + x_3 = 0
            \end{array}
            \right.\right\} = \mathcal{L} \left( \left\{
            \left( \begin{array}{c}
                1 \\ 0 \\ 0 \\ 0
            \end{array} \right),
            \left( \begin{array}{c}
                0 \\ 1 \\ -1 \\ 0
            \end{array} \right),
            \left( \begin{array}{c}
                0 \\ 0 \\ 0 \\ 1
            \end{array} \right)
            \right\}\right)
        \end{split}\end{equation*}

        Sea $\bar{e_2} = e_1$. Busco ahora $\bar{e_3}\in \left<\bar{e_1},\bar{e_2}\right>^\perp$.
        \begin{equation*}\begin{split}
            <e_2>^\perp & = \left\{x\in \bb{R}^4 \left|
            (1, 0, 0, 0)\left( \begin{array}{cccc}
                a & 1 & 0 & 0 \\
                1 & 2 & 1 & 0 \\
                0 & 1 & 1 & 0 \\
                0 & 0 & 0 & a
            \end{array}\right)
            \left( \begin{array}{c}
                x_1 \\ x_2 \\ x_3 \\ x_4
            \end{array}\right)
            = 0
            \right.\right\}\\
            & = \left\{x\in \bb{R}^4 \left|
            (a,1,0,0)
            \left( \begin{array}{c}
                x_1 \\ x_2 \\ x_3 \\ x_4
            \end{array}\right)
            = 0
            \right.\right\} \\
            &= \left\{x\in \bb{R}^4 \left|
            \begin{array}{c}
                ax_1 + x_2 = 0
            \end{array}
            \right.\right\} = \mathcal{L} \left( \left\{
            \left( \begin{array}{c}
                0 \\ 0 \\ 0 \\ 1
            \end{array} \right),
            \left( \begin{array}{c}
                0 \\ 0 \\ 1 \\ 0
            \end{array} \right),
            \left( \begin{array}{c}
                1 \\ -a \\ 0 \\ 0
            \end{array} \right)
            \right\}\right)
        \end{split}\end{equation*}

        Por tanto:
        \begin{equation*}
            \left<\bar{e_1},\bar{e_2}\right>^\perp = \left\{x\in \bb{R}^4 \left|
            \begin{array}{c}
                x_2 +x_3= 0 \\
                ax_1 + x_2 = 0 \\
            \end{array}
            \right.\right\}
            = \mathcal{L} \left( \left\{
            \left( \begin{array}{c}
                0 \\ 0 \\ 0 \\ 1
            \end{array} \right),
            \left( \begin{array}{c}
                1 \\ -a \\ a \\ 0
            \end{array} \right)
            \right\}\right)
        \end{equation*}

        Por tanto, sea $\bar{e_3} = e_4$. Por último, busco $\bar{e_4} \in Ker(g)$:
        \begin{equation*}\begin{split}
            Ker(g) & = \left\{x\in \bb{R}^4 \left|
            \left( \begin{array}{cccc}
                a & 1 & 0 & 0 \\
                1 & 2 & 1 & 0 \\
                0 & 1 & 1 & 0 \\
                0 & 0 & 0 & a
            \end{array}\right)
            \left( \begin{array}{c}
                x_1 \\ x_2 \\ x_3 \\ x_4
            \end{array}\right)
            = 0
            \right.\right\}\\
            &= \left\{x\in \bb{R}^4 \left|
            \begin{array}{c}
                x_1+x_2 = 0 \\
                x_1+2x_2+x_3 = 0\\
                x_2+x_3 = 0\\
                x_4=0
            \end{array}
            \right.\right\} = \mathcal{L} \left( \left\{
            \left( \begin{array}{c}
                1 \\ -1 \\ 1 \\ 0
            \end{array} \right)
            \right\}\right)
        \end{split}\end{equation*}

        Por tanto, $\bar{e_4} = (1, -1, 1, 0)^t$. Por tanto, calculo sus imágenes:
        \begin{gather*}
            g(\bar{e_1},\bar{e_1}) = g(e_3,e_3) = 1 \qquad
            g(\bar{e_2},\bar{e_2}) = g(e_1,e_1) = a = 1 \\
            g(\bar{e_3},\bar{e_3}) = g(e_4,e_4) = a = 1 \qquad
            g(\bar{e_4},\bar{e_4}) = 0
        \end{gather*}

        Por tanto, dado $\bar{\cc{B}}=\{\bar{e_1},\bar{e_2},\bar{e_3},\bar{e_4}\} = \{e_3,e_1,e_4,e_1-a_2+a_3\}$ base de Sylvester,
        \begin{equation*}
            A=M(F;\cc{B}) = \left( \begin{array}{cccc}
                1 & \\
                 & 1 \\
                 & & 1 \\
                 &  & & 0
            \end{array} \right)
        \end{equation*}
    \end{enumerate}
\end{ejercicio}

\begin{ejercicio}
    Dado el espacio vectorial $\bb{R}^4$ y la forma cuadrática $\phi:\bb{R}^4 \to \bb{R}$ tal que, en la base usual, $\phi(x_1, x_2, x_3, x_4) = x_1^2 + x_2^2 + x_3^2 + x_4^2 - 2x_1x_3 - 2x_3x_4$. Calcular su expresión reducida.

    Necesito calcular, en primer lugar, su matriz asociada.
    \begin{equation*}
        A = M(\phi, \cc{B}_u) = \left(\begin{array}{cccc}
            1 & 0 & -1 & 0 \\
            0 & 1 & 0 & 0 \\
            -1 & 0 & 1 & -1\\
            0 & 0 & -1 & 1
        \end{array}\right)
    \end{equation*}

    Clasifico ahora la métrica.
    \begin{equation*}
        |A| = \left|\begin{array}{ccc}
            1 & -1 & 0 \\
            -1 & 1 & -1\\
            0 & -1 & 1
        \end{array}\right|
        = \left|\begin{array}{ccc}
            1 & 0 & 0 \\
            -1 & 0 & -1\\
            0 & -1 & 1
        \end{array}\right|
        = \left|\begin{array}{ccc}
            1 & 0 \\
            -1 & -1\\
        \end{array}\right| = -1
    \end{equation*}

    Sabiendo que el sigo del determinante es un invariante y que la nulidad es 0, tenemos que:
    \begin{equation*}
        A \sim_c \left(\begin{array}{cccc}
            1 &&& \\
            &1&& \\
            &&1&\\
            &&&-1
        \end{array}\right)
        \qquad \text{o} \qquad
        A \sim_c \left(\begin{array}{cccc}
            1 &&& \\
            &-1&& \\
            &&-1&\\
            &&&-1
        \end{array}\right)
    \end{equation*}

    Sea $U=\cc{L}\{e_1, e_2\}$. Como $g$ restringido a $U$ es definida positiva, tenemos que al menos hay dos 1 en la matriz asociada a la base de Sylvester. Por tanto, estamos en el primer caso. Es decir, $Ind(\phi)=1$.

    Por tanto, la forma reducida de $\phi$, para la base de Sylvester, es de la forma:
    \begin{equation*}
        \phi(x_1, x_2, x_3, x_4) = x_1^2 + x_2^2 + x_3^2 - x_4^2
    \end{equation*}
\end{ejercicio}


\begin{ejercicio}
    Estudiar, en $\bb{R}^4$, la métrica dada por:
    \begin{equation*}
        A = M(g; \cc{B}_u) = \left( \begin{array}{cccc}
            -4 & 1 & 1 & 1 \\
            1 & -3 & 1 & 1 \\
            1 & 1 & -3 & 1 \\
            1 & 1 & 1 & -3
        \end{array}\right)
    \end{equation*}

    Calculamos en primer lugar su determinante:
    \begin{multline*}
        |A| = \left| \begin{array}{cccc}
            -4 & 1 & 1 & 1 \\
            1 & -3 & 1 & 1 \\
            1 & 1 & -3 & 1 \\
            1 & 1 & 1 & -3
        \end{array}\right|
        = \left| \begin{array}{cccc}
            0 & -11 & 5 & 5 \\
            1 & -3 & 1 & 1 \\
            0 & 4 & -4 & 0 \\
            0 & 4 & 0 & -4
        \end{array}\right|
        = -\left| \begin{array}{ccc}
            -11 & 5 & 5 \\
            4 & -4 & 0 \\
            4 & 0 & -4
        \end{array}\right| =\\= -(-11\cdot 16 +80+80) = 16
    \end{multline*}

    Por el signo del determinante, y sabiendo que $g(e_1, e_1)=-4 <0$, tenemos que:
    \begin{gather*}
        A\sim_c\left(\begin{array}{cccc}
             1&&&  \\
             &1&& \\
             &&-1& \\
             &&&-1 \\
        \end{array}\right)
        \qquad o \qquad
        A\sim_c\left(\begin{array}{cccc}
             -1&&&  \\
             &-1&& \\
             &&-1& \\
             &&&-1 \\
        \end{array}\right)
    \end{gather*}

    Veamos si la matriz es definida negativa:
    \begin{equation*}
        |-4| = -4 \qquad
        \left| \begin{array}{cc}
            -4 & 1 \\
            1 & -3\\
        \end{array}\right| = 12 -1 = 11
        \qquad
        \left| \begin{array}{cccc}
            -4 & 1 & 1 \\
            1 & -3 & 1 \\
            1 & 1 & -3 \\
        \end{array}\right| = -24
        \qquad |A|=16
    \end{equation*}
    Por tanto, como los menores principales de orden par son positivos y los de orden impar son negativos, tenemos que la métrica es definida negativa. Por tanto,
    \begin{equation*}
        A\sim_c\left(\begin{array}{cccc}
             -1&&&  \\
             &-1&& \\
             &&-1& \\
             &&&-1 \\
        \end{array}\right)
    \end{equation*}
\end{ejercicio}

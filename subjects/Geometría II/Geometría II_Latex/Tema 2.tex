\chapter{Formas bilineales simétricas y formas cuadráticas.}

\section{Formas Bilineales}
\begin{definicion}[Forma bilineal]
Sea $V^n(\bb{K})$ un espacio vectorial. La aplicación
$$T:V\times V \longrightarrow \bb{K}$$ es una forma bilineal si es lineal en cada variable. Es decir, para la primera variable se ha de cumplir:
\begin{itemize}
    \item $T(u_1+u_2,v) = T(u_1,v) + T(u_2, v)$
    \item $T(\lambda u,v) = \lambda T(u,v)$
\end{itemize}
Análogamente para la segunda variable.
\begin{ejemplo} Ejemplos de formas bilineales son:
\begin{enumerate}
    \item $T:\bb{K} \times \bb{K} \longrightarrow \bb{K}$ dado por $T(x,y) = xy$
    
    \item El producto escalar\\
    $<,>:\bb{K}^n \times \bb{K}^n \longrightarrow \bb{K}$ dado por $<x,y> = x_1y_1 + \dots + x_ny_n$

    \item $T:\bb{R}^n \times \bb{R}^n \longrightarrow \bb{K}$ dado por $T(x,y) = x_1y_2 - x_2y_1$
\end{enumerate}
\end{ejemplo}
\end{definicion}

\begin{definicion}
    Una forma bilineal $T:V\times V \longrightarrow \bb{K}$ es simétrica si:
    $$T(u,v) = T(v,u) \qquad \forall u,v \in \bb{K}$$
    También son denominadas métricas o tensores.
\end{definicion}

\begin{definicion}[EVM]
    Un espacio vectorial métrico (EVM) es un par $(V,g)$, donde $g$ es una métrica sobre $V$.
\end{definicion}

\begin{teo}
    Sea $T$ una forma bilineal simétrica sobre $V^n(\bb{K})$ y sea $\mathcal{B}=\{e_1, \dots, e_n\}$ base de $V$.

    Entonces, la matriz $A=(a_{ij})\in \mathcal{M}_n(\bb{K})$ donde $a_{ij}=T(e_i, e_j)$ determina de forma biunívoca\footnote{de forma biyectiva} la forma bilineal $T$.

    Dicha matriz se denota por $A=M(T; \mathcal{B})$.
\end{teo}

\begin{proof}
    Sea $u,v \in V^n$ dados por:
    $$u=x_1e_1 + \dots + x_ne_n \equiv (x_1, \dots, x_n)_\mathcal{B}$$
    $$v=y_1e_1 + \dots + y_ne_n \equiv (y_1, \dots, y_n)_\mathcal{B}$$

    Entonces,
    \begin{multline*}
        T(u,v) = T\left(\sum_i x_ie_i , \sum_j y_je_j \right) = \sum_{ij}x_iy_jT(e_i,e_j) = \sum_{ij}a_{ij}x_iy_{j} =\\=
        (x_1, \dots, x_n)A\left(\begin{array}{c}
            y_1 \\ \vdots \\ y_n
        \end{array} \right)
    \end{multline*}
\end{proof}

\begin{lema}
    Sea $T$ una forma bilineal y sea $A\in \mathcal{M}_n(\bb{K})$ su matriz asociada.
    $$\text{$T$ es simétrica} \Longleftrightarrow A=A^T$$
\end{lema}
\begin{proof}
    $$\text{$T$ es simétrica} \Longleftrightarrow T(e_i,e_j) = T(e_j,e_i) \Longleftrightarrow a_{ij} = a_{ji} \Longleftrightarrow A=A^T$$
\end{proof}

\subsection{Congruencia de matrices}
\begin{teo}
    Sea $T$ una forma bilineal simétrica sobre $V^n(\bb{K})$ y sean $\mathcal{B}=\{e_1, \dots, e_n\}$ y $\bar{\mathcal{B}}$ bases de $V^n(\bb{K})$. Sea $P$ la matriz de cambio de base $P=M(\bar{\mathcal{B}};\mathcal{B})$. Sean también $A=M(T, \mathcal{B})$ y $\bar{A} = M(T, \bar{\mathcal{B}})$. Entonces:
$$\bar{A} = P^tAP$$
\end{teo}
\begin{proof}
    Sea $u,v\in V$, y sea $u=x_\cc{B}=\bar{x}_{\bar{\cc{B}}}$, $v=y_\cc{B}=\bar{y}_{\bar{\cc{B}}}$. Tengo que:
    \begin{equation*}
        P\bar{x}=x \qquad P\bar{y}=y
    \end{equation*}
    Entonces:
    \begin{equation*}
        g(u,v)=x^tAy = \bar{x}^tP^t{A}P\bar{y} = \bar{x}^t\bar{A}\bar{y}  \Longrightarrow \bar{A} = P^tAP
    \end{equation*}
\end{proof}

\begin{definicion}[Congruencia de matrices] Dos matrices $A,B\in \mathcal{M}_n(\bb{K})$ se dicen congruentes si $\exists P\in \mathcal{M}_n(\bb{K})$ \textbf{regular} t.q.:
$$A=P^tBP$$
\end{definicion}

\begin{lema}
    Sean dos matrices $A,B\in \mathcal{M}_n(\bb{K})$ congruentes. Entonces:
    $$A,B \text{ congruentes} \Longrightarrow rg(A)=rg(B)$$
\end{lema}
\begin{ejemplo}La matriz asociada al producto escalar es:
    $$M(<,>,\mathcal{B}_u) = Id$$
\end{ejemplo}

\begin{observacion}
    Sean $A_1, A_2\in \mathcal{S}_n(\bb{R})$. Supongamos $A_1$ congruente a $A_2$.

    Veamos qué ocurre con la traza:
    \begin{equation*}
        tr(A_2) = tr(P^tA_1P) = tr(A_1P^tP)
    \end{equation*}
    Por tanto, no tienen la misma traza.
    
    Veamos qué ocurre con el determinante:
    \begin{equation*}
        |A_2| = |P||P^t||A_1|=|P|^2|A_1|
    \end{equation*}
    Por tanto, podemos ver que el signo del determinante es un invariante, es decir, no cambia.\\
    
    También es un invariante el número de $1$ en la matriz asociada a la base de Sylvester.
    $$k = \max \{\dim U \mid U\subset V \text{ sub. vectorial} \quad \land \quad g_{\left|U \right.} \text{ def. positiva}\}$$
\end{observacion}

\begin{prop}
    $\sim_c$ es una relación de equivalencia en $\mathcal{M}_n(\bb{K})$.
\end{prop}
\begin{proof} Demostramos las tres condiciones:
\begin{itemize}
    \item $A\sim_c A$, ya que $A=I^t AI$
    
    \item $A\sim_c B \Longrightarrow A=P^tBP \Longrightarrow B=(P^{-1})^tAP^{-1} \Longrightarrow B\sim_c A$

    \item Supongamos $A\sim_c B$ y $B\sim_c C$:
    \begin{multline*}
        \left\{\begin{array}{c}
            A\sim_c B  \\
            B\sim_c C 
        \end{array} \right\}
        \Longrightarrow
        \left\{\begin{array}{c}
            A = P^tBP  \\
            B = Q^tCQ 
        \end{array} \right\}
        \Longrightarrow\\ \Longrightarrow
        A = P^tQ^tCQP = (QP)^tC(QP)
        \Longrightarrow
        A\sim_c C
    \end{multline*}
\end{itemize}
\end{proof}

\subsection{Tipos de Métricas}

\begin{definicion}
    El núcleo de $T$ se define como:
    $$Ker(T) = \left\{ u\in V \mid T(u,v)=0 \quad 
    \forall v\in V \right\}$$

    El núcleo de $T$ es un subespacio vectorial de $V$, y su nulidad se define como:
    $$Nul(T) = \dim Ker(T)$$

    Análogamente, el núcleo se define como:
    \begin{equation*}\begin{split}
        Ker(T) &= \left\{ u\in V \mid T(u,v)=0 \quad  \forall v\in V \right\} \\
        & = \left\{ (x_1, \dots, x_n)\in V \mid (x_1, \dots, x_n)A\left(\begin{array}{c}
            y_1 \\ \vdots \\ y_n
        \end{array} \right)=0 \quad 
        \forall \left(\begin{array}{c}
            y_1 \\ \vdots \\ y_n
        \end{array} \right)\in V \right\} \\
        &= \left\{ (x_1, \dots, x_n)\in V \mid (x_1, \dots, x_n)A=0\right\}\\
        &= \left\{ \left(\begin{array}{c}
            x_1 \\ \vdots \\ x_n
        \end{array} \right) \in V \mid A\left(\begin{array}{c}
            x_1 \\ \vdots \\ x_n
        \end{array} \right)=\left(\begin{array}{c}
            0 \\ \vdots \\ 0
        \end{array} \right)\right\}
    \end{split}\end{equation*}
\end{definicion}

\begin{prop} Sea $T$ una métrica y sea $A$ la matriz asociada a $T$ en determinada base. Entonces,
    \begin{equation*}
        Nul(T) = n -rg(A)
    \end{equation*}
\end{prop}
\begin{proof}
    Tenemos que:
    \begin{equation*}
        Ker(T) = \left\{ \left(\begin{array}{c}
            x_1 \\ \vdots \\ x_n
        \end{array} \right) \in V \mid A\left(\begin{array}{c}
            x_1 \\ \vdots \\ x_n
        \end{array} \right)=\left(\begin{array}{c}
            0 \\ \vdots \\ 0
        \end{array} \right)\right\}
    \end{equation*}

    Es decir,
    \begin{equation*}
        \dim Ker(T) = Nul(T) = n - \text{num. ecuaciones lin. indep.} = n-rg(A)
    \end{equation*}
\end{proof}

\begin{definicion}
    La forma bilineal $T$ se dice que no es degenerada si
    $$Ker(T) = \{0\} \Longleftrightarrow A \text{ es regular}$$

    Análogamente,
    $$T \text{ es degenerada} \Longleftrightarrow Ker(T) \neq \{0\} \Longleftrightarrow A \text{ es singular}$$
\end{definicion}

\begin{definicion}
    Sea $u\in V$. Se define el subespacio conjugado (o ortogonal) como:
    \begin{equation*}
        <u>^{\perp} = \left\{ v\in V \mid T(u,v)=0\right\}
    \end{equation*}
\end{definicion}
Hay dos posibilidades:
\begin{itemize}
    \item $<u>^{\perp}$ es un hiperplano
    \item $<u>^{\perp} = V \Longleftrightarrow u\in Ker(T)$ 
\end{itemize}

\begin{definicion} Sea $(V,g)$  e.v. métrico. Decimos que $g$ es definida positiva si:
$$g(v,v)>0 \qquad \forall v\in V-\{0\}$$
\end{definicion}

\begin{definicion} Sea $(V,g)$  e.v. métrico. Decimos que $g$ es definida negativa si:
$$g(v,v)<0 \qquad \forall v\in V-\{0\}$$
\end{definicion}

\begin{definicion} Sea $(V,g)$  e.v. métrico. Decimos que $g$ es semidefinida positiva si:
$$g(v,v)\geq 0 \qquad \forall v\in V-\{0\}$$
\end{definicion}

\begin{definicion} Sea $(V,g)$  e.v. métrico. Decimos que $g$ es semidefinida negativa si:
$$g(v,v)\leq 0 \qquad \forall v\in V-\{0\}$$
\end{definicion}

\begin{definicion} Sea $(V,g)$  e.v. métrico. Decimos que $g$ es indefinida si:
$$\exists v,w \in V \mid g(v,v)>0 \quad \land \quad g(w,w)<0$$
\end{definicion}

\begin{definicion}
    Sea $u,v\in V$. Decimos que $u,v$ son conjugados (ortogonales) si:
    $$T(u,v)=0$$
\end{definicion}

\begin{observacion}
    Sea $(V,g)$ un espacio vectorial métrico real. Si se tiene $v\in V \mid g(v,v)=~0$, entonces tenemos:
    \begin{itemize}
        \item $g$ definida positiva/negativa $\Longrightarrow v=0$.

        \item $g$ semidefinida positiva/negativa $\Longrightarrow v\in Ker(g)$.

        \item $g$ no degenerada e indefinida $\Longrightarrow$ No se sabe.

        Ejemplo de esto último es la métrica cuya matriz asociada es $\left(\begin{array}{cc}
            1 &  \\
             & -1
        \end{array}\right)$, donde, para $v=(1,1)^t$, tenemos que $g(v,v)=0$.
    \end{itemize}
\end{observacion}

\section{Teorema de Sylvester}

\begin{lema}
    Dado $T$ forma bilineal con $k=Nul(T)$, entonces existe una base $\mathcal{B}$ de $V$ t.q.
    \begin{equation*}
        A = M(T;\mathcal{B}) = \left( \begin{array}{c|c}
            A_1 & 0_{n-k\times k} \\ \hline
            0_{n\times n-k} & 0_k
        \end{array}\right)
    \end{equation*}
    con $A_1$ regular.
\end{lema}

\begin{proof}
    Sea $\left\{e_{n-k+1}, \dots, e_n \right\}$ base de $Ker(T)$. Ampliamos dicha base a $\mathcal{B}$ base de $V$, con $\mathcal{B}~=~\left\{e_1, \dots, e_{n-k}, e_{n-k+1}, \dots, e_n \right\}$ base de $V$.

    \begin{equation*}
        a_{ij} = T(e_i, e_j) = \left\{ \begin{array}{cl}
            0 & \text{si } e_i \in Ker(T) \\
            a_{ij}  & \text{si } e_i \notin Ker(T) \Longrightarrow A_1
        \end{array}\right.
    \end{equation*}

    $A_1$ es regular porque
    $$rg(A) = n-k$$
\end{proof}

\begin{prop}
    Sea $T$ una forma bilineal simétrica.
    $$T(u,u) = 0 \;\forall u \Longrightarrow T=0$$
\end{prop}
\begin{proof}
    Sea $u,v \in V$
    \begin{equation*}
        T(u+v, u+v) = 0 = \cancel{T(u,u)} + 2T(u,v) + \cancel{T(v,v)} = 0 \Longrightarrow T(u,v)=0 \Longrightarrow T=0
    \end{equation*}
\end{proof}

\begin{teo}
    Sea $V$ un espacio vectorial y $T$ una forma bilineal simétrica no degenerada. Entonces, existe una base $\mathcal{B}$ de $V$ t.q.
    \begin{equation*}
        M(T;\mathcal{B}) = \left( \begin{array}{ccc}
            a_1 & & \\
             & \ddots &\\
             && a_n
        \end{array}\right) \qquad \text{con } a_i \neq 0\; \forall i
    \end{equation*}
\end{teo}

\begin{proof} Demostramos por inducción sobre $n$.
\begin{itemize}
    \item \underline{Para $n=1$}\\
    Se cumple, ya que toda matriz de dimensión $1$ es diagonal.
    \item \underline{Supuesto cierto para $n-1$, lo demuestro para $n$}

    Tomo $e_1\neq 0$ t.q. $T(e_1, e_1)\neq 0\footnote{Este existe $\forall T \neq 0$. Para $T=0$, se sabe que es cierto.}$ Tomo $<e_1>^{\perp} = U$ hiperplano, ya que $e_1\notin Ker(T)$.

    Tomamos ahora la forma bilineal simétrica $T_{\left|U\right.}$ sobre $U$. Esta es no degenerada, y veámoslo mediante reducción al aburdo.
    
    Supongamos que es degenerada, es decir, $Ker(T_{\left|U\right.})\neq \{0\}$. Entonces existiría $v\in U-\{0\} \mid T(v, w) \; \forall w\in U$. Además, como $v\in U$, sabemos que $T(v, e_1) = 0$. Por tanto, $v$ es ortogonal a todos los vectores de $V$, por lo que $v\in Ker(T)$. Pero $T$ es no degenerada, por lo que llegamos a una contradicción. 
    
    
    Por tanto, $T_{\left|U \right.}$ es una forma bilineal simétrica no degenerada, y $U$ tiene dimensión $n-1$. Por hipótesis de inducción, existe una base $\mathcal{B}'$ de $U$ t.q.
    \begin{equation*}
        M(T_{\left|U \right.};\mathcal{B}') = \left( \begin{array}{ccc}
            a_1 & & \\
             & \ddots &\\
             && a_{n-1}
        \end{array}\right) \qquad \text{con } a_i \neq 0\; \forall i=1, \dots, n-1
    \end{equation*}

    Sea $\mathcal{B}$ base de $V$ de la forma $\mathcal{B} =\mathcal{B}' \cup \{e_1\}$.
    \begin{equation*}
        M(T;\mathcal{B}) = \left( \begin{array}{ccc|c}
            a_1 & & & 0\\
             & \ddots & & \vdots \\
             && a_{n-1} & 0 \\ \hline
             0 & \dots & 0 & a_n
        \end{array}\right) \qquad \text{con } a_i \neq 0\; \forall i=1, \dots, n
    \end{equation*}
    donde $a_n\neq 0$ por elección de $e_1$ y el resto de los coeficientes de la columna y fila últimos son nulos porque $\mathcal{B}'$ son ortogonales a $e_1$.
\end{itemize}
    
\end{proof}

\begin{prop}
    Sea $(V,g)$ espacio vectorial métrico. Si $g$ es no degenerada, entonces:
    \begin{itemize}
        \item \underline{$\bb{K} = \bb{C}$}:
        \begin{equation*}
            \exists \mathcal{B} \text{ base de } V \mid M(g; \mathcal{B}) = \left( \begin{array}{ccc}
                1 & & \\
                 & \ddots & \\
                & & 1
            \end{array} \right)
        \end{equation*}
        \begin{proof}
            Cambiamos los vectores de la proposición anterior (supongamos $\mathcal{B}' = \{e_1, \dots, e_n\}$) por $\mathcal{B} = \left\{e_1 \cdot \frac{1}{\sqrt{g(e_1, e_1)}}, \dots, e_n \cdot \frac{1}{\sqrt{g(e_n, e_n)}} \right\}$ y obtenemos la matriz pedida.
        \end{proof}

        \item \underline{$\bb{K} = \bb{R}$}:
        \begin{equation*}
            \exists \mathcal{B} \text{ base de } V \mid M(g; \mathcal{B}) = \left( \begin{array}{cccccc}
                1 & & && & \\
                 & \ddots & &&& \\
                & & 1 &&& \\
                 &&& -1 &&\\
                 &&&& \ddots & \\
                 &&&&& -1
            \end{array} \right)
        \end{equation*}
        \begin{proof}
            Cambiamos los vectores de la proposición anterior (supongamos $\mathcal{B}' = \{e_1, \dots, e_n\}$) por:
            \begin{itemize}
                \item $e_i \cdot \frac{1}{\sqrt{g(e_1, e_1)}} \qquad \text{ si } g(e_i, e_1) > 0$

                \item $e_i \cdot \frac{1}{\sqrt{\left|g(e_1, e_1)\right|}} \qquad \text{ si } g(e_i, e_1) < 0$
            \end{itemize}
            y obtenemos la matriz buscada.
        \end{proof}
    \end{itemize}
\end{prop}

\begin{prop}
    Sea $(V^n(\bb{K}), g)$ e.v. métrico. Sea $U\subset V$ subespacio vectorial t.q. $V=U\oplus Ker(g)$. Entonces $g_{\left|U \right.}$ es no degenerada.
\end{prop}
\begin{proof}
    Denotamos por $k=\dim Ker(g) = Nul(g) = n-rg(A)$.

    Sea $\{e_1, \dots, e_k\}$ base de $Ker(g)$ y $\mathcal{B}_U=\{e_{k+1}, \dots, e_n\}$ base de $U$. Entonces $\mathcal{B} = \{e_1, \dots, e_k, e_{k+1}, \dots, e_n\}$ es una base de $V$. Debido a la suma directa, tenemos que:
    \begin{equation*}
        A=M(g, \mathcal{B}) = \left(\begin{array}{c|c}
            0_{k,k} & 0 \\ \hline
            0 & B_{n-k, n-k}
        \end{array} \right)
    \end{equation*}

    Sabemos que $rg(B)=rg(A) = n-k \Longrightarrow B$ es regular.

    Como $M(g_{\left|U \right.}; \mathcal{B}_U) = B$ y $B$ es regular, entonces $g_{\left|U \right.}$ es no degenerada.
\end{proof}

\begin{definicion} [Índice] Sea $(V,g)$ espacio vectorial métrico. Definimos el índice de $g$ como la cantidad de negativos en la diagonal de la matriz asociada a $g$ al diagonalizar la métrica.

Se denota como $Ind(g)$.
\end{definicion}

\begin{definicion} [Índice Estrella] Sea $(V,g)$ espacio vectorial métrico. Definimos el índice estrella de $g$ como:
\begin{equation*}
    Ind^\ast (g) = \max \{\dim U \mid U\subset V \text{ sub. vectorial} \quad \land \quad g_{\left|U \right.} \text{ def. negativa}\}
\end{equation*}
\end{definicion}

\begin{prop} Sea $(V,g)$ e.v. métrico.
    $$Ind^\ast (g) = Ind(g)$$
\end{prop}

\begin{coro}
    Sea $(V,g)$ espacio vectorial métrico. Sea $k$ la cantidad de negativos en la diagonal de la matriz asociada a $g$ al diagonalizar la métrica.
    Entonces:
    \begin{equation*}
        k = \max \{\dim U \mid U\subset V \text{ sub. vectorial} \quad \land \quad g_{\left|U \right.} \text{ def. positiva}\}
    \end{equation*}
\end{coro}
\begin{proof}
    Se puede demostrar a partir del la proposición anterior haciendo uso de $-g$.
\end{proof}

\begin{teo}[Teorema de Sylvester]
    Sea $T$ una forma bilineal simétrica sobre $V^n(\bb{K})$. Entonces existe una base $\mathcal{B}$ de V tal que:
    \begin{itemize}
        \item \underline{Caso complejo}:
        \begin{equation*}
            M(T,\mathcal{B}) = \left( \begin{array}{ccc|ccc}
                0&&&&& \\
                &\ddots ^k&&&& \\
                &&0&&& \\ \hline
                &&&1&& \\
                &&&&\ddots^r& \\
                &&&&&1 \\
            \end{array}\right)  \qquad \text{donde } \left\{\begin{array}{l}
                 k=Nul(T) \\
                 r=rg(T)
            \end{array}\right.
        \end{equation*}

        \begin{proof}
            Sea $\{e_1, \dots, e_k\}$ base de $Ker(T)$. Amplío dicha base a una base de $V$. Sea $\mathcal{B} = \{e_1, \dots, e_k, e_{k+1}, \dots, e_n\}$.

            Sea $U=\mathcal{L}(\{e_{k+1}, e_n\})$. Como $U\oplus Ker(g)=V$, entonces $g_{\left|U \right.}$ es no degenerada.

            Como $g_{\left|U \right.}$ es no degenerada, entonces $\exists \mathcal{B}' = \{e_{k+1}', \dots, e_{n}'\}$ base de $U$ t.q. $M(g_{\left|U \right.}; \mathcal{B}') = I$.

            Por tanto, sea $\bar{\mathcal{B}} = \{e_1, \dots, e_k, e_{k+1}', \dots, e_n'\}$. Tenemos que
            \begin{equation*}
                    M(g;\bar{\mathcal{B}}) = \left( \begin{array}{ccc|ccc}
                    0&&&&& \\
                    &\ddots ^k&&&& \\
                    &&0&&& \\ \hline
                    &&&1&& \\
                    &&&&\ddots^r& \\
                    &&&&&1 \\
                \end{array}\right)  \qquad \text{donde } \left\{\begin{array}{l}
                     k=Nul(T) \\
                     r=rg(T)
                \end{array}\right.
            \end{equation*}
        \end{proof}

        \item \underline{Caso real}:
        \begin{equation*}
            M(T,\mathcal{B}) = \left( \begin{array}{ccc|ccc|ccc}
                0&&&&&&& \\
                &\ddots ^r&&&&&&& \\
                &&0&&&&&& \\ \hline
                &&&1&&&&& \\
                &&&&\ddots^s&&&& \\
                &&&&&1&&& \\ \hline
                &&&&&&-1&& \\
                &&&&&&&\ddots ^t& \\
                &&&&&&&&-1 \\
            \end{array}\right)  \qquad \text{donde } \left\{\begin{array}{l}
                 r=Nul(T) \\
                 (s, t) \text{ signatura}  \\
                 t=\text{índice} \\
                 r+s+t=n
            \end{array}\right.
        \end{equation*}
    \end{itemize}
\end{teo}

\begin{coro} Sea $A\in \mathcal{S}(\bb{K}) \Longrightarrow \exists P\in \mathcal{M}_n(\bb{K})$ regular tal que:
\begin{itemize}
        \item \underline{$\bb{K} = \bb{C}$}:
        \begin{equation*}
            P^tAP = \left( \begin{array}{ccc|ccc}
                1&&&&& \\
                &\ddots&&&& \\
                &&1&&& \\ \hline
                &&&0&& \\
                &&&&\ddots& \\
                &&&&&0 \\
            \end{array}\right)
        \end{equation*}

        \item \underline{$\bb{K} = \bb{R}$}:
        \begin{equation*}
            P^tAP = \left( \begin{array}{ccc|ccc|ccc}
                1&&&&&&& \\
                &\ddots &&&&&&& \\
                &&1&&&&&& \\ \hline
                &&&-1&&&&& \\
                &&&&\ddots&&&& \\
                &&&&&-1&&& \\ \hline
                &&&&&&0&& \\
                &&&&&&&\ddots & \\
                &&&&&&&&0 \\
            \end{array}\right)
        \end{equation*}
    \end{itemize}
\end{coro}

\begin{definicion}[Signatura]
    Sea $(V,g)$ espacio vectorial métrico. Se define la signatura de $g$ como $(t, s)$, donde:
    \begin{center}
        $t=$número de 1 en la matriz de Sylvester \\
        $s=Ind(g)$
    \end{center}
\end{definicion}

\begin{ejemplo}
    Sea $\bb{K}=\bb{C}$. Sea el espacio vectorial métrico $(\bb{C}^2, g)$ Encontrar la base de Sylvester de la métrica $g$, sabiendo que
    \begin{equation*}
        A = M(g; \mathcal{B}_u) = \left( \begin{array}{cc}
            1 & 1 \\
            1 & 1
        \end{array} \right)
    \end{equation*}

    Sea $\mathcal{B}$ la base de Sylvester. Como $rg(A) = 1$,
    \begin{equation*}
        M(g; \mathcal{B}) = \left( \begin{array}{cc}
            1 &  \\
             & 0
        \end{array} \right)
    \end{equation*}

    Calculo en primer lugar una base del núcleo.
    \begin{equation*}
        Ker(f) = \mathcal{L}\{(1, -1)\}
    \end{equation*}

    Obtenemos una base de $\bb{C}^2$. $\mathcal{B}=\{(1,0), (1, -1)\}$
    \begin{equation*}
        M(g; \mathcal{B}) = \left( \begin{array}{cc}
            1 & 0 \\
            0 & 0
        \end{array} \right)
    \end{equation*}
\end{ejemplo}

\begin{ejemplo}
    Sea $\bb{K}=\bb{C}$. Sea el espacio vectorial métrico $(\bb{C}^3, g)$ Encontrar la base de Sylvester de la métrica $g$, sabiendo que
    \begin{equation*}
        A = M(g; \mathcal{B}_u) = \left( \begin{array}{ccc}
            0 & 1 & 1\\
            1 & 0 & 1\\
            1 & 1 & 0
        \end{array} \right)
    \end{equation*}

    Sea $\mathcal{B}$ la base de Sylvester. Como $rg(A) = 3$,
    \begin{equation*}
        M(g; \mathcal{B}) = \left( \begin{array}{ccc}
            1 &  & \\
             & 1 & \\
             &  & 1
        \end{array} \right) = I
    \end{equation*}

    Busco $\bar{e_1} \in \bb{C}^2$ de cuadrado no nulo.
    $$\bar{e_1} = (1, 1, 0) \Longrightarrow g(\bar{e_1}, \bar{e_1}) = 2 \neq 0$$
    
    $$<\bar{e_1}>^T = \left\{ x\in \bb{C}^3 \mid \bar{e_1}Ax = 0 \right\} = \left\{ x\in \bb{C}^3 \mid x_1 + x_2 + 2x_3 = 0 \right\}$$

    Busco $\bar{e_2} \in <\bar{e_1}>^T$ de cuadrado no nulo.
    $$\bar{e_2} = (1, -1, 0) \Longrightarrow g(\bar{e_2}, \bar{e_2}) = -2 \neq 0$$

    $$<\bar{e_2}>^T = \left\{ x\in \bb{C}^3 \mid \bar{e_2}Ax = 0 \right\} = \left\{ x\in \bb{C}^3 \mid -x_1 + x_2 = 0 \right\}$$
    
    Busco $\bar{e_3} \in <\bar{e_1}>^T \cap <\bar{e_2}>^T$.
    $$\bar{e_3} = (1, 1, -1) \Longrightarrow g(\bar{e_3}, \bar{e_3}) = -2 \neq 0$$

    Por tanto, sea $\bar{\mathcal{B}} = \{\bar{e_1}, \bar{e_2}, \bar{e_3}\}$
    \begin{equation*}
        M(g; \bar{\mathcal{B}}) = \left( \begin{array}{ccc}
            2 &  & \\
             & -2 & \\
             &  & -2
        \end{array} \right)
    \end{equation*}

    La base de Sylvester es:
    \begin{equation*}
        \mathcal{B} = \left\{ \frac{\bar{e_1}}{\sqrt{2}}, \frac{\bar{e_2}}{\sqrt{-2}}, \frac{\bar{e_3}}{\sqrt{-2}} \right\}
    \end{equation*}
\end{ejemplo}

\begin{teo}
    Sea $A\in\mathcal{S}_n(\bb{R})$.
    $$A\text{ es def. positiva} \Longleftrightarrow \text{Todos sus menores principales son positivos}$$

    \begin{proof} Procedemos mediante doble implicación:
    \begin{description}
        \item [$\Longrightarrow$]] Suponemos $A$ definida positiva.
        
        Sea $g$ una métrica tal que $A=M(g,\mathcal{B})$ para $\mathcal{B}=\{e_1,\dots,e_n\}$ base de V.
        
        Sabemos que, dado $U\subset V$ subespacio vectorial $\Longrightarrow g_{\left|U\right.}$ definida positiva.

        Sea $k\in\{1,\dots, n\}$ fijo. $U=\mathcal{L}\{e_1,\dots, e_k\}$.
        Tenemos que $$M(g_{\left|U\right.}; \{e_1,\dots,e_k\}) = \left(\begin{array}{ccc}
            a_{11} &\dots  & a_{1k} \\
            \vdots & \ddots & \vdots \\
            a_{k1} & \dots & a_{kk}
        \end{array}\right)$$

        Como $g_{\left|U\right.}$ es definida positiva, entonces $|M(g_{\left|U\right.}; \{e_1,\dots,e_k\})|$ es positivo. Como esto es cierto $\forall k= 1,\dots, n$, entonces todos los menores principales son positivos.

        \item [$\Longleftarrow$]] Suponemos que todos los menores principales son positivos.

        Demostramos por inducción sobre $n$.
        \begin{itemize}
            \item \underline{Para $n=1$:}
            
            Se da, ya que el cuadrado de $e_1$ es positivo, por lo que es definida positiva.
            
            \item \underline{Supuesto cierto para $n-1$, compruebo para $n$:}

            $U=\mathcal{L}\{e_1,\dots, e_{n-1}\}$.
            Tenemos que $$M(g_{\left|U\right.}; \{e_1,\dots,e_{n-1}\}) = \left(\begin{array}{ccc}
                a_{11} &\dots  & a_{1,{n-1}} \\
                \vdots & \ddots & \vdots \\
                a_{{n-1},1} & \dots & a_{{n-1},{n-1}}
            \end{array}\right)$$
            Por tanto, $g_{\left|U\right.}$ es definida positiva, ya que sé que todos sus menores principales son positivos (hipótesis de inducción).

            Tomando $\mathcal{B}'_{n-1}=\{e_1', \dots, e_{n-1}'\}$ la base de Sylvester de $U$, tenemos que:
            $$M(g_{\left|U\right.};\mathcal{B}'_{n-1}) = \left(\begin{array}{ccc}
                1 && \\
                 & \ddots & \\
                 & & 1
            \end{array}\right) = I$$

            Ampliando $\mathcal{B}'_{n-1}$ a una base de V, tengo que $\mathcal{B}'=\{e_1', \dots, e_{n-1}', e_n'\}$.

            Veamos si $\exists e'_n\in V\left|\begin{array}{l}
                g(e'_n,e'_1)=0 \\
                \vdots \\
                g(e'_n,e'_{n-1})=0 \\
            \end{array}\right\}$. Por tanto, busco un vector $e'_n$ verificando $n-1$ ecuaciones homogéneas. Por tanto, como será un SCI, existe solución. Por tanto, existe ese vector $e'_n$ buscado.

            Por tanto, $e'_n$ es ortogonal a todos los de $\mathcal{B}'_{n-1}$. Además, $e'_n\notin U$\footnote{Si $e'_n$ perteneciese a $U$, sería el vector nulo.}. Por tanto, $\mathcal{B}'$ es una base.
            $$M(g_;\mathcal{B}') = \left(\begin{array}{ccc|c}
                1 &&&0 \\
                 & \ddots & \vdots \\
                 & & 1 & 0 \\ \hline
                 0 & \dots & 0 & a
            \end{array}\right)$$
            Para ver que es definida positiva, es necesario ver que $a>0$. Como todos los menores principales son no nulos, $|M(g_;\mathcal{B}')| = a > 0$. Por tanto, $A$ es definida positiva, ya que ambas matrices son congruentes.
        \end{itemize}
    \end{description}
    \end{proof}
\end{teo}
\begin{observacion}
    Cambiando filas y columnas con cuidado, podemos reordenar la base escogida y, por tanto, facilitarnos los cálculos.
\end{observacion}

\begin{coro}
    Sea $A\in\mathcal{S}_n(\bb{R})$.
    $$A\text{ es def. negativa} \Longleftrightarrow \left\{ \begin{array}{c}
        \text{Todos sus menores principales de orden par son positivos} \\
        \text{Todos sus menores principales de orden impar son negativos}
    \end{array}\right.$$
\end{coro}
\begin{proof}
    Se demuestra haciendo uso de que $g$ definida negativa $\Longleftrightarrow -g$ definida positiva.
\end{proof}

\section{Isometría}
\begin{definicion} [Isometría]
    Sean $(V_1^n(\bb{K}), g_1)$ y $(V_2^n(\bb{K}), g_2)$ espacios vectoriales métricos. Dado $f:V_1\to V_2$ isomorfismo, decimos que es una isometría si:
    \begin{equation*}
        g_2(f(u),f(v)) = g_1(u,v) \qquad \forall u,v \in V_1
    \end{equation*}

    Luego una isometría es un isomorfismo en EVM que conserva las métricas.
\end{definicion}

\begin{definicion}
    Dos EVM $(V_1^n(\bb{K}), g_1)$ y $(V_2^n(\bb{K}), g_2)$ se dicen isométricos si \\ $\exists f:~V_1\to~V_2$ isometría.
\end{definicion}

\begin{prop}
    Sean $\mathcal{B}_1,\mathcal{B}_2$ base de $V_1$ y $V_2$ respectivamente; y sean
    \begin{equation*}
        A_1 = M(g_1, \mathcal{B}_1) \qquad A_2 = M(g_2, \mathcal{B}_2)
    \end{equation*}
    Entonces:
    \begin{equation*}
        A_1 \text{ congruente a } A_2 \Longleftrightarrow (V_1,g_1),(V_2,g_2) \text{ son isométricos}.
    \end{equation*}
\end{prop}
\begin{proof} Procedemos mediante doble implicación:
\begin{description}
    \item [$\Longrightarrow$)]
    Suponemos $A_1\sim_c A_2$, es decir, $A_1=P^tA_2P$.
    
    Sea $u=~\left(\begin{array}{c}
        x_1 \\ \vdots \\ x_n
    \end{array}\right)$ y $v=\left(\begin{array}{c}
        y_1 \\ \vdots \\ y_n
    \end{array}\right)$.

    Sea $M(f;\cc{B}_1, \cc{B}_2) = P$. Es un isomorfismo, ya que $P$ es regular.

    Las imágenes de $u,v$ son:
    \begin{equation*}
        f(u) = P\left(\begin{array}{c}
            x_1 \\ \vdots \\ x_n
        \end{array}\right)
        \qquad
        f(v) = P\left(\begin{array}{c}
            y_1 \\ \vdots \\ y_n
        \end{array}\right)
    \end{equation*}

    Por tanto,
    \begin{multline*}
        g_2(f(u), f(v)) = f(u)^t A_2 f(v) =
        (x_1, \dots, x_n)P^t \cdot A_2 \cdot P\left(\begin{array}{c}
            y_1 \\ \vdots \\ y_n
        \end{array}\right)
        =\\\stackrel{A_1=P^tA_2P}{=}
        (x_1, \dots, x_n)A_1\left(\begin{array}{c}
            y_1 \\ \vdots \\ y_n
        \end{array}\right)
    \end{multline*}
    \begin{equation*}
        g_1(u,v) = u^t A_1 v =
        (x_1, \dots, x_n) \cdot A_1 \cdot \left(\begin{array}{c}
            y_1 \\ \vdots \\ y_n
        \end{array}\right)
    \end{equation*}

    Por tanto, tenemos que $g_2(f(u), f(v)) = g_1(u,v)$, por lo que $f$ es una isometría. Por tanto, $(V_1,g_1),(V_2,g_2)$ son isométricos.

    
    \item [$\Longleftarrow$)]
    Suponemos $(V_1,g_1),(V_2,g_2)$ isométricos.
    
    Sea $P=M(f;\mathcal{B}_1, \mathcal{B}_2)$ matriz regular, ya que es un isomorfismo.
    
    Sea $u=~\left(\begin{array}{c}
        x_1 \\ \vdots \\ x_n
    \end{array}\right)$ y $v=\left(\begin{array}{c}
        y_1 \\ \vdots \\ y_n
    \end{array}\right)$. Calculamos en primer lugar $f(u), f(v)$:
    \begin{equation*}
        f(u)=P\left(\begin{array}{c}
            x_1 \\ \vdots \\ x_n
        \end{array}\right) \Longrightarrow [f(u)]^t =  (x_1, \dots, x_n) P^t
        \qquad \qquad
        f(v)=P\left(\begin{array}{c}
            y_1 \\ \vdots \\ y_n
        \end{array}\right)        
    \end{equation*}

    Por tanto,
    \begin{equation*}
        g_1(u,v) = (x_1, \dots, x_n) A_1\left(\begin{array}{c}
        y_1 \\ \vdots \\ y_n
    \end{array}\right)
    \end{equation*}
    \begin{equation*}
        g_2(f(u),f(v)) = [f(u)]^tA_2f(v) = (x_1, \dots, x_n)P^t A_2 P\left(\begin{array}{c}
        y_1 \\ \vdots \\ y_n
    \end{array}\right)
    \end{equation*}

    Debido a la isometría, sabemos que $g_2(f(u),f(v)) = g_1(u,v)$. Por tanto, tenemos que $A_1 = P^tA_2P \Longrightarrow A_1$ es congruente a $A_2$.
\end{description}
    
\end{proof}

\section{Formas Cuadráticas}
\begin{definicion}[Forma Cuadrática]
    Sea $V^n(\bb{K})$ espacio vectorial, y sea $F:V^n\to \bb{K}$. Decimos que $F$ es una forma cuadrática si, al verla en coordenadas:
    \begin{equation*}
        F(x_1,\dots,x_n) = \sum_{i,j}a_{ij}x_ix_j \text{ para cierto }a_{ij}\in\bb{K}
    \end{equation*}
\end{definicion}

\begin{prop}
    Las formas cuadráticas son un tipo de métrica.
\end{prop}
\begin{proof}
    \begin{equation*}
        F(x_1,\dots,x_n) = \sum_{i,j}a_{ij}x_ix_j  = (x_1,\dots,x_n)\left( \begin{array}{cccc}
            a_{11} & \frac{a_{12}}{2} & \dots & \frac{a_{1n}}{2} \\
            \frac{a_{12}}{2} & a_{22} &  & \vdots \\
            \vdots & & \ddots & \vdots \\
            \frac{a_{1n}}{2} & \dots & \dots & a_{nn} \\
        \end{array}\right)
        \left(\begin{array}{c}
            x_1 \\ \vdots \\ x_n
        \end{array}\right) = x^tAx
    \end{equation*}

    Como podemos ver, $A$ es simétrica.
\end{proof}

\begin{teo}[Expresión reducida]
    Sea $V^n(\bb{K})$ y $F$ una forma cuadrática. Entonces, $\exists \mathcal{B}$ en la que $F$ se calcula de la siguiente forma:
    \begin{equation*}
        F(x_1, \dots, x_n) = x_1^2 + \dots + x_r^2 - x_{r+1}^2 - x_{r+s}^2
    \end{equation*}
    donde $r$ es la cantidad de $1$ del teorema de Sylvester y $s$ es el índice.

    A esta forma se le denomina la expresión reducia de $F$.
\end{teo}


\section{Ejercicios}
Los ejercicios resueltos del presente tema están disponibles en la sección \ref{sec:EjerciciosTema2}.
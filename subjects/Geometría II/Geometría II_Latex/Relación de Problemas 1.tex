\section{Diagonalización de Endomorfismos}\label{sec:EjerciciosTema1}


\begin{ejercicio}
    Dadas las siguientes matrices $A_i\in\mathcal{M}_3(\bb{R})$:
    \begin{equation*}
        A_1 = \left( \begin{array}{ccc}
            1 & 2 & 1 \\
            2 & 0 & 2 \\
            1 & 2 & 1 \\
        \end{array}\right) \qquad
        A_2 = \left( \begin{array}{ccc}
            1 & 0 & 1 \\
            1 & 2 & 3 \\
            1 & 1 & -1 \\
        \end{array}\right) \qquad
        A_3 = \left( \begin{array}{ccc}
            3 & -6 & 6 \\
            0 & 1 & 2 \\
            2 & -6 & 3 \\
        \end{array}\right)
    \end{equation*}
    \begin{equation*}
        A_4 = \left( \begin{array}{ccc}
            9 & -18 & 0 \\
            2 & -3 & 0 \\
            2 & -6 & 3 \\
        \end{array}\right) \qquad
        A_5 = \left( \begin{array}{ccc}
            6 & -9 & 3 \\
            1 & 0 & 1 \\
            2 & -6 & 3 \\
        \end{array}\right) \qquad
        A_6 = \left( \begin{array}{ccc}
            -3 & 18 & 3 \\
            -2 & 9 & 1 \\
            -1 & 3 & 3 \\
        \end{array}\right)
    \end{equation*}

    \begin{enumerate}
        \item Estudiar si $A_i$ son diagonalizables. En caso de serlo, encontrar una matriz diagonal $D_i$ y una matriz regular $P_i$ tal que $D_i=P_i^{-1}A_iP_i$.

        \begin{enumerate}
            \item Veamos si es diagonalizable $A_1$.\\
            \begin{equation*}\begin{split}
                P_{A_1}(\lambda) & = |A_1-\lambda_0 I| = \left| \begin{array}{ccc}
                1-\lambda_0 & 2 & 1 \\
                2 & -\lambda_0 & 2 \\
                1 & 2 & 1-\lambda_0 \\
                \end{array}\right| =
                \left| \begin{array}{ccc}
                -\lambda_0 & 2 & 1 \\
                0 & -\lambda_0 & 2 \\
                \lambda_0 & 2 & 1-\lambda_0 \\
                \end{array}\right| \\
                & = \lambda_0
                \left| \begin{array}{ccc}
                -1 & 2 & 1 \\
                0 & -\lambda_0 & 2 \\
                1 & 2 & 1-\lambda_0 \\
                \end{array}\right| = \lambda_0
                \left| \begin{array}{ccc}
                0 & 4 & 2-\lambda_0 \\
                0 & -\lambda_0 & 2 \\
                1 & 2 & 1-\lambda_0 \\
                \end{array}\right|\\
                & = \lambda_0(8+\lambda_0(2-\lambda_0)) = \lambda_0(-\lambda_0^2+2\lambda_0+8) = -\lambda_0(\lambda+2)(\lambda-4)
            \end{split}\end{equation*}

            Por tanto, los valores propios son: $\{0,-2,4\}$. Como los tres son distintos, sí es diagonalizable.
            \begin{table}[H]
                \centering
                \begin{tabular}{c|c|c}
                    Valores Propios & Mult. Alg. & Mult. Geom. \\ \hline 
                    0 & 1 & 1\\
                    $-$2 & 1 & 1\\
                    4 & 1 & 1\\
                \end{tabular}
                \caption{Valores propios con sus multiplicidades}
            \end{table}
            
            Calculemos las matrices $P_1$ y $D_1$.
    
            \begin{equation*}
            V_0 = \left\{x\in\bb{R}^3 \left| \begin{array}{c}
                 x_1+2x_2+x_3=0  \\
                 x_2+x_3 = 0
            \end{array}\right.\right\} = \mathcal{L}\left(\left\{ \left(\begin{array}{c}
                    1 \\
                    0 \\
                    -1 \\
               \end{array}\right)
               \right\}\right)
            \end{equation*}
    
            \begin{equation*}
            V_{-2} = \left\{x\in\bb{R}^3 \left| \begin{array}{c}
                 3x_1+2x_2+x_3=0  \\
                 x_1+x_2+x_3=0 \\
                 x_1+2x_2+3x_3 = 0
            \end{array}\right.\right\} = \mathcal{L}\left(\left\{ \left(\begin{array}{c}
                    1 \\
                    -2 \\
                    1 \\
               \end{array}\right)
               \right\}\right)
            \end{equation*}
    
            \begin{equation*}
            V_4 = \left\{x\in\bb{R}^3 \left| \begin{array}{c}
                 -3x_1+2x_2+x_3 = 0  \\
                 x_1-2x_2+x_3 =0 \\
                 x_1+2x_2-3x_3 = 0
            \end{array}\right.\right\} = \mathcal{L}\left(\left\{ \left(\begin{array}{c}
                    1 \\
                    1 \\
                    1 \\
               \end{array}\right)
               \right\}\right)
            \end{equation*}
    
            Por tanto, las matrices $P_1$ y $D_1$ son:
            \begin{equation*}
                D_1 = \left( \begin{array}{ccc}
                    0 & 0 & 0 \\
                    0 & -2 & 0 \\
                    0 & 0 & 4 \\
                \end{array}\right) \qquad 
                P_1 = \left( \begin{array}{ccc}
                    1 & 1 & 1 \\
                    0 & -2 & 1 \\
                    -1& 1 & 1 \\
                \end{array}\right)
            \end{equation*}

            \item Veamos si es diagonalizable $A_2$.\\
            \begin{equation*}\begin{split}
                P_{A_2}(\lambda) & = |A_2-\lambda_0 I| = \left| \begin{array}{ccc}
                1-\lambda_0 & 0 & 1 \\
                1 & 2-\lambda_0 & 3 \\
                1 & 1 & -1-\lambda_0 \\
                \end{array}\right| =\\
                & = -(1-\lambda_0)(1+\lambda_0)(2-\lambda_0) +1 -(2-\lambda_0)-3(1-\lambda_0) = \\
                &= -(1-\lambda_0^2)(2-\lambda_0) -4 +4\lambda_0 = -\lambda_0^3 +2\lambda_0^2+5\lambda_0-6
            \end{split}\end{equation*}

            \begin{figure}[H]
                \centering
                \polyhornerscheme[x=1]{-x^3+2x^2+5x-6} \hspace{1cm} 
                \polyhornerscheme[x=-2]{-x^2+x+6}
            \end{figure}

            Por tanto, $P_{A_2} = (\lambda_0-1)(\lambda_0+2)(-\lambda_0+3)$. Por ende, 
            los valores propios son: $\{1, -2, 3\}$. Como los tres son distintos, sí es diagonalizable.
            \begin{table}[H]
                \centering
                \begin{tabular}{c|c|c}
                    Valores Propios & Mult. Alg. & Mult. Geom. \\ \hline 
                    1 & 1 & 1\\
                    $-$2 & 1 & 1\\
                    3 & 1 & 1\\
                \end{tabular}
                \caption{Valores propios con sus multiplicidades}
            \end{table}
            
            Calculemos las matrices $P_2$ y $D_2$.
    
            \begin{equation*}
            V_1 = \left\{x\in\bb{R}^3 \left| \begin{array}{c}
                 x_3=0  \\
                 x_1+x_2+3x_3 = 0 \\
                 x_1+x_2-2x_3 = 0
            \end{array}\right.\right\} = \mathcal{L}\left(\left\{ \left(\begin{array}{c}
                    1 \\
                    -1 \\
                    0 \\
               \end{array}\right)
               \right\}\right)
            \end{equation*}
    
            \begin{equation*}
            V_{-2} = \left\{x\in\bb{R}^3 \left| \begin{array}{c}
                 3x_1+x_3=0  \\
                 x_1+4x_2+3x_3=0 \\
                 x_1+x_2+x_3 = 0
            \end{array}\right.\right\} = \mathcal{L}\left(\left\{ \left(\begin{array}{c}
                    1 \\
                    2 \\
                    -3 \\
               \end{array}\right)
               \right\}\right)
            \end{equation*}
    
            \begin{equation*}
            V_3 = \left\{x\in\bb{R}^3 \left| \begin{array}{c}
                 -2x_1+x_3 = 0  \\
                 x_1-x_2+3x_3 =0 \\
                 x_1+x_2-4x_3 = 0
            \end{array}\right.\right\} = \mathcal{L}\left(\left\{ \left(\begin{array}{c}
                    1 \\
                    7 \\
                    2 \\
               \end{array}\right)
               \right\}\right)
            \end{equation*}
    
            Por tanto, las matrices $P_2$ y $D_2$ son:
            \begin{equation*}
                D_2 = \left( \begin{array}{ccc}
                    1 & 0 & 0 \\
                    0 & -2 & 0 \\
                    0 & 0 & 3 \\
                \end{array}\right) \qquad 
                P_2 = \left( \begin{array}{ccc}
                    1 & 1 & 1 \\
                    -1 & 2 & 7 \\
                    0 & -3 & 2 \\
                \end{array}\right)
            \end{equation*}

            \item Veamos si es diagonalizable $A_3$.
            \begin{equation*}\begin{split}
                P_{A_3}(\lambda) & = |A_3-\lambda_0 I| = \left| \begin{array}{ccc}
                3-\lambda_0 & -6 & 6 \\
                0 & 1-\lambda_0 & 2 \\
                2 & -6 & 3-\lambda_0 \\
                \end{array}\right| =\\
                & = (3-\lambda_0)^2(1-\lambda_0)-24-12(1-\lambda_0)+12(3-\lambda_0) = \\
                &= 9+\lambda_0^2-6\lambda_0 -9\lambda_0-\lambda_0^3+6\lambda_0^2 -24 -12 +12\lambda_0 +36 -12\lambda_0 = \\
                &= -\lambda_0^3 + 7\lambda_0^2-15\lambda_0 + 9
            \end{split}\end{equation*}

            \begin{figure}[H]
                \centering
                \polyhornerscheme[x=1]{-x^3+7x^2-15x+9}
            \end{figure}

            Por tanto, $P_{A_3}  = -(\lambda_0-1)(\lambda_0-3)^2$. Los valores propios son: $\{1, 3\}$.

            Calculemos la multiplicidad geométrica de $3$.
            \begin{equation*}
            V_3 = \left\{x\in\bb{R}^3 \left| \begin{array}{c}
                 -x_2+x_3=0  \\
                 -x_2+x_3 = 0 \\
                 x_1-3x_2 = 0
            \end{array}\right.\right\} = \mathcal{L}\left(\left\{ \left(\begin{array}{c}
                    3 \\
                    1 \\
                    1 \\
               \end{array}\right)
               \right\}\right)
            \end{equation*}
            \begin{table}[H]
                \centering
                \begin{tabular}{c|c|c}
                    Valores Propios & Mult. Alg. & Mult. Geom. \\ \hline 
                    1 & 1 & 1\\
                    3 & 2 & 1\\
                \end{tabular}
                \caption{Valores propios con sus multiplicidades}
            \end{table}

            Por tanto, la multiplicidad geométrica de $3$ es $n_3=1 \neq 2 = m_3$. Por tanto, $A_3$ no es diagonalizable.

            \item Veamos si es diagonalizable $A_4$.\\
            \begin{equation*}\begin{split}
                P_{A_4}(\lambda) & = |A_4-\lambda_0 I| = \left| \begin{array}{ccc}
                9-\lambda_0 & -18 & 0 \\
                2 & -3-\lambda_0 & 0 \\
                2 & -6 & 3-\lambda_0 \\
                \end{array}\right| =\\
                & = (3-\lambda_0)\left(-(9-\lambda_0)(3+\lambda_0)+36\right) = (3-\lambda_0)(\lambda_0^2-6\lambda_0+9) \\
                &= (3-\lambda_0)(\lambda_0-3)^2 = -(\lambda_0-3)^3
            \end{split}\end{equation*}

            Por tanto, el único valor propio es $\{3\}$. Calculemos su multiplicidad geométrica.
            \begin{equation*}
            V_3 = \left\{x\in\bb{R}^3 \left| \begin{array}{c}
                 x_1-3x_2=0  \\
                 x_1-3x_2 = 0 \\
                 x_1-3x_2 = 0
            \end{array}\right.\right\} \neq \bb{R}^3
            \end{equation*}

            \begin{table}[H]
                \centering
                \begin{tabular}{c|c|c}
                    Valores Propios & Mult. Alg. & Mult. Geom. \\ \hline 
                    3 & 3 & 2\\
                \end{tabular}
                \caption{Valores propios con sus multiplicidades}
            \end{table}

            Por tanto, la multiplicidad geométrica de $3$ es $n_3=2 \neq 3 = m_3$. Por tanto, $A_4$ no es diagonalizable.

            \item Veamos si es diagonalizable $A_5$.\\
            \begin{equation*}\begin{split}
                P_{A_5}(\lambda) & = |A_5-\lambda_0 I| = \left| \begin{array}{ccc}
                6-\lambda_0 & -9 & 3 \\
                1 & -\lambda_0 & 1 \\
                2 & -6 & 3-\lambda_0 \\
                \end{array}\right| =\\
                & = -\lambda_0(3-\lambda_0)(6-\lambda_0)-18-18 +6\lambda_0+9(3-\lambda_0) +6(6-\lambda_0) =\\
                &= -18\lambda_0 +3\lambda_0^2 +6\lambda_0^2-\lambda_0^3 -36 +6\lambda_0+27-9\lambda_0 + 36-6\lambda_0 =\\
                &=-\lambda_0^3+9\lambda_0^2-27\lambda_0+27 = -(\lambda_0-3)(\lambda_0-3)^2 = -(\lambda_0-3)^3
            \end{split}\end{equation*}

            \begin{figure}[H]
                \centering
                \polyhornerscheme[x=3]{-x^3+9x^2-27x+27}
            \end{figure}

            Por tanto, el único valor propio es $\{3\}$. Calculemos su multiplicidad geométrica.
            \begin{equation*}
            V_3 = \left\{x\in\bb{R}^3 \left| \begin{array}{c}
                 x_1-3x_2+x_3=0  \\
                 x_1-3x_2+x_3=0  \\
                 x_1-2x_2 = 0
            \end{array}\right.\right\} \neq \bb{R}^3
            \end{equation*}

            \begin{table}[H]
                \centering
                \begin{tabular}{c|c|c}
                    Valores Propios & Mult. Alg. & Mult. Geom. \\ \hline 
                    3 & 3 & 1\\
                \end{tabular}
                \caption{Valores propios con sus multiplicidades}
            \end{table}

            Por tanto, la multiplicidad geométrica de $3$ es $n_3=1 \neq 3 = m_3$. Por tanto, $A_5$ no es diagonalizable.

            \item Veamos si es diagonalizable $A_6$.\\
            \begin{equation*}\begin{split}
                P_{A_6}(\lambda) & = |A_6-\lambda_0 I| = \left| \begin{array}{ccc}
                -3-\lambda_0 & 18 & 3 \\
                -2 & 9-\lambda_0 & 1 \\
                -1 & 3 & 3-\lambda_0 \\
                \end{array}\right| =\\
                & = -(9-\lambda_0^2)(9-\lambda_0)-18-18 +3(9-\lambda_0)+3(3+\lambda_0)+36(3-\lambda_0) =\\
                &= -81 +9\lambda_0 +9\lambda_0^2 -\lambda_0^3 -36 +27-3\lambda_0+9+3\lambda_0+108-36\lambda_0 =\\
                &= -\lambda_0^3 +9\lambda_0^2 -27\lambda_0 + 27 = -(\lambda_0-3)^3
            \end{split}\end{equation*}

            \begin{figure}[H]
                \centering
                \polyhornerscheme[x=3]{-x^3+9x^2-27x+27}
            \end{figure}

            Por tanto, el único valor propio es $\{3\}$. Calculemos su multiplicidad geométrica.
            \begin{equation*}
            V_3 = \left\{x\in\bb{R}^3 \left| \begin{array}{c}
                 -6x_1+18x_2+3x_3=0  \\
                 -2x_1+6x_2+x_3=0  \\
                 -x_1+3x_2 = 0
            \end{array}\right.\right\}
            = \left\{x\in\bb{R}^3 \left| \begin{array}{c}
                 -2x_1+6x_2+x_3=0  \\
                 -x_1+3x_2 = 0
            \end{array}\right.\right\}
            \end{equation*}

            \begin{table}[H]
                \centering
                \begin{tabular}{c|c|c}
                    Valores Propios & Mult. Alg. & Mult. Geom. \\ \hline 
                    3 & 3 & 1\\
                \end{tabular}
                \caption{Valores propios con sus multiplicidades}
            \end{table}

            Por tanto, la multiplicidad geométrica de $3$ es $n_3=1 \neq 3 = m_3$. Por tanto, $A_6$ no es diagonalizable.

        \end{enumerate}
        

         

        \item Estudiar cuáles de las matrices de la primera fila son semejantes entre sí.
        \begin{table}[H]
            \centering
            \begin{tabular}{r|ccc|l}
                 & $A_1$ & $A_2$ & $A_3$ & \\ \hline
                 $tr(A)$ & $2$ & $2$ & $7$ & $A_1 \nsim A_3 \quad \land \quad A_2 \nsim A_3$ \\
                 $det(A)$ & $0$ & $-6$ & $\times$ & $A_1 \nsim A_2$\\
                 $P_A(\lambda)$ & $\times$ & $\times$ & $\times$ &
            \end{tabular}
            \caption{Resolución usando propiedades de las matrices semejantes}
        \end{table}
        Además, como los tres polinomios característicos son distintos, no son semejantes.

        \item Estudiar si $A_4$ y $A_5$ son semejantes.\\
        
        Tienen el mismo rango, traza y polinomio característico. Por tanto, no podemos descartar que sean semejantes.

        \begin{table}[H]
            \centering
            \begin{tabular}{c|c|c|c}
                Matriz & Valores Propios & Mult. Alg. & Mult. Geom. \\ \hline 
                $A_4$ & 3 & 3 & 2\\
                $A_5$ & 3 & 3 & 1\\
            \end{tabular}
            \caption{Valores propios con sus multiplicidades para cada matriz}
        \end{table}
    
        Como tienen multiplicidades geométricas distintas para el mismo valor propio, entonces no representan el mismo endomorfismo. Por tanto, no son semejantes.
        $$A_4 \nsim A_5$$
        
    \end{enumerate}

\end{ejercicio}

\begin{ejercicio}
    Tomamos $\bb{K}=\bb{R}$. Para todo $a$ real consideramos la matriz
    \begin{equation*}
        A = \left( \begin{array}{ccc}
            -1+a & 1 & -1+a \\
            1-a & -a & 1-a \\
            -1 & -1 & -1 \\
        \end{array}\right)
    \end{equation*}
    \begin{enumerate}
        \item Estudiar los valores de $a\in \bb{R}$ para los que $A$ es diagonalizable. \\

        Obtengo en primer lugar su polinomio característico:
        \begin{equation*}
        \begin{split}
        P_A(\lambda) & = det(A-\lambda I) = \left| \begin{array}{ccc}
                -1+a-\lambda & 1 & -1+a \\
                1-a & -a-\lambda & 1-a \\
                -1 & -1 & -1-\lambda
            \end{array}\right| \stackrel{C'_1=C_1-C_3}{=} \\  
            & = \left| \begin{array}{ccc}
                -\lambda & 1 & -1+a \\
                0 & -a-\lambda & 1-a \\
                \lambda & -1 & -1-\lambda
            \end{array}\right| = \lambda
            \left| \begin{array}{ccc}
                -1 & 1 & -1+a \\
                0 & -a-\lambda & 1-a \\
                1 & -1 & -1-\lambda
            \end{array}\right| \stackrel{F'_1=F_1+F_3}{=} \\
            & =
            \lambda
            \left| \begin{array}{ccc}
                0 & 0 & -2+a-\lambda \\
                0 & -a-\lambda & 1-a \\
                1 & -1 & -1-\lambda
            \end{array}\right| = \lambda(a-2-\lambda)(-a-\lambda)
        \end{split}
        \end{equation*}
    
        Los valores propios son: $\{0,a-2,-a\}$. Los casos a tener en cuenta son:
       \begin{itemize}
           \item Dos valores propios son iguales.
           \begin{equation*} \begin{array}{ll}
               a-2=0 & \longrightarrow a=2 \\
               -a=0 & \longrightarrow a=0 \\
               a-2=-a & \longrightarrow a=1 \\
           \end{array}\end{equation*}
           Por tanto, si $a\neq0,1,2$, entonces los tres valores propios son distintos y, por tanto, $A$ es diagonalizable.
    
           \item Caso $a=1$.
           \begin{equation*}\begin{split}
               V_{-1} & = \left\{ \left(\begin{array}{c}
                    x_1 \\
                    x_2 \\
                    x_3
               \end{array}\right) \in \bb{R}^3 \mid (A+I)\left(\begin{array}{c}
                    x_1 \\
                    x_2 \\
                    x_3
               \end{array}\right) = 0 \right\} \\
               & = \left\{ \left(\begin{array}{c}
                    x_1 \\
                    x_2 \\
                    x_3
               \end{array}\right) \in \bb{R}^3 \mid \left( \begin{array}{ccc}
                1 & 1 & 0 \\
                0 & 0 & 0 \\
                -1 & -1 & 0 \\
            \end{array}\right) \left(\begin{array}{c}
                    x_1 \\
                    x_2 \\
                    x_3
               \end{array}\right) = 0 \right\} \\
               & = \left\{ \left(\begin{array}{c}
                    x_1 \\
                    x_2  \\
                    x_3
               \end{array}\right) \in \bb{R}^3 \mid x_1+x_2=0 \right\}
           \end{split}\end{equation*}
           \begin{table}[H]
                \centering
                \begin{tabular}{c|c|c}
                    Valores Propios & Mult. Alg. & Mult. Geom. \\ \hline 
                    0 & 1 & 1\\
                    $-$1 & 2 & 2\\
                \end{tabular}
                \caption{Valores propios con sus multiplicidades}
            \end{table}
            Por tanto, para $a=1$ la matriz es diagonalizable.

            \item Caso $a=0$.
           \begin{equation*}\begin{split}
               V_0 & = \left\{ \left(\begin{array}{c}
                    x_1 \\
                    x_2 \\
                    x_3
               \end{array}\right) \in \bb{R}^3 \mid A\left(\begin{array}{c}
                    x_1 \\
                    x_2 \\
                    x_3
               \end{array}\right) = 0 \right\} \\
               & = \left\{ \left(\begin{array}{c}
                    x_1 \\
                    x_2 \\
                    x_3
               \end{array}\right) \in \bb{R}^3 \mid \left( \begin{array}{ccc}
                -1 & 1 & -1 \\
                1 & 0 & 1 \\
                -1 & -1 & -1 \\
            \end{array}\right) \left(\begin{array}{c}
                    x_1 \\
                    x_2 \\
                    x_3
               \end{array}\right) = 0 \right\} \\
               & = \left\{ \left(\begin{array}{c}
                    x_1 \\
                    x_2  \\
                    x_3
               \end{array}\right) \in \bb{R}^3 \left|
               \begin{array}{c}
                   -x_1+x_2-x_3 = 0  \\
                    x_1+x_3 = 0\\
                    x_1+x_2+x_3 = 0
               \end{array}\right.\right\} =
               \mathcal{L}\left(\left\{ \left(\begin{array}{c}
                    1 \\
                    0 \\
                    -1 \\
               \end{array}\right)
               \right\}\right)
           \end{split}\end{equation*}
           \begin{table}[H]
                \centering
                \begin{tabular}{c|c|c}
                    Valores Propios & Mult. Alg. & Mult. Geom. \\ \hline 
                    0 & 2 & 1\\
                    $-$2 & 1 & 1\\
                \end{tabular}
                \caption{Valores propios con sus multiplicidades}
            \end{table}
            Por tanto, para $a=0$ la matriz no es diagonalizable.

            \item Caso $a=2$.
           \begin{equation*}\begin{split}
               V_0 & = \left\{ \left(\begin{array}{c}
                    x_1 \\
                    x_2 \\
                    x_3
               \end{array}\right) \in \bb{R}^3 \mid A\left(\begin{array}{c}
                    x_1 \\
                    x_2 \\
                    x_3
               \end{array}\right) = 0 \right\} \\
               & = \left\{ \left(\begin{array}{c}
                    x_1 \\
                    x_2 \\
                    x_3
               \end{array}\right) \in \bb{R}^3 \mid \left( \begin{array}{ccc}
                1 & 1 & 1 \\
                -1 & -2 & -1 \\
                -1 & -1 & -1 \\
            \end{array}\right) \left(\begin{array}{c}
                    x_1 \\
                    x_2 \\
                    x_3
               \end{array}\right) = 0 \right\} \\
               & = \left\{ \left(\begin{array}{c}
                    x_1 \\
                    x_2  \\
                    x_3
               \end{array}\right) \in \bb{R}^3 \left|
               \begin{array}{c}
                   x_1+x_2+x_3 = 0  \\
                    x_1+2x_2+x_3 = 0\\
                    x_1+x_2+x_3 = 0
               \end{array}\right.\right\} =
               \mathcal{L}\left(\left\{ \left(\begin{array}{c}
                    1 \\
                    0 \\
                    -1 \\
               \end{array}\right)
               \right\}\right)
           \end{split}\end{equation*}
           \begin{table}[H]
                \centering
                \begin{tabular}{c|c|c}
                    Valores Propios & Mult. Alg. & Mult. Geom. \\ \hline 
                    0 & 2 & 1\\
                    $-$2 & 1 & 1\\
                \end{tabular}
                \caption{Valores propios con sus multiplicidades}
            \end{table}
            Por tanto, para $a=2$ la matriz no es diagonalizable.
       \end{itemize} 

       Por tanto, $A$ es diagonalizable $\forall a \in \bb{R}-\{0,2\}$.
       

        \item Diagonalizar la matriz para $a=0$, $a=1$ y $a=-1$ (si ello es posible).
        \begin{itemize}
            \item Para $a=0$\\
            $A$ no es diagonalizable.

            \item Para $a=1$\\
            Los valores propios son: $\{0, -1\}$.
            \begin{equation*}\begin{split}
            V_0 & = \left\{ \left(\begin{array}{c}
                    x_1 \\
                    x_2 \\
                    x_3
               \end{array}\right) \in \bb{R}^3 \mid A\left(\begin{array}{c}
                    x_1 \\
                    x_2 \\
                    x_3
               \end{array}\right) = 0 \right\} \\
               & = \left\{ \left(\begin{array}{c}
                    x_1 \\
                    x_2 \\
                    x_3
               \end{array}\right) \in \bb{R}^3 \mid \left( \begin{array}{ccc}
                0 & 1 & 0 \\
                0 & -1 & 0 \\
                -1 & -1 & -1 \\
            \end{array}\right) \left(\begin{array}{c}
                    x_1 \\
                    x_2 \\
                    x_3
               \end{array}\right) = 0 \right\} \\
               & = \left\{ \left(\begin{array}{c}
                    x_1 \\
                    x_2  \\
                    x_3
               \end{array}\right) \in \bb{R}^3 \left|
               \begin{array}{c}
                    x_2=0  \\
                    x_1+x_2+x_3 = 0 
               \end{array}\right. \right\} =
               \mathcal{L}\left(\left\{ \left(\begin{array}{c}
                    1 \\
                    0 \\
                    -1 \\
               \end{array}\right)
               \right\}\right)
           \end{split}\end{equation*}
           \begin{equation*}\begin{split}
            V_{-1} & = \dots =
               \mathcal{L}\left(\left\{ \left(\begin{array}{c}
                    0 \\
                    0 \\
                    1 \\
               \end{array}\right),
               \left(\begin{array}{c}
                    1 \\
                    -1 \\
                    0 \\
               \end{array}\right)
               \right\}\right)
           \end{split}\end{equation*}
           Por tanto, las matrices $P_{1}$ y $D_{1}$ son:
            \begin{equation*}
                D_{1} = \left( \begin{array}{ccc}
                    0 & 0 & 0 \\
                    0 & -1 & 0 \\
                    0 & 0 & -1 \\
                \end{array}\right) \qquad 
                P_{1} = \left( \begin{array}{ccc}
                    1 & 0 & 1 \\
                    0 & 0 & -1 \\
                    -1& 1 & 0 \\
                \end{array}\right)
            \end{equation*}

            \item Para $a=-1$\\
            Los valores propios son: $\{0, -3, 1\}$.
            \begin{equation*}\begin{split}
            V_0 & = \left\{ \left(\begin{array}{c}
                    x_1 \\
                    x_2 \\
                    x_3
               \end{array}\right) \in \bb{R}^3 \mid A\left(\begin{array}{c}
                    x_1 \\
                    x_2 \\
                    x_3
               \end{array}\right) = 0 \right\} \\
               & = \left\{ \left(\begin{array}{c}
                    x_1 \\
                    x_2 \\
                    x_3
               \end{array}\right) \in \bb{R}^3 \mid \left( \begin{array}{ccc}
                -2 & 1 & -2 \\
                2 & 1 & 2 \\
                -1 & -1 & -1 \\
            \end{array}\right) \left(\begin{array}{c}
                    x_1 \\
                    x_2 \\
                    x_3
               \end{array}\right) = 0 \right\} \\
               & = \left\{ \left(\begin{array}{c}
                    x_1 \\
                    x_2  \\
                    x_3
               \end{array}\right) \in \bb{R}^3 \left|
               \begin{array}{c}
                    2x_1-x_2+2x_3=0  \\
                    x_1+x_2+x_3 = 0 
               \end{array}\right. \right\} =
               \mathcal{L}\left(\left\{ \left(\begin{array}{c}
                    1 \\
                    0 \\
                    -1 \\
               \end{array}\right)
               \right\}\right)
           \end{split}\end{equation*}
           \begin{equation*}\begin{split}
            V_1 & = \left\{ \left(\begin{array}{c}
                    x_1 \\
                    x_2 \\
                    x_3
               \end{array}\right) \in \bb{R}^3 \mid (A-I)\left(\begin{array}{c}
                    x_1 \\
                    x_2 \\
                    x_3
               \end{array}\right) = 0 \right\} \\
               & = \left\{ \left(\begin{array}{c}
                    x_1 \\
                    x_2 \\
                    x_3
               \end{array}\right) \in \bb{R}^3 \mid \left( \begin{array}{ccc}
                -3 & 1 & -2 \\
                2 & 0 & 2 \\
                -1 & -1 & -2 \\
            \end{array}\right) \left(\begin{array}{c}
                    x_1 \\
                    x_2 \\
                    x_3
               \end{array}\right) = 0 \right\} \\
               & = \left\{ \left(\begin{array}{c}
                    x_1 \\
                    x_2  \\
                    x_3
               \end{array}\right) \in \bb{R}^3 \left|
               \begin{array}{c}
                    3x_1-x_2+2x_3=0  \\
                    x_1+x_3=0 \\
                    x_1+x_2+2x_3 = 0 
               \end{array}\right. \right\} =
               \mathcal{L}\left(\left\{ \left(\begin{array}{c}
                    1 \\
                    1 \\
                    -1 \\
               \end{array}\right)
               \right\}\right)
           \end{split}\end{equation*}
           \begin{equation*}\begin{split}
            V_{-3} & = \left\{ \left(\begin{array}{c}
                    x_1 \\
                    x_2 \\
                    x_3
               \end{array}\right) \in \bb{R}^3 \mid (A+3I)\left(\begin{array}{c}
                    x_1 \\
                    x_2 \\
                    x_3
               \end{array}\right) = 0 \right\} \\
               & = \left\{ \left(\begin{array}{c}
                    x_1 \\
                    x_2 \\
                    x_3
               \end{array}\right) \in \bb{R}^3 \mid \left( \begin{array}{ccc}
                1 & 1 & -2 \\
                2 & 4 & 2 \\
                -1 & -1 & 2 \\
            \end{array}\right) \left(\begin{array}{c}
                    x_1 \\
                    x_2 \\
                    x_3
               \end{array}\right) = 0 \right\} \\
               & = \left\{ \left(\begin{array}{c}
                    x_1 \\
                    x_2  \\
                    x_3
               \end{array}\right) \in \bb{R}^3 \left|
               \begin{array}{c}
                    x_1+x_2-2x_3=0  \\
                    x_1+2x_2+x_3 = 0 
               \end{array}\right. \right\} =
               \mathcal{L}\left(\left\{ \left(\begin{array}{c}
                    5 \\
                    -3 \\
                    1 \\
               \end{array}\right)
               \right\}\right)
           \end{split}\end{equation*}
            
           Por tanto, las matrices $P_{-1}$ y $D_{-1}$ son:
            \begin{equation*}
                D_{-1} = \left( \begin{array}{ccc}
                    0 & 0 & 0 \\
                    0 & 1 & 0 \\
                    0 & 0 & -3 \\
                \end{array}\right) \qquad 
                P_{-1} = \left( \begin{array}{ccc}
                    1 & 1 & 5 \\
                    0 & 1 & -3 \\
                    -1& -1 & 1 \\
                \end{array}\right)
            \end{equation*}
            
        \end{itemize}
        

        \item Razonar si las matrices obtenidas para $a=-2$ y para $a=4$ son semejantes.

        Para $a=-2$, los valores propios son: $\{0,2,-4\}$.

        Para $a=4$, los valores propios son: $\{0,2,-4\}$.

        Por tanto, como tienen los mismos valores propios, saldrá la misma $D$ al diagonalizarla. Por tanto, $A_{-2}\sim D \land D\sim A_4$. Al ser $\sim$ una relación de equivalencia, $A_4 \sim A_{-2}$
        
    \end{enumerate}
\end{ejercicio}

\begin{ejercicio}
    Estudiar los valores de $a$ para los que la siguiente matriz es diagonalizable. Estudiar el caso real y el caso complejo.
    \begin{equation*}
        A = \left(\begin{array}{ccc}
            1 & -2 & -2 \\
            -2 & a & 8 \\
            2 & 8 & a
        \end{array}\right)
    \end{equation*}

    \begin{equation*}\begin{split}
        P_A(\lambda) & = \left|\begin{array}{ccc}
            1-\lambda & -2 & -2 \\
            -2 & a-\lambda & 8 \\
            2 & 8 & a-\lambda
        \end{array}\right| = \left|\begin{array}{ccc}
            1-\lambda & -2 & 0 \\
            -2 & a-\lambda & 8-a+\lambda \\
            2 & 8 & -8 + a-\lambda
        \end{array}\right| = \\
        &= \left|\begin{array}{ccc}
            1-\lambda & -2 & 0 \\
            0 & 8+a-\lambda & 0 \\
            2 & 8 & -8 + a-\lambda
        \end{array}\right| = (-8+a-\lambda)(8+a-\lambda)(1-\lambda)
    \end{split}\end{equation*}

    Por tanto, los valores propios son: $\{-8+a, 8+a, 1\}$. Los casos a tener en cuenta son:
    \begin{itemize}
       \item Dos valores propios son iguales.
       \begin{equation*} \begin{array}{ll}
           -8+a=1 & \longrightarrow a=9 \\
           8+a=1 & \longrightarrow a=-7 \\
           8+a=-8+a & \longrightarrow \nexists sol \\
       \end{array}\end{equation*}
       Por tanto, si $a\neq-7,9$, entonces los tres valores propios son distintos y, por tanto, $A$ es diagonalizable.

       \item Caso $a=-7$.
           \begin{equation*}\begin{split}
               V_{1} & = \left\{ \left(\begin{array}{c}
                    x_1 \\
                    x_2 \\
                    x_3
               \end{array}\right) \in \bb{R}^3 \mid (A-I)\left(\begin{array}{c}
                    x_1 \\
                    x_2 \\
                    x_3
               \end{array}\right) = 0 \right\} \\
               & = \left\{ \left(\begin{array}{c}
                    x_1 \\
                    x_2 \\
                    x_3
               \end{array}\right) \in \bb{R}^3 \mid \left( \begin{array}{ccc}
                0 & -2 & -2 \\
                -2 & -8 & 8 \\
                2 & 8 & -8 \\
            \end{array}\right) \left(\begin{array}{c}
                    x_1 \\
                    x_2 \\
                    x_3
               \end{array}\right) = 0 \right\} \\
               & = \left\{ \left(\begin{array}{c}
                    x_1 \\
                    x_2  \\
                    x_3
               \end{array}\right) \in \bb{R}^3 \left| \begin{array}{c}
                    x_2+x_3=0 \\
                    x_1+4x_2-4x_3 = 0
               \end{array}\right| \right\}
           \end{split}\end{equation*}
           \begin{table}[H]
                \centering
                \begin{tabular}{c|c|c}
                    Valores Propios & Mult. Alg. & Mult. Geom. \\ \hline 
                    1 & 2 & 1\\
                    $-$15 & 1 & 1\\
                \end{tabular}
                \caption{Valores propios con sus multiplicidades}
            \end{table}
            Por tanto, para $a=-7$ la matriz no es diagonalizable.
            
        \item Caso $a=9$.
           \begin{equation*}\begin{split}
               V_{1} & = \left\{ \left(\begin{array}{c}
                    x_1 \\
                    x_2 \\
                    x_3
               \end{array}\right) \in \bb{R}^3 \mid (A-I)\left(\begin{array}{c}
                    x_1 \\
                    x_2 \\
                    x_3
               \end{array}\right) = 0 \right\} \\
               & = \left\{ \left(\begin{array}{c}
                    x_1 \\
                    x_2 \\
                    x_3
               \end{array}\right) \in \bb{R}^3 \mid \left( \begin{array}{ccc}
                0 & -2 & -2 \\
                -2 & 8 & 8 \\
                2 & 8 & 8 \\
            \end{array}\right) \left(\begin{array}{c}
                    x_1 \\
                    x_2 \\
                    x_3
               \end{array}\right) = 0 \right\} \\
               & = \left\{ \left(\begin{array}{c}
                    x_1 \\
                    x_2  \\
                    x_3
               \end{array}\right) \in \bb{R}^3 \left| \begin{array}{c}
                    x_2+x_3=0 \\
                    x_1+4x_2+4x_3 = 0 \\
                    -x_1+4x_2+4x_3 = 0
               \end{array}\right. \right\}
           \end{split}\end{equation*}
           \begin{table}[H]
                \centering
                \begin{tabular}{c|c|c}
                    Valores Propios & Mult. Alg. & Mult. Geom. \\ \hline 
                    1 & 2 & 1\\
                    17 & 1 & 1\\
                \end{tabular}
                \caption{Valores propios con sus multiplicidades}
            \end{table}
            Por tanto, para $a=9$ la matriz no es diagonalizable.
    \end{itemize}

    Por tanto, $A$ es diagonalizable $\forall a \in \bb{C} \backslash \{-7,9\}$.
\end{ejercicio}

\begin{ejercicio}
    Sea $A\in \mathcal{M}_4(\bb{R})$. Estudiar los valores de $a\in\bb{R}$ para los que la matriz $A$ es diagonalizable.
    \begin{equation*}
        A = \left( \begin{array}{cccc}
            1 & 0 & 1 & 0 \\
            0 & a & 0 & a \\
            1 & 1 & 1 & 0 \\
            0 & a & 0 & 1 \\
        \end{array}\right)
    \end{equation*}

    \begin{equation*}\begin{split}
        P_A(\lambda) & = \left|\begin{array}{cccc}
            1-\lambda & 0 & 1 & 0 \\
            0 & a-\lambda & 0 & a \\
            1 & 1 & 1-\lambda & 0 \\
            0 & a & 0 & 1-\lambda \\
        \end{array} \right| = \left|\begin{array}{cccc}
            2-\lambda & 0 & 1 & 0 \\
            0 & a-\lambda & 0 & a \\
            2-\lambda & 1 & 1-\lambda & 0 \\
            0 & a & 0 & 1-\lambda \\
        \end{array} \right| = \\
        & = \left|\begin{array}{cccc}
            0 & -1 & \lambda & 0 \\
            0 & a-\lambda & 0 & a \\
            2-\lambda & 1 & 1-\lambda & 0 \\
            0 & a & 0 & 1-\lambda \\
        \end{array} \right|
        = (2-\lambda) \left|\begin{array}{ccc}
            -1 & \lambda & 0 \\
            a-\lambda & 0 & a \\
            a & 0 & 1-\lambda \\
        \end{array} \right| = \\
        & = -\lambda(2-\lambda)((a-\lambda)(1-\lambda)-a^2) = -\lambda(2-\lambda)(\lambda^2-(a+1)\lambda-a^2+a)
    \end{split}\end{equation*}

    Veo el número de soluciones de la ecuación $\lambda^2-(a+1)\lambda-a^2+a = 0$
    \begin{equation*}
        \Delta = (a+1)^2 +4a^2-4a = (a-1)^2+4a^2 > 0
    \end{equation*}
    
    Por tanto, la ecuación tiene dos soluciones. Para ver los casos en los que los valores propios se repiten, veamos si $$\exists a\in \bb{R} \mid \lambda^2-(a+1)\lambda-a^2+a = 0, \text{ con } \lambda=0,2$$
    \begin{itemize}
        \item \underline{$\lambda=0$}:
        $\Longrightarrow -a^2+a = a(-a+1) = 0 \Longrightarrow a = 0,1$

        \item \underline{$\lambda=2$}:
        $\Longrightarrow 4-2a-2-a^2+a = -a^2-a+2 = 0 \Longrightarrow a = 1,-2$
    \end{itemize}

    Por tanto,
    \begin{itemize}
        \item \underline{Si $a\neq \{-2,0,1\}$}:\\
        Hay $4$ valores propios distintos, por lo que $A$ es diagonalizable.

        \item \underline{Si $a=0$}:\\
        Hay dos valores propios con $m_i=1$, pero la multiplicidad algebraica del valor propio $0$ es doble ($m_0=2$). Veamos su multiplicidad geométrica:
        \begin{equation*}\begin{split}
               V_{0} & = \left\{ \left(\begin{array}{c}
                    x_1 \\
                    x_2 \\
                    x_3 \\
                    x_4
               \end{array}\right) \in \bb{R}^4 \mid A\left(\begin{array}{c}
                    x_1 \\
                    x_2 \\
                    x_3 \\
                    x_4
               \end{array}\right) = 0 \right\} \\
               & = \left\{ \left(\begin{array}{c}
                    x_1 \\
                    x_2 \\
                    x_3 \\
                    x_4
               \end{array}\right) \in \bb{R}^4 \mid \left( \begin{array}{cccc}
                    1 & 0 & 1 & 0 \\
                    0 & 0 & 0 & 0 \\
                    1 & 1 & 1 & 0 \\
                    0 & 0 & 0 & 1 \\
            \end{array}\right) \left(\begin{array}{c}
                    x_1 \\
                    x_2 \\
                    x_3 \\
                    x_4
               \end{array}\right) = 0 \right\} \\
               & = \left\{ \left(\begin{array}{c}
                    x_1 \\
                    x_2  \\
                    x_3 \\
                    x_4
               \end{array}\right) \in \bb{R}^4 \left| \begin{array}{c}
                    x_1+x_3=0 \\
                    x_1+x_2+x_3=0\\
                    x_4=0
               \end{array}\right. \right\}
       \end{split}\end{equation*}
        \begin{table}[H]
            \centering
            \begin{tabular}{c|c|c}
                Valores Propios & Mult. Alg. & Mult. Geom. \\ \hline 
                2 & 1 & 1\\
                0 & 2 & 1\\
                - & 1 & 1
            \end{tabular}
            \caption{Valores propios con sus multiplicidades}
        \end{table}
        Por tanto, para $a=0$, $A$ no es diagonalizable.

        \item \underline{Si $a=1$}:\\
        Hay dos valores propios con $m_i=2$. Veamos su multiplicidad geométrica:
        \begin{equation*}\begin{split}
               V_{0} & = \left\{ \left(\begin{array}{c}
                    x_1 \\
                    x_2 \\
                    x_3 \\
                    x_4
               \end{array}\right) \in \bb{R}^4 \mid A\left(\begin{array}{c}
                    x_1 \\
                    x_2 \\
                    x_3 \\
                    x_4
               \end{array}\right) = 0 \right\} \\
               & = \left\{ \left(\begin{array}{c}
                    x_1 \\
                    x_2 \\
                    x_3 \\
                    x_4
               \end{array}\right) \in \bb{R}^4 \mid \left( \begin{array}{cccc}
                    1 & 0 & 1 & 0 \\
                    0 & 1 & 0 & 1 \\
                    1 & 1 & 1 & 0 \\
                    0 & 1 & 0 & 1 \\
            \end{array}\right) \left(\begin{array}{c}
                    x_1 \\
                    x_2 \\
                    x_3 \\
                    x_4
               \end{array}\right) = 0 \right\} \\
               & = \left\{ \left(\begin{array}{c}
                    x_1 \\
                    x_2  \\
                    x_3 \\
                    x_4
               \end{array}\right) \in \bb{R}^4 \left| \begin{array}{c}
                    x_1+x_3=0 \\
                    x_1+x_2+x_3=0\\
                    x_2+x_4=0
               \end{array}\right. \right\}
       \end{split}\end{equation*}
        \begin{table}[H]
            \centering
            \begin{tabular}{c|c|c}
                Valores Propios & Mult. Alg. & Mult. Geom. \\ \hline 
                0 & 2 & 1\\
                2 & 2 & -\\
            \end{tabular}
            \caption{Valores propios con sus multiplicidades}
        \end{table}
        Por tanto, para $a=1$, $A$ no es diagonalizable.

        \item \underline{Si $a=-2$}:\\
        Hay dos valores propios con $m_i=1$, pero la multiplicidad algebraica del valor propio $2$ es doble ($m_2=2$). Veamos su multiplicidad geométrica:
        \begin{equation*}\begin{split}
               V_{2} & = \left\{ \left(\begin{array}{c}
                    x_1 \\
                    x_2 \\
                    x_3 \\
                    x_4
               \end{array}\right) \in \bb{R}^4 \mid (A-2I)\left(\begin{array}{c}
                    x_1 \\
                    x_2 \\
                    x_3 \\
                    x_4
               \end{array}\right) = 0 \right\} \\
               & = \left\{ \left(\begin{array}{c}
                    x_1 \\
                    x_2 \\
                    x_3 \\
                    x_4
               \end{array}\right) \in \bb{R}^4 \mid \left( \begin{array}{cccc}
                    -1 & 0 & 1 & 0 \\
                    0 & -4 & 0 & -2 \\
                    1 & 1 & -1 & 0 \\
                    0 & -2 & 0 & -1 \\
            \end{array}\right) \left(\begin{array}{c}
                    x_1 \\
                    x_2 \\
                    x_3 \\
                    x_4
               \end{array}\right) = 0 \right\} \\
               & = \left\{ \left(\begin{array}{c}
                    x_1 \\
                    x_2  \\
                    x_3 \\
                    x_4
               \end{array}\right) \in \bb{R}^4 \left| \begin{array}{c}
                    -x_1+x_3=0 \\
                    x_1+x_2-x_3=0\\
                    2x_2+x_4=0
               \end{array}\right. \right\}
       \end{split}\end{equation*}
        \begin{table}[H]
            \centering
            \begin{tabular}{c|c|c}
                Valores Propios & Mult. Alg. & Mult. Geom. \\ \hline 
                2 & 2 & 1\\
                0 & 1 & 1\\
                - & 1 & 1
            \end{tabular}
            \caption{Valores propios con sus multiplicidades}
        \end{table}
        Por tanto, para $a=-2$, $A$ no es diagonalizable.
        
        
    \end{itemize}
    
\end{ejercicio}

\begin{ejercicio}
    Sea $A\in \mathcal{M}_n(\bb{R})$ diagonalizable. Demostrar que su matriz transpuesta también lo es. Razonar que, en ese caso, $A\sim A^t$.
    \begin{multline*}
        A \text{ diagonalizable } \Longrightarrow D = P^{-1}AP \Longrightarrow D^t = (P^{-1}AP)^t = P^t A^t (P^{-1})^t = P^t A^t (P^t)^{-1}
    \end{multline*}
    
    Además, como $D$ es diagonal, $D^t=D$. Por tanto,
    \begin{equation*}
        D = P^t A^t ((P^t)^{-1}
    \end{equation*}

    Por tanto, como $D$ es diagonal y $P^t$ es regular, $A^t$ es diagonalizable.

    Como $A\sim D \land D\sim A^t$, al ser $\sim$ una relación de equivalencia, $A \sim A^t$. $\hfill \qed$
\end{ejercicio}

\begin{ejercicio}
    Demostrar que toda $A\in \mathcal{S}_2(\bb{R})$ es diagonalizable.
    \begin{proof}
        Sea $ A=\left(\begin{array}{cc}
            a & b \\
            b & c
        \end{array}\right)$.
    
        Calculamos su polinomio característico.
        \begin{equation*}
            P_A(\lambda) = \lambda^2 - tr(A)\lambda + det(A) = \lambda^2-(a+c)\lambda + (ac-b^2)
        \end{equation*}
    
        Su discriminante es $\Delta=(a+c)^2-4(ac-b^2) = a^2+c^2+2ac-4ac+b^2 = a^2+c^2-2ac+b^2 = (a-c)^2+4b^2 \geq 0$.
        \begin{itemize}
            \item Si $\Delta > 0 \Longrightarrow A$ tiene 2 soluciones $\Longrightarrow A$ es diagonalizable.
    
            \item Si $\Delta = 0 \Longrightarrow a=c,\quad b=0 \Longrightarrow A=\left(\begin{array}{cc}
            a & 0 \\
            0 & a
        \end{array}\right)$ es diagonalizable. 
        \end{itemize}
    \end{proof}
\end{ejercicio}

\begin{observacion} En el caso de $\bb{K}=\bb{C}$ esto no es cierto. Como contraejemplo, ver el ejercicio \ref{Ejercicio7}.\ref{Ejercicio7.1}.
\end{observacion}


\begin{ejercicio}
\label{Ejercicio7}
    Sean $A_1, A_2, A_3 \in \mathcal{M}_2(\bb{C})$. Estudiar cuáles son diagonalizables.
    \begin{equation*}
        A_1 = \left( \begin{array}{cc}
            1 & i \\
            i & -1 \\
        \end{array}\right) \qquad
        A_2 = \left( \begin{array}{cc}
            1 & i \\
            -i & 1 \\
        \end{array}\right) \qquad
        A_3 = \left( \begin{array}{cc}
            1 & i \\
            i & 1 \\
        \end{array}\right)
    \end{equation*}

    \begin{enumerate}
        \item Veamos si $A_1$ es diagonalizable.
        \label{Ejercicio7.1}
        \begin{equation*}
            P_{A_1}(\lambda) = |A_1-\lambda I| = \lambda^2 - tr(A_1)\lambda + det(A_1) = \lambda^2
        \end{equation*}
        Por tanto, el único valor propio es el $\{0\}$. Calculemos su multiplicidad geométrica.
        \begin{equation*}\begin{split}
               V_{0} & = \left\{ \left(\begin{array}{c}
                    x_1 \\
                    x_2
               \end{array}\right) \in \bb{C}^2 \mid A\left(\begin{array}{c}
                    x_1 \\
                    x_2
               \end{array}\right) = 0 \right\} \\
               & = \left\{ \left(\begin{array}{c}
                    x_1 \\
                    x_2
               \end{array}\right) \in \bb{C}^2 \mid \left( \begin{array}{cc}
                    1 & i \\
                    i & -1 \\
                \end{array}\right) \left(\begin{array}{c}
                    x_1 \\
                    x_2
               \end{array}\right) = 0 \right\} \\
               & = \left\{ \left(\begin{array}{c}
                    x_1 \\
                    x_2
               \end{array}\right) \in \bb{C}^2 \left| \begin{array}{c}
                    x_1+ix_2=0 \\
                    ix_1-x_2 = 0 \\
               \end{array}\right. \right\}
               = \left\{ \left(\begin{array}{c}
                    x_1 \\
                    x_2
               \end{array}\right) \in \bb{C}^2 \left| \begin{array}{c}
                    x_1+ix_2=0 \\
               \end{array}\right. \right\}
           \end{split}\end{equation*}
           \begin{table}[H]
                \centering
                \begin{tabular}{c|c|c}
                    Valores Propios & Mult. Alg. & Mult. Geom. \\ \hline 
                    0 & 2 & 1\\
                \end{tabular}
                \caption{Valores propios con sus multiplicidades}
            \end{table}

            Por tanto, $A_1$ no es diagonalizable.
        
        \item Veamos si $A_2$ es diagonalizable.
        \begin{equation*}
            P_{A_2}(\lambda) = |A_2-\lambda I| = \lambda^2 - tr(A_2)\lambda + det(A_2) = \lambda^2 -2\lambda = \lambda(\lambda-2)
        \end{equation*}
        Por tanto, los valores propios son $\{0,2\}$.
           \begin{table}[H]
                \centering
                \begin{tabular}{c|c|c}
                    Valores Propios & Mult. Alg. & Mult. Geom. \\ \hline 
                    0 & 1 & 1\\
                    2 & 1 & 1\\
                \end{tabular}
                \caption{Valores propios con sus multiplicidades}
            \end{table}
            
            Por tanto, $A_2$ sí es diagonalizable.
            
        \item Veamos si $A_3$ es diagonalizable.
        \begin{equation*}
            P_{A_3}(\lambda) = |A_3-\lambda I| = \lambda^2 - tr(A_3)\lambda + det(A_3) = \lambda^2 -2\lambda  +2 = (\lambda-1-i)(\lambda-1+i)
        \end{equation*}
        Por tanto, los valores propios son $\{1-i,1+i\}$.
           \begin{table}[H]
                \centering
                \begin{tabular}{c|c|c}
                    Valores Propios & Mult. Alg. & Mult. Geom. \\ \hline 
                    $1-i$ & 1 & 1\\
                    $1+i$ & 1 & 1\\
                \end{tabular}
                \caption{Valores propios con sus multiplicidades}
            \end{table}
            Por tanto, $A_3$ sí es diagonalizable.
    \end{enumerate}
\end{ejercicio}

\begin{ejercicio}
    Sea $A\in \mathcal{M}_2(\bb{C})$. Consideramos la matriz traspuesta $A^t$, la matriz conjugada $\bar{A}$ y la matriz traspuesta conjugada $\bar{A}^t$,
    \begin{equation*}
        A=\left(\begin{array}{cc}
            a+ib & c+id \\
            e+if & g+ih
        \end{array}\right) \qquad
        A^t=\left(\begin{array}{cc}
            a+ib & e+if \\
            c+id & g+ih
        \end{array}\right)
    \end{equation*}
    \begin{equation*}
        \bar{A}=\left(\begin{array}{cc}
            a-ib & c-id \\
            e-if & g-ih
        \end{array}\right) \qquad
        \bar{A}^t=\left(\begin{array}{cc}
            a-ib & e-if \\
            c-id & g-ih
        \end{array}\right)
    \end{equation*}
    Decimos que $A$ es \emph{simétrica} si $A=A^t$ y que es \emph{hermítica} si $A=\bar{A}^t$.

    \begin{enumerate}
        \item Demostrar que una matriz simétrica compleja no es necesariamente diagonalizable.

        Como contraejemplo, ver el ejercicio \ref{Ejercicio7}.\ref{Ejercicio7.1}.

        \item Demostrar que toda matriz hermítica es diagonalizable y que sus valores propios son reales.

        \begin{equation*}
            A=\bar{A}^t \Longrightarrow \left\{
            \begin{array}{ll}
                a+ib = a-ib & \longrightarrow b=0 \\
                c+id = e-if & \longrightarrow c=e-if-id \\
                e+if = c-id & \\
                g+ih = g-ih & \longrightarrow h=0
            \end{array}
            \right.
        \end{equation*}

        Por tanto, $e+if = e-if -id -id \Longrightarrow 2if = -2id \Longrightarrow f=-d \Longrightarrow c=e$.

        Por tanto, dado $a.b.c.d \in \bb{R}$, una matriz $A\in \mathcal{M}_2(\bb{C})$ hermítica es:
        \begin{equation*}
            A=\left(\begin{array}{cc}
            a & c+id \\
            c-id & b
        \end{array}\right)
        \end{equation*}

        Calculamos su polinomio característico:
        \begin{equation*}
            P_A(\lambda) = |A-\lambda I| = \lambda^2 - tr(A)\lambda + det(\lambda) = \lambda^2 -(a+b)\lambda + ab-c^2-d^2
        \end{equation*}
        $$\Delta = (a+b)^2 -4(ab-c^2 - d^2) = a^2+b^2+2ab -4ab +4c^2 +4d^2 = (a-b)^2 +4(c^2+d^2) \geq 0$$

        \begin{itemize}
            \item Si $\Delta > 0 \Longrightarrow$ 
            Hay dos soluciones distintas y, por tanto, $A$ es diagonalizable.

            \item Si $\Delta = 0 \Longrightarrow a=b, \quad  c=d=0 \Longrightarrow A=\left(\begin{array}{cc}
            a & 0 \\
            0 & a
        \end{array}\right)$ es diagonalizable.
        \end{itemize}
    \end{enumerate}
\end{ejercicio}


\begin{ejercicio}
    Sea $A\in \mathcal{M}_n(\bb{R})$.
    \begin{enumerate}
        \item Demostrar que  $A$ diagonalizable $\Longrightarrow A^2-2A+I$ diagonalizable.

        $A \text{ diagonalizable } \Longrightarrow D = P^{-1}AP \Longrightarrow A=PDP^{-1}$. Por tanto,
        $$A^2-2A+I = PD^2P^{-1} -2PDP^{-1} + PIP^{-1} = P[D^2-2D + I]P^{-1}$$

        Por tanto, como $D^2-2D + I$ es diagonal y $P$ es regular, $A^2-2A+I$ es semejante a una matriz diagonal, por lo que es diagonalizable.

        \item Razonar que  $A^2-2A+I$ diagonalizable $\nRightarrow A$ diagonalizable.\\

        Sea $A=\left( \begin{array}{cc}
            1 & 0 \\
            1 & 1
        \end{array}\right)$.

        $A^2-2A + I = (A-I)^2 =\left( \begin{array}{cc}
            0 & 0 \\
            0 & 0
        \end{array}\right) = 0$. Esta matriz es diagonalizable ya que $D=P^{-1}0P$, con $D=0$ y $P=I$.

        Sin embargo, el polinomio característico de $A$ es $P_A(\lambda) = \lambda^2-2\lambda + 1 = (\lambda-1)^2$.
        $$rg(A-I) = 1 \Longrightarrow \dim V_1 = 2-1 = 1$$
        Como $n_1 = 1 \neq 2 = m_1$, $A$ no es diagonalizable.
    \end{enumerate}
\end{ejercicio}

\begin{ejercicio}
    $A\in \mathcal{M}_2(\bb{R})$ no diagonalizable y con un valor propio $a\in\bb{R}$ de multiplicidad algebraica $m_a=2$. Demostrar que $A\sim B$, con
    \begin{equation*}
        B=\left( \begin{array}{cc}
            a & 0 \\
            1 & a
        \end{array}\right)
    \end{equation*}

    \begin{proof}
        Sea $f\in End(\bb{R}^2)$ con $A=M(f, \mathcal{B})$.
        
        Como $A$ no es diagonalizable, $n_a \neq 2$, por lo que $n_a = \dim V_a = 1$.

        Busco una base de $\bb{R}^2 \; \mathcal{B} = \{v_1, v_2\}\;\; v_2 \in V_a - \{0\}$. Como $\dim V_a = 1, \{v_2\}$ base de $V_a$. Por tanto,
        $$\begin{array}{rl}
            v_1 & \longrightarrow bv_1 + cv_2 \\
            v_2 & \longrightarrow av_2
        \end{array} \qquad \qquad 
        M(f, \mathcal{B}) = A = \left( \begin{array}{cc}
            b & 0 \\
            c & a
        \end{array} \right)$$

        Como la multiplicidad algebraica de $a$ es $2$, $$P_f(\lambda) = (a-\lambda)^2 = \left| \begin{array}{cc}
            b-\lambda & 0 \\
            c & a-\lambda
        \end{array} \right| = (b-\lambda)(a-\lambda) \Longrightarrow a = b$$
        Además, si fuese $c=0$ sería diagonalizable, por lo que $c\neq 0$.

        Sea ahora $\bar{\mathcal{B}} = \{\bar{v}_1,\bar{v}_2\}$, con $\bar{v}_1=v_1$ y $\bar{v}_2 = cv_2$. Forman base ya que $c\neq 0$ y $\mathcal{B}$ es una base.
        Sabemos que $$f(\bar{v}_2) = f(cv_2) = cf(v_2) = cav_2 = a\bar{v}_2$$
        $$\begin{array}{rl}
            \bar{v}_1 & \longrightarrow a\bar{v_1} + \bar{v_2} \\
            \bar{v_2} & \longrightarrow a\bar{v_2}
        \end{array} \qquad \qquad 
        M(f, \bar{\mathcal{B}}) = B = \left( \begin{array}{cc}
            a & 0 \\
            1 & a
        \end{array} \right)$$

        Por tanto, como la matriz $A$ y la matriz $B$ representan el mismo endormorfismo en dos bases distintas ($\mathcal{B}$ y $\bar{\mathcal{B}}$ respectivamente), las matrices son semejantes ($A\sim ~B$).
    \end{proof}
\end{ejercicio}


\begin{ejercicio}
    Sea $A\in \mathcal{M}_4(\bb{R})$. Estudiar si $A$ es diagonalizable.
    \begin{equation*}
        A = \left( \begin{array}{cccc}
            1 & 1 & 1 & 1 \\
            1 & 1 & 1 & 1 \\
            1 & 1 & 1 & 1 \\
            1 & 1 & 1 & 1 \\
        \end{array}\right)
    \end{equation*}

    $rg(A)=1 \Longrightarrow \dim Im(f) = 1 \Longrightarrow \dim Ker(f)=3$.
    
    Por tanto, como $Ker(f)\neq\{0\}\Longrightarrow 0$  es un valor propio. $V_0(f)=Ker(f)$

    Los valores propios de $A$ son $\{\lambda_0 \in \bb{R} \mid |A-\lambda_0 I| = 0\}$.
    \begin{equation*}
        |A-\lambda_0 I| = \left| \begin{array}{cccc}
            1-\lambda_0 & 1 & 1 & 1 \\
            1 & 1-\lambda_0 & 1 & 1 \\
            1 & 1 & 1-\lambda_0 & 1 \\
            1 & 1 & 1 & 1-\lambda_0 \\
        \end{array}\right| = 0
    \end{equation*}

    Como podemos ver, efectivamente $\lambda_0 = 0$ es un valor propio. Para $\lambda_0=4$, los vectores columna cumplen que la suma de sus componentes es nula, es decir, pertenecen al hiperplano de $\bb{R}$ con ecuación implícita $x_1+x_2+x_3+x_4 = 0$. Como la dimensión de ese hiperplano es $3$, uno de los vectores será linealmente dependiente y por tanto el determinante es nulo.

    Como $Ker(f) = V_0,\;\dim V_0 = 3$. Por tanto, la multiplicidad algebraica de $0\;m_0$ cumple que $3 \leq m_0 \leq 4$. Pero $m_0\neq 4$ porque sino sería el único valor propio. Por tanto, $m_0=3$ y $m_4=1$.
    
    
    \begin{table}[H]
        \centering
        \begin{tabular}{c|c|c}
            Valores Propios & Mult. Alg. & Mult. Geom. \\ \hline 
            0 & 3 & 3\\
            4 & 1 & 1\\
        \end{tabular}
        \caption{Valores propios con sus multiplicidades}
    \end{table}

    Por tanto, es diagonalizable. Busquemos las matrices $P$ y $D$. Calculamos en primer lugar cada subespacio propio.
    \begin{equation*}
        V_0 = \left\{x\in\bb{R}^4 \mid x_1+x_2+x_3+x_4 = 0\right\} = \mathcal{L}\left(\left\{ \left(\begin{array}{c}
                1 \\
                - 1 \\
                0 \\
                0 \\
           \end{array}\right),
           \left(\begin{array}{c}
                1 \\
                0 \\
                -1 \\
                0 \\
           \end{array}\right),
           \left(\begin{array}{c}
                1 \\
                0 \\
                0 \\
                -1 \\
           \end{array}\right)
           \right\}\right)
    \end{equation*}

    \begin{equation*}
        V_4 = \left\{x\in\bb{R}^4 \left| 
        \begin{array}{cl}
            -3x_1+x_2+x_3+x_4 & = 0  \\
            \dots & = 0  \\
            \dots & = 0  \\
        \end{array}\right.
        \right\} = \mathcal{L}\left(\left\{ \left(\begin{array}{c}
                1 \\
                1 \\
                1 \\
                1 \\
           \end{array}\right)
           \right\}\right)
    \end{equation*}

    Por tanto, la base de vectores propios es:
    \begin{equation*}
        \mathcal{B} = \left\{
        \left(\begin{array}{c}
                1 \\
                - 1 \\
                0 \\
                0 \\
           \end{array}\right),
           \left(\begin{array}{c}
                1 \\
                0 \\
                -1 \\
                0 \\
           \end{array}\right),
           \left(\begin{array}{c}
                1 \\
                0 \\
                0 \\
                -1 \\
           \end{array}\right),
           \left(\begin{array}{c}
                1 \\
                1 \\
                1 \\
                1 \\
           \end{array}\right)
        \right\}
    \end{equation*}

    Por tanto, las matrices $P$ y $D$ son:
    \begin{equation*}
        D = \left( \begin{array}{cccc}
            0 & 0 & 0 & 0 \\
            0 & 0 & 0 & 0 \\
            0 & 0 & 0 & 0 \\
            0 & 0 & 0 & 4 \\
        \end{array}\right) \qquad 
        P = \left( \begin{array}{cccc}
            1 & 1 & 1 & 1 \\
            -1 & 0 & 0 & 1 \\
            0& -1 & 0 & 1 \\
            0 & 0 & -1 & 1 \\
        \end{array}\right)
    \end{equation*}
\end{ejercicio}

\begin{ejercicio}
    Sea $A\in \mathcal{M}_3(\bb{R})$. Estudiar si $A$ es diagonalizable.
    \begin{equation*}
        A = \left( \begin{array}{ccc}
            1 & 3 & 0 \\
            3 & -2 & -1 \\
            0 & -1 & 1 \\
        \end{array}\right)
    \end{equation*}

    Calculamos su polinomio característico:
    \begin{equation*}
    P_A(\lambda) = det(A-\lambda I) = \left| \begin{array}{ccc}
            1-\lambda & 3 & 0 \\
            3 & -2-\lambda & -1 \\
            0 & -1 & 1-\lambda \\
        \end{array}\right|=  -\lambda^3 +13\lambda - 12    
    \end{equation*}

    Sus posibles raíces en $\bb{Q}$ son: $\{\pm1, \pm2, \pm3, \pm4, \pm6, \pm12\}$.

    \begin{figure}[H]
        \centering
        \polyhornerscheme[x=1]{-x^3+13x-12}
    \end{figure}
    Las raíces de $-\lambda^2 -\lambda +12 = 0$ son $\lambda_1=-4$ y $\lambda_2=3$.
    
    Por tanto, $P_A(\lambda)=-(\lambda-1)(\lambda+4)(\lambda-3)$.
    
    \begin{table}[H]
        \centering
        \begin{tabular}{c|c|c}
            Valores Propios & Mult. Alg. & Mult. Geom. \\ \hline 
            1 & 1 & 1\\
            $-$4 & 1 & 1\\
            3 & 1 & 1\\
        \end{tabular}
        \caption{Valores propios con sus multiplicidades}
    \end{table}

    Por tanto, $A$ es diagonalizable.
\end{ejercicio}

\begin{ejercicio}\label{Ej:TodosUnos}
    Sea $A=\mathcal{M}_n(\bb{R})$ con todos sus coeficientes $1$. ¿Es diagonalizable?

    $$A = (1)_{i,j} \qquad \forall i,j\mid 1 \leq i,j \leq n$$

    $rg(A)=1 \Longrightarrow \dim Im(f) = 1 \Longrightarrow \dim Ker(f)=n-1$.
    
    Por tanto, como $Ker(f)\neq\{0\}\Longrightarrow 0$  es un valor propio. $V_0(f)=Ker(f)$

    \begin{table}[H]
        \centering
        \begin{tabular}{c|c|c}
            Valores Propios & Mult. Alg. & Mult. Geom. \\ \hline 
            0 & $n-1 \leq m_0 \leq n$ & $n-1$\\
            - & - & -\\
        \end{tabular}
        \caption{Valores propios con lo que sabemos hasta el momento.}
    \end{table}
    
    Para completar las multiplicidades, es necesario saber si hay más valores propios.
    \begin{equation*}
        (A-\lambda_0 I) = \left( \begin{array}{cccc}
            1-\lambda_0 & 1 & \dots & 1 \\
            1 & \ddots &  & \vdots \\
            \vdots &  & \ddots & 1 \\
            1 & \dots & 1 & 1-\lambda_0 \\
        \end{array}\right) = \left\{
        \begin{array}{ccc}
            1 & \text{ si } & i\neq j\\
            1-\lambda_0 & \text{ si } & i= j
        \end{array}
        \right.
    \end{equation*}

    Los valores propios de $A$ son $\{\lambda_0 \in \bb{R} \mid |A-\lambda_0 I| = 0\}$.
    \begin{equation*}
        |A-\lambda_0 I| = \left| \begin{array}{cccc}
            1-\lambda_0 & 1 & \dots & 1 \\
            1 & \ddots &  & \vdots \\
            \vdots &  & \ddots & 1 \\
            1 & \dots & 1 & 1-\lambda_0 \\
        \end{array}\right| = 0
    \end{equation*}

    $\lambda_0 = n$ es un valor propio, ya que los vectores columna formarían parte del hiperplano con ecuación implícita $x_1+\dots +x_n=0$ y, por tanto, el determinante sería nulo.

    \begin{table}[H]
        \centering
        \begin{tabular}{c|c|c}
            Valores Propios & Mult. Alg. & Mult. Geom. \\ \hline 
            0 & $n-1$ & $n-1$\\
            $n$ & 1 & 1\\
        \end{tabular}
        \caption{Valores propios con sus multiplicidades}
    \end{table}

    Por tanto, $A$ es diagonalizable.   
\end{ejercicio}

\begin{ejercicio}
    Sea $A=\mathcal{M}_3(\bb{R})$, con
    \begin{equation*}
        A = \left(\begin{array}{ccc}
            3 & -1 & 1 \\
            -1 & 3 & 1 \\
            1 & 1 & 3
        \end{array} \right)
    \end{equation*}
    Estudiar si $\exists B \in \mathcal{M}_n(\bb{R}) \mid B^2=A$. Es decir, estudiar si $\exists \sqrt{A}$.\\

    En primer lugar, vemos si $A$ es diagonalizable y, en su caso, se diagonaliza.
    \begin{equation*}\begin{split}
        P_A(\lambda) & = |A-\lambda I| = \left|\begin{array}{ccc}
            3-\lambda & -1 & 1 \\
            -1 & 3-\lambda & 1 \\
            1 & 1 & 3-\lambda
        \end{array} \right| = \left|\begin{array}{ccc}
            4-\lambda & -1 & 1 \\
            0 & 3-\lambda & 1 \\
            4-\lambda & 1 & 3-\lambda
        \end{array} \right| = \\
        &= \left|\begin{array}{ccc}
            4-\lambda & -1 & 1 \\
            0 & 3-\lambda & 1 \\
            0 & 2 & 2-\lambda
        \end{array} \right| = (4-\lambda)((3-\lambda)(2-\lambda)-2) = (4-\lambda)(\lambda^2-5\lambda+4) =\\
        &=(4-\lambda)(\lambda-4)(\lambda-1) = (4-\lambda)^2(1-\lambda)
    \end{split}\end{equation*}

    Por tanto, los valores propios son $\{1,4\}$. Calculamos los subespacios propios.
    \begin{equation*}\begin{split}
           V_{1} & = \left\{ \left(\begin{array}{c}
                x_1 \\
                x_2 \\
                x_3
           \end{array}\right) \in \bb{R}^3 \mid (A-I)\left(\begin{array}{c}
                x_1 \\
                x_2 \\
                x_3
           \end{array}\right) = 0 \right\} \\
           & = \left\{ \left(\begin{array}{c}
                x_1 \\
                x_2 \\
                x_3
           \end{array}\right) \in \bb{R}^3 \mid \left( \begin{array}{ccc}
            2 & -1 & 1 \\
            -1 & 2 & 1 \\
            1 & 1 & 2
        \end{array}\right) \left(\begin{array}{c}
                x_1 \\
                x_2 \\
                x_3
           \end{array}\right) = 0 \right\} \\
           & = \left\{ \left(\begin{array}{c}
                x_1 \\
                x_2  \\
                x_3
           \end{array}\right) \in \bb{R}^3 \left| \begin{array}{c}
                2x_1 -x_2+x_3=0 \\
                -x_1+2x_2+x_3 = 0 \\
                x_1+x_2+2x_3 = 0
           \end{array}\right. \right\} =
           \mathcal{L}\left(\left\{ \left(\begin{array}{c}
                    1 \\
                    1 \\
                    -1 \\
               \end{array}\right)
               \right\}\right)
   \end{split}\end{equation*}
   \begin{equation*}\begin{split}
           V_{4} & = \left\{ \left(\begin{array}{c}
                x_1 \\
                x_2 \\
                x_3
           \end{array}\right) \in \bb{R}^3 \mid (A-I)\left(\begin{array}{c}
                x_1 \\
                x_2 \\
                x_3
           \end{array}\right) = 0 \right\} \\
           & = \left\{ \left(\begin{array}{c}
                x_1 \\
                x_2 \\
                x_3
           \end{array}\right) \in \bb{R}^3 \mid \left( \begin{array}{ccc}
            -1 & -1 & 1 \\
            -1 & -1 & 1 \\
            1 & 1 & -1
        \end{array}\right) \left(\begin{array}{c}
                x_1 \\
                x_2 \\
                x_3
           \end{array}\right) = 0 \right\} \\
           & = \left\{ \left(\begin{array}{c}
                x_1 \\
                x_2  \\
                x_3
           \end{array}\right) \in \bb{R}^3 \left| \begin{array}{c}
                x_1+x_2-x_3 = 0
           \end{array}\right. \right\} =
           \mathcal{L}\left(\left\{
                \left(\begin{array}{c}
                    1 \\
                    0 \\
                    1 \\
               \end{array}\right),
               \left(\begin{array}{c}
                    0 \\
                    1 \\
                    1 \\
               \end{array}\right)
               \right\}\right)
   \end{split}\end{equation*}
   \begin{table}[H]
        \centering
        \begin{tabular}{c|c|c}
            Valores Propios & Mult. Alg. & Mult. Geom. \\ \hline 
            1 & 1 & 1\\
            4 & 2 & 2\\
        \end{tabular}
        \caption{Valores propios con sus multiplicidades}
    \end{table}
    Por tanto, $A$ es diagonalizable de la forma $D=P^{-1}AP$, con
    \begin{equation*}
        D=\left( \begin{array}{ccc}
            1 & 0 & 0 \\
            0 & 4 & 0 \\
            0 & 0 & 4 
        \end{array}\right) \qquad
        P=\left( \begin{array}{ccc}
            1 & 1 & 0 \\
            1 & 0 & 1 \\
            -1 & 1 & 1 
        \end{array}\right)
    \end{equation*}

    Definimos $\sqrt{D}$ como:
    \begin{equation*}
        \sqrt{D}=\left( \begin{array}{ccc}
            1 & 0 & 0 \\
            0 & 2 & 0 \\
            0 & 0 & 2 
        \end{array}\right)
    \end{equation*}

    Para calcular $B$, despejamos $A$. $A=PDP^{-1}$
    \begin{equation*}
         P\sqrt{D}P^{-1} \cdot P\sqrt{D}P^{-1} = PDP^{-1} = A \Longrightarrow B = \sqrt{A} = P\sqrt{D}P^{-1}
    \end{equation*}

    Por tanto, $\exists B=\sqrt{A} \in \mathcal{M}_3(\bb{R}) \mid B^2 = A$, y se define como
    $$B = \sqrt{A} = P\sqrt{D}P^{-1}$$
\end{ejercicio}

\begin{ejercicio}\textbf{Prueba 2022}
\begin{enumerate}
    \item Estudiar $a\in\bb{R}$ para los que $A\in\mathcal{M}_3(\bb{R})$ se diagonaliza.
    \begin{equation*}
        A = \left( \begin{array}{ccc}
            1 & -a & 1 \\
            -a & 1 & -1 \\
            0 & a & 0
        \end{array} \right)
    \end{equation*}

    Calculo en primer lugar su polinomio característico.
    \begin{multline*}
        P_A(\lambda) = |A-\lambda I| = \left| \begin{array}{ccc}
            1-\lambda & -a & 1 \\
            -a & 1-\lambda & -1 \\
            0 & a & -\lambda
        \end{array} \right|
        = \left| \begin{array}{ccc}
            1-\lambda & -a & 1 \\
            -a & 1-\lambda & -1 \\
            1-\lambda & 0 & 1-\lambda
        \end{array} \right| = \\
        = \left| \begin{array}{ccc}
            1-\lambda & -a & \lambda \\
            -a & 1-\lambda & -1+a \\
            1-\lambda & 0 & 0
        \end{array} \right| 
        = (1-\lambda)\left| \begin{array}{cc}
            -a & \lambda \\
            1-\lambda & -1+a \\
        \end{array} \right| = \\
        = (1-\lambda)\left| \begin{array}{cc}
            -a+\lambda & \lambda \\
            a-\lambda & -1+a \\
        \end{array} \right|
        = (1-\lambda)\left| \begin{array}{cc}
            -a+\lambda & \lambda \\
            0 & \lambda-1+a \\
        \end{array} \right| = (1-\lambda)(\lambda-a)(\lambda-1+a)
    \end{multline*}
    Por tanto, los valores propios son: $\{1,a,1-a\}$. Hay que tener en cuenta el caso en que dos valores propios sean iguales.
    \begin{equation*} \begin{array}{ll}
       a=1 & \longrightarrow a=1 \\
       1-a=1 & \longrightarrow a=0 \\
       a=1-a & \longrightarrow a=\frac{1}{2} \\
   \end{array}\end{equation*}
    Para $a\neq \{0,\frac{1}{2},1\}$, los tres valores propios son distintos, por lo que sí es diagonalizable.
    \begin{itemize}
        \item[-] \underline{Para $a=0$}:
        \begin{equation*}\begin{split}
               V_{1} & = \left\{ \left(\begin{array}{c}
                    x_1 \\
                    x_2 \\
                    x_3
               \end{array}\right) \in \bb{R}^3 \mid (A-I)\left(\begin{array}{c}
                    x_1 \\
                    x_2 \\
                    x_3
               \end{array}\right) = 0 \right\} \\
               & = \left\{ \left(\begin{array}{c}
                    x_1 \\
                    x_2 \\
                    x_3
               \end{array}\right) \in \bb{R}^3 \mid \left( \begin{array}{ccc}
                0 & 0 & 1 \\
                0 & 0 & -1 \\
                0 & 0 & -1
            \end{array}\right) \left(\begin{array}{c}
                    x_1 \\
                    x_2 \\
                    x_3
               \end{array}\right) = 0 \right\} \\
               & = \left\{ \left(\begin{array}{c}
                    x_1 \\
                    x_2  \\
                    x_3
               \end{array}\right) \in \bb{R}^3 \left| \begin{array}{c}
                    x_3=0 \\
               \end{array}\right. \right\}
       \end{split}\end{equation*}
        \begin{table}[H]
            \centering
            \begin{tabular}{c|c|c}
                Valores Propios & Mult. Alg. & Mult. Geom. \\ \hline 
                1 & 2 & 2\\
                0 & 1 & 1\\
            \end{tabular}
            \caption{Valores propios con sus multiplicidades}
        \end{table}
        Por tanto, para $a=0$, $A$ sí es diagonalizable.

        \item[-] \underline{Para $a=\frac{1}{2}$}:
        \begin{equation*}\begin{split}
               V_{\frac{1}{2}} & = \left\{ \left(\begin{array}{c}
                    x_1 \\
                    x_2 \\
                    x_3
               \end{array}\right) \in \bb{R}^3 \mid \left(A-\frac{1}{2}I\right)\left(\begin{array}{c}
                    x_1 \\
                    x_2 \\
                    x_3
               \end{array}\right) = 0 \right\} \\
               & = \left\{ \left(\begin{array}{c}
                    x_1 \\
                    x_2 \\
                    x_3
               \end{array}\right) \in \bb{R}^3 \mid \left( \begin{array}{ccc}
                \frac{1}{2} & -\frac{1}{2} & 1 \\
                -\frac{1}{2} & \frac{1}{2} & -1 \\
                0 & \frac{1}{2} & -\frac{1}{2}
            \end{array}\right) \left(\begin{array}{c}
                    x_1 \\
                    x_2 \\
                    x_3
               \end{array}\right) = 0 \right\} \\
               & = \left\{ \left(\begin{array}{c}
                    x_1 \\
                    x_2  \\
                    x_3
               \end{array}\right) \in \bb{R}^3 \left| \begin{array}{c}
                    \frac{1}{2}x_1-\frac{1}{2}x_2+x_3=0 \\
                    -\frac{1}{2}x_1+\frac{1}{2}x_2-x_3=0\\
                    \frac{1}{2}x_2 - \frac{1}{2}x_3 = 0
               \end{array}\right. \right\} \\
               & = \left\{ \left(\begin{array}{c}
                    x_1 \\
                    x_2  \\
                    x_3
               \end{array}\right) \in \bb{R}^3 \left| \begin{array}{c}
                    \frac{1}{2}x_1-\frac{1}{2}x_2+x_3=0 \\
                    \frac{1}{2}x_2 - \frac{1}{2}x_3 = 0
               \end{array}\right. \right\}
       \end{split}\end{equation*}
        \begin{table}[H]
            \centering
            \begin{tabular}{c|c|c}
                Valores Propios & Mult. Alg. & Mult. Geom. \\ \hline 
                1 & 1 & 1\\
                $\frac{1}{2}$ & 2 & 1\\
            \end{tabular}
            \caption{Valores propios con sus multiplicidades}
        \end{table}
        Por tanto, para $a=\frac{1}{2}$, $A$ no es diagonalizable.

        \item[-] \underline{Para $a=1$}:
        \begin{equation*}\begin{split}
               V_{1} & = \left\{ \left(\begin{array}{c}
                    x_1 \\
                    x_2 \\
                    x_3
               \end{array}\right) \in \bb{R}^3 \mid (A-I)\left(\begin{array}{c}
                    x_1 \\
                    x_2 \\
                    x_3
               \end{array}\right) = 0 \right\} \\
               & = \left\{ \left(\begin{array}{c}
                    x_1 \\
                    x_2 \\
                    x_3
               \end{array}\right) \in \bb{R}^3 \mid \left( \begin{array}{ccc}
                0 & -1 & 1 \\
                -1 & 0 & -1 \\
                0 & 1 & -1
            \end{array}\right) \left(\begin{array}{c}
                    x_1 \\
                    x_2 \\
                    x_3
               \end{array}\right) = 0 \right\} \\
               & = \left\{ \left(\begin{array}{c}
                    x_1 \\
                    x_2  \\
                    x_3
               \end{array}\right) \in \bb{R}^3 \left| \begin{array}{c}
                    x_2-x_3=0 \\
                    x_1+x_3=0
               \end{array}\right. \right\}
       \end{split}\end{equation*}
        \begin{table}[H]
            \centering
            \begin{tabular}{c|c|c}
                Valores Propios & Mult. Alg. & Mult. Geom. \\ \hline 
                1 & 2 & 1\\
                0 & 1 & 1\\
            \end{tabular}
            \caption{Valores propios con sus multiplicidades}
        \end{table}
        Por tanto, para $a=1$, $A$ no es diagonalizable.
    \end{itemize}

    \begin{itemize}
        \item \textbf{Ver si, para $a=0$ y $a=-1$, las matrices resultantes son semejantes.}
        
        Para $a=0$, los valores propios son $\{0,1,1\}$.\\
        Para $a=-1$, los valores propios son $\{1,-1,2\}$.

        Por tanto, como tienen valores distintos, tienen polinomio característico distinto y, por tanto, no son semejantes.
    \end{itemize}

    \item Sea $A\in\mathcal{M}_2(\bb{R}) \mid |A|=-1$. Demostrar que $A$ es diagonalizable.

    Sea $A=\left( \begin{array}{cc}
        a & b \\
        c & d
    \end{array} \right)$, con $|A|=ad-bc = -1$.
    $$P_A(\lambda) = \lambda^2 - tr(A)\lambda + det(A) = \lambda^2 -(a+d)\lambda-1$$
    $$\Delta = (a+d)^2 +4 \geq 4 > 0 \qquad \forall a,d \in \bb{R}$$

    Por tanto, como el discriminante es positivo, tendrá dos valores propios distintos. Por tanto, $A$ sí es diagonalizable.

    \begin{itemize}
        \item \textbf{Encontrar $A\in\mathcal{M}_2(\bb{C}) \mid |A|=-1$ t.q. $A$ no sea diagonalizable.}\\
        En primer lugar, es necesario que solo tenga un valor propio, por lo que $$\Delta=0 \Longrightarrow (a+d)^2 = -4 \Longrightarrow a+d = 2i$$

        Además, como $|A|=-1 \Longrightarrow ad-bc=-1$. Por tanto, una posible $A\in\mathcal{M}_2(\bb{C})$ es: $$A=\left( \begin{array}{cc}
            i & 1 \\
            0 & i
        \end{array} \right)$$
        $$P_A(\lambda) = \lambda^2-2i\lambda-1 = (\lambda-i)^2$$

        Por tanto, el único valor propio es $\{i\}$.
        \begin{equation*}\begin{split}
               V_{i} & = \left\{ \left(\begin{array}{c}
                    x_1 \\
                    x_2 \\
               \end{array}\right) \in \bb{C}^2 \mid (A-iI)\left(\begin{array}{c}
                    x_1 \\
                    x_2 \\
               \end{array}\right) = 0 \right\} \\
               & = \left\{ \left(\begin{array}{c}
                    x_1 \\
                    x_2 \\
               \end{array}\right) \in \bb{C}^2 \mid \left( \begin{array}{cc}
                0 & 1 \\
                0 & 0
            \end{array}\right) \left(\begin{array}{c}
                    x_1 \\
                    x_2 \\
               \end{array}\right) = 0 \right\} \\
               & = \left\{ \left(\begin{array}{c}
                    x_1 \\
                    x_2  \\
               \end{array}\right) \in \bb{C}^2 \left| \begin{array}{c}
                    x_2=0 \\
               \end{array}\right. \right\}
       \end{split}\end{equation*}
       \begin{table}[H]
            \centering
            \begin{tabular}{c|c|c}
                Valores Propios & Mult. Alg. & Mult. Geom. \\ \hline 
                $i$ & 2 & 1\\
            \end{tabular}
            \caption{Valores propios con sus multiplicidades}
        \end{table}

        Por tanto, $A$ no es diagonalizable.
    \end{itemize}
\end{enumerate}
    
\end{ejercicio}

\begin{ejercicio}
    Sea $A \in \mathcal{M}_n(\bb{K})$ t.q. $A^2=I_n$. Demostrar que $A$ es diagonalizable.
    \begin{proof}
        Haciendo uso del isomorfismo natural entre $\mathcal{M}_n(\bb{K})$ y $End(V)$ con $\dim_\bb{K}(V)=n$, consideramos $A$ como la matriz a asociada a $f\in End(V^n)$ el cual cumple que $f\circ f = Id_\bb{K}$.
        $$f\circ f = Id_\bb{K} \hspace{2cm} f(f(x))=x\quad \forall x\in V$$

        Notemos ahora que, dado un $v\in V$ cualquiera, este se puede expresar como:
        $$v\in V\quad v = \underbrace{\frac{1}{2}(v+f(v))}_{t_1} + \underbrace{\frac{1}{2}(v-f(v))}_{t_2}$$
        
        Sea $t_1 = \frac{1}{2}(v+f(v))$ y $t_2$ = $\frac{1}{2}(v-f(v))$. Nótese también que, como $f\circ f = Id_\bb{K}$,
        \begin{equation}\label{RazonamientoErroneo}\tag{$\ast$}
        \begin{split}
            f(t_1) = \frac{1}{2}(v+f(v)) = t_1& \xrightarrow{(\ast)} t_1 \text{ es un vector propio.}\quad t_1\in V_1 \\
            f(t_2) = -\frac{1}{2}(v-f(v)) = -t_2& \xrightarrow{(\ast)} t_2 \text{ es un vector propio.}\quad t_2\in V_{-1}
        \end{split}\end{equation}


        Como todo $v \in V$ se expresa como la suma de un vector de $V_1$ y otro de $V_{-1} \Longrightarrow V=V_1 + V_{-1}$. Además, como los subespacios propios de valores propios distintos son disjuntos, $V_1 \cap V_{-1} = \{0\}$. Por tanto, $V_1 \oplus V_{-1} = V$.

        Sean $\mathcal{B}_1$ y $\mathcal{B}_{-1}$ bases de $V_1$ y $V_{-1}$ respectivamente. Como $V=V_1\oplus V_{-1} \Longrightarrow \mathcal{B}_{1} \cup \mathcal{B}_{-1}$ forman una base $\mathcal{B}$ de $V$. Así, $V$ tendrá una base de vectores propios y, por tanto, $f$ es diagonalizable, por lo que $A$ también lo es.

        (\ref{RazonamientoErroneo}) Este razonamiento es válido $\forall f \neq \{Id, -Id\}$, ya que en su caso $t_1$ o $t_2$ serían nulos. En el caso en el que sea $f=Id \lor f=-Id$, su matriz asociada $A$ es diagonal y por tanto será diagonalizable.
        
    \end{proof}
\end{ejercicio}

\begin{observacion}
    La aplicación transposición,
    $$\begin{array}{cccc}
        ^t: & \mathcal{M}_n(\bb{K}) & \to & \mathcal{M}_n(\bb{K}) \\
         & A & \to & A^t
    \end{array}$$
    Tiene como matriz asociada $M(^t, \mathcal{B}) \sim D = \left( \begin{array}{ccccc}
        1 & & & &\\
        & 1 & & &\\
        & & \ddots & &\\
        & & & -1 & \\
        & & & & -1 \\  
    \end{array} \right)$
\end{observacion}

\begin{ejercicio}
    Dadas $A_1,A_2,B_1,B_2 \in \mathcal{M}_n(\bb{K})$ tal que $A_1 \sim A_2$ y $B_1\sim B_2$. Demostrar que:
    \begin{equation*}
        \left.\begin{array}{c}
             A_1 \sim A_2  \\
             B_1\sim B_2 
        \end{array}\right\} \nRightarrow A_1+B_1 \sim A_2 + B_2
    \end{equation*}

    Se demuestra buscando un contraejemplo. Sean:
    \begin{equation*}
        A_1 = \left( \begin{array}{cc}
            0 & 0 \\
            1 & 0
        \end{array} \right) \qquad
        A_2 = \left( \begin{array}{cc}
            0 & 0 \\
            1 & 0
        \end{array} \right) \qquad
        B_1 = \left( \begin{array}{cc}
            0 & 0 \\
            1 & 0
        \end{array} \right) \qquad
        B_2 = \left( \begin{array}{cc}
            0 & 1 \\
            0 & 0
        \end{array} \right)
    \end{equation*}

    Es fácil ver que $A_1 \sim A_2$. Veamos ahora que $B_1 \sim B_2$.

    Sea $f\in End(\bb{R}^2)$ y sean $\mathcal{B}=\{e_1,e_2\}$ y $ \bar{\mathcal{B}} = \{\bar{e_1}, \bar{e_2}\}$ bases de $\bb{R}^2$ tales que $B_1 = M(f;\mathcal{B})$ y $B_2 = M(f;\bar{\mathcal{B}})$.
    \begin{equation*}
        f:\left| \begin{array}{c}
            e_1 \longmapsto e_2 \\
            e_2 \longmapsto 0 
        \end{array} \right. \hspace{2cm}
        f:\left| \begin{array}{cc}
            \bar{e_1} \longmapsto 0 \\
            \bar{e_2} \longmapsto \bar{e_1} 
        \end{array} \right.
    \end{equation*}
    Son semejantes, ya que si $\bar{e_1} = e_2$ y $\bar{e_2} = e_1$, las dos matrices representan el mismo endomorfismo respecto de distintas bases.

    Por tanto, sabiendo que se dan las hipótesis,
    \begin{equation*}
        A_1 + B_1 = \left( \begin{array}{cc}
            0 & 0 \\
            2 & 0
        \end{array} \right) \qquad
        A_2+B_2 = \left( \begin{array}{cc}
            0 & 1 \\
            1 & 0
        \end{array} \right)
    \end{equation*}
    Es fácil ver que $A_1+B_1 \nsim A_2 + B_2$, ya que tienen rango distinto.\footnote{También se puede ver que tienen determinante distinto, o que la segunda es diagonalizable (por ser simétrica) y la primera no.}
\end{ejercicio}

\begin{ejercicio}
    Sea $A,B \in \mathcal{M}_3(\bb{R})$. Demostrar que no son semejantes $(A\nsim B)$.

    \begin{equation*}
        A = \left( \begin{array}{ccc}
            1 & 0 & 0 \\
            0 & 1 & 0 \\
            0 & 1 & 1
        \end{array}\right) \qquad 
        B = \left( \begin{array}{ccc}
            1 & 0 & 0 \\
            1 & 1 & 0 \\
            0 & 1 & 1
        \end{array}\right) \qquad
    \end{equation*}

    En ambos casos, tienen el mismo polinomio característico
    $$P_A(\lambda) = P_B(\lambda) = (1-\lambda)^3$$
    Por tanto, ambas matrices tienen como valor propio $\{1\}$. Veamos ahora la dimensión del subespacio propio $V_1$ en cada caso.
    \begin{equation*}
        rg(A-I) = rg\left( \begin{array}{ccc}
            0 & 0 & 0 \\
            0 & 0 & 0 \\
            0 & 1 & 0
        \end{array}\right) = 1 \qquad
        rg(B-I) = rg\left( \begin{array}{ccc}
            0 & 0 & 0 \\
            1 & 0 & 0 \\
            0 & 1 & 0
        \end{array}\right) = 2
    \end{equation*}

    Por tanto,
    \begin{table}[H]
        \centering
        \begin{tabular}{c|c|c|c}
            Valores Propios & Matriz & Mult. Alg. & Mult. Geom. \\ \hline 
            $A$ & $1$ & 3 & $3-1=2$\\
            $B$ & $1$ & 3 & $3-2=1$\\
        \end{tabular}
        \caption{Valores propios con sus multiplicidades para cada matriz}
    \end{table}

    Como tienen multiplicidades geométricas distintas para el mismo valor propio, entonces no representan el mismo endomorfismo. Por tanto, no son semejantes.
\end{ejercicio}


\begin{ejercicio}
    Sea $A\in \mathcal{M}_4(\bb{R})$. Estudiar los valores de $a\in\bb{R}$ para los que la matriz $A$ es diagonalizable.
    \begin{equation*}
        A = \left( \begin{array}{cccc}
            1 & 0 & 1 & 0 \\
            0 & a & 0 & a \\
            1 & 0 & 1 & 0 \\
            0 & a & 0 & a \\
        \end{array}\right)
    \end{equation*}

    \begin{equation*}\begin{split}
        P_A(\lambda) & = \left|\begin{array}{cccc}
            1-\lambda & 0 & 1 & 0 \\
            0 & a-\lambda & 0 & a \\
            1 & 0 & 1-\lambda & 0 \\
            0 & a & 0 & a-\lambda \\
        \end{array} \right| \stackrel{F'_1 = F_1 - F_3}{=} \left|\begin{array}{cccc}
            -\lambda & 0 & \lambda & 0 \\
            0 & a-\lambda & 0 & a \\
            1 & 0 & 1-\lambda & 0 \\
            0 & a & 0 & a-\lambda \\
        \end{array} \right|
        \stackrel{F'_2 = F_2 - F_4} {=} \\
        & = \left|\begin{array}{cccc}
            -\lambda & 0 & \lambda & 0 \\
            0 & -\lambda & 0 & \lambda \\
            1 & 0 & 1-\lambda & 0 \\
            0 & a & 0 & a-\lambda \\
        \end{array} \right| = \lambda^2 \left|\begin{array}{cccc}
            -1 & 0 & 1 & 0 \\
            0 & -1 & 0 & 1 \\
            1 & 0 & 1-\lambda & 0 \\
            0 & a & 0 & a-\lambda \\
        \end{array} \right| 
        \stackrel{C'_3 = C_3 + C_1} {=} \\
        & = \lambda^2 \left|\begin{array}{cccc}
            -1 & 0 & 0 & 0 \\
            0 & -1 & 0 & 1 \\
            1 & 0 & 2-\lambda & 0 \\
            0 & a & 0 & a-\lambda \\
        \end{array} \right|
        \stackrel{C'_4 = C_4 + C_2} {=} \lambda^2 \left|\begin{array}{cccc}
            -1 & 0 & 0 & 0 \\
            0 & -1 & 0 & 0 \\
            1 & 0 & 2-\lambda & 0 \\
            0 & a & 0 & 2a-\lambda \\
        \end{array} \right| = \\
        & = \lambda^2 \left|\begin{array}{cc}
            -1 & 0 \\
            0 & -1
        \end{array} \right|
        \left|\begin{array}{cc}
            2-\lambda & 0 \\
            0 & 2a-\lambda
        \end{array} \right| = \lambda^2(2-\lambda)(2a-\lambda)
    \end{split}\end{equation*}

    Por tanto, los valores propios son $\{0, 2, 2a\}$.
    
    \begin{itemize}
        \item \underline{Si $a\neq \{0,1\}$}:\\
        \begin{equation*}
            rg(A) = rg(A-0\cdot I) = 2 \Longrightarrow \dim V_0 = 4- 2 = 2
        \end{equation*}
        \begin{table}[H]
            \centering
            \begin{tabular}{c|c|c}
                Valores Propios & Mult. Alg. & Mult. Geom. \\ \hline 
                0 & 2 & 2\\
                2 & 1 & 1\\
                $2a$ & 1 & 1\\
            \end{tabular}
            \caption{Valores propios con sus multiplicidades para $a\neq \{0,1\}$}
        \end{table}
        Por tanto, para $a\neq \{0,1\}, A$ sí es diagonalizable.

        \item \underline{Si $a=0$}:\\
        \begin{equation*}
            rg(A) = rg(A-0\cdot I) = rg\left( \begin{array}{cccc}
            1 & 0 & 1 & 0 \\
            0 & 0 & 0 & 0 \\
            1 & 0 & 1 & 0 \\
            0 & 0 & 0 & 0 \\
        \end{array}\right) = 1 \Longrightarrow \dim V_0 = 4- 1 = 3
        \end{equation*}
        \begin{table}[H]
            \centering
            \begin{tabular}{c|c|c}
                Valores Propios & Mult. Alg. & Mult. Geom. \\ \hline 
                0 & 3 & 3\\
                2 & 1 & 1\\
            \end{tabular}
            \caption{Valores propios con sus multiplicidades para $a=0$}
        \end{table}
        Por tanto, para $a=0, A$ sí es diagonalizable.

        \item \underline{Si $a=1$}:\\
        \begin{equation*}
            rg(A-2I) = rg\left( \begin{array}{cccc}
            -1 & 0 & 1 & 0 \\
            0 & -1 & 0 & 1 \\
            1 & 0 & -1 & 0 \\
            0 & 1 & 0 & -1 \\
        \end{array}\right) = 2 \Longrightarrow \dim V_2 = 4- 2 = 2
        \end{equation*}

        \begin{equation*}
            rg(A) = rg(A-0\cdot I) = rg\left( \begin{array}{cccc}
            1 & 0 & 1 & 0 \\
            0 & 1 & 0 & 1 \\
            1 & 0 & 1 & 0 \\
            0 & 1 & 0 & 1 \\
        \end{array}\right) = 2 \Longrightarrow \dim V_0 = 4- 2 = 2
        \end{equation*}
        
        \begin{table}[H]
            \centering
            \begin{tabular}{c|c|c}
                Valores Propios & Mult. Alg. & Mult. Geom. \\ \hline 
                0 & 2 & 2\\
                2 & 2 & 2\\
            \end{tabular}
            \caption{Valores propios con sus multiplicidades para $a=1$}
        \end{table}
        Por tanto, para $a=1, A$ sí es diagonalizable.
    \end{itemize}
    
\end{ejercicio}


\begin{ejercicio}
    Determinar los valores y vectores propios de las siguientes matrices reales. ¿Son diagonizables?

    \begin{enumerate}
        \item $A = \left( \begin{array}{ccc}
            -1 & 1 & 1 \\
            1  & -1 & 1 \\
            1 & 1 & -1
        \end{array}\right)$

        \begin{multline*}
            P_A(\lambda) = \left| \begin{array}{ccc}
            -1-\lambda & 1 & 1 \\
            1  & -1-\lambda & 1 \\
            1 & 1 & -1-\lambda
        \end{array}\right| \stackrel{C'_1 = C_1+C_3}{=}\left| \begin{array}{ccc}
            -\lambda & 1 & 1 \\
            2  & -1-\lambda & 1 \\
            -\lambda & 1 & -1-\lambda
        \end{array}\right| \stackrel{F'_1 = F_1-F_3}{=} \\
        = \left| \begin{array}{ccc}
            0 & 0 & 2+\lambda \\
            2  & -1-\lambda & 1 \\
            -\lambda & 1 & -1-\lambda
        \end{array}\right| = (2+\lambda)\left| \begin{array}{cc}
            2  & -1-\lambda \\
            -\lambda & 1
        \end{array}\right| = (2+\lambda)(2-\lambda(1+\lambda))= \\  = (2+\lambda)(-\lambda^2 - \lambda +2) = (2+\lambda)^2(1-\lambda)
        \end{multline*}

        Por tanto, $A$ tiene dos valores propios: $\{-2,1\}$. Veamos sus multiplicidades geométricas:
        \begin{equation*}
            rg(A+2I) = rg\left( \begin{array}{ccc}
                1 & 1 & 1 \\
                1  & 1 & 1 \\
                1 & 1 & 1
            \end{array}\right) = 1 \Longrightarrow \dim V_{-2} = 3-1 = 2
        \end{equation*}

        \begin{table}[H]
            \centering
            \begin{tabular}{c|c|c}
                Valores Propios & Mult. Alg. & Mult. Geom. \\ \hline 
               $ -2$ & 2 & 2\\
                1 & 1 & 1\\
            \end{tabular}
            \caption{Valores propios con sus multiplicidades}
        \end{table}
        Por tanto, $A$ sí es diagonalizable.
        Los vectores propios son:
        \begin{equation*}\begin{split}
           V_{1} & = \left\{ \left(\begin{array}{c}
                    x_1 \\
                    x_2 \\
                    x_3
               \end{array}\right) \in \bb{R}^3 \mid (A-I)\left(\begin{array}{c}
                    x_1 \\
                    x_2 \\
                    x_3
               \end{array}\right) = 0 \right\} \\
               & = \left\{ \left(\begin{array}{c}
                    x_1 \\
                    x_2 \\
                    x_3
               \end{array}\right) \in \bb{R}^3 \mid \left( \begin{array}{ccc}
                    -2 & 1 & 1 \\
                    1  & -2 & 1 \\
                    1 & 1 & -2
            \end{array}\right) \left(\begin{array}{c}
                    x_1 \\
                    x_2 \\
                    x_3
               \end{array}\right) = 0 \right\} \\
               & = \left\{ \left(\begin{array}{c}
                    x_1 \\
                    x_2  \\
                    x_3
               \end{array}\right) \in \bb{R}^3 \left| \begin{array}{c}
                    -2x_1 + x_2 +x_3= 0 \\
                    x_1 -2x_2 + x_3 = 0 \\
                    x_1 +x_2 -2x_3 = 0
               \end{array}\right. \right\} \\
               & = \left\{ \left(\begin{array}{c}
                    x_1 \\
                    x_2  \\
                    x_3
               \end{array}\right) \in \bb{R}^3 \left| \begin{array}{c}
                    -2x_1 + x_2 +x_3= 0 \\
                    x_1 -2x_2 + x_3 = 0 \\
               \end{array}\right. \right\} = \mathcal{L}\left(\left\{
                    \left(\begin{array}{c}
                        1 \\
                        1 \\
                        1 \\
                   \end{array}\right)
                   \right\}\right)
       \end{split}\end{equation*}

       \begin{equation*}\begin{split}
           V_{-2} & = \left\{ \left(\begin{array}{c}
                    x_1 \\
                    x_2 \\
                    x_3
               \end{array}\right) \in \bb{R}^3 \mid (A+2I)\left(\begin{array}{c}
                    x_1 \\
                    x_2 \\
                    x_3
               \end{array}\right) = 0 \right\} \\
               & = \left\{ \left(\begin{array}{c}
                    x_1 \\
                    x_2 \\
                    x_3
               \end{array}\right) \in \bb{R}^3 \mid \left( \begin{array}{ccc}
                    1 & 1 & 1 \\
                    1  & 1 & 1 \\
                    1 & 1 & 1
            \end{array}\right) \left(\begin{array}{c}
                    x_1 \\
                    x_2 \\
                    x_3
               \end{array}\right) = 0 \right\} \\
               & = \left\{ \left(\begin{array}{c}
                    x_1 \\
                    x_2  \\
                    x_3
               \end{array}\right) \in \bb{R}^3 \left| \begin{array}{c}
                    x_1 + x_2 + x_3 = 0
               \end{array}\right. \right\} = \mathcal{L}\left(\left\{
                    \left(\begin{array}{c}
                        0 \\
                        1 \\
                        -1 \\
                   \end{array}\right),
                   \left(\begin{array}{c}
                        1 \\
                        -1 \\
                        0 \\
                   \end{array}\right)
                   \right\}\right)
       \end{split}\end{equation*}
       

        \item $B = \left( \begin{array}{ccc}
            -1 & 1 & 0 \\
            0  & -1 & 1 \\
            1 & 0 & -1
        \end{array}\right)$

        \begin{multline*}
            P_A(\lambda) = \left| \begin{array}{ccc}
                -1-\lambda & 1 & 0 \\
                0  & -1-\lambda & 1 \\
                1 & 0 & -1-\lambda
            \end{array}\right| = -(1+\lambda)^3+1 =  \\ = -\lambda^3 -3\lambda^2 - 3\lambda -1 + 1 = -\lambda^3 -3\lambda^2 - 3\lambda = -\lambda(\lambda^2+3\lambda+3)
        \end{multline*}
        $$\Delta = 9 - 12 <0 \Longrightarrow \nexists \;\;sol \in \bb{R}$$

        Por tanto, el único valor propio es $\{0\}$ con multiplicidad simple. Por tanto, $B$ no es diagonalizable.

        Los vectores propios son:
        \begin{equation*}\begin{split}
           V_{0} & = \left\{ \left(\begin{array}{c}
                    x_1 \\
                    x_2 \\
                    x_3
               \end{array}\right) \in \bb{R}^3 \mid B\left(\begin{array}{c}
                    x_1 \\
                    x_2 \\
                    x_3
               \end{array}\right) = 0 \right\} \\
               & = \left\{ \left(\begin{array}{c}
                    x_1 \\
                    x_2 \\
                    x_3
               \end{array}\right) \in \bb{R}^3 \mid \left( \begin{array}{ccc}
                -1 & 1 & 0 \\
                0  & -1 & 1 \\
                1 & 0 & -1
            \end{array}\right) \left(\begin{array}{c}
                    x_1 \\
                    x_2 \\
                    x_3
               \end{array}\right) = 0 \right\} \\
               & = \left\{ \left(\begin{array}{c}
                    x_1 \\
                    x_2  \\
                    x_3
               \end{array}\right) \in \bb{R}^3 \left| \begin{array}{c}
                    -x_1 + x_2 = 0 \\
                    -x_2 + x_3 = 0 \\
                    x_1 - x_3 = 0
               \end{array}\right. \right\}
               = \mathcal{L}\left(\left\{
                    \left(\begin{array}{c}
                        1 \\
                        1 \\
                        1 \\
                   \end{array}\right)
                   \right\}\right)
       \end{split}\end{equation*}
    \end{enumerate}
\end{ejercicio}

\begin{ejercicio}
    Encontrar, si es posible, pares de endomorfismos de $\bb{R}^3(\bb{R})$, uno diagonalizable y otro no, que tengan los siguientes polinomios característicos:

    \begin{enumerate}
        \item $P_f(\lambda) = (1-\lambda)^3$\\
        El endomorfismo diagonalizable tiene por matriz asociada:
        $$M(f, \mathcal{B}) = \left( \begin{array}{ccc}
            1 & 0 & 0 \\
            0 & 1 & 0 \\
            0 & 0 & 1 \\
        \end{array}\right)$$
        En este caso, es diagonalizable, ya que $\dim V_1 = 3-0 = 3$.

        El endomorfismo no diagonalizable tiene por matriz asociada:
        $$M(f', \mathcal{B}) = \left( \begin{array}{ccc}
            1 & 0 & 1 \\
            0 & 1 & 0 \\
            0 & 0 & 1 \\
        \end{array}\right)$$
        En este caso, no es diagonalizable ya que $\dim V_1 = 3-1 = 2 \neq 3$.

        \item $P_f(\lambda) = -(1-\lambda)^2(1+\lambda)$\\
        El endomorfismo diagonalizable tiene por matriz asociada:
        $$M(f, \mathcal{B}) = \left( \begin{array}{ccc}
            1 & 0 & 0 \\
            0 & 1 & 0 \\
            0 & 0 & -1 \\
        \end{array}\right)$$
        En este caso, es diagonalizable, ya que $\dim V_1 = 3-1 = 2$.

        El endomorfismo no diagonalizable tiene por matriz asociada:
        $$M(f', \mathcal{B}) = \left( \begin{array}{ccc}
            1 & 1 & 0 \\
            1 & 1 & 0 \\
            0 & 0 & -1 \\
        \end{array}\right)$$
        En este caso, $f'$ no es diagonalizable ya que $\dim V_1 = 3-2 = 1 \neq 2$.

        \item $P_f(\lambda) = (1-\lambda)(\lambda^2+1)$\\
        En este caso, solo hay un valor propio real, el $\{1\}$, ya que el término $\lambda^2+1$ no tiene raíces reales. Por tanto, como además tiene multiplicidad simple, el endomorfismo solo tendrá un valor propio contado con multiplicidad, por lo que no podrá ser diagonalizable. Por tanto, no es posible encontrar un endomorfismo diagonalizable con ese polinomio característico.

        El endomorfismo no diagonalizable tiene por matriz asociada:
        $$M(f', \mathcal{B}) = \left( \begin{array}{ccc}
            1 & 0 & 0 \\
            0 & 0 & 1 \\
            0 & -1 & 0 \\
        \end{array}\right)$$
    \end{enumerate}
\end{ejercicio}

\begin{ejercicio}
    Probar que todo $f\in End_\bb{R}(\bb{R}^2) \mid det(f)<0$ es diagonalizable.\\

    Su polinomio característico será:
    $$P_f(\lambda) = \lambda^2 - tr(f)\lambda + det(f)$$
    $$\Delta = tr^2(f) -4det(f) > 0 \Longleftrightarrow tr^2(f) > 4det(f) \Longleftrightarrow \frac{tr^2(f)}{4} > det(f)$$
    lo cual es cierto ya que $\frac{tr^2(f)}{4} \geq 0 > det(f)$. Por tanto, el polinomio característico tiene dos raíces distintas, por lo que hay dos valores propios distintos. Como la multiplicidad geométrica es menor o igual que la algebraica, el endomorfismo es diagonalizable.    
\end{ejercicio}

\begin{ejercicio}
    Estudiar si las siguientes matrices son semejantes entre si.
    \begin{equation*}
        A_1 = \left( \begin{array}{ccc}
            1 & 2 & 1 \\
            2 & 0 & 2 \\
            1 & 2 & 1 \\
        \end{array}\right) \qquad
        A_2 = \left( \begin{array}{ccc}
            1 & 0 & 1 \\
            1 & 2 & 3 \\
            1 & 1 & -1 \\
        \end{array}\right) \qquad
        A_3 = \left( \begin{array}{ccc}
            -1 & 2 & -1 \\
            2 & 0 & 2 \\
            3 & 2 & 3 \\
        \end{array}\right)
    \end{equation*}
    \begin{table}[H]
        \centering
        \begin{tabular}{r|ccc|l}
             & $A_1$ & $A_2$ & $A_3$ & \\ \hline
             $tr(A)$ & $2$ & $2$ & $2$ &\\
             $det(A)$ & $0$ & $-6$ & $0$ & $A_1 \nsim A_2 \land A_2 \nsim A_3$\\
        \end{tabular}
        \caption{Resolución usando propiedades de las matrices semejantes}
    \end{table}

    Calculamos el polinomio característico de $A_1$:
    \begin{multline*}
        P_{A_1}(\lambda) = \left| \begin{array}{ccc}
            1-\lambda & 2 & 1 \\
            2 & -\lambda & 2 \\
            1 & 2 & 1-\lambda \\
        \end{array}\right|
        = \left| \begin{array}{ccc}
            -\lambda & 2 & 1 \\
            0 & -\lambda & 2 \\
            \lambda & 2 & 1-\lambda \\
        \end{array}\right|
        = \left| \begin{array}{ccc}
            0 & 4 & 2-\lambda \\
            0 & -\lambda & 2 \\
            \lambda & 2 & 1-\lambda \\
        \end{array}\right| =\\=
        \lambda(8+\lambda(2-\lambda)) = \lambda(-\lambda^2+2\lambda+8)=-\lambda(\lambda-4)(\lambda+2)
    \end{multline*}

    Calculamos el polinomio característico de $A_3$:
    \begin{multline*}
        P_{A_3}(\lambda) = \left| \begin{array}{ccc}
            -1-\lambda & 2 & -1 \\
            2 & -\lambda & 2 \\
            3 & 2 & 3-\lambda \\
        \end{array}\right|
        = \left| \begin{array}{ccc}
            -\lambda & 2 & -1 \\
            0 & -\lambda & 2 \\
            \lambda & 2 & 3-\lambda \\
        \end{array}\right|
        = \left| \begin{array}{ccc}
            0 & 4 & 2-\lambda \\
            0 & -\lambda & 2 \\
            \lambda & 2 & 3-\lambda \\
        \end{array}\right| =\\=
        \lambda(8+\lambda(2-\lambda)) = \lambda(-\lambda^2+2\lambda+8)=-\lambda(\lambda-4)(\lambda+2)
    \end{multline*}

    Por tanto, ambas matrices tienen como valores propios: $\{-2,0,4\}$, por lo que son diagonalizables y semejantes a
        $$D = \left( \begin{array}{ccc}
            -2 & 0 & 0 \\
            0 & 0 & 0 \\
            0 & 0 & 4 \\
        \end{array}\right)$$

    Por tanto, como $A_1 \sim D \land D\sim A_3$, por ser $\sim$ una relación de equivalencia, $A_1 \sim A_3$.
\end{ejercicio}

\begin{ejercicio}
    Calcular $A^{12}$ y $A^{-7}$ para la matriz
    $$A = \left( \begin{array}{ccc}
            1 & 3 & 0 \\
            3 & -2 & -1 \\
            0 & -1 & 1 \\
        \end{array}\right)$$
        ¿$\exists B\in \mathcal{M}_3(\bb{R}) \mid B^2 = A$?\\

        En primer lugar, vemos si $A$ es diagonalizable.
        \begin{multline*}
            P_A(\lambda) = \left| \begin{array}{ccc}
                1-\lambda & 3 & 0 \\
                3 & -2-\lambda & -1 \\
                0 & -1 & 1-\lambda \\
            \end{array}\right|
            = -(1-\lambda)^2(2+\lambda)-(1-\lambda)-9(1-\lambda) = \\
            = -(1-\lambda)((1-\lambda)(2+\lambda)+1+9) = -(1-\lambda)(-\lambda^2-\lambda+12) = (1-\lambda)(\lambda+4)(\lambda-3)
        \end{multline*}

        Por tanto, como tiene tres valores propios distintos $\{1,3,-4\}$, es diagonalizable. Calculamos los suespacios propios.
        \begin{equation*}\begin{split}
           V_{1} & = \left\{ \left(\begin{array}{c}
                    x_1 \\
                    x_2 \\
                    x_3
               \end{array}\right) \in \bb{R}^3 \mid (A-I)\left(\begin{array}{c}
                    x_1 \\
                    x_2 \\
                    x_3
               \end{array}\right) = 0 \right\} \\
               & = \left\{ \left(\begin{array}{c}
                    x_1 \\
                    x_2 \\
                    x_3
               \end{array}\right) \in \bb{R}^3 \mid \left( \begin{array}{ccc}
                    0 & 3 & 0 \\
                    3 & -3 & -1 \\
                    0 & -1 & 0 \\
            \end{array}\right) \left(\begin{array}{c}
                    x_1 \\
                    x_2 \\
                    x_3
               \end{array}\right) = 0 \right\} \\
               & = \left\{ \left(\begin{array}{c}
                    x_1 \\
                    x_2  \\
                    x_3
               \end{array}\right) \in \bb{R}^3 \left| \begin{array}{c}
                    x_2 = 0 \\
                    3x_1 - 3x_2 - x_3 = 0
               \end{array}\right. \right\} = \mathcal{L}\left(\left\{
                    \left(\begin{array}{c}
                        1 \\
                        0 \\
                        3 \\
                   \end{array}\right)
                   \right\}\right)
       \end{split}\end{equation*}
       \begin{equation*}\begin{split}
           V_{3} & = \left\{ \left(\begin{array}{c}
                    x_1 \\
                    x_2 \\
                    x_3
               \end{array}\right) \in \bb{R}^3 \mid (A-3I)\left(\begin{array}{c}
                    x_1 \\
                    x_2 \\
                    x_3
               \end{array}\right) = 0 \right\} \\
               & = \left\{ \left(\begin{array}{c}
                    x_1 \\
                    x_2 \\
                    x_3
               \end{array}\right) \in \bb{R}^3 \mid \left( \begin{array}{ccc}
                    -2 & 3 & 0 \\
                    3 & -5 & -1 \\
                    0 & -1 & -2 \\
            \end{array}\right) \left(\begin{array}{c}
                    x_1 \\
                    x_2 \\
                    x_3
               \end{array}\right) = 0 \right\} \\
               & = \left\{ \left(\begin{array}{c}
                    x_1 \\
                    x_2  \\
                    x_3
               \end{array}\right) \in \bb{R}^3 \left| \begin{array}{c}
                    -2x_1 + 3x_2 = 0 \\
                    x_2+2x_3 = 0
               \end{array}\right. \right\} = \mathcal{L}\left(\left\{
                    \left(\begin{array}{c}
                        3 \\
                        2 \\
                        -1 \\
                   \end{array}\right)
                   \right\}\right)
       \end{split}\end{equation*}
       \begin{equation*}\begin{split}
           V_{-4} & = \left\{ \left(\begin{array}{c}
                    x_1 \\
                    x_2 \\
                    x_3
               \end{array}\right) \in \bb{R}^3 \mid (A+4I)\left(\begin{array}{c}
                    x_1 \\
                    x_2 \\
                    x_3
               \end{array}\right) = 0 \right\} \\
               & = \left\{ \left(\begin{array}{c}
                    x_1 \\
                    x_2 \\
                    x_3
               \end{array}\right) \in \bb{R}^3 \mid \left( \begin{array}{ccc}
                    5 & 3 & 0 \\
                    3 & 2 & -1 \\
                    0 & -1 & 5 \\
            \end{array}\right) \left(\begin{array}{c}
                    x_1 \\
                    x_2 \\
                    x_3
               \end{array}\right) = 0 \right\} \\
               & = \left\{ \left(\begin{array}{c}
                    x_1 \\
                    x_2  \\
                    x_3
               \end{array}\right) \in \bb{R}^3 \left| \begin{array}{c}
                    5x_1 + 3x_2 = 0 \\
                    -x_2 + 5x_3 = 0
               \end{array}\right. \right\} = \mathcal{L}\left(\left\{
                    \left(\begin{array}{c}
                        -3 \\
                        5 \\
                        1 \\
                   \end{array}\right)
                   \right\}\right)
       \end{split}\end{equation*}
       Por tanto, $A$ es diagonalizable de la forma $D=P^{-1}AP$, con:
       \begin{equation*}
           D=\left( \begin{array}{ccc}
                1 & 0 & 0  \\
                0 & 3 & 0  \\
                0 & 0 & -4  \\
           \end{array}\right) \qquad \qquad
           P=\left( \begin{array}{ccc}
                1 & 3 & -3  \\
                0 & 2 & 5  \\
                3 & -1 & 1  \\
           \end{array}\right)
       \end{equation*}
       Por tanto, despejando $A$, $A=PDP^{-1}$. Por tanto,
       $$A^{12} = PD^{12}P^{-1},\text{ con }D^{12}=\left( \begin{array}{ccc}
                1 & 0 & 0  \\
                0 & 3^{12} & 0  \\
                0 & 0 & (-4)^{12}  \\
           \end{array}\right)$$
       $$A^{-7} = PD^{-7}P^{-1},\text{ con }D^{-7}=\left( \begin{array}{ccc}
                1 & 0 & 0  \\
                0 & \frac{1}{3^7} & 0  \\
                0 & 0 & \frac{-1}{4^7}  \\
           \end{array}\right)$$
       $$B=\sqrt{A} = P\sqrt{D}P^{-1},\text{ con }\sqrt{D}=\left( \begin{array}{ccc}
            1 & 0 & 0  \\
            0 & \sqrt{3} & 0  \\
            0 & 0 & \sqrt{-4}  \\
       \end{array}\right) \in \bb{C}$$
       Por tanto, $\nexists B\in \mathcal{M}_3(\bb{R}) \mid B^2 = A$, ya que $\sqrt{-4}\notin \bb{R}$.
\end{ejercicio}

\begin{ejercicio}
    Dada la ecuación $A^2 = 9I$:
    \begin{enumerate}
        \item Resolver en $\mathcal{M}_2(\bb{R})$.\\
        Transformamos la ecuación a $I = \left( \frac{1}{3}A\right)^2$. Por tanto, buscamos $B \in \mathcal{M}_2(\bb{R}) \mid B^2 = I$.

        \begin{equation*}
            B = I_2 \quad \text{ó} \quad 
            B = -I_2
            \quad \text{ó} \quad
            B\sim\footnote{En los dos primeros casos, no hay más matrices semejantes a ellas, ya que $B=\pm I$} \left(\begin{array}{cc}
                1 & 0\\
                0 &-1
            \end{array} \right)
        \end{equation*}
        Por tanto, las solcuiones de $A^2 = 9I$ son $A=3B$, para las $B$ indicadas previamente.

        \item Resolver en $\mathcal{M}_3(\bb{R})$.\\
        Transformamos la ecuación a $I = \left( \frac{1}{3}A\right)^2$. Por tanto, buscamos $B \in \mathcal{M}_3(\bb{R}) \mid B^2 = I$.

        \begin{equation*}
            B = I_3 \quad \text{ó} \quad 
            B = -I_3
            \quad \text{ó} \quad
            B\sim \left(\begin{array}{ccc}
                1 & 0 & 0\\
                0 & 1 & 0\\
                0 & 0 & -1\\
            \end{array} \right)
            \quad \text{ó} \quad
            B\sim \left(\begin{array}{ccc}
                1 & 0 & 0\\
                0 & -1 & 0\\
                0 & 0 & -1\\
            \end{array} \right)
        \end{equation*}
        Por tanto, las soluciones de $A^2 = 9I$ son $A=3B$, para las $B$ indicadas previamente.
    \end{enumerate}
\end{ejercicio}

\begin{ejercicio}
    Sea la ecuación $A^2=0$:
    \begin{enumerate}
        \item Resolver en $\mathcal{M}_3(\bb{R})$:\\
        Sea $A=M(f;\mathcal{B})$, con $f\in End(\bb{R}^3)$ t.q. $f\circ f = 0$.

        Calculemos los valores propios. Si $\lambda_0$ es un valor propio de $A$, \begin{multline*}
            \exists v\in \bb{R}^3-\{0\} \mid f(v) = \lambda_0 v \Longrightarrow f(f(v)) = 0 = f(\lambda_0v) = \lambda_0f(v) = \lambda_0^2v\Longrightarrow \\ \Longrightarrow 0=\lambda_0^2v \Longrightarrow \lambda_0 = 0 \text{ es el único valor propio posible.}
        \end{multline*}
    
        Por tanto, $\lambda_0=0$ será una raíz del polinomio característico de $f$ con multiplicidad algebraica $m_0 \geq 1$. Por tanto, la multiplicidad geométrica debe ser $n_0\geq 1$.
        \begin{equation}
            n_0 = \dim Ker(f) \geq 1
        \end{equation}
        
        Además, como $f\circ f = 0 \Longrightarrow Im(f) \subset Ker(f)$, ya que $\forall  f(v) \in Im(f),\; f(f(v))=~0$ $ \Longrightarrow f(v) \in Ker(f)$.
        \begin{equation}\label{Ec:Subset}
            Im(f) \subset Ker(f) \Longrightarrow \dim Im(f) \leq \dim Ker(f) = n_0
        \end{equation}
        
       Además, sabemos que
       \begin{equation}\label{Ec:SumDim}
           \dim Ker(f) + \dim Im(f) = 3
       \end{equation}
    
       \begin{itemize}
           \item \underline{Supongamos $n_0 = 3$:}\\
           Como $\dim Ker(f)=3 \Longrightarrow Ker(f) = \bb{R}^3 \Longrightarrow f=0 \Longrightarrow A=0$
    
           \item \underline{Supongamos $n_0 = 1$:}\\
           Por la Ec. \ref{Ec:SumDim}, $\dim Im(f)=2$, pero por la Ec. \ref{Ec:Subset}, $2\leq 1$, lo que es una contradicción.
    
           \item \underline{Supongamos $n_0 = 2$:}\\
           Por la Ec. \ref{Ec:SumDim}, $\dim Im(f)=1$. Sea $\bar{\mathcal{B}}=\{e_1, e_2, e_3\}$ base de $\bb{R}^3$. Sea $\{e_2\}$ la base de $Im(f)$. Ampliamos a una base $\{e_2, e_3\}$ de $Ker(f)$. Además, definimos $f(e_1)=~e_2$.
           Por tanto,
           $$\bar{A} = M(f; \bar{\mathcal{B}}) = \left( \begin{array}{ccc}
               0 & 0 & 0 \\
               1 & 0 & 0 \\
               0 & 0 & 0 \\
           \end{array}\right)$$
       \end{itemize}
    
       Por tanto, las soluciones de la ecuación $A^2 = 0$ son\footnote{En este caso se dice que solo hay una clase de semejanza}:
       \begin{equation*}
           A = 0
           \qquad \text{ ó } \qquad
           A\sim \left( \begin{array}{ccc}
               0 & 0 & 0 \\
               1 & 0 & 0 \\
               0 & 0 & 0 \\
           \end{array}\right)
       \end{equation*}

        \item Resolver en $\mathcal{M}_4(\bb{R})$:\\
        Seguimos un razonamiento análogo al apartado anterior.
       \begin{itemize}
           \item \underline{Supongamos $n_0 = 4$:}\\
           Como $\dim Ker(f)=4 \Longrightarrow Ker(f) = \bb{R}^4 \Longrightarrow f=0 \Longrightarrow A=0$
    
           \item \underline{Supongamos $n_0 = 1$:}\\
           Por la análoga a la Ec. \ref{Ec:SumDim}, $\dim Im(f)=3$, pero por la Ec. \ref{Ec:Subset}, $3\leq 1$, lo que es una contradicción.
    
           \item \underline{Supongamos $n_0 = 3$:}\\
           Por la análoga a la Ec. \ref{Ec:SumDim}, $\dim Im(f)=1$. Sea $\bar{\mathcal{B}}=\{e_1, e_2, e_3, e_4\}$ base de $\bb{R}^4$. Sea $\{e_2\}$ la base de $Im(f)$. Ampliamos a una base $\{e_2, e_3, e_4\}$ de $Ker(f)$. Además, definimos $f(e_1)=~e_2$.
           Por tanto,
           $$\bar{A} = M(f; \bar{\mathcal{B}}) = \left( \begin{array}{cccc}
               0 & 0 & 0 & 0\\
               1 & 0 & 0 & 0\\
               0 & 0 & 0 & 0\\
               0 & 0 & 0 & 0\\
           \end{array}\right)$$

           \item \underline{Supongamos $n_0 = 2$:}\\
           Por la análoga a la Ec. \ref{Ec:SumDim}, $\dim Im(f)=2$. Sea $\bar{\mathcal{B}}=\{e_1, e_2, e_3, e_4\}$ base de $\bb{R}^4$. Sea $\{e_3, e_4\}$ la base de $Im(f)$ y de $Ker(f)$. Además, definimos $f(e_1)=~e_3,\;f(e_2)=~e_4$.
           Por tanto,
           $$\bar{A} = M(f; \bar{\mathcal{B}}) = \left( \begin{array}{cccc}
               0 & 0 & 0 & 0\\
               0 & 0 & 0 & 0\\
               1 & 0 & 0 & 0\\
               0 & 1 & 0 & 0\\
           \end{array}\right)$$
       \end{itemize}
    
       Por tanto, las soluciones de la ecuación $A^2 = 0$ son, con 2 clases de semejanza,
       \begin{equation*}
           A = 0
           \qquad \text{ ó } \qquad
           A\sim \left( \begin{array}{cccc}
               0 & 0 & 0 & 0\\
               1 & 0 & 0 & 0\\
               0 & 0 & 0 & 0\\
               0 & 0 & 0 & 0\\
           \end{array}\right)
           \qquad \text{ ó } \qquad
           A\sim \left( \begin{array}{cccc}
               0 & 0 & 0 & 0\\
               0 & 0 & 0 & 0\\
               1 & 0 & 0 & 0\\
               0 & 1 & 0 & 0\\
           \end{array}\right)
       \end{equation*}

       \item Resolver en $\mathcal{M}_2(\bb{R})$:\\
        Seguimos un razonamiento análogo al apartado anterior.
       \begin{itemize}
           \item \underline{Supongamos $n_0 = 2$:}\\
           Como $\dim Ker(f)=2 \Longrightarrow Ker(f) = \bb{R}^2 \Longrightarrow f=0 \Longrightarrow A=0$
    
           \item \underline{Supongamos $n_0 = 1$:}\\
           Por la análoga a la Ec. \ref{Ec:SumDim}, $\dim Im(f)=1$. Sea $\bar{\mathcal{B}}=\{e_1, e_2\}$ base de $\bb{R}^2$. Sea $\{e_2\}$ la base de $Im(f)$ y $Ker(f)$. Además, definimos $f(e_1)=~e_2$.
           Por tanto,
           $$\bar{A} = M(f; \bar{\mathcal{B}}) = \left( \begin{array}{cc}
               0 & 0\\
               1 & 0\\
           \end{array}\right)$$
       \end{itemize}
    
       Por tanto, las soluciones de la ecuación $A^2 = 0$, con 1 clase de semejanza, son:
       \begin{equation*}
           A = 0
           \qquad \text{ ó } \qquad
           A\sim \left( \begin{array}{cc}
               0 & 0\\
               1 & 0\\
           \end{array}\right)
       \end{equation*}
       
    \end{enumerate}
    
\end{ejercicio}

\begin{ejercicio}
    Resolver la ecuación $f^3=f \in End_\bb{R}\bb{R}^n$.\\
    
    Calculemos los posibles valores propios. Si $\lambda_0$ es un valor propio de $f$,
    \begin{multline*}
        \exists v\in \bb{R}^3-\{0\} \mid f(v) = \lambda_0 v \Longrightarrow f^3(v) = \lambda_0^3v\Longrightarrow \lambda_0^3v=\lambda_0v
        \Longrightarrow \\ \Longrightarrow
        \lambda_0^3 = \lambda_0 \Longrightarrow \lambda_0 = \{-1,0,1\} \text{ son los posibles valores propios.}
    \end{multline*}
    \begin{itemize}
        \item \underline{Caso 1:} $f\in Aut(\bb{R}^n)$\\
        Como $f^3 = f\Longrightarrow f(f^2-Id) = 0$. Además, como $f$ es biyectiva, $\exists f^{-1}$. Por tanto, las soluciones son $f^2=Id$.
        
        Por tanto, las soluciones tomando $A=M(f, \mathcal{B})$ son:
        \begin{equation*}
            A\sim \left(\begin{array}{cccccc}
                1&&&&&\\
                &\ddots^{r\;veces}&&&&\\
                &&1&&&\\
                &&&-1&&\\
                &&&&\ddots^{s\;veces}&\\
                &&&&&-1\\
            \end{array} \right), \text{ con $r$ y $s$ tal que $r+s=n$}
        \end{equation*}

        \item \underline{Caso 2:} $f$ no biyectiva. Es decir, $ker(f)\neq \{0\}$\\
        Sea $v\in Ker(f)\cap Im(f) -\{0\}$
        \begin{itemize}
            \item $\exists w \in V \mid f(w) = v$
            \item $f(v)=0$
        \end{itemize}
        Por tanto, $f^2(w)=0 \Longrightarrow f^3(w) = f(0) = 0 \Longrightarrow f^3(w) = f(w) = 0$. Por tanto, como $f(w)=v=0 \Longrightarrow v=0$. Pero $v\neq 0$, por lo que llegamos a una contradicción. Por tanto, $Ker(f)\cap Im(f)=\{0\}$. En conclusión,
        $$Ker(f)\oplus Im(f) = V$$

        La restricción $h:=f|_{Im(f)}:Im(f) \to Im(f)$ tiene núcleo trivial.
        $$Ker(h)=\{0\}$$
        Por tanto, nos referimos al apartado anterior, ya que la restricción $h$ es un automorfismo.
        
        Por tanto, existe una base de $Im(f)\;\{e_1, \dots, e_r\}$ t.q. $f(e_i)=\pm e_i\quad \forall i$.

        Sea $\{e_{r+1}, \dots, e_n\}$ base de $Ker(f)$.  Por tanto, $\{e_1, \dots, e_r, e_{r+1}, \dots, e_n\}$ es una base de $V$. Por tanto, $f$ es diagonalizable.
        
        Por tanto, las soluciones tomando $A=M(f, \mathcal{B})$ son:
        \begin{equation*}
            A\sim \left(\begin{array}{ccccccccc}
                1&&&&&&&&\\
                &\ddots^{x\;veces}&&&&&&&\\
                &&1&&&&&&\\
                &&&-1&&&&&\\
                &&&&\ddots^{y\;veces}&&&&\\
                &&&&&-1&&&\\
                &&&&&&0&&\\
                &&&&&&&\ddots^{t\;veces}&\\
                &&&&&&&&0\\
            \end{array} \right) \text{ con $x,y,t\mid x+y+t=n$}
        \end{equation*}
    \end{itemize}
\end{ejercicio}

\begin{ejercicio}
    Sea $f\in End(\bb{R}^3)$ que respecto a la base usual $\mathcal{B}_u$ tiene la matriz asociada
    \begin{equation*}
        M(f, \mathcal{B}_u) = A = \left(\begin{array}{ccc}
            1+\alpha & -\alpha & \alpha \\
            2+\alpha & -\alpha & \alpha-1 \\
            2 & -1 & 0
        \end{array} \right)
    \end{equation*}
    \begin{enumerate}
        \item Estudiar para qué valores de $\alpha \in \bb{R}$ la matriz es diagonalizable.
        \begin{multline*}
            P_f(\lambda) = \left|\begin{array}{ccc}
            1+\alpha-\lambda & -\alpha & \alpha \\
            2+\alpha & -\alpha-\lambda & \alpha-1 \\
            2 & -1 & -\lambda
        \end{array} \right| =
        \left|\begin{array}{ccc}
            1+\alpha-\lambda & -\alpha & 0 \\
            2+\alpha & -\alpha-\lambda & -1-\lambda \\
            2 & -1 & -1-\lambda
        \end{array} \right| = \\ =
        \left|\begin{array}{ccc}
            1+\alpha-\lambda & -\alpha & 0 \\
            \alpha & 1-\alpha-\lambda & 0 \\
            2 & -1 & -1-\lambda
        \end{array} \right|
        = -(1+\lambda)\left|\begin{array}{cc}
            1+\alpha-\lambda & -\alpha\\
            \alpha & 1-\alpha-\lambda\\
        \end{array} \right| = \\ = -(1+\lambda)\left[ (1-\lambda+\alpha)(1-\lambda-\alpha)+\alpha^2 \right]
        = -(1+\lambda)\left[ (1-\lambda)^2 - \alpha^2 +\alpha^2 \right] = \\=
        -(1+\lambda)(1-\lambda)^2
        \end{multline*}

        Por tanto, independientemente del valor de $\alpha$, los valores propios son: $\{1, -1\}$. Veamos la multiplicidad geométrica del $1$:
        \begin{equation*}
            rg(A-I) = rg\left(\begin{array}{ccc}
            \alpha & -\alpha & \alpha \\
            2+\alpha & -\alpha-1 & \alpha-1 \\
            2 & -1 & -1
        \end{array} \right)
        \end{equation*}
        Calculamos el rango mediante determinantes.
        \begin{multline*}
            det(A-I) = \left|\begin{array}{ccc}
            \alpha & -\alpha & \alpha \\
            2+\alpha & -\alpha-1 & \alpha-1 \\
            2 & -1 & -1
        \end{array} \right| = \alpha \left|\begin{array}{ccc}
            1 & -1 & 1 \\
            2+\alpha & -\alpha-1 & \alpha-1 \\
            2 & -1 & -1
        \end{array} \right| =\\=
        \alpha \left|\begin{array}{ccc}
            1 & -1 & 0 \\
            2+\alpha & -\alpha-1 & -2 \\
            2 & -1 & -2
        \end{array} \right|
        =\alpha \left|\begin{array}{ccc}
            1 & -1 & 0 \\
            \alpha & -\alpha & 0 \\
            2 & -1 & -2
        \end{array} \right| = \alpha \cdot 0 = 0
        \end{multline*}
        \begin{equation*}
            \left| \begin{array}{cc}
                \alpha & -\alpha \\
                2 & -1
            \end{array}\right| = -\alpha +2\alpha = \alpha = 0 \Longleftrightarrow \alpha = 0
        \end{equation*}

        \begin{itemize}
            \item \underline{Para $a\neq 0$:}\\
            Si $\alpha\neq 0\Longrightarrow rg(A-I)=2 \Longrightarrow \dim V_1 = 3-2 = 1$.
            \begin{table}[H]
                \centering
                \begin{tabular}{c|c|c}
                    Valores Propios & Mult. Alg. & Mult. Geom. \\ \hline 
                    1 & 2 & 1\\
                    $-$1 & 1 & 1\\
                \end{tabular}
                \caption{Valores propios con sus multiplicidades para $\alpha\neq 0$}
            \end{table}
            Por tanto, para $\alpha\neq 0, f$ no es diagonalizable.

            \item \underline{Para $\alpha=0$:}\\
            \begin{equation*}
                rg(A-I) = rg\left(\begin{array}{ccc}
                    0 & 0 & 0 \\
                    2 & -1 & -1 \\
                    2 & -1 & -1
                \end{array} \right) = 1
            \end{equation*}
            Como $rg(A-I)=1 \Longrightarrow \dim V_1 = 3-1 = 2$.
            \begin{table}[H]
                \centering
                \begin{tabular}{c|c|c}
                    Valores Propios & Mult. Alg. & Mult. Geom. \\ \hline 
                    1 & 2 & 2\\
                    $-$1 & 1 & 1\\
                \end{tabular}
                \caption{Valores propios con sus multiplicidades para $\alpha= 0$}
            \end{table}
            Por tanto, para $\alpha= 0, f$ sí es diagonalizable.
        \end{itemize}
        
        \item Diagonalizar $f$ dando la matriz de cambio de base.\\
        Diagonalizamos para para $\alpha=0$, ya que es el único valor para el cual $f$ es diagonalizable.
        \begin{equation*}
            M(f, \mathcal{B}_u) = A = \left(\begin{array}{ccc}
                1 & 0 & 0 \\
                2 & 0 & -1 \\
                2 & -1 & 0
            \end{array} \right)
        \end{equation*}
        \begin{equation*}\begin{split}
           V_{1} & = \left\{ \left(\begin{array}{c}
                    x_1 \\
                    x_2 \\
                    x_3
               \end{array}\right) \in \bb{R}^3 \mid (A-I)\left(\begin{array}{c}
                    x_1 \\
                    x_2 \\
                    x_3
               \end{array}\right) = 0 \right\} \\
               & = \left\{ \left(\begin{array}{c}
                    x_1 \\
                    x_2 \\
                    x_3
               \end{array}\right) \in \bb{R}^3 \mid \left( \begin{array}{ccc}
                    0 & 0 & 0 \\
                    2 & -1 & -1 \\
                    2 & -1 & -1
            \end{array}\right) \left(\begin{array}{c}
                    x_1 \\
                    x_2 \\
                    x_3
               \end{array}\right) = 0 \right\} \\
               & = \left\{ \left(\begin{array}{c}
                    x_1 \\
                    x_2  \\
                    x_3
               \end{array}\right) \in \bb{R}^3 \left| \begin{array}{c}
                    2x_1 -x_2 - x_3 = 0
               \end{array}\right. \right\} = \mathcal{L}\left(\left\{
                    \left(\begin{array}{c}
                        0 \\
                        -1 \\
                        1 \\
                   \end{array}\right),
                   \left(\begin{array}{c}
                        1 \\
                        1 \\
                        1 \\
                   \end{array}\right)
                   \right\}\right)
       \end{split}\end{equation*}
       \begin{equation*}\begin{split}
           V_{-1} & = \left\{ \left(\begin{array}{c}
                    x_1 \\
                    x_2 \\
                    x_3
               \end{array}\right) \in \bb{R}^3 \mid (A+I)\left(\begin{array}{c}
                    x_1 \\
                    x_2 \\
                    x_3
               \end{array}\right) = 0 \right\} \\
               & = \left\{ \left(\begin{array}{c}
                    x_1 \\
                    x_2 \\
                    x_3
               \end{array}\right) \in \bb{R}^3 \mid \left( \begin{array}{ccc}
                    2 & 0 & 0 \\
                    2 & 1 & -1 \\
                    2 & -1 & 1
            \end{array}\right) \left(\begin{array}{c}
                    x_1 \\
                    x_2 \\
                    x_3
               \end{array}\right) = 0 \right\} \\
               & = \left\{ \left(\begin{array}{c}
                    x_1 \\
                    x_2  \\
                    x_3
               \end{array}\right) \in \bb{R}^3 \left| \begin{array}{c}
                    x_1 = 0\\
                    2x_1 + x_2 - x_3 = 0
               \end{array}\right. \right\} = \mathcal{L}\left(\left\{
                    \left(\begin{array}{c}
                        0 \\
                        1 \\
                        1 \\
                   \end{array}\right)
                   \right\}\right)
       \end{split}\end{equation*}
       Por tanto, $D=P^{-1}AP$, con:
       \begin{equation*}
        D = \left(\begin{array}{ccc}
            1 & 0 & 0\\
            0 & 1 & 0\\
            0 & 0 & -1\\
        \end{array} \right) \qquad \qquad
        P = \left(\begin{array}{ccc}
            0 & 1 & 0\\
            -1 & 1 & 1\\
            1 & 1 & 1\\
        \end{array} \right)
    \end{equation*}
    siendo $D=M(f, \mathcal{B}')$ la matriz asociada a $f$ y $P=M(\mathcal{B}', \mathcal{B}_u)$ la matriz de cambio de base.   
    \end{enumerate}
\end{ejercicio}

\begin{ejercicio}
    Sea $f\in End(V)$. Probar:
    \begin{enumerate}
        \item $f(V_\lambda) = V_\lambda$, para todo valor propio $\lambda\neq 0$.
        \begin{proof}
            Veamos en primer lugar que $f(V_\lambda) \subset V_\lambda$:
            \begin{equation}\label{Ec:9.a.Subset}
                \forall f(v) \in f(V_\lambda), f(v) = \lambda v \in V_\lambda \Longrightarrow f(V_\lambda) \subset V_\lambda
            \end{equation}
            donde hemos usado que $\lambda$ es un escalar y que $V_\lambda$ es un subespacio vectorial, por lo que es cerrado para el producto por escalares.
    
            Veamos ahora que tienen la misma dimensión.
            $$Ker(V_\lambda) = \{v\in V_\lambda \mid \lambda v = 0\} = \{0\},\text{ ya que } \lambda \neq 0$$
            \begin{equation}\label{Ec:9.a.Dim}
                \dim Im(V_\lambda) = \dim V_\lambda - \dim Ker(V_\lambda) = \dim V_\lambda
            \end{equation}
    
            Por tanto, por las ecuaciones \ref{Ec:9.a.Subset} y \ref{Ec:9.a.Dim},
            $$f(V_\lambda) = V_\lambda \qquad \forall \lambda\neq 0$$
        \end{proof}

        \item $f \in Aut(V) \Longleftrightarrow 0$ no es un valor propio de $f$.
        \begin{proof} Procedemos mediante doble implicación:
            \begin{description}
                \item [$\Longrightarrow )$] Suponemos $f$ automorfismo.\\
                Por tanto, $f$ es homomorfismo, por lo que $Ker(f) = \{0\}$.
                Por tanto,
                $$\nexists v \in V-\{0\} \mid f(v) = 0v = 0$$
                Por tanto, el $0$ no es un valor propio de $f$.

                \item [$\Longleftarrow )$] Suponemos que el 0 no es un valor propio de $f$.\\
                Entonces, $Ker(f) = \{0\}$, por lo que $f$ es un homomorfismo.

                Además, como $\dim Im(f) = \dim V - \dim Ker(f) = \dim V$, también es un epimorfismo.

                Por tanto, $f$ es un isomorfismo, por lo que es un automorfismo.
            \end{description}
        \end{proof}

        \item Sea $f \in Aut(V)$. $\lambda$ es un valor propio de $f \Longleftrightarrow \lambda^{-1}$ es un valor propio de $f^{-1}$.
         \begin{proof} Procedemos mediante doble implicación:
            \begin{description}
                \item [$\Longrightarrow )$] Suponemos $\lambda$ valor propio de $f$.\\
                En primer lugar, hay que destacar que como $f$ es un automorfismo, y por lo demostrado en el apartado anterior, $\lambda\neq 0$.

                Como $\lambda$ es un valor propio, $\exists v \in V-\{0\} \mid f(v) = \lambda v$.
                \begin{multline*}
                    (f^{-1}\circ f)(v) = v \Longrightarrow f^{-1}(f(v)) = f^{-1}(\lambda v) = \lambda f^{-1}(v) =v
                    \Longrightarrow \\ \Longrightarrow
                    f^{-1}(v) = \lambda ^{-1} v \Longrightarrow \lambda^{-1} \text{ es un valor propio de } f^{-1}
                \end{multline*}

                \item [$\Longleftarrow )$] Suponemos $\lambda^{-1}$ valor propio de $f^{-1}$.\\
                En primer lugar, hay que destacar que como $f$ es un automorfismo, $f^{-1}$ también lo es y, por lo demostrado en el apartado anterior, $\lambda^{-1} \neq 0$.

                Como $\lambda^{-1}$ es un valor propio, $\exists v \in V-\{0\} \mid f^{-1}(v) = \lambda^{-1} v$.
                \begin{multline*}
                    (f\circ f^{-1})(v) = v \Longrightarrow f(f^{-1}(v)) = f(\lambda^{-1} v) = \lambda^{-1} f(v) =v
                    \Longrightarrow \\ \Longrightarrow
                    f(v) = \lambda v \Longrightarrow \lambda \text{ es un valor propio de } f
                \end{multline*}
            \end{description}
            
        \end{proof}
        
    \end{enumerate}
    
\end{ejercicio}

\begin{ejercicio}
    Sea $f\in End(\bb{R}^2) \mid nul(f) = 1$. Probar que $f$ es diagonalizable si y solo si $Ker(f) \cap Im(f) = \{0\}$.
    \begin{proof}Procedemos mediante doble implicación:
        \begin{description}
            \item [$\Longrightarrow )$]Suponemos $f$ diagonalizable.
            
            Como $nul(f) = \dim Ker(f) = 1$, sea $\{e_2\}$ base de $Ker(f)$.
            $$f(e_2) = 0$$
            Amplío a una base de $\bb{R}^2, \mathcal{B} = \{e_1,e_2\}$
            $$f(e_1) = ae_1 + be_2$$
            $$M(f;\mathcal{B}) = \left(\begin{array}{cc}
                a & 0 \\
                b & 0
            \end{array} \right)$$
            Su polinomio característico es: $P_f(\lambda) = \lambda(\lambda-a)$, por lo que tiene como valores propios $\{0,a\}$.
        
            Por el Teorema Fundamental de Diagonalización, como $f$ es diagonalizable y $\dim V_0 = \dim Ker(f) = 1$, la multiplicidad algebraica del 0 $m_0 = 1$. Por tanto, $$a\neq 0$$

            Veamos ahora el valor de $Ker(f) \cap Im(f)$. Supongamos $x \in Ker(f) \cap Im(f):$
            \begin{equation*}\begin{split}
                x \in Ker(f) & \Longrightarrow f(x) = 0 \Longrightarrow
                 \left(\begin{array}{cc}
                    a & 0 \\
                    b & 0
                \end{array} \right)
                 \left(\begin{array}{c}
                    x_1 \\
                    x_2
                \end{array} \right) = 
                \left(\begin{array}{c}
                    ax_1 \\
                    bx_1
                \end{array} \right) = 0 \Longrightarrow\\
               &  \qquad \Longrightarrow ax_1 = 0 \Longrightarrow x_1 = 0 \\
               x \in Im(f) & \Longrightarrow \exists y\in \bb{R}^2 \mid f(y) = x \Longrightarrow
                 \left(\begin{array}{cc}
                    a & 0 \\
                    b & 0
                \end{array} \right)
                 \left(\begin{array}{c}
                    y_1 \\
                    y_2
                \end{array} \right) = 
                \left(\begin{array}{c}
                    ay_1 \\
                    by_1
                \end{array} \right) =
                \left(\begin{array}{c}
                    0 \\
                    x_2
                \end{array} \right) \Longrightarrow\\
               &  \qquad \Longrightarrow \left\{\begin{array}{c}
                   ay_1 = 0 \Longrightarrow y_1 = 0   \\
                   by_1 = x_2 \Longrightarrow x_2 = 0 
               \end{array} \right. \\
            \end{split}\end{equation*}
            Por tanto, como $x_1=x_2 = 0$, $x=0$ y por tanto $Ker(f) \cap Im(f)=\{0\}$.

            \item [$\Longleftarrow )$]Suponemos $Ker(f) \cap Im(f)=\{0\}$. Veamos que $f$ es diagonalizable.

            Veamos que $V = Ker(f) \oplus~Im(f)$:
            \begin{multline*}
                \dim \bb{R}^2 = \dim Im(f) + \dim Ker(f) = 
                \dim(Im(f) + Ker(f)) + \dim(Im(f) \cap Ker(f)) = \\ = \dim(Im(f) + Ker(f))
            \end{multline*}
            Como $Im(f)\subset \bb{R}^2$ y $Ker(f)\subset \bb{R}^2 \Longrightarrow Im(f) + Ker(f) \subset \bb{R}^2$. Como además, tienen la misma dimensión, se da la igualdad, $Im(f) + Ker(f) = \bb{R}^2$. Por último, como $Ker(f) \cap Im(f)=\{0\}$, la suma es directa.
            $$Im(f) \oplus Ker(f) = \bb{R}^2$$

            Como $\dim Ker(f)=1$, $\dim Im(f) = 1$. Sea $\{e_2\}$ base de $Ker(f)$ y $\{e_1\}$ base de $Im(f)$. Como es suma directa, la unión de las bases también es una base. Por tanto, $\mathcal{B} = \{e_1, e_2\}$ es una base de $\bb{R}^2$.
            $$f(e_1) = ae_1,\; a\neq 0 \qquad f(e_2) = 0 \qquad M(f;\mathcal{B}) = \left(\begin{array}{cc}
                a & 0 \\
                0 & 0
            \end{array} \right)$$
            Por tanto, los valores propios del endomorfismo son $\{0,a\}$. Como $a \neq 0$, tiene dos valores propios distinos. Por tanto, $f$ es diagonalizable.\qedhere
        \end{description}
    \end{proof}
\end{ejercicio}

\begin{ejercicio}
    Sea $f\in End(V) \mid f^2=f$. Probar que
    $$V = Ker(f) \oplus Ker(f-1_V).$$
    Deducir que $f$ es diagonalizable.
    \begin{proof}
        Demostramos en primer lugar que $Ker(f) \cap Ker(f-1_V) = \{0\}$. Sea $v\in Ker(f) \cap Ker(f-1_V)$.
        \begin{equation*}
            \begin{split}
                v&\in Ker(f) \Longrightarrow f(v) = 0\\
                v&\in Ker(f-1_V) \Longrightarrow (f-1_V)(v) = 0 = f(v) - v \Longrightarrow f(v) = v
            \end{split}
        \end{equation*}
        Por tanto, $v=f(v) = 0\Longrightarrow Ker(f) \cap Ker(f-1_V) = \{0\}$.
    
        Veamos ahora que $V=Ker(f) + Ker(f-1_V)$. Sea $v\in V$:
        $$v=\underbrace{v-f(v)}_{t_1} + \underbrace{f(v)}_{t_2}$$
    
        Veamos que $t_1 = v-f(v) \in Ker(f)$:
        $$f(t_1) = f(v-f(v)) = f(v) - f^2(v) = f(v) - f(v) = 0 \Longrightarrow t_1 \in Ker(f)$$
        
         Veamos que $t_2 = f(v) \in Ker(f-1_V)$:
        $$(f-1_V)(t_2) =  (f-1_V)(f(v)) = f(v)-f(v) = 0\Longrightarrow t_2 \in Ker(f-1_V)$$
    
        Por tanto, como $V=Ker(f) + Ker(f-1_V)$ y, además, son conjuntos disjuntos, $$V=Ker(f) \oplus Ker(f-1_V)$$
    
        Veamos ahora que $f$ es diagonalizable:
        \begin{equation}\label{RazonamientoErroneo2}\tag{$\ast\ast$}
        \begin{split}
            f(t_1) = 0 & \xrightarrow{(\ast\ast)} t_1 \text{ es un vector propio.}\quad t_1\in V_0 \\
            f(t_2) = f(f(v)) = f(v) = t_2 & \xrightarrow{(\ast\ast)} t_2 \text{ es un vector propio.}\quad t_2\in V_{1}
        \end{split}\end{equation}
    
        Es fácil ver que $V_0 = Ker(f)$ y $V_1 = Ker(f-1_V)$. Por tanto, sea $\mathcal{B}_0=\{v_1, \dots, v_k\}$ base de $V_0 = Ker(f)$ y sea $\mathcal{B}_1=\{u_1, \dots, u_t\}$ base de $V_1 = Ker(f-1_V)$. $\mathcal{B}_0 \cup \mathcal{B}_1$ será base de $V$ por ser suma directa, y además todos sus elementos serán vectores propios, por lo que será diagonalizable.
    
        (\ref{RazonamientoErroneo2}) Este razonamiento es válido siempre que $f\neq \{Id, 0\}$, ya que $t_1$ o $t_2$ serían vectores nulos. Sin embargo, tanto $f=Id$ como $f=0$ son diagonalizables.
    \end{proof}
\end{ejercicio}

\begin{ejercicio}
    Sea $A \in \mathcal{M}_n(\bb{R})$ t.q. $|A|<0$. Estudiar si $\exists B\in \mathcal{M}_n(\bb{R}) \mid B^2=A$.\\

    Supongamos que sí existe. Tomando determinantes,
    $$|B^2| = |A| \Longrightarrow |B|^2 = |A| \Longrightarrow |B|^2 < 0$$
    Por tanto, llegamos a una contradicción, ya que no podemos encontrar $|B|\in \bb{R}$. Por tanto, $\nexists B \mid B^2=A$.
\end{ejercicio}

\begin{ejercicio}
    Dada $A\in \mathcal{M}_n(\bb{K})$, sea el espacio vectorial $\upsilon \subset \mathcal{M}_n(\bb{K})$ dado por:
    $$\upsilon = \left\{ p(A) \mid p\in \bb{K}[\lambda]\right\}$$
    Demostrar que $\dim \upsilon \leq n$.
    \begin{proof}
        Como $\dim \mathcal{M}_n(\bb{K}) = n^2 \Longrightarrow \dim \upsilon \leq n^2$.

        Aplicando el Teorema de Cayley-Hamilton,
        $$P_A(A) = 0 = \pm A^n + c_{n-1}A^{n-1} + \dots + c_1A + c_0\cdot I$$
        Por tanto, $A^n$ es combinación lineal de $\{I, A,\dots, A^{n-1}\}$.
    
        Divido $p(\lambda)$ un polinomio cualquiera entre $P_A(\lambda)$:
        $$p(\lambda) = P_A(\lambda)q(\lambda) + r(\lambda)\qquad grd(r)<n$$
    
        Por tanto, $p(A) = \cancelto{0}{P_A(A)q(A)} + r(A)$.
    
        Por tanto, el subespacio $\upsilon$ tiene como sistema de generadores $\{I, A, \dots, A^{n-1}\} \Longrightarrow$
        $\Longrightarrow \dim \upsilon \leq n$
    \end{proof}
\end{ejercicio}

\begin{ejercicio}
    Sea $A\in \mathcal{M}_n(\bb{R}) \subset \mathcal{M}_n(\bb{C})$.

    \begin{enumerate}
        \item ¿Puede ser que sea diagonalizable en $\bb{R}$ pero no en $\bb{C}$?\\
        No, ya que $\mathcal{M}_n(\bb{R}) \subset \mathcal{M}_n(\bb{C})$.
        
        \item ¿Puede ser que sea diagonalizable en $\bb{C}$ pero no en $\bb{R}$?\\
        Sí, y como ejemplo tenemos la matriz $A=\left(\begin{array}{cc}
            0 & 1 \\
            1 & 0
        \end{array} \right)$, con polinomio característico $P_A(\lambda) = \lambda^2 + 1$.

        \item ¿Puede ser que sea no sea diagonalizable en $\bb{R}$ ni en $\bb{C}$?\\
        Sí, y como ejemplo tenemos la matriz $A=\left(\begin{array}{cc}
            0 & 0 \\
            1 & 0
        \end{array} \right)$, con polinomio característico $P_A(\lambda) = \lambda^2$
    \end{enumerate}
\end{ejercicio}

\begin{ejercicio}
    Sea $A\in \mathcal{M}_n(\bb{K})$ cuyos valores propios son $\{-1,1,2,-2\}$ con multiplicidad algebraica cualquiera. Calcular $$rg(A+3I).$$

    $\lambda_0$ valor propio $\Longrightarrow |A-\lambda_0I| = 0 \Longrightarrow (A-\lambda_0 I)$ no es regular.
    
    Como el $-3$ no es un valor propio, entonces la matriz $(A+3I)$ es regular y, por tanto,
    $$rg(A+3I) = n$$
\end{ejercicio}

\begin{ejercicio}
    Sea $A\in \mathcal{M}_n(\bb{R})$.
    $$A=\left( \begin{array}{cccc}
        1+a & 1 & \dots & 1 \\
        1 & 1+a & \dots & 1 \\
        \vdots & \vdots & \ddots & \dots \\
        1 & 1 & \dots & 1+a \\
    \end{array}\right)$$
    Estudiar para qué valores de $a\in \bb{R}$ la matriz $A$ es diagonalizable.\\

    Sea $B=\left( \begin{array}{cccc}
        1 & 1 & \dots & 1 \\
        1 & 1 & \dots & 1 \\
        \vdots & \vdots & \ddots & \vdots \\
        1 & 1 & \dots & 1 \\
    \end{array}\right)$.
    
    Por tanto, sea $p(x)=x+a$. Como $A=B+aI=p(B)$, y $B$ es diagonalizable (ver Ej. \ref{Ej:TodosUnos}) $\Longrightarrow A$ es diagonalizable $\forall a\in \bb{R}$.
\end{ejercicio}

\begin{ejercicio}
    Sea $f\in End(\bb{V}^3(\bb{R}))$ que en cierta base tiene como matriz asociada
    $$A=\left(\begin{array}{ccc}
        1+a & 1+a & 1 \\
        -a & -a & -1 \\
        a & a-1 & 0
    \end{array}\right)$$
    Estudiar para qué valores de $a\in \bb{R}$, el endomorfismo $f$ es diagonalizable.
    \begin{multline*}
        P_f(\lambda) = |A-\lambda I|
        = \left|\begin{array}{ccc}
            1+a-\lambda & 1+a & 1 \\
            -a & -a-\lambda & -1 \\
            a & a-1 & -\lambda
        \end{array}\right|
        = \left|\begin{array}{ccc}
            1-\lambda & 1-\lambda & 0 \\
            -a & -a-\lambda & -1 \\
            a & a-1 & -\lambda
        \end{array}\right| =\\
        = \left|\begin{array}{ccc}
            1-\lambda & 0 & 0 \\
            -a & -\lambda & -1 \\
            a & -1 & -\lambda
        \end{array}\right| = (1-\lambda)(\lambda^2 - 1) = (1-\lambda)(\lambda+1)(\lambda-1) = -(\lambda+1)(\lambda-1)^2
    \end{multline*}
    Por tanto, los valores propios de $f$ son: $\{1, -1\}$.

    \begin{equation*}
        rg(A-I) = rg\left(\begin{array}{ccc}
            a & 1+a & 1 \\
            -a & -a-1 & -1 \\
            a & a-1 & -1
        \end{array}\right) = rg\left(\begin{array}{ccc}
            a & 1+a & 1 \\
            0 & 0 & 0 \\
            0 & 2 & 2
        \end{array}\right) = \left\{
        \begin{array}{cc}
            2 & \text{ si } a\neq 0 \\
            1 & \text{ si } a= 0
        \end{array}
        \right.
    \end{equation*}

    Por tanto, $\dim V_1 = 3 - rg(A-I) = \left\{
        \begin{array}{cc}
            1 & \text{ si } a\neq 0 \\
            2 & \text{ si } a= 0
        \end{array}
        \right.$
    \begin{table}[H]
        \centering
        \begin{tabular}{c|c|c}
            Valores Propios & Mult. Alg. & Mult. Geom. \\ \hline 
            $-1$ & 1 & 1\\
            1 & 2 & $\left\{
        \begin{array}{cc}
            1 & \text{ si } a\neq 0 \\
            2 & \text{ si } a= 0
        \end{array}
        \right.$\\
        \end{tabular}
        \caption{Valores propios con sus multiplicidades}
    \end{table}

    Por tanto, se puede ver que $f$ solo es diagonalizable si $a=0$.
\end{ejercicio}

\begin{ejercicio}
    Sea $A\in \mathcal{M}_n(\bb{K})$. Demostrar que $\exists p(\lambda)\in \bb{K}[\lambda] \mid p(A)=A^{-1}$.
    \begin{proof}
        Usando el Teorema de Cayley-Hamilton:
        \begin{equation*}
            P_A(A) = 0 = (-1)^n A^n + (-1)^{n-1}tr(A)A^{n-1} + \dots + c_1A + det(A)I
        \end{equation*}
    
        Como $A$ es regular, $\exists A^{-1}$. Multiplicando por $A^{-1}$,
        \begin{equation*}
            A^{-1}P_A(A) = 0 = (-1)^n A^{n-1} + (-1)^{n-1}tr(A)A^{n-2} + \dots + c_2A + c_1I + det(A)A^{-1}
        \end{equation*}
    
        Despejando $A^{-1}$,
        \begin{multline*}
            -det(A)A^{-1} = (-1)^n A^{n-1} + (-1)^{n-1}tr(A)A^{n-2} + \dots + c_2A + c_1I
            \Longrightarrow \\ \Longrightarrow
            A^{-1} = -\frac{(-1)^n A^{n-1} + (-1)^{n-1}tr(A)A^{n-2} + \dots + c_2A + c_1I}{det(A)}
        \end{multline*}
        donde he podido despejarla ya que $det(A)\neq 0$ por ser regular. Por tanto, he despejado el polinomio $A^{-1}$ en función de $A$, teniendo así el polinomio buscado.
    \end{proof}
\end{ejercicio}

\begin{ejercicio}
    Encontrar una matriz $A\in \mathcal{M}_n(\bb{R})$, siendo $n$ par, tal que no tenga ningún valor propio.\\

    La matriz $A_1=\left(\begin{array}{cc}
        0 & -1 \\
        1 & 0
    \end{array} \right)$ no tiene valores propios reales.

    La matriz $A_2=\left(\begin{array}{c|c}
        A_1 & 0 \\ \hline
        0 & A_1
    \end{array} \right)$ tampoco tiene valores propios reales, ya que su polinomio característico es $P_{A_2}(\lambda) = P_{A_1}(\lambda)P_{A_1}(\lambda)$, que tampoco tiene raíces reales. Por tanto, la solución es:
    \begin{equation*}
        A = \left(\begin{array}{c|c|c}
            A_1 & 0 & 0 \\ \hline
            0  & \ddots & 0 \\ \hline
            0 & 0 & A_1
        \end{array}\right)
    \end{equation*}
    ya que $P_{A}(\lambda) = P_{A_1}(\lambda)\dots P_{A_1}(\lambda)$.
\end{ejercicio}

\begin{comment}
\begin{ejercicio}
    Sea $V^3(\bb{R})$ un espacio vectorial. Denotemos por $\mathcal{B}=\{u_1, u_2, u_3\}$ una base de $V$. Sea $f\in End(V)$ del que se sabe:
    \begin{itemize}
        \item $f$ transforma el vector $6u_1 + 2u_2 + 5u_3$ en sí mismo.

        \item $U=\{(x_1, x_2, x_3)_\mathcal{B} \mid 2x_1 + 11x_2 -7x_3 = 0\}$ es un subespacio propio de $f$.

        \item La traza de $f$ es $5$.
    \end{itemize}
    Hallar los valores propios de $f$ y calcular $M(f; \mathcal{B})$.\\

    Como $f((6,2,5)_\mathcal{B}) = (6,2,5)_\mathcal{B} \Longrightarrow \lambda_0=1$ es un valor propio de $f$.

    \begin{equation*}
        U = \mathcal{L}\left(\left\{
        \left(\begin{array}{ccc}
            11 \\ -2 \\ 0
        \end{array} \right),
        \left(\begin{array}{ccc}
            0 \\ 7 \\ 11
        \end{array} \right)
        \right\}\right) \Longrightarrow
        \left\{ \begin{array}{c}
             f((11,-2,0)_\mathcal{B}) = \lambda_0(11,-2,0)_\mathcal{B} \\
             f((0,7,11)_\mathcal{B}) = \lambda_0(0,7,11)_\mathcal{B}
        \end{array}\right.
    \end{equation*}

    Por tanto, las ecuaciones quedan:
    \begin{equation*}
        \left\{\begin{array}{c}
            6f(u_1) + 2f(u_2) + 5f(u_3) = 6u_1 + 2u_2 + 5u_3 \\
            11f(u_1) - 2f(u_2) = \lambda_0(11u_1 - 2u_2)\\
            7f(u_2) + 11f(u_3) = \lambda_0(7u_2 + 11u_3)
        \end{array} \right.
    \end{equation*}
    
    \vspace{2cm} TERMINAR \vspace{2cm}
\end{ejercicio}
\end{comment}

\begin{ejercicio}
    Sea $A,C \in \mathcal{M}_2(\bb{\bb{K}})$ tal que $A=AC-CA$. Demostrar que $A^2=0_2$.\\

    Veamos en primer lugar que $tr(A)=0$:
    $$tr(A) = tr(AC-CA) = tr(AC) - tr(CA) = 0$$
    
    Por el Teorema de Cayley-Hamilton:
    $$P_A(A) = 0 = A^2 -\cancel{tr(A)A} + det(A)I = A^2 + det(A)I \Longrightarrow A^2 = -det(A)I$$

    Además, tenemos:
    \begin{gather}
        \label{Ej41_Ec1} A^2 = A(AC-CA) = A^2C - ACA \\
        \label{Ej41_Ec2} A^2 = (AC-CA)A = ACA - CA^2
    \end{gather}

    Sumando las ecuaciones \ref{Ej41_Ec1} y \ref{Ej41_Ec2},
    $$2A^2 = A^2C- CA^2 = -det(A)IC +Cdet(A)I = 0 \Longrightarrow A^2 = 0_2$$
\end{ejercicio}

\begin{ejercicio}
    Sea $A\in \mathcal{M}_2(\bb{R}) \mid A^3=0_2$. Demostrar que $A^2=0$.\\

    Por el Teorema de Cayley-Hamilton,
    \begin{equation*}
        P_A(A) = 0 = A^2 - tr(A)A + det(A)I \Longrightarrow \cancel{A^3} - tr(A)A^2 + det(A)A = 0 \Longrightarrow A^2 = A\frac{det(A)}{tr(A)}
    \end{equation*}

    Veamos el valor de $det(A):$
    \begin{equation*}
        0 = |A^3| = |A|^3 \Longrightarrow |A|=0
    \end{equation*}

    Por tanto, $A^2 = A\cdot 0 = 0$.
\end{ejercicio}


\begin{ejercicio}
    Sean $A,P,Q \in \mathcal{M}_2(\bb{R})$ tal que
    $$P^{-1}AP = Q^{-1}AQ = \left( \begin{array}{cc}
        2 & 0 \\
        0 & \sqrt{3}
    \end{array} \right)$$
    Demostrar que entonces las columnas de $P$ son proporcionales a las columnas de $Q$.
    \begin{comment}
    \begin{proof}
        Sea $X=P^{-1}AP$, $Y=Q^{-1}AQ$ y $D=\left( \begin{array}{cc}
            2 & 0 \\
            0 & \sqrt{3}
        \end{array} \right) = X=Y$
        
        Usando el isomorfismo entre endomorfismos y matrices cuadradas, sea $f\in~End(V^2)$ tal que $M(f, \mathcal{B}) = A$. Como $A\sim X$ y $A\sim  Y$, las tres matrices representan al mismo endomorfismo pero en distintas bases.
        $$X=M(f; \mathcal{B}_1) \qquad Y=M(f; \mathcal{B}_2) \qquad A=M(f; \mathcal{B}_3)$$

        Debido a la matriz $D$, sabemos que $V_2 \oplus V_{\sqrt{3}} = V^2(\bb{R})$. Los valores propios del endomorfismo con multiplicidad algebraica y geométrica 1 son: $\{2,  \sqrt{3}\}$. Como $X=Y=D$, sabemos que $\mathcal{B}_1$ y $\mathcal{B}_2$ son bases de vectores propios.

        Como la multiplicidad geométrica es $1$, sea $\{e_1\}$ base de $V_2$ y $\{e_2\}$ base de $V_{\sqrt{3}}$. Debido a la suma directa, sea $\mathcal{B}_1 =~\{e_1, e_2\}$.
        
        \vspace{5cm}
        Sea ahora $\mathcal{B}_2 = \{\bar{e_1}, \bar{e_1}\}$, con $\bar{e_1} \in V_2,\;\bar{e_1}\in V_{\sqrt{3}}$.
        \begin{itemize}
            \item Como $\bar{e_1} \in V_2 \Longrightarrow \bar{e_1} = ae_1$, con $a\neq 0$.
            \item Como $\bar{e_2} \in V_{\sqrt{3}} \Longrightarrow \bar{e_2} = be_2$, con $b\neq 0$.
        \end{itemize}
        Por tanto, los vectores de $\mathcal{B}_2$ son proporcionales a los vectores de $\mathcal{B}_1$. Veamos ahora que las columnas de $P$ y $Q$ también son proporcionales:
        \begin{itemize}
            \item $P=M(\mathcal{B}_1; \mathcal{B}_3) \Longrightarrow P$ tiene como columnas los vectores de $\mathcal{B}_1$ en la base $\mathcal{B}_3$.

            \item $Q=M(\mathcal{B}_2; \mathcal{B}_3) \Longrightarrow Q$ tiene como columnas los vectores de $\mathcal{B}_2$  en la base $\mathcal{B}_3$.
        \end{itemize}
        
        Por tanto, como los vectores de $\mathcal{B}_2$ son proporcionales a los vectores de $\mathcal{B}_1$, las columnas de $P$ también son proporcionales a las columnas de $Q$.
    \end{proof}
    \end{comment}

\end{ejercicio}
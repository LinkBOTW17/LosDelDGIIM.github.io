\chapter{Introducción a las bases de datos.}

\begin{definicion}[Directorio]
Archivo especial que permite agrupar archivos según las necesidades de los usuarios.
\end{definicion}

\begin{definicion}[Archivo] Un archivo es un conjunto de información respecto de un mismo tema. Está compuesto de registros homogéneos que contienen información sobre el tema.
\end{definicion}

\begin{definicion}[Registro] Estructura dentro del archivo que contiene la información correspondiente a un elemento individual. Se divide en campos.
\end{definicion}

\begin{definicion}[Campo] Dato que representa una información unitaria o independiente.
\end{definicion}

\subsection*{Clasificación de archivos según el tipo de registros}
\begin{enumerate}
    \item \underline{Longitud fija.}
    \item \underline{Longitud variable.}
    \begin{itemize}
        \item Con delimitador. Estos suelen ser caracteres especiales como el salto de línea.
        \item Con cabecera: cada registro contiene un campo inicial que almacena el número de bytes del registro.
    \end{itemize}
    \item \underline{Longitud indefinida.}
    
    El sistema operativo no realiza ninguna gestión sobre la longitud de los registros ya que el archivo no tiene realmente ninguna estructura interna.

    En cada operación de lectura/escritura se transfiere una determinada subcadena del archivo y será el programa de usuario quien indique al SO el principio y final de ese registro.
\end{enumerate}

\section{Organización de Archivos}

Existen diversas formas de organizar los archivos:
\subsection{Organización secuencial}
Los registros están almacenados físicamente de forma contigua.

Tiene como ventajas que es sencilla de usar y usa el espacio de una forma óptima. Además, permite leer los registros de forma secuencial de forma óptima.
\begin{itemize}
    \item \underline{Añadir registros}: Solo se puede añadir al final.
    \item \underline{Consultar registros}: Para consultar el registro $n$, es necesario leer los $n-1$ anteriores.
    \item \underline{Modificación de registros}: Solo es posible si no se aumenta su longitud
    \item \underline{Eliminación de registros}: No es posible eliminar un registro del archivo, pero se puede realizar un borrado lógico; es decir, marcarlo de tal forma que al leer se identifique como no válido.
\end{itemize}

\subsection{Organización secuencial encadenada}

También llamado ``celdas enlazadas''. Junto a cada registro se almacena un puntero con la dirección física del registro siguiente, dando lugar a una cadena de registros.

El último registro de la cadena contiene una marca especial en el lugar del puntero indicando que ya no hay más registros en el archivo.

Permite insertar y borrar registros con facilidad, pero las consultas solo pueden realizarse de forma secuencial.
\begin{itemize}
    \item \underline{Añadir registros}: Tras localizar la posición en la que se desea intertar, se siguien los siguientes pasos:
    \begin{enumerate}
        \item Escribir el registro en una zona libre de memoria.
        \item Modificar la dirección a la que apunta el registro anterior al creado.
        \item Modificar la dirección a la que apunta el creado al siguiente.
    \end{enumerate}
    
    \item \underline{Consultar registros}: Es secuencial. Al leer cada registro se lee la posición del siguiente.
    
    \item \underline{Modificación de registros}: Si el cambio no implica un aumento de longitud del registro, éste puede reescribirse en el mismo espacio.
    
    En caso contrario, se debe insertar el registro y luego borrar la versión anterior al cambio.
    
    \item \underline{Eliminación de registros}: Se asigna al puntero del registro anterior la dirección del registro siguiente.
\end{itemize}


\subsection{Organización secuencial indexada}

Está formado por dos estructuras: zona de registros y zona de índices.
\begin{itemize}
    \item \underline{Zona de registros}:

    Zona donde se encuentras los registros del archivo de forma secuencial. Está dividida en tramos, y es necesario que los registros estén ordenados según el valor de una llave.

    \item \underline{Zona de índices}: Zona en la que por cada tramo de la zona de registros hay un registro que contiene:
    \begin{itemize}
        \item La dirección del primer registro del tramo.
        \item La llave del último registro del tramo. Al estar los registros ordenados, esta es la mayor llave del tramo.
    \end{itemize}
\end{itemize}
La gestión de la estructura la realiza el sistema operativo o un software especial, por lo que el usuario de esta estructura no necesita conocer la existencia de ambas zonas, pudiendo contemplar ambas como un todo.

Las operaciones sobre archivos se describen a continuación:
\begin{itemize}
    \item \underline{Consulta de registros}: Usando la zona de índices se localiza el tramo, y posteriormente se realiza la búsqueda secuencial en dicho tramo.
    \item \underline{Inserción, borrado y modificación}: no son posibles.
\end{itemize}


\subsection{Organización directa o aleatoria}
Es un archivo para el cual existe una función de transformación que genera la dirección de cada registro a partir de un campo que se usa como llave. El nombre de ``aleatorio'' se debe a que normalmente no existe ninguna vinculación aparente entre el orden lógico de los registros y su orden físico. Este tipo de organización es útil para archivos donde los accesos deben realizarse por llave.

Las distintas \underline{funciones de transformación} o \underline{métodos de direccionamiento} son:
\begin{enumerate}
    \item \underline{Direccionamiento directo} Solo es factible cuando la llave es numérica y su rango de valores no es mayor que el rango de direcciones en el archivo.

    La dirección es la propia llave.

    Como inconveniente, se puede dar el caso de direcciones sin usar.

    \item \underline{Direccionamiento asociado} Debe construirse una tabla en la que se almacena para cada llave la dirección donde se encuentra el registro correspondiente. Dicha tabla se debe guardar mientras exista el archivo.

    \item \underline{Direccionamiento calculado o \textit{``hashing''}}:
    La dirección de cada registro se obtiene realizando una transformación sobre la llave. Se usa cuando:
    \begin{itemize}
        \item La llave no es numérica. En ese caso, se necesita una conversión previa para obtener un número a partir de ella. Por ejemplo se usa el equivalente decimal al código ASCII del carácter, y ya con dicho código se opera.

        \item La llave es numérica pero toma valores en un rango inadecuado para usarse directamente como dirección.
    \end{itemize}
\end{enumerate}


Es fundamental elegir adecuadamente la {función de transformación} o {método de} {direccionamiento}, ya que se pueden dar los siguientes casos:
\begin{enumerate}
    \item Que haya direcciones que no se corresponden con ninguna llave y, por tanto, habrá zonas de disco sin utilizar.

    \item Que haya direcciones que se correspondan con más de una llave. En este caso se dice que las llaves son \textbf{sinónimas} para esa transformación o que se produce una \textbf{colisión}.
\end{enumerate}

El problema de los sinónimos se resuelve de la siguiente forma:
\begin{enumerate}
    \item Se reserva una zona de desbordamiento gestionada de otra forma (secuencial) donde se guardan los registros cuya dirección ya está ocupada por una clave sinónima.

    \item Los registros cuya dirección ya está ocupada por una clave sinónima se guardan en la próxima dirección de memoria libre.
\end{enumerate}

Las operaciones sobre archivos se describen a continuación:
\begin{itemize}
    \item \underline{Consulta de registros}: Por llave, aplicándole la transformación correspondiente. En el caso de que no se encuentre en su dirección, se procede según se haya resuelto el problema de sinónimos.
    
    \item \underline{Borrado}: Ha de ser lógico.
    
    \item \underline{Inserción y modificación}: Es posible, realizando la transformación de la llave correspondiente.
\end{itemize}



\section{Bases de datos}

Los problemas asociados a las gestiones de archivos mencionadas son:
\begin{enumerate}
    \item \underline{Dificultad de mantenimiento.} Si hay dos archivos con información duplicada, si se modifica uno ha de moficarse el otro. En caso contrario, habría información incoherente.
    \item \underline{Redundancia}. Hay redundancia si un dato se puede deducir a partir de otros datos, y estaríamos en el caso anterior.
    \item \underline{Rigidez de búsqueda}. La búsqueda ha de realizarse de una forma concreta, sin poder combinarse directa y secuencial.
    \item \underline{Dependencia con los programas} Si un programa lee dichos archivos y se modifican los archivos, ha de modificarse también el programa.
    \item \underline{Seguridad}.
\end{enumerate}

Para resolver dichos problemas, surgen las bases de datos.
\begin{definicion}[Base de Datos]
    Sistema formado por un conjunto de datos y un paquete software para gestión de dicho conjunto de datos de tal modo que:
    \begin{enumerate}
        \item Se controla el almacenamiento de datos redundantes.
        \item Los datos resultan independientes de los programas que los usan.
        \item Las relaciones entre los datos se almacenan junto a ellos.
        \item Se puede acceder a los datos de diversas formas.
    \end{enumerate}

    En una base de datos se almacenan además de las entidades, las relaciones existentes entre ellas.
\end{definicion}

\begin{definicion}[Clave primaria]
    Se dice que uno o más atributos de una entidad es un identificador o clave primaria si el valor de dichos atributos determina de forma unívoca cada uno de los elementos de dicha entidad, y no existe ningún subconjunto de él que permita identificar a la entidad de manera única.
\end{definicion}
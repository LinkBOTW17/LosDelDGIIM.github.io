\chapter{Generación y Depuración de Aplicaciones}
\section{Plataformas}

\begin{definicion}
    Combinación de hardware y/o software para ejecutar aplicaciones software.
    
    Ejemplos de plataformas son los sistemas operativos (Windows, Mac, etc.) o los navegadores.
\end{definicion}

La clasificación del software se realiza según la plataforma en la que se puede ejecutar.
\begin{enumerate}
    \item \underline{Dependiente de la plataforma particular} para la cual se desarrolla y ejecuta (bien sea una plataforma hardware, sistema operativo o máquina virtual).

    \item \underline{Multiplataforma}, cuando el software se ha desarrollado y opera en varias plataformas.

    El software multiplataforma se puede clasificar en dos tipos:
    \begin{enumerate}
        \item Aplicaciones que requieren su creación o compilación para cada plataforma específica donde se ejecutarán.

        \item Aplicaciones que directamente se pueden ejecutar en más de una plataforma sin preparación especial.
        
        Ejemplo de esto son los programas en Java o aplicaciones escritas en lenguaje interpretado solo usando los paquetes estándar.
    \end{enumerate}
\end{enumerate}


\subsection{Plataforma JAVA}

Tiene un alto nivel de abstracción, ya que incluso incluye un lenguaje de programación. Requiere la instalación de JVM \textit{(Java Virtual Machine)} en cada SO. Por tanto, los programas Java son multiplataforma, pero no la JVM (hay una para cada sistema operativo).

Los programas no se ejecutan directamente en el sistema operativo, sino en la JVM. No obstante, dispone de la \textit{Java Native Interface (JNI)} para cceder a funciones específicadas de cada SO.


Hay un compilador \textit{JIT (Just In Time)} que traduce instrucciones en Java a instrucciones nativas del procesador. Esto permite que las aplicaciones JAVA se ejecuten de forma rápida.


\subsection{Plataforma Android}

La arquitectura de Android está formada por distintas capas, todas ellas basadas en software libre. Cada una de las capas utiliza elementos de la capa inferior para realizar sus funciones, concepto que se conoce como pila de sofware. Estas capas, de más inferior a superior, son:
\begin{enumerate}
    \item \underline{Kernel de Linux}: Administración de memoria de bajo nivel, gestión de subprocesos, etc.
    
    \item \underline{Capa de abstracción de hardware} \textit{(HAL, Hardware Abstraction Layer)}: Módulos de biblioteca para un tipo específico de componente de hardware, como dispositivos Bluetooth.

    \item Tiempo de ejecución de Android \textit{(ART, Android Runtime)}: Ejecuta varias máquinas virtuales que optimizan los procesos.

    \item \underline{Bibliotecas C/C++ nativas.} Muchos componentes y servicios centrales del sistema Android, como el ART y la HAL, se basan en código que requiere bibliotecas nativas escritas en C/C++.

    \item \underline{Marco de trabajo de la API de Java}. El conjunto de funciones del SO Android está disponible mediante API escritas en Java.

    \item \underline{Apps del sistema}: Apps centrales como un navegador, email, calendario, etc.
\end{enumerate}


\section{Framework de Desarrollo de Aplicaciones}
\subsection{Herramientas básicas para el desarrollo software en Linux}

Linux tiene herramientas de ayuda en cada fase del desarrollo de las aplicaciones:
\begin{enumerate}
    \item \underline{Generador de código fuente}: Editores de texto como \verb|gedit|, \verb|nano|, etc.

    \item \underline{Sangrado código fuente}: \verb|indent| sangra un programa en C sintácticamente correcto.

    \item \underline{Compilación código fuente}: Tiene compiladores como \verb|g++| o \verb|gcc|.

    \item \underline{Gestión de software basado en módulos}: Para ello está la utilidad \verb|make|.

    \item \underline{Gestión de bibliotecas}
    
    \item \underline{Control de versiones}
\end{enumerate}

\begin{definicion}[IDE]

Un entorno integrado de diseño (IDE, \textit{Integrated Development Environment}) es aplicación que proporciona un conjunto de herramientas relacionadas para el desarrollo del software. Permite:
\begin{itemize}
    \item Crear el código
    \item Gestionar el ciclo de vida. Compilar, enlazar, etc.
    \item Despliegue (instalación) y depuración del código.
\end{itemize}

La comodidad que aportan contrasta con el uso de las herramientas aisladas en Linux.
\end{definicion}
Algunas características de los IDE a destacar son:
\begin{itemize}
    \item Maximizar la productividad, ya que incluye diversas herramientas.

    \item Acelerar el aprendizaje de lenguajes de programación; ya que por ejemplo el código se puede analizar mientras se edita para conocer de forma inmediata errores léxicos, sintácticos, etc.
\end{itemize}


\begin{definicion}[Programación Visual]
    Hace uso de IDEs que permiten a los programadores crear nuevas aplicaciones combinando bloques de código con diagramas de flujo, normalmente basados en UML.
\end{definicion}

\subsection{Framework de desarrollo software}
\begin{definicion}[Framework de desarrollo software] Abstracción usada para crear aplicaciones que proporciona software con funcionalidad genérica que el usuario puede modificar mediante código.
\end{definicion}

Por ejemplo, algunos de los software genéricos que pueden incluir son la gestión de la base de datos, la autenticación de usuarios o la generación de interfaces de usuario, entre otras.

Tienen como objetivo facilitar y reducir el tiempo el desarrollo evitando dedicar tiempo a detalles de bajo nivel, pero debido a la complejidad de las APIs conllevan un tiempo extra de aprendizaje inicialmente.



\section{Técnicas de Depuración de Programas}

\begin{definicion}[Depuración]
    Actividad consistente en encontrar y solucionar errores cometidos en el diseño y codificación de programas y solucionarlos.
\end{definicion}

Los objetivos de la depuración son:
\begin{itemize}
    \item Incrementar la productividad con una detección de errores más rápida.
    \item Mejorar la calidad de los programas.
    \item Prevenir errores.
    \item Mejorar lenguajes de programación y herramientas.
\end{itemize}


La depuración consiste en la ejecución de programas como máquinas de estados. Cada \textbf{estado} viene determinado por el valor de las variables y la posición de ejecución, por lo que cada estado depende de los anteriores.

Las fases de la generación del fallo son:
\begin{enumerate}
    \item \underline{Creación de un defecto}: Pieza de código creada por el programador que \textit{puede} causar infección como consecuencia de un error.

    \item \underline{El defecto produce infección}: Los estados alcanzados difieren de los previstos.

    \item \underline{Propagación de la infección}

    \item \underline{La infección causa el fallo}: Error observable externamente en el comportamiento de un programa que detecta el programador.
\end{enumerate}

De esto se deduce que la ausencia de fallos nunca asegura la ausencia de defectos.

\begin{definicion}[Cadena de infección] Todas las instrucciones que se han ejecutado desde que se produjo el defecto hasta que se ha detectado el fallo. Al depurar, buscamos retroceder en la cadena de infección hasta localizar el fallo.
\end{definicion}


Las fases de la depuración son:
\begin{enumerate}
    \item Registrar el problema.
    \item Reproducir el fallo (the failure): más complicado para programas no-deterministas y de larga ejecución.
    \item Automatizar y simplificar el caso de prueba.
    \item Encontrar posibles orígenes de la infección.
    \item Centrar la búsqueda en los orígenes más probables.
    \item Aislar la cadena de infección.
    \item Corregir el defecto: sencillo si está claro el defecto que produce el fallo.
\end{enumerate}


Los errores más comunes son escribir código desestructurado y programar sin pensar. Las características del código estructurado son:
\begin{itemize}
    \item Uso de funciones.
    \item Uso de construcciones iterativas (\verb|while|, \verb|for|) en vez de \verb|goto|.
    \item Uso de variables con nombre significativo.
    \item Uso de tipos de dato abstracto \textit{(ADTs, Abstract data types)} o clases y objetos.
\end{itemize}


\begin{definicion}[Depurador]
    Herramienta software que ayuda a ejecución controlada de un programa para ayudar a encontrar el defectos que producen un fallos.
\end{definicion}

Las características de los depuradores son:
   \begin{itemize}
       \item Permitir interacción con el programador.
       \item Proporciona un conjunto de instrucciones propias.
       \item Permite detectar errores en tres categorías:
       \begin{itemize}
           \item \underline{Sintácticos}. Detectados durante la compilación o enlazado.
           \item \underline{Ejecución}. Errores que provocan el fin del programa. Dividir entre 0, acceder fuera del espacio de memoria reservado (\textit{segmentation fault}), etc.

           \item \underline{Lógicos}. Errores que debe detectar el programador, que no generan fallos en ejecución pero que muestran que el algoritmo no está bien implementado.
       \end{itemize}
   \end{itemize}
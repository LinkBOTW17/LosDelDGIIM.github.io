\chapter{Sistemas de Cómputo}
\begin{definicion}[Firmware]
    El firmware de un sistema de cómputo es la lógica de más bajo nivel (controla los circuitos electrónicos de un dispositivo $\longrightarrow$ hardware de más bajo nivel) $\longrightarrow$ conjunto de instrucciones máquina grabadas en una memoria de solo lectura (microprogramas/microinstrucciones).
\end{definicion}
\begin{definicion}[BIOS]
    La BIOS es el firmware de una computadora para activarla desde su encendido y preparar el entorno para cargar el S.O. en la memoria RAM y en el disco duro.    
\end{definicion}

\begin{definicion}[Sistema Operativo]
    Es una capa intermedia entre las ``utilidades y herramientas'' y el hardware para que:
    \begin{itemize}
        \item Los usuarios finales no se tengan que preocupar del hardware y vean el ordenador como un conjunto de aplicaciones.
        \item Los programadores desarrollen programas de aplicación utilizando utilidades y otros programas del sistema. Tienen la ayuda del SO: editores, enlazadores, compiladores, depuradores...
        \item Acceder a los dispositivos de E/S de forma transparente a las instrucciones específicas que utiliza cada dispositivo.
        \item Acceder a los ficheros de forma controlada.
        \item Detectar distintos tipos de errores durante la ejecución.
        \item Acceder a recursos del sistema sin conflictos.
        \item Monitorizar el sistema y proporcionar estadísticas de uso.
    \end{itemize}
    En definitiva, facilitar el uso del ordenador de una forma eficiente.
\end{definicion}

\begin{definicion} [Interrupciones]
    Señal recibida por el procesador que indica que debe ``interrumpir'' la ejecución del código actual y pasar a ejecutar un código específico para tratar esa situación (suspensión temporal de procesos).
    \begin{itemize}
         \item \underline{Instrucciones por hardware} Son las habituales de las operaciones de E/S. No son producidas por instrucciones del proceso, sino por señales del dispositivo de E/S para indicar al procesador que necesita ser atendido.

         \item \underline{Interrucpiones por software o \emph{trap}} Son lo mismo que las llamadas al sistema generadas por un programa durante su ejecución. (Siempre que hay una instrucción que requiere del SO para su ejecución, por ejemplo, acceso al disco duro, donde se necesita ejecutar una rutina del SO)
    \end{itemize}
\end{definicion}

\begin{definicion} [Excepciones] Interrupciones síncronas causadas típicamente por un error en un programa (división por 0, acceso a direcciones de memoria protegidas,~ ...). En esos casos, se produce una ejecución del SO para garantizar la integridad de los datos, intentar solucionar el error o al menos informar de ello al usuario.
    
\end{definicion}
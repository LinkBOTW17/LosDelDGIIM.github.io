\chapter{Introducción a los Sistemas Operativos}
\begin{definicion}[Procesamiento en serie]
    Se dio a finales de los 40 y a principios de los 50. Se caracteriza por:
    \begin{itemize}
        \item Tarjetas perforadas
        \item No había SO: se cargaba el programa en código máquina utilizando las tarjetas
        \item Los errores se examinaban a mano $\longrightarrow$ reiniciar desde el principio.
        \item Los usuarios pedían cita
    \end{itemize}
\end{definicion}

\begin{definicion}[Sistemas por lotes (Sistemas Batch)]
    Se caracteriza por la existencia de un software específico llamado \emph{monitor}.
    \begin{itemize}
        \item Agrupación de trabajos similares (evitar los problemas del procesamiento en serie)
        \item Parte del monitor estaba cargado en memoria (monitor residente) y otras utilizadas se cargaban como subrutinas del programa del usuario.
    \end{itemize}
    El proceso seguido era:
    \begin{enumerate}
        \item  Monitor lee un trabajo del core.
        \item El trabajo se carga en memoria.
        \item El procesador ejecuta el trabajo.
        \item Vuelta al 1).
    \end{enumerate}
    Los problemas eran:
    \begin{itemize}
        \item El procesador estaba ocioso
        \item Recursos infravalorados.
    \end{itemize}
\end{definicion}

\begin{definicion}[Multiprogramación]
    Cuando un proceso se pone en espera, en lugar de estar el procesador inactivo esperando, el SO cambie a otro proceso y lo ejecute (y así sucesivamente). Implica:
    \begin{itemize}
        \item Tener cargado en memoria todos los procesos
        \item Tener una planificación de la CPU (decidir entre varios procesos activos)
    \end{itemize}
\end{definicion}

\begin{definicion}[SO de Tiempo Compartido]
    SO multiprogramado donde se establecen turnos de CPU y el SO va decidiendo que procesos va a ejecutarse en el siguiente ciclo. Logra eficiencia a la hora de ejecutar todos los procesos en el menor tiempo posible.
    
    Un ejemplo de tiempo compartido es el DMA, ya que mediante el uso de interrupciones, el controlador del dispositivo de E/S se encarga de la operación de E/S mientras el procesador puede estar encargándose de otros procesos (el controlador va activando interrupciones).
\end{definicion}

\begin{definicion} [Dispatcher]
    Es un módulo del SO que se encarga de intercambiar procesos.
    \begin{table}[H]
        \centering
        \begin{tabular}{|c|c}
        \cline{1-1}
         \hspace{2cm} & Proceso A            \\ \cline{1-1}
         \hspace{2cm} & \textbf{Dispatcher}  \\ \cline{1-1}
         \hspace{2cm} & Proceso B            \\ \cline{1-1}
         \hspace{2cm} & \textbf{Dispatcher}  \\ \cline{1-1}
         \hspace{2cm} & Proceso A            \\ \cline{1-1}
        \end{tabular}
    \end{table}
\end{definicion}

\begin{definicion}[Proceso]
    Está formado por un programa ejecutable y datos que necesita el SO para ejecutar el programa. Está relacionado con:
    \begin{itemize}
        \item La operación en lotes en sistema multiprogramados (interrupciones).

        \item Tiempo compartido (múltiples procesos ejecutándose utilizando pequeños intervalos de tiempo)

        \item Sistemas de transacciones en tiempo real
        \begin{itemize}
            \item Proceso $\longrightarrow$ Uso de transacciones (los procesos se suspenden a través de las por interrupciones, se almacena su estado y el procesador pasa a ejecutar otro proceso) 
        \end{itemize}
    \end{itemize}

    Pueden producirse errores relacionados con:
    \begin{itemize}
        \item Sincronizaciones inadecuadas (se interrumpe un proceso y no se restaure adecuadamente).

        \item Violaciones de exclusión mutua (hay recursos del sistema que sólo pueden utilizarse por un único proceso simultáneamente).

        \item Operaciones no deterministas de los programas (protección del espacio de memoria de los procesos).

        \item Interbloqueos (que existan varios programas que se estén esperando entre sí debido al uso que se está haciendo de los recursos).
    \end{itemize}

    Para evitar estos errores, los procesos constan de:
    \begin{enumerate}
        \item Un programa ejecutable.
        \item Datos necesarios que requiere el SO para ejecutarlo (incluyendo el contexto de la ejecución). Se denomina bloque de control de proceso (PCB) formado por:
        \begin{enumerate}
            \item Identificador de proceso
            \item Contexto de ejecución: contenido de los registros del procesador.
            \item Memoria donde reside el programa y sus datos.
            \item Información relacionada con recursos del sistema.
            \item Estado: en que situación se encuentra el proceso en cada momento (modelo de estados).
            \item Otra información.
        \end{enumerate}
    \end{enumerate}
\end{definicion}

\begin{definicion}[Traza de ejecución (de un proceso)]
    Es la secuencia de instrucciones que se ejecutan en cada ciclo para ese proceso.
\end{definicion}

\section{Modelo de dos estados}
Tiene el inconveniente de que es demasiado simple.

Un proceso puede estar en ejecución o no (los cambios de un estado a otro los realiza el dispatcher).

\section{Modelo de cinco estados}
    Cabe contemplar que los estados que NO están en ejecución pueden estar:
    \begin{description}
        \item[Nuevos] que todavía no han sido aceptados para ejecución. 
        \item[Preparados] para ejecutarse.
        \item[Bloqueados] (ej: porque están esperando a que termine una operación de E/S)
        \item[Terminados] debido a que se han extraídos del conjunto de procesos aceptados para ejecución.
    \end{description}

\section{Gestión de Procesos}
\begin{itemize}
    \item \underline{Lista de procesos}: posiciones de memoria en la que se almacena la dirección de memoria de inicio de cada proceso.

    \item Cada proceso es una estructura de datos para coordinar su ejecución.

    \item El PC almacena la dirección de la siguiente instrucción que debe ejecutarse del proceso que indica el registro ``índice de proceso'' (IP).

    \item El registro base contiene la dirección incial del proceso.

    \item El registro límite incluye el tamaño del proceso
\end{itemize}

Para crear un proceso, se ha de inicializar el PCB de la siguiente forma:
\begin{enumerate}
    \item Asignar un identificador único al proceso.
    \item Asignar un nuevo PCB.
    \item Asignar la memoria correspondiente.
    \item Inicializar PCB.
    \begin{itemize}
        \item PCB $\longrightarrow $ Dirección inicial de comienzo del programa.
        \item SP $\longrightarrow $ Dirección de la pila de sistema.
        \item Memoria donde reside el programa
        \item El resto de campos se inicializan a valores por defecto.
    \end{itemize}
\end{enumerate}


\section{Cambios de modo}
\begin{definicion} [Cambios de Modo] Cuando se produce el cambio de modo usuario a modo supervisor/kernel o viceversa.
\end{definicion}
Tras una interrupción, excepción o llamada al sistema; puede haber un cambio de modo de usuario a kernel.
\begin{enumerate}
    \item El hardware guarda automáticamente el PC y PSW como mínimo. El bit de modo cambia a kernel.
    \item Se determina la ruta del SO que debe ejecutarse y se almacena en el PC la dirección de comienzo.
    \item Se ejecuta la rutina. Suele empezar por guardar toda la información y terminar restaurándola.
    \item Volver a la rutina del SO previa. El hardware restaura el PC y PSW.
\end{enumerate}

\section{Cambios de Contexto}
\begin{definicion} [Cambio de contexto] Cuando se pasa la ejecución de un proceso a otro.
\end{definicion}
Se producen al cambiar de proceso, y son llevados a cabo por el Dispatcher. Los pasos a seguir, tras una interrupción, excepción o llamada al sistema, son:
\begin{enumerate}
    \item Guardar los registros en el PCB del proceso ejecutándose.
    \item Actualizar el campo del estado del proceso al nuevo estado e insertar el PCB en la cola.
    \item Seleccionar un nuevo proceso de los ``Preparados''. Se encarga el Planificador de la CPU o el \textit{Scheduler}.
    \item Actualizar el estado del nuevo proceso a ``Ejecutándose'', y sacarlo de ``Preparados''.
    \item Cargar los registros del procesador con el PCB del proceso.
\end{enumerate}




Hasta el momento, hemos visto que la tarea fundamental del SO es gestionar procesos:
\begin{itemize}
    \item Reservarles recursos
    \item Permitirles intercambiar información
    \item Garantizar uso recursos y protección
    \item Sincronizar recursos
\end{itemize}

\section{Hebras o Hilos}
\begin{definicion}[Hebra]
    Tiene como objetivo separar:
    \begin{itemize}
        \item El control de la asignación de los recursos (procesos)

        \item De la ejecución de los programas (división de los procesos en hilos que se pueden ejecutar de forma independiente)       
    \end{itemize}

    Para ello,
    \begin{itemize}
        \item Sigue habiendo un único bloque de control del proceso (PCB) y espacio de direcciones del usuario.

        \item Cada hilo tiene una pila y un bloque de control (TCB, \textit{Thread Control Block})  que contiene valores de los registros, prioridad, resto de información de estado, etc.
    \end{itemize}

    Por tanto, todos los hilos de un proceso comparten el estado y los recursos asociados al proceso. Por tanto, es necesario planificar la sincronización de las actividades de los hilos.
\end{definicion}

Con el concepto de hebra, pasamos a \emph{sistemas multihilo}, los cuales separan el punto de control de la asignación de recursos (procesos) y el de ejecución de los programas (cada proceso se divide en hebras, de forma que la ejecución de una hebra se produce de forma independiente al resto de hebras).

\begin{definicion} [Modelo Monohilo]
    Un único bloque de control (PCB), espacio de direcciones de usuario, pilas de usuario y kernel.
\end{definicion}

\begin{definicion}[Modelo Multihilo]
    Hay un único bloque de control del proceso (PCB) y espacio de direcciones del usuario. Cada hilo tiene su propio bloque de control (TCB), sus propias pilas de usuario y kernel.
\end{definicion}

\subsection{Modelo de 5 estados para hebras}
\begin{itemize}
    \item Cuando se crea un espacio, se crea un hilo para dicho proceso.

    \item Posteriormente los hilos pueden generar nuevos hilos para el mismo proceso.

    \item Cada nuevo hilo, se coloca en la cola de \textbf{preparados}.

    \item Cuando un hilo necesita esperar a un evento, se \textbf{bloquea}.

    \item Cuando se produce el evento en espera, el hilo pasa al estado \textbf{preparado}.

    \item Cuando termina la ejecución del hilo, se liberan los registros y pilas asociados.
\end{itemize}

\subsection{Ventajas de las Hebras}
\begin{itemize}
    \item Menor tiempo de creación de una hebra en un proceso ya creado que crear un nuevo proceso.

    \item Menor tiempo de finalización de una hebra que de un proceso.

    \item Menor tiempo de cambio de contexto entre hebras de un mismo proceso.

    \item Facilitan la comunicación entre hebras de un mismo proceso.
\end{itemize}

\section{Gestión de Memoria}
El concepto de multiprogramación se basa en compartir la memoria entre todos los procesos de forma eficiente y segura. Es segura, ya que cada proceso debe tener su espacio de memoria.

Normalmente los programadores desconocen a priori en qué posiciones de memoria se va a almacenar el programa durante su ejecución (direcciones absolutas).

Es conveniente poder intercambiar los procesos en la memoria principal para hacer más eficiente el uso del procesador $\longrightarrow$ trabajar con direcciones relativas $\longrightarrow$ hardware del procesador y del SO que se encarga de ir traduciendo de direcciones relativas a direcciones absolutas.

\subsection{Carga de procesos en memoria}
El primer paso que es necesario para generar un proceso consiste en cargarlo en memoria $\longrightarrow$~\textbf{cargador}.
\begin{enumerate}
    \item \underline{Carga absoluta}: Durante la compilación se añaden al código máquina direcciones absolutas de memoria. $\longrightarrow$ El programa no es reubicable (transparencia~34~$\longrightarrow$ X a N).

    \item \underline{Carga reubicable}: La carga del programa puede realizarse en cualquier espacio de la memoria principal. Para ello, el compilador genera direcciones lógicas relativas (de 0 a M).
    \begin{enumerate}
        \item \underline{Reubicación estática}: El cargador decide dónde se va a ubicar en memoria el programa y añade la dirección base a las relativas (compilador).
        \item \underline{Reubicación dinámica en tiempo real}: la traducción a direcciones absolutas se realiza en tiempo de ejecución $\longrightarrow$ es necesario un hardware del procesador.
    \end{enumerate}
\end{enumerate}

\subsection{Espacio de direcciones de memoria}
Pueden ser direcciones lógics o físicas. Diapositiva 37.
\begin{itemize}
    \item Las direcciones que genera un procesador son direcciones lógicas (relativas a la posición 0).
    \item Hay un hardware para traducir las direcciones lógicas a direcciones físicas. Es el MMU: \textit{memory management unit}.
    \item Mapa de memoria de un ordenador: espacio de memoria direccionable.
    \item Mapa de memoria de un proceso: Tamaño total de direcciones lógicas del proceso $\longrightarrow$ correspondencia con el tamaño de direcciones físicas.
\end{itemize}

\subsection{Fragmentación}
El problema de la fragmentación consiste en dividir el espacio de memoria de los procesos en fragmentos más pequeños que no se tengan que cargar necesariamente en posiciones de memorias contiguas.

Ha de realizarse adecuadamente para evitar la fragmentación tanto interna como externa:
\begin{itemize}
    \item \underline{Fragmentación Interna}:

    Espacio inutilizable dentro de un bloque de la memoria.
    \item \underline{Fragmentación Interna}:

    Espacio inutilizable fuera de un bloque de la memoria.
\end{itemize}

\begin{enumerate}
    \item \textbf{Paginación}
    \begin{itemize}
        \item El espacio lógico de direcciones de un proceso se divide en elementos del mismo tamaño (páginas).
        \item El espacio físico en memoria se divide en fragmentos del mismo tamaño (marcos de páginas).
        \item Tamaño de página $=$ Tamaño de marco de página $=2^n$
        \item Las direcciones lógicas son $(p,d)$, mientras que las direcciones físicas son~$(m,d)$ donde $p=$número de página, $m=$número de marco, $d=$~desplazamiento.
    \end{itemize}

    La tabla de páginas contiene la información necesaria para traducir cada dirección lógica a física:
    \begin{itemize}
        \item Hay una por proceso.
        \item Tiene una entrada por página del proceso.
        \item Tiene los siguientes elementos:
        \begin{itemize}
            \item Número de marco asociado a la página
            \item Bits de protección (modo de acceso autorizado)
            \item Bits adicionales, de haberlos.
        \end{itemize}
    \end{itemize}

    La ejecución de cada instrucción requiere por tanto dos accesos a memoria. En primer lugar, para la traducción, se accede a la tabla de páginas; y en segunda lugar ya se accede a memoria.
    \begin{itemize}
        \item El \textbf{RBTP} \textit{(registro base de la tabla de procesos)} está en el PCB del proceso y apunta a la tabla de páginas.

        \item El búfer de traducción adelantada, \textbf{TLB} \textit{(translation look-aside buffer)} resuelve el problema de acceder dos veces a memoria. Es una caché hardware.
    \end{itemize}
    
    \item \textbf{Segmentación}

    Similar a la paginación, solo que en este caso cada página no mide lo mismo.

    La tabla de segmentos contiene:
    \begin{itemize}
        \item Número de segmento
        \item Dirección base, en la que empieza el segmento.
        \item Longitud del segmento; es decir, desplazamiento máximo.
        \item Bits de protección:
        \begin{itemize}
            \item Bit de presencia: activado si la página se encuentra en memoria.
            \item Bit de modificado: el contenido de la página se ha modificado desde que se cargó en memoria.
            \item Bits de protección $RWX$
            \item Bits de accedido
        \end{itemize}
    \end{itemize}

    La tabla de segmentos de encuentra en la memoria principal. Adicionalmente, tenemos:
    \begin{itemize}
        \item El \textbf{RBTS} \textit{(registro base de la tabla de segmentos)} está en el PCB del proceso y apunta a la tabla de segmentos.

        \item El \textit{registro Longitud de la Tabla de Segmentos} \textbf{(STLR)}  indica el número de segmentos del proceso. Sirve para comprobar si un segmento $s$ es válido. Se almacena en el PCB.
    \end{itemize}
    
\end{enumerate}
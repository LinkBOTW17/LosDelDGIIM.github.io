\section{Definición de un monitor}
El concepto de semáforo se desarrolló previamente en el Seminario 1 de prácticas\footnote{Por lo que el lector debería estar familiarizado con ellos.}. Los semáforos presentan dos grandes limitaciones:
\begin{enumerate}
    \item Están basados en variables compartidas del programa, por lo que no fomentan la modularidad de los programas, impidiendo su reutilización.
    \item Las operaciones de los semáforos (\verb|sem_wait| y \verb|sem_signal|) se encuentran dispersas a lo largo del código del programa concurrente. Además, estas instrucciones no solo afectan al bloque de código en el que se encuentran, sino a cualquier otro bloque que use el mismo semáforo.
\end{enumerate}
En definitiva, los semáforos no son un buen mecanismo de programación concurrente, y además la verificación de programas que usan semáforos es muy complicada. 

Era necesario encontrar un nuevo mecanismo de programación concurrente que permitiera la encapsulación de la información y de la sincronización entre procesos, así como programar las operaciones de sincronización (como \verb|wait| o \verb|signal|) dentro de bloques o procedimientos que se ejecuten con instrucciones atómicas, para que las instrucciones de sincronización no se encuentren desperdigadas por el programa.
Fue Charles Antony Richard Hoare quien inventó los monitores, concepto en el que ahondaremos a lo largo de este Capítulo.\\

La idea básica de monitor es un módulo que contiene un conjunto de variables a las que llamaremos \textit{variables permanentes}\footnote{A pesar de su nombre, no serán constantes, sino que podremos modificar su valor.}, de forma que dichas variables solo podrán ser alteradas dentro de los procedimientos del módulo monitor. Garantizaremos que la ejecución de cada uno de esos procedimientos se ejecute la mayor parte del tiempo como una única instrucción atómica, salvo que se produzca algo por lo que interrumpir la ejecución del procedimiento.\\

Podemos pensar en un monitor como en un tipo de dato abstracto que define tipos y variables permanentes propias del monitor, así como un conjunto de procedimientos dentro de dicho módulo. No debemos pensar en los monitores como en una clase, ya que no pueden hacer lo mismo que ellas (no se pueden instanciar y tampoco existe polimorfismo o ligadura dinámica).

\subsubsection{Ventajas}
A continuación, los programas concurrentes estarán formados tanto por procesos que se ejecutarán de forma concurrente, como por monitores, los cuales velarán por la sincronización y acceso a variables compartidas de dichos procesos, de forma que no se produzcan condiciones de carrera o comportamientos indeseados. Podremos modelar tantas relaciones de interacción entre los procesos de un programa concurrente como queramos. De esta forma, el uso de los monitores o de procedimientos asociados a monitores no restringen las posibilidades del modelado de un sistema concurrente. 

Los procesos de un programa concurrente no tendrán que llamar a operaciones de sincronización, sino que llamarán a procedimientos del monitor, los cuales realizarán la funcionalidad deseada sobre las variables compartidas garantizando la sincronización entre los procesos.\\

Además, los monitores nos permiten una alta reusabilidad de código, ya que podremos reutilizar un monitor ya creado para resolver problemas similares. Sin embargo, la reutilización de código no es similar a la usada en programación orientada a objetos mediante instancias de una misma clase, sino que se hará por copias parametrizables: tendremos una definición de un monitor basada en parámetros, y cuando necesitemos usar un monitor, crearemos una copia de dicha definición parametrizándola (pasándole los parámetros que necesitemos para resolver nuestro problema). De esta forma, no es reutilización por instanciación, sino por \textit{parametrización}.\\

Los procesos que usemos en los progrmas concurrentes no verán el acceso a las variables compartidas, sino que será realizado por los procedimientos del monitor, garantizando que se hacen como deben hacerse, evitando condiciones de carrera. De esta forma, los monitores garantizan la ocultación de las variables compartidas, haciéndolas transparentes a los procesos del sistema concurrente.\\

Finalmente, existen unos axiomas que nos permiten verificar los programas concurrentes que usen monitores de forma sencilla. Dichas demostraciones estarán basadas en el uso de los invariables globales. Ahondaremos en la verificación de programas concurrentes que utilicen monitores más adelante.

\subsection{Concepto de monitor}\label{sec:concepto_de_monitor}
A modo de resumen para comenzar a definir lo que es un monitor, podemos decir que:
\begin{itemize}
    \item Es un módulo con un conjunto de variables permanentes que solo pueden ser modificadas por los procedimientos del monitor.
    \item Cada uno de los procedimientos\footnote{Podemos pensar en ellos como en los ``métodos'' de una clase, haciendo hincapié en que los monitores \textbf{no son} clases.} de un monitor se ejecutan en exclusión mutua (garantizando el acceso a las variables compartidas sin condiciones de carrera). Sin embargo, estos no tienen por qué ejecutarse completamente, sino que pueden interrumpirse y en algún momento futuro seguir ejecutándose en exclusión mutua.
    \item La ejecución de los procedimientos de un monitor modifican el estado interno del mismo (esto es, el conjunto de las variables permanentes asociadas al monitor).
    \item El estado inicial del monitor (de sus variables permanentes) se establece mediante la ejecución de un procedimiento especial, al que llamaremos \textit{código de inicialización}. Este se ejecuta tras la declaración de una variable de tipo monitor y da valores iniciales a las variables permanentes.
\end{itemize}
De esta forma, un monitor puede visualizarse de forma intuitiva en la Tabla~\ref{esq:monitor_1}, como un conjunto que englobla:
\begin{itemize}
    \item Un conjunto de variables, llamadas \textit{variables permanentes}, que no son accesibles desde fuera del monitor.
    \item Un conjunto de procedimientos que el monitor proporciona como servicio a los procesos de un programa concurrente (para por ejemplo, acceder a las variables permanentes que serán las variables que compartan dichos procesos), llamados \textit{procedimientos exportados} o \textit{exportables}.
    \item Un procedimiento especial llamado \textit{código de inicialización}, que permite inicializar las variable permanentes.
\end{itemize}

\begin{table}[H]
\centering
\begin{tabular}{|l|}
\hline
Variables \\
permanentes \\
\hline
Procedimientos \\
exportados \\
\hline
Código de \\ 
inicialización \\
\hline
\end{tabular}
\caption{Esquema de un monitor.}
\label{esq:monitor_1}
\end{table}

\begin{ejemplo}
    Aunque todavía no entendemos muy bien qué es un monitor, daremos a continuación un ejemplo de uso del mismo para ilustrar la definición que queremos dar de monitor, pese a que algunas cosas del ejemplo no podamos entenderlas todavía y deberemos dejarlas para más adelante\footnote{Como el tipo de dato \texttt{cond}.}.\\

    En este ejemplo, queremos solventar un problema mediante el paradigma productor/consumidor. Tendremos dos procesos, un productor y un consumidor, de forma que el productor escribirá en un buffer (o vector) que usaremos como cola cíclica (esto es, que si nos pasamos de la posición final, volvemos al inicio y con planificación FIFO), mientras que el consumidor irá leyendo los datos de dicho buffer. Siendo \verb|Buf| una variable de tipo monitor que luego definiremos en este ejemplo, el código del productor y del consumidor será el siguiente (pensando en que tenemos que usar procedimientos del monitor para el acceso a las variables compartidas, en este caso el buffer):
    \begin{minted}[escapeinside=\#\#]{pascal}
        Proceso Prod1::
          var d : tipo_dato;

          while true do begin
            d = producir();
            Buf.insertar(d); {mete d en el buffer}
          end do
    \end{minted}
    \begin{minted}[escapeinside=\#\#]{pascal}
        Proceso Cons1::
          var x : tipo_dato;

          while true do begin
            Buf.retirar(x); {retira del buffer en x}
            consumir(x);
          end do
    \end{minted}
    El código del monitor será el siguiente en pseudo-pascal (hemos omitido el código de inicialización):
    \begin{minted}[escapeinside=\#\#]{pascal}
        Monitor Buf
          var
            -elementos_ocupados : int;
            -frente, atras: 0..N-1;
            -no_vacio, no_lleno : cond;

          +insertar(d : tipo_dato);
          +retirar(var x : tipo_dato);
    \end{minted}
    Donde vemos 5 variables permanentes: \verb|elementos_ocupados|, que mide la cantidad de posiciones ocupadas del buffer, \verb|frente|, que marca la casilla en la que el productor insertará el próximo dato (por tanto, ha de estar siempre vacía), \verb|atras|, que marca la casilla de la que leerá el consumidor, \verb|no_vacio| y \verb|no_lleno|, variables de tipo \verb|cond|, las cuales aprenderemos lo que hacen más adelante.
    
    Contamos además con dos procedimientos: \verb|insertar|, que inserta un dato en el buffer en caso de que haya hueco (si no hay hueco, se bloquea hasta que el consumidor lea un dato y deje un hueco libre):
    \begin{minted}[escapeinside=\#\#]{pascal}
        procedure insertar(d : tipo_dato) begin
          if((frente + 1) mod N = atras) then no_lleno.wait();
          introducir(buf, frente, d);  {inserta d en la posicion frente en el buffer}
          elementos_ocupados += 1;
          frente = (frente + 1) mod N;
          no_vacio.signal();
        end
    \end{minted}
    Y con el procedimiento \verb|retirar|, que retira un dato del buffer y lo devuelve como resultado del procedimiento, siempre que esto sea posible (es decir, si no hay ningún dato que leer en el buffer, se bloquea esperando a que el productor ponga algún dato):
    \begin{minted}[escapeinside=\#\#]{pascal}
        procedure retirar(var x : tipo_dato) begin
          if(frente = atras) then no_vacio.wait();
          eliminar(buf, atras, x);  {inserta buf[atras] en x y lo borra del buffer}
          elementos_ocupados -= 1;
          atras = atras mod N + 1;
          no_lleno.signal();
        end
    \end{minted}
    Como hemos ya comentado mientras mostrábamos los pseudocódigos del ejemplo, hay que establecer condiciones que identifiquen las dos condiciones inseguras del ejemplo: que el buffer esté lleno o que el buffer esté vacío:
    \begin{itemize}
        \item Si \verb|frente = atras|, entonces el último dato que se ha de consumir está en una casilla vacía en la que el productor escribirá. Se trata de la situación en la que el buffer está vacío. Debemos por tanto, evitar que el consumidor lea un dato del buffer.
        \item Si \verb|(frente + 1) mod N = atras|, entonces el siguiente dato a introducir en el buffer está justo delante del dato a consumir. Se trata de la situación en la que el buffer está lleno. Debemos por tanto, evitar que el productor introduzca un dato en el buffer\footnote{Definimos anteriormente que \texttt{frente} siempre apunta a una casilla vacía, por lo que como máximo el buffer tendrá ocupados $N-1$ elementos.}.
    \end{itemize}
    Los procesos del programa llaman a los procedimientos del monitor, y no tienen acceso directo al buffer, por lo que no pueden saber cuándo este está lleno o vacío. De esta forma, lo que sucederá es que los procedimientos internos del monitor realizarán una sincronización interna mediante el uso de llamadas bloqueantes:
    \begin{itemize}
        \item Si el buffer está lleno y el productor se dispone a escribir un dato, quedará el proceso bloqueado hasta que un consumidor lea un dato. Este señalará (\verb|signal|) al productor, desbloqueándolo.
        \item Si el buffer está vacío y el consumidor se disopne a leer un dato, quedará bloqueado el proceso que ejecute el procedimiento del monitor. Cuando el productor escriba un dato, enviará una señal al consumidor, desbloqueándolo.
    \end{itemize}
    Esta funcionalidad se consigue mediante las variables de tipo \verb|cond|. Se verán a continuación, pero para entenderlas por ahora digamos que necesitamos tener una variable de tipo \verb|cond| por cada razón por la que queremos bloquear un proceso\footnote{En el caso de productor/consumidor, queremos bloquear un proceso si sucede alguno de los dos puntos superiores, condiciones inseguras, luego nos harán falta dos variables de tipo \texttt{cond}. En otros problemas, el número de variables de tipo \texttt{cond} podría ser otro.}.\\

    El código de los procedimientos es ejecutado por los propios procesos que ejecutan cada proceso (productor o consumidor, en este caso) del programa concurrente. Por tanto, si el productor ejecuta un procedimiento del monitor con un \verb|wait|, dicho proceso se bloqueará y no podrá ejecutar código hasta desbloquearse.\\

    Para que el código que hemos visto funcione adecuadamente, nos falta introducir un último concepto en los monitores, y es que mientras se ejecuta un procedimiento de un monitor, no se puede ejecutar ningún otro, sino que han de ejecutarse en \textbf{exclusión mutua}.
\end{ejemplo}

\subsection{Características de programación con monitores}
Una vez ilustrado el uso de la herramienta que estamos construyendo en este Capítulo mediante el ejemplo anterior, vamos ahora a introducir la noción de que solo puede ejecutarse a la vez un único procedimiento de un monitor.\\

Como ya hemos visto, los procedimientos de los monitores no tienen por qué ejecutarse de principio a fin, sino que un proceso puede comenzar a ejecutar un procedimiento, bloquearse (dejando por tanto libre al monitor) y que otro proceso comience a ejecutar un procedimiento de dicho monitor, sucediéndose un entrelazamiento de las trazas de ejecución de los procedimientos.\\

Cuando un proceso se encuentra ejecutando un procedimiento del monitor, decimos que el monitor está \emph{ocupado}. En caso contrario, diremos que este está \emph{libre}. Notemos que si un proceso se bloquea mientras ejecuta un procedimiento del monitor, el monitor tiene que quedar libre, ya que si no no habría forma de volver a despertar a dicho proceso (tenemos que ejecutar un \verb|signal| sobre la misma variable \verb|cond| que bloqueó al proceso\footnote{Se explicará más adelante.}). La situación de bloquear a un proceso y dejar que entre otro al monitor es delicada y debe hacerse con cuidado, para garantizar que solo haya un único proceso ejecutando un procedimiento del monitor al mismo tiempo.\\

Los monitores son objetos \emph{pasivos}. Esto es, no tienen una hebra dentro que ejecute su código, sino que simplemente proporciona código (sus procedimientos) a otros procesos para que sean ellos quien ejecuten el código del monitor.\\

Para implementar una librería con monitores en un lenguaje de programación base, este debe tener la propiedad de ser \emph{reentrante}.
\begin{definicion}
    Un lenguaje de programación tiene la propiedad de ser reentrante si, siempre que tengamos un proceso ejecutando una función y este se bloquea, sea capaz de conservar la siguiente instrucción a ejecutar y el valor de sus variables locales tras desbloquearse. Es decir, el proceso no debe enterarse localmente\footnote{Las variables locales a la función deben mantenerse, pero puede haber variables globales que sí hayan cambiado.} de que nada haya cambiado mientras estaba bloqueado.
\end{definicion}
Notemos que debemos tener esta propiedad en el lenguaje de programación con el que trabajemos para poder hacer uso de funciones bloqueantes (como \verb|wait|) dentro de los procedimientos de un monitor, algo básico en el funcionamiento de este. Afortunadamente, actualmente todos los lenguajes de programación que encontramos en el mercado son reentrantes.\\

\subsubsection{Copias paramétricas de un monitor}
El siguiente ejemplo nos ilustra cómo podemos crear nuevos monitores a partir de uno ya creado, fijando parámetros que use el código de inicialización.

\begin{ejemplo}
    Aunque los monitores están pensados para programas concurrentes (ya que no tiene sentido su uso en programas secuenciales), usaremos en este ejemplo un monitor en un programa secuencial, ya que solo nos interesa la forma en la que los monitores inicializan sus variables permanentes\footnote{Además, no hemos terminado de desarrollar cómo es que solo puede ejecutarse a la vez un único procedimiento del monitor, por lo que no entendemos hasta ahora cómo es que sirven para sincronizar programas concurrentes.}.\\

    Tenemos un programa en el que necesitamos dos variables, las cuales queremos consultar e incrementar mediante un incremento previamente fijado que no cambiará. Para ello, creamos un monitor de acceso a una variable, con parámetros de entrada, para luego poder crear dos copias parametrizadas del mismo. El código del monitor será algo parecido a:
    \begin{minted}[escapeinside=\#\#]{pascal}
        class monitor VariableProtegida(inicio, incremento : integer);
          var x, inc : integer;

          procedure incremento();
          begin
            x = x + inc;
          end

          procedure valor(var v : integer);
          begin
            v = x;
          end
          
          begin
            x = inicio; inc = incremento;
          end
    \end{minted}
    De esta forma, podemos usar dos copias del monitor de la forma:
    \begin{minted}[escapeinside=\#\#]{pascal}
        var mv1 : VariableProtegida(0,1);   {empieza en 0 e incrementa en 1}
            mv2 : VariableProtegida(10,4);  {empieza en 10 e incrementa en 4}
            a, b : integer;
        begin
          mv1.incremento();  {+=1}
          mv1.valor(a);      {a=1}
          mv2.incremento();  {+=4}
          mv2.valor(b);      {b=14}
        end
    \end{minted}
\end{ejemplo}

\subsection{Exclusión mutua en los procedimientos de un monitor}
Si tenemos varios procesos del programa concurrente que quieren hacer uso de procedimientos del monitor a la vez, solo podremos dejar pasar un proceso al monitor (suponiendo que este se encuentre libre). Para los otros procesos, almacenaremos su llamada al procedimiento. 

Para ello, todos los monitores tienen implementada una cola con planificación FIFO, llamada \emph{cola de entrada al monitor}. Si tenemos dos procesos que quieren acceder a un procedimiento de un monitor libre, solo podrá hacerlo un proceso. La llamada al procedimiento del monitor del otro proceso quedará almacenada en la cola de entrada al monitor, y este pasará a ejecutar el procedimiento deseado una vez el proceso anterior haya dejado libre el monitor. 
\begin{observacion}
    En esta asignatura, supondremos que la cola de entrada al monitor es suficientemente larga como para albergar a todos los procesos que necesiten esperar a que el monitor quede libre.
\end{observacion}

Podemos representar la vida de un proceso de un programa concurrente que hace uso de monitores para sincronizar a sus procesos con el siguiente diagrama:
\begin{figure}[H]
\centering
\begin{tikzpicture}[node/.style={rectangle, draw, minimum size=1cm, align=center},
                    edge/.style={-stealth}]
    \node[node] (A) {Procedimiento\\terminado};
    \node[node, left =of A] (B) {\begin{tabular}{c}Ejecutando código\\del proceso\\(fuera del monitor)\end{tabular}};
    \node[node, left =of B] (C) {\begin{tabular}{c}Llamando a\\procedimiento\\del monitor\end{tabular}};
    \node[node, below =of C] (D) {\begin{tabular}{c}Esperando en\\cola de entrada\\al monitor\end{tabular}};
    \node[node, right =of D] (E) {\begin{tabular}{c}Monitor\\libre\end{tabular}};
    \node[node, right =of E, xshift=.6cm] (F) {\begin{tabular}{c}Ejecutando procedimiento\\(dentro del monitor)\end{tabular}};
    
    \draw[edge] (A) -- (B);
    \draw[edge] (B) -- (C);
    \draw[edge] (C) -- (D);
    \draw[edge] (D) -- (E);
    \draw[edge] (E) -- (F);
    \draw[edge] (F) -- (A);
\end{tikzpicture}
\caption{Vida de un proceso en un programa concurrente con monitores.}
\end{figure}

De esta forma, podemos ahora reescribir la descripción gráfica de monitor que hicimos en la tabla~\ref{esq:monitor_1}, incluyendo ahora la cola de entrada al monitor, tal y como vemos en la tabla~\ref{esq:monitor_2}
\begin{table}[H]
\centering
\begin{tabular}{|l|}
\hline
Cola del\\
monitor \\
\hline
Variables \\
permanentes \\
\hline
Procedimientos \\
exportados \\
\hline
Código de \\ 
inicialización \\
\hline
\end{tabular}
\caption{Esquema de un monitor incluyendo la cola de entrada.}
\label{esq:monitor_2}
\end{table}

\subsection{Operaciones de sincronización}
Las operaciones de sincronización entre los procesos de un programa concurrente se programan, como ya hemos visto, dentro de los procedimientos del monitor. Son instrucciones que permiten detener la ejecución de un procedimiento de un monitor y bloquear en una cola al proceso que ha hecho la llamada del procedimiento del monitor. Tenemos para realizar esta acción dos operaciones principales: \verb|wait| y \verb|signal|.\\

Sin embargo, las operaciones \verb|wait| y \verb|signal| que manejamos en monitores no se parecen a las que usábamos en los semáforos:
\begin{itemize}
    \item En los semáforos, la ejecución de \verb|wait| ofrecía la posibilidad de bloquear al proceso, ya que no lo hacía si el entero de dentro del semáforo era mayor estricto que $0$. Por contra, en monitores la llamada \verb|wait| siempre será bloqueante.
    \item Las operaciones \verb|wait| y \verb|signal| eran relativas a un semáforo: hacía falta usar un semáforo por cada razón que tuviéramos dentro de un programa concurrente para bloquear a uno o varios procesos (en el caso del productor/consumidor, usar dos semáforos). Sin embargo, con un solo monitor podremos bloquear procesos por tantas razones como queramos, usando un nuevo tipo de dato.
\end{itemize}

\subsubsection{Tipo de dato \texttt{cond}}
En los monitores, para poder usar las operaciones de \verb|wait| y \verb|signal|, será necesario utilizar una variable de tipo de dato condición, o \verb|cond|. 

Las variables tipo \verb|cond| se encuentran junto con las variables permanentes de un monitor. Estas no se inicializan a ningún valor.\\

En nuestro monitor, tendremos varias razones por las que queramos bloquear a los procesos concurrentes de nuestro programa por alguna determinada razón, hasta que se cumpla una condición determinada, a la que llamaremos \textit{condición de sincronización}. Por ejemplo, en el problema del productor/consumidor:
\begin{itemize}
    \item Queremos bloquear a cualquier productor que intente escribir si la estructura de datos intermedia que usamos está llena. Desbloquearemos a un proceso productor cuando se vacíe un hueco en dicha estructura.
    \item Además, queremos bloquear a cualquier consumidor que intente leer de la estructura de datos intermedia cuando esta esté vacía. Desbloquearemos a un consumidor cuando algún productor haya escrito algún dato.
\end{itemize}
Por cada razón o condición distinta por la que queramos bloquear a los procesos de un programa concurrente en relación a una misma variable compartida (para evitar estados inseguros), crearemos una variable de tipo \verb|cond|. Es decir, una variable por cada una de las razones por las que queramos que esperen los procesos. En el ejemplo del productor/consumidor, son necesarias únicamente dos variables de tipo \verb|cond|.\\

Las variables de tipo \verb|cond| admiten 4 métodos (aunque solo recomendamos usar los dos primeros):
\begin{description}
    \item [wait] Bloquea al proceso que ejecuta este método. Dicho proceso pasa a una cola asociada a la variable condición correspondiente con planificación FIFO\@.
    \item [signal] En caso de haber algún proceso bloqueado en la cola asociada a la variable condición correspondiente, lo desbloquea\footnote{Hemos de tener cuidado con esto, se explicará más adelante.}. Si esta cola está vacía, es equivalente a una operación nula\footnote{Esto es, equivalente a la instrucción \texttt{;}.}.
    \item [queue] Devuelve un booleano que indica (\verb|true|) si la cola asociada a la variable condición contiene al menos un proceso bloqueado.
    \item [signal\_all] Desbloquea de una sola vez a todos los procesos bloqueados en la cola asociada a la variable condición. El orden de dicha cola no se mantiene para realizar la petición de acceso al monitor, por lo que se produce competencia entre los procesos para entrar al monitor, incumpliendo la propiedad de equidad entre procesos. Depende de la semántica de las señales del lenguaje\footnote{Se explicará más adelante qué es esto.}. Se recomienda \textbf{no usarla}.
\end{description}
De esta forma, la representación gráfica final de un monitor es la que se muestra en la tabla~\ref{esq:monitor_3}:
\begin{table}[H]
\centering
\begin{tabular}{|l|}
\hline
Cola del\\
monitor \\
\hline
Variables \\
permanentes \\
\hline
Variables \\
condición y\\
colas de procesos\\
bloqueados\\
\hline
Procedimientos \\
exportados \\
\hline
Código de \\ 
inicialización \\
\hline
\end{tabular}
\caption{Esquema de un monitor incluyendo las variables condición.}
\label{esq:monitor_3}
\end{table}

\subsubsection{Semánticas de señales}
Como hemos comentado ya, los monitores solo permiten que un único proceso se encuentre ejecutando un procedimiento del mismo. En este caso, decíamos que el monitor está ocupado. En caso de que un proceso que estaba ejecutando el procedimiento ejecute un \verb|wait| (o salga del procedimiento), hay que dejar el monitor libre para dejar pasar a otro. Se trata de un momento muy delicado, ya que se pueden producir condiciones de carrera entre los procesos que quieran conseguir el monitor. Esta situación la hemos solucionado ya con la cola de entrada al monitor, ya que con la planificación FIFO, solo podrá entrar un único proceso al monitor.\\

Si ahora el proceso nuevo que ejecuta el monitor ejecuta un \verb|signal| sobre una variable condición (recordemos que tenemos al menos un proceso bloqueado), desbloqueará al primer proceso de su cola asociada, que estaba ejecutando código del monitor, por lo que ahora tenemos dos procesos ejecutando código del monitor: el proceso que señala y el señalado (el recién desbloqueado). Esta condición no puede darse, ya que los procedimientos de un monitor deben ejecutarse en exclusión mutua. Una solución a este problema es que el procedimiento que señala se bloquee en la cola de entrada al monitor\footnote{Veremos más soluciones.}, dejando paso al recién desbloqueado.

Esta solución plantea un problema, y es que si el proceso señalador se bloquea, puede que en algún momento deje al monitor libre, por lo que se meta un proceso de la cola de entrada al monitor, planteando nuevamente la situación en la que tenemos dos procesos en el monitor (el desbloqueado y el primero que estaba esperando entrar al monitor). Deberá haber un mecanismo que elija quién de los dos acaba finalmente con el monitor. En caso de que sea el proceso que estaba esperando en la cola de entrada, diremos que se produce un \emph{robo de señal}, donde el proceso recién desbloqueado debe irse al final de la cola de entrada al monitor.\\

Para solucionar este segundo problema, algunos lenguajes implementan una \emph{semántica desplazante}\footnote{Se trabajará más adelante sobre las semánticas de las señales.} en las señales: el proceso que ejecuta el \verb|signal| le pasa el monitor al proceso que recibió la señal (el primero en la cola de bloqueados de la variable condición correspondiente), sin liberar en ningún momento el monitor, de forma que el proceso señalado tiene prioridad. Se dice que la señal usada con la operación \verb|signal| tiene \emph{semántica desplazante}.\\

Cabe destacar que \textbf{no todos los lenguajes con monitores tienen señales con semántica desplazante}, por lo que en dichos lenguajes pueden sucederse robos de señales. En esta sección y en la siguiente, supondremos que estaremos trabajando siempre con señales \verb|signal| con semántica desplazante, de forma que el proceso señalador se bloqueará tras ejecutar un \verb|signal| y podemos pensar que es enviado a la cola de entrada al monitor. En una futura sección veremos los tipos de semánticas de señales que podemos encontrarnos (cada uno hará que el comportamiento de \verb|signal| sea distinto).\\

\noindent
Como comentario final a la descripción de un monitor y para motivar la siguiente sección:
\begin{itemize}
    \item Se presupone que el programador de monitores es un programador experto, de forma que el compilador en ningún momento se dedicará a comprobar si hemos programado de forma correcta un monitor o un procedimiento de él, más allá de la sintaxis del código.
    \item No deben programarse operaciones \verb|wait| indebidas ni omitirse operaciones \verb|signal| necesarias. Para comprobar esto, usaremos nuestro sistema de verificación formal.
\end{itemize}

\section{Verificación de programas con monitores}
En la verificación de los programas concurrentes que hemos manejado hasta ahora, hemos primero demostrado la corrección secuencial de cada proceso que forma parte de un programa secuencial, para luego demostrar la no interferencia entre los mismos. 

Sin embargo, ahora que introducimos los monitores, esto no podrá ser nunca más así, ya que un programador nunca puede conocer a priori la traza que genera un proceso que forma parte de un programa concurrente con monitores, ya que al ejecutar procedimientos de monitores, estos pueden quedar bloqueados y se ejecutarían en medio instrucciones de otros procesos que podrían alterar las variables compartidas del programa, falseando alguna precondición o poscondición del proceso bloqueado, por lo que tras desbloquearse, no podemos esperar nada de dicho proceso.\\

Es por tanto que ahora la estrategia a seguir en las demostraciones es mediante un Invariante de Monitor.

\subsection{Invariante de monitor}
\begin{definicion}[Invariante de Monitor]
    Un Invariante de Monitor (IM) es una relación entre las variables permanentes de un monitor que debe ser cierta en cualquier estado del programa concurrente, excepto cuando un proceso esté ejecutando código de un procedimiento del monitor.
\end{definicion}
De esta forma, un IM puede no ser cierto durante la ejecución de un procedimiento por parte de un proceso, pero este ha de cumplirse antes y después de la ejecución de dicho procedimiento.\\

Si conseguimos probar la existencia de un IM en un programa concurrente, entonces bastará con probar cada una de las secciones de código secuenciales entre llamadas a procedimientos del monitor. Para probar finalmente la corrección de los procesos, usaremos que los IM se mantienen antes y después de las llamadas a procedimientos, para conseguir probar finalmente la corrección de cada uno de los procesos. Si nuestro IM estaba relacionado con la solución al problema, como el acceso a variables compartidas estará controlado por los monitores, al final del programa todos los IMs demostrados se seguirán compliendo, por lo que tendremos probada la corrección de nuestro programa concurrente.

Es decir, primero demostraremos que por cada monitor que usamos se verifica un IM, y luego pasaremos a probar la corrección de cada proceso que interviene en el programa concurrente, usando para ello dichos IMs. Finamente, tendremos probado el programa concurrente.

\subsubsection{Esquema de demostración}
Suponiendo que hemos encontrado una relación matemática entre las variables permanentes de un monitor y queremos probar que se trata de un IM\footnote{A continuación, llamaremos a dicha condición \texttt{IM}, pese a no haber demostrado aún que se trate de verdad de un IM.}, lo primero será probar que $IM$ se cumple en el estado inicial del monitor, esto es, justo después de la inicialización de las variables permanentes, por lo que tendremos que probar que se verifica el \textbf{triple de inicialización de variables}:
\begin{equation*}
    \{V\}\ \text{código de inicialización}\ \{IM\}
\end{equation*}
Posteriormente, deberemos probar que $IM$ se mantiene antes y después de la llamada a cada procedimiento. Es decir, notando por $IN$ a las precondiciones que tenemos antes de la ejecución de un procedimiento y por $OUT$ a las poscondiciones que deseamos tener tras dicho procedimiento, debemos demostrar los \textbf{triples de procedimientos del monitor}, es decir, demostrar un triple
\begin{equation*}
    \{IM \land IN\}\ \text{procedimiento}\ \{IM \land OUT\}
\end{equation*}
por cada procedimiento que tenga nuestro monitor.

\subsubsection{Cuidado con las intereferencias}
Terminaremos de ver esto más adelante, pero es necesario darnos cuenta de un detalle, y es que si un procedimiento modifica el valor de alguna variable compartida que se usa en otro proceso, debemos demostrar la no interferencia entre dichas instrucciones. Ilustramos esto con el siguiente ejemplo.
\begin{ejemplo}
    Si tenemos un monitor llamado \verb|Buf| con un procedimientos \verb|retirar(x)|, de forma que modifica el valor del parámetro que le pasamos, ante el siguiente código (si \verb|x| es una variable compartida):
    \begin{minted}[escapeinside=\#\#]{pascal}
        cobegin y = x; #||# Buf.retirar(x); coend
    \end{minted}
    Tenemos que probar que al cambiar el valor de \verb|x| con el procedimiento \verb|retirar|, no hay interferencia con la instrucción de la izquierda. Es decir, tenemos que probar:
    \begin{align*}
        &NI(pre(y=x), Buf.retirar(x)) \\
        &NI(pos(y=x), Buf.retirar(x))
    \end{align*}
    Sin embargo, en caso de ejecutar el siguiente código:
    \begin{minted}[escapeinside=\#\#]{pascal}
        z = x;
        cobegin y=z; #||# Buf.retirar(x); coend
    \end{minted}
    No tendríamos que hacerlo, ya que el uso de variables disjuntas nos garantiza la no interferencia entre dichas instrucciones.
\end{ejemplo}

\subsection{Axiomas para operaciones de sincronización desplazantes}\label{sec:ax_desplazantes}
Sabemos ya demostrar toda la corrección de un programa secuencial que usa monitores, salvo por un detalle, y es que no sabemos nada sobre cómo demostrar los triples:
\begin{align*}
    \{P\}\ &c.wait();\ \{Q\} \\
    \{P\}\ &c.signal();\ \{Q\}
\end{align*}
para cualesquiera asertos $P$ y $Q$.

En esta subsección, trataremos de dar axiomas para la comprobación de dichos triples, razonándolos de forma intuitiva y mediante el uso de Invariantes de Monitores.\\

\subsubsection{Axioma de operación \texttt{wait}}
Comenzaremos primero con el triple $\{P\}\ c.wait();\ \{Q\}$. Para necesitar ejecutar una instrucción \verb|wait| en un procedimiento de un monitor, lo que sucede es que estamos cerca de un estado inseguro del programa (intuitivamente, que $IM$ está a punto de incumplirse), pero no llegamos a él, porque para ello ejecutamos esta operación, para impedir que el proceso ejecute una instrucción que falsee el $IM$. Por tanto, el proceso se bloquea, dejando libre el monitor, por lo que entra otro proceso a ejecutar otro procedimiento. 

Solo podremos desbloquear al proceso cuando nos alejemos de dicho estado inseguro, por lo que además de cumplirse el $IM$, deberá cumplirse una condición un tanto más estricta que el $IM$ (que nos indique que estamos lejos de aquel estado inseguro por el cual se bloqueó el proceso). Dicha condición recibe el nombre de \textit{condición de sincronización}, y la notaremos por $C$\footnote{Notemos que según hemos definido $C$, ha de verificarse que $C\rightarrow IM$.}. 

Resumiendo:
\begin{itemize}
    \item Antes de ejecutar la operación \verb|wait|, hemos de estar en un estado seguro del programa, por lo que ha de cumplirse el $IM$.
    \item Tras ejecutar la operación \verb|wait| (es decir, después de que el proceso haya sido desbloqueado), ha de cumplirse la condición de sincronización $C$.
\end{itemize}
Teniendo en cuenta que además se puede cumplir un invariante local al que llamamos $L$ (esto es, relaciones entre variables locales del procedimiento del monitor) antes y después\footnote{Gracias a que estamos en lenguajes reentrantes.} de dicha instrucción \verb|wait|.

De esta forma, acabamos de razonar de forma intuitiva el \textbf{Axioma de la operación wait}:
\begin{equation*}
    \{IM \land L\}\ c.wait();\ \{C \land L\}
\end{equation*}

\subsubsection{Axioma de operación \texttt{signal}}
Si nos disponemos a ejecutar una instrucción \verb|signal| en nuestro código, es porque el estado del programa se ha alejado de la condición insegura de la que hablábamos en la subsección anterior, que falsearía el valor de verdad de $IM$. Por tanto, el programa ha llegado a un punto en el que se cumple la condición de sincronización $C$, y ya puede desbloquear al proceso que anteriormente bloqueó. Tras su desbloqueo, este proceso podría ejecutar una instrucción que volviera a acercarnos a un estado inseguro, pero sin llegar a él (ya que $C$ era suficientemente restrictiva), por lo que como poscondición de la instrucción \verb|signal| no podremos garantizar $C$, sino solo podremos asegurar que se sigue cumpliendo $IM$.

Añadiendo la posibilidad de tener un invariante local $L$ y que si la cola de la variable condición está vacía, la operación \verb|signal| es una instrucción nula, llegamos al \textbf{Axioma de la operación signal}:
\begin{equation*}
    \{\lnot vacio(c) \land C \land L\}\ c.signal();\ \{IM \land L\}
\end{equation*}

o equivalentemente:
\begin{equation*}
    \{c.queue() \land C \land L\}\ c.signal();\ \{IM \land L\}
\end{equation*}
En caso de cumplirse que $c.queue() = false$, entonces negaría la precondición del triple, haciéndolo la regla cierta por un razonamiento por vacuidad.\\

\begin{observacion}
    Notemos que el axioma de la operación \verb|signal| funciona porque hemos suspuesto que \textbf{tenemos semántica desplazante}, ya que después de ejecutar \verb|signal| desbloqueamos al proceso que teníamos bloqueado, cediéndole el monitor, por lo que dicho proceso seguirá procesando su procedimiento. Cuando el proceso señalador vuelva al monitor, solo podremos garantizar que se cumple $IM$, ya que tanto el proceso señalado como cualquier otro que se introduzca en el monitor después del señalado (veremos más adelante si esto es posible), pueden cambiar la condición $C$, por lo que solo podemos esperar $IM$.
\end{observacion}~\\

Una vez vistos ya todos los axiomas sobre verificación de operaciones de sincronización de monitores, estamos listos para desmotrar la corrección de un $IM$. Lo haremos en el siguiente ejemplo.

\begin{ejemplo}
    En este ejemplo, queremos programar un monitor que simule el funcionamiento de un semáforo. Para ello, se nos ha ocurrido el siguiente código:
    \begin{minted}[escapeinside=\#\#]{pascal}
        Monitor Semaforo;
          var s : integer;
              c : cond;

          procedure P;
          begin
            if s=0 then
              c.wait;
            else
              null;
            end if
            s = s - 1;
          end

          procedure V;
          begin
            s = s + 1;
            c.signal;
          end

          begin   {código de inicialización}
            s = 0;
          end
    \end{minted}
    Donde hemos llamado \verb|P| a la función \verb|sem_wait| del semáforo y por \verb|V| a la función \verb|sem_signal|.\\

    Procedemos a realizar la demostración de que existe un Invariante de Monitor que se mantiene tras la inicialización de las variables permanentes de nuestro monitor y antes y después de cada procedimiento, con la finalidad de poder usar dicho IM en las demostraciones de cualquier programa concurrente que use el semáforo que acabamos de implementar mediante un monitor.
    \begin{proof}
        Tratamos de demostrar que este monitor tiene como IM el aserto
        \begin{equation*}
            IM \equiv \{s \geq 0\}
        \end{equation*}

        \begin{enumerate}
            \item Primero, tenemos que demostrar el triple de inicialización de variables:
                \begin{equation*}
                    \{V\}\ s=0;\ \{s\geq 0\}
                \end{equation*}
                Como el triple $\{V\}\ s=0;\ \{s=0\}$ es cierto por el axioma de asignación y tenemos que $\{s=0\}\rightarrow\{s\geq 0\}$, usando la primera regla de la consecuencia tenemos demostrado el triple.
            \item Posteriormente, demostraremos el triple de procedimiento del monitor para el procedimiento \verb|P|: $\{IM\}\ P\ \{IM\}$. Para ello, primero tendremos que probar el triple
                \begin{equation*}
                    \{IM\}\ \texttt{if\ }s=0\texttt{\ then\ }c.wait;\texttt{\ else\ }null;\texttt{\ end if}\ \{s>0\}
                \end{equation*}
                Luego usaremos la regla del \verb|if|, por lo que será suficiente con probar los triples:
                \begin{gather*}
                    \{IM \land s=0\}\ c.wait;\ \{s>0\} \\
                    \{IM \land s\neq0\}\ null;\ \{s>0\}
                \end{gather*}
                \begin{enumerate}
                    \item Comenzamos por el segundo, por ser más sencillo. Tenemos:
                        \begin{equation*}
                            {\{IM \land s\neq0\} \equiv \{s\geq 0 \land s\neq0\}\equiv \{s>0\}}
                        \end{equation*}
                        Por tanto, basta probar el triple ${\{s>0\}\ null;\ \{s>0\}}$, que es cierto por el axioma de la sentencia nula.
                    \item Para el primer triple, buscamos aplicar el axioma de la operación \verb|wait|, por lo que tenemos que buscar la condición de sincronización. Para ello, buscamos la precondición del \verb|signal| asociado a la misma variable condición, que se encuentra en el procedimiento \verb|V|. Para hallar la precondición de la instrucción \verb|c.signal|, tendremos que demostrar alguna instrucción de dicho procedimiento, con el fin de hallar la precondición.

                        Sobre el código de \verb|V|, vemos que antes de \verb|c.signal| se ejecuta una primera instrucción \verb|s=s+1;|. Suponemos que \verb|V| tiene como precondición $IM$, por lo que buscamos una poscondición para \verb|s=s+1;|:
                        \begin{equation*}
                            \{IM\}\equiv \{s\geq 0\}\ s=s+1;
                        \end{equation*}
                        Puede comprobarse con el axioma de asignación que la poscondición buscada es $\{s>0\}$. Por tanto, esta será la condición de sincronización de la variable condición \verb|c|:
                        \begin{equation*}
                            C \equiv \{s>0\}
                        \end{equation*}
                        Por tanto, por el axioma de la operación \verb|wait| usado con $L=\{V\}$, tenemos que el siguiente triple es cierto:
                        \begin{equation*}
                            \{IM\}\ c.wait;\ \{s>0\}
                        \end{equation*}
                        Como $\{IM \land s=0\}\to \{IM\}$, por la segunda regla de la consecuencia, tenemos demostrado el triple que buscábamos.
                \end{enumerate}
                Una vez demostrados los dos triples, tenemos probado el triple del \verb|if|, por lo que solo faltará probar el triple 
                \begin{equation*}
                    \{s>0\}\ s=s-1;\ \{IM\}
                \end{equation*}
                Para tener probado el triple del procedimiento \verb|P|.

                Como $\{IM\} \equiv \{s\geq 0\}$, basta aplicar el axioma de asignación, para obtener $\{s>0\}\ s=s-1;\ \{s\geq 0\}$.

                Aplicando finalmente la regla de composición sobre el triple del \verb|if| y este último triple, tenemos ya probado $\{IM\}\ P\ \{IM\}$.

            \item Finalmete, hemos de probar el triple $\{IM\}\ V\ \{IM\}$ para garantizar al fin que $IM$ es un IM\@. Para ello, hemos de probar el triple
                \begin{equation*}
                    \{IM\}\ s=s+1; c.signal;\ \{IM\}
                \end{equation*}
                Basta con probar los triples
                \begin{gather*}
                    \{IM\}\ s=s+1;\ \{s>0\} \\
                    \{s>0\}\ c.signal;\ \{IM\}
                \end{gather*}
                y aplicar la regla de composición. El primer triple ya lo demostramos en la demostración del triple del procedimiento \verb|P|, luego bastará probar el segundo, el cual es cierto gracias al axioma de la operación \verb|signal|.

                Acabamos de probar que $\{IM\}\ V\ \{IM\}$, que era el último procedimiento del monitor, luego $IM$ es un IM\@.
        \end{enumerate}
    \end{proof}
\end{ejemplo}

\begin{ejercicio*}
    Se pide demostrar que el siguiente monitor funciona como un semáforo de Habermann. Un semáforo de Habermann se trata de un semáforo normal que lleva la cuenta de:
    \begin{itemize}
        \item El número de llamadas realizadas a \verb|signal|, \verb|nv|.
        \item El número de llamadas realizadas a \verb|wait|, \verb|na|.
        \item El número de llamadas completadas a \verb|wait|, \verb|np|.
    \end{itemize}
    En este caso, llamaremos \verb|P| al procedimiento que simule la operación \verb|wait| y \verb|V| al procedimiento que simule la operación \verb|signal|.
    
    \begin{minted}[escapeinside=\#\#]{pascal}
        Monitor Semaforo;
          var na, np, nv : int;
              c : cond;

          procedure P;
          begin
            na = na + 1;
            if(na > nv) then c.wait();
            np = np + 1;
          end

          procedure V;
          begin
            nv = nv + 1;
            if(na > np) then c.signal();
          end

          begin
            na = 0; np = 0; nv = 0;
          end
    \end{minted}
    Buscamos un IM para preceder a la demostración del mismo. Notemos que las variables permanentes han de cumplir:
    \begin{itemize}
        \item $np \leq na$, ya que para completar una llamada \verb|P| hay que realizar una llamada.
        \item $np \leq nv$, ya que, como inicialmente $np = na = nv = 0$, para poder completar una llamada a \verb|P|, hay que previamente haber hecho una llamada a \verb|V|.
        \item $np \geq \min(na,nv)$, para no detener innecesariamente a los preocesos, cumpliendo la hipótesis de progreso finito.
    \end{itemize}
    De estas tres propiedades, deducimos que el invariante a usar es:
    \begin{equation*}
        \{IM\} \equiv \{np = \min(na,nv)\}
    \end{equation*}
    Pasemos ahora a la demostración del monitor:
    \begin{enumerate}
        \item En primer lugar, probaremos el triple de inicialización de variables:
            \begin{gather*}
                \{V\} \\
                na = 0;\ np=0;\ nv=0; \\
                \{na = 0 \land np = 0 \land nv = 0\} \rightarrow \{IM\}
            \end{gather*}
        \item Posteriormente, probaremos el triple del procedimiento \verb|P|:
            \begin{equation*}
                \{IM \land L\}\ P\ \{C \land L\}
            \end{equation*}

            para ello:
            \begin{gather*}
                \{IM\}\equiv \{np=\min(na,nv)\} \\
                na = na + 1;\\
                \{np = \min(na-1,nv)\} \\
                \texttt{if\ }(na > nv) \texttt{\ then\ } \\
                \{na > nv \land np = \min(na-1,nv)\} \rightarrow \{na-1\geq nv \land np = \min(na-1,nv)\} \rightarrow \\ 
                \rightarrow\{na > nv \land np = nv\} \rightarrow \{np = \min(na,nv)\} \\
                c.wait();
            \end{gather*}
            Donde en la precondición de la operación \verb|c.wait();| hemos necesitado comprobar que $IM$ se seguía compliendo.

            Y para poder seguir, hemos de buscar la precondición de la operación \verb|c.signal();| asociada a la misma variable condición, con lo que comenzamos a demostrar el procedimiento \verb|V|:
            \begin{gather*}
                \{IM\} \equiv \{np = \min(na,nv)\} \\
                nv = nv + 1; \\
                \{np = \min(na,nv-1)\} \\
                \texttt{if\ } (na > np) \texttt{\ then\ } \\
                \{na > np \land np = \min(na,nv-1)\} \rightarrow \{na > np \land np = nv-1\}
            \end{gather*}
            Luego tenemos ya la poscondición de la operación \verb|c.wait();|, con lo que podemos volver por donde íbamos:
            \begin{gather*}
                \{IM\} \equiv \{np = \min(na,nv)\} \\
                na = na + 1;\\
                \{np = \min(na-1,nv)\} \\
                \texttt{if\ }(na > nv) \texttt{\ then\ } \\
                \{na > nv \land np = \min(na-1,nv)\} \rightarrow \{na-1\geq nv \land np = \min(na-1,nv)\} \rightarrow \\ 
                \rightarrow\{np = nv\} \rightarrow \{np = \min(na,nv)\} \\
                c.wait(); \\
                \{na > np \land np = nv - 1\} \\
                \texttt{else do} \\
                \{na \leq nv \land np = \min(na-1,nv)\} \rightarrow \{na-1 < nv \land np = \min(na-1,nv)\} \rightarrow \\
                \rightarrow \{na-1 < nv \land np = na-1\} \rightarrow \{np < nv \land np = na-1\} \\
                \texttt{null;} \\
                \{np < nv \land np = na -1\} \\
                \texttt{endif}
            \end{gather*}
            Como las poscondiciones de cada bloque del \verb|if| son distintas, debemos relajarlas para buscar dos poscondiciones iguales, para poder aplicar la regla del \verb|if|. Notemos que:
            \begin{gather*}
                \{na > np \land np = nv-1\} \rightarrow \{na > nv-1 \land np = nv-1\} \rightarrow \\ \rightarrow \{na \geq nv \land np = nv - 1\} \\
                \{np < nv \land np = na-1\} \rightarrow \{na-1 < nv \land np = na-1\} \rightarrow\\\rightarrow \{na \leq nv \land np = na-1\}
            \end{gather*}
            Podemos por tanto, relajar ambas poscondiciones a la poscondición
            \begin{equation*}
                \{np+1 = \min(na,nv)\}
            \end{equation*}
            Con lo que ahora sí podemos aplicar la regla del \verb|if|, con lo que podemos finalizar la demostración del triple del procedimiento \verb|P|:
            \begin{gather*}
                \{IM\} \equiv \{np = \min(na,nv)\} \\
                na = na + 1;\\
                \{np = \min(na-1,nv)\} \\
                \texttt{if\ }(na > nv) \texttt{\ then\ } \\
                \{np = \min(na,nv)\} \\
                c.wait(); \\
                \{np+1 = \min(na,nv)\} \\
                \texttt{else do} \\
                \{np+1 = \min(na,nv)\} \\
                \texttt{null;} \\
                \{np+1 = \min(na,nv)\} \\
                \texttt{endif} \\
                \{np+1 = \min(na,nv)\} \\
                np = np + 1; \\
                \{np = \min(na,nv)\} \equiv \{IM\}
            \end{gather*}
        \item Para probar ahora el triple del procedimiento \verb|V|:
            \begin{gather*}
                \{IM\} \equiv \{np = \min(na,nv)\} \\
                nv = nv + 1; \\
                \{np = \min(na,nv-1)\} \\
                \texttt{if\ } (na > np) \texttt{\ then\ } \\
                \{na > np \land np = \min(na,nv-1)\} \rightarrow \{na > np \land np = nv-1\}\\
                c.signal(); \\
                \{np = \min(na,nv)\} \\
                \texttt{else do} \\
                \{na \leq np \land np = \min(na,nv-1)\}\rightarrow\{np = na\} \rightarrow \{np = \min(na,nv)\} \\
                \texttt{null;} \\
                \{np = \min(na,nv)\} \\
                \texttt{endif} \\
                \{np = \min(na,nv)\} \equiv \{IM\}
            \end{gather*}
    \end{enumerate}
    Con lo que tenemos provado que $IM$ es un IM\@.
\end{ejercicio*}

\subsection{Regla de concurrencia para programas con monitores}
Consideramos un programa concurrente en el que tenemos $n$ procesos ejecutándose que podemos representar como triples de Hoare ciertos $\{P_i\}\ S_i\ \{Q_i\}$ con \newline $i \in \{1, \ldots, n\}$ de forma que ninguna variable en $P_i$ o en $Q_i$ es modificada por ningún $S_j$ con $i\neq j$. Si en dicho código tenemos $m$ monitores de forma que para cada uno hemos conseguido probar un IM, $IM_k$ con $1 \leq k \leq m$, entonces podemos aplicar la \textbf{regla de concurrencia para programas con monitores}:
\begin{equation*}
    \dfrac{\{P_i\}\ S_i\ \{Q_i\} \quad 1 \leq i \leq n}{\begin{array}{c}
            \{MI_1 \land \ldots \land MI_m \land P_1 \land \ldots\land P_n\} \\
            cobegin\ S_1\ ||\ S_2\ ||\ \ldots\ ||\ S_n\ coend \\
            \{MI_1 \land \ldots \land MI_m \land Q_1 \land  \ldots\land Q_n\} 
        \end{array}}
\end{equation*}
Obteniendo así la verificación de nuestro programa concurrente.

\section{Patrones de uso de un monitor}
Al programar programas concurrentes que nos resuelvan un problema, encontramos muchas veces ciertos pequeños problemas a resolver que se repiten a menudo. En esta sección, destacamos tres de estos problemas, describiéndolos, planteando una solución mediante un monitor a utilizar en ellos y demostrando que el monitor realiza el funcionamiento esperado.

\subsection{Espera única}
Puede suceder que en un programa concurrente queramos que un proceso espere a que otro proceso haya realizado una cierta acción para seguir con su ejecución. Llamaremos al proceso que tiene que esperar al otro \textit{consumidor} y a dicho otro \textit{productor}.\\

El problema se resuelve de forma muy sencilla, con una variable compartida que indique si el productor ha realizado ya su acción por la que el consumidor debe esperar o si no lo ha hecho todavía. Cuando el consumidor se acerque a la zona en la que debe esperar al productor, consultará la variable compartida y en caso de que esta indique que no se ha realizado la acción, bloquearemos al proceso. Si se ha realizado la acción, no haremos nada.

Además, cuando el productor haya realizado la acción que ha de realizar, cambiaremos el valor de dicha variable compartida, indicando que ya se ha realizado la acción. En caso de que el consumidor se haya bloqueado antes de realizar la acción, lo desbloquearemos.\\

Observemos que estamos haciendo uso de una variable compartida, que es modificada por un proceso. Debemos por tanto acceder a ella en exclusión mutua. Esto es garantizado por el uso del monitor.

\subsubsection{Monitor a usar}
El monitor que usaremos tendrá dos procedimentos exportables, uno llamado \verb|esperar| que será el que use el proceso consumidor, y otro llamado \verb|notificar|, que será el que use el proceso productor para avisar al consumidor de que ya ha realizado la acción.

De esta forma, podemos ver el código monitor en pseudo código:
\begin{minted}[escapeinside=\#\#]{pascal}
    monitor EU;
    var terminado : boolean;  {si terminado = true, se ha realizado la acción}

        {variable auxiliar para la demostracion}
        autorizado : boolean; {si autorizado = true, consumidor puede ejecutarse}

        c : cond;

       procedure esperar(); begin
          if (not terminado) then
             c.wait();
          autorizado = true;
       end

       procedure notificar(); begin
          terminado = true; 
          c.signal();
       end

       begin {codigo de inicializacion}
          terminado = false; autorizado = false;
       end
\end{minted}

Para su demostración, usaremos el invariante
\begin{equation*}
    \{IM\} \equiv \{terminado = false \Longrightarrow autorizado = false\}
\end{equation*}
\begin{proof}
    Hemos pues de demostrar que se cumplen los triples:
    \begin{gather*}
        \{V\}\ codigo\ inicializacion\ \{IM\} \\
        \{IM\}\ \texttt{esperar}\ \{IM\} \\
        \{IM\}\ \texttt{notificar}\ \{IM\} 
    \end{gather*}
    Comenzamos por el primero:
    \begin{equation*}
        \{V\}\ terminado = false;\ autorizado=false;\ \{IM\}
    \end{equation*}
    Para ello, hemos de aplicar la regla de la composición sobre los triples:
    \begin{equation*}
        \{V\}\ terminado=false;\ \{terminado = false\} \\
    \end{equation*}
    \begin{multline*}
        \{terminado = false\}\ autorizado=false;\ \\ \{terminado = false \land autorizado = false\} \rightarrow \{IM\}
    \end{multline*}
    Que son ambos ciertos, el primero por el axioma de asignación y el segundo por el axioma de asignación y la primera regla de la consecuencia.\\

    A continuación, probaremos los triples de los procedimientos del monitor, recordando que para hallar la poscondición de \verb|c.wait()| hemos primero de localizar la precondición de \verb|c.signal()|, por lo que hemos de realizar las dos demostraciones ``al mismo tiempo'', aunque a continuación escribiremos primero la de \verb|notificar| y luego la de \verb|esperar|:
    \begin{gather*}
        \texttt{notificar(); begin} \\
        \{IM\} \\
        \texttt{terminado = true;} \\
        \{terminado = true\} \equiv \{C\} \\
        \texttt{c.signal();} \\
        \{IM\} \\
        \texttt{end}
    \end{gather*}
    Con lo que ya sabemos que la condición de sincronización de \verb|c| es: 
    \begin{equation*}
        \{C\}\equiv \{terminado = true\}
    \end{equation*}
    Demostramos ahora $\{IM\}\ \texttt{esperar}\ \{IM\}$:
    \begin{gather*}
        \texttt{esperar(); begin} \\
        \{IM\} \\
        \texttt{if not terminado then} \\
        \{IM \land terminado = false\} \rightarrow \{terminado = false \land autorizado = false\} \\
        \texttt{c.wait();} \\
        \{terminado = true\} \\
        \texttt{else} \\
        \{IM \land terminado = true\} \rightarrow \{terminado = true\} \\
        \texttt{null;} \\
        \{terminado = true\} \\
        \texttt{end} \\
        \{terminado = true\} \\
        \texttt{autorizado = true;} \\
        \{terminado = true \land autorizado = true\} \rightarrow \\ 
        \rightarrow \{autorizado = true \Longrightarrow terminado = true\} \equiv \{IM\}  \\
        \texttt{end}
    \end{gather*}
\end{proof}

\subsection{Exclusión mutua}
Es muy habitual que en programas concurrentes tengamos una o varias regiones de código que queramos que se ejecuten en exclusión mutua, llamadas secciones críticas. Es decir, mientras que un proceso se encuentre ejecutando una sección crítica, ningún otro proceso podrá estar ejecutando a la vez la misma sección crítica\footnote{Podemos tener dos secciones críticas con distintos códigos pero que sean referidas al acceso para la misma variable compartida. En dicho caso, pensamos que las secciones críticas son iguales, ya que solo puede haber un proceso que ejecute una u otra a la vez.}.\\

Usualmente, querremos tener exclusiones mutuas cuando varios procesos de un programa hagan uso de un recurso compartido, tal como una variable compartida, una salida a un fichero o imprimir información en un entorno gráfico.

\subsubsection{Monitor a usar}
El monitor que resuelve el problema de la exclusión mutua tendrá dos procedimientos exportables, \verb|entrar|, que se ejecutará antes de cualquier sección crítica, y \verb|salir|, que se ejecutará al final de cada sección crítica.

De esta forma, tenemos el monitor:
\begin{minted}[escapeinside=\#\#]{pascal}
    monitor EM;
    var ocupada : boolean;  {ocuapda = true si hay un proceso en seccion critica}
        cola : cond;

       procedure entrar(); begin
          if ocupada then
             cola.wait();
          ocupada = true;
       end

       procedure salir(); begin
          ocupada = false;
          cola.signal();
       end

       begin
          ocupada = false;
       end
\end{minted}
Para demostrarlo, primero definiremos la variable $num_{sc}$ como el número de procesos que se encuentran ejecutando la sección crítica. Una vez definido, el invariante a usar será
\begin{equation*}
    \{IM\} \equiv \{(ocupada = false \Longleftrightarrow num_{sc} = 0) \land 0\leq num_{sc} \leq 1\}
\end{equation*}
La demostración se deja como ejercicio al lector.
% // TODO: Hacer la demostración y quitar lo de que se deja como ejercicio

\subsection{Productores/Consumidores}
Volvemos otra vez al paradigma del productor/consumidor, el cual hemos explicado varias veces ya en este documento. Ahora, admitiremos la existencia de varios procesos productores que querrán generar datos y escribirlos en una variable compartida, así como varios consumidores, que querrán leer dicha variable compartida y realizar los cálculos pertinentes.

\subsubsection{Monitor a usar}
Usaremos un monitor con dos procedimientos: \verb|escribir|, que permitirá escribir en la variable compartida el valor indicado, y \verb|leer|, que permitirá leer el valor de la variable compartida. La ventaja de usar un monitor es que los códigos de sincronización solo los tenemos que realizar en el monitor, dejando limpios los códigos de los procesos.

Como tenemos dos condiciones de sincronización, bloquear a productores que quieran escribir en una variable que no se ha leído, o bloquear a consumidores que quieran leer de una variable cuyo valor ya ha sido leído, necesitaremos dos variables de tipo \verb|cond|:
\begin{minted}[escapeinside=\#\#]{pascal}
    monitor PC;
    var valor : integer;  {variable compartida a usar}
        pendiente : boolean;  {si pendiente = true, valor escrito y no leido}
        cola_prod, cola_cons : cond;

       procedure escribir(v : integer); begin
          if pendiente then
             cola_prod.wait();
          valor = v;
          pendiente = true;
          cola_cons.signal();
       end

       procedure leer() : integer; begin
          if (not pendiente) then
             cola_cons.wait();
          result = valor;
          pendiente = false;
          cola_prod.signal();
       end

       begin
          pendiente = false;
       end
\end{minted}
Para la demostración, debemos definir primero:
\begin{align*}
    E &= \text{\ número\ de\ llamadas\ a\ escribir\ \textbf{completadas}.} \\
    L &= \text{\ número\ de\ llamadas\ a\ leer\ \textbf{completadas}.} \\
\end{align*}
De esta forma, podemos definir el invariante
\begin{gather*}
    \{IM\} \equiv 
    \left\{E-L = \left\{\begin{array}{rl}
                0 & \text{\ si\ } pendiente = false \\
                1 & \text{\ si\ } pendiente = true \\
    \end{array}\right\}\right\} \equiv  \\
    \equiv \{(pendiente = false \land E-L=0) \lor (pendiente = true \land E-L=1)\}
\end{gather*}
La demostración se deja como ejercicio para el lector.
% // TODO: Hacer la demostración y quitar lo de que se deja como ejercicio

Asímismo, notemos que la demostración es similar a la del monitor para exclusión mutua, teniendo en cuenta que:
\begin{align*}
    E - L &= num_{sc} \\
    pendiente &= \lnot libre
\end{align*}

\begin{ejercicio*}
    Si ahora los productores no escriben sobre una misma variable compartida sino sobre un buffer (por ejemplo, un array con planificación FIFO), plantear un monitor que solucione el problema de los productores/consumidores, así como desmostrar que dicho monitor funciona correctamente.\newline
    (\textbf{Pista:} para la demostración, sustituir en el IM ``1'' por el tamaño del buffer)
\end{ejercicio*}

\section{Semánticas de señales}
Como comentamos ya al inicio del Capítulo, las operaciones \verb|signal| de las variables condición son operaciones delicadas, ya que son ejecutadas por un proceso que se encuentra ejecutando algún procedimiento del monitor y que lo que hacen es desbloquear a algún proceso se que se encontraba ejecutando código del monitor, por lo que (si no hacemos nada), tendremos dos procesos distintos ejecutando a la vez procedimientos (podría ser el mismo procedimiento) de un mismo monitor, algo que no puede darse.\\

Para solucionar el problema que nos plantea la operación \verb|signal|, plantearemos distintas \textit{semánticas de señales} \verb|signal|. Esto es, plantearemos varios paradigmas en los que la señal \verb|signal| tendrá un significado u otro, de forma que su finalidad sea siempre \textit{sacar al primero proceso de la cola de la variable condición} en cuestión y ponerlo en otro sitio que permita que dicho proceso entre al monitor en algún futuro próximo, garantizando la propiedad de vivacidad de que dicho proceso en algún momento volverá a entrar al monitor.\\

Además, clasificaremos los distintos tipos de semánticas de señales que veremos en \textbf{no desplazantes} y \textbf{desplazantes}.

\begin{definicion}
    Diremos que \textbf{una semántica de señal es desplazante} si, siempre que un proceso ejecute una operación \verb|signal| sobre una variable condición \verb|c|, en dicho instante cederá el monitor al primer proceso de la cola de bloqueados de la variable condición \verb|c| sin que el monitor quede libre en ningún momento\footnote{Evitando que entre al monitor cualquier otro proceos de la cola de entrada al monitor.}. Además, debe garantizarse la propiedad de vivacidad de que el proceso señalador volverá a entrar al monitor en algún momento.
\end{definicion}

Veremos ahora todas las posibles semánticas de señales que podemos encontrarnos, entendiendo que el proceso \textit{señalador} es aquel que ejecuta la operación \verb|signal| sobre una variable condición; y que el proceso \textit{señalado} es aquel que estaba primero en la cola de bloqueados de la misma variable condición.

\subsection{Señalar y Continuar (SC)}
El proceso señalador no se bloquea tras ejecutar \verb|signal|, sino que sigue con la ejecución del procedimiento en cuestión. El proceso señalado se bloquea hasta que se pueda adquirir de nuevo el monitor.\\

Como podemos ver, se trata de una semántica no desplazante, ya que el proceso señalador no se bloquea tras ejecutar la instrucción \verb|signal|. En relación al proceso señalado, se produce una \textit{competición} contra el resto de procesos que esperan en la cola de entrada al monitor. Esta ``competición'' que hemos mencionado depende de la implementación que se haga, destacando dos posibilidades:
\begin{enumerate}
    \item El proceso señalado pasa al final de la cola de entrada al monitor.
    \item Después de que el proceso señalador deje libre el monitor (ya sea porque termina el procedimiento o se bloquea debido a una instrucción \verb|wait|), se desbloquea al proceso señalado y al primero de la cola de entrada al monitor, se sucede una condición de carrera entre ambos y el vencedor (el que primero llega al monitor), acaba ejecutándolo. El perdedor acaba al final de la cola de entrada al monitor.

        Como podemos entender, la primera opción es mejor, ya que en la segunda puede suceder que un proceso pierda varias veces la competición por el monitor, incumpliendo la condición de equidad entre procesos; además de que no sabemos qué sucede tras desbloquear a dicho proceso, debido a que la competición es transparente para el programador.
\end{enumerate}
De esta forma, podemos representar la vida de un proceso que forma parte de un programa concurrente que usa monitores con semántica de señales de señalar y continuar con el diagrama de la Figura~\ref{fig:sc}.

% // TODO: Ver si fijar o no estos 4 grafos
\begin{figure}%[H]
\centering
\begin{tikzpicture}[node/.style={rectangle, draw, minimum size=1cm, align=center},
                    edge/.style={-stealth}]
    \node[node] (A) {Procedimiento\\terminado};
    \node[node, left =of A] (B) {\begin{tabular}{c}Ejecutando código\\del proceso\\(fuera del monitor)\end{tabular}};
    \node[node, left =of B] (C) {\begin{tabular}{c}Llamando a\\procedimiento\\del monitor\end{tabular}};
    \node[node, below =of C] (D) {\begin{tabular}{c}Esperando en\\cola de entrada\\al monitor\end{tabular}};
    \node[node, right =of D] (E) {\begin{tabular}{c}Monitor\\libre\end{tabular}};
    \node[node, right =of E, xshift=.6cm] (F) {\begin{tabular}{c}Ejecutando procedimiento\\(dentro del monitor)\end{tabular}};

    \node[node, below right =of F] (G) {\texttt{c.signal()}};
    \node[node, below =of F] (H) {\texttt{c.wait()}};
    \node[node, left =of H] (I) {\begin{tabular}{c}Bloqueado en una\\cola condición $c$\end{tabular}};
    \node[node, left =of I, xshift=-1.3cm] (J) {Señalado};
    
    \draw[edge] (A) -- (B);
    \draw[edge] (B) -- (C);
    \draw[edge] (C) -- (D);
    \draw[edge] (D) -- (E);
    \draw[edge] (E) -- (F);
    \draw[edge] (F) -- (A);

    % \draw[edge] (F) to[out=0, in=100] (G);
    % \draw[edge] (G) to[out=180, in=-50] (F);

    \draw[edge] (F) to[out=0, in=90] (G);
    \draw[edge] (G) to[out=180, in=-90] ([xshift=1.5cm]F.south);

    \draw[edge] (F) -- (H);
    \draw[edge] (H) -- (I);
    \draw[edge] (I) -- (J);
    \draw[edge] (J) -- (D);
\end{tikzpicture}
\caption{Vida de un proceso en un programa con monitores con semántica SC.}
\label{fig:sc}
\end{figure}

\subsubsection{A tener en cuenta para bloquear}
Como se trata de una semántica no desplazante, debemos tener cuidado con un detalle:

En semántica desplazante, bloqueamos a un proceso porque se acerca a un estado inseguro del programa. Cuando nos alejemos de dicho estado, realizaremos la instrucción \verb|signal|, por lo que el proceso que bloqueamos se podrá ejecutar al habernos alejado del estado inseguro del programa.\\

Por otra parte, ahora también desbloquearemos al proceso por alejarnos de dicho estado inseguro, pero el proceso señalado no se ejecutará tras esto, por lo que puede que se ejecuten procesos en medio que vuelvan a acercarse a este estado inseguro, luego tendremos que volver a comprobar si estamos cerca o no de dicho estado inseguro.\\

\noindent
Por tanto, en semánticas se señales no desplazantes, en vez de bloquear a procesos con:
\begin{minted}[escapeinside=\#\#]{pascal}
    if (cerca de estado inseguro) then
       c.wait();
\end{minted}
Tendremos que plantear el código
\begin{minted}[escapeinside=\#\#]{pascal}
    while (cerca de estado inseguro) do begin
       c.wait();
    end
\end{minted}
Ya que cuando el proceso señalado vuelva no sabremos si la condición por la que lo bloqueamos es cierta o no. Puede suceder que bloqueemos a un proceso por acercarse a un estado inseguro, que lo desbloqueemos cuando el estado del programa se aleje de dicho estado, que se ejecuten procesos que se acerquen a dicho estado y que cuando el proceso que fue señalado entre al monitor, se encuentre cerca del mismo estado inseguro por el que tuvo que bloquearse, teniendo que hacerlo nuevamente.

\subsection{Señalar y Salir (SS)}
El proceso señalador \textbf{finaliza} la ejecución de su procedimiento tras la ejecución de una instrucción \verb|signal|. El proceso señalado se desbloquea posteriormente y en todo este tiempo el monitor no queda libre en ningún momento. Se trata, por tanto, de una semántica desplazante, ya que el proceso señalador cede el monitor al proceso señalado.\\

Notemos que en señalar y salir, el proceso señalador \textbf{finaliza} la ejecución de su procedimiento. Es decir, cualquier instrucción que haya tras una operación \verb|signal| no se ejecutará nunca. Obliga por tanto a una disciplina de programación en la que si queremos realizar un \verb|signal| dentro de un procedimiento, esta instrucción debe ser la última dentro del procedimiento, para poder ejecutar antes todas las instrucciones que deseemos. 

Si recordamos el ejemplo de la Sección~\ref{sec:concepto_de_monitor}, observamos que en este las instrucciones \verb|signal| se encuentran al final de los procedimientos de forma natural. Por tanto, la semántica SS funciona bien para este monitor.\\

Al igual que hicimos para SC, podemos representar la vida de un proceso que forma parte de un programa concurrente que usa monitores con semántica de señales de señalar y salir con el diagrama de la Figura~\ref{fig:ss}.
\begin{figure}%[H]
\centering
\begin{tikzpicture}[node/.style={rectangle, draw, minimum size=1cm, align=center},
                    edge/.style={-stealth}]
    \node[node] (A) {Procedimiento\\terminado};
    \node[node, left =of A] (B) {\begin{tabular}{c}Ejecutando código\\del proceso\\(fuera del monitor)\end{tabular}};
    \node[node, left =of B] (C) {\begin{tabular}{c}Llamando a\\procedimiento\\del monitor\end{tabular}};
    \node[node, below =of C, yshift=-1.5cm] (D) {\begin{tabular}{c}Esperando en\\cola de entrada\\al monitor\end{tabular}};
    \node[node, right =of D] (E) {\begin{tabular}{c}Monitor\\libre\end{tabular}};
    \node[node, right =of E, xshift=.6cm] (F) {\begin{tabular}{c}Ejecutando procedimiento\\(dentro del monitor)\end{tabular}};

    \node[node, below =of B] (G) {\texttt{c.signal()}};
    \node[node, below right =of F] (H) {Señalado};
    \node[node, below left =of F] (I) {\texttt{c.wait()}};
    \node[node, below =of F, yshift=.1cm] (J) {\begin{tabular}{c}Bloqueado en una\\cola condición $c$\end{tabular}};

    \draw[edge] (A) -- (B);
    \draw[edge] (B) -- (C);
    \draw[edge] (C) -- (D);
    \draw[edge] (D) -- (E);
    \draw[edge] (E) -- (F);
    \draw[edge] (F) -- (A);

    \draw[edge] (F) -- (G);
    \draw[edge] (G) -- (B);

    \draw[edge] (F) -- (I);
    \draw[edge] (I) -- (J);

    \draw[edge] (H) -- (F);
    \draw[edge] (J) -- (H);
\end{tikzpicture}
\caption{Vida de un proceso en un programa con monitores con semántica SS.}
\label{fig:ss}
\end{figure}

\subsection{Señalar y Esperar (SE)}
El proceso señalador se bloquea al final de la cola de entrada al monitor y cede el monitor al proceso señalado. El monitor no queda libre en ningún momento. Se trata de otro tipo de señal con semántica desplazante.\\

Puede considerarse una semántica \textit{injusta}, ya que cada vez que un proceso ejecuta la operación \verb|signal|, debe irse al final de la cola de entrada al monitor, teniendo que esperar entre todos los procesos nuevamente (el proceso probablemente tuvo ya que esperar para entrar al monitor para ejecutar el procedimiento que contenía la operación \verb|signal|).

Podemos representar la vida de un proceso que forma parte de un programa concurrente que usa monitores con semántica de señales de señalar y esperar con el diagrama de la Figura~\ref{fig:se}.

\begin{figure}%[H]
\centering
\begin{tikzpicture}[node/.style={rectangle, draw, minimum size=1cm, align=center},
                    edge/.style={-stealth}]
    \node[node] (A) {Procedimiento\\terminado};
    \node[node, left =of A] (B) {\begin{tabular}{c}Ejecutando código\\del proceso\\(fuera del monitor)\end{tabular}};
    \node[node, left =of B] (C) {\begin{tabular}{c}Llamando a\\procedimiento\\del monitor\end{tabular}};
    \node[node, below =of C] (D) {\begin{tabular}{c}Esperando en\\cola de entrada\\al monitor\end{tabular}};
    \node[node, right =of D] (E) {\begin{tabular}{c}Monitor\\libre\end{tabular}};
    \node[node, right =of E, xshift=.6cm] (F) {\begin{tabular}{c}Ejecutando procedimiento\\(dentro del monitor)\end{tabular}};

    \node[node, below =of F, yshift=-1.5cm] (J) {\begin{tabular}{c}Bloqueado en una\\cola condición $c$\end{tabular}};
    \node[node, right =of J] (H) {Señalado};
    \node[node, left =of J] (I) {\texttt{c.wait()}};

    \node[node, below =of E, yshift=.3cm] (G) {\texttt{c.signal()}};

    \draw[edge] (A) -- (B);
    \draw[edge] (B) -- (C);
    \draw[edge] (C) -- (D);
    \draw[edge] (D) -- (E);
    \draw[edge] (E) -- (F);
    \draw[edge] (F) -- (A);

    \draw[edge] ([xshift=-2cm]F.south) to[out=-90, in=0] (G);
    \draw[edge] (G) to[out=180, in=-90] (D);

    \draw[edge] (F) -- (I);
    \draw[edge] (I) -- (J);

    \draw[edge] (H) -- (F);
    \draw[edge] (J) -- (H);
\end{tikzpicture}
\caption{Vida de un proceso en un programa con monitores con semántica SE.}
\label{fig:se}
\end{figure}

\subsection{Señalar y Espera Urgente (SU)}
El proceso señalador se bloquea en una nueva cola del monitor llamada \textit{cola de procesos urgentes}, cediendo el monitor al proceso señalado. El monitor no queda libre en ningún momento, por lo que se trata de una señal con semántica desplazante.\\

La nueva cola de procesos urgentes actúa como una nueva cola de entrada al monitor, pero con mayor preferencia que la cola de entrada que ya teníamos. De esta forma, es similar a la semántica de señalar y esperar pero sin ser tan \textit{injusta}, ya que da preferencia a los procesos que se bloquearon tras ejecutar un \verb|signal| y deja en segunda posición a los proceos que esperan para ejecutar procedimientos del monitor.\\

Esta semántica modifica la última representación gráfica que teníamos de un monitor en la Tabla~\ref{esq:monitor_3}, dejándola finalmente en la que observamos en la Tabla~\ref{esq:monitor_4}.

\begin{table}[H]
\centering
\begin{tabular}{|l|}
\hline
Cola del\\
monitor \\
\hline
Cola de\\
urgentes \\
\hline
Variables \\
permanentes \\
\hline
Variables \\
condición y\\
colas de procesos\\
bloqueados\\
\hline
Procedimientos \\
exportados \\
\hline
Código de \\ 
inicialización \\
\hline
\end{tabular}
\caption{Esquema de un monitor incluyendo la cola de urgentes.}
\label{esq:monitor_4}
\end{table}

Podemos representar la vida de un proceso que forma parte de un programa concurrente que usa monitores con semántica de señalar y espera urgente con el diagrama de la Figura~\ref{fig:su}.\\

Finalmente, recordemos que con semántica SU, si la cola de una variable condición está vacía y realizamos un \verb|signal| sobre la misma, será equivalente a una operación nula, con lo que en este caso el proceso señalador no se irá a la cola de bloqueados.

\begin{figure}%[H]
\centering
\begin{tikzpicture}[node/.style={rectangle, draw, minimum size=1cm, align=center},
                    edge/.style={-stealth}]
    \node[node] (A) {Procedimiento\\terminado};
    \node[node, left =of A] (B) {\begin{tabular}{c}Ejecutando código\\del proceso\\(fuera del monitor)\end{tabular}};
    \node[node, left =of B] (C) {\begin{tabular}{c}Llamando a\\procedimiento\\del monitor\end{tabular}};
    \node[node, below =of C] (D) {\begin{tabular}{c}Esperando en\\cola de entrada\\al monitor\end{tabular}};
    \node[node, right =of D] (E) {\begin{tabular}{c}Monitor\\libre\end{tabular}};
    \node[node, right =of E, xshift=.6cm] (F) {\begin{tabular}{c}Ejecutando procedimiento\\(dentro del monitor)\end{tabular}};

    \node[node, below =of D] (G) {\texttt{c.wait()}};
    \node[node, below =of G] (H) {\begin{tabular}{c}Bloqueado en una\\cola condición $c$\end{tabular}};
    \node[node, right =of H] (I) {Señalado};

    \node[node, below =of F, xshift=4cm] (J) {\begin{tabular}{c}Monitor\\libre\end{tabular}};
    \node[node, below =of J] (K) {\begin{tabular}{c}Esperando en\\cola de urgentes\end{tabular}};
    \node[node, left =of K] (L) {\texttt{c.signal()}};

    \draw[edge] (A) -- (B);
    \draw[edge] (B) -- (C);
    \draw[edge] (C) -- (D);
    \draw[edge] (D) -- (E);
    \draw[edge] (E) -- (F);
    \draw[edge] (F) -- (A);

    \draw[edge] ([xshift=-2cm]F.south) to[out=-90, in=0] (G);
    \draw[edge] (G) -- (H);
    \draw[edge] (H) -- (I);
    \draw[edge] (I) to[out=0, in=-90] ([xshift=-1cm]F.south);

    % \draw[edge] (F) to[out=0, in=90] (J);
    % \draw[edge] (J) -- (K);
    % \draw[edge] (K) -- (L);
    % \draw[edge] (L) -- (F);

    \draw[edge] (J) to[out=90, in=0] (F);
    \draw[edge] (K) -- (J);
    \draw[edge] (L) -- (K);
    \draw[edge] (F) -- (L);
\end{tikzpicture}
\caption{Vida de un proceso en un programa con monitores con semántica SU.}
\label{fig:su}
\end{figure}

\subsection{Comparativa}\label{cap:comp}
Antes de comparar las semánticas de señales que ya hemos descrito, destacamos un último tipo de semántica de señales, las \textit{señales automáticas} (SA). En estas, el programador programa las operaciones \verb|wait| y es el compilador quien analiza los códigos programados y decide cuándo desbloquear a los procesos, por lo que se trata de unas señales implícitas (ya que en ningún moemnto se especifica cuándo hacer una instrucción \verb|signal|). 

A pesar de que este tipo de semántica aparece en la bibliografía, no aparece en ningún lenguaje de programación conocido, por lo que no tendrá relevancia en la asignatura ni en la comparativa que ahora realizaremos.\\

En primer lugar, hemos de destacar que cualquiera de las primeras 4 semánticas vistas es capaz de resolver cualquier problema, por lo que nos guiaremos por la sencillez de uso de cada una y la eficiencia que cada una nos aporta a los programas concurrentes para elegir una u otra semántica.
\begin{itemize}
    \item En cuato a sencillez de uso, SC, SE y SU presentan la misma facilidad, mientras que SS nos condiciona a un paradigma de programación en el que las instrucciones \verb|signal| sean las últimas que se ejecuten en los procedimientos de un monitor que contenga alguna operación \verb|signal|.
    \item En cuanto a eficiencia:
        \begin{itemize}
            \item Las semánticas SE y SU resultan ineficientes cuando las instrucciones \verb|signal| se encuentran al final de los procedimientos (ya que habrá procesos que ejecutan un \verb|signal|, se bloquearán, esperarán en la cola correspondiente de entrada al monitor, y cuando por fin tengan acceso a él no realizarán nada, sino que simplemente terminarán la ejecución del procedimiento), por lo que se saturará la cola de entrada al monitor y se realizarán cambios de contexto de formas innecesarias.
            \item La semántica SC es poco eficiente, al tener que obligarnos a usar un bucle de comprobación para comprobar si estamos cerca de una situación insegura cada vez que entramos al monitor, como consecuencia de no tener semántica desplazante, pudiendo producirse robos de señal.
        \end{itemize}
        Por tanto, si no nos resulta incómodo tener que colocar todas las instrucciones \verb|signal| como última instrucción de los procedimientos en los que aparecen, SS será probablemente la semántica de señales más eficientes.
\end{itemize}

Para concluir la comparativa de las semánticas de señales, veremos el siguiente ejemplo, que muestra que a veces la semántica elegida condiciona la forma en la que programamos los monitores.

\begin{ejemplo}
    Queremos programar en un programa concurrente una \textit{barrera parcial}.

    A partir del concepto de barrera que ya manejamos en la asignatura de Arquitectura de Computadores\footnote{Consultar los apuntes si no se conoce el concepto.}, ahora no queremos tener a los procesos esperando a que todos los procesos que intervienen en el programa lleguen a un punto en específico, sino solo queremos que lo hagan $n$ procesos.

    De esta forma, cuando el primer proceso llegue al punto de la barrera, este se bloqueará. Estos les sucederá a los primeros $n-1$ procesos en llegar a la barrera. Cuando el proceso $n$-ésimo llegue a esta, se abrirá la barrera, dejando pasar a estos $n$ primeros procesos que llegaron a esta. La barrera no debe dejar pasar al proceso número $n+1$, sino que este deberá esperar al proceso número $2n$ para voler a abrir la barrera.\\

    Una vez descrito el problema a resolver, planteamos el siguiente monitor como primera solución:
    \begin{minted}[escapeinside=\#\#]{pascal}
        Monitor BP;
           var cola : cond;
               contador : integer;

           procedure barrera(); begin
              contador = contador + 1;
              if (contador < n) then
                 cola.wait();
              else begin
                 for i=1 to n-1 do
                    cola.signal();
                 end
                 contador = 0;
              end
              {Funcionalidad tras pasar la barrera parcial}
           end

           begin
              contador = 0;
           end
    \end{minted}
Veamos ahora el comportamiento del monitor, según la semántica de señales que hayamos escogido:
\begin{description}
    \item [Señalar y Continuar.] Cuando llega el $n$-ésimo proceso de un grupo, este desbloquea a los últimos $n-1$ procesos bloqueados, que pasan a competir por el monitor. Puede suceder que algún proceso del siguiente grupo adelante a uno de este grupo, por lo que no sería una solución válida.
    \item [Señalar y Salir.] El $n$-ésimo proceso solo podría despertar al primer proceso que llegó a la barrera (ya que tras ejecutar \verb|signal|, este terminaría la ejecución de su procedimiento), sin reestablecer el contador a 0, por lo que cualquier proceso que pase por la barrera despertaría al siguiente proceso bloqueado, de forma que los últimos $n$ procesos que ejecuten el protocolo \verb|barrera| quedarán bloqueados de forma indefinida.
    \item [Señalar y Esperar.] El $n$-ésimo proceso solo podría despertar al primer proceso que llegó a la barrera, volviendo el proceso señalador a la cola de entrada al monitor. El $n+1$-ésimo proceso solo podría despertar al segundo proceso, y se iría a la cola de entrada al monitor. Cuando el $2n$-ésimo proceso entre al monitor, este ya no ejecutará \verb|signal| (ya que como la cola está vacía, es equivalente a una instrucción nula), poniendo por fin el contador a 0 y comenzando otra vez con dicho comportamiento.

        No es una solución válida.
    \item [Señalar y Espera Urgente.] El $n$-ésimo proceso desbloquea al primer proceso que llegó a la barrera y a continuación se bloquearía en la cola de urgentes, cediendo el monitor al primer proceso, que finalizaría la ejecución del procedimiento y volvería a entrar el $n$-ésimo proceso, por tener la cola de urgentes mayor prioridad que la cola de entrada al monitor. El proceso se repetiría hasta que el $n$-ésimo proceso finalie la ejecución del procedimiento, restaurando la barrera parcial para los siguientes $n$ procesos.

        Sería un funcionamiento correcto de la barrera, pero hace que el $n$-ésimo proceso realice $n-1$ cambios de contexto innecesarios.
\end{description}
Mostramos ahora una versión alternativa para el monitor que resuelve el problema de la barrera parcial:
\begin{minted}[escapeinside=\#\#]{pascal}
    Monitor BP;
       var cola : cond;
           contador : integer;

       procedure barrera(); begin
          contador = contador + 1;
          if (contador < n) then
             cola.wait();
          contador = contador - 1;
          {Funcionalidad tras pasar la barrera parcial}
          if (contador > 0) then
             cola.signal();
       end

       begin
          contador = 0;
       end
\end{minted}
Vemos su comportamiento:
\begin{itemize}
    \item No funciona con la semántica SC\@.
    \item Funciona con cualquier semántica desplazante, ya que tras llegar a $n$ procesos en la barrera, el $n$-ésimo desbloquea al primero, que desbloquea al segundo, que desbloquea al tercero, \ldots, hasta llegar al $n-1$, que realiza su funcionalidad y viceversa hasta el $n$-ésimo proceso, que deja la barrera en el estado inicial para los siguientes grupos de procesos.
\end{itemize}
\end{ejemplo}~\\

En general, tenemos que tener cuidado con la semántica de señales que estemos empleando, sobre todo si las instrucciones \verb|signal| tienen instrucciones detrás. Además, la semántica SC suele complicar como norma general los diseños de los monitores.

\begin{ejercicio*}
    Determinar para qué tipo de semánticas (SC o desplazantes) funcionan los siguientes monitores que tratan de simular el comportamiento de un semáforo:
    \setlength{\columnsep}{1cm}
    \begin{multicols}{2}
        \begin{minted}[escapeinside=\#\#]{pascal}
            Monitor semaforo_FIFO1;
            var c : cond;
                s : int;

            procedure P;
            begin
                if (s = 0) then
                    c.wait();
                s := s - 1;
            end;

            procedure V;
            begin
                s := s + 1;
                c.signal();
            end

            begin
                s := 0;
            end
        \end{minted}
        \begin{minted}[escapeinside=\#\#]{pascal}
            Monitor semaforo_FIFO2;
            var c : cond;
                s : int;

            procedure P;
            begin
                while (s = 0) do
                    c.wait();
                end do;
                s := s - 1;
            end;

            procedure V;
            begin
                c.signal();
                s := s + 1;
            end

            begin
                s := 0;
            end
        \end{minted}
    \end{multicols}
    En primer lugar, notamos que el monitor de la izquierda realiza una instrucción \verb|if| antes de llamar a la operación \verb|wait|, luego solo sirve para semánticas desplazantes.

    Posteriormente, observamo que el monitor de la derecha realiza la operación \verb|wait| dentro de un bucle, por lo que este sirve para semánticas SC\@. Sin embargo, no sirve para semánticas desplazantes, ya que la operación \verb|signal| se realiza antes de incrementar la variable \verb|s|, por lo que nunca se realizaría dicho incremento, al bloquear al proceso señalador.
\end{ejercicio*}

\subsection{Axiomas para operaciones de sincronización no desplazantes}
Como comentamos ya en la Sección~\ref{sec:ax_desplazantes}, los axiomas que sabemos para demostrar las operaciones de sincronización en monitores solo aplican cuando trabajamos con señales con semánticas desplazantes. Por tanto, si nos encontramos trabajando con semánticas de señales no desplazantes como SC, necesitamos unas nuevas reglas a aplicar.

\begin{description}
    \item [Axioma de la operación \texttt{wait}.]~\\
        Sea $c$ una variable de tipo condición, $L$ un invariante local del procedimiento en el que nos encontremos y $IM$ el invariante del monitor que usamos para demostrar la corrección del mismo, se verifica que
        \begin{equation*}
            \{IM \land L\}\ c.wait();\ \{IM \land L\}
        \end{equation*}
        Es decir, antes y después de la operación \verb|wait| solo podemos asegurar que se cumple el IM (ya que podrían producirse robos de señal), por lo que para asegurarnos de que estamos lejos de un estado inseguro del programa, debemos hacer uso de un bucle \verb|while|, tal y como expusimos anteriormente:
        \begin{minted}[escapeinside=\#\#]{pascal}
            while (cerca de estado inseguro) do begin
               c.wait();
            end
        \end{minted}
    \item [Axioma de la operación \texttt{signal}.]~\\
        Como estamos ante una semántica no desplazante, el proceso señalador seguirá ejecutando el procedimiento que estaba ejecutando cuando ejecutó la operación \verb|signal|, por lo que nada habrá cambiado:
        \begin{equation*}
            \{P\}\ c.signal();\ \{P\}
        \end{equation*}
        Por tanto, la operación \verb|signal| tiene el mismo comportamiento que la instrucción nula en semánticas no desplazantes, ya que podemos poner como pre y poscondición el mismo aserto $P$ y el triple será cierto independientemente del aserto escogido.
\end{description}

\subsection{Intercambio de señales en programas que usan monitores}
En el Capítulo~\ref{cap:comp}, mencionamos que todas las semánticas de señales eran equivalentes para resolver cualquier problema. Esto se debe a que si se cumplen unas determinadas condiciones, entonces todas las señales (las automáticas, las desplazantes y las no desplazantes) son equivalentes. Esto significa que si estamos trabajando con una semántica, podemos reemplazar las señales de dicha semántica por cierto código que asemeja el comportamiento de otra semántica de señales.\\

Las condiciones que deben cumplirse dentro de un monitor para que podamos simular una semántica mediante otra son:
\begin{enumerate}
    \item Se tiene que exigir como poscondición de la operación \verb|wait| el invariante del monitor y nada más estricto (esto es, no podremos usar directamente la condición de sincronización tras la operación \verb|wait|, sino solamente el invariante del monitor\footnote{Notemos que esto no inclumple el axioma de la operación \texttt{wait}, ya que $C\rightarrow IM$.}).
    \item Con señales continuas, cuando realizamos un \verb|signal| no hay desplazamiento, por lo que para que se puedan asemejar a las señales desplazantes, hemos de obligar a que la llamada a \verb|signal| se haga siempre antes de suspender el proceso y salir del monitor.

        Esto es, que la operación \verb|signal| sea la última instrucción en el procedimiento de un monitor o que tras dicha operación haya una instrucción \verb|wait|.
    \item Finalmente, para que sean equivalentes los códigos, no podemos usar la operación \verb|signal_all|.
\end{enumerate}

\subsection{Señales \texttt{wait} con prioridad}
Como hemos visto hasta ahora, tras ejecutar una operación \verb|wait| sobre una variable condición \verb|c|, pasamos a la cola de bloqueados de la misma, con planificación FIFO\@. Sin embargo, en algunos lenguajes de programación aparece la operación \verb|wait| con prioridad sobre las variables de tipo condición, con el fin de solucionar situaciones en las que no queremos que tras una operación \verb|signal| se desbloquee el primero de la cola, sino aquel que tuviera más prioridad y lleve más tiempo en la misma.\\

En el caso descrito, la operación \verb|wait| aceptará un parámetro entero no negativo llamado \textit{prioridad}, de forma que a menor sea el valor de dicho parámetro, mayor prioridad tendrá en la cola de bloqueados. En este caso, la cola FIFO de la variable condición se sustituye por una cola con prioridad.\\

Mostramos a continuación un ejemplo para motivar por qué algunos lenguajes implementan esta operación \verb|wait|.
\begin{ejemplo}
    Queremos programar una alarma para procesos de tal manera que el proceso se bloquee al fijar la alarma y que este se desbloquee al cumplirse la hora previamente fijada. Para ello, implementaremos la alarma haciendo uso de un monitor con dos procedimientos:
    \begin{itemize}
        \item \verb|tick|, cableado a la interrupción de reloj del sistema operativo, con el fin de que el reloj sepa el instante en el que se encuentra en cada momento.
        \item \verb|despiertame|, que servirá para que cada proceso fije el momento en el que quiere ser despertado.
    \end{itemize}
    Para simplificar el ejemplo, usaremos como unidad de medida un ``tick''. Planteamos el siguiente monitor como solución al problema, usando señales \verb|wait| con prioridad:
    \begin{minted}[escapeinside=\#\#]{pascal}
        monitor despertador;
        var ahora : Long_integer;
            despertar : cond;  {variable condición prioritaria}

        procedure despiertame(n : integer);
        var alarma : Long_integer;
        begin
            alarma := ahora + n;    {hora para despertar}
            while ahora < alarma do 
                despertar.wait(alarma);
            end do;
            despertar.signal();
        end

        procedure tick();   {cableada a INT CLK}
        begin
            ahora := ahora + 1;
            despertar.signal();
        end

        begin
            ahora := 0;
        end
    \end{minted}
    De esta forma, lo que estamos haciendo es ordenar la cola de bloqueados de la variable condición \verb|despertar| poniendo al inicio de la misma el primer proceso a desbloquear.

    Por ejemplo, ante el uso del monitor por los procesos que observamos en la Tabla~\ref{tab:ejm_uso_monitor},
    \begin{table}
    \centering
    \begin{tabular}{|c|c|c|c|}
        \hline
        Proceso & Instante & Llamada & Cola de variable condición \\
        \hline
        P1 & 0 & \verb|despiertame(10)| & (P1) \\
        \hline
        P2 & 1 & \verb|despiertame(3)| & (P2 | P1) \\
        \hline
        P3 & 2 & \verb|despiertame(5)| & (P2 | P3 | P1) \\
        \hline
        P4 & 3 & \verb|despiertame(1)| & (P2 | P4 | P3 | P1) \\
        \hline
    \end{tabular}
    \caption{Ejemplo de uso del monitor}
    \label{tab:ejm_uso_monitor}
    \end{table}
    (donde P4 y P2 tienen la misma prioridad, pero como P4 se bloquea después, decidimos colocarlo en la cola después de P2) cuando la interrupción de reloj del sistema llame al procedimiento \verb|tick|, se desbloqueará al primer proceso de la cola de la variable condición, P2, que saldrá del bucle \verb|while| y desbloqueará al siguiente proceso de la cola, P4, que hará lo mismo, desbloqueando a P3, y como para P3 \verb|ahora < alarma|, volverá a bloquearse, quedando la cola de la variable condición como (P3 | P1).\\

    Notemos que con el uso de la operación \verb|wait| con prioridad minimizamos el número de procesos a desbloquear en cada nuevo instante, ya que tenemos la cola de bloqueados ordenada de forma que al inicio están los procesos que antes se desbloquearán, con lo que si encontramos el primer proceso que no se desbloquea, hemos terminado en dicho instante de despertar a los procesos.\\

    Si tratamos de programar este mismo problema sin operaciones \verb|wait| con prioridad, sino con la operación \verb|wait| que venimos usando a lo largo de este Capítulo, el procedimiento \verb|despiertame| quedaría como:
    \begin{minted}[escapeinside=\#\#]{pascal}
        procedure despiertame(n : integer);
        var alarma : Long_integer;
        begin
            alarma := ahora + n;    {hora para despertar}
            while ahora < alarma do 
                despertar.signal();
                despertar.wait();
            end do;
            despertar.signal();
        end
    \end{minted}
    Lo que sucede es que tenemos que estar continuamente desbloqueando a los procesos para comprobar si deben o no despertarse, al no tener ningún orden en dicha cola.
\end{ejemplo}

\section{Implementación de los monitores}
Podemos implementar todas las funcionalidades de un monitor usando semáforos. En esta sección, explicaremos cómo implementar un monitor con semántica de señales SU, que nos garantice el comportamiento de un proceso descrito en la Figura~\ref{fig:su}.\\

Para conseguir simular el comportamiento de un monitor, hemos de conseguir:
\begin{itemize}
    \item Tener una cola de entrada al monitor, que implementaremos usando un semáforo llamado \verb|mutex|.
    \item Tener una cola de procesos urgentes, que implementaremos con un semáforo llamado \verb|next|.
    \item Contabilizar el número de procesos bloqueados en la cola de urgentes\footnote{Con la finalidad de saber si dicha cola está o no vacía}, con una variable entera \verb|next_count|.
    \item Implementación de las variables condición.

        Para ello, por cada variable condición que queramos tener en un monitor, crearemos un semáforo nuevo y una variable entera que controle el número de procesos que bloquea dicho semáforo. Además, crearemos unas nuevas funciones \verb|wait| y \verb|signal| (llamadas \verb|x_wait| y \verb|x_signal|) que simulen el comportamiento de las operaciones \verb|wait| y \verb|signal| de las variables compartidas.
\end{itemize}

Inicializaremos las variables que nos permiten controlar el monitor de la forma:
\begin{minted}[escapeinside=\#\#]{pascal}
    mutex := 1;
    next := 0;
    next_count := 0;
\end{minted}

\subsubsection{Procedimientos del monitor}
Cada vez que nos dispongamos a crear un procedimiento nuevo para el monitor, deberemos ejecutar cierto código antes y después del mismo, con la finalidad de garantizar la exclusión mutua dentro del monitor. Para ello, crearemos dos funciones, \verb|entrada| y \verb|salida|, las cuales deberemos invocar antes y después del cuerpo del procedimiento a programar:
\begin{minted}{pascal}
    procedure P();
    begin
        entrada();
        {cuerpo del procedimiento}
        salida();
    end
\end{minted}
De esta forma, el código de entrada al monitor sería:
\setlength{\columnsep}{1cm}
\begin{minted}{pascal}
    procedure entrada();
    begin
        sem_wait(mutex);    {garantizar exclusión mutua}
    end
\end{minted}    
Y el de salida:
\begin{minted}{pascal}
    procedure salida();
    begin
        if(next_count <> 0)    {hay procesos en cola de urgentes}
            sem_signal(next);
        else
            sem_signal(mutex);   {liberamos el monitor}
        end if
    end
\end{minted}

\subsubsection{Variables condición}
Por cada variable condición a usar dentro del monitor necesitamos tener un par semáforo, entero. Mostramos ahora cómo podemos implementar las operaciones \verb|wait| y \verb|signal| de las variables condición usando semáforos.

Para ello, crearemos una función por cada operación a simular, la cual recibirá dos parámetros: el semáforo que simula la variable condición (\verb|x_sem|) y la cantidad de procesos que dicho semáforo tiene bloqueados (\verb|x_count|):
\begin{minted}{pascal}
    procedure x_wait(x_sem : semaphore, x_count : integer);
    begin
        x_count := x_count + 1;  {un bloqueado más}

        if(next_count <> 0) then  {hay procesos en cola de urgentes}
            sem_signal(next);
        else
            sem_signal(mutex);  {deja libre el monitor}
        end if

        sem_wait(x_sem);
        x_count := x_count - 1;
    end
\end{minted}
\begin{minted}{pascal}
    procedure x_signal(x_sem : semaphore, x_count : integer);
    begin
        if(x_count <> 0) then  {si no hay bloqueado no hace nada}
            next_count := next_count + 1;
            sem_signal(x_sem);
            sem_wait(next);
            next_count := next_count - 1;
        end if
    end
\end{minted}
Y con todas estas funciones tenemos \textbf{casi} implementados los monitores. Hay que tener en cuenta que los semáforos no tienen una política de planificación fija, sino que es el sistema operativo quien planifica los procesos desbloqueados por el semáforo.

Es necesario por tanto, usar semáforos con colas FIFO para tener totalmente implementados los monitores.

\section{Introducción}

\begin{ejercicio}
    Considerar el siguiente fragmento de programa para 2 procesos \verb|P1| y \verb|P2|: Los dos procesos
    pueden ejecutarse a cualquier velocidad. ¿Cuáles son los posibles valores resultantes para la
    variable \verb|x|? Suponer que \verb|x| debe ser cargada en un registro para incrementarse y que cada
    proceso usa un registro diferente para realizar el incremento.
    \setlength{\columnsep}{2cm} % Ajusta el espacio entre columnas
    \begin{multicols}{2}
        \begin{minted}{pascal}
        { variables compartidas }
        var x : integer := 0 ;
        Process P1;
        var i: integer;
        begin
          begin
            for i:= 1 to 2 do begin
              x:= x + 1;
            end
          end
        end
        \end{minted}
        
        \begin{minted}{pascal}
            

        Process P2;
        var j: integer;
        begin
          begin
            for j:= 1 to 2 do begin
              x:= x + 1;
            end
          end
        end
        \end{minted}
    \end{multicols}
\end{ejercicio}


\begin{ejercicio}
    ¿Cómo se podría hacer la copia del fichero \verb|f| en otro \verb|g|, de forma concurrente, utilizando la
    instrucción concurrente \verb|cobegin-coend|? Para ello, suponer que:
    \begin{enumerate}
        \item Los archivos son una secuencia de items de un tipo arbitrario \verb|T|, y se encuentran ya abiertos
        para lectura (\verb|f|) y escritura (\verb|g|). Para leer un ítem de \verb|f| se usa la llamada a función \verb|leer(f)| y
        para saber si se han leído todos los ítems de \verb|f|, se puede usar la llamada \verb|fin(f)| que devuelve
        verdadero si ha habido al menos un intento de leer cuando ya no quedan datos. Para
        escribir un dato \verb|x| en \verb|g| se puede usar la llamada a procedimiento \verb|escribir(g,x)|.

        \item El orden de los items escritos en \verb|g| debe coincidir con el de \verb|f|.
        \item Dos accesos a dos archivos distintos pueden solaparse en el tiempo.
    \end{enumerate}
\end{ejercicio}

\begin{ejercicio}\label{ej:3}
    Construir, utilizando las instrucciones concurrentes \verb|cobegin-coend| y \verb|fork-join|, programas concurrentes que se correspondan con los grafos de precedencia que se muestran en la figura \ref{fig:grafoEj2}.
    \begin{figure}
        \centering
        \begin{subfigure}{0.3\textwidth}
            \centering
            \resizebox{\linewidth}{!}{
                \begin{tikzpicture}[
                    node/.style={circle, draw, minimum size=0.5cm},
                    edge/.style={-stealth}
                    ]
        
                    % Nodos
                    \node[node] (P0) {P0};
                    \node[node, below left=of P0] (P1) {P1};
                    \node[node, below right=of P0] (P2) {P2};
                    \node[node, below=of P1] (P3) {P3};
                    \node[node, below left=of P3] (P4) {P4};
                    \node[node, below right=of P3] (P5) {P5};
                    \node[node, below=of P5] (P6) {P6};
        
                    % Aristas
                    \draw[edge] (P0) -- (P1);
                    \draw[edge] (P0) -- (P2);
                    \draw[edge] (P1) -- (P3);
                    \draw[edge] (P3) -- (P4);
                    \draw[edge] (P3) -- (P5);
                    \draw[edge, bend right] (P4) to (P6);
                    \draw[edge] (P5) -- (P6);
                    \draw[edge, bend left] (P2) to (P6);
                
                \end{tikzpicture}
            }
            \caption{DAG del apartado \ref{ej:3.1}.}
            \label{fig:grafoEj3.1}
            
        \end{subfigure}
        \begin{subfigure}{0.3\textwidth}
            \centering
            \resizebox{\linewidth}{!}{
                \begin{tikzpicture}[
                    node/.style={circle, draw, minimum size=1cm},
                    edge/.style={-stealth}
                    ]
        
                    % Nodos
                    \node[node] (P0) {P0};
                    \node[node, below left=of P0] (P1) {P1};
                    \node[node, below right=of P0] (P2) {P2};
                    \node[node, below=of P1] (P3) {P3};
                    \node[node, below left=of P1] (P4) {P4};
                    \node[node, below right=of P1] (P5) {P5};
                    \node[node, below=of P5] (P6) {P6};
        
                    % Aristas
                    \draw[edge] (P0) -- (P1);
                    \draw[edge] (P0) -- (P2);
                    \draw[edge] (P1) -- (P3);
                    \draw[edge] (P1) -- (P4);
                    \draw[edge] (P1) -- (P5);
                    \draw[edge, bend right] (P4) to (P6);
                    \draw[edge] (P5) -- (P6);
                    \draw[edge, bend left] (P2) to (P6);
                    \draw[edge, bend right] (P3) to (P6);
                
                \end{tikzpicture}
            }
            \caption{DAG del apartado \ref{ej:3.2}.}
            \label{fig:grafoEj3.2}
            
        \end{subfigure}
        \begin{subfigure}{0.3\textwidth}
            \centering
            \resizebox{\linewidth}{!}{
                \begin{tikzpicture}[
                    node/.style={circle, draw, minimum size=1cm},
                    edge/.style={-stealth}
                    ]
        
                    % Nodos
                    \node[node] (P0) {P0};
                    \node[node, below left=of P0] (P1) {P1};
                    \node[node, below right=of P0] (P2) {P2};
                    \node[node, below=of P1] (P3) {P3};
                    \node[node, below left=of P3] (P4) {P4};
                    \node[node, below right=of P3] (P5) {P5};
                    \node[node, below=of P5] (P6) {P6};
        
                    % Aristas
                    \draw[edge] (P0) -- (P1);
                    \draw[edge] (P0) -- (P2);
                    \draw[edge] (P1) -- (P3);
                    \draw[edge] (P3) -- (P4);
                    \draw[edge] (P3) -- (P5);
                    \draw[edge, bend right] (P4) to (P6);
                    \draw[edge] (P5) -- (P6);
                    \draw[edge] (P2) to (P5);
                
                \end{tikzpicture}
            }
            \caption{DAG del apartado \ref{ej:3.3}.}
            \label{fig:grafoEj3.3}
            
        \end{subfigure}
        \caption{Grafos de precedencia del ejercicio \ref{ej:3}.}
        \label{fig:grafoEj3}
    \end{figure}
    \begin{enumerate}
        \item \label{ej:3.1}
         Grafo de precedencia de la figura \ref{fig:grafoEj3.1}:
        \item \label{ej:3.2}
        Grafo de precedencia de la figura \ref{fig:grafoEj3.2}:
         \item \label{ej:3.3}
         Grafo de precedencia de la figura \ref{fig:grafoEj3.3}:
    \end{enumerate}
\end{ejercicio}



\begin{ejercicio} \label{ej:4}
    Dados los siguientes fragmentos de programas concurrentes, obtener sus grafos de precedencia asociados:
    \begin{figure}
        \centering
        \begin{subfigure}[b]{0.45\textwidth}
            \centering
            \begin{minted}{pascal}
                begin
                    P0 ;
                    cobegin
                        P1 ;
                        P2 ;
                        cobegin
                            P3 ; P4 ; P5 ; P6 ;
                        coend ;
                        P7 ;
                    coend ;
                    P8 ;
                end ;
            \end{minted}
            \caption{Programa 1.}
            \label{code:prog1_Ej4}
        \end{subfigure}\hfill
        \begin{subfigure}[b]{0.45\textwidth}
            \centering
            \begin{minted}{pascal}
                begin
                    P0 ;
                    cobegin
                        begin
                            cobegin
                                P1 ; P2 ;
                            coend
                            P5 ;
                        end
                        begin
                            cobegin
                                P3 ; P4 ;
                            coend
                            P6 ;
                        end
                    coend
                    P7 ;
                end
            \end{minted}
            \caption{Programa 2.}
            \label{code:prog2_Ej4}
        \end{subfigure}
        \caption{Programas concurrentes del ejercicio \ref{ej:4}.}
    \end{figure}
    
    \begin{enumerate}
        \item Programa de la figura \ref{code:prog1_Ej4}.
        \item Programa de la figura \ref{code:prog2_Ej4}.
    \end{enumerate}
\end{ejercicio}
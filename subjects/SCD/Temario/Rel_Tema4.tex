\section{Sistemas de Tiempo Real}

\begin{comment}
        \begin{tikzpicture}[scale=0.8]
            % Eje de abscisas
            \draw[-Stealth] (0, 0) -- (18, 0) node[right] {};
            \foreach \x in {0, 2, 4, 6, 8, 10, 12, 14, 16} {
                \draw (\x, 0) -- (\x, -0.2) node[below] {\x};
            }

            % Eje de ordenadas
            \draw[-Stealth] (0, 0) -- (0, 8.5) node[above] {};
            \draw (0, 2) -- (-0.2, 2) node[left] {$\tau_4$};
            \draw (0, 4) -- (-0.2, 4) node[left] {$\tau_3$};
            \draw (0, 6) -- (-0.2, 6) node[left] {$\tau_2$};
            \draw (0, 8) -- (-0.2, 8) node[left] {$\tau_1$};

            \draw[thick, gray] (0, 2) -- (2, 2);

            \fill (1, 2) circle (2pt) node[above] {\verb|P(s1)|};
        \end{tikzpicture}       
\end{comment}

\begin{ejercicio}\label{ej:rel4_1}
    Dado el conjunto de tareas periódicas y sus atributos temporales que se indica en la Tabla~\ref{tab:4_1}, determinar si se puede planificar el conjunto de dichas tareas utilizando un esquema de planificación basado en planificación cíclica. Diseña el plan cíclico determinando el marco secundario, y el entrelazamiento de las tareas sobre un cronograma.
    \begin{table}[H]
    \centering
    \begin{tabular}{|c|c|c|c|}
        \hline
        Tarea & $C_i$ & $T_i$ & $D_i$ \\
        \hline
        $\tau_1$ & 10 & 40 & 40 \\
        \hline
        $\tau_2$ & 18 & 50 & 50 \\
        \hline
        $\tau_3$ & 10 & 200 & 200 \\
        \hline
        $\tau_4$ & 20 & 200 & 200 \\
        \hline
    \end{tabular}
    \caption{Tareas periódicas y sus atributos temporales.}
    \label{tab:4_1}
    \end{table}

    La única forma de asegurar la planificabilidad del conjunto de tareas por ejecutivo cíclico es creando un ejecutivo cíclico capaz de planificar dicho conjunto. Para ello, comenzamos calculando el valor del hiperperiodo:
    \begin{equation*}
        T_M = mcm(40,50,200) = 200
    \end{equation*}
    A continuación hemos de elegir un valor para el ciclo secundario, que según el desarrollo visto en teoría este ha de ser un divisor de 200 y ha de cumplir que:
    \begin{equation*}
        \max\{10,18,20\} = 20 \leq T_s \leq 40 = \min\{40,50,200\}
    \end{equation*}
    Elegimos por ejemplo $T_s = 40$.

    Ahora es cuestión de cuadrar bien las tareas dentro de los ciclos secundarios de forma que ninguna incumpla sus \textit{deadlines}. Como $T_1=40=T_s$, la tarea $\tau_1$ deberá estar en cada ciclo secundario. Además, como $T_3=T_4=200=T_M$, solo habrá que tener en cuenta una vez dentro del hiperperiodo a las tareas $\tau_3$ y $\tau_4$. El ejecutivo cíclico que hemos elaborado es el de la Figura~\ref{fig:ej1}.

    \begin{figure}[H]
        \centering
        \begin{tikzpicture}[scale=0.7]
            % Eje de abscisas
            \draw[-Stealth] (0, 0) -- (20.5, 0) node[right] {};
            \draw (0, 0) -- (0, -0.2) node[below] {0};
            \draw (1, 0) -- (1, -0.2) node[below] {};
            \draw (2, 0) -- (2, -0.2) node[below] {};
            \draw (3, 0) -- (3, -0.2) node[below] {};
            \draw (4, 0) -- (4, -0.2) node[below] {40};
            \draw (5, 0) -- (5, -0.2) node[below] {};
            \draw (6, 0) -- (6, -0.2) node[below] {};
            \draw (7, 0) -- (7, -0.2) node[below] {};
            \draw (8, 0) -- (8, -0.2) node[below] {80};
            \draw (9, 0) -- (9, -0.2) node[below] {};
            \draw (10, 0) -- (10, -0.2) node[below] {};
            \draw (11, 0) -- (11, -0.2) node[below] {};
            \draw (12, 0) -- (12, -0.2) node[below] {120};
            \draw (13, 0) -- (13, -0.2) node[below] {};
            \draw (14, 0) -- (14, -0.2) node[below] {};
            \draw (15, 0) -- (15, -0.2) node[below] {};
            \draw (16, 0) -- (16, -0.2) node[below] {160};
            \draw (17, 0) -- (17, -0.2) node[below] {};
            \draw (18, 0) -- (18, -0.2) node[below] {};
            \draw (19, 0) -- (19, -0.2) node[below] {};
            \draw (20, 0) -- (20, -0.2) node[below] {200};

            % Eje de ordenadas
            \draw[-Stealth] (0, 0) -- (0, 8.5) node[above] {};
            \draw (0, 2) -- (-0.2, 2) node[left] {$\tau_1$};
            \draw (0, 4) -- (-0.2, 4) node[left] {$\tau_2$};
            \draw (0, 6) -- (-0.2, 6) node[left] {$\tau_3$};
            \draw (0, 8) -- (-0.2, 8) node[left] {$\tau_4$};

            \draw[thick, gray] (0, 2) -- (1, 2);
            \draw[thick, gray] (1, 4) -- (2.8, 4);
            \draw[thick, gray] (2.8, 6) -- (3.8, 6);
            \draw[thick, gray] (4, 2) -- (5, 2);
            \draw[thick, gray] (5, 8) -- (7, 8);
            \draw[thick, gray] (8, 4) -- (9.8, 4);
            \draw[thick, gray] (9.8, 2) -- (10.8, 2);
            \draw[thick, gray] (12, 4) -- (13.8, 4);
            \draw[thick, gray] (13.8, 2) -- (14.8, 2);
            \draw[thick, gray] (16, 2) -- (17, 2);
            \draw[thick, gray] (17, 4) -- (18.8, 4);

            \draw[thick, black] (4, 0) -- (4, 8.5);
            \draw[thick, black] (8, 0) -- (8, 8.5);
            \draw[thick, black] (12, 0) -- (12, 8.5);
            \draw[thick, black] (16, 0) -- (16, 8.5);
            \draw[thick, black] (20, 0) -- (20, 8.5);

            \fill (5, 4) circle (2pt);
            \fill (10, 4) circle (2pt);
            \fill (15, 4) circle (2pt);
            \fill (20, 4) circle (2pt);
        \end{tikzpicture}
        \caption{Ejecutivo cíclico para el Ejercicio~\ref{ej:rel4_1}.}
        \label{fig:ej1}
    \end{figure}
\end{ejercicio}

\begin{ejercicio}\label{ej:rel4_2}
    El siguiente conjunto de tareas periódicas se puede planificar con ejecutivos cíclicos. Determina si esto es cierto calculando el marco secundario que debería tener. Dibuja el cronograma que muestre las ocurrencias de cada tarea y su entrelazamiento. ¿Cómo se tendría que implementar? (escribe el pseudo-código de la implementación)
    \begin{table}[H]
    \centering
    \begin{tabular}{|c|c|c|c|}
        \hline
        Tarea & $C_i$ & $T_i$ & $D_i$ \\
        \hline
        $\tau_1$ & 2 & 6 & 6 \\
        \hline
        $\tau_2$ & 2 & 8 & 8 \\
        \hline
        $\tau_3$ & 3 & 12 & 12 \\
        \hline
    \end{tabular}
    \caption{Tareas periódicas y sus atributos temporales.}
    \label{tab:4_2}
    \end{table}

    Calculamos en primer lugar el hiperperiodo:
    \begin{equation*}
        T_M = mcm(6,8,12) = 24
    \end{equation*}
    Posteriormente, el ciclo secundario, que ha de ser un divisor de 24 y tiene que cumplir que:
    \begin{equation*}
        \max\{2,3\} = 3 \leq T_s \leq 6 = \min\{6,8,12\}
    \end{equation*}
    Por tanto, los posibles candidatos a $T_s$ son 3, 4 y 6. Si elegimos $T_s=6$, podemos considerar el ejecutivo cíclico de la Figura~\ref{fig:ej2}.
    \begin{figure}[H]
        \centering
        \begin{tikzpicture}[scale=0.5]
            % Eje de abscisas
            \draw[-Stealth] (0, 0) -- (24.5, 0) node[right] {};
            \draw (0, 0) -- (0, -0.2) node[below] {0};
            \draw (1, 0) -- (1, -0.2) node[below] {};
            \draw (2, 0) -- (2, -0.2) node[below] {2};
            \draw (3, 0) -- (3, -0.2) node[below] {};
            \draw (4, 0) -- (4, -0.2) node[below] {4};
            \draw (5, 0) -- (5, -0.2) node[below] {};
            \draw (6, 0) -- (6, -0.2) node[below] {6};
            \draw (7, 0) -- (7, -0.2) node[below] {};
            \draw (8, 0) -- (8, -0.2) node[below] {8};
            \draw (9, 0) -- (9, -0.2) node[below] {};
            \draw (10, 0) -- (10, -0.2) node[below] {10};
            \draw (11, 0) -- (11, -0.2) node[below] {};
            \draw (12, 0) -- (12, -0.2) node[below] {12};
            \draw (13, 0) -- (13, -0.2) node[below] {};
            \draw (14, 0) -- (14, -0.2) node[below] {14};
            \draw (15, 0) -- (15, -0.2) node[below] {};
            \draw (16, 0) -- (16, -0.2) node[below] {16};
            \draw (17, 0) -- (17, -0.2) node[below] {};
            \draw (18, 0) -- (18, -0.2) node[below] {18};
            \draw (19, 0) -- (19, -0.2) node[below] {};
            \draw (20, 0) -- (20, -0.2) node[below] {20};
            \draw (21, 0) -- (21, -0.2) node[below] {};
            \draw (22, 0) -- (22, -0.2) node[below] {22};
            \draw (23, 0) -- (23, -0.2) node[below] {};
            \draw (24, 0) -- (24, -0.2) node[below] {24};

            % Eje de ordenadas
            \draw[-Stealth] (0, 0) -- (0, 6.5) node[above] {};
            \draw (0, 2) -- (-0.2, 2) node[left] {$\tau_1$};
            \draw (0, 4) -- (-0.2, 4) node[left] {$\tau_2$};
            \draw (0, 6) -- (-0.2, 6) node[left] {$\tau_3$};

            \draw[thick, gray] (0, 2) -- (2, 2);
            \draw[thick, gray] (2, 6) -- (5, 6);
            \draw[thick, gray] (6, 4) -- (8, 4);
            \draw[thick, gray] (8, 2) -- (10, 2);
            \draw[thick, gray] (10, 4) -- (12, 4);
            \draw[thick, gray] (12, 2) -- (14, 2);
            \draw[thick, gray] (14, 6) -- (17, 6);
            \draw[thick, gray] (18, 2) -- (20, 2);
            \draw[thick, gray] (20, 4) -- (22, 4);

            \draw[thick, black] (6, 0) -- (6, 6.5);
            \draw[thick, black] (12, 0) -- (12, 6.5);
            \draw[thick, black] (18, 0) -- (18, 6.5);
            \draw[thick, black] (24, 0) -- (24, 6.5);

            \fill (8, 4) circle (2pt);
            \fill (16, 4) circle (2pt);
            \fill (24, 4) circle (2pt);
        \end{tikzpicture}
        \caption{Ejecutivo cíclico para el Ejercicio~\ref{ej:rel4_2}.}
        \label{fig:ej2}
    \end{figure}
    Este podría implementarse en lenguaje C++ mediante el código de la Figura~\ref{cod:ej2}.
    \begin{figure}
    \centering
    \begin{minted}{c++}
        void Ejecutivo(){
           const milliseconds tmp_secundario(6);
           const int nciclos = 4,  // nº de ciclos secundarios
             siguiente_instante = clock::now();  
           int frame = 0;   // nº del siguiente ciclo secundario

           while(true){
              for(frame = 1; frame <= nciclos; frame++){
                 switch(frame){
                    case 0: T1(); T3(); break;
                    case 1: T2(); T1(); T2(); break;
                    case 2: T1(); T3(); break;
                    case 3: T1(); T2(); break;
                 }

                 siguiente_instante += tmp_secundario;
                 sleep_until(siguiente_instante);
              }
           }
        }
    \end{minted}
    \caption{Implementación del ejecutivo cíclico para el Ejercicio~\ref{ej:rel4_2}.}
    \label{cod:ej2}
    \end{figure}

\end{ejercicio}

\begin{ejercicio}\label{ej:rel4_3}
    Comprobar si el conjunto de procesos periódicos que se muestra en la siguiente tabla es planificable con el algoritmo RMS utilizando el test basado en el factor de utilización del tiempo del procesador. Si el test no se cumple, ¿debemos descartar que el sistema sea planificable?    
    \begin{table}[H]
    \centering
    \begin{tabular}{|c|c|c|}
        \hline
        Tarea & $C_i$ & $T_i$ \\
        \hline
        $\tau_1$ & 9 & 30 \\
        \hline
        $\tau_2$ & 10 & 40 \\
        \hline
        $\tau_3$ & 10 & 50 \\
        \hline
    \end{tabular}
    \caption{Tareas periódicas y sus atributos temporales.}
    \label{tab:4_3}
    \end{table}

    Para comprobar si el conjunto de procesos periódicos mostrado en la Tabla~\ref{tab:4_3} es planificable con el algoritmo RMS podemos calcular su factor de utilización de CPU y hacer uso del test de Liu y Layland:
    \begin{gather*}
        U = \sum_{i=1}^{3}\dfrac{C_i}{T_i} = \dfrac{9}{30} + \dfrac{10}{40} + \dfrac{10}{50} = \dfrac{3}{10} + \dfrac{1}{4} + \dfrac{1}{5} = 0.75 \\
        U_0(3) = 3\cdot \left(2^{\frac{1}{3}}-1\right) = 0.779
    \end{gather*}
    Como $U \leq U_0(3)$, tenemos que el conjunto de tareas es planificable con RMS\@. Si el test no se cumple, al ser solo una condición suficiente, no tenemos ninguna información sobre el conjunto de tareas. En este caso, lo único que podemos asegurar es que si $U>1$ entonces el conjunto de tareas no es planificable. Sin embargo, si $U$ está entre $U_0(3)$ y $1$ no tenemos información de lo que sucede y tendremos que recurrir al diagrama de Gantt para determinar si el conjunto de tareas es o no planificable.
\end{ejercicio}

\begin{ejercicio}\label{ej:rel4_4}
    Considérese el siguiente conjunto de tareas compuesto por tres tareas periódicas:
    \begin{table}[H]
    \centering
    \begin{tabular}{|c|c|c|}
        \hline
        Tarea & $C_i$ & $T_i$ \\
        \hline
        $\tau_1$ & 10 & 40 \\
        \hline
        $\tau_2$ & 20 & 60 \\
        \hline
        $\tau_3$ & 20 & 80 \\
        \hline
    \end{tabular}
    \caption{Tareas periódicas y sus atributos temporales.}
    \label{tab:4_4}
    \end{table}
    Comprueba la planificabilidad del conjunto de tareas con el algoritmo RMS utilizando el test basado en el factor de utilización. Calcular el hiperperiodo y construir el correspondiente cronograma.\\

    Calculamos el factor de utilización de la CPU por parte del conjunto de tareas:
    \begin{equation*}
        U = \sum_{i=1}^{3}\dfrac{C_i}{T_i} = \dfrac{20}{60} + \dfrac{20}{80} + \dfrac{20}{80} = \dfrac{1}{3} + \dfrac{1}{4} + \dfrac{1}{4} = 0.83 > 0.779 = U_0(3)
    \end{equation*}
    Como no es menor que la constante que nos ofrece el Test de Liu y Layland, no podemos decir nada sobre la planificabilidad del conjunto de tareas por RMS\@. El hiperperiodo es:
    \begin{equation*}
        T_M = mcm(40,60,80) = 240
    \end{equation*}
    Construimos el cronograma, que puede observarse en la Figura~\ref{fig:ej4}.
    \begin{figure}[H]
        \centering
        \begin{tikzpicture}[scale=0.6]
            % Eje de abscisas
            \draw[-Stealth] (0, 0) -- (25, 0) node[right] {};

            \foreach \x in {0, 2, ..., 22} {
                \pgfmathtruncatemacro{\xa}{\x + 1}
                \pgfmathtruncatemacro{\label}{\xa * 10}
                \draw (\x, 0) -- (\x, -0.2) node[below] {};
                \draw (\xa, 0) -- (\xa, -0.2) node[below] {\label};
            }
            \draw (24, 0) -- (24, -0.2) node[below] {};

            % Eje de ordenadas
            \draw[-Stealth] (0, 0) -- (0, 6.5) node[above] {};
            \draw (0, 2) -- (-0.2, 2) node[left] {$\tau_1$};
            \draw (0, 4) -- (-0.2, 4) node[left] {$\tau_2$};
            \draw (0, 6) -- (-0.2, 6) node[left] {$\tau_3$};

            \draw[thick, gray] (0, 2) -- (1, 2);
            \draw[thick, gray] (4, 2) -- (5, 2);
            \draw[thick, gray] (8, 2) -- (9, 2);
            \draw[thick, gray] (12, 2) -- (13, 2);
            \draw[thick, gray] (16, 2) -- (17, 2);
            \draw[thick, gray] (20, 2) -- (21, 2);

            \draw[thick, gray] (1, 4) -- (3, 4);
            \draw[thick, gray] (6, 4) -- (8, 4);
            \draw[thick, gray] (13, 4) -- (15, 4);
            \draw[thick, gray] (18, 4) -- (20, 4);

            \draw[thick, gray] (3, 6) -- (4, 6);
            \draw[thick, gray] (5, 6) -- (6, 6);
            \draw[thick, gray] (9, 6) -- (11, 6);
            \draw[thick, gray] (17, 6) -- (18, 6);
            \draw[thick, gray] (21, 6) -- (22, 6);

            % deadline de t1
            \fill (4, 2) circle (2pt);
            \fill (8, 2) circle (2pt);
            \fill (12, 2) circle (2pt);
            \fill (16, 2) circle (2pt);
            \fill (20, 2) circle (2pt);
            \fill (24, 2) circle (2pt);

            % deadline de t2
            \fill (6, 4) circle (2pt);
            \fill (12, 4) circle (2pt);
            \fill (18, 4) circle (2pt);
            \fill (24, 4) circle (2pt);

            % deadline de t3
            \fill (8, 6) circle (2pt);
            \fill (16, 6) circle (2pt);
            \fill (24, 6) circle (2pt);
        \end{tikzpicture}       
        \caption{Diagrama de Gantt para las tareas para el Ejercicio~\ref{ej:rel4_4}.}
        \label{fig:ej4}
    \end{figure}

\end{ejercicio}

\begin{ejercicio}\label{ej:rel4_5}
    Comprobar la planificabilidad y construir el cronograma de acuerdo al algoritmo de planificación RMS del siguiente conjunto de tareas periódicas.
    \begin{table}[H]
    \centering
    \begin{tabular}{|c|c|c|}
        \hline
        Tarea & $C_i$ & $T_i$ \\
        \hline
        $\tau_1$ & 20 & 60 \\
        \hline
        $\tau_2$ & 20 & 80 \\
        \hline
        $\tau_3$ & 20 & 120 \\
        \hline
    \end{tabular}
    \caption{Tareas periódicas y sus atributos temporales.}
    \label{tab:4_5}
    \end{table}

    Calculamos el factor de utilización de la CPU\@:
    \begin{equation*}
        U = \sum_{i=1}^{3} \dfrac{C_i}{T_i} = \dfrac{20}{60} + \dfrac{20}{80} + \dfrac{20}{120} = \dfrac{1}{3} + \dfrac{1}{4} + \dfrac{1}{6} = 0.75 \leq 0.779 = U_0(3)
    \end{equation*}
    Por tanto, el conjunto de tareas es planificable gracias al test de Liu y Layland. Mostramos el cronograma en la Figura~\ref{fig:ej5}, donde hemos usado un hiperperiodo de $mcm(60,80,120) = 240$.
    \begin{figure}[H]
        \centering
        \begin{tikzpicture}[scale=0.6]
            % Eje de abscisas
            \draw[-Stealth] (0, 0) -- (25, 0) node[right] {};

            \foreach \x in {0, 2, ..., 22} {
                \pgfmathtruncatemacro{\xa}{\x + 1}
                \pgfmathtruncatemacro{\label}{\x * 10}
                \draw (\x, 0) -- (\x, -0.2) node[below] {\label};
                \draw (\xa, 0) -- (\xa, -0.2) node[below] {};
            }
            \draw (24, 0) -- (24, -0.2) node[below] {240};

            % Eje de ordenadas
            \draw[-Stealth] (0, 0) -- (0, 6.5) node[above] {};
            \draw (0, 2) -- (-0.2, 2) node[left] {$\tau_1$};
            \draw (0, 4) -- (-0.2, 4) node[left] {$\tau_2$};
            \draw (0, 6) -- (-0.2, 6) node[left] {$\tau_3$};

            \draw[thick, gray] (0, 2) -- (2, 2);
            \draw[thick, gray] (2, 4) -- (4, 4);
            \draw[thick, gray] (4, 6) -- (6, 6);
            \draw[thick, gray] (6, 2) -- (8, 2);
            \draw[thick, gray] (8, 4) -- (10, 4);
            \draw[thick, gray] (12, 2) -- (14, 2);
            \draw[thick, gray] (14, 6) -- (16, 6);
            \draw[thick, gray] (16, 4) -- (18, 4);
            \draw[thick, gray] (18, 2) -- (20, 2);

            \fill (4, 2) circle (2pt);
            \fill (8, 2) circle (2pt);
            \fill (12, 2) circle (2pt);
            \fill (16, 2) circle (2pt);
            \fill (20, 2) circle (2pt);
            \fill (24, 2) circle (2pt);

            \fill (8, 6) circle (2pt);
            \fill (16, 6) circle (2pt);
            \fill (24, 6) circle (2pt);

            \fill (6, 4) circle (2pt);
            \fill (12, 4) circle (2pt);
            \fill (18, 4) circle (2pt);
            \fill (24, 4) circle (2pt);
        \end{tikzpicture}       
        \caption{Diagrama de Gantt para las tareas para el Ejercicio~\ref{ej:rel4_5}.}
        \label{fig:ej5}
    \end{figure}
\end{ejercicio}

\begin{ejercicio}\label{ej:rel4_6}
    Determinar si el siguiente conjunto de tareas puede planificarse con la política de planificación RMS y con la política EDF, utilizando los tests de planificabilidad adecuados para cada uno de los dos casos. Comprobar también la planificabilidad en ambos casos construyendo los dos cronogramas.
    \begin{table}[H]
    \centering
    \begin{tabular}{|c|c|c|}
        \hline
        Tarea & $C_i$ & $T_i$ \\
        \hline
        $\tau_1$ & 1 & 5 \\
        \hline
        $\tau_2$ & 1 & 10 \\
        \hline
        $\tau_3$ & 2 & 20 \\
        \hline
        $\tau_4$ & 10 & 20 \\
        \hline
        $\tau_5$ & 7 & 100 \\
        \hline
    \end{tabular}
    \caption{Tareas periódicas y sus atributos temporales.}
    \label{tab:4_6}
    \end{table}

    Calculamos el factor de utilización de la CPU\@:
    \begin{gather*}
        U = \sum_{i=1}^{5}\dfrac{C_i}{T_i} = \dfrac{1}{5} + \dfrac{1}{10} + \dfrac{2}{20} + \dfrac{10}{20} + \dfrac{7}{100} = \dfrac{1}{5} + \dfrac{1}{10} + \dfrac{1}{10} + \dfrac{1}{2} + \dfrac{7}{100} = 0.97\\
        U_0(5) = 5\cdot \left(2^{\frac{1}{5}}-1\right) = 0.74
    \end{gather*}
    Como $U>U_0(5)$, no sabemos si el conjunto de tareas es planificable o no por RMS\@. Sin embargo, como $U\leq 1$, sabemos que sí será planificabe por EDF\@.

    Construimos los dos cronogramas para comprobar la planificabilidad en cada caso, considerando que $T_M=mcm(5,10,20,100)=100$.
    \begin{description}
        \item [RMS.] A pesar de ser $T_M=100$, solo hace falta observar hasta 20, tal y como hacemos en la Figura~\ref{fig:ej6}, ya que cada $20 = mcm(5,10,20)$ el comportamiento de las tareas $\tau_1$, $\tau_2$, $\tau_3$ y $\tau_4$ será el mismo, a diferencia de la tarea $\tau_5$, que cada 20 unidades temporales conseguirá 2 unidades de cómputo. Como $C_5=7$ y $\nicefrac{100}{20} = 4$, la tarea $\tau_5$ logrará completar su ejecución dentro de su \textit{deadline}, de forma que hasta $t=80$ el cronograma será repetir 4 veces el de la Figura~\ref{fig:ej6} y desde $t=80$ hasta $t=100$ el único cambio será que $\tau_5$ se ejecuta durante una unidad de tiempo en vez de dos.

            Pese a que el test de Liu y Layland no nos diga nada sobre la planificabilidad de las tareas, observamos que sí que son planificables por RMS\@.
    \begin{figure}[H]
        \centering
        \begin{tikzpicture}[scale=0.7]
            % Eje de abscisas
            \draw[-Stealth] (0, 0) -- (21, 0) node[right] {};

            \foreach \x in {0, 2, ..., 18} {
                \pgfmathtruncatemacro{\xa}{\x + 1}
                \draw (\x, 0) -- (\x, -0.2) node[below] {\x};
                \draw (\xa, 0) -- (\xa, -0.2) node[below] {};
            }
            \draw (20, 0) -- (20, -0.2) node[below] {20};

            % Eje de ordenadas
            \draw[-Stealth] (0, 0) -- (0, 11) node[above] {};
            \draw (0, 2) -- (-0.2, 2) node[left] {$\tau_1$};
            \draw (0, 4) -- (-0.2, 4) node[left] {$\tau_2$};
            \draw (0, 6) -- (-0.2, 6) node[left] {$\tau_3$};
            \draw (0, 8) -- (-0.2, 8) node[left] {$\tau_4$};
            \draw (0, 10) -- (-0.2, 10) node[left] {$\tau_5$};

            \draw[thick, gray] (0, 2) -- (1, 2);
            \draw[thick, gray] (1, 4) -- (2, 4);
            \draw[thick, gray] (2, 6) -- (4, 6);
            \draw[thick, gray] (4, 8) -- (5, 8);
            \draw[thick, gray] (5, 2) -- (6, 2);
            \draw[thick, gray] (6, 8) -- (10, 8);
            \draw[thick, gray] (10, 2) -- (11, 2);
            \draw[thick, gray] (11, 4) -- (12, 4);
            \draw[thick, gray] (12, 8) -- (15, 8);
            \draw[thick, gray] (15, 2) -- (16, 2);
            \draw[thick, gray] (16, 8) -- (18, 8);
            \draw[thick, gray] (18, 10) -- (20, 10);

            \fill (5, 2) circle (2pt);
            \fill (10, 2) circle (2pt);
            \fill (15, 2) circle (2pt);
            \fill (20, 2) circle (2pt);
            \fill (10, 4) circle (2pt);
            \fill (20, 4) circle (2pt);
            \fill (20, 6) circle (2pt);
            \fill (20, 8) circle (2pt);
        \end{tikzpicture}       
        \caption{Diagrama de Gantt para el Ejercicio~\ref{ej:rel4_6} por RMS\@.}
        \label{fig:ej6}
    \end{figure}
    \item [EDF.] El cronograma de la Figura~\ref{fig:ej6} coincide con el diagrama que obtendríamos según EDF, teniendo en cuenta que en $t=2$ tanto $\tau_3$ como $\tau_4$ tienen su \textit{deadline} igual de próximo, por lo que da igual cual de las dos pasa a ejecución (en este caso hemos elegido a $\tau_3$). Algo similar sucede en $t=11$, con $\tau_2$, $\tau_3$ y $\tau_4$.
    \end{description}
\end{ejercicio}

\begin{ejercicio}\label{ej:rel4_7}
    Describe razonadamente si el siguiente conjunto de tareas puede planificarse o no puede planificarse en un sistema monoprocesador usando un ejecutivo cíclico o usando algún algoritmo basado en prioridades estáticas o dinámicas.
    \begin{table}[H]
    \centering
    \begin{tabular}{|c|c|c|}
        \hline
        Tarea & $C_i$ & $T_i$ \\
        \hline
        $\tau_1$ & 1 & 5 \\
        \hline
        $\tau_2$ & 1 & 10 \\
        \hline
        $\tau_3$ & 2 & 10 \\
        \hline
        $\tau_4$ & 11 & 20 \\
        \hline
        $\tau_5$ & 5 & 100 \\
        \hline
    \end{tabular}
    \caption{Tareas periódicas y sus atributos temporales.}
    \label{tab:4_7}
    \end{table}

    Calculamos el factor de utilización de la CPU\@:
    \begin{equation*}
        U = \sum_{i=1}^{5}\dfrac{C_i}{T_i} = \dfrac{1}{5} + \dfrac{1}{10} + \dfrac{2}{10} + \dfrac{11}{20} + \dfrac{5}{100} = \dfrac{1}{5} + \dfrac{1}{10} + \dfrac{1}{5} + \dfrac{11}{20} + \dfrac{1}{20} = 1.1 > 1
    \end{equation*}
    Por tanto, el conjunto de tareas no puede ser planificado por ningún esquema.
\end{ejercicio}

\subsubsection{Problemas adicionales}

\begin{ejercicio}\label{ej:rel4_8}
    Para el conjunto de tareas cuyos datos se muestran más abajo, se pide:
    \begin{itemize}
        \item Dibujar el gráfico de ejecución y obtener el tiempo de respuesta de cada tarea.
        \item Determinar, mediante inspección del gráfico anterior, cuántas veces interfiere la tarea $\mathcal{T}_1$ a la tarea $\mathcal{T}_3$ durante un intervalo temporal dado por el tiempo de respuesta de esta última tarea.
        \item Hacer lo mismo que en el apartado anterior pero para las tareas $\mathcal{T}_1$ y $\mathcal{T}_2$.
    \end{itemize}
    \begin{table}[H]
    \centering
    \begin{tabular}{|c|c|c|c|}
        \hline
        Tarea & $C_i$ & $T_i$ & $D_i$ \\
        \hline
        $\tau_1$ & 1 & 3 & 2 \\
        \hline
        $\tau_2$ & 3 & 6 & 5 \\
        \hline
        $\tau_3$ & 2 & 13 & 13 \\
        \hline
    \end{tabular}
    \caption{Tareas periódicas y sus atributos temporales.}
    \label{tab:4_8}
    \end{table}

    \noindent
    Suponiendo que nos piden dibujar el gráfico según el algoritmo EDF, tenemos que:
    \begin{equation*}
        mcm(3,6,13) = 78
    \end{equation*}
    \begin{figure}[H]
        \centering
        \begin{tikzpicture}[scale=0.2]
            % Eje de abscisas
            \draw[-Stealth] (0, 0) -- (79, 0) node[right] {};

            \foreach \x in {0, 4, ..., 76} {
                \pgfmathtruncatemacro{\xa}{\x + 2}
                \draw (\x, 0) -- (\x, -0.2) node[below] {\x};
                \draw (\xa, 0) -- (\xa, -0.2) node[below] {};
            }
            \draw (78, 0) -- (78, -0.2) node[below] {};

            % Eje de ordenadas
            \draw[-Stealth] (0, 0) -- (0, 8) node[above] {};
            \draw (0, 2) -- (-0.2, 2) node[left] {$\tau_1$};
            \draw (0, 4) -- (-0.2, 4) node[left] {$\tau_2$};
            \draw (0, 6) -- (-0.2, 6) node[left] {$\tau_3$};

            \draw[thick, gray] (0, 2) -- (1, 2);
            \draw[thick, gray] (1, 4) -- (4, 4);
            \draw[thick, gray] (4, 2) -- (5, 2);
            \draw[thick, gray] (5, 6) -- (6, 6);
            \draw[thick, gray] (6, 2) -- (7, 2);
            \draw[thick, gray] (7, 4) -- (10, 4);
            \draw[thick, gray] (10, 2) -- (11, 2);
            \draw[thick, gray] (11, 6) -- (12, 6);
            \draw[thick, gray] (12, 2) -- (13, 2);
            \draw[thick, gray] (13, 4) -- (16, 4);
            \draw[thick, gray] (16, 2) -- (17, 2);
            \draw[thick, gray] (17, 6) -- (18, 6);
            \draw[thick, gray] (18, 2) -- (19, 2);
            \draw[thick, gray] (19, 4) -- (22, 4);
            \draw[thick, gray] (22, 2) -- (23, 2);
            \draw[thick, gray] (23, 6) -- (24, 6);
            \draw[thick, gray] (24, 2) -- (25, 2);
            \draw[thick, gray] (25, 4) -- (28, 4);
            \draw[thick, gray] (28, 2) -- (29, 2);
            \draw[thick, gray] (29, 6) -- (30, 6);
            \draw[thick, gray] (30, 2) -- (31, 2);
            \draw[thick, gray] (31, 4) -- (34, 4);
            \draw[thick, gray] (34, 2) -- (35, 2);
            \draw[thick, gray] (35, 6) -- (36, 6);
            \draw[thick, gray] (36, 2) -- (37, 2);
            \draw[thick, gray] (37, 4) -- (40, 4);
            \draw[thick, gray] (40, 2) -- (41, 2);
            \draw[thick, gray] (41, 6) -- (42, 6);
            \draw[thick, gray] (42, 2) -- (43, 2);
            \draw[thick, gray] (43, 4) -- (46, 4);
            \draw[thick, gray] (46, 2) -- (47, 2);
            \draw[thick, gray] (47, 6) -- (48, 6);
            \draw[thick, gray] (48, 2) -- (49, 2);
            \draw[thick, gray] (49, 4) -- (52, 4);
            \draw[thick, gray] (52, 2) -- (53, 2);
            \draw[thick, gray] (53, 6) -- (54, 6);
            \draw[thick, gray] (54, 2) -- (55, 2);
            \draw[thick, gray] (55, 4) -- (58, 4);
            \draw[thick, gray] (59, 2) -- (60, 2);
            \draw[thick, gray] (60, 6) -- (61, 6);
            \draw[thick, gray] (61, 2) -- (62, 2);
            \draw[thick, gray] (62, 4) -- (65, 4);
            \draw[thick, gray] (65, 2) -- (66, 2);
            \draw[thick, gray] (66, 6) -- (67, 6);
            \draw[thick, gray] (67, 2) -- (68, 2);
            \draw[thick, gray] (68, 4) -- (71, 4);
            \draw[thick, gray] (71, 2) -- (72, 2);
            \draw[thick, gray] (72, 6) -- (73, 6);
            \draw[thick, gray] (73, 2) -- (74, 2);
            \draw[thick, gray] (74, 4) -- (77, 4);
            \draw[thick, gray] (77, 2) -- (78, 2);

            \fill (2, 2) circle (2pt);
            \draw[fill] (3, 2) -- ++(2pt, 0) -- ++(-1pt, 3pt) -- cycle;
            \fill (5, 2) circle (2pt);
            \draw[fill] (6, 2) -- ++(2pt, 0) -- ++(-1pt, 3pt) -- cycle;
            \fill (8, 2) circle (2pt);
            \draw[fill] (9, 2) -- ++(2pt, 0) -- ++(-1pt, 3pt) -- cycle;
            \fill (11, 2) circle (2pt);
            \draw[fill] (12, 2) -- ++(2pt, 0) -- ++(-1pt, 3pt) -- cycle;
            \fill (14, 2) circle (2pt);
            \draw[fill] (15, 2) -- ++(2pt, 0) -- ++(-1pt, 3pt) -- cycle;
            \fill (17, 2) circle (2pt);
            \draw[fill] (18, 2) -- ++(2pt, 0) -- ++(-1pt, 3pt) -- cycle;
            \fill (20, 2) circle (2pt);
            \draw[fill] (21, 2) -- ++(2pt, 0) -- ++(-1pt, 3pt) -- cycle;
            \fill (23, 2) circle (2pt);
            \draw[fill] (24, 2) -- ++(2pt, 0) -- ++(-1pt, 3pt) -- cycle;
            \fill (26, 2) circle (2pt);
            \draw[fill] (27, 2) -- ++(2pt, 0) -- ++(-1pt, 3pt) -- cycle;
            \fill (29, 2) circle (2pt);
            \draw[fill] (30, 2) -- ++(2pt, 0) -- ++(-1pt, 3pt) -- cycle;
            \fill (32, 2) circle (2pt);
            \draw[fill] (33, 2) -- ++(2pt, 0) -- ++(-1pt, 3pt) -- cycle;
            \fill (35, 2) circle (2pt);
            \draw[fill] (36, 2) -- ++(2pt, 0) -- ++(-1pt, 3pt) -- cycle;
            \fill (38, 2) circle (2pt);
            \draw[fill] (39, 2) -- ++(2pt, 0) -- ++(-1pt, 3pt) -- cycle;
            \fill (41, 2) circle (2pt);
            \draw[fill] (42, 2) -- ++(2pt, 0) -- ++(-1pt, 3pt) -- cycle;
            \fill (44, 2) circle (2pt);
            \draw[fill] (45, 2) -- ++(2pt, 0) -- ++(-1pt, 3pt) -- cycle;
            \fill (47, 2) circle (2pt);
            \draw[fill] (48, 2) -- ++(2pt, 0) -- ++(-1pt, 3pt) -- cycle;
            \fill (50, 2) circle (2pt);
            \draw[fill] (51, 2) -- ++(2pt, 0) -- ++(-1pt, 3pt) -- cycle;
            \fill (53, 2) circle (2pt);
            \draw[fill] (54, 2) -- ++(2pt, 0) -- ++(-1pt, 3pt) -- cycle;
            \fill (56, 2) circle (2pt);
            \draw[fill] (57, 2) -- ++(2pt, 0) -- ++(-1pt, 3pt) -- cycle;
            \fill (59, 2) circle (2pt);
            \draw[fill] (60, 2) -- ++(2pt, 0) -- ++(-1pt, 3pt) -- cycle;
            \fill (62, 2) circle (2pt);
            \draw[fill] (63, 2) -- ++(2pt, 0) -- ++(-1pt, 3pt) -- cycle;
            \fill (65, 2) circle (2pt);
            \draw[fill] (66, 2) -- ++(2pt, 0) -- ++(-1pt, 3pt) -- cycle;
            \fill (68, 2) circle (2pt);
            \draw[fill] (69, 2) -- ++(2pt, 0) -- ++(-1pt, 3pt) -- cycle;
            \fill (71, 2) circle (2pt);
            \draw[fill] (72, 2) -- ++(2pt, 0) -- ++(-1pt, 3pt) -- cycle;
            \fill (74, 2) circle (2pt);
            \draw[fill] (75, 2) -- ++(2pt, 0) -- ++(-1pt, 3pt) -- cycle;
            \fill (77, 2) circle (2pt);
            \draw[fill] (78, 2) -- ++(2pt, 0) -- ++(-1pt, 3pt) -- cycle;
            \fill (5, 4) circle (2pt);
            \draw[fill] (6, 4) -- ++(2pt, 0) -- ++(-1pt, 3pt) -- cycle;
            \fill (11, 4) circle (2pt);
            \draw[fill] (12, 4) -- ++(2pt, 0) -- ++(-1pt, 3pt) -- cycle;
            \fill (17, 4) circle (2pt);
            \draw[fill] (18, 4) -- ++(2pt, 0) -- ++(-1pt, 3pt) -- cycle;
            \fill (23, 4) circle (2pt);
            \draw[fill] (24, 4) -- ++(2pt, 0) -- ++(-1pt, 3pt) -- cycle;
            \fill (29, 4) circle (2pt);
            \draw[fill] (30, 4) -- ++(2pt, 0) -- ++(-1pt, 3pt) -- cycle;
            \fill (35, 4) circle (2pt);
            \draw[fill] (36, 4) -- ++(2pt, 0) -- ++(-1pt, 3pt) -- cycle;
            \fill (41, 4) circle (2pt);
            \draw[fill] (42, 4) -- ++(2pt, 0) -- ++(-1pt, 3pt) -- cycle;
            \fill (47, 4) circle (2pt);
            \draw[fill] (48, 4) -- ++(2pt, 0) -- ++(-1pt, 3pt) -- cycle;
            \fill (53, 4) circle (2pt);
            \draw[fill] (54, 4) -- ++(2pt, 0) -- ++(-1pt, 3pt) -- cycle;
            \fill (59, 4) circle (2pt);
            \draw[fill] (60, 4) -- ++(2pt, 0) -- ++(-1pt, 3pt) -- cycle;
            \fill (65, 4) circle (2pt);
            \draw[fill] (66, 4) -- ++(2pt, 0) -- ++(-1pt, 3pt) -- cycle;
            \fill (71, 4) circle (2pt);
            \draw[fill] (72, 4) -- ++(2pt, 0) -- ++(-1pt, 3pt) -- cycle;
            \fill (77, 4) circle (2pt);
            \draw[fill] (78, 4) -- ++(2pt, 0) -- ++(-1pt, 3pt) -- cycle;
            \fill (13, 6) circle (2pt);
            \fill (26, 6) circle (2pt);
            \fill (39, 6) circle (2pt);
            \fill (52, 6) circle (2pt);
            \fill (65, 6) circle (2pt);
            \fill (78, 6) circle (2pt);
        \end{tikzpicture}       
        \caption{Diagrama de Gantt para el Ejercicio~\ref{ej:rel4_8}\@.}
        \label{fig:ej8}
    \end{figure}
    \begin{itemize}
        \item $\tau_1$ interfiere a $\tau_3$:
            \begin{equation*}
                \left\lfloor \frac{T_3}{T_1} \right\rfloor = \left\lfloor \frac{13}{3} \right\rfloor = 4\text{\ veces}
            \end{equation*}
        \item $\tau_2$ interfiere a $\tau_3$:
            \begin{equation*}
                \left\lfloor \frac{T_3}{T_2} \right\rfloor = \left\lfloor \frac{13}{6} \right\rfloor = 2\text{\ veces}
            \end{equation*}
    \end{itemize}
    Calculamos ahora los tiempos de respuesta de cada tarea, sabiendo que:
    \begin{equation*}
        R_i = C_i + \sum_{j\neq i} \left\lfloor \frac{T_i}{T_j} \right\rfloor C_j
    \end{equation*}
    \begin{align*}
        R_1 &= C_1 = 1 \\
        R_2 &= C_2 + \left\lfloor \frac{T_2}{T_1} \right\rfloor C_1 =  3 + \left\lfloor \frac{6}{3} \right\rfloor \cdot 1 = 3 + 2 = 5 \\
        R_3 &= C_3 + \left\lfloor \frac{T_3}{T_1} \right\rfloor C_1 + \left\lfloor \frac{T_3}{T_2} \right\rfloor C_2 =  2 + \left\lfloor \frac{13}{3} \right\rfloor \cdot 1 + \left\lfloor \frac{13}{6}\right\rfloor \cdot 3 = 2 + 4\cdot 1 + 2\cdot 3  = 12
    \end{align*}
\end{ejercicio}

\begin{ejercicio}\label{ej:rel4_9}
    Verificar la planificabilidad del siguiente conjunto de tareas utilizando para ello el algoritmo del ``primero el del tiempo límite más cercano'' (EDF).
    \begin{table}[H]
    \centering
    \begin{tabular}{|c|c|c|}
        \hline
        Tarea & $C_i$ & $T_i$ \\
        \hline
        $\tau_1$ & 1 & 4 \\
        \hline
        $\tau_2$ & 2 & 6 \\
        \hline
        $\tau_3$ & 3 & 8 \\
        \hline
    \end{tabular}
    \caption{Tareas periódicas y sus atributos temporales.}
    \label{tab:4_9}
    \end{table}

    Calculamos primero el factor de utilización de la CPU\@:
    \begin{equation*}
        U = \sum_{i=1}^{3} \dfrac{C_i}{T_i} = \dfrac{1}{4} + \dfrac{2}{6} + \dfrac{3}{8} = \dfrac{1}{4} + \dfrac{1}{3} + \dfrac{3}{8} = 0.95 \leq 1
    \end{equation*}
    Por lo que el conjunto de tareas es planificable por EDF, tal y como vemos en la Figura~\ref{fig:ej9} (considerando $mcm(4,6,8)=24$).
    \begin{figure}[H]
        \centering
        \begin{tikzpicture}[scale=0.6]
            % Eje de abscisas
            \draw[-Stealth] (0, 0) -- (25, 0) node[right] {};

            \foreach \x in {0, 2, ..., 22} {
                \pgfmathtruncatemacro{\xa}{\x + 1}
                \draw (\x, 0) -- (\x, -0.2) node[below] {\x};
                \draw (\xa, 0) -- (\xa, -0.2) node[below] {};
            }
            \draw (24, 0) -- (24, -0.2) node[below] {24};

            % Eje de ordenadas
            \draw[-Stealth] (0, 0) -- (0, 8) node[above] {};
            \draw (0, 2) -- (-0.2, 2) node[left] {$\tau_1$};
            \draw (0, 4) -- (-0.2, 4) node[left] {$\tau_2$};
            \draw (0, 6) -- (-0.2, 6) node[left] {$\tau_3$};

            \draw[thick, gray] (0, 2) -- (1, 2);
            \draw[thick, gray] (1, 4) -- (3, 4);
            \draw[thick, gray] (3, 6) -- (6, 6);
            \draw[thick, gray] (6, 2) -- (7, 2);
            \draw[thick, gray] (7, 4) -- (9, 4);
            \draw[thick, gray] (9, 2) -- (10, 2);
            \draw[thick, gray] (10, 6) -- (13, 6);
            \draw[thick, gray] (13, 2) -- (14, 2);
            \draw[thick, gray] (14, 4) -- (16, 4);
            \draw[thick, gray] (16, 2) -- (17, 2);
            \draw[thick, gray] (17, 6) -- (20, 6);
            \draw[thick, gray] (20, 2) -- (21, 2);
            \draw[thick, gray] (21, 4) -- (23, 4);

            \fill (4, 2) circle (2pt);
            \fill (8, 2) circle (2pt);
            \fill (12, 2) circle (2pt);
            \fill (16, 2) circle (2pt);
            \fill (20, 2) circle (2pt);
            \fill (24, 2) circle (2pt);
            \fill (6, 4) circle (2pt);
            \fill (12, 4) circle (2pt);
            \fill (18, 4) circle (2pt);
            \fill (24, 4) circle (2pt);
            \fill (8, 6) circle (2pt);
            \fill (16, 6) circle (2pt);
            \fill (24, 6) circle (2pt);
        \end{tikzpicture}       
        \caption{Diagrama de Gantt para el Ejercicio~\ref{ej:rel4_9}\@.}
        \label{fig:ej9}
    \end{figure}
\end{ejercicio}

\begin{ejercicio}\label{ej:rel4_10}
    Verificar la planificabilidad utilizando el algoritmo EDF de asignación dinámica de prioridades a las tareas y construir el diagrama de ejecución de tareas del siguiente conjunto:
    \begin{table}[H]
    \centering
    \begin{tabular}{|c|c|c|c|}
        \hline
        Tarea & $C_i$ & $D_i$ & $T_i$ \\
        \hline
        $\tau_1$ & 2 & 5 & 6 \\
        \hline
        $\tau_2$ & 2 & 4 & 8 \\
        \hline
        $\tau_3$ & 4 & 8 & 12 \\
        \hline
    \end{tabular}
    \caption{Tareas periódicas y sus atributos temporales.}
    \label{tab:4_10}
    \end{table}

    Calculamos el factor de utilización de la CPU\@:
    \begin{equation*}
        U = \sum_{i=1}^{3} \dfrac{C_i}{T_i} = \dfrac{2}{6} + \dfrac{2}{8} + \dfrac{4}{12} = \dfrac{1}{3} + \dfrac{1}{4} + \dfrac{1}{3} \leq 1
    \end{equation*}
    Por lo que el conjunto de tareas es planificable por EDF, tal y como vemos en la Figura~\ref{fig:ej10} (considerando $mcm(6,8,12)=24$).
    \begin{figure}[H]
        \centering
        \begin{tikzpicture}[scale=0.6]
            % Eje de abscisas
            \draw[-Stealth] (0, 0) -- (25, 0) node[right] {};

            \foreach \x in {0, 2, ..., 22} {
                \pgfmathtruncatemacro{\xa}{\x + 1}
                \draw (\x, 0) -- (\x, -0.2) node[below] {\x};
                \draw (\xa, 0) -- (\xa, -0.2) node[below] {};
            }
            \draw (24, 0) -- (24, -0.2) node[below] {24};

            % Eje de ordenadas
            \draw[-Stealth] (0, 0) -- (0, 8) node[above] {};
            \draw (0, 2) -- (-0.2, 2) node[left] {$\tau_1$};
            \draw (0, 4) -- (-0.2, 4) node[left] {$\tau_2$};
            \draw (0, 6) -- (-0.2, 6) node[left] {$\tau_3$};

            \draw[thick, gray] (0, 4) -- (2, 4);
            \draw[thick, gray] (2, 2) -- (4, 2);
            \draw[thick, gray] (4, 6) -- (8, 6);
            \draw[thick, gray] (8, 2) -- (10, 2);
            \draw[thick, gray] (10, 4) -- (12, 4);
            \draw[thick, gray] (12, 2) -- (14, 2);
            \draw[thick, gray] (14, 6) -- (18, 6);
            \draw[thick, gray] (18, 4) -- (20, 4);
            \draw[thick, gray] (20, 2) -- (22, 2);

            \fill (5, 2) circle (2pt);
            \draw[fill] (6, 2) -- ++(2pt, 0) -- ++(-1pt, 3pt) -- cycle;
            \fill (11, 2) circle (2pt);
            \draw[fill] (12, 2) -- ++(2pt, 0) -- ++(-1pt, 3pt) -- cycle;
            \fill (17, 2) circle (2pt);
            \draw[fill] (18, 2) -- ++(2pt, 0) -- ++(-1pt, 3pt) -- cycle;
            \fill (23, 2) circle (2pt);
            \draw[fill] (24, 2) -- ++(2pt, 0) -- ++(-1pt, 3pt) -- cycle;
            \fill (4, 4) circle (2pt);
            \draw[fill] (8, 4) -- ++(2pt, 0) -- ++(-1pt, 3pt) -- cycle;
            \fill (12, 4) circle (2pt);
            \draw[fill] (16, 4) -- ++(2pt, 0) -- ++(-1pt, 3pt) -- cycle;
            \fill (20, 4) circle (2pt);
            \draw[fill] (24, 4) -- ++(2pt, 0) -- ++(-1pt, 3pt) -- cycle;
            \fill (8, 6) circle (2pt);
            \draw[fill] (12, 6) -- ++(2pt, 0) -- ++(-1pt, 3pt) -- cycle;
            \fill (20, 6) circle (2pt);
            \draw[fill] (24, 6) -- ++(2pt, 0) -- ++(-1pt, 3pt) -- cycle;
        \end{tikzpicture}       
        \caption{Diagrama de Gantt para el Ejercicio~\ref{ej:rel4_10}\@.}
        \label{fig:ej10}
    \end{figure}
\end{ejercicio}

\begin{ejercicio}\label{ej:rel4_11}
    Verificar la planificabilidad del conjunto de tareas descrito en el Ejercicio~\ref{ej:rel4_10} utilizando el algoritmo del ``plazo de respuesta máximo (D)'' (algoritmo \textit{deadline monotomic} o DM).\\

    Para ello, mostramos en la Figura~\ref{fig:ej11} el cronograma, donde vemos que el algoritmo falla ya que incumple el primer \textit{deadline} para la tarea $\tau_3$, al no poder completar su ejecución dentro de plazo.
    \begin{figure}[H]
        \centering
        \begin{tikzpicture}[scale=0.6]
            % Eje de abscisas
            \draw[-Stealth] (0, 0) -- (25, 0) node[right] {};

            \foreach \x in {0, 2, ..., 22} {
                \pgfmathtruncatemacro{\xa}{\x + 1}
                \draw (\x, 0) -- (\x, -0.2) node[below] {\x};
                \draw (\xa, 0) -- (\xa, -0.2) node[below] {};
            }
            \draw (24, 0) -- (24, -0.2) node[below] {24};

            % Eje de ordenadas
            \draw[-Stealth] (0, 0) -- (0, 8) node[above] {};
            \draw (0, 2) -- (-0.2, 2) node[left] {$\tau_1$};
            \draw (0, 4) -- (-0.2, 4) node[left] {$\tau_2$};
            \draw (0, 6) -- (-0.2, 6) node[left] {$\tau_3$};

            \draw[thick, gray] (0, 4) -- (2, 4);
            \draw[thick, gray] (2, 2) -- (4, 2);
            \draw[thick, gray] (4, 6) -- (6, 6);
            \draw[thick, gray] (6, 2) -- (8, 2);

            \fill (5, 2) circle (2pt);
            \draw[fill] (6, 2) -- ++(2pt, 0) -- ++(-1pt, 3pt) -- cycle;
            \fill (11, 2) circle (2pt);
            \draw[fill] (12, 2) -- ++(2pt, 0) -- ++(-1pt, 3pt) -- cycle;
            \fill (17, 2) circle (2pt);
            \draw[fill] (18, 2) -- ++(2pt, 0) -- ++(-1pt, 3pt) -- cycle;
            \fill (23, 2) circle (2pt);
            \draw[fill] (24, 2) -- ++(2pt, 0) -- ++(-1pt, 3pt) -- cycle;
            \fill (4, 4) circle (2pt);
            \draw[fill] (8, 4) -- ++(2pt, 0) -- ++(-1pt, 3pt) -- cycle;
            \fill (12, 4) circle (2pt);
            \draw[fill] (16, 4) -- ++(2pt, 0) -- ++(-1pt, 3pt) -- cycle;
            \fill (20, 4) circle (2pt);
            \draw[fill] (24, 4) -- ++(2pt, 0) -- ++(-1pt, 3pt) -- cycle;
            \fill (8, 6) circle (2pt);
            \draw[fill] (12, 6) -- ++(2pt, 0) -- ++(-1pt, 3pt) -- cycle;
            \fill (20, 6) circle (2pt);
            \draw[fill] (24, 6) -- ++(2pt, 0) -- ++(-1pt, 3pt) -- cycle;
        \end{tikzpicture}       
        \caption{Diagrama de Gantt para el Ejercicio~\ref{ej:rel4_11}\@.}
        \label{fig:ej11}
    \end{figure}
\end{ejercicio}

\begin{ejercicio}\label{ej:rel4_12}
    Indicar cuáles de las siguientes afirmaciones son correctas respecto de los algoritmos para resolver el problema de \textit{inversión de prioridad} de las tareas en sistemas de tiempo real de \textit{misión crítica}:
    \begin{enumerate}[label=(\alph*)]
        \item Suponiendo que las tareas se planifican con el protocolo de \textit{herencia de prioridad}: la prioridad heredada por una tarea sólo se mantiene mientras dicha tarea esté utilizando un recurso compartido con otra tarea más prioritaria.
        \item Con el protocolo de \textit{techo de prioridad}, cuando una tarea adquiere un recurso no puede verse interrumpida, hasta que termine su ejecución, por otras tareas que se activan después que ésta y que vayan a utilizar en el futuro un recurso con límite de prioridad igual o inferior.
        \item Con el protocolo de techo de prioridad (PPP), una tarea no puede comenzar a ejecutarse si no están libres todos los recursos que va a utilizar durante su primer ciclo.
        \item Si consideramos una tarea periódica que utilice el protocolo de techo de prioridad inmediato para cambiar su prioridad dinámica cuando accede a recursos, siempre se cumplirá que dicha tarea no puede ser interrumpida por otra menos prioritaria que ella.
        \item Con el protocolo de techo de prioridad las tareas más prioritarias del sistema pueden ser interrumpidas durante cada ciclo de su ejecución como máximo 1 vez cuando acceden a recursos que comparten con otras tareas menos prioritarias.
        \item El protocolo de techo de prioridad original producirá siempre tiempos de respuesta menores para las tareas que el algoritmo de \textit{herencia de prioridad}.
    \end{enumerate}

    Las respuestas son:
    \begin{enumerate}[label=(\alph*)]
        \item Falsa.
        \item Verdadera.
        \item Verdadera.
        \item Falsa.
        \item Verdadera.
        \item Falsa.
    \end{enumerate}
\end{ejercicio}

\begin{ejercicio}\label{ej:rel4_13}
    Calcular la utilización máxima del procesador que se puede asignar al \textit{Servidor Esporádico} para garantizar la planificabilidad del siguiente conjunto de tareas periódicas utilizando RM.
    \begin{table}[H]
    \centering
    \begin{tabular}{|c|c|c|}
        \hline
        $\mathcal{T}_1$ & 1 & 5 \\
        \hline
        $\mathcal{T}_2$ & 2 & 8 \\
        \hline
    \end{tabular}
    \caption{Conjunto de tareas periódicas.}
    \label{tab:4_13}
    \end{table}
\end{ejercicio}

\begin{ejercicio}\label{ej:rel4_14}
    Calcular la utilización máxima del procesador que puede ser asignada al \textit{Servidor Diferido} (SD) para garantizar la planificabilidad del conjunto de tareas periódicas dado en el Ejercicio~\ref{ej:rel4_13} anterior.\\

    Sabemos que el factor de utilización de la CPU por parte de las tareas periódicas del Ejercicio~\ref{ej:rel4_13} es:
    \begin{equation*}
        U_p = \dfrac{1}{5} + \dfrac{2}{8} = \dfrac{1}{5} + \dfrac{1}{4} = 0.45
    \end{equation*}
    Por tanto, buscamos el mayor valor de $U_s$ que nos cumpla la desigualdad (para $n=2$):
    \begin{equation*}
        U_p \leq n\cdot \left({\left(\dfrac{U_s+2}{2U_s+1}\right)}^{\frac{1}{n}}-1\right)
    \end{equation*}
    Despejando $U_s$:
    \begin{gather*}
        U_p \leq n\cdot \left({\left(\dfrac{U_s+2}{2U_s+1}\right)}^{\frac{1}{n}}-1\right) \\
        {\left(\dfrac{U_p}{n}+1\right)}^{n} \leq \dfrac{U_s+2}{2U_s+1} \\
        (2Us+1){\left(\dfrac{U_p}{n}+1\right)}^{n} \leq U_s+2 \\
        U_s\left[2{\left(\dfrac{U_p}{n}+1\right)}^{n}-1\right] \leq 2-{\left(\dfrac{U_p}{n}+1\right)}^{n} 
    \end{gather*}
    Llegamos a que:
    \begin{equation*}
        U_s \leq \dfrac{2-{\left(\dfrac{U_p}{n}+1\right)}^{n}}{2{\left(\dfrac{U_p}{n}+1\right)}^{n}-1} = \dfrac{2-{\left(\dfrac{0.45}{2}+1\right)}^{2}}{2{\left(\dfrac{0.45}{2}+1\right)}^{2}-1} = \dfrac{2-1.5}{2\cdot 1.5-1} = 0.25
    \end{equation*}
    Por tanto, la utilización máxima del procesador que puede ser asignada al servidor diferido es de $0.25$.
\end{ejercicio}

\begin{ejercicio}\label{ej:rel4_15}
    Junto con las tareas periódicas que se muestran en el Ejercicio~\ref{ej:rel4_13} definir un plan para planificar las siguientes tareas aperiódicas utilizando un \textit{SD} (tarea sondeante) que posea una utilización máxima del tiempo del procesador y prioridad intermedia:
    \begin{table}[H]
    \centering
    \begin{tabular}{|c|c|c|}
        \hline
        & $t_a$ & $C_i$ \\
        \hline
        $J_1$ & 2 & 2 \\
        \hline
        $J_2$ & 7 & 2 \\
        \hline
        $J_3$ & 17 & 1 \\
        \hline
    \end{tabular}
    \caption{Tareas periódicas y sus atributos temporales.}
    \label{tab:4_15}
    \end{table}

    Buscamos unos valores $C_s$ y $T_s$ de una nueva tarea sondeante $\tau_s$ que nos permita cumplir el test de Liu y Layland junto a las tareas del Ejercicio~\ref{ej:rel4_13}:
    \begin{equation*}
        U_p + \dfrac{C_s}{T_s} = 0.45 + \dfrac{C_s}{T_s} \leq U_0(3) = 0.779
    \end{equation*}
    Por tanto, el mayor factor de utilización de la CPU por parte de la tarea sondeante será de:
    \begin{equation*}
        \dfrac{C_s}{T_s} \leq 0.779 - 0.45 = 0.33
    \end{equation*}
    Como la prioridad de la tarea sondeante no tiene nada que ver con $T_s$, podemos elegir a nuestro gusto $C_s$ y $T_s$. Elegimos $C_s$ de acorde con el mayor tiempo de cómputo, luego escogemos $C_s = C_1 = 2$ y $T_s = 6$.
\end{ejercicio}

\begin{ejercicio}\label{ej:rel4_16}
    Resolver ahora el mismo problema de planificación descrito en el Ejercicio~\ref{ej:rel4_15} utilizando ahora un \textit{Servidor Esporádico} que tenga una utilización máxima y prioridad intermedia.
\end{ejercicio}

\begin{ejercicio}\label{ej:rel4_17}
    Resolver el mismo problema de planificación descrito en el Ejercicio~\ref{ej:rel4_15} utilizando ahora un \textit{Servidor Diferido} que tenga una utilización máxima y prioridad intermedia.\\

    Basta elegir unos valores de $C_s$ y $T_s$ de forma que $\frac{C_s}{T_s} = 0.25$, ya que en el Ejercicio~\ref{ej:rel4_14} vimos que la utilización máxima que podía tener el servidor diferido era $0.25$. Elegimos $C_s$ de forma que tenga el valor del mayor tiempo de cómputo de las tareas aperiódicas a las que sirva, por lo que elegimos $C_s=2$ y $T_s = 8$.
\end{ejercicio}

\begin{ejercicio}\label{ej:rel4_18}
    Utilizar un \textit{Servidor Esporádico} con capacidad $C_s=2$ y periodo $T_s=6$ para planificar las siguiente tareas:
    \begin{table}[H]
    \centering
    \begin{tabular}{|c|c|c|}
        \hline
        & $C_i$ & $T_i$ \\
        \hline
        $\mathcal{T}_1$ & 1 & 4 \\
        \hline
        $\mathcal{T}_2$ & 2 & 6 \\
        \hline
                        & $a_i$ & $C_i$ \\
        \hline
        $J_1$ & 2 & 2 \\
        \hline
        $J_2$ & 5 & 1 \\
        \hline
        $J_3$ & 10 & 2 \\
        \hline
    \end{tabular}
    \caption{Tareas periódicas y sus atributos temporales.}
    \label{tab:4_18}
    \end{table}
\end{ejercicio}

\newpage
\chapter{Ecuaciones y sistemas}

\begin{definicion}[Ecuación Diferencial y solución]
    Una ecuación diferencial viene dada por una función
    \Func{\Phi}{D\subseteq \mathbb{R}^3}{\mathbb{R}}{(t,x,y)}{\Phi(t,x,y)}
    \textbf{continua} donde $D$ es un \textbf{abierto}\footnote{Por convenio, ya que las ecuaciones diferenciales no se comportan bien en los bordes.} \textbf{conexo}\footnote{Es lógico pensarlo, ya que estamos estudiando movimientos.} de $\mathbb{R}^3$.

\noindent
Una \textbf{solución} de dicha ecuación diferencial será una función
\Func{x}{I}{\mathbb{R}}{t}{$x(t)$}
con $I\subseteq \mathbb{R}^2$ \textbf{intervalo}\footnote{Es lógico, pues usualmente $t$ será el tiempo.} \textbf{abierto}\footnote{Pudiendo trabajar con funciones continuas en un cerrado y derivables en el abierto, evitando así por ejemplo tangentes verticales} tal que:
\begin{enumerate}[label=\roman*)]
    \item $x$ sea derivable en $I$\footnote{Necesario para poder considerar $x'(t)$ en la propia ecuación.}.
    \item $(t,x(t),x'(t)) \in D \quad\forall t\in I$\footnote{En vistas de III).}.
    \item $\Phi(t,x(t),x'(t))=0 \quad \forall t\in I$.
\end{enumerate}
\end{definicion}

\begin{ejemplo}
    Dada la ecuación diferencial:
    \begin{equation*}
        x'(t) = \dfrac{1}{x(t)}
    \end{equation*}
    y la expresión:
    \begin{equation*}
        x(t) = \sqrt{2t-38}
    \end{equation*}
    Probar que dicha expresión es solución de la ecuación diferencial.\\

    La ecuación diferencial viene dada por
    \Func{\Phi}{\mathbb{R}\times\mathbb{R}^+\times \mathbb{R}}{\mathbb{R}}{(t,x,y)}{y-\frac{1}{x}}
    \begin{itemize}
        \item Cogiendo $I = \left]19, +\infty\right[$, tenemos I).
        \item Por ser la raíz una función continua y creciente, tenemos que ${x(I) = \left]0,+\infty\right[=\mathbb{R}^+}$, de donde se cumple II).
        \item 
            \begin{equation*}
                x'(t) = \dfrac{1}{\sqrt{2t-38}} = \dfrac{1}{x(t)} \qquad \forall t\in I
            \end{equation*}
    \end{itemize}
\end{ejemplo}

\begin{notacion}
    La notación que hemos estado utilizando para las ecuaciones diferenciales no es la que usaremos a lo largo del curso:\\

    Lo que hasta ahora hemos notado y entendido por:
    \begin{equation*}
        \Phi(t,x(t),x'(t)) = 0
    \end{equation*}

    Como en el caso:
    \begin{equation*}
      x'(t) = 3x(t)  
    \end{equation*}
    Lo notaremos ahora por:
    \begin{gather*}
        \Phi(t,x,x') = 0\\
        x' = 3x
    \end{gather*}

    Que recordamos tiene por solución:
    \begin{equation*}
        x(t) = c \cdot e^{3t}, \quad c \in \mathbb{R}
    \end{equation*}
\end{notacion}

\begin{ejemplo}
    Para cierto $\lm \in \mathbb{R}$, resolver:
    \begin{enumerate}
        \item $x' = \lm x$
            \begin{equation*}
                x(t) = c \cdot e^{\lm t} \quad c \in \mathbb{R}
            \end{equation*}
        \item $x' = \lm t$
            \begin{equation*}
                x(t) = \frac{\lm}{2} t^2 + c \quad c \in \mathbb{R}
            \end{equation*}
    \end{enumerate}
\end{ejemplo}

\section{Crecimiento proporcional}
En el capítulo anterior nos preguntábamos si todas las soluciones de la ecuación $x'=\lm x$ para cierto $\lm \in \mathbb{R}$ eran de la forma
\begin{equation*}
    x(t) = c \cdot e^{\lm t} \quad c\in \mathbb{R}
\end{equation*}
Ahora daremos respuesta a dicha cuestión, comentando además utilidades de la misma ecuación.

\begin{prop}
    Dada $x(t)$, solución de $x'=\lm x$ para cierto $\lm \in \mathbb{R}$, definida en un intervalo abierto $I$, existe $c\in \mathbb{R}$ tal que
    \begin{equation*}
        x(t) = c \cdot e^{\lm t} \quad t\in I
    \end{equation*}
\end{prop}
Antes de dar paso a la demostración, observemos que si suponemos cierta la tesis:
\begin{equation*}
    x(t) = c \cdot e^{\lm t} \Longleftrightarrow  e^{-\lm t} x(t) = c \Longrightarrow \dfrac{d}{dt}(e^{-\lm t}x(t)) = 0
\end{equation*}
\begin{proof}
    Definimos la función
    \Func{f}{I}{\mathbb{R}}{t}{e^{-\lm t}x(t)}
    que es derivable por ser producto de funciones derivables.
    \begin{align*}
        \dfrac{df(t)}{dt} = \dfrac{d}{dt}(e^{-\lm t}x(t)) &= -\lm e^{-\lm t} x(t) + e^{-\lm t} x'(t)\\
                          &= -\lm e^{-\lm t}x(t) + \lm e^{-\lm t}x(t) = 0 \quad \forall t\in I
    \end{align*}
    Por ser $f$ continua y definida en un intervalo, llegamos a que es constante, luego $\exists c\in \mathbb{R} \mid f(t) = e^{-\lm t}x(t) = c \quad \forall t\in I$, de donde deducimos que:
    \begin{equation*}
        x(t) = c \cdot e^{\lm t} \quad \forall t\in I
    \end{equation*}
\end{proof}

% // TODO: Copiar lo del capital y lo de desintegración

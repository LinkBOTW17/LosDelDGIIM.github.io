\newpage
\chapter{Ecuaciones y sistemas}

\begin{definicion}[Ecuación Diferencial y solución]
    Una ecuación diferencial viene dada por una función
    \Func{\Phi}{D\subseteq \mathbb{R}^3}{\mathbb{R}}{(t,x,y)}{\Phi(t,x,y)}
    \textbf{continua} donde $D$ es un \textbf{abierto}\footnote{Por convenio, ya que las ecuaciones diferenciales no se comportan bien en los bordes.} \textbf{conexo}\footnote{Es lógico pensarlo, ya que estamos estudiando movimientos.} de $\mathbb{R}^3$.

\noindent
Una \textbf{solución} de dicha ecuación diferencial será una función
\Func{x}{I}{\mathbb{R}}{t}{$x(t)$}
con $I\subseteq \mathbb{R}^2$ \textbf{intervalo}\footnote{Es lógico, pues usualmente $t$ será el tiempo.} \textbf{abierto}\footnote{Pudiendo trabajar con funciones continuas en un cerrado y derivables en el abierto, evitando así por ejemplo tangentes verticales} tal que:
\begin{enumerate}[label=\roman*)]
    \item $x$ sea derivable en $I$\footnote{Necesario para poder considerar $x'(t)$ en la propia ecuación.}.
    \item $(t,x(t),x'(t)) \in D \quad\forall t\in I$\footnote{En vistas de III).}.
    \item $\Phi(t,x(t),x'(t))=0 \quad \forall t\in I$.
\end{enumerate}
\end{definicion}

\begin{ejemplo}
    Dada la ecuación diferencial:
    \begin{equation*}
        x'(t) = \dfrac{1}{x(t)}
    \end{equation*}
    y la expresión:
    \begin{equation*}
        x(t) = \sqrt{2t-38}
    \end{equation*}
    Probar que dicha expresión es solución de la ecuación diferencial.\\

    La ecuación diferencial viene dada por
    \Func{\Phi}{\mathbb{R}\times\mathbb{R}^+\times \mathbb{R}}{\mathbb{R}}{(t,x,y)}{y-\frac{1}{x}}
    \begin{itemize}
        \item Cogiendo $I = \left]19, +\infty\right[$, tenemos I).
        \item Por ser la raíz una función continua y creciente, tenemos que ${x(I) = \left]0,+\infty\right[=\mathbb{R}^+}$, de donde se cumple II).
        \item 
            \begin{equation*}
                x'(t) = \dfrac{1}{\sqrt{2t-38}} = \dfrac{1}{x(t)} \qquad \forall t\in I
            \end{equation*}
    \end{itemize}
\end{ejemplo}

\begin{notacion}
    La notación que hemos estado utilizando para las ecuaciones diferenciales no es la que usaremos a lo largo del curso:\\

    Lo que hasta ahora hemos notado y entendido por:
    \begin{equation*}
        \Phi(t,x(t),x'(t)) = 0
    \end{equation*}

    Como en el caso:
    \begin{equation*}
      x'(t) = 3x(t)  
    \end{equation*}
    Lo notaremos ahora por:
    \begin{gather*}
        \Phi(t,x,x') = 0\\
        x' = 3x
    \end{gather*}

    Que recordamos tiene por solución:
    \begin{equation*}
        x(t) = c \cdot e^{3t}, \quad c \in \mathbb{R}
    \end{equation*}
\end{notacion}

\begin{ejemplo}
    Para cierto $\lm \in \mathbb{R}$, resolver:
    \begin{enumerate}
        \item $x' = \lm x$
            \begin{equation*}
                x(t) = c \cdot e^{\lm t} \quad c \in \mathbb{R}
            \end{equation*}
        \item $x' = \lm t$
            \begin{equation*}
                x(t) = \frac{\lm}{2} t^2 + c \quad c \in \mathbb{R}
            \end{equation*}
    \end{enumerate}
\end{ejemplo}

\section{Crecimiento proporcional}


En el capítulo anterior nos preguntábamos si todas las soluciones de la ecuación $x'=\lm x$ para cierto $\lm \in \mathbb{R}$ eran de la forma
\begin{equation*}
    x(t) = c \cdot e^{\lm t} \quad c\in \mathbb{R}
\end{equation*}
Ahora daremos respuesta a dicha cuestión, comentando además utilidades de la misma ecuación.

\begin{prop}
    Dada $x(t)$, solución de $x'=\lm x$ para cierto $\lm \in \mathbb{R}$, definida en un intervalo abierto $I$, existe $c\in \mathbb{R}$ tal que
    \begin{equation*}
        x(t) = c \cdot e^{\lm t} \quad t\in I
    \end{equation*}
\end{prop}
Antes de dar paso a la demostración, observemos que si suponemos cierta la tesis:
\begin{equation*}
    x(t) = c \cdot e^{\lm t} \Longleftrightarrow  e^{-\lm t} x(t) = c \Longrightarrow \dfrac{d}{dt}(e^{-\lm t}x(t)) = 0
\end{equation*}
\begin{proof}
    Definimos la función
    \Func{f}{I}{\mathbb{R}}{t}{e^{-\lm t}x(t)}
    que es derivable por ser producto de funciones derivables.
    \begin{align*}
        \dfrac{df(t)}{dt} = \dfrac{d}{dt}(e^{-\lm t}x(t)) &= -\lm e^{-\lm t} x(t) + e^{-\lm t} x'(t)\\
                          &= -\lm e^{-\lm t}x(t) + \lm e^{-\lm t}x(t) = 0 \quad \forall t\in I
    \end{align*}
    Por ser $f$ continua y definida en un intervalo, llegamos a que es constante, luego $\exists c\in \mathbb{R} \mid f(t) = e^{-\lm t}x(t) = c \quad \forall t\in I$, de donde deducimos que:
    \begin{equation*}
        x(t) = c \cdot e^{\lm t} \quad \forall t\in I
    \end{equation*}
\end{proof}

Nos centraremos ahora en la idea intuitiva de derivada como representación de la variación de una variable dependiente para dar lugar a los siguientes dos ejemplos que nos muestran cómo surge la ecuación diferencial $x' = \lm x$ para cierto $\lm \in \mathbb{R}$ de forma natural.

\begin{ejemplo}
    Si ingresamos un capital inicial $C(0) = 100$ en un banco que nos da el 2\% anual de interés, al cabo de un año tendremos en nuestra cuenta:
    \begin{equation*}
        C(1) = C(0) + \dfrac{2}{100}C(0) = 100 + \dfrac{2}{100}\cdot 100 = 102
    \end{equation*}
Sin embargo, si acudimos a otro banco que nos ofrece la misma tasa anual de interés pero que nos realiza pagos semestrales, al cabo de un año conseguiremos reunir:
\begin{align*}
    C(\nicefrac{1}{2}) &= C(0) + \dfrac{1}{2}\dfrac{2}{100}C(0) = 100 + \dfrac{1}{2}\dfrac{2}{100}\cdot 100 = 101 \\
    C(1) &= C(\nicefrac{1}{2}) + \dfrac{1}{2}\dfrac{2}{100}C(\nicefrac{1}{2}) = 101 + \dfrac{1}{2}\dfrac{2}{100}\cdot 101 = 102.01 > 102
\end{align*}
En general, si $\Delta t$ es la fracción del año en la que se nos hacen los pagos ($\Delta t = 1$ significa que es pago anual), la fórmula es:
\begin{equation*}
    C(t+\Delta t) = C(t) + \dfrac{2}{100}\Delta t C(t)
\end{equation*}

de donde:
\begin{equation*}
    \dfrac{C(t+\Delta t) - C(t)}{\Delta t} = \dfrac{2}{100} C(t)
\end{equation*}
Si ahora vamos a un banco que nos haga pagos continuos, dicha fracción de tiempo será mínima, por lo que podemos pensar que $\Delta t \rightarrow 0$ y hacer un paso al límite (no riguroso) para obtener que:
\begin{equation*}
    \lim_{\Delta t\to0} \dfrac{C(t+\Delta t) - C(t)}{\Delta t} = C'(t) = \dfrac{2}{100}C(t)
\end{equation*}
De donde tenemos
\begin{equation*}
    C' = \dfrac{2}{100} C
\end{equation*}

o de forma general, si notamos por $I\in \mathbb{R}$ a la tasa de interés:
\begin{equation*}
    C' = I C
\end{equation*}
Cuyas soluciones ya sabemos que se tratan de cualquier función de la familia:
\begin{equation*}
    C(t) = c \cdot e^{It} \quad c\in \mathbb{R}
\end{equation*}
Por tanto, el crecimiento del dinero en un banco que nos ofrezca un interés continuo es exponencial. En este ejemplo, vemos cómo el crecimiento del dinero en una cuenta bancaria es exponencial a la propia cantidad de dinero de la que disponemos en la misma.
\end{ejemplo}

\begin{ejemplo}
    La teoría física de la radioactividad nos dice que las sustancias radioactivas van perdiendo masa a una velocidad proporcional a la masa de la propia sustancia. Es decir, si representamos la masa de una sustancia radioactiva como variable dependiente del tiempo $m=m(t)$:
    \begin{equation*}
        m'(t) = -\lm m(t)
    \end{equation*}
    Para cierto parámetro $\lm \in \mathbb{R}^+$ relativo a la sustancia radioaciva que consideremos.

    Gracias al estudio anteriormente realizado de la ecuación diferencial $m' = -\lm m$, sabemos ya que soluciones de esta son cualesquiera funciones de la forma:
    \begin{equation*}
        m(t) = c\cdot e^{-\lm t}\quad c\in \mathbb{R}^+
    \end{equation*}
En este ejemplo, hemos vuelto a observar la interpretación de la derivada como cuantía de la variación de una determinada variable dependiente.
\end{ejemplo}

Más adelante en estos mismos apuntes podremos observer el uso de la interpretación de la derivada como pendiente de la recta tangente a una curva.

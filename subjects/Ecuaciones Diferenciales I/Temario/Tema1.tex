\newpage
\chapter{Ecuaciones y sistemas}

\begin{definicion}[Ecuación Diferencial y solución]
    Una ecuación diferencial viene dada por una función
    \Func{\Phi}{D\subseteq \mathbb{R}^3}{\mathbb{R}}{(t,x,y)}{\Phi(t,x,y)}
    \textbf{continua} donde $D$ es un \textbf{abierto}\footnote{Por convenio, ya que las ecuaciones diferenciales no se comportan bien en los bordes.} \textbf{conexo}\footnote{Es lógico pensarlo, ya que estamos estudiando movimientos.} de $\mathbb{R}^3$.

\noindent
Una \textbf{solución} de dicha ecuación diferencial será una función
\Func{x}{I}{\mathbb{R}}{t}{$x(t)$}
con $I\subseteq \mathbb{R}^2$ \textbf{intervalo}\footnote{Es lógico, pues usualmente $t$ será el tiempo.} \textbf{abierto}\footnote{Pudiendo trabajar con funciones continuas en un cerrado y derivables en el abierto, evitando así por ejemplo tangentes verticales} tal que:
\begin{enumerate}[label=\roman*)]
    \item $x$ sea derivable en $I$\footnote{Necesario para poder considerar $x'(t)$ en la propia ecuación.}.
    \item $(t,x(t),x'(t)) \in D \quad\forall t\in I$\footnote{En vistas de III).}.
    \item $\Phi(t,x(t),x'(t))=0 \quad \forall t\in I$.
\end{enumerate}
\end{definicion}

\begin{ejemplo}
    Dada la ecuación diferencial:
    \begin{equation*}
        x'(t) = \dfrac{1}{x(t)}
    \end{equation*}
    y la expresión:
    \begin{equation*}
        x(t) = \sqrt{2t-38}
    \end{equation*}
    Probar que dicha expresión es solución de la ecuación diferencial.\\

    La ecuación diferencial viene dada por
    \Func{\Phi}{\mathbb{R}\times\mathbb{R}^+\times \mathbb{R}}{\mathbb{R}}{(t,x,y)}{y-\frac{1}{x}}
    \begin{itemize}
        \item Cogiendo $I = \left]19, +\infty\right[$, tenemos I).
        \item Por ser la raíz una función continua y creciente, tenemos que ${x(I) = \left]0,+\infty\right[=\mathbb{R}^+}$, de donde se cumple II).
        \item 
            \begin{equation*}
                x'(t) = \dfrac{1}{\sqrt{2t-38}} = \dfrac{1}{x(t)} \qquad \forall t\in I
            \end{equation*}
    \end{itemize}
\end{ejemplo}

\begin{notacion}
    La notación que hemos estado utilizando para las ecuaciones diferenciales no es la que usaremos a lo largo del curso:\\

    Lo que hasta ahora hemos notado y entendido por:
    \begin{equation*}
        \Phi(t,x(t),x'(t)) = 0
    \end{equation*}

    Como en el caso:
    \begin{equation*}
      x'(t) = 3x(t)  
    \end{equation*}
    Lo notaremos ahora por:
    \begin{gather*}
        \Phi(t,x,x') = 0\\
        x' = 3x
    \end{gather*}

    Que recordamos tiene por solución:
    \begin{equation*}
        x(t) = c \cdot e^{3t}, \quad c \in \mathbb{R}
    \end{equation*}
\end{notacion}

\begin{ejemplo}
    Para cierto $\lm \in \mathbb{R}$, resolver:
    \begin{enumerate}
        \item $x' = \lm x$
            \begin{equation*}
                x(t) = c \cdot e^{\lm t} \quad c \in \mathbb{R}
            \end{equation*}
        \item $x' = \lm t$
            \begin{equation*}
                x(t) = \frac{\lm}{2} t^2 + c \quad c \in \mathbb{R}
            \end{equation*}
    \end{enumerate}
\end{ejemplo}

\section{Crecimiento proporcional}


En el capítulo anterior nos preguntábamos si todas las soluciones de la ecuación $x'=\lm x$ para cierto $\lm \in \mathbb{R}$ eran de la forma
\begin{equation*}
    x(t) = c \cdot e^{\lm t} \quad c\in \mathbb{R}
\end{equation*}
Ahora daremos respuesta a dicha cuestión, comentando además utilidades de la misma ecuación.

\begin{prop}
    Dada $x(t)$, solución de $x'=\lm x$ para cierto $\lm \in \mathbb{R}$, definida en un intervalo abierto $I$, existe $c\in \mathbb{R}$ tal que
    \begin{equation*}
        x(t) = c \cdot e^{\lm t} \quad t\in I
    \end{equation*}
\end{prop}
Antes de dar paso a la demostración, observemos que si suponemos cierta la tesis:
\begin{equation*}
    x(t) = c \cdot e^{\lm t} \Longleftrightarrow  e^{-\lm t} x(t) = c \Longrightarrow \dfrac{d}{dt}(e^{-\lm t}x(t)) = 0
\end{equation*}
\begin{proof}
    Definimos la función
    \Func{f}{I}{\mathbb{R}}{t}{e^{-\lm t}x(t)}
    que es derivable por ser producto de funciones derivables.
    \begin{align*}
        \dfrac{df(t)}{dt} = \dfrac{d}{dt}(e^{-\lm t}x(t)) &= -\lm e^{-\lm t} x(t) + e^{-\lm t} x'(t)\\
                          &= -\lm e^{-\lm t}x(t) + \lm e^{-\lm t}x(t) = 0 \quad \forall t\in I
    \end{align*}
    Por ser $f$ continua y definida en un intervalo, llegamos a que es constante, luego $\exists c\in \mathbb{R} \mid f(t) = e^{-\lm t}x(t) = c \quad \forall t\in I$, de donde deducimos que:
    \begin{equation*}
        x(t) = c \cdot e^{\lm t} \quad \forall t\in I
    \end{equation*}
\end{proof}

Nos centraremos ahora en la idea intuitiva de derivada como representación de la variación de una variable dependiente para dar lugar a los siguientes dos ejemplos que nos muestran cómo surge la ecuación diferencial $x' = \lm x$ para cierto $\lm \in \mathbb{R}$ de forma natural.

\begin{ejemplo}
    Si ingresamos un capital inicial $C(0) = 100$ en un banco que nos da el 2\% anual de interés, al cabo de un año tendremos en nuestra cuenta:
    \begin{equation*}
        C(1) = C(0) + \dfrac{2}{100}C(0) = 100 + \dfrac{2}{100}\cdot 100 = 102
    \end{equation*}
Sin embargo, si acudimos a otro banco que nos ofrece la misma tasa anual de interés pero que nos realiza pagos semestrales, al cabo de un año conseguiremos reunir:
\begin{align*}
    C(\nicefrac{1}{2}) &= C(0) + \dfrac{1}{2}\dfrac{2}{100}C(0) = 100 + \dfrac{1}{2}\dfrac{2}{100}\cdot 100 = 101 \\
    C(1) &= C(\nicefrac{1}{2}) + \dfrac{1}{2}\dfrac{2}{100}C(\nicefrac{1}{2}) = 101 + \dfrac{1}{2}\dfrac{2}{100}\cdot 101 = 102.01 > 102
\end{align*}
En general, si $\Delta t$ es la fracción del año en la que se nos hacen los pagos ($\Delta t = 1$ significa que es pago anual), la fórmula es:
\begin{equation*}
    C(t+\Delta t) = C(t) + \dfrac{2}{100}\Delta t C(t)
\end{equation*}

de donde:
\begin{equation*}
    \dfrac{C(t+\Delta t) - C(t)}{\Delta t} = \dfrac{2}{100} C(t)
\end{equation*}
Si ahora vamos a un banco que nos haga pagos continuos, dicha fracción de tiempo será mínima, por lo que podemos pensar que $\Delta t \rightarrow 0$ y hacer un paso al límite (no riguroso) para obtener que:
\begin{equation*}
    \lim_{\Delta t\to0} \dfrac{C(t+\Delta t) - C(t)}{\Delta t} = C'(t) = \dfrac{2}{100}C(t)
\end{equation*}
De donde tenemos
\begin{equation*}
    C' = \dfrac{2}{100} C
\end{equation*}

o de forma general, si notamos por $I\in \mathbb{R}$ a la tasa de interés:
\begin{equation*}
    C' = I C
\end{equation*}
Cuyas soluciones ya sabemos que se tratan de cualquier función de la familia:
\begin{equation*}
    C(t) = c \cdot e^{It} \quad c\in \mathbb{R}
\end{equation*}
Por tanto, el crecimiento del dinero en un banco que nos ofrezca un interés continuo es exponencial. En este ejemplo, vemos cómo el crecimiento del dinero en una cuenta bancaria es exponencial a la propia cantidad de dinero de la que disponemos en la misma.
\end{ejemplo}

\begin{ejemplo}
    La teoría física de la radioactividad nos dice que las sustancias radioactivas van perdiendo masa a una velocidad proporcional a la masa de la propia sustancia. Es decir, si representamos la masa de una sustancia radioactiva como variable dependiente del tiempo $m=m(t)$:
    \begin{equation*}
        m'(t) = -\lm m(t)
    \end{equation*}
    Para cierto parámetro $\lm \in \mathbb{R}^+$ relativo a la sustancia radioaciva que consideremos.

    Gracias al estudio anteriormente realizado de la ecuación diferencial $m' = -\lm m$, sabemos ya que soluciones de esta son cualesquiera funciones de la forma:
    \begin{equation*}
        m(t) = c\cdot e^{-\lm t}\quad c\in \mathbb{R}^+
    \end{equation*}
En este ejemplo, hemos vuelto a observar la interpretación de la derivada como cuantía de la variación de una determinada variable dependiente.
\end{ejemplo}

Más adelante en estos mismos apuntes podremos observer el uso de la interpretación de la derivada como pendiente de la recta tangente a una curva.

% // TODO: Arreglar
% // Apuntes de clase
\section{Interpretación geométrica}
Las funciones que vamos a considerar para dar una interpretación geométrica.
\Func{f}{\mathbb{R}^2}{\mathbb{R}}{(t,x)}{f(t,x)}
continua
\begin{equation*}
    x' = f(x)
\end{equation*}
(Aunque debemos pensar en $x'(t) = f(t,x(t))$.)

% DIbujar dibujo de bloc de notas

Interpretaremos la función $f$ como un campo de direcciones, a cada punto del plano $(t,x)$ le asignamos un número $f(t,x)$ e interpretaremos dicho número como la pendiente de la recta que pasa por dicho punto.

A partir de dicha función podemos ir construyendo un ``campo de direcciones'', una regla que a cada punto del plano le asigna una recta que pasar por dicho punto.

Notemos que tenemos $x'(t) = f(t,x(t))$, por lo que a la derecha tenemos la pendiente del campo de direcciones; mientras que a la izquierda tenemos la pendiente de la recta tangente a la curva $x=x(t)$ en $(t,x(t))$.

% DIbujar campo de direcciones

En resumen, podemos imaginarnos la solución diferencial como un campo de direcciones, mientras que las soluciones son curvas que ``peinan'' dicho campo.

\begin{ejemplo}
    Consideremos la ecuación diferencial:
    \begin{equation*}
        x' = 0
    \end{equation*}
    Cuyas soluciones son todas las funciones constantes:
    \begin{equation*}
        x(t) = c \quad c\in \mathbb{R}, \quad t\in \mathbb{R}
    \end{equation*}
    En este caso, tenemos $f(t,x)=0$, $\forall (t,x)\in \mathbb{R}^2$, por lo que a cada punto de $\mathbb{R}^2$ le asociaremos una recta horizontal que pase por el mismo.

    % // TODO: Poner dibujo de wolframalpha
\end{ejemplo}

\begin{ejemplo}
    Consideremos:
    \begin{equation*}
        x' = 1
    \end{equation*}
    Cuyas soluciones son de la forma:
    \begin{equation*}
        x(t) = x + c \quad c\in \mathbb{R}, \quad t\in \mathbb{R}
    \end{equation*}
    En este caso, tenemos $f(t,x)=1$, $\forall (t,x)\in \mathbb{R}^2$, por lo que a cada punto le asociamos una recta a 45º de inclinación.
    % // TODO: Poner dibujo de wolframalpha
\end{ejemplo}

\begin{ejemplo}
    Consideramos:
    \begin{equation*}
        x'=x
    \end{equation*}
    Cuyas soluciones son:
    \begin{equation*}
        x(t) = c\cdot e^{t} \quad c\in \mathbb{R}, \quad t\in \mathbb{R}
    \end{equation*}
    En este caso, tenemos $f(t,x)=x$, $\forall (t,x)\in \mathbb{R}^2$.

    Para pensar en el campo de direcciones, pensemos en ir dibujándolo por cada recta horizontal del plano:
    \begin{itemize}
        \item Cuando $x=0$, tendremos rectas horizontales, luego para todos los puntos del eje de abscisas tenemos siempre rectas horizontales.
        \item Cuando $x=1$, tendremos rectas a 45º de inclinación.
        \item Cuando $x=-1$, tendremos rectas a -45º de inclinación.
        \item Cuando $x=2$, tendremos rectas más inclinadas que la de 45º.
        \item Cuando $x=-2$, tendremos rectas más inclinadas que la de -45º.
        \item \ldots
    \end{itemize}
    % // TODO: Poner dibujo de wolframalpha
\end{ejemplo}

\begin{ejemplo}
    Consideramos:
    \begin{equation*}
        x' = t^2 + x^2
    \end{equation*}
    Cuyas soluciones se han demostrado que no tienen fórmula que pueda escribirse con funciones elementales clásicas.

    Vamos a dibujar su campo de direcciones:
    Tenemos $f(t,x) = t^2 + x^2$, $\forall (t,x)\in \mathbb{R}^2$. 

    Para visualizar el campo de direcciones, pensaremos en circunferencias centradas en el origen con distinto radio $r$:
    \begin{itemize}
        \item Para $r = 0$, tenemos que $f(0,0) = 0$, luego tenemos en el origen la recta horizontal.
        \item Para $r = 1$, tenemos que $t^2 + x^2 = r^2 = 1$, luego tenemos una recta a 45º de inclinación en cada punto de dicha circunferencia.
        \item Para $r = 2$, tenemos que $t^2 + x^2 = r^2 = 4$, luego tendríamos una recta de mayor inclinación en dichos puntos.
        \item \ldots
    \end{itemize}
    % // TODO: Poner dibujo de wolframalpha
    A partir del campo de direcciones, intuimos que las soluciones son funciones crecientes cuya forma se asemeja a $x^3$, se van a $-\infty$ y a $+\infty$ y además de forma muy rápida.
\end{ejemplo}

\section{Funciones implícitas}
\begin{ejemplo}
    Veamos que la fórmula
    \begin{equation}\label{eq:ejm1}
        x^7+3x+t^2 = 0
    \end{equation}
    nos define una función implícita, de forma que dado un número $t\in \mathbb{R}$ nos proporcione un único valor $x\in \mathbb{R}$ que verifique la fórmula~(\ref{eq:ejm1}).

    Si conseguimos probarlo, tendremos la función $x=x(t)$.

    \begin{proof}
        Para probar la existencia:\newline
        Dado $t\in \mathbb{R}$, podemos definir el polinomio
        \begin{equation*}
            p_t(x) = x^7+3x+t^2
        \end{equation*}
        Y por ser un polinomio de grado impar, conocemos que al menos tiene una raíz $x\in \mathbb{R}$ tal que se verifica la fórmula~(\ref{eq:ejm1})\footnote{Es contenido de Álgebra I, pero puede deducirse por los límites en los extremos y el Teorema de Bolzano.}.\\

        \noindent
        Para ahora probar la unicidad, derivamos el polinomio $p_t(x)$:
        \begin{equation*}
            p_t'(x) = 7x^6+3 > 0 \quad \forall x\in \mathbb{R}
        \end{equation*}
        luego es una función estrictamente creciente, por lo que a lo sumo tiene una raíz.
    \end{proof}
\end{ejemplo}

Nos preguntamos ahora si la función $x=x(t)$ implícita dada por la fórmula~(\ref{eq:ejm1}) puede derivarse y si cualquier fórmula que nos inventemos nos dará una ecuación implícita.

Como primer resultado, vemos que hay fórmulas que no nos definen funciones implícitas:
\begin{equation*}
    x^2+t^2 = -1
\end{equation*}

Por ejemplo, la ecuación
\begin{equation*}
    x^2+t^2 = 1
\end{equation*}
Nos fabrica dos funciones implícitas:
\begin{align*}
    x_1(t) &= +\sqrt{1-t^2}, \quad t\in [-1,1] \\
    x_2(t) &= -\sqrt{1-t^2}, \quad t\in [-1,1] \\
\end{align*}

Tenemos funciones que están definidas por una ecuación que relaciona las variables $t$ y $x$.

También hay funciones implícitas que no se pueden derivar:
La fórmula
\begin{equation*}
    x^3-t^2 = 0
\end{equation*}
nos define la función $x(t) = t^{\nicefrac{2}{3}}$, $t\in \mathbb{R}$ y dicha función no es derivable en 0.

Dada una fórmula que nos relacione dos variables:
\begin{equation*}
    F(t,x) = 0
\end{equation*}
no podemos asegurar que la función implícita que nos da sea derivable, puede pasar cualquier cosa: a veces no define ecuación implícita, a veces define varias, \ldots no hay una teoría general de ecuaciones implícitas.\\

% // TODO: Pasar esto a teo
El Teorema de la Función Implícita es un teorema local, que da conciones para que exista la función implícita y sea única, dentro de un entorno suficientemente pequeño.

\Func{F}{G\subseteq \mathbb{R}^2}{\mathbb{R}}{(t,x)}{F(t,x)}
El dominio $G$ tiene que ser un abierto y la función $F$ ha de ser una función de clase $C^1(G)$. Es decir, que existan las parciales respecto a $t$ y respecto a $x$ y que ambas sean continuas. 

Como el teorema es local, necesitamos un punto $(t_0,x_0)\in G$ en el que estudiar la función, luego se cumplirá la ecuación $F(t_0,x_0)=0$. Además, tenemos que obligar a que la parcial respecto a la variable que queremos hacer dependiente no se anule: $\dfrac{\partial F(t,x)}{\partial x} \neq 0$.

En dicho caso, $\exists x:I\rightarrow\mathbb{R}$ con $I$ un intervalo abierto de forma que $t_0\in I$ tal que $x(t_0) = x_0$, $x\in C^1(I)$, se cumple que $(t,x(t)) \in G$ $\forall t\in I$ y que $F(t,x(t)) = 0$ $\forall t\in I$


\begin{ejemplo}
    En el caso
    \begin{equation*}
        x^2+t^2 = 1
    \end{equation*}
    Tenemos la función 
    \Func{F}{\mathbb{R}^2}{\mathbb{R}}{(t,x)}{x^2 + t^2 - 1}
    que es de clase $C^1(\mathbb{R}^2)$.
    
    Ahora, buscamos los puntos en los que centrarnos, puntos de la circunferencia de centro 1.
    No puede ser donde se anule $x$, ya que la parcial se anularía, luego podemos considerar la circunferencia menos donde corta al eje de abscisas.


    Dado dicho punto $t_0$, el teorema nos dice que hay un intervalo abierto que contiene a $t_0$ de forma que \ldots
\end{ejemplo}

\begin{ejemplo}
    En el caso
    \begin{equation*}
        x^7+3x+t^2 = 0
    \end{equation*}
    ya sabemos que existe la función implícita, pero no sabemos si dicha función es derivable o no.

    Usaremos el Teorema de la función implícita para probar que la función $x=x(t)$ que nos da la fórmula es derivable:

    Sea $G = \mathbb{R}^2$ y $F(t,x) = x^7+3x+t^2$ que es función de clase $C^1(\mathbb{R}^2)$:
    \begin{equation*}
        \dfrac{\partial F}{\partial x}(t,x) = 7x^6 + 3 > 0
    \end{equation*}
    Por lo que podemos elegir cualquier punto en el que $F(t,x) = 0$ para aplicar el teorema.
\end{ejemplo}

\begin{ejemplo}
    \begin{equation*}
        x^3-t^2 = 0
    \end{equation*}
    Sabemos que define una función implícita:
    \begin{equation*}
        x(t) = t^{\nicefrac{2}{3}}, \quad t\in \mathbb{R}
    \end{equation*}
    Donde tenemos $F(t,x) = x^3- t^2$

    \begin{equation*}
        \dfrac{\partial F}{\partial x}(0,0) = 0
    \end{equation*}
    Luego hemos comprobado con el teorema por qué el ejemplo anterior falló en 0.
\end{ejemplo}

\subsection{Derivación implícita}
\begin{ejemplo}
    Tenemos:
    \begin{equation*}
        F(t,x(t)) = 0, \quad t\in I
    \end{equation*}
    Con $F$ y $x$ funcion de clase C1, luego podemos derivar e igualar a cero (derivada de función constante es 0).

    Para derivarlo:
    \begin{equation*}
        \dfrac{d}{ds}[ F(t(s),x(s))] = \dfrac{\partial F}{\partial t}(t(s), x(s))t'(s) + \dfrac{\partial F}{\partial x}(t(s),x(s))x'(s)
    \end{equation*}

    \begin{equation*}
        \dfrac{dF}{dt}(t,x(t)) = \dfrac{\partial F}{\partial t}(t,x(t)) + \dfrac{\partial F}{\partial x}(t,x(t))x'(t) = 0
    \end{equation*}
    La derivada viene dada por una ecuación diferencial de primer orden que no está en forma normal.

    Tenemos que tener cuidado ya que $\dfrac{\partial F}{\partial x}(t,x(t))$ podría ser 0, pero bajo las hipótesis del Teorema de la ecuación implícita tenemos garantizado que en un entorno es distinto de cero, por lo que podemos ponerla en forma normal.

    Esa era la condición transversal. Geométricamente podemos pensar que si no se da la condición transversal, tenemos una curva que no puede ponerse como función (al tener curva tangente con pendiente infinita).\\

    Aplicar el teorema de la función implícita cuando no se pueda despejar la propia función de la fórmula.
\end{ejemplo}


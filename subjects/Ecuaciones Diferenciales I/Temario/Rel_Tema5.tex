\section{Sistemas Lineales}

\begin{ejercicio}\label{ej:5.1}
    Calcula la solución general del sistema lineal homogéneo $x' = Ax$ para las siguientes matrices
    \[
        A =
        \begin{pmatrix}
            1 & 0 & 1 \\
            0 & 1 & 0 \\
            1 & 0 & 1
        \end{pmatrix},
        \quad
        A =
        \begin{pmatrix}
            1 & -1 & 2 \\
            -1 & 1 & 0 \\
            -1 & 0 & 1
        \end{pmatrix}.
    \]
\end{ejercicio}

\begin{ejercicio}\label{ej:5.2}
    Se considera el problema de valores iniciales para el sistema de ecuaciones de segundo orden
    \[
        \begin{cases}
            x'' = A(t)x + b(t),\\
            x(t_0) = x_0,\\
            x'(t_0) = v_0,
        \end{cases}
    \]
    con $A \in C(I, \bb{R}^{N\times N})$, $b \in C(I, \bb{R}^N)$, $I$ intervalo abierto, $t_0 \in I$, $x_0, v_0 \in \bb{R}^N$. Demuestra que la sucesión de iterantes
    \[
        x_{n+1}(t) = \int_{t_0}^t (t - s)[A(s)x_n(s) + b(s)]ds + x_0 + v_0(t - t_0)
    \]
    con inicialización $x_0(t) = x_0$ converge uniformemente en compactos de $I$ a una solución del problema. Demuestra que esta solución es única.
\end{ejercicio}

\begin{ejercicio}\label{ej:5.3}
    Construye un sistema de ecuaciones lineal y homogéneo que tenga como soluciones linealmente independientes
    \[
        x_1 = (1, \sen t)^t, \quad x_2 = (\sen t, 1)^t.
    \]
    Justifica razonadamente si los coeficientes de tal sistema pueden estar o no definidos en toda la recta real.
\end{ejercicio}

\begin{ejercicio}\label{ej:5.4}
    Consideremos el sistema
    \[
        \begin{cases}
            x' = Ax + b(t),
        \end{cases}
    \]
    donde $A$ es una matriz cuadrada de orden $N$ y $b(t) = e^{\mu t}v$, con $v \in \bb{R}^N$. Si $\mu$ no es valor propio de $A$, prueba que existe una solución particular de la forma $x(t) = e^{\mu t}w$. Usa esta idea y el principio de superposición para encontrar una solución particular del sistema
    \[
        \begin{cases}
            x_1' = 2x_1 - 3x_2 + 3e^{2t},\\
            x_2' = x_1 - 2x_2 - 8e^{-3t}.
        \end{cases}
    \]
\end{ejercicio}

\begin{ejercicio}\label{ej:5.5}
    Se considera el sistema
    \[
        \begin{pmatrix}
            x'\\
            y'
        \end{pmatrix}
        =
        \begin{pmatrix}
            a(t) & b(t)\\
            -b(t) & a(t)
        \end{pmatrix}
        \begin{pmatrix}
            x\\
            y
        \end{pmatrix},
    \]
    donde $a, b : I \to \bb{R}$ son continuas. Demuestra que este sistema se puede reformular como una ecuación escalar compleja del tipo $z' = \alpha(t)z$ donde la incógnita $z = z(t)$ puede tomar valores complejos y la función $\alpha : I \to \bb{C}$ se determina a partir de $a(t)$ y $b(t)$. Utiliza este hecho para resolver el sistema original.
\end{ejercicio}

\begin{ejercicio}\label{ej:5.6}
    Dadas las funciones
    \[
        x(t) = \int_0^\infty \frac{e^{-\sigma}}{\sqrt{\sigma}}\sen(t\sigma)~d\sigma, \quad y(t) = \int_0^\infty \frac{e^{-\sigma}}{\sqrt{\sigma}}\cos(t\sigma)~d\sigma,
    \]
    demuestra que $(x, y)$ es solución de un sistema lineal homogéneo. Resuelve el sistema con las condiciones iniciales adecuadas para calcular las integrales.
    \begin{observacion}
        Tenga en cuenta que
        $\displaystyle \int_0^\infty e^{-t^2}~dt = \frac{\sqrt{\pi}}{2}$.
    \end{observacion}
\end{ejercicio}

\begin{ejercicio}\label{ej:5.7}
    Dada una matriz fundamental $\Phi(t)$ del sistema $x' = A(t)x$, con $A : I \to \bb{R}^{N\times N}$ continua y una función $\Psi : I \to \bb{R}^{N\times N}$ de clase $C^1$, demuestra que:
    \[
        \Psi(t) \text{ es matriz fundamental} \Longleftrightarrow \exists C \in \bb{R}^{N\times N} \text{ constante, con } |C|\neq 0 \text{, tal que } \Psi(t) = \Phi(t)C.
    \]
\end{ejercicio}

\begin{ejercicio}\label{ej:5.8}
    Dos tanques del mismo volumen $V$ contienen inicialmente agua salada con concentración $C_1$, $C_2$ respectivamente
    Se produce un transvase de agua entre los dos tanques a una velocidad fija de $\unitfrac[k]{l}{min}$ y en ambas direcciones, de manera que el volumen $V$ de cada tanque se mantiene constante. Plantea y resuelve el sistema de ecuaciones diferenciales que rige la cantidad de sal $Q_1$, $Q_2$ en cada tanque. Explica el comportamiento a largo plazo.
\end{ejercicio}

\begin{ejercicio}\label{ej:5.9}
    Dada $A \in C(\bb{R}, \bb{R}^{N\times N})$, demuestra que si la matriz $A(t)$ conmuta con $B(t) = \displaystyle \int_0^t A(s)~ds$, entonces $\Phi(t) = \exp\left(\displaystyle \int_0^t A(s)~ds\right)$ es matriz fundamental del sistema $x' = A(t)x$.
\end{ejercicio}

\begin{ejercicio}\label{ej:5.10}
    Por el método de variación de constantes, encuentra la solución general del sistema completo $x' = Ax + b(t)$ en los siguientes casos
    \begin{enumerate}
        \item $A\in \bb{R}^{2\times 2}$, $b\in \bb{R}^{2\times 1}$, donde:
        \[
            A =
            \begin{pmatrix}
                0 & 1\\
                -2 & 3
            \end{pmatrix},
            \quad
            b(t) =
            \begin{pmatrix}
                e^t \\ 2
            \end{pmatrix}.
        \]
        \item $A\in \bb{R}^{2\times 2}$, $b\in \bb{R}^{2\times 1}$, donde:
        \[
            A =
            \begin{pmatrix}
                2 & 1\\
                -4 & 2
            \end{pmatrix},
            \quad
            b(t) =
            \begin{pmatrix}
                te^{2t} \\ -e^{2t}
            \end{pmatrix}.
        \]
    \end{enumerate}
\end{ejercicio}

\begin{ejercicio}\label{ej:5.11}
    En este ejercicio probaremos un resultado sobre independencia lineal para funciones que son productos de polinomios y exponenciales. Lo haremos en tres pasos:
    \begin{enumerate}
        \item Demuestra que si $p(t)$ es un polinomio no nulo, $\alpha\in \bb{R}^*$ y $m = 1, 2, \ldots$, entonces
        \[
            \frac{d^m}{dt^m}\left[p(t)e^{\alpha t}\right] = q(t)e^{\alpha t},
        \]
        donde $q(t)$ es otro polinomio no nulo.
        \item Se supone que $p_1, \ldots, p_r$ son polinomios y $\alpha_1, \ldots, \alpha_r$ números distintos entre sí ($\alpha_i \neq \alpha_j$ si $i \neq j$). Entonces si la identidad
        \[
            p_1(t)e^{\alpha_1t} + \ldots + p_r(t)e^{\alpha_rt} = 0
        \]
        es válida en algún intervalo $I$ se cumplirá
        \[
            p_1 \equiv p_2 \equiv \ldots \equiv p_r \equiv 0.
        \]
        \item Dados números naturales $n_1, \ldots, n_r$ las funciones
        \[
            e^{\alpha_1t}, te^{\alpha_1t}, \ldots, t^{n_1}e^{\alpha_1t}, \ldots, e^{\alpha_rt}, te^{\alpha_rt}, \ldots, t^{n_r}e^{\alpha_rt}
        \]
        son linealmente independientes en $I$.
    \end{enumerate}
\end{ejercicio}

\begin{ejercicio}\label{ej:5.12}
    Se considera el operador diferencial
    \[
        L[y] = y^{(k)} + a_{k-1}y^{(k-1)} + \ldots + a_1y' + a_0y
    \]
    donde $a_0, a_1, \ldots, a_{k-1}$ son números reales.
    \begin{enumerate}
        \item Demuestra, para cada $m \geq 0$, la identidad
        \[
            L\left[t^me^{\lambda t}\right] = \sum_{h = 0}^m \binom{m}{h}t^{m-h}p^{(h)}(\lambda)e^{\lambda t}
        \]
        donde $p(\lambda) = \lambda^k + a_{k-1}\lambda^{k-1} + \ldots + a_1\lambda + a_0$.
        \item Utiliza esta identidad y el ejercicio anterior para obtener un sistema fundamental de la ecuación $L[y] = 0$. Se distinguirá el caso de raíces complejas.
        \item Resuelve la ecuación
        \[
            y^{(5)} - y^{(4)} + 2y''' - 2y'' + y' - y = 0.
        \]
        \item Se pasa la ecuación del apartado anterior a un sistema $x' = Ax$, con $x \in \bb{R}^5$, por el cambio $x_1 = y$, $x_2 = y'$, $x_3 = y''$, $x_4 = y'''$, $x_5 = y^{(4)}$. Diseña dos posibles estrategias para calcular $e^{At}$. ¿Cuál sería más conveniente?
    \end{enumerate}
\end{ejercicio}

\begin{ejercicio}\label{ej:5.13}
    ¿Es cierta la identidad $e^Ae^B = e^{A+B}$ para matrices arbitrarias $A, B \in \bb{R}^{N\times N}$?
\end{ejercicio}

\begin{ejercicio}\label{ej:5.14}
    Calcula $e^A$ para las matrices
    \[
        A =
        \begin{pmatrix}
            a & b\\
            -b & a
        \end{pmatrix},
        \quad
        A =
        \begin{pmatrix}
            0 & b_1 & 0 & \ldots & 0\\
            0 & 0 & b_2 & \ldots & 0\\
            \vdots & \vdots & \vdots & \ddots & \vdots\\
            0 & 0 & 0 & \ldots & b_{N-1}\\
            0 & 0 & 0 & \ldots & 0
        \end{pmatrix}.
    \]
\end{ejercicio}

\begin{ejercicio}\label{ej:5.15}
    Dada $A \in \bb{R}^{N\times N}$ se considera el sistema $x' = Ax$
    \begin{enumerate}
        \item Prueba que $e^{(t-t_0)A}$ es matriz fundamental principal en $t = t_0$.
        \item $(e^{tA})^{-1} = e^{-tA}$.
        \item Se considera ahora el problema de valores iniciales $x' = Ax+b(t)$, $x(t_0) = x_0$ donde $b : I \to \bb{R}^N$ es una función continua. Prueba que la solución está dada por la fórmula
        \[
            x(t) = e^{tA}x_0 + \int_{t_0}^t e^{(t-s)A}b(s)~ds.
        \]
    \end{enumerate}
\end{ejercicio}

\begin{ejercicio}\label{ej:5.16}
    Dada una matriz $A \in \bb{R}^{N\times N}$ define $\sen(A)$ y $\cos(A)$. Calcula: $$\cos\left(\begin{pmatrix} t & 1\\ 0 & t \end{pmatrix}\right)$$
\end{ejercicio}



\begin{comment}
1 En este ejercicio probaremos un resultado sobre independencia lineal para funciones que son productos de polinomios
y exponenciales. Lo haremos en tres pasos:
a) Demuestra que si p(t) es un polinomio no nulo, α 6= 0 es un n´umero y m = 1, 2, . . . , entonces
d
m
dtm

p(t)e
αt
= q(t)e
αt
,
donde q(t) es otro polinomio no nulo.
b) Se supone que p1, . . . , pr son polinomios y α1, . . . , αr n´umeros distintos entre s´ı(αi 6= αj si i 6= j). Entonces si
la identidad
p1(t)e
α1t + · · · + pr(t)e
αrt = 0
es v´alida en alg´un intervalo I se cumplir´a
p1 ≡ p2 ≡ · · · ≡ pr ≡ 0.
c) Dados n´umeros naturales n1, . . . , nr las funciones
e
α1t
, teα1t
, . . . , tn1 e
α1t
, . . . , eαrt
, teαrt
, . . . , tnr e
αrt
son linealmente independientes en I.
2 Se considera el operador diferencial
L[y] = y
(k) + ak−1y
(k−1) + · · · + a1y
0 + a0y
donde a0, a1, . . . , ak−1 son n´umeros reales.
a) Demuestra, para cada m ≥ 0, la identidad
L

t
me
λt
=
"Xm
h=0

m
h

t
m−h
p
(h)
(λ)
#
e
λt
donde p(λ) = λ
k + ak−1λ
k−1 + · · · + a1λ + a0.
b) Utiliza esta identidad y el ejercicio anterior para obtener un sistema fundamental de la ecuaci´on L[y] = 0. Se
distinguir´a el caso de ra´ıces complejas.
c) Resuelve la ecuaci´on
y
(5) − y
(4) + 2y
000 − 2y
00 + y
0 − y = 0.
d) Se pasa la ecuaci´on del apartado anterior a un sistema x
0 = Ax, con x ∈ R
5
, por el cambio x1 = y, x2 = y
0
, x3 =
y
00, x4 = y
000, x5 = y
(4). Dise˜na dos posibles estrategias para calcular e
At. ¿Cu´al ser´ıa m´as conveniente?.
3 ¿Es cierta la identidad e
Ae
B = e
A+B para matrices arbitrarias A, B ∈ R
N×N ?
4 Calcula e
A para las matrices A =

a b
−b a 
y A =


0 b1 0 · · · 0
0 0 b2 · · · 0
· · · · · · ·
0 0 0 · · · bN−1
0 0 0 · · · 0


.
5 Dada A ∈ R
N×N se considera el sistema x
0 = Ax
a) Prueba que e
(t−t0)A es matriz fundamental principal en t = t0.
b) (e
tA)
−1 = e
−tA.
c) Se considera ahora el problema de valores iniciales x
0 = Ax+b(t), x(t0) = x0 donde b : I → R
N es una funci´on
continua. Prueba que la soluci´on est´a dada por la f´ormula
x(t) = e
tAx0 +
Z t
t0
e
(t−s)Ab(s)ds.
6 Dada una matriz A ∈ R
N×N define sen(A) y cos(A). Calcula cos 
t 1
0 t

\end{comment}
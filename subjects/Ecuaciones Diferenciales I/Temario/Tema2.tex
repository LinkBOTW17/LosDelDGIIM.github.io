\newpage
\chapter{Cambio de Variables}

Dada una ecuación de primer orden en forma normal definida en un subconjunto del plano que suponemos que es una funcion continua:

\begin{equation*}
    \dfrac{dx}{dt} = x' = f(t,x)
\end{equation*}

\begin{gather*}
    f:D\rightarrow\mathbb{R}\\
    (t,x)\mapsto f(t,x)
\end{gather*}
$D$ abierto y conexo.

Un cambio de variable es un cambio en el plano y en el tiempo.

Viene dado por unas ecuaciones (funciones en R):
\begin{gather*}
    s=\varphi_1(t,x) \\
    y = \varphi_2(t,x)
\end{gather*}
Dos dan las nuevas variables en función de las variables antiguas.
Escribiremos la ecuación como:

\begin{equation*}
    \dfrac{dy}{ds} = fgorro(s,y)
\end{equation*}
Buscamos ver quién es $fgorro$. Es lo que veremos hoy.
Cuando escribimos las ecuaciones, las escribimos en la plano.
Por ahora, $x$ es variable independiente.

Podremos definir:
\begin{gather*}
    \varphi = (\varphi_i, \varphi_2), \varphi:D\rightarrow D_1 \\
    (t,x) \mapsto (s,y)
\end{gather*}
Con $D_1$ abierto y conexo.

Queremos hacer un cambio a un nuevo dominio en el plano


¿Qué hace falta para que sea un buen cambio de variable?
1. Que sea biyectivo (la $\varphi$). Es equivalente a decir que tenga inversa $\psi$.
\begin{gather*}
    \psi = \varphi^{-1} \\
    \varphi \circ \psi = id_{D_1} \\
    \psi \circ \varphi = id_{D} \\
\end{gather*}

2. Que podamos hacer cálculo diferencial en ambos lados y poder transportarlo.
Para ello, que el cambio de variable sea diferencial, es decir, $\varphi, \psi \in C^1$
\begin{gather*}
    \varphi : D\rightarrow D_1 \\
    \psi : D_1 \rightarrow D
\end{gather*}
(que una función de $\mathbb{R}^2$ sea de clase C1 significa que podemos hacer las parciales y que las dos son continuas)

Se resume diciendo que $\varphi$ es un difeomorfismo de clase $C^1$.
Los difeomorfismos son los buenos cambios de variable.
% // TODO: Hacer definición de difemorfismos (depende del contexto), homeomorfismo diferenciable con inversa diferenciable.

3. Además, tendremos que buscar cómo poner $y$ en función de $s$ (se verá en un futuro).
Supondremos haciendo los cálculos que se puede hacer y luego lo veremos.
*Hay difeomorfismos que pueden psar una curva en explícitas a una curva que no sea una gráfica de una función.


Supondremos que podemos poner la $y$ en función de la $s$:
Si hacemos $y=y(s)$, tenemos que calcular:
\begin{equation*}
    \dfrac{dy}{ds} = \dfrac{dy}{dt}\dfrac{dt}{ds}
\end{equation*}

Viendo que $y = \varphi_2(t,x)$, vemos que la única variable independiente es la $t$.
Estamos pensando en $y(t) = \varphi_2(t,x(t))$.
\begin{equation*}
    \dfrac{dy}{dt}(t) = \dfrac{\partial \varphi_2}{\partial t}(t,x(t)) + \dfrac{\partial\varphi_2}{\partial x}(t,x(t)) x'(t)
\end{equation*}
Sin embargo, $x$ es la solución de la ecuación diferencial, luego donde pone $x'(t)$ puedo poner $f(t,x)$ (aplicando el teorema de la función inversa)
\begin{equation*}
    \dfrac{ds}{dt} = \dfrac{\partial\varphi_1}{\partial t}(t,x) + \dfrac{\partial\varphi_1}{\partial x}(t,x) f(t,x)
\end{equation*}
Donde escribimos $x$ pero estamos pensando en $x(t)$.
\begin{equation*}
    \dfrac{dy}{ds} = \dfrac{dy}{dt}\dfrac{dt}{ds} = \dfrac{\dfrac{\partial\varphi_2}{\partial t}(t,x) + \dfrac{\partial\varphi_2}{\partial x}(t,x)f(t,x)}{\dfrac{\partial\varphi_1}{\partial t}(t,x) + \dfrac{\partial\varphi_1}{\partial x}(t,x) f(t,x)} 
\end{equation*}
La $fgorro$ no es eso exactamente, ya que en una ecuación diferencial no podemos quedarnos a medias: hemos pasado a $(s,y)$ pero seguimos teniendo cosas en $(t,x)$.
Para ello, usamos $\psi$:
\begin{equation*}
    (t,x) = \psi(s,y)
\end{equation*}

\begin{equation*}
    \dfrac{dy}{ds} = \dfrac{\dfrac{\partial\varphi_2}{\partial t}(\psi(s,y)) + \dfrac{\partial\varphi_2}{\partial x}(\psi(s,y))f(\psi(s,y))}{\dfrac{\partial\varphi_1}{\partial t}(\psi(s,y)) + \dfrac{\partial\varphi_1}{\partial x}(\psi(s,y)) f(\psi(s,y))}  = fgorro(s,y)
\end{equation*}
Que es la ecuación transportada, la $fgorro$. Para ello, tendremos que tener el denominador distinto de 0.\\

Este es el resultado y ahora le daremos rigor a todo esto.

\begin{ejemplo}
    \begin{equation*}
        \dfrac{dx}{dt} = \dfrac{\sen(x+t-3)}{{(x-2t+1)}^{2}} = f(t,x)
    \end{equation*}
    (La ecuacion sin dominio son dos ecuciones diferenciales)
    Siendo el dominio de $f$ todos los puntos donde no se anula el denominador.
    El conjunto de puntos donde se anula el demoninador viene dado por la recta:
    \begin{equation*}
        x-2t+1 = 0
    \end{equation*}
    Que divide el plano en dos componentes conexas. Para el dominio de la función $f$ hemos de quedarnos con un semiplano de los dos. Nos quedaremos con el de la izquierda, al que le llamamos $D$:
    \begin{equation*}
        D = \{(t,x)\in \mathbb{R}^2 \mid x -2t+1>0 \}
    \end{equation*}
    (fijado un punto de la recta, vemos donde queda el semiplano)
    % // TODO: Dibujar
    Vamos a aplicarle un cambio de variable:
    \begin{equation*}
        \left\{\begin{array}{rll}
            y &= x+t-3 &= \varphi_2(t,x) \\
            s &= x-2t+1 &= \varphi_1(t,x)
        \end{array}\right.
    \end{equation*}
    Veamos que $\varphi = (\varphi_1,\varphi_2)$ es un difeomorfismo de todo el plano en todo el plano:
    \begin{enumerate}
        \item Biyectivo: Sí, se puede despejar de manera única (es un sistema lineal compatible determinado).
        \item De clase C1: Sí, son polinomios.
        \item Que la inversa sea de clase C1: Sí, al despejar sale que es un polinomio también.
    \end{enumerate}
    Por tanto, $\varphi:\mathbb{R}^2\rightarrow\mathbb{R}^2$ es una biyección.\\
    Sin embargo, nos interesa verlo como un difeomorfismo de $D$. Buscamos el dominio del nuevo semiplano:
    La recta $x-2t+1=0$ se transforma en la recta $s=0$.
    % // TODO: Dibujar
    (como regla general, calcular la imagen y buscar cual es la frontera del dominio).

    Tomamos el lado de la derecha, ya que $D = \{\ldots \mid s>0\}$

    En definitiva:
    \begin{equation*}
        D_1 = \{(s,y)\in \mathbb{R}^2 \mid s>0\}
    \end{equation*}
    Además, sabemos que $D_1$ es abierto y conexo.\\

    Ahora, buscamos $fgorro$. Podríamos usar la fórmula pero vamos a repetir los cálculos:

    \begin{equation*}
        \dfrac{dy}{ds} = \dfrac{dy}{dt}\dfrac{dt}{ds} = \dfrac{\nicefrac{dy}{dt}}{\nicefrac{ds}{dt}}
    \end{equation*}

    Pensando que tanto $y$ como $x$ dependen de $t$:
    \begin{gather*}
        \dfrac{dy}{dt} = x' + 1 \\
        \dfrac{ds}{dt} = x' - 2 
    \end{gather*}
    \begin{equation*}
        \dfrac{dy}{ds} = \dfrac{x'+1}{x'-2}
    \end{equation*}
    Ahora, usamos que $x$ es solución de la ecuación diferencial:
    \begin{equation*}
        \dfrac{dy}{ds} = \dfrac{x'+1}{x'-2} = \dfrac{\dfrac{\sen(x+t-3)}{{(x-2t+1)}^{2}}+1}{\dfrac{\sen(x+t-3)}{{(x-2t+1)}^{2}}-2}
    \end{equation*}
    Ahora, falta poner la ecuación en función de $(s,y)$. Para ello, componemos con la $\psi$:
    \begin{equation*}
        \dfrac{dy}{ds} = \dfrac{\dfrac{\sen y}{s^2}+1}{\dfrac{\sen y}{s^2}-2} = \dfrac{\sen y+s^2}{\sen y -2s^2} = fgorro(s,y)
    \end{equation*}

Ahora surge que tenemos que poner la $y$ en función de $s$. La ecuación diferencial no está definida en todo el semiplano: el denominador de la expresión no puede anularse.
Los puntos:
\begin{equation*}
    \sen y -2s^2 = 0
\end{equation*}
% // TODO: Dibujarlo
No pueden entrar en el dominio de la ecuación diferencial.

Ocurre a lo que planteábamos antes: la curva en explícitas puede pasar a una curva que no se pueda poner en explícitas: que el denominador no se anule. Veremos gráficamente que es por esta razón.

Al introducir un cambio de variable en una ecuación diferencial, pueden surgir singularidades.

\end{ejemplo}

Ahora, desarrollo formal:
% // TODO: Pasar a teorema

Supongo que tengo una ecuacion diferencial
\begin{equation*}
    x' = f(t,x)
\end{equation*}
con $f:D\rightarrow\mathbb{R}$ continua con $D\subseteq \mathbb{R}^2$ abierto y conexo.\\
Primero, definimos:

\begin{definicion}[Cambio de variable admisible]
    Un cambio de variable admisible es una transformación $\varphi$:
    \begin{gather*}
        \varphi:D\rightarrow D_1 \\
        (t,x) \mapsto (s,y)
    \end{gather*}
    con $D$, $D_1\subseteq \mathbb{R}^2$ abiertos conexos, $\varphi$ es $C^1$-difeomorfismo, y además:
    \begin{equation*}
        \dfrac{\partial\varphi_1}{\partial t}(t,x) + \dfrac{\partial\varphi_1}{\partial x}(t,x)f(t,x) \neq 0 \qquad \forall (t,x) \in D
    \end{equation*}

    (Tiene una explicación geométrica: permitir que podamos poner las solucions $x=x(t)$ como $y=y(s)$, ya que los difeomorfismos llevan curvas en curvas pero no curvas en explícitas en curvas en explícitas)
\end{definicion}

Entonces, la ecuación $x'=f(t,x)$ es \textit{equivalente}\footnote{Quiere decir que siempre que tengamos una solución en $I\subseteq D$, tendremos otra solución en $J\subseteq D_1$ tal que $\varphi$ lleva un gráfico en otro gráfico} a la ecuación:
\begin{equation*}
    \dfrac{dy}{ds} = fgorro(s,y)
\end{equation*}
donde:
\begin{equation*}
    fgorro(s,y) = \dfrac{\dfrac{\partial\varphi_2}{\partial t}(\psi(s,y))+\dfrac{\partial\varphi_2}{\partial x}(\psi(s,y))f(\psi(s,y))}{\dfrac{\partial\varphi_1}{\partial t}(\psi(s,y)) + \dfrac{\partial\varphi_1}{\partial x}(\psi(s,y))f(\psi(s,y))}
\end{equation*}
donde $fgorro:D_1 \rightarrow\mathbb{R}$ continua
(Si hacemos que $\varphi$ sea de clase $C^1$, las parciales serán continuas y haremos que $fgorro$ sea continua)

\begin{proof}
    Supongamos que $x(t)$ es solución de la ecuación $x'=f(t,x)$ definida en un intervalo abierto $I\subseteq \mathbb{R}$.\\

    Teníamos:
    \begin{gather*}
        s = \varphi_1(t,x(t)) \\
        y = \varphi_2(t,x(t))
    \end{gather*}
    Buscamos poner $t = T(s)$.\\

    Defino
    \begin{gather*}
        S:I\rightarrow\mathbb{R} \\
        S(t) = \varphi_1(t,x(t))
    \end{gather*}
    que es derivable (por la regla de la cadena), con derivada distinta de 0:
    \begin{equation*}
        S'(t) = \dfrac{\partial\varphi_1}{\partial t}(t,x) + \dfrac{\partial\varphi_1}{\partial x}(t,x)x'(t) = \dfrac{\partial\varphi_1}{\partial t}(t,x) + \dfrac{\partial\varphi_1}{\partial x}(t,x)f(t,x) \neq 0 \qquad \forall t\in I
    \end{equation*}
    ya que el cambio era admisible. Podemos ahora aplicar el Teorema de la función inversa sobre $S$:\\

    Defino $J=S(I)$ intervalo abierto y que da:
    \begin{gather*}
        T:J\rightarrow I \\
        t = T(s)
    \end{gather*}
    \begin{gather*}
        T(S(t)) = t \quad \forall t\in I \\
        S(T(s)) = s \quad s\in J
    \end{gather*}
    Teníamos $s$ en función de $t$ y ahora hemos puesto $t$ en función de $s$ utilizando la primera ecuación del cambio de variable. Ahora, podemos definir (usando la segunda ecuación del cambio):
    \begin{gather*}
        y:J\rightarrow\mathbb{R} \\
        y(s) = \varphi_2(T(s),x(T(s)))
    \end{gather*}
    Nos falta derivar $y$ respecto a $s$ para comprobar que sea solución de la ecuación diferencial % // TODO: poner un label aquí haciendo referencia a la ecuacion diferencia dy/ds = muchas fracciones

    \begin{align*}
        y'(s) &= \dfrac{\partial\varphi_2}{\partial t}(T(s),x(T(s)))\cdot T'(s) + \dfrac{\partial \varphi_2}{\partial x}(T(s),x(T(s)))\cdot x'(T(s))\cdot T'(s) \\
              &= T'(s) \left(\dfrac{\partial\varphi_2}{\partial t}(T(s),x(T(s)))+ \dfrac{\partial \varphi_2}{\partial x}(T(s),x(T(s)))\cdot x'(T(s))\right) \\
              &= T'(s) \left(\dfrac{\partial\varphi_2}{\partial t}(T(s),x(T(s)))+ \dfrac{\partial \varphi_2}{\partial x}(T(s),x(T(s)))\cdot f(T(s),x(T(s)))\right) \\
    \end{align*}
    Ahora:
    \begin{equation*}
        \psi(s,y(s)) = (T(s), x(T(s)))
    \end{equation*}
    y ya tenemos el numerador
    % // TODO: Reescribir sustituyendo
    Falta ver que $T'(s)$ es el denominador:
    \begin{gather*}
        T=S^{-1} \\
        S:I\rightarrow J \\
        S(t) = \varphi_1(t,x(t))
    \end{gather*}
    Aplicamos la regla de derivación de la función inversa:
    \begin{equation*}
        T'(s) = \dfrac{1}{S'(T(s))} =  \dfrac{1}{\dfrac{\partial\varphi_1}{\partial t}(T(s),x(T(s))) + \dfrac{\partial\varphi_1}{\partial x}(T(s),x(T(s)))f(T(s),x(T(s)))} 
    \end{equation*}
    Ahora:
    \begin{equation*}
        \psi(s,y(s)) = (T(s), x(T(s)))
    \end{equation*}
    y ya tenemos el denominador
    % // TODO: Reescribir sustituyendo

    Ahora, falta ver que si tenemos una solución en $D_1$, volvemos a tener una solución en $D$.\\

    Bastaría aplicar el mismo proceso, pero al revés. Sin embargo, debemos comprobar que si $\varphi$ es admisible para la ecuación
    \begin{equation*}
        x'=f(t,x)
    \end{equation*}
    entonces, $\psi$ es admisible para la ecuación
    \begin{equation*}
        y' = fgorro(s,y)
    \end{equation*}
    % // TODO: Hacer:
    % Demostrar que si \varphi es admisible para x' entonces \psi es admisible para y', es decir, que la condición de \neq 0 se mantiene:
    \begin{equation*}
        \dfrac{\partial\psi_1}{\partial s}(s,y) + \dfrac{\partial\psi_1}{\partial y}(s,y)fgorro(s,y) \neq 0 \qquad \forall (s,y) \in  D_1
    \end{equation*}
    Usando que:
    \begin{equation*}
        \varphi'(\psi(s,y))\psi'(s,y) = Id
    \end{equation*}
    % Demostrado esto, veríamos que la condición es equivalente.
\end{proof}

Falta aprender estrategias para buscar el cambio de variable adecuado.
Estrategia: aprender a resolver ecuaciones sencillas y pasar ecuaciones más complicadas a estas mediante cambios de variable.

\section{Cálculo de primitivas}
Buscamos resolver el ecuaciones diferenciales sencillas. Las ecuaciones diferenciales más sencillas que podemos encontrarnos son el cálculo de primitivas, es decir, cuando la derivada de $x$ sólo está en función de $t$.\\

Pensamos en la ecuación diferencial:
\begin{equation*}
    x' = p(t)
\end{equation*}
con: $p:I\subseteq \mathbb{R}\rightarrow\mathbb{R}$ continua, el dominio de la ecuación diferencial es $D = I\times \mathbb{R}\subseteq \mathbb{R}^2$

Sabemos que dicha ecuación diferencial tiene solución, gracias por el Teorema Fundamental del Cálculo (Todas las funciones continuas tienen primitiva y además se construye así).

% // TODO: COpiar TFC.

Si $p$ es continua, entonces $\exists P$ tal que:
\begin{gather*}
    P\in C^1(I) \\
    P'(t) = p(t) \\
    P(t) = \displaystyle\int_{t_0}^{t} p(s)~ds 
\end{gather*}
Por tanto, fijado $t_0 \in I$, las soluciones de la ecuación diferencial son de la forma:
\begin{equation*}
    x(t) = k + \int_{t_0}^{t} p(s)~ds  \qquad k\in \mathbb{R}
\end{equation*}
(tenemos una curva y al mover $k$ rellenamos toda la banda $I\times \mathbb{R}$) 

Tenemos una primera clase de ecuaciones diferenciales que sabemos resolver a nivel teórico (porque hay integrales que no pueden calcularse).

Observacion: al integrar, escribimos otra variable ($s$):
\begin{equation*}
    \int_{t_0}^{t} p(s)~ds 
\end{equation*}
\begin{equation*}
    \sum_{n=0}^{N}p_n
\end{equation*}







TFC\@:
\begin{equation*}
    \dfrac{d}{dt}\left(\displaystyle\int_{t_0}^{t} f(s)~ds \right) = f(t)
\end{equation*}

Regla de Barrow:
\begin{equation*}
    \displaystyle\int_{t_0}^{t} F'(s)~ds  = F(t) - F(t_0)
\end{equation*}

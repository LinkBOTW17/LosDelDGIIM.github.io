\section{Ecuación Lineal de Orden Superior}

\begin{ejercicio}\label{ej:4.1}
    Encuentra funciones $a, b \in C(I)$ de modo que $t, t^2$ sean soluciones de una ecuación lineal
    \[
        x'' + a(t)x' + b(t)x = 0
    \]
    con $a, b \in C(I)$. Discute si el intervalo $I$ puede ser toda la recta real o no.
\end{ejercicio}

\begin{ejercicio}\label{ej:4.2}
    Encuentra un sistema fundamental de soluciones de la ecuación $3x'' - 2x' - 8x = 0$
    \begin{observacion}
        Busca soluciones de la forma $e^{\lambda t}$.
    \end{observacion}
    Por el método de variación de constantes, encuentra la solución general de la ecuación $3x'' - 2x' - 8x = \cosh(t)$.
\end{ejercicio}

\begin{ejercicio}\label{ej:4.3}
    Encuentra la solución general de la ecuación
    \[
        y'' + \frac{2}{x}\cdot y' + y = \frac{1}{x},
    \]
    sabiendo que dos soluciones de la ecuación homogénea son $\frac{\sen x}{x}$, $\frac{\cos x}{x}$.
\end{ejercicio}

\begin{ejercicio}\label{ej:4.4}
    Se considera la ecuación
    \[
        (1 + t)x'' - (1 + 2t)x' + tx = t e^t.
    \]
    Se pide:
    \begin{enumerate}
        \item Comprueba que $z_0(t) = e^t$ es una solución particular de la ecuación homogénea.
        \item Efectúa el cambio $x = uz_0$ en la ecuación completa para reducir su orden y poder integrarla.
    \end{enumerate}
\end{ejercicio}

\begin{ejercicio}\label{ej:4.5}
    Consideremos la ecuación
    \[
        x^2 y'' - 7xy' + 16y = 0.
    \]
    Encuentra una solución particular de tipo potencia ($y_1(x) = x^m$) y usa la fórmula de Liouville para encontrar la solución general.
\end{ejercicio}

\begin{ejercicio}\label{ej:4.6}
    Fijado $I\subset \bb{R}$, sean $\varphi_1(t), \varphi_2(t), \ldots, \varphi_k(t)\in C^k(I)$ que cumplen:
    $$W(\varphi_1, \varphi_2, \ldots, \varphi_k)(t) \neq 0~\forall t \in I$$
    Demuestra que existe una ecuación lineal homogénea
    \[
        x^{(k)} + a_{k-1}(t)x^{(k-1)} + \cdots + a_1(t)x' + a_0(t)x = 0
    \]
    con $a_{k-1}, \ldots, a_1, a_0 \in C(I)$ tal que $\varphi_1(t), \varphi_2(t), \ldots, \varphi_k(t)$ es un sistema fundamental. ¿Es cierta esta conclusión cuando $W(\varphi_1, \varphi_2, \ldots, \varphi_k)(t) = 0$ para cada $t \in I$?
\end{ejercicio}

\begin{ejercicio}\label{ej:4.7}
    Se considera la ecuación
    \[
        y' + y^2 + \alpha(t)y + \beta(t) = 0
    \]
    donde $\alpha, \beta : I \to \bb{R}$ son funciones continuas. Dada una solución $y(t)$ definida en un intervalo abierto $J \subset I$ se define
    \[
        x(t) = c\cdot \exp\left(\int_{t_0}^t y(s)ds\right)
    \]
    donde $c\in \bb{R}$ es una constante y $t_0 \in J$. Demuestra que $x(t)$ es solución de una ecuación lineal y homogénea de segundo orden.
\end{ejercicio}

\begin{ejercicio}\label{ej:4.8}
    Se considera la ecuación $x'' + a(t)x = 0$ donde $a \in C^1(I)$.
    \begin{enumerate}
        \item Dadas dos soluciones $x_1(t)$ y $x_2(t)$ de la ecuación anterior, demuestra que la función producto $z(t) = x_1(t)x_2(t)$ es solución de la ecuación de tercer orden
        \[
            z''' + 4a(t)z' + 2a'(t)z = 0.
        \]
        \item Se supone que $x_1(t)$ y $x_2(t)$ forman un sistema fundamental para la ecuación de segundo orden, demuestra que las funciones $x_1^2(t)$, $x_1(t)x_2(t)$, $x_2^2(t)$ forman un sistema fundamental de la ecuación de tercer orden.
        \begin{observacion}
            Prueba la identidad
            \[
                \begin{vmatrix}
                    v_1^2 & w_1^2 & v_1w_1\\
                    2v_1v_2 & 2w_1w_2 & v_2w_1 + v_1w_2\\
                    v_2^2 & w_2^2 & v_2w_2
                \end{vmatrix} = (w_1v_2 - v_1w_2)^3.
            \]
        \end{observacion}
    \end{enumerate}
\end{ejercicio}

\begin{ejercicio}\label{ej:4.9}
    Demuestra que las funciones $f_j(t) = |t - j|$, $j = 1, \ldots, n$ son linealmente independientes en $I = ]0, \infty[$.
    \begin{observacion}
        Las funciones $f_j$ son derivables en cada intervalo $]0, 1[$, $]1, 2[$, $\ldots$.
    \end{observacion}
\end{ejercicio}

\begin{ejercicio}\label{ej:4.10}~
    \begin{enumerate}
        \item Encuentra dos funciones $f_1, f_2 \in C^1(I)$ que sean linealmente independientes en $I$ mientras que sus derivadas son linealmente dependientes.
        \item Demuestra que si $f_1, f_2, \ldots, f_n \in C^1(I)$ son funciones tales que $f_0, f_1, \ldots, f_n$ son linealmente independientes entonces $f_1^0, f_0^1, \ldots, f_0^n$ son también linealmente independientes. La notación $f^0$ se emplea para la función constante $f^0(t)=~1$.
    \end{enumerate}
\end{ejercicio}

\begin{ejercicio}\label{ej:4.11}~
    \begin{enumerate}
        \item Dada la ecuación del oscilador armónico $x'' + \omega^2 x = 0$ con $\omega > 0$, demuestra que las funciones $\cos \omega t$, $\sen \omega t$ forman un sistema fundamental.
        \item Consideramos ahora el oscilador forzado $x'' + \omega^2 x = A \sen(\Omega t) + B \cos(\Omega t)$, donde $\Omega > 0$ es un número real. Demuestra que esta ecuación admite una solución del tipo $x(t) = a \cos(\Omega t) + b \sen(\Omega t)$ si $\Omega \neq \omega$.
        \item Resuelve la ecuación $x'' + \omega^2 x = \sum\limits_{i = 1}^n \alpha_i \sen(\Omega_i t + \varphi_i)$ cuando $\omega \neq \Omega_i$ para cada $i$.
        \item ¿Cómo son las soluciones en el caso $\Omega_i = \omega$ para algún $i$?
    \end{enumerate}
\end{ejercicio}

\begin{ejercicio}\label{ej:4.12}
    Se considera la ecuación
    \[
        x^{(k)} + a_{k-1}x^{(k-1)} + \cdots + a_1x' + a_0x = \sum_{i = 1}^n A_i e^{\lambda_i t}
    \]
    donde $a_0, \ldots, a_{k-1}$ y $\lambda_1, \ldots, \lambda_n$ son constantes. Encuentra la condición necesaria y suficiente para que la ecuación admita una solución del tipo $x(t) = \sum\limits_{i = 1}^n c_i e^{\lambda_i t}$.
\end{ejercicio}
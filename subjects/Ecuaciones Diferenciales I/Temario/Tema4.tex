\newpage
\chapter{Ecuación Lineal de Orden Superior}

\noindent
Lo que queda de curso se limita a estudiar las ecuaciones diferenciales lineales. Es decir, las que son de la forma:
\begin{equation}\label{eq:linealsup}
    x^{(m)} + a_{m-1}(t) x^{(m-1)} + \cdots + a_1(t) x' + a_0(t)x = b(t) \qquad m\geq 1
\end{equation}
con $a_0,a_1,\ldots, a_{m-1},b:I\rightarrow\mathbb{R}$ funciones continuas definidas en un intervalo $I$ abierto.

% // TODO: Definir orden de la ecuación.
% // TODO: Cambiar todos los x'' en el muelle por puntitos

\begin{definicion}
    Una ecuación diferencial lineal se dice: 
    \begin{itemize}
        \item homogénea si $b(t) = 0$ $\forall t\in I$.
        \item completa si $b(t) \neq 0$ para algún $t\in I$.
    \end{itemize}
\end{definicion}

% // TODO: Juntar todo en un ejeplo
\begin{ejemplo}
    El caso $m=1$ ha sido ya estudiado, se trata de la ecuación lineal de primer orden:
    \begin{equation*}
        x' + a_0(t)x = b(t)
    \end{equation*}
    Por tanto, esta sección estará dedicada a las ecuaciones lineal de orden 2 o mayor. En especial, las de orden 2, por $F=ma$.
\end{ejemplo}

\begin{ejemplo}
    La ecuación del oscilador armónico representa un objeto que está atado a una pared con un muelle. % // TODO: Hacer dibujo y dar forma al ejemplo
    \begin{equation*}
        mx'' + kx = 
    \end{equation*}
    Para $x=0$, decimos que hemos estidrado el muelle. Sin embargo, si lo estiramos, el muelle tira haciea dentro, tira hacia la izqda. Si lo comprimimos, tiraría en la otra dirección, con lo que tenemos $mx'' = F(x)$ (masa por aceleración ($x''$)), conocida como la Ley de Hooke (la ley de Hooke es $F(x)=-kx$).

    Se trata de una fuerza recuperadora. $k$ es la constante de elasticidad, que cuantifiac qué tan duro está un muelle (si el muelle es muy duro, $k$ es muy grande).

    Que podemos escribirla como:
    \begin{equation*}
        x'' + \dfrac{k}{m} x = 0
    \end{equation*}

    con:
    \begin{equation*}
        a_1(t) = 0 \qquad a_0(t) = \dfrac{k}{m}\qquad  b(t) = 0
    \end{equation*}
\end{ejemplo}

\begin{ejemplo}
    Las vibraciones de los puentes se manejan en ecuaciones de 4o orden. Un ejemplo de esta es:
    \begin{equation*}
        x^{(iv)} + e^t x^{(iii)} - x = \sen t
    \end{equation*}
    Con $m=4$, y:
    \begin{equation*}
        b(t) = \sen t \qquad a_3(t) = e^t \qquad a_2(t)=a_1(t) = 0 \qquad a_0(t) = -1
    \end{equation*}
    con $I=\mathbb{R}$
\end{ejemplo}

Esta sección va a estar basada en un Teorema que no se demostrará en este Capítulo, sino en el siguiente. Se trata de un Teorema de existencia y unicidad.

% // TODO: Poner esto al final de la asignatura
% Si el lector está interesado en hacerlo de forma deductiva, se recomienda pasar a leer el Teorema~\ref{teo:PONER REFERENCIA DEL TEOREMA}

\begin{definicion}[Condición inicial]
    Una condición inicial para una ecuación lineal de orden $m$ es tomar $t_0\in I$ y dar los valores:
    \begin{equation*}
        x(t_0) = \alpha_0 \qquad x'(t_0) = \alpha_1,\qquad  \ldots, \qquad  x^{(m-1)}(t_0) = \alpha_{m-1}
    \end{equation*}
    con $\alpha_0,\alpha_1,\ldots,\alpha_{m-1}\in \mathbb{R}$.
\end{definicion}
De esta forma, para dar una condición inical para una ecuación de orden $m$, habrá que dar $m$ condiciones: el valor de la función en un punto y el valor de sus sucesivas derivadas hasta $m-1$ en dicho punto.\\

Notemos la importancia de dar los valores en el mismo punto $t_0\in I$, ya que si no, no se trata de una condición inicial, sino una condición de contorno.

\begin{ejemplo}
    En el oscilador armónico:
    \begin{equation*}
        x'' + \dfrac{k}{m}x = 0
    \end{equation*}
    una condición inicial es:
    \begin{equation*}
        x(t_0) = 0 \qquad x'(t_0) = 1
    \end{equation*}
    Donde:
    \begin{itemize}
        \item El muelle parte del origen (posición 0).
        \item El muelle comienza con velocidad 1 (por lo que le hemos dado un impulso al muelle).
    \end{itemize}
    De forma intuitiva, vemos que con estas condiciones iniciales tenemos una única solución de la ecuación que cumple estas mismas.

    Por otra parte, la condición
    \begin{equation*}
        x(t_0) = 1 \qquad x'(t_0) = 0
    \end{equation*}
    Significa que:
    \begin{itemize}
        \item El muelle parte de una unidad (posición 1, el muelle está estirado).
        \item El muelle no tiene velocidad inicial.
    \end{itemize}

    En este caso, podemos pensar en la condición de que para dar una condición inicial la ordenada sea la misma intuitivamente como que tenemos que describir el movimiento (estamos en ecuaciones diferenciales) completo en un mismo instante.
\end{ejemplo}

\begin{teo}[Existencia y unicidad de las soluciones]\label{teo:existencia_unicidad}
    Dada una ecuación de la forma~(\ref{eq:linealsup}) siendo los coeficientes funciones continuas en un intervalo abierto $I$. Dada una condición inicial $t_0\in I$ y números $\alpha_0, \alpha_1,\ldots, \alpha_{m-1}\in \mathbb{R}$.

    Entonces, existe una única solución $x$ de~(\ref{eq:linealsup}), definida en todo el intervalo $I$ que cumple las condiciones iniciales, es decir:
    \begin{equation*}
        x(t_0) = \alpha_0 \qquad x'(t_0) = \alpha_1,\qquad  \ldots, \qquad  x^{(m-1)}(t_0) = \alpha_{m-1}
    \end{equation*}
\end{teo}
Notemos que $x\in C^m(I)$.\\
Notemos que este teorema es global, nos garantiza que dada una ecuación diferencial definida en un conjunto $D=I\times J$, entonces las soluciones estarán siempre definidas en todo el intervalo $I$.\\

Notemos que de esta forma, las condiciones iniciales nos etiquetan las soluciones.

\begin{ejemplo}
    \begin{equation*}
        x' = x^2 \qquad x(0) = 1
    \end{equation*}

    $D=\mathbb{R}^2$ sus soluciones son:
    \begin{equation*}
        x(t) = \dfrac{1}{1-t} \qquad t\in \left]-\infty,1\right[
    \end{equation*}

    Que no están definidas en el mismo intervalo $D$.
\end{ejemplo}

Notemos que en~(\ref{eq:linealsup}) no consideramos un coeficiente $a_m(t)$, ya que queremos poder dicha ecuación en forma normal.

\begin{ejemplo}
    Consideremos
    \begin{equation*}
        tx'-x = 0 \qquad x(0) = 0
    \end{equation*}

    que no es una ecuación de la forma~(\ref{eq:linealsup}). Notemos que no podemos aplicar el Teorema y que, de hecho, esta ecuación tiene infinitas soluciones, por representar todas las rectas que pasan por el origen:
    % // TODO: Graficar las rectas que pasan por el origen.

    No tenemos unicidad

    Si ahora consideramos como condición inicial $x(0)=1$, no tenemos existencia.
\end{ejemplo}

Por tanto, lo de considerar $a_m(t)=1$ tiene su importancia.

Sin embargo, si dividimos entre $a_m(t)$ podemos aplicar el Teorema, pero debemos restringir el dominio y tener cuidado con las condiciones iniciales, ya que quizas no se cumpla la condicion inicial.

\begin{equation*}
    x' - \dfrac{1}{t}x = 0 
\end{equation*}
A pesar de esto, si en este caso tenemos $t_0\neq 0$, podemos dividir entre $a_m(t)$ y considerar la ecuación resultado, que sí es una ecuación lineal de orden superior.

El Teorema será difícil de demostrar para orden mayor o igual que 2. Sin embargo, no lo estanto para $m=1$.

\begin{prop}[Teorema debilitado]
    Dada una ecuación de la forma
    \begin{equation*}
        x' + a_0(t)x = b(t)
    \end{equation*}
    con $a_0,b:I\rightarrow\mathbb{R}$ funciones continuas, sea una condicion \ldots

    Dada una ecuación de la forma~(\ref{eq:linealsup}) de orden $1$ siendo los coeficientes funciones continuas en un intervalo abierto $I$. Dada una condición inicial $t_0\in I$ y número $\alpha_0 \in \mathbb{R}$.

    Entonces, existe una única solución $x$ de~(\ref{eq:linealsup}), definida en todo el intervalo $I$ que cumple las condiciones iniciales, es decir:
    \begin{equation*}
        x(t_0) = \alpha_0 
    \end{equation*}
    
    \begin{proof}
        Probemos pues existencia y unicidad:
        \begin{description}
            \item [Existencia.]~\\
                Tenemos una ecuación de la forma
                \begin{equation*}
                    x' + a_0(t)x = b(t) \qquad x(t_0) = \alpha_0
                \end{equation*}
                Sea:
                \begin{equation*}
                    x(t) = e^{-A_0(t)} [\alpha_0 + F(t)]
                \end{equation*}

                donde:
                \begin{equation*}
                    A_0(t) = \int_{t_0}^{t} a_0(s)~ds 
                \end{equation*}

                está bien definida porque $a_0$ es continua, luego TFC, tenemos que $A_0\in C^1(I)$.
                \begin{equation*}
                    F(t) = \int_{t_0}^{t} e^{A_0(s)}b(s)~ds 
                \end{equation*}

                por otra parte, $b$ es continua, producto de dos funciones continuas más TFC, tenemos que $F\in C^1(I)$.

                Por tanto, tenemos que $x\in C^1(I)$.

                Veamos que cumple la condición inicial (para esto, hemos tomado unas primitivas específicas y no cualquiera, las centradas en $t_0$):
                \begin{equation*}
                    x(t_0) = e^0 (\alpha_0 + 0) = \alpha_0
                \end{equation*}

                Finalmente, demostremos que cumple la ecuación diferencial. Aplicando la derivada del producto:
                \begin{equation*}
                    x'(t) = -a_0(t)e^{-A_0(t)} [\alpha_0 + F(t)] + e^{-A_0(t)}e^{A_0(t)}b(t) \AstIg -a_0(t) x(t) + b(t)
                \end{equation*}
                Donde en $(\ast)$ hemos aplicado que:
                \begin{equation*}
                    x(t) = e^{-A_0(t)} [\alpha_0 + F(t)]
                \end{equation*}
            \item [Unicidad.]~\\
                La unicidad se basa en usar que la lineal de orden 1 tiene un factor integrante. Supongamos, pues que tenemos dos soluciones, $x_1,x_2:I\rightarrow\mathbb{R}$ de la ecuación lineal de orden 1 y buscamos probar que $x_1(t) = x_2(t)$ $\forall t\in I$.
                
                Para ello, definimos:
                \begin{equation*}
                    y(t) = x_1(t) - x_2(t)
                \end{equation*}
                
                función de clase $C^1(I)$ que cumple:
                \begin{equation*}
                    y'(t) = x_1'(t) - x_2'(t)
                \end{equation*}
                
                luego:
                \begin{equation*}
                    y'(t) + a_0(t)y(t) = 0
                \end{equation*}
                Y además, la función $y$ cumple una condición inicial: 
                \begin{equation*}
                    y(t_0) = x_1(t_0) - x_2(t_0) = \alpha_0 - \alpha_0 = 0
                \end{equation*}
                Para ver que $y$ es constantemente igual a 0, multiplicaremos por el factor integrante de $y$ (como $y$ cumple la misma ecuacion, en el capitulo anterior vimos el factor integrante):
                \begin{equation*}
                    e^{A_0(t)} y'(t)+ a_0(t) e^{A_0(t)} y(t) = 0
                \end{equation*}
                \begin{equation*}
                    \dfrac{d}{dt}\left(e^{A_0(t)}y(t)\right) = 0
                \end{equation*}
                Como consecuencia, tenemos que:
                \begin{equation*}
                    e^{A_0(t)} y(t) = c \qquad c\in \mathbb{R} \qquad \forall t\in I
                \end{equation*}

                como $y(t_0) = 0$, deducimos que $c=0$, es decir, que:
                \begin{equation*}
                    y(t) = 0 \qquad \forall t\in I
                \end{equation*}

                con lo que $x_1(t) = x_2(t)$ $\forall t\in I$.
        \end{description}
    \end{proof}
\end{prop}

Lo que sucede a partir de segundo orden es que no hay fórmula.

Se puede demostrar que la ecuación diferencial de segundo orden no tiene fórmula.
Resulta que cualquiera de 2o orden lineal se puede pasar a Riccati, con lo que ninguna tiene.

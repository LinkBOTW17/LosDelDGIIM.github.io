\section{Ecuaciones y sistemas}

\begin{ejercicio}
    En Teoría del Aprendizaje, se supone que la velocidad a la que se memoriza una materia es proporcional a la
    cantidad que queda por memorizar. Suponemos que \(M\) es la cantidad total de materia a memorizar y \(A(t)\) es la
    cantidad de materia memorizada a tiempo \(t\). Determine una ecuación diferencial para \(A(t)\). Encuentre soluciones
    de la forma \(A(t) = a + be^{\lambda t}\).\\

    Tras interpretar el enunciado, deducimos que:
    \begin{equation*}
        A' = c(M - A),
    \end{equation*}
    donde $c\in \bb{R}$ es la constante de proporcionalidad. Esta es la ecuación diferencial que buscamos.

    % // TODO: Terminar la segunda parte
\end{ejercicio}


\begin{ejercicio}
    Interprete cada enunciado como una ecuación diferencial:
    \begin{enumerate}
        \item El grafo de \(y(x)\) verifica que la pendiente de la recta tangente en un punto es el cuadrado de la distancia del punto al origen.
        \item El grafo de \(y(x)\) verifica en cada punto que la distancia del origen al punto de corte de la recta tangente con el eje de ordenadas coincide con la distancia del origen al punto de corte de la recta normal con el eje de abscisas.
    \end{enumerate}
\end{ejercicio}


\begin{ejercicio}
    En ciertas reacciones químicas, la velocidad a la que se forma un nuevo compuesto viene dada por la ecuación
    \begin{equation*}
        x' = k(x - \alpha)(\beta - x),
    \end{equation*}
    donde \(x(t)\) es la cantidad de compuesto a tiempo \(t\), \(k > 0\) es una constante de proporcionalidad y \(\beta > \alpha > 0\). Usando el campo de direcciones, prediga el comportamiento de \(x(t)\) cuando \(t \to +\infty\).
\end{ejercicio}


\begin{ejercicio}
    Encuentre la familia de trayectorias ortogonales a las familias de curvas siguientes, teniendo en cuenta que para resolver las ecuaciones que aparecen en \ref{itm:ej1.4_b} y \ref{itm:ej1.4_c} habrá que esperar a la siguiente lección:
    \begin{enumerate}
        \item \(xy = k\),
        \item\label{itm:ej1.4_b} \(y = kx^4\),
        \item\label{itm:ej1.4_c} \(y = e^{kx}\).
    \end{enumerate}
\end{ejercicio}


\begin{ejercicio}
    Haga un dibujo aproximado del campo de direcciones asociado a la ecuación
    \begin{equation*}
        x' = t + x^3.
    \end{equation*}
    Dibuje la curva donde las soluciones alcanzan un punto crítico. Considerando una solución tal que \(x(0) = 0\), demuestre que tal solución alcanza en 0 un mínimo local estricto y que de hecho es el mínimo global.
\end{ejercicio}


\begin{ejercicio} Resuelva los siguientes apartados:
    \begin{enumerate}
        \item Estudie cuántas funciones diferenciables \(y(x)\) se pueden extraer de la curva
        \begin{equation*}
            C \equiv x^2 + 2y^2 + 2x + 2y = 1,
        \end{equation*}
        dando su intervalo maximal de definición.
        \item Usando derivación implícita, encuentre una ecuación diferencial de la forma \(y' = f(x, y)\) que admita como soluciones a las funciones del apartado anterior.
        \item La misma cuestión para una ecuación del tipo \(g(y, y') = 0\).
    \end{enumerate}
\end{ejercicio}


\begin{ejercicio}
    Una persona, partiendo del origen, se mueve en la dirección del eje \(x\) positivo tirando de una cuerda de longitud \(s\) atada a una piedra.
    Se supone que la cuerda se mantiene tensa en todo momento, y que la piedra es arrastrada desde el punto de partida \((0, s)\).
    La trayectoria que describe la piedra es una curva clásica llamada tractriz.
    Encuentre una ecuación diferencial para la misma.
    \begin{observacion}
        Se supone que la cuerda se mantiene tangente a la trayectoria de la piedra en todo momento.
    \end{observacion}
\end{ejercicio}


\begin{ejercicio}
    Demuestre que si \(x(t)\) es una solución de la ecuación diferencial
    \begin{equation}\label{eq:ej1.8_1}
        x'' + x = 0,
    \end{equation}
    entonces también cumple, para alguna constante \(c \in \bb{R}\),
    \begin{equation}\label{eq:ej1.8_2}
        (x')^2 + x^2 = c.
    \end{equation}

    Encuentre una solución de $(x')^2 + x^2 = 1$ que no sea solución de \eqref{eq:ej1.8_1}.\\

    \begin{proof}
        Sea $I\subset \bb{R}$ el intervalo de definición de $x(t)$ solución de \eqref{eq:ej1.8_1}. Definimos la función auxiliar
        \Func{f}{I}{\bb{R}}{t}{(x'(t))^2 + x^2(t).}

        Por ser $x$ una solición de una ecuación diferencial de segundo orden, tenemos que $x\in C^2(I)$. Por tanto, $x,~x'\in C^1(I)$ y, por tanto $f$ es derivable. Calculamos su derivada:
        \begin{equation*}
            f'(t) = 2x'(t)x''(t) + 2x(t)x'(t) = 2x'(t)~\left[ x''(t) + x(t) \right] = 2x'(t)\cdot 0 = 0.
        \end{equation*}

        Por tanto, $f'(t)=0$ para todo $t\in I$, lo que implica que $f$ es constante en $I$. Es decir, existe $c\in \bb{R}$ tal que
        \begin{equation*}
            (x'(t))^2 + x^2(t) = c \quad \forall t\in I.
        \end{equation*}

        Por tanto, queda demostrado lo pedido.
    \end{proof}

    Para la segunda parte, sea la solución $x(t) = 1$ para todo $t\in \bb{R}$. Entonces, tenemos que:
    \begin{align*}
        (x'(t))^2 + x^2(t) &= 0^2 + 1^2 = 1,\\
        x''(t) + x(t) &= 0 + 1 = 1
    \end{align*}
\end{ejercicio}



\begin{ejercicio}
    Una nadadora intenta atravesar un río pasando de la orilla \(y = -1\) a la orilla opuesta \(y = 1\).
    La corriente es uniforme, con velocidad \(v_R > 0\) y paralela a la orilla.
    Por otra parte, la nadadora se mueve a velocidad constante \(v_N > 0\) y apunta siempre hacia una torre situada en el punto \(T = (2, 1)\).
    Las ecuaciones
    \begin{align*}
        \dfrac{dx}{dt} &= v_R + v_N \cdot \dfrac{2 - x}{\sqrt{(2 - x)^2 + (1 - y)^2}},\\
        \dfrac{dy}{dt} &= v_N \cdot \dfrac{1 - y}{\sqrt{(2 - x)^2 + (1 - y)^2}},
    \end{align*}
    describen la posición \((x, y)\) de la nadadora en el instante \(t\); es decir \(x = x(t)\), \(y = y(t)\).
    \begin{enumerate}
        \item Explique cómo se ha obtenido este sistema.
        \item Encuentre la ecuación diferencial de la órbita \(y = y(x)\).
    \end{enumerate}
\end{ejercicio}


\begin{ejercicio}
    Encuentre una ecuación diferencial de segundo orden que admita como soluciones a las siguientes familias de funciones, donde \(c_1, c_2 \in \bb{R}\):
    \begin{enumerate}
        \item \(x = c_1e^t + c_2e^{-t}\),
        \item \(x = c_1\cosh t + c_2\senh t\).
    \end{enumerate}
\end{ejercicio}


\begin{ejercicio}
    Dada la ecuación de Clairaut:
    \begin{equation*}
        x = tx_0 + \varphi(x_0)
    \end{equation*}
    \begin{enumerate}
        \item Encuentre una familia uniparamétrica de soluciones rectilíneas.
        \item Suponiendo que \(\varphi(x) = x^2\), demuestre que \(x(t) = -\dfrac{t^2}{4}\) también es solución.
        \item ¿Qué relación hay entre esta solución y las que se han encontrado antes?
    \end{enumerate}
\end{ejercicio}

\begin{ejercicio}
    Resuelva los problemas 6 y 7 de la página 33 (sección 2.6) del libro de Ahmad-Ambrosetti.
\end{ejercicio}


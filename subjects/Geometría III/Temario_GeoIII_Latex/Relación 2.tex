\section{El Espacio Afín Euclídeo.}\label{Rel:Tema2}

\begin{ejercicio}
    Sean $\cc{A}$ un espacio afín euclídeo y S un subespacio afín suyo. Dado un punto $p \in \cc{A}$ demuestra que existe $q_0 \in S$ tal que
    \begin{equation*}
        d(p, q_0) = d(p, S) := \inf\{d(p, q) : q \in S\}.
    \end{equation*}
\end{ejercicio}

\begin{ejercicio}[Hiperplano afín de puntos equidistantes] Dados tres puntos $p, q, r \in~\bb{R}^n$, demostrar que se cumple la igualdad
    \begin{equation*}
        d(p,q)^2 - d(q,r)^2 = 2\langle\vec{rm_{pq}}, \vec{qp}\rangle,
    \end{equation*}
    donde $m_{pq}$ es el punto medio entre $p$ y $q$. Utilizar esta igualdad para probar lo siguiente: si $p \neq q$, entonces el conjunto de los puntos de $\bb{R}^n$ que se encuentran a la misma distancia de $p$ y de $q$ coincide con el hiperplano $m_{pq} + \cc{L}(\{\vec{p q}\})^\perp$.
\end{ejercicio}

\begin{ejercicio}
    Dados los siguientes pares de rectas de $\bb{R}^2$, estudia su posición relativa. Si se cortan, determina el ángulo que forman; en otro caso, calcula la distancia entre ellas.
    \begin{enumerate}
        \item $R=\{(x,y)\in \bb{R}^2 \mid x=y\}$, $S=\{(x,y)\in \bb{R}^2\mid 2x-y=0\}$.
        \item $R=\{(x,y)\in \bb{R}^2 \mid x-y=1\}$, $S=\{(2\lambda,1+2\lambda) \mid \lambda\in \bb{R}\}$.
    \end{enumerate}
\end{ejercicio}

\begin{ejercicio}
    En el espacio vectorial $\cc{P}_2(\bb{R})$, consideramos el producto escalar definido como $\langle p(x), q(x)\rangle = \int_0^1 p(x)q(x)~dx$.
    Dotamos $\cc{P}_2(\bb{R})$ de la estructura afín canónica (que lo convierte en un espacio afín euclídeo) y consideramos las siguientes rectas afines:
    \begin{equation*}
        S = \{p(x) \in \cc{P}_2(\bb{R}) \mid p(0) = 5, p''(8) = 4\}
        \qquad \text{y} \qquad
        T = \cc{L}\{5x^2 - 2x, 2x^2 - x + 1\}.
    \end{equation*} 
    Comprueba que S y T se cortan en un punto, y calcula el ángulo que forman.

    El punto de corte es el $(5,0,2)$. 
\end{ejercicio}

\begin{ejercicio}
    En el espacio vectorial $\cc{M}_{2\times 2}(\bb{R})$, consideramos el producto escalar definido como $\langle M, N\rangle = tr(M^tN)$. Dotamos $\cc{M}_{2\times 2}(\bb{R})$ de la estructura afín canónica (que lo convierte en un espacio afín euclídeo). Calcula el ángulo que forman los siguientes hiperplanos afines:
    \begin{equation*}
        S = \{M \in \cc{M}_{2\times 2}(\bb{R}) \mid tr(M) = 2\}
        \quad \text{y}\quad 
        T = \left\{\left(\begin{array}{cc}
            a & b \\ c & d
        \end{array}\right) \in \cc{M}_{2\times 2}(\bb{R}) \mid a = 1 \right\}
    \end{equation*}


    Tenemos que $S$ viene dado por:
    \begin{equation*}
        S = \left\{\left(\begin{array}{cc}
            a & b \\ c & d
        \end{array}\right) \in \cc{M}_{2\times 2}(\bb{R}) \mid a + d = 2 \right\}
    \end{equation*}

    Calculo el vector normal a $S$: $n_s=\left(\begin{array}{cc}
            1 & 0 \\ 0 & 1
        \end{array}\right)$, y el vector normal a $T$ es $n_T=\left(\begin{array}{cc}
            1 & 0 \\ 0 & 0
        \end{array}\right)$.
\end{ejercicio}


\begin{ejercicio}
    En $\bb{R}^2$ consideramos dos triángulos $T_1 = \{a_1, a_2, a_3\}$ y $T_2=~\{b_1, b_2, b_3\}$. Demuestra que:
    \begin{enumerate}
        \item Existen seis aplicaciones afines de $\bb{R}^2$ en $\bb{R}^2$ que llevan $T_1$ en $T_2$.

        Las permutaciones, que son $3!=6$.

        \item Una de las aplicaciones afines anteriores $f : \bb{R}^2 \to \bb{R}^2$ es isometría si y sólo si
        \begin{equation*}
            d(a_i, a_j ) = d(f(a_i), f(a_j )), \qquad \forall i, j \in \{1, 2, 3\},
        \end{equation*}
        donde $d(\cdot, \cdot)$ es la función distancia de $\bb{R}^2$.
    \end{enumerate}
\end{ejercicio}

\begin{ejercicio}
    En $\bb{R}^3$, considera el plano afín $\Pi = \{(x, y, z) \in \bb{R}^3 \mid x + y - z = -1\}$. Sea $s : \bb{R}^3 \to \bb{R}^3$ la simetría respecto de $\Pi$. Calcula la imagen mediante $s$ de la recta dada por las ecuaciones $r\equiv \frac{x-1}{2} =\frac{y+1}{-3} = z - 1$.

    Sabemos que $r=(1,-1,1)+\cc{L}(2, -3, 1)$. Como el vector director de la recta no es un vector director del plano, tenemos que son secantes. Calculemos la intersección:
    \begin{equation*}
        \left\{\right.
    \end{equation*}
    Esta intersección es $(1, -1, 1)$.

    Para calcular la simetría, calculamos el simétrico de $(2,-3,1)$. Para ello, lo descomponemos en $\vec{Pi}, \vec{Pi}^\perp$. Dado $T\in \vec{Pi}$, $N\in \vec{Pi}^\perp$ unitario, tenemos que $V=T+\alpha N$, para cierto $\alpha\in \bb{R}$. Calculemos $\alpha$:
    \begin{equation*}
        \langle V,N\rangle = \langle V,T\rangle + \alpha \langle N,N\rangle = \alpha
    \end{equation*}
\end{ejercicio}

\begin{ejercicio}
     Una aplicación afín $f : (A, \langle , \rangle) \to (A', \langle , \rangle')$ entre espacios afines euclidianos se dice que preserva la ortogonalidad si para cualesquiera rectas secantes $R, S$ en $\cc{A}$ tales que $R \perp S$, entonces $f(\bb{R}) \perp f(S)$.

     Probar que si $f : (A, \langle , \rangle) \to (A', \langle , \rangle')$ es una aplicación afín biyectiva, entonces $f$ es una semejanza (composición de un movimiento rígido y una homotecia) si, y sólo si, $f$ preserva la ortogonalidad.
\end{ejercicio}

\begin{ejercicio}
    Encuentra, si existe, un movimiento rígido de $\bb{R}^2$ que lleve la recta $\{(x, y) \in \bb{R}^2 \mid x = 0\}$ en la recta $\{(x, y) \in \bb{R}^2 \mid y = 1\}$, y también lleve la recta $\{(x, y) \in \bb{R}^2 \mid y =0\}$ en la recta $\{(x, y) \in \bb{R}^2 \mid x = 1\}$.

    Componemos una traslación en $v=(1,1)$ con un giro de 90 grados.

    También sirve una simetría respecto de la recta $y=-x+1$.
\end{ejercicio}

\begin{ejercicio}
    Sean $f_1, f_2$ las simetrías (ortogonales) de $\bb{R}^2$ respecto de las rectas $R_1 = \{(x, y) \in \bb{R}^2 \mid x - y = 2\}$ y $R_2=\{(x, y) \in \bb{R}^2 \mid x - 2y = 1\}$, respectivamente. Calcula $f_1 \circ f_2$ y descríbela geométricamente.
\end{ejercicio}

\begin{ejercicio}
    Considera un espacio afín euclídeo $\cc{A}$ de dimensión 3, y sea $f$ un movimiento rígido de $\cc{A}$ tal que $f(1, 0, 1) = (2, -3, 1)$ en coordenadas de un sistema de referencia euclídeo fijo. Si sabemos que $f$ es la simetría respecto de un plano, calcula dicho plano.
\end{ejercicio}

\begin{ejercicio}
    Sea $\cc{A}$ un espacio afín euclídeo de dimensión 2, y sean $R_1$ y $R_2$ dos rectas de $\cc{A}$. Prueba que siempre es posible encontrar un movimiento rígido $f : \cc{A} \to \cc{A}$ que lleve $R_1$ en $R_2$. Estudia de qué tipo es $f$, según la posición relativa de $R_1$ y $R_2$. ¿Se puede elegir siempre directo? ¿E inverso?
\end{ejercicio}

\begin{ejercicio}
    Sean $p$ y $q$ dos puntos distintos en un espacio afín euclídeo. Demuestra que existe una única simetría respecto de un hiperplano que lleva $p$ en $q$.
\end{ejercicio}

\begin{ejercicio}
     Sean $R_1$ y $R_2$ dos rectas que se cruzan en un espacio afín euclídeo tridimensional $\cc{A}$. Demuestra que existe una única recta afín $R$ que interseca de manera ortogonal a $R_1$ y $R_2$. Prueba además que la distancia de $R_1$ a $R_2$ es exactamente la distancia entre los puntos dados por $R_1 \cap R$ y $R_2 \cap R$.
\end{ejercicio}

\begin{ejercicio}
    Consideremos la aplicación $f : \bb{R}^2 \to \bb{R}^2$ que viene dada por $f(x, y) = (y -2, x+ 1)$. ¿Es $f$ un movimiento rígido? En tal caso, clasifícalo.
\end{ejercicio}

\begin{ejercicio}
     Demostrar que si $p$ y $q$ son dos puntos de un espacio afín euclídeo $\cc{A}$, entonces siempre existe un movimiento rígido $f : \cc{A} \to \cc{A}$ tal que $f(p) = q$. De forma más general, probar que si $\cc{A}$ tiene dimensión finita y $S$, $S'$ son dos subespacios afines de $\cc{A}$ de dimensión $m$, entonces existe un movimiento rígido $f : \cc{A} \to \cc{A}$ tal que $f(S) = S_0$.
\end{ejercicio}

\begin{ejercicio}
     Consideremos la aplicación $f : \bb{R}^2 \to \bb{R}^2$ que viene dada por $f(x, y) = (2y - 1, -2x + 3)$. ¿Es $f$ un movimiento rígido? En tal caso, clasifícalo.
\end{ejercicio}

\begin{ejercicio}
    Sean $f_1, f_2 : \bb{R}^2\to \bb{R}^2$ las isometrías dadas, respectivamente, por las simetrías respecto de las rectas de ecuación $x + y = 0$ y $x + 2y = 2$.
    \begin{enumerate}
        \item Calcula explícitamente $f_1$ y $f_2$ en coordenadas usuales.
        \item Clasifica el movimiento rígido $g = f_1 \circ f_2$.
    \end{enumerate}
\end{ejercicio}

\begin{ejercicio}
    Demuestra que las siguientes aplicaciones son movimientos rígidos del plano y clasifícalos.
    \begin{enumerate}
        \item $f\left(x, y\right) = \left(3 - \frac{3x}{5} + \frac{4y}{5}, 1 - \frac{4x}{5} - \frac{3y}{5}\right).$
        \item $f\left(x, y\right) = \left(\frac{x}{2} -\frac{\sqrt{3} y}{2} + 1,\frac{\sqrt{3} x}{2}+ \frac{y}{2} + 2\right).$
        \item $f\left(x, y\right) = \left(-\frac{x}{2} + \frac{\sqrt{3 y}}{2} + 1,\frac{\sqrt{3} x}{2} + \frac{y}{2} - 1\right).$
        \item $f\left(x, y\right) = \left(\frac{3x}{5} + \frac{4y}{5} + 2, \frac{4x}{5} - \frac{3y}{5} + 5\right).$
    \end{enumerate}
\end{ejercicio}

\begin{ejercicio}
    Demuestra que las siguientes aplicaciones son movimientos rígidos del espacio y clasifícalos.
    \begin{enumerate}
        \item $f\left(x, y, z\right) = \left(2 + y, x, 1 + z\right)$.
        
        Simetría respecto del plano $z=3x-5$.
        
        \item $f\left(x, y, z\right) = \left(\frac{x}{2} -\frac{\sqrt{3} z}{2} + 2, y + 2,\frac{\sqrt{3} x}{2} + \frac{z}{2} + 2\right)$.

        Simetría respecto del plano $3x+2y-z=5$ con deslizamiento
        
        \item $f\left(x, y, z\right) = \left(-\frac{4x}{5} + \frac{3z}{5} + 3, y, \frac{3x}{5} + \frac{4z}{5} - 1\right)$.
        \item $f\left(x, y, z\right) = \left(-\frac{4x}{5} + \frac{3z}{5} + 3, y + 4, \frac{3x}{5} + \frac{4z}{5} - 1\right)$.
        \item $f\left(x, y, z\right) = \left(\frac{2y}{\sqrt{5}} + \frac{z}{\sqrt{5}},\frac{\sqrt{5}x}{3} - \frac{2y}{3\sqrt{5}} + \frac{4z}{3\sqrt{5}}, -\frac{2x}{3} - \frac{y}{3} + \frac{2z}{3}\right)$.
        \item $f\left(x, y, z\right) = \left(\frac{\sqrt{5} x}{3} - \frac{2y}{3\sqrt{5}} + \frac{4z}{3\sqrt{5}}, \frac{2y}{\sqrt{5}} + \frac{z}{\sqrt{5}}, -\frac{2x}{3}- \frac{y}{3} + \frac{2z}{3}\right)$.
    \end{enumerate}
\end{ejercicio}

\begin{ejercicio}
    Calcula en coordenadas usuales de $\bb{R}^2$ el giro centrado en el punto $c = (1, 2)$ y de ángulo $\frac{2\pi}{3}$.
\end{ejercicio}

\begin{ejercicio}
    Calcula la simetría con deslizamiento respecto de la recta $x - y = 1$ de $\bb{R}^2$ y vector de desplazamiento $v = (-2, -2)$.
\end{ejercicio}

\begin{ejercicio}
    Sea $f : \bb{R}^2\to \bb{R}^2$ la aplicación afín dada por
    \begin{equation*}
        f(-1, -1) = (0, 0),\qquad  f(-1, -2) = (1, 0),\qquad f(0, -1) = (0, 1).
    \end{equation*}
    Demuestra que $f$ es un movimiento rígido y clasifícalo.
\end{ejercicio}

\begin{ejercicio}
    Sea $\cc{R}$ el sistema de referencia de $\bb{R}^2$ con origen en el punto $(1, 1)$ y base asociada $\{(1, 1),(-1, 1)\}$. Consideremos la aplicación afín $f$ tal que, si $(x, y)$ son las coordenadas de un punto genérico $p$ en el sistema de referencia $\cc{R}$, entonces las coordenadas de $f(p)$ en el sistema de referencia usual vienen dadas por
    \begin{equation*}
        \left(\begin{array}{c}
            -1 \\ 3
        \end{array}\right)+
        \left(\begin{array}{cc}
            1 & 1 \\ 1 & -1
        \end{array}\right)
        \left(\begin{array}{c}
            x \\ y
        \end{array}\right).
    \end{equation*}
    ¿Es $f$ un movimiento rígido? En caso afirmativo, clasifícalo.
\end{ejercicio}

\begin{ejercicio}
    Sean $f_1, f_2 : \bb{R}^3\to \bb{R}^3$ las isometrías dadas respectivamente por las simetrías respecto de los planos de ecuaciones $x + y = 1$ y $x - z = 2$.
    \begin{enumerate}
        \item Calcula explícitamente $f_1$ y $f_2$ en coordenadas usuales
        \item Clasifica el movimiento rígido $g = f_1 \circ f_2$.
    \end{enumerate}
\end{ejercicio}

\begin{ejercicio}
    Calcula en coordenadas usuales la isometría de $\bb{R}^3$ dada por el movimiento helicoidal alrededor de la recta $R \equiv (1, 2, 1) + \cc{L}(1, 0, -1)$ con giro de ángulo $\frac{\pi}{2}$ y vector de traslación $v = (-2, 0, 2)$.
\end{ejercicio}

\begin{ejercicio}
     Clasifica el siguiente movimiento rígido de $\bb{R}^3$:
     \begin{equation*}
         f(x, y, z) = \frac{1}{3}(2x + 2y + z + 2, x - 2y + 2z - 2, 2x - y - 2z - 4)
     \end{equation*}
     y calcula sus elementos notables (o geométricos).
\end{ejercicio}

\begin{ejercicio}
     Calcula la simetría con deslizamiento respecto del plano de ecuación $x + y + z = 1$ de $\bb{R}^3$ y con vector de traslación $v = (2, -1, -1)$.
\end{ejercicio}

\begin{ejercicio}
     Clasifica el movimiento rígido de $\bb{R}^3$ dado por
     \begin{equation*}
         f(x,y,z) = \left(
         \frac{2x+2y+z+3}{3}, \frac{-2x+y+2z}{3}, \frac{-x+2y-2z-3}{3}
         \right)
     \end{equation*}
\end{ejercicio}

\begin{ejercicio}
    Calcula la isometría de $\bb{R}^3$ dada por la composición de un giro de ángulo $\frac{\pi}{2}$ respecto del eje $R \equiv (1, 2, 0) + \cc{L}(\{(0, 1, 0)\})$ y la simetría respecto del plano $y = -1$.
\end{ejercicio}

\begin{ejercicio}
    Para cada $\alpha \in \bb{R}$ se considera el movimiento rígido de $\bb{R}^3$ dado por:
    \begin{equation*}
        f_\alpha(x,y,z)=\frac{1}{3}(-x + 2y + 2z + 2, 2x + 2y - z - 1, 2x - y + 2z - \alpha).
    \end{equation*}
    Clasificar, según los valores de $\alpha$, qué tipo de movimiento es $f_\alpha$, calculando en cada caso el conjunto de puntos fijos.
\end{ejercicio}

\begin{ejercicio}
    Sea $\cc{R}$ es el sistema de referencia con origen en el punto $(1, 0, 1)$ y base asociada $\{(1, 0, 0),(1, 1, 0),(1, 1, 1)\}$. Determina si la siguiente aplicación afín $f : \bb{R}^3\to \bb{R}^3$, que en coordenadas respecto del sistema de referencia afín $\cc{R}$ está dada por
    \begin{equation*}
        f\left(\begin{array}{c}
            x \\ y \\ z
        \end{array}\right) = \frac{1}{5}
        \left[
        \left(\begin{array}{c}
            -9 \\ 16 \\ -7
        \end{array}\right) +
        \left(\begin{array}{ccc}
            5 & 2 & -2 \\
            0 & 7 & 8 \\
            0 & -4 & -1
        \end{array}\right)
        \left(\begin{array}{c}
            x \\ y \\ z
        \end{array}\right)
        \right]
    \end{equation*}
    es un movimiento rígido y, en caso afirmativo, clasifícalo.
\end{ejercicio}

\begin{ejercicio}
    Decide de forma razonada qué tipo de movimiento rígido es:
    \begin{enumerate}
        \item La composición de dos simetrías ortogonales en el plano euclídeo $\bb{R}^2$.
        \item La composición de dos simetrías ortogonales con deslizamiento en el plano euclídeo $\bb{R}^2$.
        \item La composición de un giro y una simetría en el plano euclídeo $\bb{R}^2$.
        \item La composición de un giro y una simetría con deslizamiento en el plano euclídeo~$\bb{R}^2$.
        \item La composición de dos simetrías ortogonales en el espacio euclídeo $\bb{R}^3$.
        \item La composición de un giro y una simetría en el espacio euclídeo $\bb{R}^3$.
        \item La composición de un giro y una traslación en el espacio euclídeo $\bb{R}^3$.
        \item La composición de dos simetrías centrales en el espacio euclídeo $\bb{R}^3$.
    \end{enumerate}
\end{ejercicio}

\begin{ejercicio}
    Demuestra que la composición de dos simetrías respecto de dos puntos distintos es una traslación.
\end{ejercicio}

\begin{ejercicio}
    Sea $T$ un triángulo en un espacio afín euclídeo $\cc{A}$ con vértices $a, b, c \in \cc{A}$. La recta que pasa por el vértice $a$ y con vector director
    \begin{equation*}
        v_a=\frac{1}{\|\vec{ab}\|} \vec{ab} + \frac{1}{\|\vec{ac}\|}\vec{ac}
    \end{equation*}
    la llamamos bisectriz que pasa por $a$. Si se definen de manera análoga las bisectrices que pasan por los vértices $b$ y $c$, prueba que las tres rectas se cortan en un mismo punto, que llamaremos incentro del triángulo.
\end{ejercicio}

\begin{ejercicio}
    Calcula el baricentro, ortocentro, circuncentro e incentro del triángulo de $\bb{R}^2$ que tiene por vértices a los puntos $(0, 0)$,$(1, 0)$ y $(0, 1)$.
\end{ejercicio}

\begin{ejercicio}
    ¿Está el incentro de cualquier triángulo alineado con el baricentro, ortocentro y circuncentro?
\end{ejercicio}

\begin{ejercicio}[Ejercicio de Examen 2022-23]
    Demostrar que, en un triángulo isósceles, los 4 puntos notables de un triángulo están alineados.


    Tenemos que ver que los 4 puntos están en una misma recta. Esto se debe a que la misma altura
\end{ejercicio}
\section{Hipercuádricas afines.}\label{Rel:Tema3}


\begin{ejercicio}
    Clasifica las hipercuádricas de un espacio afín euclídeo de dimensión 1.

    Tenemos que la forma cuadrática de toda hipercuádrica en cierto sistema de referencia euclídeo $\cc{R}$ es una de las siguientes:
    \begin{enumerate}
        \item $\frac{x^2}{a^2}=0$.

        Es un único punto doble, el origen.

        
        \item $\frac{x^2}{a^2}=1$.

        Son dos puntos, $x=\pm a$.
        
        \item $\frac{x^2}{a^2}=-1$.
        
        No es posible, por lo que es el vacío.
    \end{enumerate}
\end{ejercicio}


\begin{ejercicio}
    En el semiplano $P = \{(x, y, z) \in \bb{R}^3\mid y = 0, x \geq 0\}$ tomamos una circunferencia $C$ de centro $(c, 0, 0)$ y radio $r > 0$ con $C > r > 0$. Se llama toro de revolución generado por $C$ a la superficie $T$ obtenida al rotar $C$ alrededor del eje $z$. Dibujar $T$ y describir la superficie como el conjunto de soluciones de una ecuación con 3 incógnitas. ¿Es dicha ecuación la de una cuádrica?
\end{ejercicio}


\begin{ejercicio}
    Sea $\cc{A}$ un espacio afín de dimensión $n$ y $S$ un subespacio afín suyo de dimensión $m > 0$. Demuestra que:
    \begin{enumerate}
        \item Existe un sistema de referencia $\cc{R}$ de $\cc{A}$ tal que las ecuaciones implícitas de $S$ en dicho sistema son $x_{m+1} = 0, \dots , x_n = 0$.
    \item  Si $H$ es una hipercuádrica de $\cc{A}$ entonces $H \cap S$ es una hipercuádrica de $S$ o bien vacío o todo $S$.
    \end{enumerate}
\end{ejercicio}



\begin{ejercicio}
    Sean $H$ una hipercuádrica y $R$ una recta de un espacio afín $\cc{A}$. Prueba que $R \cap H$ puede ser vacío, un punto, dos puntos o toda la recta. Da un ejemplo conocido de cada uno de los casos cuando $\cc{A}$ tiene dimensión dimensión 2 y dimensión 3.
\end{ejercicio}



\begin{ejercicio}
     Construir explícitamente un isomorfismo afín $f : \bb{R}^n \to \bb{R}^n$ tal que $f(C) = C'$ en cada uno de los siguientes casos:
     \begin{enumerate}
         \item $n=2$, $C=\left\{(x,y)\in \bb{R}^2\left|\dfrac{x^2}{a^2} + \dfrac{y^2}{b^2}=1\right.\right\}$, $C'=\left\{(x,y)\in \bb{R}^2\left|x^2 - y^2=1\right.\right\}$.
         \item $n=3$, $C=\left\{(x,y,z)\in \bb{R}^3\left|x^2 + y^2+z^2=1\right.\right\}$, $C'=\left\{(x,y,z)\in \bb{R}^3\left|\dfrac{x^2}{a^2} + \dfrac{y^2}{b^2} + \dfrac{z^2}{c^2}=1\right.\right\}$.
         \item $n=2$, $C=\left\{(x,y)\in \bb{R}^2\left|x^2-y=0=1\right.\right\}$, $C'=\left\{(x,y)\in \bb{R}^2\left|x-y^2=0\right.\right\}$.
         \item $n=3$, $C=\left\{(x,y,z)\in \bb{R}^3\left|ax^2+by^2=1\right.\right\}$, $C'=\left\{(x,y,z)\in \bb{R}^3\left|x^2+y^2=1\right.\right\}$.
     \end{enumerate}
\end{ejercicio}



\begin{ejercicio} Cuestiones sobre elipses:
\begin{enumerate}
    \item Dados dos puntos distintos $p_1, p_2$ de un plano afín euclídeo $\cc{A}$ y $r > \frac{1}{2} d(p_1, p_2)$, demuestra que $H=\{p \in \cc{A} \mid d(p, p_1) + d(p, p_2) = r\}$ es una elipse. Los puntos $p_1, p_2$ reciben el nombre de focos de la elipse. Se llama centro de la elipse al punto medio de sus focos y vértices a los puntos de intersección de la elipse con la recta que pasa por sus focos.

    \item Prueba que toda elipse se puede escribir como en el apartado anterior, para ciertos puntos $p_1, p_2 \in \cc{A}$.

    \item Demuestra que toda elipse $H$ es simétrica con respecto a la recta $R_{p_1p_2}$ que pasa por sus focos y con respecto a la mediatriz de sus focos.

    \item Prueba que, para cada punto $p$ de una elipse $H$, la recta tangente a $H$ en $p$ forma ángulos iguales con las rectas que pasan por $p$ y cada uno de sus focos.
\end{enumerate}
\end{ejercicio}



\begin{ejercicio} Cuestiones sobre hipérbolas:
\begin{enumerate}
    \item Dados dos puntos distintos$ p_1, p_2$ de un plano afín euclídeo $\cc{A}$ y $r > \frac{1}{2} d(p_1, p_2)$, demuestra que $H=\{p \in \cc{A} \mid |d(p, p_1) - d(p, p_2)| = r\}$ es una hipérbola. Los puntos $p_1, p_2$ reciben el nombre de focos de la hipérbola. Se llama centro de la hipérbola al punto medio de sus focos y vértices a los puntos de intersección de la hipérbola con la recta que pasa por sus focos.

    \item Prueba que toda hipérbola se puede escribir como en el apartado anterior, para ciertos puntos $p_1, p_2 \in \cc{A}$.

    \item  Demuestra que toda hipérbola $H$ es simétrica con respecto a la recta $R_{p_1p_2}$ que pasa por sus focos y con respecto a la mediatriz de sus focos.

    \item  Prueba que, para cada punto $p$ de una hipérbola $H$, la recta tangente a $H$ en $p$ forma ángulos iguales con las rectas que pasan por $p$ y cada uno de sus focos.
\end{enumerate}
\end{ejercicio}


\begin{ejercicio} Cuestiones sobre parábolas:
\begin{enumerate}
    \item  Sean $p_0$ un punto de un plano afín euclídeo $\cc{A}$ y $R$ una recta de $\cc{A}$ que no contiene a $p_0$. Demuestra que $H=\{p \in \cc{A} \mid d(p, p_0) = d(p, R)\}$ es una parábola. El punto $p_0$ recibe el nombre de foco de la parábola y $R$ se llama directriz de la parábola. Se llama vértice de la parábola al punto de la parábola más próximo al foco o, equivalentemente, a la directriz.

    \item Prueba que toda parábola se puede escribir como en el apartado anterior, para cierto punto $p_0$ y cierta recta $R$ de $\cc{A}$.

    \item Demuestra que toda parábola $H$ es simétrica con respecto a la recta que pasa por su foco y es perpendicular a su directriz (esta recta se llama eje de la parábola).

    \item Prueba que, para cada punto $p$ de una parábola $H$, la recta tangente a $H$ en $p$ forma ángulos iguales con la recta que pasa por $p$ y su foco y con la recta que pasa por $p$ y es paralela al eje de la parábola.
\end{enumerate}
\end{ejercicio}

\begin{ejercicio}
    Consideremos las siguientes rectas de $\bb{R}^3$:
    \begin{equation*}
        R_1 = (1, 0, 0) + \cc{L}\{(0, 1, 1)\}\hspace{1.5cm}
        R_2 = (1, 0, 0) + \cc{L}\{(0, -1, 1)\}
    \end{equation*}
    \begin{enumerate}
        \item  Demuestra que la superficie generada al rotar $R_1$ (o bien $R_2$) alrededor del eje $z$ es el hiperboloide de una hoja que tiene ecuación $x^2+y^2-z^2=1$.

        \item Deduce que si $H$ es cualquier hiperboloide de una hoja de $\bb{R}^3$ y $p \in H$ entonces existen dos rectas distintas contenidas en $H$ que pasan por $p$.
    \end{enumerate}
\end{ejercicio}


\begin{ejercicio}
    Prueba que cualquier plano de $\bb{R}^3$ corta al hiperboloide de una hoja que tiene ecuación $x^2+y^2-z^2 = 1$.
\end{ejercicio}



\begin{ejercicio}
    Encuentra, si existe, una parábola de $\bb{R}^2$ que pase por los puntos $(2, 0), (0, 1), (3, 1)$ y $(0, 0)$.

    Tenemos que una hipercuádrica es del tipo:
    \begin{equation*}
        Ax^2+By^2+Cxy+Dx+Ey+F=0
    \end{equation*}

    Establecemos las condiciones de contorno dadas:
    \begin{equation*}
        \begin{split}
            (2,0)&\longrightarrow 4A+2D+F=0\\
            (0,1)&\longrightarrow B+E+F=0\\
            (3,1)&\longrightarrow 9A+B+3C+3D+E+F=0\\
            (0,0)&\longrightarrow F=0
        \end{split}
    \end{equation*}

    Por tanto, el sistema a revolver queda:
    \begin{equation*}
        \left\{\begin{array}{l}
            F=0\\
            2A+D=0 \\
            B+E=0\\
            9A+B+3C+3D+E=0
        \end{array}\right\}
        \Longrightarrow
        \left\{\begin{array}{l}
            F=0\\
            D=-2A \\
            E=-B\\
            9A+B+3C-6A-B=0
        \end{array}\right\}
        \Longrightarrow
        \left\{\begin{array}{l}
            F=0\\
            D=-2A \\
            E=-B\\
            C=-A
        \end{array}\right\}
    \end{equation*}

    Por tanto, nuestra hipercuádrica queda:
    \begin{equation*}
        Ax^2+By^2-Axy-2Ax-By=0 \qquad A,B\in \bb{R}
    \end{equation*}

    Por tanto, hay gran cantidad de hipercuádricas que pasan por los 4 puntos. Clasifiquémoslas en función de $A,B\in \bb{R}$:
    \begin{equation*}
        \begin{split}
            0 &= Ax^2+By^2-Axy-2Ax-By =\\
            &= A[x^2-x(y+2)]+By^2-By =\\
            &= A\left[x-\frac{y+2}{2}\right]^2 - A\cdot \frac{(y+2)^2}{4} +By^2-By =\\
            &= A\left(x-\frac{y+2}{2}\right)^2 + \left(B-\frac{A}{4}\right)y^2 -(A+B)y -A\\
        \end{split}
    \end{equation*}

    Como en una parábola existe un sistema de referencia $\cc{R}'$ tal que su ecuación reducida en dicho sistema es $\wt{x}^2=2\wt{y}$, entonces necesito que la segunda coordenada no esté elevado al cuadrado. Por tanto,
    \begin{equation*}
        B-\frac{A}{4} = 0 \Longleftrightarrow A=4B
    \end{equation*}

    Tenemos entonces que la hipercuádrica queda:
    \begin{equation*}
        0 = 4B\left(x-\frac{y+2}{2}\right)^2 -5By -4B
        = 4B\left(x-\frac{y+2}{2}\right)^2 -2\cdot \frac{5By+4B}{2}
    \end{equation*}

    Aplicamos entonces el cambio de sistema de referencia a $\cc{R}'$, de forma que:
    \begin{equation*}
        M(Id, \cc{R}_0, \cc{R}') = \left(\begin{array}{c|cc}
            1 & 0 & 0 \\ \hline
            -2\sqrt{B} & 2\sqrt{B} & -\sqrt{B}\\ 
            2B & 0 & 5B
        \end{array}\right)
    \end{equation*}

    Tenemos que se trata de un sistema de referencia para todo $B\in \bb{R}^\ast$, por lo que $B\neq 0$. En dicho sistema, si las coordenadas de un punto son ${(\wt{x}, \wt{y})}_{\cc{R}'}$, tenemos que:
    \begin{equation*}
        \wt{x}^2 = 2\wt{y}^2
    \end{equation*}

    Es decir, se trata de una parábola. Por tanto, tenemos que las parábolas que pasan por dichos puntos son de la forma:
    \begin{equation*}
        4Bx^2 + By^2-4Bxy -8Bx-By=0 \Longleftrightarrow 4x^2 + y^2-4xy -8x-y=0
    \end{equation*}

    Y por tanto, hemos determinado de forma única dicha parábola. Notemos que hemos simplificado porque $B\neq 0$.
\end{ejercicio}


\begin{ejercicio}
    Clasifica euclídeamente las siguientes cónicas de $\bb{R}^2$ y obtén, en cada caso, un sistema de referencia euclídeo en el cual su expresión sea reducida:
    \begin{enumerate}
        \item $-125 - 220x - 14x^2 - 40y - 96xy + 14y^2 = 0.$
        
        Tenemos que la matriz asociada a dicha cónica es:
        \begin{equation*}
            \wh{A} = \left(\begin{array}{c|cc}
                -125 & -110 & -20  \\ \hline
                -110 & -14 & -48 \\
                -20 &  -48 & 14
            \end{array}\right)
            = \left(\begin{array}{c|c}
                a & z \\ \hline
                z^t & A
            \end{array}\right)
        \end{equation*}
        
        Diagonalizamos $A$. Su polinomio característico es:
        \begin{equation*}
            \lambda^2 -2500 = 0 \Longleftrightarrow \lm = \pm 50
        \end{equation*}

        Sus vectores propios asociados son:
        \begin{equation*}
            V_{50} = \left\{(x,y)\in \bb{R}^2 \mid \left(\begin{array}{cc}
                -64 & -48 \\
                -48 & -36\\ 
            \end{array}\right)\left(\begin{array}{c}
                x\\y
            \end{array}\right)=0 \right\} = \cc{L}\left\{\frac{1}{5}(-3,4)\right\}
        \end{equation*}
        \begin{equation*}
            V_{-50} = \left\{(x,y)\in \bb{R}^2 \mid \left(\begin{array}{cc}
                36 & -48 \\
                -48 & 64\\ 
            \end{array}\right)\left(\begin{array}{c}
                x\\y
            \end{array}\right)=0 \right\} = \cc{L}\left\{\frac{1}{5}(4,3)\right\}
        \end{equation*}

        Además, para que el vector $\wt{z}$ (primera columna) sea entero nulo, buscamos que:
        \begin{equation*}
            Ac + z = 0
            =  \left(\begin{array}{cc}
                -14 & -48 \\
                -48 & 14
            \end{array}\right)\left(\begin{array}{c}
                x\\y
            \end{array}\right) + \left(\begin{array}{c}
                -110 \\ -20
            \end{array}\right) = \left(\begin{array}{c}
                0\\0
            \end{array}\right)
        \end{equation*}
        Esto se da si y solo si:
        \begin{equation*}
            \left(\begin{array}{cc}
                -14 & -48 \\
                -48 & 14
            \end{array}\right)\left(\begin{array}{c}
                x\\y
            \end{array}\right) = \left(\begin{array}{c}
                110 \\ 20
            \end{array}\right) \Longrightarrow c=(-1,-2)
        \end{equation*}

        Por tanto, definimos el sistema de referencia $\cc{R}'=\left\{(-1, -2), \left\{\dfrac{1}{5}(-3,4), \dfrac{1}{5}(4,3)\right\}\right\}$.

        Para calcular el nuevo valor de $\wt{a}$, sabemos que el determinante es invariante mediante semejanzas.
        \begin{equation*}
            |\wh{A}|=-62500= \wt{a}\cdot 50\cdot -50\Longrightarrow \wt{a}=25
        \end{equation*}
        
        
        Por tanto, la matriz asociada a la hipercuádrica en $\cc{R}'$ es:
        \begin{equation*}
            \left(\begin{array}{c|cc}
                25 & 0 & 0  \\ \hline
                0 & 50 & 0 \\
                0 & 0 & -50
            \end{array}\right)
        \end{equation*}
        Por tanto, su ecuación en dicho sistema de referencia es:
        \begin{equation*}
            50\wt{x}^2 -50\wt{y}^2 = -25 \Longleftrightarrow 2\wt{x}^2 - 2\wt{y}^2=-1
        \end{equation*}

        Por tanto, se trata de una hipérbola que corta al eje $X$ de ese nuevo sistema de referencia. Sus ejes son:
        \begin{equation*}
            e_1 = (-1,-2) + \cc{L}\{(-3,4)\}
            \qquad
            e_2 = (-1,-2) + \cc{L}\{(4,3)\}
        \end{equation*}
        Su centro es el punto $(-1,-2)$. La longitud de los semiejes (que son iguales, por ser equilátera) es $a=b=\frac{1}{\sqrt{2}}$. La mitad de la distancia focal es $c=1$. Entonces, los focos son:
        \begin{equation*}
            F_1={(1,0)}_{\cc{R}'}
            \qquad
            F_2={(-1,0)}_{\cc{R}'}
        \end{equation*}

        Las ecuaciones de las asíntotas en $\cc{R}'$ son:
        \begin{equation*}
            \frac{x}{a}+\frac{y}{b}=0
            \Longrightarrow y = -\frac{b}{a}x = -x
            \hspace{2cm}
            \frac{x}{a}-\frac{y}{b}=0
            \Longrightarrow y = \frac{b}{a}x = x
        \end{equation*}
        
        \item $3x^2 + 2xy + 3y^2 + 4\sqrt{2}x + 4\sqrt{2}y + 2 = 0.$

        Tenemos que la matriz asociada a dicha cónica es:
        \begin{equation*}
            \wh{A} = \left(\begin{array}{c|cc}
                2 & 2\sqrt{2} & 2\sqrt{2}  \\ \hline
                2\sqrt{2} & 3 & 1 \\
                2\sqrt{2} &  1 & 3
            \end{array}\right)
            = \left(\begin{array}{c|c}
                a & z \\ \hline
                z^t & A
            \end{array}\right)
        \end{equation*}
        
        Diagonalizamos $A$. Su polinomio característico es:
        \begin{equation*}
            \lambda^2 -6\lm +8 = 0 \Longleftrightarrow \left\{\begin{array}{l}
                \lm_1 = 4\\
                \lm_2 = 2
            \end{array}\right.
        \end{equation*}

        Sus vectores propios asociados son:
        \begin{equation*}
            V_{4} = \left\{(x,y)\in \bb{R}^2 \mid \left(\begin{array}{cc}
                -1 & 1 \\
                1 & -1\\ 
            \end{array}\right)\left(\begin{array}{c}
                x\\y
            \end{array}\right)=0 \right\} = \cc{L}\left\{\frac{1}{\sqrt{2}}(1,1)\right\}
        \end{equation*}
        \begin{equation*}
            V_{2} = \left\{(x,y)\in \bb{R}^2 \mid \left(\begin{array}{cc}
                1 & 1\\
                1 & 1
            \end{array}\right)\left(\begin{array}{c}
                x\\y
            \end{array}\right)=0 \right\} = \cc{L}\left\{\frac{1}{\sqrt{2}}(-1,1)\right\}
        \end{equation*}

        Además, para que el vector $\wt{z}$ (primera columna) sea entero nulo, buscamos que:
        \begin{equation*}
            Ac + z = 0
            =  \left(\begin{array}{cc}
                3 & 1 \\
                1 & 3
            \end{array}\right)\left(\begin{array}{c}
                x\\y
            \end{array}\right) + \left(\begin{array}{c}
                2\sqrt{2} \\ 2\sqrt{2}
            \end{array}\right) = \left(\begin{array}{c}
                0\\0
            \end{array}\right)
        \end{equation*}
        Esto se da si y solo si:
        \begin{equation*}
            \left(\begin{array}{cc}
                3 & 1 \\
                1 & 3
            \end{array}\right)\left(\begin{array}{c}
                x\\y
            \end{array}\right) = \left(\begin{array}{c}
                -2\sqrt{2} \\ -2\sqrt{2}
            \end{array}\right) \Longrightarrow c=\left(-\frac{\sqrt{2}}{2},-\frac{\sqrt{2}}{2}\right)
        \end{equation*}

        Por tanto, definimos el sistema de referencia $\cc{R}'=\left\{\left(-\frac{\sqrt{2}}{2},-\frac{\sqrt{2}}{2}\right), \left\{\dfrac{1}{\sqrt{2}}(1,1), \dfrac{1}{\sqrt{2}}(-1,1)\right\}\right\}$.

        Para calcular el nuevo valor de $\wt{a}$, sabemos que el determinante es invariante mediante semejanzas.
        \begin{equation*}
            |\wh{A}|=-16= \wt{a}\cdot 4\cdot 2\Longrightarrow \wt{a}=-2
        \end{equation*}
        
        
        Por tanto, su matriz asociada en $\cc{R}'$ es:
        \begin{equation*}
            \left(\begin{array}{c|cc}
                -2 & 0 & 0  \\ \hline
                0 & 4 & 0 \\
                0 & 0 & 2
            \end{array}\right)
        \end{equation*}
        Por tanto, su ecuación en dicho sistema de referencia es:
        \begin{equation*}
            4\wt{x}^2 +2\wt{y}^2 = 2 \Longleftrightarrow 2\wt{x}^2 +\wt{y}^2=1
        \end{equation*}

        Por tanto, se trata de una elipse. Sus ejes son:
        \begin{equation*}
            e_1 = \left(-\frac{\sqrt{2}}{2},-\frac{\sqrt{2}}{2}\right) + \cc{L}\{(1,1)\}
            \qquad
            e_2 = \left(-\frac{\sqrt{2}}{2},-\frac{\sqrt{2}}{2}\right) + \cc{L}\{(-1,1)\}
        \end{equation*}
        Su centro es el punto $\left(-\frac{\sqrt{2}}{2},-\frac{\sqrt{2}}{2}\right)$. La longitud de los semiejes es $b=\frac{1}{\sqrt{2}}$, $a=1$. La mitad de la distancia focal es $c=\sqrt{a^2-b^2}=\frac{1}{\sqrt{2}}$. Entonces, los focos son:
        \begin{equation*}
            F_1={(0,\nicefrac{-1}{\sqrt{2}})}_{\cc{R}'}
            \qquad
            F_2={(0,\nicefrac{1}{\sqrt{2}})}_{\cc{R}'}
        \end{equation*}
        
        \item $-2\sqrt{2} + 12 x + 3\sqrt{2}x^2 + 4 y + 2\sqrt{2} x y + 3\sqrt{2}y^2=0$.
    
        Tenemos que la matriz asociada a dicha cónica es:
        \begin{equation*}
            \wh{A} = \left(\begin{array}{c|cc}
                -2\sqrt{2} & 6 & 2  \\ \hline
                6 & 3\sqrt{2} & \sqrt{2} \\
                2 &  \sqrt{2} & 3\sqrt{2}
            \end{array}\right)
            = \left(\begin{array}{c|c}
                a & z \\ \hline
                z^t & A
            \end{array}\right)
        \end{equation*}
        
        Diagonalizamos $A$. Su polinomio característico es:
        \begin{equation*}
            \lambda^2 -6\sqrt{2}\lm +16 = 0 \Longleftrightarrow \left\{\begin{array}{l}
                \lm_1 = 4\sqrt{2}\\
                \lm_2 = 2\sqrt{2}
            \end{array}\right.
        \end{equation*}

        Sus vectores propios asociados son:
        \begin{equation*}
            V_{4\sqrt{2}} = \left\{(x,y)\in \bb{R}^2 \mid \left(\begin{array}{cc}
                -\sqrt{2} & \sqrt{2} \\
                \sqrt{2} & -\sqrt{2}\\ 
            \end{array}\right)\left(\begin{array}{c}
                x\\y
            \end{array}\right)=0 \right\} = \cc{L}\left\{\frac{1}{\sqrt{2}}(1,1)\right\}
        \end{equation*}
        \begin{equation*}
            V_{2\sqrt{2}} = \left\{(x,y)\in \bb{R}^2 \mid \left(\begin{array}{cc}
                \sqrt{2} & \sqrt{2}\\
                \sqrt{2} & \sqrt{2}
            \end{array}\right)\left(\begin{array}{c}
                x\\y
            \end{array}\right)=0 \right\} = \cc{L}\left\{\frac{1}{\sqrt{2}}(-1,1)\right\}
        \end{equation*}

        Además, para que el vector $\wt{z}$ (primera columna) sea entero nulo, buscamos que:
        \begin{equation*}
            Ac + z = 0
            =  \left(\begin{array}{cc}
                3\sqrt{2} & \sqrt{2} \\
                \sqrt{2} & 3\sqrt{2}
            \end{array}\right)\left(\begin{array}{c}
                x\\y
            \end{array}\right) + \left(\begin{array}{c}
                6 \\ 2
            \end{array}\right) = \left(\begin{array}{c}
                0\\0
            \end{array}\right)
        \end{equation*}
        Esto se da si y solo si:
        \begin{equation*}
            \left(\begin{array}{cc}
                3\sqrt{2} & \sqrt{2} \\
                \sqrt{2} & 3\sqrt{2}
            \end{array}\right)\left(\begin{array}{c}
                x\\y
            \end{array}\right) = \left(\begin{array}{c}
                -6 \\ -2
            \end{array}\right) \Longrightarrow c=\left(-\sqrt{2}, 0\right)
        \end{equation*}

        Por tanto, definimos el sistema de referencia $\cc{R}'=\left\{\left(-\sqrt{2}, 0\right), \left\{\dfrac{1}{\sqrt{2}}(1,1), \dfrac{1}{\sqrt{2}}(-1,1)\right\}\right\}$.

        Para calcular el nuevo valor de $\wt{a}$, sabemos que el determinante es invariante mediante semejanzas.
        \begin{equation*}
            |\wh{A}|=-128\sqrt{2}= \wt{a}\cdot 4\sqrt{2}\cdot 2\sqrt{2}\Longrightarrow \wt{a}=-8\sqrt{2}
        \end{equation*}
        
        
        Por tanto, su matriz asociada en $\cc{R}'$ es:
        \begin{equation*}
            \left(\begin{array}{c|cc}
                -8\sqrt{2} & 0 & 0  \\ \hline
                0 & 4\sqrt{2} & 0 \\
                0 & 0 & 2\sqrt{2}
            \end{array}\right)
        \end{equation*}
        Por tanto, su ecuación en dicho sistema de referencia es:
        \begin{equation*}
            4\sqrt{2}\wt{x}^2 +2\sqrt{2}\wt{y}^2 = 8\sqrt{2} \Longleftrightarrow \frac{\wt{x}^2}{2} +\frac{\wt{y}^2}{4}=1
        \end{equation*}

        Por tanto, se trata de una elipse. Sus ejes son:
        \begin{equation*}
            e_1 = \left(-\frac{\sqrt{2}}{2},-\frac{\sqrt{2}}{2}\right) + \cc{L}\{(1,1)\}
            \qquad
            e_2 = \left(-\frac{\sqrt{2}}{2},-\frac{\sqrt{2}}{2}\right) + \cc{L}\{(-1,1)\}
        \end{equation*}
        Su centro es el punto $\left(-\sqrt{2},0\right)$. La longitud de los semiejes es $b=\sqrt{2}$, $a=2$. La mitad de la distancia focal es $c=\sqrt{a^2-b^2}=\sqrt{2}$. Entonces, los focos son:
        \begin{equation*}
            F_1={(0,-\sqrt{2})}_{\cc{R}'}
            \qquad
            F_2={(0,\sqrt{2})}_{\cc{R}'}
        \end{equation*}

        \item $4x^2+y^2+4xy+2x+1=0$.
    
        Tenemos que la matriz asociada a dicha cónica es:
        \begin{equation*}
            \wh{A} = \left(\begin{array}{c|cc}
                1 & 1 & 0  \\ \hline
                1 & 4 & 2 \\
                0 &  2 & 1
            \end{array}\right)
            = \left(\begin{array}{c|c}
                a & z \\ \hline
                z^t & A
            \end{array}\right)
        \end{equation*}
        
        Diagonalizamos $A$. Su polinomio característico es:
        \begin{equation*}
            \lambda^2 -5\lm = 0 \Longleftrightarrow \left\{\begin{array}{l}
                \lm_1 = 5\\
                \lm_2 = 0
            \end{array}\right.
        \end{equation*}

        Por tanto, se trata de una parábola. Sus vectores propios asociados son:
        \begin{equation*}
            V_{5} = \left\{(x,y)\in \bb{R}^2 \mid \left(\begin{array}{cc}
                -1 & 2 \\
                2 & -4\\ 
            \end{array}\right)\left(\begin{array}{c}
                x\\y
            \end{array}\right)=0 \right\} = \cc{L}\left\{\frac{1}{\sqrt{5}}(2,1)\right\}
        \end{equation*}
        \begin{equation*}
            V_{0} = \left\{(x,y)\in \bb{R}^2 \mid \left(\begin{array}{cc}
                4 & 2 \\
                2 & 1\\ 
            \end{array}\right)\left(\begin{array}{c}
                x\\y
            \end{array}\right)=0 \right\} = \cc{L}\left\{\frac{1}{\sqrt{5}}(1,-2)\right\}
        \end{equation*}

        El eje entonces de la parábola es:
        \begin{equation*}
            \begin{split}
                e&=\left\{(x,y)^t\in \bb{R}^2 \mid (0,2,1)\wh{A}(1,x,y)^t=0\right\}=\\
                &= \left\{(x,y)^t\in \bb{R}^2 \mid (0,2,1) \left(\begin{array}{c|cc}
                1 & 1 & 0  \\ \hline
                1 & 4 & 2 \\
                0 &  2 & 1
            \end{array}\right)
             \left(\begin{array}{c}
                1 \\
                x \\
                y
            \end{array}\right)
            =0\right\} =\\
            &= \left\{(x,y)^t\in \bb{R}^2 \mid \left(\begin{array}{ccc}
                2 & 10 & 5
            \end{array}\right)
             \left(\begin{array}{c}
                1 \\
                x \\
                y
            \end{array}\right)
            =0\right\} =\\
            &= \left\{(x,y)^t\in \bb{R}^2 \mid 2+10x+5y=0\right\} =\\
            &= \left(\nicefrac{-6}{5}, 2\right)+\cc{L}\{(-1,2)\}
            \end{split}
        \end{equation*}

        Para calcular el vértice de la parábola, tenemos que resolver el siguiente sistema de ecuaciones:
        \begin{equation*}
            \left\{
            \begin{array}{l}
                2+10x+5y=0 \\
                4x^2+y^2+4xy+2x+1=0
            \end{array}
            \right\}\Longrightarrow v=\left(\frac{-29}{50},\frac{19}{25}\right)
        \end{equation*}

        Por tanto, definimos el sistema de referencia $\cc{R}'=\left\{\left(\dfrac{-29}{50},\dfrac{19}{25}\right), \left\{\dfrac{1}{\sqrt{5}}(2,1), \dfrac{1}{\sqrt{5}}(1,-2)\right\}\right\}$.

        En $\cc{R}'$, su matriz asociada queda:
        \begin{equation*}
            \left(\begin{array}{c|cc}
                0 & 0 & \lm  \\ \hline
                0 & 5 & 0 \\
                \lm & 0 & 0
            \end{array}\right)
        \end{equation*}

        Para calcular el valor de $\lm$, sabemos que el determinante es invariante mediante semejanzas.
        \begin{equation*}
            |\wh{A}|=-1= -5\lm^2 \Longrightarrow \lm = \pm \frac{1}{\sqrt{5}}
        \end{equation*}

        Por tanto, su ecuación en dicho sistema de referencia es:
        \begin{equation*}
            5x^2=-2\lm y \Longleftrightarrow \frac{5}{-\lm}x^2 = 2y
        \end{equation*}
        Como el coeficiente de $x^2$ es positivo, tenemos que $\lm =-\dfrac{1}{\sqrt{5}}$. Si la distancia del vértice al foco es $c$, entonces:
        \begin{equation*}
            \frac{1}{2c} = -\frac{\lm}{5} \Longrightarrow c = -\frac{\lm}{10} = \frac{1}{10\sqrt{5}}
        \end{equation*}

        Por tanto, consideramos los siguientes puntos:
        \begin{gather*}
            P_1 = \left(\dfrac{-29}{50},\dfrac{19}{25}\right) + \frac{1}{10\sqrt{5}}\cdot \frac{1}{\sqrt{5}}(1,-2)
            = \left(\dfrac{-14}{25},\dfrac{18}{25}\right) \\
            P_2 = \left(\dfrac{-29}{50},\dfrac{19}{25}\right) - \frac{1}{10\sqrt{5}}\cdot \frac{1}{\sqrt{5}}(1,-2)
            = \left(\dfrac{-3}{5},\dfrac{4}{5}\right)
        \end{gather*}

        Por tanto, supongamos que la directriz pasa por $P_2$ es decir, $d'=P_2+\cc{L}\{(2,1)\}$. Su ecuación cartesiana es:
        \begin{equation*}
            \left|\begin{array}{cc}
                2 & x+\nicefrac{3}{5} \\
                1 & y-\nicefrac{4}{5}
            \end{array}\right| = 0 = 2y-\frac{8}{5}-x-\frac{3}{5} = 2y-x-\frac{11}{5}
        \end{equation*}
    
        Comprobemos si lo que hemos supuesto como directriz corta a la parábola o no:
        \begin{equation*}
            \left\{
            \begin{array}{l}
                2y-x-\frac{11}{5}=0 \\
                4x^2+y^2+4xy+2x+1=0
            \end{array}
            \right.
        \end{equation*}
    
        Como dicho sistema \emph{ tiene dos soluciones} reales, entonces no era la directriz. Por tanto, la directriz es $d=\left(\dfrac{-14}{25},\dfrac{18}{25}\right)+\cc{L}\{(2,1)\}$; y el foco es $P_2~=~F=~\left(\dfrac{-3}{5},\dfrac{4}{5}\right)$.
    
        \item $-3+2x+x^2-6y-6xy+y^2=0$.
            
        Tenemos que la matriz asociada a dicha cónica es:
        \begin{equation*}
            \wh{A} = \left(\begin{array}{c|cc}
                -3 & 1 & -3  \\ \hline
                1 & 1 & -3 \\
                -3 &  -3 & 1
            \end{array}\right)
            = \left(\begin{array}{c|c}
                a & z \\ \hline
                z^t & A
            \end{array}\right)
        \end{equation*}
        
        Diagonalizamos $A$. Su polinomio característico es:
        \begin{equation*}
            \lambda^2 -2\lm -8 = 0 \Longleftrightarrow \left\{\begin{array}{l}
                    \lm_1 = 4\\
                    \lm_2 = -2
                \end{array}\right.
        \end{equation*}
    
        Sus vectores propios asociados son:
        \begin{equation*}
            V_{4} = \left\{(x,y)\in \bb{R}^2 \mid \left(\begin{array}{cc}
                -3 & -3 \\
                -3 & -3
            \end{array}\right)\left(\begin{array}{c}
                x\\y
            \end{array}\right)=0 \right\} = \cc{L}\left\{\frac{1}{\sqrt{2}}(1,-1)\right\}
        \end{equation*}
        \begin{equation*}
            V_{-2} = \left\{(x,y)\in \bb{R}^2 \mid \left(\begin{array}{cc}
                3 & -3 \\
                -3 & 3
            \end{array}\right)\left(\begin{array}{c}
                x\\y
            \end{array}\right)=0 \right\} = \cc{L}\left\{\frac{1}{\sqrt{2}}(1,1)\right\}
        \end{equation*}
    
        Además, para que el vector $\wt{z}$ (primera columna) sea entero nulo, buscamos que:
        \begin{equation*}
            Ac + z = 0
            =  \left(\begin{array}{cc}
                1 & -3 \\
                -3 & 1
            \end{array}\right)\left(\begin{array}{c}
                x\\y
            \end{array}\right) + \left(\begin{array}{c}
                1 \\ -3
            \end{array}\right) = \left(\begin{array}{c}
                0\\0
            \end{array}\right)
        \end{equation*}
        Esto se da si y solo si:
        \begin{equation*}
            \left(\begin{array}{cc}
                1 & -3 \\
                -3 & 1
            \end{array}\right)\left(\begin{array}{c}
                x\\y
            \end{array}\right) = \left(\begin{array}{c}
                -1 \\ 3
            \end{array}\right) \Longrightarrow c=(-1,0)
        \end{equation*}
    
        Por tanto, definimos el sistema de referencia $\cc{R}'=\left\{(-1, 0), \left\{\dfrac{1}{\sqrt{2}}(1,-1), \dfrac{1}{\sqrt{2}}(1,1)\right\}\right\}$.
    
        Para calcular el nuevo valor de $\wt{a}$, sabemos que el determinante es invariante mediante semejanzas.
        \begin{equation*}
            |\wh{A}|=32= \wt{a}\cdot 4\cdot -2\Longrightarrow \wt{a}=-4
        \end{equation*}
        
        
        Por tanto, la matriz asociada a la hipercuádrica en $\cc{R}'$ es:
        \begin{equation*}
            \left(\begin{array}{c|cc}
                -4 & 0 & 0  \\ \hline
                0 & 4 & 0 \\
                0 & 0 & -2
            \end{array}\right)
        \end{equation*}
        Por tanto, su ecuación en dicho sistema de referencia es:
        \begin{equation*}
            4\wt{x}^2 -2\wt{y}^2 = 4 \Longleftrightarrow \wt{x}^2 - \frac{1}{2}\wt{y}^2=1
        \end{equation*}
    
        Por tanto, se trata de una hipérbola que corta al eje $X$ de ese nuevo sistema de referencia. Sus ejes son:
        \begin{equation*}
            e_1 = (-1,0) + \cc{L}\{(1,-1)\}
            \qquad
            e_2 = (-1,0) + \cc{L}\{(1,1)\}
        \end{equation*}
        Su centro es el punto $(-1,0)$. La longitud de los semiejes es $a=\sqrt{2}$, $b=1$. La mitad de la distancia focal es $c=\sqrt{3}$. Entonces, los focos son:
        \begin{equation*}
            F_1={(\sqrt{3},0)}_{\cc{R}'}
            \qquad
            F_2={(-\sqrt{3},0)}_{\cc{R}'}
        \end{equation*}
    
        Las ecuaciones de las asíntotas en $\cc{R}'$ son:
        \begin{equation*}
            \frac{x}{b}+\frac{y}{a}=0
            \Longrightarrow y = -{\sqrt{2}}x
            \hspace{2cm}
            \frac{x}{b}-\frac{y}{a}=0
            \Longrightarrow y = {\sqrt{2}}x
        \end{equation*}
    \end{enumerate}
\end{ejercicio}


\begin{ejercicio}
    Clasifica euclídeamente las siguientes cuádricas de $\bb{R}^3$ y obtén, en cada caso, un sistema de referencia euclídeo en el cual su expresión sea reducida:
    \begin{enumerate}
        \item $3x^2 + 4y^2 + 2z^2 - 4xy + 4xz - 6x - 6y + 3 = 0.$

        \item $3x^2 + y^2 + 3z^2 - 2xz + 2\sqrt{2}x + 2y + 2\sqrt{2}z + 2 = 0.$
    \end{enumerate}
\end{ejercicio}




\begin{ejercicio}
    Clasifica las siguientes cónicas afines de $\bb{R}^2$ y determina un sistema de referencia donde su expresión sea reducida:
    \begin{enumerate}
        \item $4x^2 - 2xy + 2y^2 + x - 3y - 3 = 0.$        
        \item $2x^2 - 4xy + 2y^2 + 12x - 18y + 11 = 0.$
        \item $2x^2 + 12xy + 18y^2 + 4x - 6y + 10 = 0$
        \item $x^2 - 4xy + 4y^2 + 4x - 9y + 2 = 0.$
        \item $x^2 - 4xy + 4y^2 + 2x - 4y + 1 = 0.$
        \item $3x^2 + 6xy + 3y^2 - 12x - 11y + 11 = 0.$
    \end{enumerate}
\end{ejercicio}


\begin{ejercicio}
    Clasifica las siguientes cónicas afines de $\bb{R}^3$ y determina un sistema de referencia donde su expresión sea reducida:
    \begin{enumerate}
        \item $3x^2 + 4y^2 + 21z^2 + 6xy + 12xz + 18yz - 12x - 14y - 29z + 14 = 0.$

        Empleamos el método de completar cuadrados:
        \begin{equation*}
            \begin{split}
                0 &= 3x^2 + 4y^2 + 21z^2 + 6xy + 12xz + 18yz - 12x - 14y - 29z + 14 =\\
                &= 3x^2+6x(y+2z-2) + 4y^2 + 21z^2 + 18yz - 14y - 29z + 14 =\\
                &= 3[x^2+2x(y+2z-2)] + 4y^2 + 21z^2 + 18yz - 14y - 29z + 14 =\\
                &= 3(x + y+2z-2)^2-3(y+2z-2)^2 + 4y^2 + 21z^2 + 18yz - 14y - 29z + 14 =\\
                &= 3(x + y+2z-2)^2-3(y^2+4z^2+4+4yz-4y-8z) +\\&\hspace{0.5\linewidth}+4y^2 + 21z^2 + 18yz - 14y - 29z + 14 =\\
                &= 3(x + y+2z-2)^2 +y^2 + 9z^2 + 6yz - 2y - 5z + 2 =\\
                &= 3(x + y+2z-2)^2 +y^2 + 2y(3z-1) + 9z^2- 5z + 2 =\\
                &= 3(x + y+2z-2)^2 + [y+(3z-1)]^2 -(3z-1)^2 + 9z^2- 5z + 2 =\\
                &= 3(x + y+2z-2)^2 + [y+3z-1]^2 +z +1 =\\
                &= [\sqrt{3}(x + y+2z-2)]^2 + [y+3z-1]^2 -2\left[-\frac{z+1}{2}\right]
            \end{split}
        \end{equation*}

        Aplicamos entonces el cambio de sistema de referencia a $\cc{R}'$, de forma que:
        \begin{equation*}
            M(Id, \cc{R}_0, \cc{R}') = \left(\begin{array}{c|ccc}
                1 & 0 & 0 & 0 \\ \hline
                -2\sqrt{3} & \sqrt{3} & \sqrt{3} & 2\sqrt{3} \\
                -1 & 0 & 1 & 3 \\
                \nicefrac{-1}{2} & 0 & 0 & \nicefrac{-1}{2}
            \end{array}\right)
        \end{equation*}

        Para hallar el sistema de referencia $\cc{R}'$, tenemos que:
        \begin{equation*}
            M(Id, \cc{R}', \cc{R}_0) = \left(\begin{array}{c|ccc}
                1 & 0 & 0 & 0 \\ \hline
                -2\sqrt{3} & \sqrt{3} & \sqrt{3} & 2\sqrt{3} \\
                -1 & 0 & 1 & 3 \\
                \nicefrac{-1}{2} & 0 & 0 & \nicefrac{-1}{2}
            \end{array}\right)^{-1}
            = \left(\begin{array}{c|ccc}
                1 & 0 & 0 & 0 \\ \hline
                0 & \nicefrac{\sqrt{3}}{3} & -1 & -2 \\
                4 & 0 & 1 & 6 \\
                -1 & 0 & 0 & -2
            \end{array}\right)
        \end{equation*}

        Es decir, si $\cc{R}_0 = \{O, \{e_1,e_2,e_3\}\}$, entonces:
        \begin{equation*}
            \cc{R}' = \left\{(0, 4, -1), \left\{\left(\nicefrac{\sqrt{3}}{3}, 0, 0\right), \left(-1, 1, 0\right), \left(-2,6,-2\right)\right\}\right\}
        \end{equation*}

        En dicho sistema, si las coordenadas de un punto son ${(\wt{x}, \wt{y}, \wt{z})}_{\cc{R}'}$, tenemos que:
        \begin{equation*}
            \wt{x}^2 + \wt{y}^2 = 2\wt{z}
        \end{equation*}

        Por tanto, se trata de un paraboloide elíptico.
        
        \item $3x^2 + 2y^2 + 7z^2 + 6xy + 12xz + 6yz - 12x - 10y - 10z + 12 = 0.$
        \item $x^2 + 2y^2 - 2z^2 - 2xy + 2xz - 4yz + 4y - 12z = 0.$
        \item $x^2 + 4y^2 + z^2 - 4xy + 2xz - 4yz + 2x - 3y + 4 = 0.$

        Empleamos el método de completar cuadrados:
        \begin{equation*}
            \begin{split}
                0 &= x^2 + 4y^2 + z^2 - 4xy + 2xz - 4yz + 2x - 3y + 4 =\\
                &= x^2 + 2x(-2y+z+1)+ 4y^2 + z^2 - 4yz - 3y + 4 =\\
                &= (x-2y+z+1)^2-(-2y+z+1)^2+ 4y^2 + z^2 - 4yz - 3y + 4 =\\
                &= (x-2y+z+1)^2-(4y^2+z^2+1-4yz-4y+2z)+ 4y^2 + z^2 - 4yz - 3y + 4 =\\
                &= (x-2y+z+1)^2 +y + 3-2z\\
                &= (x-2y+z+1)^2 -2\cdot \frac{-y - 3+2z}{2}\\
            \end{split}
        \end{equation*}

        Aplicamos entonces el cambio de sistema de referencia a $\cc{R}'$, de forma que:
        \begin{equation*}
            M(Id, \cc{R}_0, \cc{R}') = \left(\begin{array}{c|ccc}
                1 & 0 & 0 & 0 \\ \hline
                1 & 1 & -2 & 1 \\
                0 & 0 & 1 & 0 \\
                \nicefrac{-3}{2} & 0 & \nicefrac{-1}{2} & 1
            \end{array}\right)
        \end{equation*}
        Notemos que hemos usado $\wt{y}=y$, aunque bastaría con poner valores de $\wt{y}$ de forma que la matriz anterior fuese regular, para que fuese un cambio de sistema de referencia.

        Para hallar el sistema de referencia $\cc{R}'$, tenemos que:
        \begin{equation*}
            M(Id, \cc{R}', \cc{R}_0) = \left(\begin{array}{c|ccc}
                1 & 0 & 0 & 0 \\ \hline
                1 & 1 & -2 & 1 \\
                0 & 0 & 1 & 0 \\
                \nicefrac{-3}{2} & 0 & \nicefrac{-1}{2} & 1
            \end{array}\right)^{-1}
            = \left(\begin{array}{c|ccc}
                1 & 0 & 0 & 0 \\ \hline
                \nicefrac{-5}{2} & 1 & \nicefrac{3}{2} & -1 \\
                0 & 0 & 1 & 0 \\
                \nicefrac{3}{2} & 0 & \nicefrac{1}{2} & 1
            \end{array}\right)
        \end{equation*}
        

        Es decir, si $\cc{R}_0 = \{O, \{e_1,e_2,e_3\}\}$, entonces:
        \begin{equation*}
            \cc{R}' = \left\{\left(-\frac{5}{2}, 0, \frac{3}{2}\right), \left\{\left(1,0,0\right), \left(\frac{3}{2}, 1, \frac{1}{2}\right), \left(-1,0,1\right)\right\}\right\}
        \end{equation*}

        En dicho sistema, si las coordenadas de un punto son ${(\wt{x}, \wt{y}, \wt{z})}_{\cc{R}'}$, tenemos que:
        \begin{equation*}
            \wt{x}^2=2 \wt{z}^2
        \end{equation*}

        Por tanto, se trata de un cilindro hiperbólico.

        
        \item $x^2 + 11y^2 + 9z^2 - 6xy + 4xz - 8yz + 2x - 2y + 2z + 3 = 0.$

        Empleamos el método de completar cuadrados:
        \begin{equation*}
            \begin{split}
                0 &= x^2 + 11y^2 + 9z^2 - 6xy + 4xz - 8yz + 2x - 2y + 2z + 3 =\\
                &= x^2 + 2x(-3y+2z+1) + 11y^2 + 9z^2 - 8yz - 2y + 2z + 3 =\\
                &= [x+(1-3y+2z)]^2 -(1-3y+2z)^2 + 11y^2 + 9z^2 - 8yz - 2y + 2z + 3 =\\
                &= (x+1-3y+2z)^2 -(1+9y^2+4z^2-6y+4z-12yz) + 11y^2 + 9z^2-\\&\hspace{0.6\linewidth} - 8yz - 2y + 2z + 3=\\
                &= (x+1-3y+2z)^2 + 2y^2 + 5z^2 +4yz +4y - 2z + 2=\\
                &= (x+1-3y+2z)^2 + 2y^2 +4y(z+1)+ 5z^2 - 2z + 2=\\
                &= (x+1-3y+2z)^2 + 2[y+(z+1)]^2 -2(z+1)^2 + 5z^2 - 2z + 2=\\
                &= (x+1-3y+2z)^2 + 2(y+z+1)^2 + 3z^2 - 6z =\\
                &= (x+1-3y+2z)^2 + 2(y+z+1)^2 + 3(z^2-2z+1)-3=\\
                &= (x+1-3y+2z)^2 + 2(y+z+1)^2 + 3(z-1)^2-3=\\
                &= \frac{1}{3}(x+1-3y+2z)^2 + \frac{2}{3}(y+z+1)^2 + \frac{3}{3}(z-1)^2-1
            \end{split}
        \end{equation*}

        Aplicamos entonces el cambio de sistema de referencia a $\cc{R}'$, de forma que:
        \begin{equation*}
            M(Id, \cc{R}_0, \cc{R}') = \frac{1}{\sqrt{3}}\left(\begin{array}{c|ccc}
                \sqrt{3} & 0 & 0 & 0 \\ \hline
                1 & 1 & -3 & 2 \\
                \sqrt{2} & 0 & \sqrt{2} & \sqrt{2} \\
                -\sqrt{3} & 0 & 0 & \sqrt{3}
            \end{array}\right)
        \end{equation*}

        Para hallar el sistema de referencia $\cc{R}'$, tenemos que:
        \begin{equation*}
            M(Id, \cc{R}', \cc{R}_0) = \left[\frac{1}{\sqrt{3}}\left(\begin{array}{c|ccc}
                \sqrt{3} & 0 & 0 & 0 \\ \hline
                1 & 1 & -3 & 2 \\
                \sqrt{2} & 0 & \sqrt{2} & \sqrt{2} \\
                -\sqrt{3} & 0 & 0 & \sqrt{3}
            \end{array}\right)\right]^{-1}
            = \left(\begin{array}{c|ccc}
                1 & 0 & 0 & 0 \\ \hline
                -9 & \sqrt{3} & \nicefrac{3\sqrt{6}}{2} & -5 \\
                -2 & 0 & \nicefrac{\sqrt{6}}{2} & -1 \\
                1 & 0 & 0 & 1
            \end{array}\right)
        \end{equation*}

        Es decir, si $\cc{R}_0 = \{O, \{e_1,e_2,e_3\}\}$, entonces:
        \begin{equation*}
            \cc{R}' = \left\{(-9,-2,1), \left\{\left(\sqrt{3},0,0\right), \left(\frac{3\sqrt{6}}{2}, \frac{\sqrt{6}}{2}, 0\right), \left(-5,-1,1\right)\right\}\right\}
        \end{equation*}

        En dicho sistema, si las coordenadas de un punto son ${(\wt{x}, \wt{y}, \wt{z})}_{\cc{R}'}$, tenemos que:
        \begin{equation*}
            \wt{x}^2 + \wt{y}^2 + \wt{z}^2 = 1
        \end{equation*}

        Por tanto, se trata de un elipsoide.
        \item $xy + xz + yz - 1 = 0.$

        Empleamos el método de completar cuadrados:
        \begin{equation*}
            \begin{split}
                0 &= xy + xz + yz - 1=\\
                &= x(y+z) +yz -1 =\\
                &= x(y+z) + \frac{1}{4}(y+z)^2 - \frac{1}{4}(y-z)^2 -1=\\
                &= (y+z)\left[x + \frac{1}{4}(y+z)\right] - \frac{1}{4}(y-z)^2-1 =\\
                &= \frac{1}{4}\left(y+z+x+\frac{1}{4}(y+z)\right)^2 - \frac{1}{4}\left(y+z-x-\frac{1}{4}(y+z)\right)^2 - \frac{1}{4}(y-z)^2-1 =\\
                &= \frac{1}{4}\left(x+\frac{5}{4}(y+z)\right)^2 - \frac{1}{4}\left(-x+\frac{3}{4}(y+z)\right)^2 - \frac{1}{4}(y-z)^2-1 =\\
                &= \frac{1}{4}\left(-x+\frac{3}{4}(y+z)\right)^2 + \frac{1}{4}(y-z)^2 -\frac{1}{4}\left(x+\frac{5}{4}(y+z)\right)^2+1
            \end{split}
        \end{equation*}

        Aplicamos entonces el cambio de sistema de referencia a $\cc{R}'$, de forma que:
        \begin{equation*}
            M(Id, \cc{R}_0, \cc{R}') = \frac{1}{2}\left(\begin{array}{c|ccc}
                2 & 0 & 0 & 0 \\ \hline
                0 & -1 & \nicefrac{3}{4} & \nicefrac{3}{4} \\
                0 & 0 & 1 & -1 \\
                0 & 1 & \nicefrac{5}{4} & \nicefrac{5}{4}
            \end{array}\right)
        \end{equation*}

        Para hallar el sistema de referencia $\cc{R}'$, tenemos que:
        \begin{equation*}
            M(Id, \cc{R}', \cc{R}_0) = \left[\frac{1}{2}\left(\begin{array}{c|ccc}
                2 & 0 & 0 & 0 \\ \hline
                0 & -1 & \nicefrac{3}{4} & \nicefrac{3}{4} \\
                0 & 0 & 1 & -1 \\
                0 & 1 & \nicefrac{5}{4} & \nicefrac{5}{4}
            \end{array}\right)\right]^{-1}
            = \left(\begin{array}{c|ccc}
                1 & 0 & 0 & 0 \\ \hline
                0 & \nicefrac{-5}{4} & 0 & \nicefrac{3}{4} \\
                0 & \nicefrac{1}{2} & 1 & \nicefrac{1}{2} \\
                0 & \nicefrac{1}{2} & -1 & \nicefrac{1}{2}
            \end{array}\right)
        \end{equation*}
        

        Es decir, si $\cc{R}_0 = \{O, \{e_1,e_2,e_3\}\}$, entonces:
        \begin{equation*}
            \cc{R}' = \left\{O, \left\{\left(-\frac{5}{4}, \frac{1}{2}, \frac{1}{2}\right), \left(0,1,-1\right), \left(\frac{3}{4}, \frac{1}{2}, \frac{1}{2}\right)\right\}\right\}
        \end{equation*}

        En dicho sistema, si las coordenadas de un punto son ${(\wt{x}, \wt{y}, \wt{z})}_{\cc{R}'}$, tenemos que:
        \begin{equation*}
            \wt{x}^2 + \wt{y}^2 - \wt{z}^2 = -1
        \end{equation*}

        Por tanto, se trata de un hiperboloide de dos hojas. Vemos que corta a la recta $\wt{x}=\wt{y}=0$ (eje $\wt{z}$).
    \end{enumerate}
\end{ejercicio}


\begin{ejercicio}
    Clasifica, según los valores del parámetro $a \in \bb{R}$, la siguientes cónicas de $\bb{R}^2$:
    \begin{enumerate}
        \item $x^2 + ay2 + 2xy - 2x + a = 0 .$
        \item $x^2 + axy + y^2 + 1 = 0 .$
    \end{enumerate}
\end{ejercicio}


\begin{ejercicio}
    En $\bb{R}^3$ consideramos el punto $F = (0, 0, 1)$ y el plano afín $S$ de ecuación $x-z = 0$. Definimos el conjunto:
    \begin{equation*}
        C = \{p \in \bb{R}^3 \mid d(p, F) = d(p, S)\}.
    \end{equation*}
    Demostrar que $C$ es una cuádrica y clasificarla.
\end{ejercicio}
\section{Teorema de Cambio de Variable}

\subsection{Repaso teórico}
Repasamos ahora los conceptos teóricos necesarios para realizar cambios de variable en varias variables. A diferencia de los cambios de variable en integrales simples, la oportunidad de aplicar estos cambios son muy puntuales, en situaciones donde tengamos claras simetrías.

\subsubsection{Teoremas}
\begin{definicion}[Difeomorfismo]
    Dados dos abiertos $\Omega, G \subseteq \mathbb{R}^N$, una función ${\phi:\Omega\to G}$ es un difeomorfismo de clase $C^1$ cuando $\phi$ es biyectiva y, tanto $\phi$ como $\phi^{-1}$ son de clase $C^1$.
\end{definicion}

\begin{prop}
    Si $\phi:\Omega\to G$ es un difeomorfismo, entonces preserva los conjuntos medibles.
\end{prop}

\begin{teo}[Teorema de cambio de variable para funciones medibles positivas]\ \\
    Sea $\phi:\Omega\to G$ un difeomorfismo de clase $C^1$ entre dos abiertos de $\mathbb{R}^N$. Dado un conjunto medible $E\subseteq \Omega$ y una función medible positiva $f:\phi(E)\to[0,\infty]$, consideramos la función $g:E\to [0,\infty]$ definida por $g(t)=f(\phi(t))|detJ\phi(t)|$ para todo $t\in E$. Entonces $g$ es medible y su integral sobre $E$ coincide con la de $f$ sobre $\phi(E)$, es decir:
    \begin{equation*}
        \int_{\phi(E)} f(x)dx = \int_E f(\phi(t))|detJ\phi(t)|dt
    \end{equation*}
\end{teo}

\begin{teo}[Teorema de cambio de variable]\ \\
    Sea $\phi:\Omega\to G$ un difeomorfismo de clase $C^1$ entre dos abiertos de $\mathbb{R}^N$. Dado un conjunto medible $E\subseteq \Omega$ y una función medible $f:\phi(E)\to\mathbb{R}$, se considera la función $g:E\to \mathbb{R}$ definida por $g(t)=f(\phi(t))|detJ\phi(t)|$ para todo $t\in E$. Entonces $f$ es integrable en $\phi(E)$ si, y sólo si, $g$ es integrable en $E$, en cuyo caso se tiene:
    \begin{equation*}
        \int_{\phi(E)} f(x)dx = \int_E f(\phi(t))|detJ\phi(t)|dt
    \end{equation*}
\end{teo}

Normalmente tendremos un conjunto medible $A\subseteq \mathbb{R}^N$ y una función medible $f:A\to \mathbb{R}$, y queremos usar el cambio de variable $\phi:\Omega\to G$, por lo que si no tenemos que $A\subseteq G$ podemos pensar que no podemos aplicarlo. Sin embargo, basta con que $\lm(A\setminus G)= 0$.

Los cambios de variable que veremos a continuación se basan en que ${\lm(\mathbb{R}^N\setminus G)= 0}$, luego podremos aplicarlos sea cual sea el conjunto $A$.

\subsubsection{Cambios usuales}
Normalmente, sólo aplicaremos los siguientes cambios de variable.

\begin{teo}[Cambio de variable a coordenadas polares]\ \\
    Dado un conjunto medible $A\subseteq \mathbb{R}^2$ y una función medible $f:A\to \mathbb{R}$, consideramos el conjunto medible:
    \begin{equation*}
        E = \{(\rho, \theta)\in \mathbb{R}^+ \times \left]-\pi,\pi\right[ \mid (\rho\cos\theta, \rho\sen\theta)\in A\}
    \end{equation*}
    y la función $g:E\to \mathbb{R}$ dada por $g(\rho, \theta)=\rho f(\rho\cos\theta, \rho\sen\theta)$, para todo $(\rho, \theta)\in E$. Entonces $f\in \cc{L}_1(A)$ si, y sólo si, $g\in \cc{L}_1(E)$, en cuyo caso se tiene:
    \begin{equation*}
        \int_A f(x,y)d(x,y) = \int_E \rho f(\rho\cos\theta, \rho\sen\theta)d(\rho,\theta)
    \end{equation*}
\end{teo}

Donde hemos usado el cambio de variable dado por la función\newline ${\phi:\Omega=\mathbb{R}^+ \times \left]-\pi,\pi\right[ \to \mathbb{R}^2}$ definida por:
\begin{equation*}
    \phi(\rho,\theta) = (\rho\cos\theta, \rho\sen\theta)\qquad \forall (\rho,\theta)\in \Omega
\end{equation*}
de esta forma, se tiene que:
\begin{align*}
    \rho &= \sqrt{x^2+y^2} \\
    \theta &= 2\arctg \dfrac{y}{\sqrt{x^2+y^2}+x}
\end{align*}
y que $detJ\phi(\rho,\theta)=\rho >0\qquad \forall (\rho, \theta)\in \Omega$.\\

Se trata de un cambio de variable útil cuando el conjunto sobre el que se integra presenta una simetría central, o que se pueda simplificar el integrando sustituyendo varios de sus términos por $\rho$.

\begin{teo}[Cambio de variable a coordenadas cilíndricas]\ \\
    Dado un conjunto medible $A\subseteq \mathbb{R}^3$ y una función medible $f:A\to \mathbb{R}$, consideramos el conjunto medible:
    \begin{equation*}
        E = \{(\rho, \theta,z)\in \mathbb{R}^+ \times \left]-\pi,\pi\right[\times\mathbb{R} \mid (\rho\cos\theta, \rho\sen\theta,z)\in A\}
    \end{equation*}
    y la función $g:E\to \mathbb{R}$ dada por $g(\rho, \theta,z)=\rho f(\rho\cos\theta, \rho\sen\theta,z)$, para todo $(\rho, \theta,z)\in E$. Entonces $f\in \cc{L}_1(A)$ si, y sólo si, $g\in \cc{L}_1(E)$, en cuyo caso se tiene:
    \begin{equation*}
        \int_A f(x,y,z)d(x,y,z) = \int_E \rho f(\rho\cos\theta, \rho\sen\theta,z)d(\rho,\theta,z)
    \end{equation*}
\end{teo}

Donde hemos usado el cambio de variable dado por la función\newline ${\phi:\Omega=\mathbb{R}^+ \times \left]-\pi,\pi\right[\times\mathbb{R} \to \mathbb{R}^3}$ definida por:
\begin{equation*}
    \phi(\rho,\theta,z) = (\rho\cos\theta, \rho\sen\theta,z)\qquad \forall (\rho,\theta,z)\in \Omega
\end{equation*}
de esta forma, se tiene que:
\begin{align*}
    \rho &= \sqrt{x^2+y^2} \\
    \theta &= 2\arctg \dfrac{y}{\sqrt{x^2+y^2}+x}
\end{align*}
y que $detJ\phi(\rho,\theta,z)=\rho >0\qquad \forall (\rho, \theta,z)\in \Omega$.\\

Se trata de un cambio de variable útil cuando el conjunto sobre el que se integra presenta una simetría axial, o que se pueda simplificar el integrando sustituyendo varios de sus términos por $\rho$.

\begin{teo}[Cambio de variable a coordenadas esféricas]\ \\
    Dado un conjunto medible $A\subseteq \mathbb{R}^3$ y una función medible $f:A\to \mathbb{R}$, consideramos el conjunto medible:
    \begin{equation*}
        E = \{(r, \theta,\varphi)\in \mathbb{R}^+ \times \left]-\pi,\pi\right[\times\left]-\nicefrac{\pi}{2},\nicefrac{\pi}{2}\right[ \mid (r\cos\varphi\cos\theta,r\cos\varphi\sen\theta,r\sen\varphi )\in A\}
    \end{equation*}
    y la función $g:E\to \mathbb{R}$ dada por $g(r, \theta,\varphi)=r^2\cos\varphi f(r\cos\varphi\cos\theta, r\cos\varphi\sen\theta,r\sen\varphi)$, para todo $(r, \theta,\varphi)\in E$. Entonces $f\in \cc{L}_1(A)$ si, y sólo si, $g\in \cc{L}_1(E)$, en cuyo caso se tiene:
    \begin{equation*}
        \int_A f(x,y,z)d(x,y,z) = \int_E r^2\cos\varphi f(r\cos\varphi\cos\theta, r\cos\varphi\sen\theta,r\sen\varphi)d(r,\theta,\varphi)
    \end{equation*}
\end{teo}

Donde hemos usado el cambio de variable dado por la función\newline ${\phi:\Omega=\mathbb{R}^+ \times \left]-\pi,\pi\right[\times\left]-\nicefrac{\pi}{2},\nicefrac{\pi}{2}\right[ \to \mathbb{R}^3}$ definida por:
\begin{equation*}
    \phi(r,\theta,\varphi) = (r\cos\varphi\cos\theta, r\cos\varphi\sen\theta,\sen\varphi)\qquad \forall (r,\theta,\varphi)\in \Omega
\end{equation*}
de esta forma, se tiene que:
\begin{align*}
    r &= \sqrt{x^2+y^2+z^2} \\
    \theta &= 2\arctg \dfrac{y}{\sqrt{x^2+y^2}+x} \\
    \varphi &= \arcsen \dfrac{z}{\sqrt{x^2+y^2+z^2}}
\end{align*}
y que $detJ\phi(r,\theta,\varphi)=r^2\cos\varphi >0\qquad \forall (r, \theta,\varphi)\in \Omega$.\\

Se trata de un cambio de variable útil cuando el conjunto sobre el que se integra presenta una simetría central, o que se pueda simplificar el integrando sustituyendo varios de sus términos por $r$.

\subsection{Ejercicios}

\begin{ejercicio}
    En cada uno de los siguientes casos, probar que la función \( f \) es integrable en el conjunto \( A \) y calcular su integral:

    \begin{enumerate}
        \item \( A = \left\{ (x, y) \in \mathbb{R}^2 \mid x^2 + y^2 > 1 ,~y > 0 \right\} \)
        \[ f(x, y) = \frac{x + y}{(x^2 + y^2)^\alpha} \quad \forall (x, y) \in A \quad (\alpha \in \mathbb{R},~\alpha > \nicefrac{3}{2}) \]

        Aplicaremos el cambio de variable a coordenadas polares. Para ello, en primer
        lugar obtenemos el conjunto $E$ siguiente:
        \begin{align*}
            E &= \left\{ (\rho, \theta) \in \bb{R}^+\times \left]-\pi, \pi\right[ ~\mid (\rho\cos\theta, \rho\sen\theta) \in A \right\} \\
            &= \left\{ (\rho, \theta) \in \bb{R}^+\times \left]-\pi, \pi\right[ ~\mid \rho > 1 ,~\sen\theta > 0 \right\} \\
            &= \left] 1, +\infty \right[ \times \left] 0, \pi \right[
        \end{align*}

        Definimos por tanto la función $g: E \to \bb{R}^2$ como:
        \begin{equation*}
            g(\rho, \theta) = \rho f(\rho\cos\theta, \rho\sen\theta) = \frac{\rho^2(\cos\theta + \sen\theta)}{\rho^{2\alpha}}
        \end{equation*}

        Por el Teorema de Cambio de Variable, tenemos que $f\in \cc{L}_1(A)$ si y solo si $g\in \cc{L}_1(E)$, y en tal caso:
        \begin{equation*}
            \int_A f = \int_E g
        \end{equation*}

        Veamos si $g\in \cc{L}_1(E)$. Por el Teorema de Tonelli, tenemos que:
        \begin{align*}
            \int_E |g(\rho, \theta)|~d(\rho, \theta)
            &\leq 2\int_E \rho^{2-2\alpha}~d(\rho, \theta)
            = 2\int_0^\pi \left( \int_1^{+\infty} \rho^{2-2\alpha}~d\rho \right)~d\theta
            =\\&= 2\pi \int_1^{+\infty} \rho^{2-2\alpha}~d\rho
            = 2\pi \left[ \frac{\rho^{3-2\alpha}}{3-2\alpha} \right]_1^{+\infty}
            \AstIg \frac{2\pi}{2\alpha-3} < \infty
        \end{align*}
        donde, en $(\ast)$, hemos usado que $\alpha > \nicefrac{3}{2}$, por lo que $3-2\alpha < 0$.
        Por tanto, $g\in \cc{L}_1(E)$, y por el Teorema de Cambio de Variable, tenemos que $f\in \cc{L}_1(A)$ con:
        \begin{align*}
            \int_A f &= \int_E g = \int_0^\pi \left( \int_1^{+\infty} \frac{\rho^2(\cos\theta + \sen\theta)}{\rho^{2\alpha}}~d\rho \right)~d\theta =\\
            &= \int_0^\pi \left( \cos\theta + \sen\theta \right) \left( \int_1^{+\infty} \rho^{2-2\alpha}~d\rho \right)~d\theta =\\
            &= \int_0^\pi \left( \cos\theta + \sen\theta \right) \left[ \frac{\rho^{3-2\alpha}}{3-2\alpha} \right]_1^{+\infty}~d\theta =\\
            &= \frac{1}{2\alpha-3} \int_0^\pi \left( \cos\theta + \sen\theta \right) ~d\theta =\\
            &= \frac{1}{2\alpha-3} \left[ \sen\theta - \cos\theta \right]_0^\pi = \frac{2}{2\alpha-3}
        \end{align*}

        \item \( A = \left\{ (x, y, z) \in \mathbb{R}^3 \mid 0 < x^2 + y^2 < 1 ,~z > 1 \right\} \)
        \[ f(x, y, z) = z^\alpha (x^2 + y^2)^\beta \quad \forall (x, y, z) \in A \quad (\alpha, \beta \in \mathbb{R},~\alpha < -1 < \beta) \]

        Aplicaremos el cambio de variable a coordenadas cilíndricas. Para ello, en primer
        obtener el conjunto $E$ siguiente:
        \begin{align*}
            E &= \left\{ (\rho, \theta, z) \in \bb{R}^+\times \left]-\pi, \pi\right[ \times \bb{R} ~\mid (\rho\cos\theta, \rho\sen\theta, z) \in A \right\} \\
            &= \left\{ (\rho, \theta, z) \in \bb{R}^+\times \left]-\pi, \pi\right[ \times \bb{R} ~\mid 0 < \rho < 1 ,~z > 1 \right\} \\
            &= \left] 0, 1 \right[ \times \left] -\pi, \pi \right[ \times \left] 1, +\infty \right[
        \end{align*}

        Definimos por tanto la función $g: E \to \bb{R}^3$ como:
        \begin{equation*}
            g(\rho, \theta, z) = \rho f(\rho\cos\theta, \rho\sen\theta, z) = z^\alpha \rho^{2\beta+1}
        \end{equation*}

        Por el Teorema de Cambio de Variable para funciones medibles positivas, junto con el Teorema de Tonelli, tenemos que:
        \begin{align*}
            \int_A f &= \int_E g = \int_{-\pi}^\pi \left( \int_0^1 \left( \int_1^{+\infty} z^\alpha \rho^{2\beta+1}~dz \right)~d\rho \right)~d\theta =\\
            &= \left(\int_{-\pi}^\pi ~d\theta \right) \left( \int_0^1 \rho^{2\beta+1}~d\rho \right) \left( \int_1^{+\infty} z^\alpha ~dz \right) =\\
            &= 2\pi \left[ \frac{\rho^{2\beta+2}}{2\beta+2} \right]_0^1 \left[ \frac{z^{\alpha+1}}{\alpha+1} \right]_1^{+\infty} =\\
            &= -\frac{2\pi}{2\beta+2}\cdot \frac{1}{\alpha+1}
            = -\frac{\pi}{(\beta+1)(\alpha+1)}
        \end{align*}

        Por tanto, como la integral es finita, $f\in \cc{L}_1(A)$ y su valor es el calculado.

        \item \( A = \left\{ (x, y, z) \in (\mathbb{R}^+)^3 \mid x^2 + y^2 + z^2 > 1 \right\} \)
        \[ f(x, y, z) = \frac{x y z}{(x^2 + y^2 + z^2)^4} \quad \forall (x, y, z) \in A \]

        Aplicaremos el cambio de variable a coordenadas esféricas. Para ello, en primer
        lugar obtenemos el conjunto $E$ siguiente:
        \begin{align*}
            E &= \left\{ (r, \theta, \varphi) \in \bb{R}^+\times \left]-\pi, \pi\right[ \times \left] \nicefrac{-\pi}{2},\nicefrac{\pi}{2} \right[ ~\mid (r\cos\varphi\cos\theta, r\cos\varphi\sen\theta, r\sen\varphi) \in A \right\} \\
            &= \left\{ (r, \theta, \varphi) \in \bb{R}^+\times \left]-\pi, \pi\right[ \times \left] \nicefrac{-\pi}{2},\nicefrac{\pi}{2} \right[ ~\mid r > 1,~\varphi,\theta\in \left]0,\nicefrac{\pi}{2}\right[ \right\} \\
            &= \left] 1, +\infty \right[ \times \left] 0, \nicefrac{\pi}{2} \right[ \times \left] 0,\nicefrac{\pi}{2} \right[
        \end{align*}

        Definimos por tanto la función $g: E \to \bb{R}^3$ como:
        \begin{align*}
            g(r, \theta, \varphi) &= r^2\cos\varphi\cdot f(r\cos\varphi\cos\theta, r\cos\varphi\sen\theta, r\sen\varphi)
            = \frac{r^5\cos^3\varphi\cos\theta\sen\theta\sen\varphi}{r^8} =\\
            &= \frac{\cos^3\varphi\cos\theta\sen\theta\sen\varphi}{r^3}
        \end{align*}

        Veamos que $g\in \cc{L}_1(E)$ usando el Teorema de Tonelli:
        \begin{align*}
            \int_E |g(r, \theta, \varphi)|~d(r, \theta, \varphi)
            &\leq \int_E \frac{1}{r^3}~d(r, \theta, \varphi)
            = \int_{0}^{\nicefrac{\pi}{2}} \left( \int_{0}^{\nicefrac{\pi}{2}} \left( \int_1^{+\infty} \frac{dr}{r^3} \right)~d\theta \right)~d\varphi
            =\\&= \frac{\pi^2}{4} \int_{1}^{+\infty} \frac{dr}{r^3}
            = \frac{\pi^2}{4} \left[ -\frac{1}{2r^2} \right]_1^{+\infty} = \frac{\pi^2}{8} < \infty
        \end{align*}

        Por tanto, $g\in \cc{L}_1(E)$, y por el Teorema de Cambio de Variable, tenemos que $f\in \cc{L}_1(A)$. Para calcular la integral, hay dos formas de proceder:
        Calcular la integral de forma normal, usando el Teorema de Cambio de Variable y el Teorema de Fubini.
        \begin{align*}
            \int_A f &= \int_E g = \int_{0}^{\nicefrac{\pi}{2}} \left( \int_{0}^{\nicefrac{\pi}{2}} \left( \int_1^{+\infty} \frac{\cos^3\varphi\cos\theta\sen\theta\sen\varphi}{r^3}~dr \right)~d\theta \right)~d\varphi =\\
            &= \left( \int_{0}^{\nicefrac{\pi}{2}} \cos^3\varphi\sen\varphi~d\varphi \right) \left( \int_{0}^{\nicefrac{\pi}{2}} \cos\theta\sen\theta~d\theta \right) \left( \int_1^{+\infty} \frac{dr}{r^3} \right) =\\
            &= \left[-\dfrac{\cos^4\varphi}{4}\right]_{0}^{\nicefrac{\pi}{2}} \cdot \left[-\dfrac{\cos(2\theta)}{4}\right]_{0}^{\nicefrac{\pi}{2}} \cdot \left[-\dfrac{1}{2r^2}\right]_1^{+\infty} =\\
            &= \frac{1}{4} \cdot \left(\frac{1}{4}+\frac{1}{4}\right) \cdot \dfrac{1}{2} = \frac{1}{16}
        \end{align*}

        \begin{comment}

            \item[Opción 2.] Usar que es simétrica respecto de la variable $x$. Para ello, definimos en primer lugar $A_0,~A_+$ y $A_-$ como:
            \begin{align*}
                A_0 &= \left\{ (x, y, z) \in A \mid x=0 \right\} \\
                A_+ &= \left\{ (x, y, z) \in A \mid x > 0 \right\} \\
                A_- &= \left\{ (x, y, z) \in A \mid x < 0 \right\}
            \end{align*}

            Por ser $A_0$ un hiperplano en $\bb{R}^3$, tenemos que $\lm_3(A_0) = 0$. Por otro lado, tenemos que:
            \begin{gather*}
                A_- = \left\{ (-x, y, z) \in \bb{R}^3 \mid (x, y, z) \in A_+ \right\} \\
                f(-x, y, z) = -f(x, y, z) \quad \forall (x, y, z) \in A_+
            \end{gather*}

            Por tanto, como $A=A_+ \uplus A_- \uplus A_0$, por la aditividad de la integral de Lebesgue, tenemos que:
            \begin{align*}
                \int_A f &= \int_{A_+} f + \int_{A_-} f + \int_{A_0} f = \int_{A_+} f - \int_{A_+} f = 0
            \end{align*}
        \end{description}
    \end{comment}
    \end{enumerate}
\end{ejercicio}

\begin{ejercicio}
    En cada uno de los siguientes casos, estudiar la integrabilidad de la función \( f \) en el conjunto \( A \):

    \begin{enumerate}
        \item \( A = \mathbb{R}^2 \setminus \{(0, 0)\} \)
        \[ f(x, y) = \frac{\sen x \sen y}{(x^2 + y^2)^{\nicefrac{3}{2}}} \quad \forall (x, y) \in A \]

        En primer lugar, descomponemos el conjunto $A$ en dos conjuntos disjuntos:
        \begin{align*}
            A_1 &= \left\{ (x, y) \in \bb{R}^2 \mid x^2 + y^2 > 1 \right\} \\
            A_2 &= \left\{ (x, y) \in \bb{R}^2 \mid x^2 + y^2 \leq 1 \right\}\setminus \{(0, 0)\}
        \end{align*}

        tenemos que $A=A_1 \uplus A_2$. Por tanto, $f\in \cc{L}_1(A)$ si y solo si $f\in \cc{L}_1(A_1)$ y $f\in \cc{L}_1(A_2)$. Estudiemos en primer lugar la integrabilidad de $f$ en $A_1$.
        Para ello, aplicaremos el cambio de variable a coordenadas polares. Obtenemos el conjunto $E_1$ siguiente:
        \begin{align*}
            E_1 &= \left\{ (\rho, \theta) \in \bb{R}^+\times \left]-\pi, \pi\right[ ~\mid (\rho\cos\theta, \rho\sen\theta) \in A_1 \right\} \\
            &= \left\{ (\rho, \theta) \in \bb{R}^+\times \left]-\pi, \pi\right[ ~\mid \rho > 1 \right\} \\
            &= \left]1,\infty\right[\times ]-\pi, \pi[
        \end{align*}

        Definimos por tanto la función $g_1: E_1 \to \bb{R}$ como:
        \begin{equation*}
            g_1(\rho, \theta) = \rho f(\rho\cos\theta, \rho\sen\theta) = \rho\cdot \frac{\sen(\rho\cos\theta) \sen(\rho\sen\theta)}{\rho^3}
            = \frac{\sen(\rho\cos\theta) \sen(\rho\sen\theta)}{\rho^2}
        \end{equation*}

        Por el Teorema de Cambio de Variable, tenemos que $f\in \cc{L}_1(A_1)$ si y solo si $g_1\in \cc{L}_1(E_1)$. Veamos si $g_1\in \cc{L}_1(E_1)$:
        \begin{align*}
            \int_{E_1} |g_1(\rho, \theta)|~d(\rho, \theta)
            &\leq \int_{E_1} \frac{1}{\rho^2}~d(\rho, \theta)
            = \int_{-\pi}^\pi \left( \int_1^{+\infty} \frac{d\rho}{\rho^2} \right)~d\theta
            = 2\pi \int_1^{+\infty} \frac{d\rho}{\rho^2} =\\
            &= 2\pi \left[ -\frac{1}{\rho} \right]_1^{+\infty} = 2\pi < \infty
        \end{align*}
        
        Por tanto, $g_1\in \cc{L}_1(E_1)$, y por el Teorema de Cambio de Variable, tenemos que $f\in \cc{L}_1(A_1)$.
        \begin{observacion}
            Es cierto que no es fácil que se nos ocurra descomponer $A$ en $A_1$ y $A_2$. No obstante,
            el razonamiento seguido sería el lógico, y si no se hubiese descompuesto en ambos conjuntos aquí nos encontraríamos
            con un problema, puesto que la función $x\mapsto \frac{1}{x^2}$ no es integrable en $\left]0,c\right[$ para ningún $c\in \bb{R}^+$.
            Por esto, tomando $c=1$, conseguimos que ahora sí sea integrable en $A_1$. Para $A_2$ será necesario un razonamiento más sutil.
        \end{observacion}

        Estudiemos ahora la integrabilidad de $f$ en $A_2$. Para ello, aplicamos el cambio de variable a coordenadas polares. Obtenemos el conjunto $E_2$ siguiente:
        \begin{align*}
            E_2 &= \left\{ (\rho, \theta) \in \bb{R}^+\times \left]-\pi, \pi\right[ ~\mid (\rho\cos\theta, \rho\sen\theta) \in A_2 \right\} \\
            &= \left\{ (\rho, \theta) \in \bb{R}^+\times \left]-\pi, \pi\right[ ~\mid \rho \leq 1 \right\} \\
            &= \left]0,1\right]\times ]-\pi, \pi[
        \end{align*}

        Definimos por tanto la función $g_2: E_2 \to \bb{R}$ como:
        \begin{equation*}
            g_2(\rho, \theta) = \rho f(\rho\cos\theta, \rho\sen\theta) = \rho\cdot \frac{\sen(\rho\cos\theta) \sen(\rho\sen\theta)}{\rho^3}
            = \frac{\sen(\rho\cos\theta) \sen(\rho\sen\theta)}{\rho^2}
        \end{equation*}

        Por el Teorema de Cambio de Variable, tenemos que $f\in \cc{L}_1(A_1)$ si y solo si $g_2\in \cc{L}_1(E_2)$. Para acotar $g_2$,
        hacemos uso de la Desigualdad del Valor Medio para funciones de una variable, de la que deducimos que:
        \begin{equation*}
            |\sen(t) - \sen(0)| \leq |t-0| \quad \forall t\in \bb{R} \Longrightarrow |\sen(t)| \leq |t| \quad \forall t\in \bb{R}
        \end{equation*}

        Por tanto, tenemos que:
        \begin{align*}
            |g_2(\rho, \theta)| &= \left| \frac{\sen(\rho\cos\theta) \sen(\rho\sen\theta)}{\rho^2} \right|
            \leq \frac{|\rho\cos\theta| \cdot |\rho\sen\theta|}{\rho^2}
            \leq 1 \quad \forall (\rho, \theta) \in E_2
        \end{align*}

        Por tanto, tenemos que:
        \begin{align*}
            \int_{E_2} |g_2(\rho, \theta)|~d(\rho, \theta)
            &\leq \int_{E_2} 1~d(\rho, \theta)
            = \lm(E_2) = \lm\left(\left]0,1\right]\right) \cdot \lm\left(]-\pi, \pi[\right) = 2\pi < \infty
        \end{align*}

        Por tanto, $g_2\in \cc{L}_1(E_2)$, y por el Teorema de Cambio de Variable, tenemos que $f\in \cc{L}_1(A_2)$.
        Como $f\in \cc{L}_1(A_1)$ y $f\in \cc{L}_1(A_2)$, tenemos que $f\in \cc{L}_1(A)$.

        \item \( A = \left\{ (x, y, z) \in \mathbb{R}^3 \mid z > x^2 + y^2 \right\} \)
        \[ f(x, y, z) = (x^3 + y^3) \cos (xy) e^{-z} \quad \forall (x, y, z) \in A \]

        Aplicaremos el cambio de variable a coordenadas cilíndricas. Para ello, en primer
        lugar obtenemos el conjunto $E$ siguiente:
        \begin{align*}
            E &= \left\{ (\rho, \theta, z) \in \bb{R}^+\times \left]-\pi, \pi\right[ \times \bb{R} ~\mid (\rho\cos\theta, \rho\sen\theta, z) \in A \right\} \\
            &= \left\{ (\rho, \theta, z) \in \bb{R}^+\times \left]-\pi, \pi\right[ \times \bb{R} ~\mid z > \rho^2 \right\}
        \end{align*}

        Definimos por tanto la función $g: E \to \bb{R}^3$ para cada $(\rho, \theta, z) \in E$ como:
        \begin{equation*}
            g(\rho, \theta, z) = \rho f(\rho\cos\theta, \rho\sen\theta, z) = \rho^4(\cos^3\theta + \sen^3\theta) \cos(\rho^2\cos\theta\sen\theta)e^{-z}
        \end{equation*}

        Por el Teorema de Cambio de Variable, tenemos que $f\in \cc{L}_1(A)$ si y solo si $g\in \cc{L}_1(E)$. Para estudiar la integrabilidad de $g$, veamos si $g\in \cc{L}_1(E)$,
        para lo cual usaremos el Teorema de Tonelli dos veces:
        \begin{align*}
            \int_E |g(\rho, \theta, z)|~d(\rho, \theta, z)
            &\leq \int_E \rho^4e^{-z}~d(\rho, \theta, z)
            = \int_{-\pi}^\pi \left( \int_0^{+\infty} \left( \int_{\rho^2}^{+\infty} \rho^4e^{-z}~dz \right)~d\rho \right)~d\theta =\\
            &= 2\pi \int_0^{+\infty}\rho^4 \left( \int_{\rho^2}^{+\infty}e^{-z}~dz \right)~d\rho
            = 2\pi \int_0^{+\infty}\rho^4 \left[ -e^{-z} \right]_{\rho^2}^{+\infty}~d\rho =\\
            &= 2\pi \int_0^{+\infty}\rho^4e^{-\rho^2}~d\rho
        \end{align*}

        Para ver si dicha última función es integrable, podemos hacer uso del Criterio de Comparación con la función $x\mapsto e^{-x}$, integrable en $\bb{R}^+_0$.
        Como ambas son continuas, sabemos que son localmente integrables en $\bb{R}^+_0$. Tenemos el siguiente límite:
        \begin{equation*}
            \lim_{\rho\to+\infty} \frac{\rho^4e^{-\rho^2}}{e^{-\rho}} = \lim_{\rho\to+\infty} \rho^4e^{-\rho^2+\rho} = 0
        \end{equation*}

        Por tanto, por el Criterio de Comparación, como la función empleada para comparar es integrable, tenemos que:
        \begin{equation*}
            \int_0^{+\infty}\rho^4e^{-\rho^2}~d\rho < \infty
        \end{equation*}

        Por tanto, tenemos que:
        \begin{equation*}
            \int_E |g(\rho, \theta, z)|~d(\rho, \theta, z) = 2\pi \int_0^{+\infty}\rho^4e^{-\rho^2}~d\rho < \infty
        \end{equation*}

        Por tanto, $g\in \cc{L}_1(E)$, y por el Teorema de Cambio de Variable, tenemos que $f\in \cc{L}_1(A)$.

        \item \( A = \mathbb{R}^3 \)
        \[ f(x, y, z) = \frac{1}{\left(1 + x^2 + y^2 + z^2\right)^\alpha} \quad \forall (x, y, z) \in A \quad (\alpha \in \mathbb{R}^+) \]

        Apliquemos el cambio de variable a coordenadas esféricas. Para ello, en primer
        lugar obtenemos el conjunto $E$ siguiente:
        \begin{align*}
            E &= \left\{ (r, \theta, \varphi) \in \bb{R}^+\times \left]-\pi, \pi\right[ \times \left] \nicefrac{-\pi}{2},\nicefrac{\pi}{2} \right[ ~\mid (r\cos\varphi\cos\theta, r\cos\varphi\sen\theta, r\sen\varphi) \in A \right\} \\
            &= \bb{R}^+\times \left]-\pi, \pi\right[ \times \left] \nicefrac{-\pi}{2},\nicefrac{\pi}{2} \right[
        \end{align*}

        Definimos por tanto la función $g: E \to \bb{R}$ para cada $(r, \theta, \varphi) \in E$ como:
        \begin{equation*}
            g(r, \theta, \varphi) = r^2\cos\varphi f(r\cos\varphi\cos\theta, r\cos\varphi\sen\theta, r\sen\varphi) =
            \frac{r^2\cos\varphi}{\left(1 + r^2\right)^\alpha}
        \end{equation*}

        Por el Teorema de Cambio de Variable, tenemos que $f\in \cc{L}_1(A)$ si y solo si $g\in \cc{L}_1(E)$. Para estudiar la integrabilidad de $g$, usamos el Teorema de Tonelli:
        \begin{align*}
            \int_A f(x, y, z)~d(x, y, z) &= \int_E g(r, \theta, \varphi)~d(r, \theta, \varphi) =\\
            &= \int_{-\pi}^\pi \left( \int_{\nicefrac{-\pi}{2}}^{\nicefrac{\pi}{2}} \left( \int_0^{+\infty} \frac{r^2\cos\varphi}{\left(1 + r^2\right)^\alpha}~dr \right)~d\varphi \right)~d\theta =\\
            &= 2\pi \int_{\nicefrac{-\pi}{2}}^{\nicefrac{\pi}{2}} \left( \int_0^{+\infty} \frac{r^2\cos\varphi}{\left(1 + r^2\right)^\alpha}~dr \right)~d\varphi =\\
            &= 2\pi \left(\int_{\nicefrac{-\pi}{2}}^{\nicefrac{\pi}{2}} \cos\varphi~d\varphi\right) \left(\int_0^{+\infty} \frac{r^2}{\left(1 + r^2\right)^\alpha}~dr\right) =\\
            &= 4\pi \left(\int_0^{+\infty} \frac{r^2}{\left(1 + r^2\right)^\alpha}~dr\right)
        \end{align*}

        Para estudiar la integrabilidad de la última integral, como tiene extensión continua en $\bb{R}^0_0$ sabemos que será integrable en $[0,1]$,
        luego nos centraremos en estudiar su integrabilidad en $[1,+\infty[$. Para ello,
        podemos hacer uso de la función $r\mapsto r^{2-2\alpha}$.
        Como ambas son continuas, sabemos que son localmente integrables en $[1,+\infty[$. Tenemos el siguiente límite:
        \begin{equation*}
            \lim_{r\to+\infty} \dfrac{\frac{r^2}{\left(1 + r^2\right)^\alpha}}{r^{2-2\alpha}}
            = \lim_{r\to+\infty} \dfrac{r^{2\alpha}}{\left(1 + r^2\right)^\alpha} = 1
        \end{equation*}

        Por tanto, por el Criterio de Comparación, tenemos que l función cuya integrabilidad queremos estudiar es integrable si y solo si lo es la función $r\mapsto r^{2-2\alpha}$.
        Esta última sabemos que es integrable en $[1,+\infty[$ si y solo si $2-2\alpha < -1 \iff \alpha > \nicefrac{3}{2}$.
        Por tanto, la función $f$ es integrable en $A$ si y solo si $\alpha > \nicefrac{3}{2}$.
    \end{enumerate}
\end{ejercicio}

\begin{ejercicio}
    Calcular el volumen de la llamada \emph{bóveda de Viviani}:
    \[ B = \left\{ (x, y, z) \in \mathbb{R}^3 \mid (2x - 1)^2 + 4y^2 \leq 1 ,~x^2 + y^2 + z^2 \leq 1 ,~z \geq 0 \right\} \]

    Tenemos que $B$ es cerrado y acotado, luego es compacto. Por tanto, $B\in \cc{M}_3$.
    El conjunto $B$ podemos expresarlo como sigue:
    \begin{align*}
        B &= \left\{ (x, y, z) \in \mathbb{R}^3 \mid (2x - 1)^2 + 4y^2 \leq 1,~0\leq z\leq \sqrt{1 - x^2 - y^2} \right\}
    \end{align*}

    Definimos $A=\left\{ (x, y) \in \bb{R}^2 \mid (2x - 1)^2 + 4y^2 \leq 1 \right\}$ y la siguiente función:
    \Func{f}{A}{\bb{R}}{(x, y)}{\sqrt{1 - x^2 - y^2}}

    Como $A\in \cc{M}_2$ por ser cerrado y $f$ es medible por ser continua,
    tenemos que $\text{Sg}(f)\in \cc{M}_3$. Además, sabemos que $\lm_3(B)=\lm_3(\text{Sg}(f))$, 
    por lo que:
    \begin{align*}
        \lm_3(B) &= \int_A \sqrt{1 - x^2 - y^2}~d(x, y)
    \end{align*}

    Apliquemos el cambio de variable a coordenadas polares. Para ello, en primer
    lugar obtenemos el conjunto $E$ de las coordenadas polares de los puntos de $A$.
    Si $\rho\in \bb{R}^+,~\theta\in \left]-\pi, \pi\right[$, se tiene que
    $(\rho\cos\theta, \rho\sen\theta)\in A$ si y solo si:
    \begin{align*}
        (\rho\cos\theta, \rho\sen\theta) &\in A \iff (2\rho\cos\theta - 1)^2 + 4\rho^2\sen^2\theta \leq 1 \iff\\
        \iff 4\rho^2\cos^2\theta - 4\rho\cos\theta + 1 + 4\rho^2\sen^2\theta &\leq 1 \iff 4\rho^2 - 4\rho\cos\theta \leq 0 \iff\\
        \iff \rho^2 - \rho\cos\theta &\leq 0 \iff \rho(\rho - \cos\theta) \leq 0 \iff \rho \leq \cos\theta
    \end{align*}

    Por tanto, el conjunto $E$ es:
    \begin{align*}
        E &= \left\{ (\rho, \theta) \in \bb{R}^+\times \left]\nicefrac{-\pi}{2}, \nicefrac{\pi}{2}\right[ ~\mid \rho \leq \cos\theta \right\}
    \end{align*}

    Definimos por tanto la función $g: E \to \bb{R}$ para cada $(\rho, \theta) \in E$ como:
    \begin{equation*}
        g(\rho, \theta) = \rho f(\rho\cos\theta, \rho\sen\theta) = \rho\sqrt{1 - \rho^2\cos^2\theta - \rho^2\sen^2\theta} = \rho\sqrt{1 - \rho^2}
    \end{equation*}

    Usando el Teorema de Cambio de Variable, junto con el Teorema de Tonelli, tenemos que:
    \begin{align*}
        \lm_3(B) &= \int_A \sqrt{1 - x^2 - y^2}~d(x, y) = \int_E \rho\sqrt{1 - \rho^2}~d(\rho, \theta) =\\
        &= \int_{-\nicefrac{\pi}{2}}^{\nicefrac{\pi}{2}} \left( \int_0^{\cos\theta} \rho\sqrt{1 - \rho^2}~d\rho \right)~d\theta =\\
        &= \red{-\frac{1}{2}}\int_{-\nicefrac{\pi}{2}}^{\nicefrac{\pi}{2}} \left( \int_0^{\cos\theta} \red{-2}\rho\sqrt{1 - \rho^2}~d\rho \right)~d\theta =\\
        &= {-\frac{1}{2}}\int_{-\nicefrac{\pi}{2}}^{\nicefrac{\pi}{2}} \left[ \dfrac{\left(1-\rho^2\right)^{\nicefrac{3}{2}}}{\nicefrac{3}{2}} \right]_0^{\cos\theta}~d\theta
        = {-\frac{1}{3}}\int_{-\nicefrac{\pi}{2}}^{\nicefrac{\pi}{2}} \left[ \left(1-\rho^2\right)^{\nicefrac{3}{2}} \right]_0^{\cos\theta}~d\theta =\\
        &= {-\frac{1}{3}}\int_{-\nicefrac{\pi}{2}}^{\nicefrac{\pi}{2}} \left(1-\cos^2\theta\right)^{\nicefrac{3}{2}} - 1~d\theta 
        = {-\frac{1}{3}}\int_{-\nicefrac{\pi}{2}}^{\nicefrac{\pi}{2}} |\sen\theta|^3~d\theta +\frac{1}{3} \left[\theta\right]_{-\nicefrac{\pi}{2}}^{\nicefrac{\pi}{2}} =\\
        &= \frac{\pi}{3} -\frac{2}{3}\int_{0}^{\nicefrac{\pi}{2}} \sen^3\theta~d\theta
        = \frac{\pi}{3} -\frac{2}{3}\int_{0}^{\nicefrac{\pi}{2}} \left(1-\cos^2\theta\right)\sen\theta~d\theta =\\
        &= \frac{\pi}{3} -\frac{2}{3}\left(\int_{0}^{\nicefrac{\pi}{2}} \sen\theta~d\theta - \int_{0}^{\nicefrac{\pi}{2}} \cos^2\theta\sen\theta~d\theta\right) =\\
        &= \frac{\pi}{3} -\frac{2}{3}\left(\left[-\cos\theta\right]_{0}^{\nicefrac{\pi}{2}} + \left[\dfrac{\cos^3\theta}{3}\right]_{0}^{\nicefrac{\pi}{2}}\right)
        = \frac{\pi}{3} -\frac{2}{3}\left[-\cos\theta + \dfrac{\cos^3\theta}{3}\right]_{0}^{\nicefrac{\pi}{2}} =\\
        &= \frac{\pi}{3} -\frac{2}{3}\left[-\cos\left(\nicefrac{\pi}{2}\right) + \dfrac{\cos^3\left(\nicefrac{\pi}{2}\right)}{3} + \cos(0) - \dfrac{\cos^3(0)}{3}\right] =\\
        &= \frac{\pi}{3} -\frac{2}{3}\cdot \frac{2}{3} = \dfrac{3\pi - 4}{9}
    \end{align*}
\end{ejercicio}
    

\begin{ejercicio}[Extraordinaria DGIIM 22-23]
    Dados $\alpha, \beta \in \bb{R}$ con $0 < \alpha < \beta$, probar que el siguiente conjunto es medible y calcular su volumen:
    \[ E = \left\{ (x, y, z) \in \bb{R}^3 \mid x^2 + y^2 \leq \alpha^2,~x^2 + y^2 + z^2 \leq \beta^2 \right\} \]

    Tenemos que $E$ es intersección de dos cerrados,
    luego es cerrado y, por tanto, $E\in \cc{M}_3$. Para calcular su volumen,
    dados $(x,y)\in \bb{R}^2$, calculamos la sección vertical siguiente:
    \begin{align*}
        E_{(x,y)} &= \left\{ z\in \bb{R} \mid (x, y, z) \in E \right\} = \left\{ z\in \bb{R} \mid x^2 + y^2 + z^2 \leq \beta^2,~x^2 + y^2 \leq \alpha^2 \right\}
    \end{align*}

    Consideramos el conjunto $A = \left\{ z\in \bb{R} \mid x^2 + y^2 \leq \alpha^2 \right\}$, y tenemos claramente que
    $E_{(x,y)} = \emptyset$ para todo $(x,y)\in \bb{R}^2\setminus A$. Para $(x,y)\in A$, tenemos que:
    \begin{align*}
        E_{(x,y)} &= \left\{ z\in \bb{R} \mid x^2 + y^2 + z^2 \leq \beta^2 \right\} = \left[ -\sqrt{\beta^2 - x^2 - y^2}, \sqrt{\beta^2 - x^2 - y^2} \right]
    \end{align*}

    Por tanto, por el Teorema de Tonelli en el caso particular de una función característica, tenemos que:
    \begin{align*}
        \lm_3(E) &= \int_{\bb{R}^2} \lm_1(E_{(x,y)})~d(x, y) = \int_{A} 2\sqrt{\beta^2 - x^2 - y^2}~d(x, y) =\\
        &= 2\int_{A} \sqrt{\beta^2 - x^2 - y^2}~d(x, y)
    \end{align*}

    Para resolver esta integral, realizamos un cambio de variable a coordenadas polares. Calculemos $E$, el conjunto de coordenadas polares de los puntos de $A$:
    \begin{align*}
        E &= \left\{ (\rho, \theta) \in \bb{R}^+\times \left]-\pi, \pi\right[ ~\mid \rho \leq \alpha \right\} = \left]0, \alpha\right]\times \left]-\pi, \pi\right[
    \end{align*}

    Por el Teorema de Cambio de Variable, junto con el Teorema de Tonelli, tenemos que:
    \begin{align*}
        \lm_3(E) &= 2\int_{A} \sqrt{\beta^2 - x^2 - y^2}~d(x, y) = 2\int_{E} \rho\sqrt{\beta^2 - \rho^2}~d(\rho, \theta) =\\
        &= 2\int_{-\pi}^\pi \left( \int_0^\alpha \rho\sqrt{\beta^2 - \rho^2}~d\rho \right)~d\theta = 4\pi \int_0^\alpha \rho\sqrt{\beta^2 - \rho^2}~d\rho =\\
        &= \red{-} \dfrac{4\pi}{\red{2}} \int_0^\alpha \red{-2}\rho\sqrt{\beta^2 - \rho^2}~d\rho =
        -2\pi\left[ \dfrac{{(\beta^2-\rho^2)}^{\nicefrac{3}{2}}}{\nicefrac{3}{2}} \right]_0^\alpha =\\
        &= -\dfrac{4\pi}{3} \left[ {(\beta^2-\rho^2)}^{\nicefrac{3}{2}}\right]_0^\alpha
        = -\dfrac{4\pi}{3} \left[ {(\beta^2-\alpha^2)}^{\nicefrac{3}{2}} - \beta^3 \right]
        = \dfrac{4\pi}{3} \left[ \beta^3 - {(\beta^2-\alpha^2)}^{\nicefrac{3}{2}} \right]
    \end{align*}
\end{ejercicio}



\begin{ejercicio}
    Sea $\Omega=\left\{(x,y)\in \bb{R}^2 \mid x^2+y^2\leq 1,~x^2+y^2\leq 2x\right\}$, y sea la función $f:\Omega\to \bb{R}$ definida por:
    \[ f(x,y) = x \quad \forall (x,y)\in \Omega \]
    Probar que la función \( f \) es integrable en \( \Omega \) y calcular su integral.

    \begin{observacion}
        Notemos que este es el apartado \ref{ej:2.4.2.2} del Ejercicio~\ref{ej:2.4.2}.
        No obstante, se incluye aquí para mostrar otra forma de resolverlo, haciendo uso de coordenadas polares.
    \end{observacion}

    Sabemos que $\Omega$ es cerrado y acotado, luego es compacto y, por tanto, $\Omega\in~\cc{M}_2$. Además, como $f$ es continua y $\Omega$ es compacto, tenemos que $f\in \cc{L}_1(\Omega)$.
    Para calcular su integral, haremos uso de coordenadas polares. Para ello, en primer lugar, obtenemos el conjunto $E$ de las coordenadas polares de los puntos de $\Omega$.
    \begin{align*}
        E &= \left\{ (\rho, \theta) \in \bb{R}^+\times \left]-\pi, \pi\right[ ~\mid \rho\leq 1,~\rho^2\leq 2\rho\cos\theta \right\} =\\
        &= \left\{ (\rho, \theta) \in \bb{R}^+\times \left]-\pi, \pi\right[ ~\mid \rho\leq 1,~\rho\leq 2\cos\theta \right\}
    \end{align*}

    Calculamos ahora las secciones horizontales de $E$. Fijado $\theta\in \left]-\pi, \pi\right[$, tenemos:
    \begin{align*}
        E^{\theta} &= \left\{ \rho\in \bb{R}^+ \mid \rho\leq 1,~\rho\leq 2\cos\theta \right\} = \emptyset \qquad \text{si } \cos\theta < 0 \Longleftrightarrow \theta\in \left]-\pi, \pi\right[\setminus \left]\nicefrac{-\pi}{2}, \nicefrac{\pi}{2}\right[\\
        E^{\theta} &= [0, 1] \qquad \text{si } 1\leq 2\cos\theta \Longleftrightarrow \theta\in \left[\nicefrac{-\pi}{3}, \nicefrac{\pi}{3}\right] \\
        E^{\theta} &= [0, 2\cos\theta] \qquad \text{si } 0\leq 2\cos\theta \leq 1 \Longleftrightarrow \theta\in \left[\nicefrac{-\pi}{2}, \nicefrac{\pi}{2}\right]\setminus \left[\nicefrac{-\pi}{3}, \nicefrac{\pi}{3}\right]
    \end{align*}

    Por tanto, por el Teorema de Cambio de Variable, junto con el Teorema de Tonelli, tenemos que:
    \begin{align*}
        \int_{\Omega}& f(x, y)~d(x, y) = \int_{E} \rho^2\cos\theta~d(\rho, \theta) =\\
        &= \int_{\nicefrac{-\pi}{2}}^{\nicefrac{-\pi}{3}}\left( \int_0^{2\cos\theta} \rho^2\cos\theta~d\rho \right)~d\theta + \int_{\nicefrac{-\pi}{3}}^{\nicefrac{\pi}{3}}\left( \int_0^1 \rho^2\cos\theta~d\rho \right)~d\theta +\\&\hspace{2cm}+ \int_{\nicefrac{\pi}{3}}^{\nicefrac{\pi}{2}}\left( \int_0^{2\cos\theta} \rho^2\cos\theta~d\rho \right)~d\theta =\\
        &= \int_{\nicefrac{-\pi}{2}}^{\nicefrac{-\pi}{3}}\cos\theta \left[ \dfrac{\rho^3}{3} \right]_0^{2\cos\theta}~d\theta + \int_{\nicefrac{-\pi}{3}}^{\nicefrac{\pi}{3}}\cos\theta \left[ \dfrac{\rho^3}{3} \right]_0^1~d\theta + \int_{\nicefrac{\pi}{3}}^{\nicefrac{\pi}{2}}\cos\theta \left[ \dfrac{\rho^3}{3} \right]_0^{2\cos\theta}~d\theta=\\
        &= \dfrac{8}{3}\int_{\nicefrac{-\pi}{2}}^{\nicefrac{-\pi}{3}}\cos^4\theta~d\theta + \dfrac{1}{3}\int_{\nicefrac{-\pi}{3}}^{\nicefrac{\pi}{3}}\cos\theta~d\theta + \dfrac{8}{3}\int_{\nicefrac{\pi}{3}}^{\nicefrac{\pi}{2}}\cos^4\theta~d\theta =\\
        &= \dfrac{16}{3}\int_{\nicefrac{\pi}{3}}^{\nicefrac{\pi}{2}}\cos^4\theta~d\theta + \dfrac{2}{3}\int_{0}^{\nicefrac{\pi}{3}}\cos\theta~d\theta =\\
        &= \dfrac{16}{3}\int_{\nicefrac{\pi}{3}}^{\nicefrac{\pi}{2}}\cos^4\theta~d\theta + \dfrac{2}{3}\left[\sen\theta\right]_{0}^{\nicefrac{\pi}{3}}
        = \dfrac{16}{3}\int_{\nicefrac{\pi}{3}}^{\nicefrac{\pi}{2}}\cos^4\theta~d\theta + \dfrac{\sqrt{3}}{3}
    \end{align*}
    
    La resolución de la integral restante no es del todo sencilla, ya que requiere una buena idea. No obstante, al ser frecuente es didáctico resolverla. Tenemos que:
    \begin{align*}
        \cos^4(\theta) &= \left(\cos^2(\theta)\right)^2 = \left(\dfrac{1+\cos(2\theta)}{2}\right)^2 = \dfrac{1}{4} + \dfrac{1}{2}\cos(2\theta) + \dfrac{1}{4}\cos^2(2\theta) =\\
        &= \dfrac{1}{4} + \dfrac{1}{2}\cos(2\theta) + \dfrac{1}{4}\dfrac{1+\cos(4\theta)}{2} = \dfrac{1}{4} + \dfrac{1}{2}\cos(2\theta) + \dfrac{1}{8} + \dfrac{1}{8}\cos(4\theta)
    \end{align*}

    Por tanto, tenemos que:
    \begin{align*}
        \int_{\nicefrac{\pi}{3}}^{\nicefrac{\pi}{2}}\cos^4\theta~d\theta &= \int_{\nicefrac{\pi}{3}}^{\nicefrac{\pi}{2}}\left(\dfrac{1}{4} + \dfrac{1}{2}\cos(2\theta) + \dfrac{1}{8} + \dfrac{1}{8}\cos(4\theta)\right)~d\theta =\\
        &= \left[\dfrac{\theta}{4} + \dfrac{1}{4}\sen(2\theta) + \dfrac{\theta}{8} + \dfrac{1}{32}\sen(4\theta)\right]_{\nicefrac{\pi}{3}}^{\nicefrac{\pi}{2}} =\\
        &= \dfrac{\pi}{8} + \dfrac{\pi}{16} -\dfrac{\pi}{12} - \dfrac{\pi}{24} - \dfrac{\sqrt{3}}{8} + \dfrac{\sqrt{3}}{64}
        = \dfrac{\pi}{16} - \dfrac{7\sqrt{3}}{64}
    \end{align*}

    Por tanto, tenemos que:
    \begin{align*}
        \int_{\Omega} f(x, y)~d(x, y) = \dfrac{16}{3}\int_{\nicefrac{\pi}{3}}^{\nicefrac{\pi}{2}}\cos^4\theta~d\theta + \dfrac{\sqrt{3}}{3} = \dfrac{16}{3}\left(\dfrac{\pi}{16} - \dfrac{7\sqrt{3}}{64}\right) + \dfrac{\sqrt{3}}{3} = \dfrac{\pi}{3} - \dfrac{\sqrt{3}}{4}
    \end{align*}
\end{ejercicio}



\begin{ejercicio}[Parcial DGIIM 23-24]
    Sea el conjunto $A$ y la función $g:A\to \bb{R}$ definidos por:
    \begin{gather*}
        A=\left\{(x,y,z)\in \bb{R}^3 \mid x>0,~z>0,~x^2+y^2+z^2> 1\right\}\\
        g(x,y,z) = \dfrac{\sqrt{x^2+y^2}}{(x^2+y^2+z^2)^{3}} \quad \forall (x,y,z)\in A
    \end{gather*}
    Probar que la función \( g \) es integrable en \( A \) y calcular su integral.

    Sabemos que $A$ es abierto, luego es medible y, por tanto, $A\in \cc{M}_3$. Además, como $g$ es continua, tenemos que $g$ es medible.
    Para calcular su integral, haremos uso de coordenadas esféricas escribiendo:
    \begin{equation*}
        x = r\cos\theta\cos\varphi, \quad y = r\sen\theta\cos\varphi, \quad z = r\sen\varphi
    \end{equation*}
    con $r\in \bb{R}^+$, $\theta\in \left]-\pi, \pi\right[$ y $\varphi\in \left]-\nicefrac{\pi}{2}, \nicefrac{\pi}{2}\right[$.
    Para ello, en primer lugar, obtenemos el conjunto $E$ de las coordenadas esféricas de los puntos de $A$.
    \begin{align*}
        E &= \left\{ (r, \theta, \varphi) \in \bb{R}^+\times \left]-\pi, \pi\right[ \times \left] \nicefrac{-\pi}{2},\nicefrac{\pi}{2} \right[ ~\mid r\cos\theta\cos\varphi > 0,~\sen\varphi > 0,~r^2 > 1 \right\} =\\
        &= \left\{ (r, \theta, \varphi) \in \bb{R}^+\times \left]-\pi, \pi\right[ \times \left] \nicefrac{-\pi}{2},\nicefrac{\pi}{2} \right[ ~\mid \cos\theta\cos\varphi > 0,~\sen\varphi > 0,~r > 1 \right\} =\\
        &= \left]1, +\infty\right[\times \left]\nicefrac{-\pi}{2}, \nicefrac{\pi}{2}\right[ \times \left]0, \nicefrac{\pi}{2}\right[
    \end{align*}

    Definimos por tanto la función $h: E \to \bb{R}$ para cada $(r, \theta, \varphi) \in E$ como:
    \begin{align*}
        h(r, \theta, \varphi) &=g(r\cos\theta\cos\varphi, r\sen\theta\cos\varphi, r\sen\varphi)\cdot r^2\cos\varphi =\\
        &= \dfrac{\sqrt{r^2\cos^2\theta\cos^2\varphi + r^2\sen^2\theta\cos^2\varphi}}{r^6} \cdot r^2\cos\varphi = \dfrac{\sqrt{r^2\cos^2\varphi}}{r^4}\cdot \cos\varphi =\\&= \dfrac{\cos^2\varphi}{r^3}
    \end{align*}

    Usando el Teorema de Cambio de Variable, tenemos que $g\in \cc{L}_1(A)$ si y solo si $h\in \cc{L}_1(E)$. Para estudiar la integrabilidad de $h$, usamos el Teorema de Tonelli:
    \begin{align*}
        \int_E h(r, \theta, \varphi)~d(r, \theta, \varphi) &=
        \int_1^{+\infty} \left( \int_{\nicefrac{-\pi}{2}}^{\nicefrac{\pi}{2}} \left( \int_0^{\nicefrac{\pi}{2}} \dfrac{\cos^2\varphi}{r^3}~d\varphi \right)~d\theta \right)~dr =\\
        &= \left(\int_1^{+\infty} \dfrac{1}{r^3}~dr\right) \left(\int_{\nicefrac{-\pi}{2}}^{\nicefrac{\pi}{2}}~d\theta\right) \left(\int_0^{\nicefrac{\pi}{2}} \cos^2\varphi~d\varphi\right) =\\
        &= \left[\dfrac{r^{-2}}{-2}\right]_1^{+\infty} \cdot \pi \cdot \int_0^{\nicefrac{\pi}{2}} \dfrac{1+\cos(2\varphi)}{2}~d\varphi =\\
        &= \dfrac{1}{2} \cdot \pi \cdot \left[\dfrac{\varphi}{2} + \dfrac{\sen(2\varphi)}{4}\right]_0^{\nicefrac{\pi}{2}} = \dfrac{\pi}{2} \cdot \dfrac{\pi}{4} = \dfrac{\pi^2}{8} < +\infty
    \end{align*}

    Por tanto, como $h\in \cc{L}_1(E)$, tenemos que $g\in \cc{L}_1(A)$, con:
    \begin{equation*}
        \int_A g(x, y, z)~d(x, y, z) = \int_E h(r, \theta, \varphi)~d(r, \theta, \varphi) = \dfrac{\pi^2}{8}
    \end{equation*}
\end{ejercicio}


\begin{ejercicio}[Incidencias DGIIM 22-23 y 23-24]
    Considerar el conjunto dado por $\Omega=\left\{\left(x,y\right)\in \bb{R}^2 \mid x^2 + y^2 < 1,~0<x<y\right\}$.
    Probar que la función
    $g:\Omega\to \bb{R}$ definida por
    \begin{equation*}
        g(x,y) = \dfrac{x\sqrt{x^2+y^2}\arctan\left(\dfrac{y}{x}\right)}{\sqrt{1+x^2+y^2}}\qquad \forall (x,y)\in \Omega
    \end{equation*}
    es integrable en $\Omega$ y calcular su integral.\\

    Sabemos que $\Omega$ es abierto, luego es medible y, por tanto, $\Omega\in \cc{M}_2$. Además, como $g$ es continua, tenemos que $g$ es medible.
    Para calcular su integral, haremos uso de coordenadas polares escribiendo:
    \begin{equation*}
        x = \rho\cos\theta, \quad y = \rho\sen\theta
    \end{equation*}
    con $\rho\in \bb{R}^+$ y $\theta\in \left]-\pi, \pi\right[$. Para ello, en primer lugar, obtenemos el conjunto $E$ de las coordenadas polares de los puntos de $\Omega$.
    \begin{align*}
        E &= \left\{ (\rho, \theta) \in \bb{R}^+\times \left]-\pi, \pi\right[ ~\mid \rho^2 < 1,~0<\rho\cos\theta<\rho\sen\theta \right\} =\\
        &= \left\{ (\rho, \theta) \in \bb{R}^+\times \left]-\pi, \pi\right[ ~\mid \rho < 1,~1<\tg\theta,~\cos\theta>0 \right\} =\\
        &= \left\{ (\rho, \theta) \in \bb{R}^+\times \left]-\pi, \pi\right[ ~\mid \rho < 1,~\arctan(1)<\theta,\cos\theta>0\right\} =\\
        &= \left]0, 1\right[\times \left]\nicefrac{\pi}{4}, \nicefrac{\pi}{2}\right[
    \end{align*}

    Definimos por tanto la función $h: E \to \bb{R}$ para cada $(\rho, \theta) \in E$ como:
    \begin{align*}
        h(\rho, \theta) &= \rho f(\rho\cos\theta, \rho\sen\theta)=\\
        &= \rho\cdot \dfrac{\rho\cos\theta\cdot \rho \cdot \arctan\left(\dfrac{\rho\sen\theta}{\rho\cos\theta}\right)}{\sqrt{1+\rho^2}}
        = \dfrac{\rho^3\cos\theta\arctan\left(\tg\theta\right)}{\sqrt{1+\rho^2}} = \dfrac{\rho^3\cos\theta\arctan\left(\sen\theta\right)}{\sqrt{1+\rho^2}} =\\
        &= \dfrac{\rho^3}{\sqrt{1+\rho^2}}\cdot \theta\cos\theta
    \end{align*}

    Usando el Teorema de Cambio de Variable, tenemos que $g\in \cc{L}_1(\Omega)$ si y solo si $h\in \cc{L}_1(E)$. Para estudiar la integrabilidad de $h$, usaremos el Teorema de Tonelli:
    \begin{align*}
        \int_E h(\rho, \theta)~d(\rho, \theta) &=
        \int_0^1 \left( \int_{\nicefrac{\pi}{4}}^{\nicefrac{\pi}{2}} \rho^3\theta\cos\theta~d\theta \right)~d\rho =\\
        &= \left( \int_0^1 \dfrac{\rho^3}{\sqrt{1+\rho^2}} ~d\rho \right) \left( \int_{\nicefrac{\pi}{4}}^{\nicefrac{\pi}{2}} \theta\cos\theta~d\theta \right)
    \end{align*}

    Para resolver la primera integral, tenemos dos opciones:
    \begin{description}
        \item[Opción 1.] Funciones hiperbólicas.
        
        Consideramos el seno hiperbólico, dado por $\senh(t)=\dfrac{e^t-e^{-t}}{2}$. Entonces, tenemos que:
        \begin{align*}
            \senh(x)=0 &\iff e^x-e^{-x}=0 \iff e^x = e^{-x} \iff x=0\\
            \senh(x)=1 &\iff e^x-e^{-x}=2 \iff e^x -\dfrac{1}{e^x}=2 \iff e^{2x}-2e^x-1=0 \iff \\&\qquad \iff e^x = \dfrac{2+\sqrt{4+4}}{2} = 1+\sqrt{2} \iff x = \ln(1+\sqrt{2})
        \end{align*}
        
        Aplicamos la versión elemental del Teorema de Cambio de Variable con $\varphi:\left[0,\ln(1+\sqrt{2})\right]\to [0,1]$ dada por $\varphi(t)=\senh(t)$, con $\varphi\in C^1\left([0,\ln(1+\sqrt{2})\right)$, $\varphi(0)=0$ y $\varphi\left(\ln(1+\sqrt{2})\right)=1$. Por tanto, por el Teorema de Cambio de Variable, tenemos que:
        \begin{align*}
            \int_0^1 \dfrac{\rho^3}{\sqrt{1+\rho^2}} ~d\rho &= \int_0^{\ln(1+\sqrt{2})} \dfrac{\senh^3(t)}{\sqrt{1+\senh^2(t)}}\cdot \cosh(t)~dt =\\
            &= \int_0^{\ln(1+\sqrt{2})} \dfrac{\senh^3(t)}{\sqrt{\dfrac{1}{\cosh^2(t)}}}\cdot \cosh(t)~dt
            = \int_0^{\ln(1+\sqrt{2})} \senh^3(t)\cosh^2(t)~dt =\\
            &= \int_0^{\ln(1+\sqrt{2})} \senh^3(t)\left(1+\senh^2(t)\right)~dt =\\
            &= \int_0^{\ln(1+\sqrt{2})} \senh^3(t) + \senh^5(t)~dt
        \end{align*}

        Se podría continuar por este camino resolviendo las dos integrales que nos han quedado mediante integración por partes,
        pero como vemos es un camino largo, por lo que buscaremos otras opciones.

        \item [Opción 2.] Cambio de variable trigonométrico.
        
        Aplicamos la versión elemental del Teorema de Cambio de Variable con $\varphi:\left[0,\nicefrac{\pi}{4}\right]\to [0,1]$ dada por $\varphi(t)=\tg(t)$, con $\varphi\in C^1\left([0,\nicefrac{\pi}{4}]\right)$, $\varphi(0)=0$ y $\varphi\left(\nicefrac{\pi}{4}\right)=1$. Por tanto, por el Teorema de Cambio de Variable, tenemos que:
        \begin{align*}
            \int_0^1 \dfrac{\rho^3}{\sqrt{1+\rho^2}} ~d\rho &= \int_0^{\nicefrac{\pi}{4}} \dfrac{\tg^3(t)}{\sqrt{1+\tg^2(t)}}\cdot \dfrac{1}{\cos^2(t)}~dt =\\
            &= \int_0^{\nicefrac{\pi}{4}} \dfrac{\tg^3(t)}{\sqrt{\dfrac{1}{\cos^2(t)}}}\cdot \dfrac{1}{\cos^2(t)}~dt
            = \int_0^{\nicefrac{\pi}{4}} \dfrac{\tg^3(t)}{\cos(t)}~dt =\\
            &= \int_0^{\nicefrac{\pi}{4}} \dfrac{\sen^3(t)}{\cos^4(t)}~dt
            = \int_0^{\nicefrac{\pi}{4}} \dfrac{\sen(t)\left(1-\cos^2(t)\right)}{\cos^4(t)}~dt =\\
            &= \int_0^{\nicefrac{\pi}{4}} \dfrac{\sen(t)}{\cos^4(t)} - \dfrac{\sen(t)}{\cos^2(t)}~dt
        \end{align*}

        Como vemos, en este caso la integral tampoco es sencilla de resolver, por lo que buscaremos otra opción.

        \item [Opción 3.] Versión elemental del Método de Integración por Partes.
        
        Aplicamos la versión elemental del Teorema de Cambio de Variable con las funciones $F,G:\bb{R}\to \bb{R}$ dadas por:
        \begin{equation*}
            F(t) = \rho^2, \quad G'(t) = \dfrac{\rho}{\sqrt{1+t^2}}
        \end{equation*}

        Entonces, tenemos que:
        \begin{align*}
            \int_0^1 \dfrac{\rho^3}{\sqrt{1+\rho^2}} ~d\rho &= \left[\rho^2\cdot \sqrt{1+\rho^2}\right]_0^1 - \int_0^1 2\rho\cdot \sqrt{1+\rho^2}~d\rho =\\
            &= \left[\rho^2\cdot \sqrt{1+\rho^2} - \dfrac{2}{3}\left(1+\rho^2\right)^{\nicefrac{3}{2}}\right]_0^1
            = \sqrt{2} - \dfrac{2}{3}\left(2^{\nicefrac{3}{2}}\right)+\dfrac{2}{3}
            =\\&= \sqrt{2} + \dfrac{2}{3}\left(-2\sqrt{2} + 1\right)
            = \dfrac{2-\sqrt{2}}{3}
        \end{align*}

        \item [Opción 4.] Sumando y restando $\rho$ en el numerador.
        \begin{align*}
            \int_0^1 \dfrac{\rho^3}{\sqrt{1+\rho^2}} ~d\rho &= \int_0^1 \dfrac{\rho^3+\rho-\rho}{\sqrt{1+\rho^2}} ~d\rho =\\
            &= \int_0^1 \dfrac{\rho^3+\rho}{\sqrt{1+\rho^2}} ~d\rho - \int_0^1 \dfrac{\rho}{\sqrt{1+\rho^2}} ~d\rho =\\
            &= \int_0^1 \rho\sqrt{1+\rho^2} ~d\rho - \int_0^1 \dfrac{\rho}{\sqrt{1+\rho^2}} ~d\rho =\\
            &= \left[\dfrac{1}{3}\left(1+\rho^2\right)^{\nicefrac{3}{2}}\right]_0^1 - \left[\sqrt{1+\rho^2}\right]_0^1 =\\
            &= \dfrac{1}{3}\left(2^{\nicefrac{3}{2}}-1\right) - \left(\sqrt{2}-1\right) = \dfrac{2-\sqrt{2}}{3}
        \end{align*}

        \item [Opción 5.] Cambio de variable especial.
        
        Aplicamos la versión elemental del Teorema de Cambio de Variable con la función $\varphi:\left[1,\sqrt{2}\right]\to [0,1]$ dada por $\varphi(t)=\sqrt{t^2-1}$.
        Tenemos que $\varphi\in~C^1\left([1,\sqrt{2}]\right)$, $\varphi(1)=0$ y $\varphi\left(\sqrt{2}\right)=1$. Por tanto, por el Teorema de Cambio de Variable, tenemos que:
        \begin{align*}
            \int_0^1 \dfrac{\rho^3}{\sqrt{1+\rho^2}} ~d\rho &= \int_1^{\sqrt{2}} \dfrac{(t^2-1)^{\nicefrac{3}{2}}}{\sqrt{t^2}}\cdot \dfrac{t}{\sqrt{t^2-1}}~dt =\\
            &= \int_1^{\sqrt{2}} t^2-1 ~dt
            = \left[\dfrac{t^3}{3}-t\right]_1^{\sqrt{2}} = \dfrac{2\sqrt{2}}{3}-\sqrt{2}-\dfrac{1}{3}+1 = \dfrac{2-\sqrt{2}}{3}
        \end{align*}
    \end{description}
    Como vemos, las dos primeras formas de resolver la integral son bastante complicadas, mientras que las tres últimas son bastante sencillas. Dejamos todas para que el lector pueda ver las diferentes formas de resolver la integral, y que tirar por una opción u otra puede hacer que la resolución de un problema sea mucho más sencilla.\\

    Retomando nuestro problema original, para calcular la segunda integral, usaremos la versión elemental del Método de Integración por Partes con las funciones $F,G:\bb{R}\to \bb{R}$ dadas por:
    \begin{equation*}
        F(t) = \theta, \quad G'(t) = \cos\theta
    \end{equation*}

    Tenemos que:
    \begin{align*}
        \int_{\nicefrac{\pi}{4}}^{\nicefrac{\pi}{2}} \theta\cos\theta~d\theta &= \left[\theta\sen\theta\right]_{\nicefrac{\pi}{4}}^{\nicefrac{\pi}{2}} - \int_{\nicefrac{\pi}{4}}^{\nicefrac{\pi}{2}} \sen\theta~d\theta =\\
        &= \left[\theta\sen\theta\right]_{\nicefrac{\pi}{4}}^{\nicefrac{\pi}{2}} + \left[\cos\theta\right]_{\nicefrac{\pi}{4}}^{\nicefrac{\pi}{2}} = \dfrac{\pi}{2} - \dfrac{\pi}{4}\cdot \dfrac{\sqrt{2}}{2} -\dfrac{\sqrt{2}}{2}
        = \dfrac{\pi}{2} - \dfrac{\sqrt{2}}{2}\left(\dfrac{\pi}{4}+1\right)
    \end{align*}


    Por tanto, tenemos que:
    \begin{align*}
        \int_{\Omega} h(\rho, \theta)~d(\rho, \theta) &=
        \dfrac{2-\sqrt{2}}{3} \cdot \left(\dfrac{\pi}{2} - \dfrac{\sqrt{2}}{2}\left(\dfrac{\pi}{4}+1\right)\right)
    \end{align*}

    Por tanto, como $h\in \cc{L}_1(E)$, tenemos que $g\in \cc{L}_1(\Omega)$, con:
    \begin{equation*}
        \int_{\Omega} g(x, y)~d(x, y) = \int_{\Omega} h(\rho, \theta)~d(\rho, \theta) = \dfrac{2-\sqrt{2}}{3} \cdot \left(\dfrac{\pi}{2} - \dfrac{\sqrt{2}}{2}\left(\dfrac{\pi}{4}+1\right)\right)
    \end{equation*}
    

\end{ejercicio}
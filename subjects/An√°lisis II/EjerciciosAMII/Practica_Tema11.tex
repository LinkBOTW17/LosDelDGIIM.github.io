\section{Cálculo de Integrales Simples}

\subsection{Repaso teórico}
Repasamos ahora los conceptos teóricos necesarios para realizar el cálculo de integrales simples. Destacamos 4 reglas que nos ayudan a resolver el cálculo de integrales, todas ellas tendrán su versión elemental\footnote{Las cuales ya se vieron en Cálculo II.} y su versión general.

\subsubsection{Regla de Barrow}
\begin{teo}[Regla de Barrow versión elemental]\ \\
    Si $f:J\rightarrow\mathbb{R}$ es una función continua y $G$ una primitiva de $f$, se tiene:
    \begin{equation*}
        \int_{a}^{b} f(x)~dx  = G(b) - G(a) = {[G(x)]}_a^b\qquad \forall a,b\in J
    \end{equation*}
\end{teo}

\begin{teo}[Versión general de la Regla de Barrow]\ \\
    Si $f\in \cc{L}_1(J)$ y $G:J\rightarrow\mathbb{R}$ es una primitiva de $f$, entonces $G$ tiene límite, tanto en $\alpha$ como el $\beta$, y se verifica que:
    \begin{equation*}
        \displaystyle\int_{\alpha}^{\beta} f(t)~dt = \lim_{x\to\beta} G(x) -\lim_{x\to\alpha} G(x) = {[G(x)]}_\alpha^\beta
    \end{equation*}
\end{teo}
\begin{coro}
    Si $f\in \cc{L}_1^{\text{loc}}(J)$ y $G$ una primitiva de $f$, entonces se tiene:
    \begin{equation*}
        \int_{a}^{b} f(x)~dx  = G(b) - G(a) = {[G(x)]}_a^b\qquad \forall a,b\in J
    \end{equation*}
\end{coro}
Notemos que ninguno de los dos teoremas anteriores nos permite averiguar si $f$ es integrable o no, ya que suponen que lo es para llegar a la tesis. A continuación, vemos un criterio que nos permite comprobar esto, como consecuencia de la versión general de la regla de Barrow.

\begin{teo}[Criterio de integrabilidad]\ \\
    Dada una función $f:J\rightarrow\red{\mathbb{R}^+_0}$, sea $G$ una primitiva de $f$. Entonces $f\in \cc{L}_1$ si, y sólo si, $G$ tiene límite en $\alpha$ y $\beta$, en cuyo caso:
    \begin{equation*}
        \displaystyle\int_{\alpha}^{\beta} f(t)~dt  = {[G(x)]}_\alpha^\beta
    \end{equation*}
\end{teo}
Notemos que sólo es válido para funciones con codominio $\mathbb{R}^+_0$. Sin embargo, si tenemos una función con imagen negativa $f$ de forma que $-f$ (función con imagen positiva) cumpla las hipótesis del criterio, $-f$ será integrable, luego $f$ también. De esta forma, si tenemos una función que pasa de ser positiva a negativa (o viceversa) un número finito de veces, podemos, en cada trozo donde el signo de su imagen es constante, aplicar el criterio, obteniendo que la función es integrable (en caso de que cada uno de sus ``trozos'' cumpla con las hipótesis del criterio).

\subsubsection{Criterio de comparación}
A continuación, desarrollaremos el criterio de comparación con paso al límite, que nos permite averiguar si una función dada es integrable o no, comparándola con funciones conocidas. Para poder realizar dicho estudio, es necesario desarrollar un ``catálogo'' de funciones conocidas.

\begin{ejercicio*}
    Dados $s\in \mathbb{R}$ y $c\in \mathbb{R}^+$, la función $x\mapsto x^s$ es:
    \begin{itemize}
        \item $\cc{L}_1^{\text{loc}}(\mathbb{R}^+)$.
        \item integrable en $]0,c[$ si, y sólo si, $s > -1$.
        \item integrable en $]c,+\infty[$ si, y sólo si, $s < -1$.
    \end{itemize}
\end{ejercicio*}

\begin{ejercicio*}
    Dados $s\in \mathbb{R}^*$ y $c\in \mathbb{R}$, la función $x\mapsto e^{sx}$ es:
    \begin{itemize}
        \item $\cc{L}_1^{\text{loc}}(\mathbb{R})$.
        \item integrable en $]-\infty, c[$ si, y sólo si, $s>0$.
        \item integrable en $]c, +\infty[$ si, y sólo si, $s<0$.
    \end{itemize}
\end{ejercicio*}

\begin{ejercicio*}
    Sea $\rho \in \mathbb{R}^+$, la función $x\mapsto e^{-\rho|x|}\in \cc{L}_1(\mathbb{R})$.
\end{ejercicio*}

\begin{observacion}
    Si $f$ y $g$ son dos funciones medibles tales que $g\in \cc{L}_1(A)$ para un cierto $A\subset \mathbb{R}$ y se tiene que $|f|\leq |g|$, entonces $f\in \cc{L}_1(A)$.
    
\end{observacion}
\begin{proof}
    \begin{equation*}
        \int_A |f| \leq \int_A |g| < \infty
    \end{equation*}
\end{proof}


\begin{prop}[Criterio de comparación por paso al límite]\ \\
    Sea $I=[a, \beta[$ con $a\in \mathbb{R}$ y $a<\beta \leq +\infty$ y $f,g\in \cc{L}_1^{\text{loc}}(I)$ con $g(x)\neq 0$ para todo $x\in I$
    \begin{itemize}
        \item Si $\lim\limits_{x\to\beta}\dfrac{|f(x)|}{|g(x)|} = L \in \mathbb{R}^+ $, entonces $f\in \cc{L}_1(I) \Longleftrightarrow g\in \cc{L}_1(I)$
        \item Si $\lim\limits_{x\to\beta} \dfrac{|f(x)|}{|g(x)|} = 0 $, entonces $g\in \cc{L}_1(I)\Longrightarrow f\in \cc{L}_1(I)$
        \item Si $\dfrac{|f(x)|}{|g(x)|}\rightarrow +\infty \ (x\rightarrow\beta)$, entonces: $g\notin \cc{L}_1(I) \Longrightarrow f\notin\cc{L}_1(I)$
    \end{itemize}
    En el caso $I=\left]\alpha,b\right]$ con $b\in \mathbb{R}$ y $-\infty\leq \alpha < b$, se verifica el resultado análogo, con $\alpha$ en lugar de $\beta$.
\end{prop}

\subsubsection{Integración por partes}
Este método es muy útil para calcular integrales de funciones que ya sabemos que son integrables, pero no nos da una forma de comprobar que lo sean.
\begin{teo}[Fórmula de integración por partes versión elemental]\ \\
    Si $F,G:J\rightarrow\mathbb{R}$ son funciones de clase $C^1$ en $J$, se tiene:
    \begin{equation*}
        \displaystyle\int_{a}^{b} F(t)G'(t)~dt  = {[F(x)G(x)]}_a^b - \displaystyle\int_{a}^{b} F'(t)G(t)~dt  \qquad a,b\in J
    \end{equation*}
\end{teo}

A continuación, destacamos el resultado general, que nombramos como ``primera versión''.

\begin{teo}[Fórmula de integración por partes (primera versión general)]\ \\
    Dadas $F,G:J\rightarrow\mathbb{R}$, supongamos que, para cada intervalo compacto $K\subset J$ las restricciones $F_{\big|K}$ y $G_{\big|K}$ son absolutamente continuas. Entonces, $FG'$ y $GF'$ son localmente integrables en $J$ con:
    \begin{equation*}
        \displaystyle\int_{a}^{b} F(t)G'(t)~dt = {[F(x)G(x)]}_a^b - \displaystyle\int_{a}^{b} F'(t)G(t)~dt \qquad a,b\in J
    \end{equation*}
\end{teo}

Como podemos ver, pese a ser satisfactorio teóricamente, no nos aporta utilidad práctica. Por ello, destacamos el siguiente teorema, más débil pero de gran utilidad práctica.

\begin{teo}[Fórmula de Integración por partes (segunda versión general)]\ \\
    Sean $F,G:J\rightarrow\mathbb{R}$ dos funciones derivables en $J=\left]\alpha,\beta\right[$, tales que $F'G$ y $FG'$ son integrables en $J$. Entonces, $FG$ tiene límite, tanto en $\alpha$ como en $\beta$, y se verifica que:
    \begin{equation*}
        \displaystyle\int_{\alpha}^{\beta} F(t)G'(t)~dt = {[F(x)G(x)]}_\alpha^\beta - \displaystyle\int_{\alpha}^{\beta} F'(t)G(t)~dt 
    \end{equation*}
\end{teo}

\subsubsection{Cambio de variable}
\begin{teo}[Cambio de variable versión elemental]\ \\
    Dados dos intervalos no triviales $I,J\subset \mathbb{R}$, sea $\varphi:I\rightarrow J$ una función de clase $C^1$ en $I$ y $f:J\rightarrow\mathbb{R}$ una función continua. Entonces:
    \begin{equation*}
        \displaystyle\int_{\varphi(a)}^{\varphi(b)} f(x)~dx = \displaystyle\int_{a}^{b} f(\varphi(t))\varphi'(t)~dt\qquad \forall a,b\in I 
    \end{equation*}
\end{teo}

\begin{teo}[Fórmula de cambio de variable]\ \\
    Dados dos intervalos no triviales $I,J\subset \mathbb{R}$, sea $\varphi:I\rightarrow J$ tal que $\varphi_{\big|H}$ es absolutamente continua, para todo intervalo compacto $H\subset I$.
    Si $f:I\rightarrow\mathbb{R}$ es localmente integrable en $J$ y verifica que $(f\circ \varphi)\varphi'$ es localmente integrable en $I$, entonces:
    \begin{equation*}
        \displaystyle\int_{\varphi(a)}^{\varphi(b)} f(x)~dx = \displaystyle\int_{a}^{b} f(\varphi(t))\varphi'(t)~dt\qquad \forall a,b\in I
    \end{equation*}
\end{teo}

Sin embargo, al igual que sucedía antes, carece de utilidad práctica (no nos dice si alguna de las dos es integrable o no, y las integrales siguen siendo en conjuntos compactos). Vemos ahora otro segundo teorema que podemos usar, que además caracteriza la integabilidad de la función $f$ con la función $(f\circ \varphi)\varphi'$. 

\begin{teo}[Teorema de cambio de variable]\ \\
    Dado un intervalo abierto no vacío $I\subset \mathbb{R}$, sea $\varphi:I\rightarrow\mathbb{R}$ una función de clase $C^1$ en $I$, con $\varphi'(t)\neq 0$ para todo $t\in I$, y sea $J=\varphi(I)$.\newline
    Entonces, una función $f:J\rightarrow\mathbb{R}$ es integrable en $J$ si, y sólo si, $(f\circ \varphi)\varphi'$ es integrable en $I$, en cuyo caso:
    \begin{equation*}
        \int_J f = \int_I (f\circ\varphi)|\varphi'|
    \end{equation*}
\end{teo}

Este teorema suele usarse de la siguiente forma:\newline
$I=\left]\alpha,\beta\right[$ con $-\infty\leq \alpha<\beta\leq +\infty$ y $J=\left]\gamma,\delta\right[$ con $-\infty\leq \gamma<\delta\leq +\infty$\newline
$\{\gamma,\delta\} = \{\wt{\alpha},\wt{\beta}\}$ con $\varphi(t)\to\wt{\alpha}$ $(t\to\alpha)$ y $\varphi(t)\to\wt{\beta}$ $(t\to\beta)$\newline
Entonces:
\begin{equation*}
    \displaystyle\int_{\wt{\alpha}}^{\wt{\beta}} f(x)~dx = \displaystyle\int_{\alpha}^{\beta} f(\varphi(t))\varphi'(t)~dt 
\end{equation*}












\newpage
\subsection{Ejercicios}
\begin{ejercicio}
    En cada uno de los siguientes casos, probar que la función $f$ es integrable en el intervalo $J$ y calcular su integral:
    \begin{enumerate}
        \item $f(x)=x^2\ln x \qquad \forall x\in J=\left]0,1\right[$.
        
        Veamos varias formas de demostrar que $f\in \cc{L}_1(J)$.
        \begin{description}
            \item[Opción 1.] Extendiendo la función a $\ol{J}$.
            
            Veamos el valor de los siguientes límites:
            \begin{align*}
                \lim_{x\to 0} f(x) &= \lim_{x\to 0} x^2\ln x = \lim_{x\to 0} \dfrac{\ln x}{x^{-2}} \Hop \lim_{x\to 0} \dfrac{\frac{1}{x}}{-2x^{-3}} = \lim_{x\to 0} -\dfrac{x^2}{2} = 0\\
                \lim_{x\to 1} f(x) &= \lim_{x\to 0} x^2\ln x = 0
            \end{align*}

            Por tanto, tenemos que $f$ tiene límite en $0$ y $1$, por lo que extendemos $f$ a $\ol{J}$ de forma continua.
            Sea $\ol{f}:\ol{J}\to \bb{R}$ su extensión continua. Tenemos que $\ol{f}$ es continua en un compacto,
            luego $\ol{f}\in \cc{L}_1\left(\ol{J}\right)$. Por tanto, restringuiendo a $J$, tenemos que $f\in \cc{L}_1(J)$.

            \item[Opción 2.] Criterio de comparación.
            
            Sea $\wt{f},g:]0,1] \to \bb{R}$ dadas por $\wt{f}(x)=f(x)$ para todo $x\in J$, $\wt{f}(1)=0$ y $g(x)=x$ para todo $x\in~]0,1]$. Tenemos que
            $\wt{f},g$ son continuas en $]0,1]$, por lo que $\wt{f},g\in \cc{L}^{\text{loc}}_1(]0,1])$. Además, $g(x)\neq 0$ para todo $x\in~]0,1]$. Estamos entonces en las hipótesis del Criterio de Comparación:
            \begin{equation*}
                \lim_{x\to 0} \dfrac{|\wt{f}|}{g}
                = \lim_{x\to 0} \dfrac{-x^2\ln x}{x}
                = \lim_{x\to 0} -x\ln x
                \Hop \lim_{x\to 0} \dfrac{\frac{1}{x}}{\frac{1}{x^2}}
                = \lim_{x\to 0} x = 0
            \end{equation*}
            Como $g\in \cc{L}_1(]0,1])$, tenemos que $\wt{f}\in \cc{L}_1(]0,1])$, y por tanto, $f\in \cc{L}_1(J)$.
        \end{description}
        
        Para calcular la integral buscada, empleamos la integración por partes.
        Sean $F,G:J\to \bb{R}$ las funciones dadas por:
        \begin{equation*}
            F(x)=\ln x \hspace{1cm} G(x)=\dfrac{x^3}{3}
        \end{equation*}

        Tenemos que $F,G$ son derivables en $J$, con:
        \begin{equation*}
            F'(x)=\dfrac{1}{x} \hspace{1cm} G'(x)=x^2
        \end{equation*}

        Como hemos demostrado, $FG'=f\in \cc{L}_1(J)$. Además, $F'G=\frac{x^2}{3}\in \cc{L}_1(J)$ por ser una función continua en $\ol{J}$.
        Por tanto, por el Teorema de Integración por Partes, tenemos que:
        \begin{equation*}
            \int_0^1 x^2\ln x~dx
            = \left[\dfrac{x^3}{3}\ln x\right]_0^1 - \int_0^1 \dfrac{x^3}{3}\dfrac{1}{x}~dx
            \AstIg \frac{1}{3}\left[x^3\ln x-\frac{x^3}{3}\right]_0^1
            = \frac{1}{3}\left[-\frac{1}{3}\right] = -\frac{1}{9}
        \end{equation*}
        donde en $(\ast)$ hemos empleado la versión elemental de la Regla de Barrow. Además, en la última igualdad, hemos usado que:
        \begin{equation*}
            \lim_{x\to 0} x^3\ln x = \lim_{x\to 0} \dfrac{\ln x}{x^{-3}} \Hop \lim_{x\to 0} \dfrac{\frac{1}{x}}{-3x^{-4}} = \lim_{x\to 0} -\dfrac{x^3}{3} = 0
        \end{equation*}

        Veamos ahora otra forma de demostrar que $f\in \cc{L}_1(J)$.
        \begin{description}
            \item[Opción 3.] Uso del Criterio de Integrabilidad. 
            
            Probaremos ahora que $f\in \cc{L}_1(J)$.
            Probemos que $G$ es una primitiva de $f$, donde:
            \Func{G}{J}{\bb{R}}{x}{\dfrac{1}{3}\left[x^3\ln x-\dfrac{x^3}{3}\right]}
            
            Tenemos que $G$ es derivable en $J$, con:
            \begin{equation*}
                G'(x) = \dfrac{1}{3}\left[3x^2\ln x + x^2 - x^2\right] = x^2\ln x = f(x) \qquad \forall x\in J
            \end{equation*}

            Por tanto, tenemos que $G$ es una primitiva de $f$, y hemos visto que $G$ tiene límite en $0$ y $1$.
            Empleando el Criterio de Integrabilidad de la Regla de Barrow para $-f$, tenemos que $f\in \cc{L}_1(J)$, como queríamos demostrar.

            \begin{observacion}
                Notemos que $G$ la hemos deducido a partir del método de integración por partes, para lo cual hemos usado que $f$ era integrable.
                Notemos que el proceso sería aun así correcto, puesto que en esta segunda opción demostramos
                que, efectivamente, $G$ es una primitiva de $f$. Se podría haber escrito esta segunda opción al principio,
                y entonces $G$ podríamos decir que habría sido una intuición o, usando la 
                expresión empleada en clase, una primitiva que ``nos ha caído de la chimenea''.
            \end{observacion}
        \end{description}

        \item $f(x)=e^{-x}\cos(2x) \qquad \forall x\in J=\bb{R}^+$.
        
        Sabiendo que la función $x\mapsto e^{-x}$ es integrable en $\bb{R}^+$, tenemos que:
        \begin{equation*}
            \int_{\bb{R}^+} |f| \leq \int_{\bb{R}^+} e^{-x} < +\infty
        \end{equation*}

        Por tanto, tenemos que $f\in \cc{L}_1(\bb{R}^+)$. Para calcular su integral, usamos la fórmula de integración por partes.
        Sean $F,G:\bb{R}^+\to \bb{R}$ las funciones dadas por:
        \begin{equation*}
            F(x) = \cos(2x) \hspace{1cm} G(x) = -e^{-x}
        \end{equation*}

        Tenemos que $F,G$ son derivables en $\bb{R}^+$, con:
        \begin{equation*}
            F'(x) = -2\sen(2x) \hspace{1cm} G'(x) = e^{-x}
        \end{equation*}

        Tenemos que $FG' = f\in \cc{L}_1(\bb{R}^+)$ como hemos visto antes, y $F'G = 2e^{-x}\sen(2x)\in \cc{L}_1(\bb{R}^+)$ por el mismo razonamiento que $f$.
        Por tanto, por la fórmula de integración por partes, tenemos que:
        \begin{align*}
            \int_0^{+\infty} e^{-x}\cos(2x)~dx
            &= \left[-e^{-x}\cos(2x)\right]_0^{+\infty} - \int_0^{+\infty} 2e^{-x}\sen(2x)~dx
            =\\&= \left[-e^{-x}\cos(2x)\right]_0^{+\infty} - 2\int_0^{+\infty} e^{-x}\sen(2x)~dx
        \end{align*}

        Usamos ahora la fórmula de integración por partes para la integral que nos queda.
        Sean $G_1=G$ y $F_1:\bb{R}^+\to \bb{R}$ la función dada por $F_1(x) = \sen(2x)$.
        Tenemos que $F_1,G_1$ son derivables en $\bb{R}^+$, con:
        \begin{equation*}
            F_1'(x) = 2\cos(2x) \hspace{1cm} G_1'(x) = -e^{-x}
        \end{equation*}

        Tenemos entonces que:
        \begin{align*}
            \int_0^{+\infty}& e^{-x}\cos(2x)~dx
            = \left[-e^{-x}\cos(2x)\right]_0^{+\infty} - 2\int_0^{+\infty} e^{-x}\sen(2x)~dx
            =\\&= \left[-e^{-x}\cos(2x)\right]_0^{+\infty} - 2\left[-e^{-x}\sen(2x)\right]_0^{+\infty} - 4\int_0^{+\infty}e^{-x}\cos(2x)~dx
            =\\&= \left[-e^{-x}(\cos(2x)-2\sen(2x))\right]_0^{+\infty} - 4\int_0^{+\infty}e^{-x}\cos(2x)~dx
        \end{align*}

        Despejando el valor de la integral buscada, obtenemos:
        \begin{equation*}
            \int_0^{+\infty} e^{-x}\cos(2x)~dx = \frac{1}{5}\left[-e^{-x}(\cos(2x)-2\sen(2x))\right]_0^{+\infty} \AstIg \frac{1}{5}\cdot (0+1) = \frac{1}{5}
        \end{equation*}
        donde en $(\ast)$ hemos usado los siguientes límites:
        \begin{equation*}
            \lim_{x\to \infty} -e^{-x}(\cos(2x)+2\sen(2x)) = 0
            \hspace{1cm}
            \lim_{x\to 0} -e^{-x}(\cos(2x)+2\sen(2x)) = -e^0=1
        \end{equation*}


        \item $f(x)=\dfrac{1}{x^4-1} \qquad \forall x\in J=\left]2,+\infty\right[$.
        
        Sea $\wt{f}:\ol{J}\to \bb{R}$ dada por $\wt{f}(x)=f(x)$ para todo $x\in J$ y $\wt{f}(2)=\nicefrac{1}{15}$.
        Sea $g:\ol{J}\to \bb{R}$ dada por $g(x)=x^{-4}$ para todo $x\in \ol{J}$. Tenemos que $\wt{f},g\in \cc{L}_1^{\text{loc}}\left(\ol{J}\right)$ por ser continuas.
        Además, $g(x)\neq 0$ para todo $x\in \ol{J}$. Estamos entonces en las hipótesis del Criterio de Comparación:
        \begin{equation*}
            \lim_{x\to \infty} \dfrac{|\wt{f}|}{|g|} = \lim_{x\to \infty} \dfrac{\nicefrac{1}{x^4-1}}{\nicefrac{1}{x^4}} = \lim_{x\to \infty} \dfrac{x^4}{x^4-1} = 1
        \end{equation*}

        Como $g\in \cc{L}_1\left(\ol{J}\right)$, tenemos que $\wt{f}\in \cc{L}_1\left(\ol{J}\right)$, y por tanto, $f\in \cc{L}_1(J)$.\\

        Aplicamos ahora la descomposisión en fracciones simples de $f$ para calcular su integral.
        Tenemos que $x^4-1=(x^2-1)(x^2+1)=(x-1)(x+1)(x^2+1)$, por lo que, siendo $A,B,C,D\in \bb{R}$, tenemos que:
        \begin{align*}
            \dfrac{1}{x^4-1}
            &= \dfrac{A(x+1)(x^2+1) + B(x-1)(x^2+1) + (Cx+D)(x-1)(x+1)}{(x-1)(x+1)(x^2+1)}
            =\\&= \dfrac{A}{x-1} + \dfrac{B}{x+1} + \dfrac{Cx+D}{x^2+1}
        \end{align*}

        Igualando el denominador, tenemos:
        \begin{itemize}
            \item \ul{$x=1$}: Tenemos $1=4A$, por lo que $A=\nicefrac{1}{4}$.
            \item \ul{$x=-1$}: Tenemos $1=-4B$, por lo que $B=\nicefrac{-1}{4}$.
            \item \ul{$x=0$}: Tenemos $1=A-B-D$, por lo que $D=A-B-1=\nicefrac{-1}{2}$.
            \item \ul{$x=-2$}: Tenemos $1=-5A -15B+3(-2C+D)$, por lo que:
            \begin{equation*}
                C = -\frac{1}{6}\left(1+5A+15B-3D\right)
                = -\frac{1}{6}\left(1+5\cdot \frac{1}{4}-15\cdot \frac{1}{4}+3\cdot \frac{1}{2}\right)=0
            \end{equation*}
        \end{itemize}

        Por tanto, y usando la linealidad de la integral, tenemos que:
        \begin{align*}
            \int_2^{+\infty} \dfrac{1}{x^4-1}~dx
            &= \int_2^{+\infty} \dfrac{A}{x-1}~dx
            + \int_2^{+\infty} \dfrac{B}{x+1}~dx
            + \int_2^{+\infty} \dfrac{Cx+D}{x^2+1}~dx
            =\\&= \frac{1}{4}\int_2^{+\infty} \dfrac{1}{x-1}~dx
            - \frac{1}{4}\int_2^{+\infty} \dfrac{1}{x+1}~dx
            - \frac{1}{2}\int_2^{+\infty} \dfrac{1}{x^2+1}~dx
            \AstIg\\&\AstIg \frac{1}{4}\left[\ln(x-1)\right]_2^{+\infty}
            - \frac{1}{4}\left[\ln(x+1)\right]_2^{+\infty}
            - \frac{1}{2}\left[\arctan x\right]_2^{+\infty}
            =\\&= \frac{1}{4}\left[\ln(x-1) - \ln(x+1)\right]_2^{+\infty}
            - \frac{1}{2}\left[\arctan x\right]_2^{+\infty}
            =\\&= \frac{1}{4}\left[\ln\left(\frac{x-1}{x+1}\right)\right]_2^{+\infty}
            - \frac{1}{2}\left[\arctan x\right]_2^{+\infty}
            =\\&= \frac{1}{4}\left[0 - \ln\left(\frac{1}{3}\right)\right] - \frac{1}{2}\left[\frac{\pi}{2} - \arctan 2\right]
            =\\&= \frac{1}{4}\ln 3 - \frac{\pi}{4} + \frac{1}{2}\arctan 2 
        \end{align*}
        donde en $(\ast)$ empleamos la versión general de la Regla de Barrow. Además, también se usa que:
        \begin{equation*}
            \lim_{x\to +\infty} \ln\left(\frac{x-1}{x+1}\right)
            = \ln\left(\lim_{x\to +\infty}\frac{x-1}{x+1}\right)
            = \ln(1) = 0
        \end{equation*}
        donde ahora hemos empleado que la función logaritmo es continua.
        


        \item $f(x)=\dfrac{1}{e^x+e^{-x}} \qquad \forall x\in J=\bb{R}$.
        
        Sea el cambio de variable $x=\varphi(t)$ dado por $\varphi:\bb{R}^+\to \bb{R}$ dada por $\varphi(t)=\ln t$.
        Tenemos que $\varphi\in C^1(\bb{R}^+)$, $\varphi$ es biyectiva y $\varphi'(t)\neq 0$ para todo $t\in \bb{R}^+$.
        Además, tenemos que:
        \begin{equation*}
            \lim_{t\to 0} \varphi(t) = -\infty \hspace{1cm} \lim_{t\to +\infty} \varphi(t) = +\infty
        \end{equation*}

        Por tanto, estamos en las hipótesis del Teorema de Cambio de Variable, por lo que $f\in \cc{L}_1(\bb{R})$ si, y sólo si, $(f\circ \varphi)\varphi'$ es integrable en $\bb{R}^+$.
        Tenemos que:
        \begin{equation*}
            (f\circ \varphi)(t)\cdot \varphi'(t)
            = \dfrac{1}{e^{\ln t}+e^{-\ln t}}\cdot \dfrac{1}{t}
            = \dfrac{1}{t + \nicefrac{1}{t}}\cdot \dfrac{1}{t}
            = \dfrac{1}{t^2 + 1}
        \end{equation*}

        Por tanto, tenemos que $(f\circ \varphi)\varphi'$ es integrable en $\bb{R}^+$, por lo que $f\in \cc{L}_1(\bb{R})$. Además, tenemos que:
        \begin{equation*}
            \int_{-\infty}^{+\infty} \dfrac{1}{e^x+e^{-x}}~dx
            = \int_0^{+\infty} \dfrac{1}{t^2+1}~dt
            = \left[\arctan t\right]_0^{+\infty}
            = \frac{\pi}{2}
        \end{equation*}


        \item $f(x)=\dfrac{1}{x^2 + \sqrt{x}} \qquad \forall x\in J=\left]0,1\right[$.
        
        Sea el cambio de variable $x=\varphi(t)$ dado por $\varphi:]0,1[\to \bb{R}$ dada por $\varphi(t)=t^2$ para todo $t\in ]0,1[$.
        Tenemos que $\varphi\in C^1(]0,1[)$, $\varphi$ es biyectiva y $\varphi'(t)\neq 0$ para todo $t\in ]0,1[$. Además, tenemos que:
        \begin{equation*}
            \lim_{t\to 0} \varphi(t) = 0 \hspace{1cm} \lim_{t\to 1} \varphi(t) = 1
        \end{equation*}

        Por tanto, estamos en las hipótesis del Teorema de Cambio de Variable, por lo que $f\in \cc{L}_1(]0,1[)$ si, y sólo si, $(f\circ \varphi)\varphi'$ es integrable en $]0,1[$.
        Tenemos que:
        \begin{equation*}
            (f\circ \varphi)(t)\cdot \varphi'(t)
            = \dfrac{1}{t^4 + \sqrt{t^2}}\cdot 2t
            = \dfrac{2t}{t^4 + t}
            = \dfrac{2}{t^3 + 1}
        \end{equation*}

        Tenemos que la extensión continua de $f$ a $\ol{J}$ es integrable por ser continua en un compacto, por lo que $f\in \cc{L}_1(J)$. Además, tenemos que:
        \begin{equation*}
            \int_0^1 \dfrac{1}{x^2 + \sqrt{x}}~dx
            = \int_0^1 \dfrac{2}{t^3 + 1}~dt
            = 2\cdot \int_0^1 \dfrac{1}{t^3 + 1}~dt
        \end{equation*}

        Para calcular la integral, descomponemos en fracciones simples. Para ello, en primer lugar hallamos las raíces del denominador:
        \begin{figure}[H]
            \centering
            \polyhornerscheme[x=-1]{x^3+1}
        \end{figure}

        Por tanto, $t^3+1=(t+1)(t^2-t+1)$. Además, no tiene más raíces reales,
        puesto que el discriminante del segundo factor es
        $\Delta = 1^2-4\cdot 1\cdot 1 = -3 < 0$. Entonces, queda:
        \begin{align*}
            \dfrac{1}{t^3 + 1}
            &= \dfrac{A}{t+1} + \dfrac{Bt+C}{t^2-t+1}
            = \dfrac{A(t^2-t+1) + (Bt+C)(t+1)}{(t+1)(t^2-t+1)}
            \qquad A,B,C\in \bb{R}
        \end{align*}

        Igualando el numerador, tenemos:
        \begin{itemize}
            \item \ul{$t=-1$}: Tenemos $1=3A$, por lo que $A=\nicefrac{1}{3}$.
            \item \ul{$t=0$}: Tenemos $1=A+C$, por lo que $C=1-A=\nicefrac{2}{3}$.
            \item \ul{$t=1$}: Tenemos $1=A+2(B+C)=A+2B+2C$, por lo que:
            \begin{equation*}
                B = \frac{1-A-2C}{2} = \frac{1-\nicefrac{1}{3}-2\cdot \nicefrac{2}{3}}{2} = -\frac{1}{3}
            \end{equation*}
        \end{itemize}

        Por tanto, y usando la linealidad de la integral, tenemos que:
        \begin{align*}
            \int_0^1 \dfrac{1}{x^2 + \sqrt{x}}~dx
            &= 2\cdot \int_0^1 \dfrac{1}{t^3 + 1}~dt
            = \frac{2}{3}\cdot \int_0^1 \frac{1}{t+1}~dt
            + \frac{2}{3}\cdot \int_0^1 \frac{-t+2}{t^2-t+1}~dt
        \end{align*}

        Resolvemos ahora la nueva integral que tenemos, cuyo denominador tiene raíces complejas.
        \begin{align*}
            \int_0^1 \dfrac{-t+2}{t^2-t+1}~dt
            &= \red{-\frac{1}{2}} \int_0^1 \dfrac{\red{2}t \red{-4}}{t^2-t+1}~dt
            = -\frac{1}{2}\int_0^1 \dfrac{2t-1}{t^2-t+1}~dt
            -\frac{1}{2}\int_0^1 \dfrac{-3}{t^2-t+1}~dt
            =\\&= -\frac{1}{2}\left[\ln|t^2-t+1|\right]_0^1
            +\frac{3}{2}\int_0^1 \dfrac{1}{t^2-t+1}~dt
        \end{align*}

        Buscamos ahora encontrar un binomio al cuadrado en el denominador para resolver esta última integral. Para ello, completamos cuadrados:
        \begin{align*}
            t^2-t+1&=\left(t-\frac{1}{2}\right)^2+\frac{3}{4}
            = \frac{3}{4}\left[1+\frac{4}{3}\left(t-\frac{1}{2}\right)^2\right]
            = \frac{3}{4}\left[1+\left(\frac{2}{\sqrt{3}}\left(t-\frac{1}{2}\right)\right)^2\right]
        \end{align*}

        Por tanto, tenemos que:
        \begin{align*}
            \int_0^1 \dfrac{1}{t^2-t+1}~dt
            &= \int_0^1 \dfrac{4}{3}\cdot \dfrac{1}{1+\left(\frac{2}{\sqrt{3}}\left(t-\frac{1}{2}\right)\right)^2}~dt
            =\\&= \dfrac{4}{3}\cdot \red{\dfrac{\sqrt{3}}{2}}\cdot \int_0^1 \dfrac{\red{\dfrac{2}{\sqrt{3}}}}{1+\left(\frac{2}{\sqrt{3}}\left(t-\frac{1}{2}\right)\right)^2}~dt
            =\\&= \frac{2\sqrt{3}}{3}\cdot \left[\arctan\left(\frac{2}{\sqrt{3}}\left(t-\frac{1}{2}\right)\right)\right]_0^1
            = \frac{2\sqrt{3}}{9}\cdot \pi
        \end{align*}

        Resolviendo ahora los resultados que hemos obtenido, tenemos que:
        \begin{align*}
            \int_0^1 &\dfrac{1}{x^2 + \sqrt{x}}~dx
            = \frac{2}{3}\cdot \left[\ln(t+1)\right]_0^1
            + \frac{2}{3} \left(-\frac{1}{2}\left[\ln|t^2-t+1|\right]_0^1 + \frac{3}{2}\cdot \frac{2\sqrt{3}}{9}\cdot \pi\right)
            =\\&= \frac{2}{3}\ln 2 + \frac{2}{3}\cdot \frac{3}{2}\cdot \frac{2\sqrt{3}}{9}\cdot \pi
            =\\&= \frac{2}{3}\ln 2 + \frac{2\sqrt{3}}{9}\cdot \pi
        \end{align*}


        \item $f(x)=\dfrac{1}{x^2\sqrt{1+x^2}} \qquad \forall x\in J=\left]1,+\infty\right[$.  
        
        Sea el cambio de variable $x=\varphi(t)$ dado por $\varphi:\left]\nicefrac{\pi}{4},\nicefrac{\pi}{2}\right[\to \bb{R}$ dada por $\varphi(t)=\tg(t)$ para todo $t\in \left]\nicefrac{\pi}{4},\nicefrac{\pi}{2}\right[$.
        Tenemos que:
        \begin{equation*}
            \lim_{t\to \frac{\pi}{2}} \varphi(t) = +\infty \hspace{1cm} \lim_{t\to \frac{\pi}{4}} \varphi(t) = 1
        \end{equation*}

        Además, $\varphi\in C^1\left(\left]\nicefrac{\pi}{4},\nicefrac{\pi}{2}\right[\right)$, $\varphi\left(\left]\nicefrac{\pi}{4},\nicefrac{\pi}{2}\right[\right)=J$ y $\varphi'(t)\neq 0$ para todo $t\in \left]\nicefrac{\pi}{4},\nicefrac{\pi}{2}\right[$.
        Por tanto, estamos en las hipótesis del Teorema de Cambio de Variable, por lo que $f\in \cc{L}_1\left(\left]1,+\infty\right[\right)$ si, y sólo si, $(f\circ \varphi)\varphi'$ es integrable en $\left]\nicefrac{\pi}{4},\nicefrac{\pi}{2}\right[$.
        Tenemos que:
        \begin{equation*}
            (f\circ \varphi)(t)\cdot \varphi'(t) = \frac{1}{\tg^2(x)\cdot \sqrt{1+\tg^2(x)}}\cdot \frac{1}{\cos^2(x)}
            = \frac{1}{\sen^2(x)\cdot \sqrt{\frac{1}{\cos^2(x)}}}
            = \frac{\cos (x)}{\sen^2(x)}
        \end{equation*}

        La función obtenida tiene límite en $\nicefrac{\pi}{4}$ y $\nicefrac{\pi}{2}$, por lo que admite una extensión continua a $\left[\nicefrac{\pi}{4},\nicefrac{\pi}{2}\right]$,
        por lo que es integrable en $\left(\left]\nicefrac{\pi}{4},\nicefrac{\pi}{2}\right[\right)$. Por tanto, $(f\circ \varphi)\varphi'$ es integrable en $\left]\nicefrac{\pi}{4},\nicefrac{\pi}{2}\right[$, por lo que $f\in \cc{L}_1\left(J\right)$.

        Por el Teorema de Cambio de Variable, y posteriormente por la Regla de Barrow, tenemos que:
        \begin{align*}
            \int_1^{+\infty} \dfrac{1}{x^2\sqrt{1+x^2}}~dx
            =& \int_{\nicefrac{\pi}{4}}^{\nicefrac{\pi}{2}} \dfrac{\cos (x)}{\sen^2(x)}~dx
            = \left[-\frac{1}{\sen x}\right]_{\nicefrac{\pi}{4}}^{\nicefrac{\pi}{2}}
            =\\&= -\frac{1}{\sen\left(\nicefrac{\pi}{2}\right)} + \frac{1}{\sen\left(\nicefrac{\pi}{4}\right)}
            = \sqrt{2}-1
        \end{align*}


        \item $f(x)=\dfrac{1}{1+\cos x + \sen x} \qquad \forall x\in J=\left]0,\dfrac{\pi}{2}\right[$.
        
        La función $f$ tiene límite en los extremos de $J$, por lo que admite una extensión continua a $\ol{J}$ compacto, por lo que $f\in \cc{L}_1(J)$.

        Sea ahora el cambio de variable $x=\varphi(t)$ dado por $\varphi:]0,1[\to \bb{R}$ dada por $\varphi(t)=2\arctan t$ para todo $t\in~]0,1[$, y tenemos que $\varphi\in C^1(]0,1[)$ y $f$ es continua,
        por lo que estamos en las hipótesis de la versión elemental del Teorema de Cambio de Variable.
        
        Antes, veamos cómo expresar $\sen x$ y $\cos x$ en función de $t$.
        Como $x=\varphi (t)$, tenemos que:
        \begin{equation*}
            \frac{x}{2} = \arctan t \Longrightarrow
            \tg\left(\frac{x}{2}\right) = t \qquad \forall t\in ]0,1[
        \end{equation*}

        Usamos ahora la fórmula de la tangente del ángulo mitad:
        \begin{equation*}
            \tg x = \tg\left(2\cdot \frac{x}{2}\right)
            = \frac{2\cdot \tg\left(\frac{x}{2}\right)}{1-\tg^2\left(\frac{x}{2}\right)}
            \AstIg \frac{2t}{1-t^2}
        \end{equation*}
        donde en $(\ast)$ hemos usado de nuevo el cambio de variable. Por tanto, empleando el razonamiento
        sobre un triángulo rectángulo, tenemos que el cateto opuesto mide $2t$ y
        el cateto contiguo mide $1-t^2$, por lo que la hipotenusa mide:
        \begin{equation*}
            \sqrt{(2t)^2 + (1-t^2)^2} = \sqrt{4t^2 + 1 - 2t^2 + t^4}
            = \sqrt{t^4 + 2t^2 + 1} = 1 + t^2
        \end{equation*}

        Por tanto, tenemos que:
        \begin{equation*}
            \sen x = \frac{2t}{1+t^2} \hspace{1cm} \cos x = \frac{1-t^2}{1+t^2}
        \end{equation*}

        Retomando el uso del Teorema de Cambio de Variable, tenemos que:
        \begin{align*}
            \int_0^{\frac{\pi}{2}} \dfrac{1}{1+\cos x + \sen x}~dx
            &= \int_0^1 \dfrac{1}{1+\frac{1-t^2}{1+t^2} + \frac{2t}{1+t^2}}\cdot \frac{2}{1+t^2}~dt
            =\\&= \int_0^1 \dfrac{2}{1+t^2+1-t^2+2t}~dt
            = \int_0^1 \dfrac{2}{2t+2}~dt
            =\\&= \int_0^1 \dfrac{1}{t+1}~dt
            = \left[\ln(t+1)\right]_0^1
            = \ln 2
        \end{align*}

        \item $f(x)=\dfrac{1}{x^3\sqrt{x^2-1}} \qquad \forall x\in J=\left]1,+\infty\right[$.
        
        Sea $I=\left]0,\nicefrac{\pi}{2}\right[$, y consideramos 
        el cambio de variable $x=\varphi(t)$ dado por $\varphi:I\to \bb{R}$ dada por $\varphi(t)=\frac{1}{\cos x}$ para todo $t\in I$.
        Tenemos que $\varphi\in C^1(I)$, $\varphi(I)=J$ y $\varphi'(t)\neq 0$ para todo $t\in I$.
        Además, tenemos que:
        \begin{equation*}
            \lim_{t\to 0} \varphi(t) = 1 \hspace{1cm} \lim_{t\to \frac{\pi}{2}} \varphi(t) = +\infty
        \end{equation*}

        Por tanto, estamos en las hipótesis del Teorema de Cambio de Variable, por lo que $f\in \cc{L}_1(J)$ si, y sólo si, $(f\circ \varphi)\varphi'$ es integrable en $I$.
        Tenemos que:
        \begin{align*}
            (f\circ \varphi)(t)\cdot \varphi'(t)
            &= \frac{1}{\dfrac{1}{\cos^3(t)}\cdot \sqrt{\dfrac{1}{\cos^2(t)}-1}}\cdot \frac{\sen (t)}{\cos^2(t)}
            =\\&= \frac{\sen(t)\cos(t)}{\sqrt{\tg^2(t)}}
            = \frac{\sen(t)\cos(t)}{\tg(t)}
            = \cos^2(t) \hspace{2cm} \forall t\in I
        \end{align*}

        La función obtenida es integrable en $I$ por ser acotada y ser $I$ un conjunto de medida finita (acotado).
        Por tanto, $(f\circ \varphi)\varphi'$ es integrable en $I$, por lo que $f\in \cc{L}_1(J)$.
        Usando el Teorema de Cambio de Variable, y posteriormente la Regla de Barrow, tenemos que:
        \begin{align*}
            \int_1^{+\infty} \dfrac{1}{x^3\sqrt{x^2-1}}~dx
            &= \int_0^{\nicefrac{\pi}{2}} \cos^2(t)~dt
            \AstIg \int_0^{\nicefrac{\pi}{2}} \frac{1+\cos(2t)}{2}~dt
            =\\&= \frac{1}{2}\left[t+\frac{\sen(2t)}{2}\right]_0^{\nicefrac{\pi}{2}}
            = \frac{1}{2}\left[\frac{\pi}{2}\right] = \frac{\pi}{4}
        \end{align*}
        donde en $(\ast)$ hemos empleado la conocida
        forma de expresar el cuadrado del coseno en función del
        coseno del ángulo doble\footnote{Para evitar la extensión innecesaria de estos apuntes, no se incluirá la fácil demostración de esta fórmula.}.
    \end{enumerate}
\end{ejercicio}

\begin{ejercicio}
    En cada uno de los siguientes casos, estudiar la integrabilidad de la función $f$ en el intervalo $J$:
    \begin{enumerate}
        \item $f(x)=\dfrac{x^a}{e^x-1}\qquad \forall x\in J=\bb{R}^+.\hspace{1cm}(a\in \bb{R})$
        
        En estos casos, calcular la integral de $f$ en $J$ será un proceso demasiado complejo,
        por lo que buscamos aplicar el criterio de comparación. Para ello, sabemos que $f\in \cc{L}_1^{\text{loc}}(J)$ por ser $f$ continua.
        Partimos entonces el conjunto $J$ en dos subconjuntos $J_1=\left]0,1\right]$ y $J_2=\left[1,+\infty\right[$,
        y los estudiamos por separado.
        \begin{itemize}
            \item \ul{$J_1=\left]0,1\right]$}:
            
            Buscamos $g\in \cc{L}_1^{\text{loc}}(J_1)$, con $g(x)\neq 0$ para todo $x\in J_1$, y tal que:
            \begin{equation*}
                \lim_{x\to 0} \dfrac{|f(x)|}{|g(x)|} = \lim_{x\to 0} \dfrac{x^a}{(e^x-1)|g(x)|}
                = L\in \bb{R}^+
            \end{equation*}

            Sea entonces $g(x)=x^{a-1}>0$, función continua luego $g\in \cc{L}_1^{\text{loc}}(J_1)$.
            Tenemos que:
            \begin{equation*}
                \lim_{x\to 0} \dfrac{x^a}{(e^x-1)|g(x)|}
                = \lim_{x\to 0} \dfrac{x}{e^x-1}
                \Hop \lim_{x\to 0} \dfrac{1}{e^x}
                = 1
            \end{equation*}

            Por tanto, por el Criterio de Comparación, tenemos que $f\in \cc{L}_1(J_1)$ si y
            sólo si $g\in \cc{L}_1(J_1)$, y sabemos que esto equivale a que $a-1>-1$, es decir, $a>0$.
            Por tanto, $f\in \cc{L}_1(J_1)$ si, y sólo si, $a\in \bb{R}^+$.

            \item \ul{$J_2=\left[1,+\infty\right[$}:
            
            Sea ahora $g(x)=e^{\nicefrac{-x}{2}}>0$, función continua luego $g\in \cc{L}_1^{\text{loc}}(J_2)$. además, como $\nicefrac{-1}{2}<0$, tenemos que $g\in \cc{L}_1(J_2)$.
            Tenemos que:
            \begin{equation*}
                \lim_{x\to +\infty} \dfrac{x^a}{(e^x-1)|g(x)|}
                = \lim_{x\to +\infty} \dfrac{x^a}{(e^x-1)e^{\nicefrac{-x}{2}}}
                = \lim_{x\to +\infty} \dfrac{x^a}{e^{\nicefrac{x}{2}}-e^{\nicefrac{-x}{2}}}
                = 0
            \end{equation*}

            Por tanto, por el Criterio de Comparación, tenemos que $g\in \cc{L}_1(J_2)$ implica
            que $f\in \cc{L}_1(J_2)$, por lo que siempre se tiene.
        \end{itemize}

        Por tanto, $f\in \cc{L}_1(J)$ si, y sólo si, $a\in \bb{R}^+$.


        \item $f(x)=x^n e^{-x^2}\cos x \qquad \forall x\in J=\bb{R}.\hspace{1cm}(n\in \bb{N})$
        
        En este caso, la función $f$ es continua en todo $\bb{R}$, por lo que $f\in \cc{L}_1^{\text{loc}}(J)$.
        Partimos entonces el conjunto $J$ en dos subconjuntos $J_1=\bb{R}_0^-$ y $J_2=\bb{R}_0^+$, y los estudiamos por separado.
        \begin{itemize}
            \item \ul{$J_1=\bb{R}_0^-$}:
            
            Sea $g:\bb{R}_0^-\to \bb{R}$ dada por $g(x)=e^x>0$, función continua luego $g\in \cc{L}_1^{\text{loc}}(J_1)$.
            Tenemos que:
            \begin{align*}
                \lim_{x\to -\infty} \dfrac{|f(x)|}{|g(x)|} &= \lim_{x\to -\infty} \dfrac{|x^n e^{-x^2}\cos x|}{|e^x|}
                = \lim_{x\to -\infty} \dfrac{|x^n\cos x|}{e^{x^2+x}}
                =\\&= \lim_{x\to \infty} \dfrac{|(-x)^n\cos(-x)|}{e^{x^2-x}}
                = 0
            \end{align*}

            Por tanto, por el Criterio de Comparación, como $g\in \cc{L}_1(J_1)$, tenemos que $f\in \cc{L}_1(J_1)$.
            
            \item \ul{$J_2=\bb{R}_0^+$}:
            
            Sea $g:\bb{R}_0^+\to \bb{R}$ dada por $g(x)=e^{-x}>0$, función continua luego $g\in \cc{L}_1^{\text{loc}}(J_2)$.
            Tenemos que:
            \begin{align*}
                \lim_{x\to \infty} \dfrac{|f(x)|}{|g(x)|}
                &= \lim_{x\to \infty} \dfrac{|x^n e^{-x^2}\cos x|}{|e^{-x}|}
                = \lim_{x\to \infty} \dfrac{|x^n\cos x|}{e^{x^2-x}}
                = 0
            \end{align*}

            Por tanto, por el Criterio de Comparación, como $g\in \cc{L}_1(J_2)$, tenemos que $f\in \cc{L}_1(J_2)$.
        \end{itemize}

        Por tanto, $f\in \cc{L}_1(J)$ en todo caso, independientemente del valor de $n\in \bb{N}$.

        \item $f(x)=\dfrac{x^\rho}{1-\cos x} \qquad \forall x\in J=\left]0,\pi\right[.\hspace{1cm}(\rho\in \bb{R})$
        
        Definiendo $f(\pi) = \frac{\pi^\rho}{2}$, extendemos $f$ para que sea continua,
        luego localmente integrable, en $]0,\pi]$. Por tanto, $f\in \cc{L}_1^{\text{loc}}(]0,\pi])$.
        Definimos ahora $g:]0,\pi]\to \bb{R}$ dada por $g(x)=x^{\rho-2}>0$, función continua luego $g\in \cc{L}_1^{\text{loc}}(]0,\pi])$.
        Tenemos que:
        \begin{align*}
            \lim_{x\to 0} \dfrac{|f(x)|}{|g(x)|}
            &= \lim_{x\to 0} \dfrac{|x^\rho|}{|(1-\cos x)|\cdot |x^{\rho-2}|}
            = \lim_{x\to 0} \dfrac{x^2}{1-\cos x}
            =\\& \Hop \lim_{x\to 0} \dfrac{2x}{\sen x}
            \Hop \lim_{x\to 0} \dfrac{2}{\cos x} = 2
        \end{align*}

        Por tanto, por el Criterio de Comparación, tenemos que $f\in \cc{L}_1(]0,\pi])$ si, y sólo si, $g\in \cc{L}_1(]0,\pi])$, y sabemos que esto equivale a que $\rho-2>-1$, es decir, $\rho>1$.
        Por tanto, $f\in \cc{L}_1(]0,\pi])$ si, y sólo si, $\rho>1$.

        \item $f(x)=\dfrac{x^a{(1-x)}^b~\ln(1+x^2)}{\ln^2(x)} \qquad \forall x\in J=\left]0,1\right[.\hspace{1cm}(a,b\in \bb{R})$
        
        Sabemos que $f$ es continua en $J$, por lo que $f\in \cc{L}_1^{\text{loc}}(J)$.
        Partimos entonces el conjunto $J$ en dos subconjuntos $J_1=\left]0,\nicefrac{1}{2}\right]$ y $J_2=\left[\nicefrac{1}{2},1\right[$, y los estudiamos por separado.
        \begin{itemize}
            \item \ul{$J_1=\left]0,\nicefrac{1}{2}\right]$}:
            
            Para estudiar este caso, haremos uso del siguiente límite:
            \begin{equation*}
                \lim_{x\to 0} \dfrac{\ln(1+x^2)}{x^2}
                \Hop
                \lim_{x\to 0} \dfrac{\nicefrac{2x}{1+x^2}}{2x}
                = \lim_{x\to 0} \dfrac{1}{1+x^2} = 1
            \end{equation*}

            Definimos la función $g_1:\left]0,\nicefrac{1}{2}\right]\to \bb{R}$ dada por $g_1(x)=x^{a+2}$, función continua luego $g_1\in \cc{L}_1^{\text{loc}}(J_1)$.
            Tenemos que:
            \begin{align*}
                \lim_{x\to 0} \dfrac{|f(x)|}{|g_1(x)|}
                &= \lim_{x\to 0} \dfrac{|x^a{(1-x)}^b~\ln(1+x^2)|}{|x^{a+2}|\cdot |\ln^2(x)|}
                = \lim_{x\to 0} \dfrac{{(1-x)}^b}{\ln^2(x)}\cdot \dfrac{\ln(1+x^2)}{x^2}
                =\\&= \lim_{x\to 0} \dfrac{{(1-x)}^b}{\ln^2(x)} = 0
            \end{align*}

            Por tanto, por el Criterio de Comparación, tenemos que $g_1\in \cc{L}_1(J_1)\Longrightarrow f\in \cc{L}_1(J_1)$.
            Sabemos que $g\in \cc{L}_1(J_1)$ si, y sólo si, $a+2>-1$, es decir, $a>-3$.
            Por tanto, \ul{para $a>-3$, $f\in \cc{L}_1(J_1)$}.\\

            Para $a<-3$, sea $c\in \left]a, -3\right[$, y definimos $h_1:\left]0,\nicefrac{1}{2}\right]\to \bb{R}$ dada por $h_1(x)=x^{c+2}$, función continua luego $h_1\in \cc{L}_1^{\text{loc}}(J_1)$.
            Tenemos que:
            \begin{align*}
                \lim_{x\to 0} \dfrac{|f(x)|}{|h_1(x)|}
                &= \lim_{x\to 0} \dfrac{|x^a{(1-x)}^b~\ln(1+x^2)|}{|x^{c+2}|\cdot |\ln^2(x)|}
                = \lim_{x\to 0} \dfrac{{(1-x)}^b}{x^{c-a}\ln^2(x)}\cdot \dfrac{\ln(1+x^2)}{x^2}
                =\\&= \lim_{x\to 0} \dfrac{{(1-x)}^b}{x^{c-a}\ln^2(x)}
                = \infty
            \end{align*}

            Por tanto, por el Criterio de Comparación, tenemos que $f\in \cc{L}_1(J_1)\Longrightarrow h_1\in \cc{L}_1(J_1)$.
            No obstante, como $c+2<-3+2=-1$, sabemos que $h_1\notin \cc{L}_1(J_1)$, por lo que \ul{para $a<-3$, $f\notin \cc{L}_1(J_1)$}.\\

            Por último, veamos para el caso de $a=-3$. Definimos $h_2:\left]0,\nicefrac{1}{2}\right]\to \bb{R}$ dada por $h_2(x)=\dfrac{1}{x\ln^2(x)}$, función continua luego $h_2\in \cc{L}_1^{\text{loc}}(J_1)$.
            Del Criterio de Integrabilidad, tenemos que $h_2\in \cc{L}_1(J_1)$, con:
            \begin{equation*}
                \int_0^{\nicefrac{1}{2}} \dfrac{1}{x\ln^2(x)}~dx
                = \left[-\dfrac{1}{\ln(x)}\right]_0^{\nicefrac{1}{2}}
                = -\dfrac{1}{\ln\left(\nicefrac{1}{2}\right)} + 0
                = \dfrac{1}{\ln(2)}
            \end{equation*}

            Consideramos ahora el límite siguiente:
            \begin{equation*}
                \lim_{x\to 0} \dfrac{|f(x)|}{|h_2(x)|}
                = \lim_{x\to 0} \dfrac{|x^{-3}{(1-x)}^b~\ln(1+x^2)|}{|\ln^2(x)|\cdot \left|\frac{1}{x\ln^2(x)}\right|}
                = \lim_{x\to 0} {(1-x)}^b\cdot \dfrac{\ln(1+x^2)}{x^2} = 1
            \end{equation*}

            Por tanto, por el Criterio de Comparación, tenemos que $f\in \cc{L}_1(J_1)$ si, y sólo si, $h_2\in \cc{L}_1(J_1)$. Por tanto, \ul{para $a=-3$, $f\in \cc{L}_1(J_1)$}.
            
            
            \item \ul{$J_2=\left[\nicefrac{1}{2}, 1\right[$}:
            
            Definimos la función $g_2:\left[\nicefrac{1}{2},1\right[\to \bb{R}$ dada por $g_2(x)={(1-x)}^{b-2}$, función continua luego $g_2\in \cc{L}_1^{\text{loc}}(J_2)$.
            Tenemos que:
            \begin{align*}
                \lim_{x\to 1} \dfrac{|f(x)|}{|g_2(x)|}
                &= \lim_{x\to 1} \dfrac{|x^a{(1-x)}^b~\ln(1+x^2)|}{|{(1-x)}^{b-2}|\cdot |\ln^2(x)|}
                = \lim_{x\to 1} x^a\ln(1+x^2) \cdot \left(\dfrac{1-x}{\ln(x)}\right)^2
                =\\&=
                \ln(2)\lim_{x\to 1} \left(\dfrac{1-x}{\ln(x)}\right)^2
                \Hop \ln(2)\lim_{x\to 1} \left(\dfrac{-1}{\nicefrac{1}{x}}\right)^2 = \ln(2)\in \bb{R}^+
            \end{align*}

            Por tanto, por el Criterio de Comparación, tenemos que $f\in \cc{L}_1(J_2)$ si, y sólo si, $g_2\in \cc{L}_1(J_2)$, y sabemos que esto equivale a que $b-2>-1$, es decir, $b>1$.
        \end{itemize}

        En resumen, tenemos que $f\in \cc{L}_1(J)$ si, y sólo si, $a\geq -3$ y $b>1$.
    \end{enumerate}
\end{ejercicio}

\begin{ejercicio}[Parcial DGIIM 23-24] Estudiar la integrabilidad
    en la semirrecta $J=\left]1,+\infty\right[$ de la función $f:J\to \bb{R}$ definida por
    \begin{equation*}
        f(x) = \frac{\ln x}{(x-1)^{\nicefrac{3}{2}}}       \qquad \forall x\in J
    \end{equation*}


    Fijado $c\in \left]1,+\infty\right[$, estudiamos la integrabilidad por separado:
    \begin{itemize}
        \item \underline{$I_1 = \left]1, c\right]$}:
        
            Tenemos que $f\in \cc{L}_1^{loc}(I_1)$ por ser continua. Intuimos que hemos de usar el Criterio de Comparación, pero no sabemos con qué función comparar.
            Dado $a\in \bb{R}$,
            definimos $g_a:I_1\to \bb{R}$ dada por $g_a(x)=(x-1)^{a}$, función continua luego $g_a\in \cc{L}_1^{loc}(I_1)$.
            Además, para que tengamos $g_a\in \cc{L}_1(I_1)$, suponemos $a>-1$.
            \begin{equation*}
                \lim_{x\to 1} \dfrac{|f(x)|}{|g_a(x)|} = \lim_{x\to 1} \dfrac{\ln x}{(x-1)^{\nicefrac{3}{2}}\cdot (x-1)^{a}}
                = \lim_{x\to 1} \dfrac{\ln x}{(x-1)^{\nicefrac{3}{2}+a}}
            \end{equation*}

            Como $a>-1$, tenemos que $\frac{3}{2}+a>\frac{3}{2}-1=\frac{1}{2}>0$, por lo que:
            \begin{equation*}
                \lim_{x\to 1} \dfrac{|f(x)|}{|g_a(x)|} = \left[\dfrac{0}{0}\right]
                \Hop
                \lim_{x\to 1} \dfrac{\nicefrac{1}{x}}{(\nicefrac{3}{2}+a)(x-1)^{\nicefrac{1}{2}+a}}
            \end{equation*}

            Para que dicho límite sea $0$, necesitamos que $\frac{1}{2}+a<0$, es decir, $a<\nicefrac{-1}{2}$.
            Por tanto, nos bastará con escoger $a\in \left]-1, \nicefrac{-1}{2}\right[$, por ejemplo $a=\nicefrac{-3}{4}$. Sea $h_1(x)=(x-1)^{\nicefrac{-3}{4}}$, con $h_1\in \cc{L}_1(I_1)$, tenemos que:
            \begin{equation*}
                \lim_{x\to 1} \dfrac{|f(x)|}{|h_1(x)|}
                =\lim_{x\to 1} \dfrac{\nicefrac{1}{x}}{\frac{3}{4}\cdot (x-1)^{\nicefrac{-1}{4}}} = 0
            \end{equation*}
            Por tanto, por el Criterio de Comparación, como $h_1\in \cc{L}_1(I_1)$, tenemos que $f\in \cc{L}_1(I_1)$.

            \begin{observacion}
                Evidentemente, en un examen este razonamiento
                se haría en sucio y tan solo se entregaría el razonamiento hecho directamente con la función $h_1$.
                No obstante, por simples objetivos didácticos, se ha querido mostrar el proceso completo para que el lector sepa razonar cómo obtener dicha función.
            \end{observacion}


        \item \underline{$I_2 = \left[c, +\infty\right[$}: Tenemos que $g\in \cc{L}_1^{loc}(I_2)$ por ser continua.
        
        De nuevo, intuimos que hemos de usar el Criterio de Comparación, pero no sabemos con qué función comparar.
        Tenemos que $f\in \cc{L}_1^{loc}(I_2)$ por ser continua, y dado $a\in \bb{R}$, definimos $g_a:I_2\to \bb{R}$ dada por $g_a(x)=(x-1)^{a}$, función continua luego $g_a\in \cc{L}_1^{loc}(I_2)$.
        En este caso, suponemos $a<-1$ para tener que $g_a\in \cc{L}_1(I_2)$.
        \begin{equation*}
            \lim_{x\to +\infty} \dfrac{|f(x)|}{|g_a(x)|} = \lim_{x\to +\infty} \dfrac{\ln x}{(x-1)^{\nicefrac{3}{2}}\cdot (x-1)^{a}}
            = \lim_{x\to +\infty} \dfrac{\ln x}{(x-1)^{\nicefrac{3}{2}+a}}
        \end{equation*}

        Para que dicho límite no sea infinito, necesitamos una indeterminación,
        por lo que buscamos que el denominador también diverga, por lo que necesitamos que $\frac{3}{2}+a>0$, es decir, $a>\nicefrac{-3}{2}$.
        Supongamos por tanto $a\in \left]\nicefrac{-3}{2},~-1\right[$, y tenemos que:
        \begin{equation*}
            \lim_{x\to +\infty} \dfrac{|f(x)|}{|g_a(x)|} = \left[\dfrac{\infty}{\infty}\right]
            \Hop
            \lim_{x\to +\infty} \dfrac{\nicefrac{1}{x}}{(\nicefrac{3}{2}+a)(x-1)^{\nicefrac{1}{2}+a}}
        \end{equation*}

        Al ser un límite con $x\to \infty$ de funciones racionales,
        sabemos que este límite será $0$ si y solo si $-1<\frac{1}{2}+a$, es decir, $\nicefrac{-3}{2}<a$, algo que ya tenemos.
        Por tanto, tan solo necesitamos imponer que $a\in \left]\nicefrac{-3}{2},~-1\right[$, por ejemplo $a=\nicefrac{-5}{4}$. Sea $h_2(x)=(x-1)^{\nicefrac{-5}{4}}$, con $h_2\in \cc{L}_1(I_2)$, tenemos que:
        \begin{equation*}
            \lim_{x\to +\infty} \dfrac{|f(x)|}{|h_2(x)|}
            =\lim_{x\to +\infty} \dfrac{\nicefrac{1}{x}}{\frac{5}{4}\cdot (x-1)^{\nicefrac{-1}{4}}}
            = \dfrac{4}{5}\cdot \lim_{x\to +\infty} \dfrac{(x-1)^{\nicefrac{1}{4}}}{x} = 0
        \end{equation*}

        Por tanto, por el Criterio de Comparación, como $h_2\in \cc{L}_1(I_2)$, tenemos que $f\in \cc{L}_1(I_2)$.
        \begin{observacion}
            De nuevo, se ha querido mostrar el proceso completo para que el lector sepa razonar cómo obtener dicha función.
            En un examen, tan solo se entregaría el razonamiento hecho directamente con la función $h_2$.
        \end{observacion}
    \end{itemize}
    En definitiva, tenemos que $f\in \cc{L}_1(J)$, que es lo que queríamos demostrar.
\end{ejercicio}

\begin{ejercicio}[Ordinaria DGIIM 22-23] Estudiar la integrabilidad
    en la semirrecta $J=\left]1,+\infty\right[$ de la función $f:J\to \bb{R}$ definida por
    \begin{equation*}
        f(x) = \frac{\ln^2 x}{(x-1)^{\nicefrac{5}{2}}}       \qquad \forall x\in J
    \end{equation*}


    Fijado $c\in \left]1,+\infty\right[$, estudiamos la integrabilidad por separado:
    \begin{itemize}
        \item \underline{$I_1 = \left]1, c\right]$}:
        
            Tenemos que $f\in \cc{L}_1^{loc}(I_1)$ por ser continua.
            Sea $g_1:I_1\to \bb{R}$ dada por $g_1(x)=(x-1)^{\nicefrac{-1}{2}}$, función continua luego $g_1\in \cc{L}_1^{loc}(I_1)$.
            Tenemos que:
            \begin{align*}
                \lim_{x\to 1} \dfrac{|f(x)|}{|g_1(x)|}
                &= \lim_{x\to 1} \dfrac{\ln^2 x}{(x-1)^{\nicefrac{5}{2}}\cdot (x-1)^{\nicefrac{-1}{2}}}
                = \lim_{x\to 1} \dfrac{\ln^2 x}{(x-1)^2}
                =\\&= \left(\lim_{x\to 1} \dfrac{\ln x}{x-1}\right)^2
                = \left(\lim_{x\to 1} \dfrac{\nicefrac{1}{x}}{1}\right)^2 = 1
            \end{align*}

            Por tanto, por el Criterio de Comparación, tenemos que $f\in \cc{L}_1(I_1)$ si, y sólo si, $g_1\in \cc{L}_1(I_1)$, y sabemos que esto equivale a que $\nicefrac{-1}{2}>-1$ (que es cierto). Por tanto, \ul{para $I_1$, $f\in \cc{L}_1(I_1)$}.


        \item \underline{$I_2 = \left[c, +\infty\right[$}: Tenemos que $g\in \cc{L}_1^{loc}(I_2)$ por ser continua.
        
        Tenemos que $f\in \cc{L}_1^{loc}(I_2)$ por ser continua, y definimos $g_2:I_2\to \bb{R}$ dada por $g_2(x)=(x-1)^{\nicefrac{-3}{2}}$, función continua luego $g_2\in \cc{L}_1^{loc}(I_2)$.
        Tenemos que:
        \begin{align*}
            \lim_{x\to +\infty} \dfrac{|f(x)|}{|g_2(x)|}
            &= \lim_{x\to +\infty} \dfrac{\ln^2 x}{(x-1)^{\nicefrac{5}{2}}\cdot (x-1)^{\nicefrac{-3}{2}}}
            = \lim_{x\to +\infty} \dfrac{\ln^2 x}{x-1}
            \Hop\\&\Hop
            \lim_{x\to +\infty} \dfrac{2\ln x\cdot \frac{1}{x}}{1}
            = 0
        \end{align*}
        Por tanto, por el Criterio de Comparación, como $g_2\in \cc{L}_1(I_2)$ (ya que $\nicefrac{-3}{2}<-1$), tenemos que $f\in \cc{L}_1(I_2)$.
    \end{itemize}
    En definitiva, tenemos que $f\in \cc{L}_1(J)$, que es lo que queríamos demostrar.
\end{ejercicio}

\begin{ejercicio}[Parcial DGIIM 23-24] Probar que la función $g:\bb{R}_0^+$definida por
    \begin{equation*}
        g(x) = \frac{e^{2x}-e^x}{e^{3x}+1}
        \qquad \forall x\in \bb{R}_0^+
    \end{equation*}
    es integrable en $\bb{R}_0^+$ y calcular su integral.\\

    Sean $I=\left[1,+\infty\right[$, $J=\bb{R}^+_0$, definimos $\varphi:I\longrightarrow J$ dada por:
    \begin{equation*}
        \varphi(t) = \ln t \qquad \forall t\in I
    \end{equation*}
    
    Tenemos que $\varphi \in C^1(I)$ y $g$ continua, por lo que definiendo $h=(g\circ \varphi)\varphi'$, por el Teorema de cambio de variable tenemos que $g\in \cc{L}_1(J)\Longleftrightarrow h\in \cc{L}_1(I)$.
    \begin{equation*}
        h(t) = [(g\circ \varphi)\varphi'](t) = \dfrac{t^2-t}{t(t^3+1)} = \dfrac{t-1}{t^3 + 1} \hspace{1cm} \forall t\in I
    \end{equation*}
    Como $3-1>1$, sabemos que $h\in \cc{L}_1(I)$, de donde tenemos que $g\in \cc{L}_1(J)$. Calculamos ahora la integral de $g$ usando también el Teorema de cambio de variable:
    \begin{equation*}
        \int_0^{+\infty} g(x)~dx = \int_1^{+\infty} h(t)~dt = \int_1^{+\infty} \dfrac{t-1}{t^3+1}~dt
    \end{equation*}

    Para resolver esta integral, descomponemos la fracción en fracciones simples. Tenemos que:
    \begin{figure}[H]
        \centering
        \polyhornerscheme[x=-1]{x^3+1}
    \end{figure}

    Por tanto, tenemos que $t^3+1 = (t+1)(t^2-t+1)$, y sabemos que $t^2-t+1$ no tiene raíces reales ya que $\Delta = 1-4<0$.
    Por tanto, podemos descomponer la fracción en fracciones simples de la siguiente forma:
    \begin{equation*}
        \dfrac{t-1}{t^3+1} = \dfrac{A}{t+1} + \dfrac{Bt+C}{t^2-t+1} = \dfrac{A(t^2-t+1) + (Bt+C)(t+1)}{t^3+1}
    \end{equation*}

    Igualando numeradores, tenemos que:
    \begin{itemize}
        \item \ul{Para $t=-1$}: $-2 = 3A \Longrightarrow A=\nicefrac{-2}{3}$.
        \item \ul{Para $t=0$}: $-1 = A + C \Longrightarrow C=\nicefrac{-1}{3}$.
        \item \ul{Para $t=1$}: $0 = A + 2B + 2C \Longrightarrow B=\frac{-A-2C}{2}=\nicefrac{2}{3}$.
    \end{itemize}

    Por tanto, usando también la versión general de la Regla de Barrow, tenemos que:
    \begin{align*}
        \int_0^{+\infty} g(x)~dx &= \int_1^{+\infty} \dfrac{t-1}{t^3+1}~dt
        =\\&= -\frac{2}{3}\int_1^{+\infty} \dfrac{dt}{t+1} + \frac{1}{3}\int_1^{+\infty} \dfrac{2t-1}{t^2-t+1}~dt
        =\\&= -\frac{2}{3}\left[\ln(t+1)\right]_1^{+\infty} + \frac{1}{3}\left[\ln(t^2-t+1)\right]_1^{+\infty}
        =\\&= \frac{1}{3}\left[\ln\left(\frac{t^2-t+1}{(t+1)^2}\right)\right]_1^{+\infty}
        =\\&= -\frac{1}{3}\ln\left(\frac{1}{4}\right)
        = \frac{1}{3}\ln 4
    \end{align*}
    donde hemos hecho uso de que:
    \begin{equation*}
        \lim_{t\to+\infty}\ln\left(\frac{t^2-t+1}{(t+1)^2}\right) =
        \ln\left(\lim_{t\to+\infty}\frac{t^2-t+1}{(t+1)^2}\right) =
        \ln(1) = 0
    \end{equation*}
\end{ejercicio}


\begin{ejercicio}[Incidencias DGIIM 22-23 y 23-24]
    Estudiar la integrabilidad en $\bb{R}^+$ de la función $f:\bb{R}^+\to \bb{R}$ dada por:
    \begin{equation*}
        f(x) = \dfrac{(1+x)\log(1+x^2)\cos x}{x^2\sqrt{x}}
    \end{equation*}

    Vemos que no podemos extender $f$ de forma continua al $0$ por diverger esta, donde hemos hecho uso de que:
    \begin{align*}
        \lim_{x\to 0} \dfrac{\ln(1+x^2)}{x^2}=1
    \end{align*}

    Buscamos entonces aplicar el Criterio de Comparación. Para ello, partimos el conjunto $\bb{R}^+$ en dos subconjuntos $J_1=\left]0,1\right]$ y $J_2=\left[1,+\infty\right[$, y los estudiamos por separado.
    Como $f$ es continua en todo $\bb{R}^+$, tenemos que $f\in \cc{L}_1^{\text{loc}}(\bb{R}^+)$. Estudiamos ahora cada uno de los conjuntos:
    \begin{itemize}
        \item \ul{$J_1=\left]0,1\right]$}:
        
        Sea $g_1:J_1\to \bb{R}$ dada por $g_1(x)=x^{\nicefrac{-3}{4}}$, función continua luego $g_1\in \cc{L}_1^{\text{loc}}(J_1)$. Tenemos que:
        \begin{align*}
            \lim_{x\to 0} \dfrac{|f(x)|}{|g_1(x)|}
            &= \lim_{x\to 0} \dfrac{f(x)}{g_1(x)}
            = \lim_{x\to 0}\dfrac{\log(1+x^2)}{x^2}\cdot
            \lim_{x\to 0}\dfrac{(1+x)\cos x}{\sqrt{x}\cdot x^{\nicefrac{-3}{4}}}
            = \lim_{x\to 0}\dfrac{(1+\cos(x))}{x^{\nicefrac{-1}{4}}} = 0
        \end{align*}
        Como $g_1\in \cc{L}_1(J_1)$ por ser $\nicefrac{-3}{4}>-1$, tenemos que $f\in \cc{L}_1(J_1)$.

        \item \ul{$J_2=\left[1,+\infty\right[$}:
        
        Sea $g_2:J_2\to \bb{R}$ dada por $g_2(x)=x^{\nicefrac{-4}{3}}$, función continua luego $g_2\in \cc{L}_1^{\text{loc}}(J_2)$. Tenemos que:
        \begin{align*}
            \lim_{x\to +\infty} \dfrac{|f(x)|}{|g_2(x)|}
            &= \lim_{x\to +\infty}\dfrac{(1+x)\log(1+x^2)|\cos x|}{x^2\sqrt{x}\cdot x^{\nicefrac{-4}{3}}}\\
            &= \lim_{x\to +\infty}\dfrac{(1+x)\log(1+x^2)|\cos x|}{x^{\nicefrac{7}{6}}}=0
        \end{align*}
        donde en el último paso hemos usado que $\nicefrac{7}{6}>1$. Por tanto, por el Criterio de Comparación, como $g_2\in \cc{L}_1(J_2)$ por ser $\nicefrac{-4}{3}<-1$, tenemos que $f\in \cc{L}_1(J_2)$.
    \end{itemize}

    Por tanto, como $f\in \cc{L}_1(J_1)$ y $f\in \cc{L}_1(J_2)$, tenemos que $f\in \cc{L}_1(\bb{R}^+)$.
\end{ejercicio}


\begin{ejercicio}[Parcial DGIIM 22-23]
    Estudiar la integrabilidad en el intervalo $J=\left]0,\nicefrac{\pi}{2}\right[$ de la función $f:J\to \bb{R}$ dada por:
    \begin{equation*}
        f(x) = \dfrac{(\pi-2x)\log x}{\cos x~\sqrt{\sen x}}
        \qquad \forall x\in J
    \end{equation*}

    Veamos si podemos extender $f$ de forma continua al $\nicefrac{\pi}{2}$. Para ello, calculamos el límite:
    \begin{equation*}
        \lim_{x\to \nicefrac{\pi}{2}} f(x)
        = \lim_{x\to \nicefrac{\pi}{2}} \dfrac{\log x}{\sqrt{\sen x}}\cdot \lim_{x\to \nicefrac{\pi}{2}} \dfrac{\pi-2x}{\cos x}
        \Hop \log\left(\nicefrac{\pi}{2}\right)\cdot \lim_{x\to \nicefrac{\pi}{2}} \dfrac{-2}{-\sen x}
        = -2\log\left(\nicefrac{\pi}{2}\right)
    \end{equation*}

    Por tanto, podemos extender $f$ de forma continua al $\nicefrac{\pi}{2}$. De esta forma, y como $f$ es continua en todo $J$ (tenemos que $f\in \cc{L}_1^{\text{loc}}(J)$), podemos estudiar la integrabilidad de $f$ en $J$ tan solo aplicando el Criterio de Comparación en el $0$.
    Veamos en primer lugar ciertos límites:
    \begin{align*}
        &\lim_{x\to 0} \dfrac{\pi-2x}{\cos x}=\pi\\
        &\lim_{x\to 0} \left(\frac{x}{\sen x}\right)^{\nicefrac{1}{2}}=\left(\lim_{x\to 0} \frac{x}{\sen x}\right)^{\nicefrac{1}{2}}=1\\
        &\lim_{x\to 0} \dfrac{\log x}{x^{\nicefrac{-1}{4}}}
        \Hop  \lim_{x\to 0} \dfrac{1}{x\cdot \frac{-1}{4}x^{\nicefrac{-5}{4}}}
        = -4 \lim_{x\to 0} x^{\nicefrac{1}{4}} = 0
    \end{align*}

    Definimos por tanto $g:J\to \bb{R}$ dada por $g(x)=x^{\nicefrac{-3}{4}}$, función continua luego $g\in \cc{L}_1^{\text{loc}}(J)$. Tenemos que:
    \begin{align*}
        \lim_{x\to 0} \dfrac{|f(x)|}{|g(x)|}
        &= \lim_{x\to 0} \dfrac{(\pi-2x)\log x}{\cos x~\sqrt{\sen x}\cdot x^{\nicefrac{-3}{4}}}
        = \lim_{x\to 0} \dfrac{(\pi-2x)\log x}{\cos x~\sqrt{\sen x}\cdot x^{\nicefrac{-1}{2}}\cdot x^{\nicefrac{-1}{4}}}
        =\\&= \lim_{x\to 0} \dfrac{\log x}{x^{\nicefrac{-1}{4}}}
        = \pi\cdot 1\cdot 0 = 0
    \end{align*}

    Por tanto, por el Criterio de Comparación, como $g\in \cc{L}_1(J)$ (por ser $\nicefrac{-3}{4}>-1$), tenemos que $f\in \cc{L}_1(J)$.
\end{ejercicio}


\begin{ejercicio}[Extraordinaria DGIIM 22-23]
    Estudia la integrabilidad en $\bb{R}^+$ de la función definida por:
    \begin{equation*}
        f(x) = \dfrac{e^{-x}\sqrt{x}\cos x}{\log(1+x)} \qquad \forall x\in \bb{R}^+
    \end{equation*}

    Fijado $c\in \bb{R}^+$, estudiamos la integrabilidad por separado en los intervalos $I_1=\left]0,c\right]$ y $I_2=\left[c,+\infty\right[$.
    \begin{itemize}
        \item \ul{$I_1=\left]0,c\right]$}:
        
        Sea $g_1:I_1\to \bb{R}$ dada por $g_1(x)=x^{\nicefrac{-1}{2}}$, función continua luego $g_1\in \cc{L}_1^{\text{loc}}(I_1)$. Tenemos que:
        \begin{align*}
            \lim_{x\to 0} \dfrac{|f(x)|}{|g_1(x)|}
            &= \lim_{x\to 0} \dfrac{e^{-x}\sqrt{x}\cos x}{\log(1+x)\cdot x^{\nicefrac{-1}{2}}}
            = \lim_{x\to 0} \dfrac{e^{-x}(\sqrt{x})^2|\cos x|}{\log(1+x)}
            =\\&= \lim_{x\to 0} \dfrac{e^{-x}x|\cos x|}{\log(1+x)}
            = \lim_{x\to 0} e^{-x}|\cos x| \cdot \lim_{x\to 0} \dfrac{x}{\log(1+x)}
            = 0\cdot 1 = 0
        \end{align*}

        Por tanto, por el Criterio de Comparación, como $g_1\in \cc{L}_1(I_1)$ (por tener que $\nicefrac{-1}{2}>-1$), tenemos que $f\in \cc{L}_1(I_1)$.

        \item \ul{$I_2=\left[c,+\infty\right[$}:
        
        Sea $g_2:I_2\to \bb{R}$ dada por $g_2(x)=x^{\nicefrac{-3}{2}}$, función continua luego $g_2\in \cc{L}_1^{\text{loc}}(I_2)$. Tenemos que:
        \begin{align*}
            \lim_{x\to +\infty} \dfrac{|f(x)|}{|g_2(x)|}
            = \lim_{x\to +\infty} \dfrac{e^{-x}\sqrt{x}|\cos x|}{\log(1+x)\cdot x^{\nicefrac{-3}{2}}}
            = \lim_{x\to +\infty} \dfrac{e^{-x}x^2|\cos x|}{\log(1+x)} = 0
        \end{align*}
        donde al final se ha hecho uso de la escala de infinitos. Por tanto, por el Criterio de Comparación, como $g_2\in \cc{L}_1(I_2)$ (por tener que $\nicefrac{-3}{2}<-1$), tenemos que $f\in \cc{L}_1(I_2)$.
    \end{itemize}

    En definitiva, tenemos que $f\in \cc{L}_1(\bb{R}^+)$.
\end{ejercicio}


\begin{ejercicio}[Ordinaria DGIIM 23-24]
    Estudiar la integrabilidad en $\bb{R}^+$ de la función $f:\bb{R}^+\to \bb{R}$ dada por:
    \begin{equation*}
        f(x) = \dfrac{\left(e^{\nicefrac{x}{2}} -1\right)\sqrt{x}\cos x}{e^x \log(1+x^2)}
        \qquad \forall x\in \bb{R}^+
    \end{equation*}

    Como $f$ es continua en todo $\bb{R}^+$, tenemos que $f\in \cc{L}_1^{\text{loc}}(\bb{R}^+)$.
    Fijado $c\in \bb{R}^+$, estudiamos la integrabilidad por separado en los intervalos $I_1=\left]0,c\right]$ y $I_2=\left[c,+\infty\right[$.
    \begin{itemize}
        \item \ul{$I_1=\left]0,c\right]$}:
        
        Sea $g_1:I_1\to \bb{R}$ dada por $g_1(x)=x^{\nicefrac{-1}{2}}$, función continua luego $g_1\in \cc{L}_1^{\text{loc}}(I_1)$. Tenemos que:
        \begin{align*}
            \lim_{x\to 0} \dfrac{|f(x)|}{|g_1(x)|}
            &= \lim_{x\to 0} \dfrac{\left(e^{\nicefrac{x}{2}} -1\right)\sqrt{x}|\cos x|}{e^x \log(1+x^2)\cdot x^{\nicefrac{-1}{2}}}
            =\\&= \lim_{x\to 0} \dfrac{\left(e^{\nicefrac{x}{2}} -1\right)x|\cos x|}{e^x \log(1+x^2)}
            = \lim_{x\to 0} \dfrac{\left(e^{\nicefrac{x}{2}} -1\right)x^2|\cos x|}{e^x x\log(1+x^2)} =\\
            &= \lim_{x\to 0} \dfrac{|\cos x|}{e^x} \cdot \lim_{x\to 0} \dfrac{x^2}{\log(1+x^2)} \cdot \lim_{x\to 0} \dfrac{e^{\nicefrac{x}{2}} -1}{x}=\\
            &= 1\cdot 1\cdot \lim_{x\to 0} \dfrac{e^{\nicefrac{x}{2}}}{2} = 1\cdot 1\cdot \dfrac{1}{2} = \dfrac{1}{2}
        \end{align*}

        Por tanto, por el Criterio de Comparación, como $g_1\in \cc{L}_1(I_1)$ (por tener que $\nicefrac{-1}{2}>-1$), tenemos que $f\in \cc{L}_1(I_1)$.

        \item \ul{$I_2=\left[c,+\infty\right[$}:
        
        Sea $g_2:I_2\to \bb{R}$ dada por $g_2(x)=e^{\nicefrac{-x}{4}}$, función continua luego $g_2\in \cc{L}_1^{\text{loc}}
        (I_2)$. Tenemos que:
        \begin{align*}
            \lim_{x\to +\infty} \dfrac{|f(x)|}{|g_2(x)|}
            &= \lim_{x\to +\infty} \dfrac{\left(e^{\nicefrac{x}{2}} -1\right)\sqrt{x}|\cos x|}{e^x \log(1+x^2)\cdot e^{\nicefrac{-x}{4}}}
            =\\&= \lim_{x\to +\infty} \dfrac{\left(e^{\nicefrac{x}{2}} -1\right)\sqrt{x}|\cos x|}{e^{\nicefrac{3x}{4}} \log(1+x^2)}
            = 0
        \end{align*}
        donde en el último paso hemos hecho uso de la escala de infinitos, puesto que $\nicefrac{3}{4}>\nicefrac{1}{2}$. Por tanto, por el Criterio de Comparación, como $g_2\in \cc{L}_1(I_2)$ (por tener que $\nicefrac{-1}{4}<0$), tenemos que $f\in \cc{L}_1(I_2)$.
    \end{itemize}
\end{ejercicio}
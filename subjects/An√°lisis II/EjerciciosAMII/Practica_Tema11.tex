\section{Cálculo de Integrales Simples}

\subsection{Repaso teórico}
Repasamos ahora los conceptos teóricos necesarios para realizar el cálculo de integrales simples. Destacamos 4 reglas que nos ayudan a resolver el cálculo de integrales, todas ellas tendrán su versión elemental\footnote{Las cuales ya se vieron en Cálculo II.} y su versión general.

\subsubsection{Regla de Barrow}
\begin{teo}[Regla de Barrow versión elemental]
    Si $f:J\rightarrow\mathbb{R}$ es una función continua y $G$ una primitiva de $f$, se tiene:
    \begin{equation*}
        \int_{a}^{b} f(x)~dx  = G(b) - G(a) = {[G(x)]}_a^b\qquad \forall a,b\in J
    \end{equation*}
\end{teo}

\begin{teo}[Versión general de la Regla de Barrow]
    Si $f\in \cc{L}_1(J)$ y $G:J\rightarrow\mathbb{R}$ es una primitiva de $f$, entonces $G$ tiene límite, tanto en $\alpha$ como el $\beta$, y se verifica que:
    \begin{equation*}
        \displaystyle\int_{\alpha}^{\beta} f(t)~dt = \lim_{x\to\beta} G(x) -\lim_{x\to\alpha} G(x) = {[G(x)]}_\alpha^\beta
    \end{equation*}
\end{teo}
Notemos que ninguno de los dos teoremas anteriores nos permite averiguar si $f$ es integrable o no, ya que suponen que lo es para llegar a la tesis. A continuación, vemos un criterio que nos permite comprobar esto, como consecuencia de la versión general de la regla de Barrow.

\begin{teo}[Criterio de integrabilidad]
    Dada una función $f:J\rightarrow\red{\mathbb{R}^+_0}$, sea $G$ una primitiva de $f$. Entonces $f\in \cc{L}_1$ si, y sólo si, $G$ tiene límite en $\alpha$ y $\beta$, en cuyo caso:
    \begin{equation*}
        \displaystyle\int_{\alpha}^{\beta} f(t) ~dt  = {[G(x)]}_\alpha^\beta
    \end{equation*}
\end{teo}
Notemos que sólo es válido para funciones con codominio $\mathbb{R}^+_0$. Sin embargo, si tenemos una función con imagen negativa $f$ de forma que $-f$ (función con imagen positiva) cumpla las hipótesis del criterio, $-f$ será integrable, luego $f$ también. De esta forma, si tenemos una función que pasa de ser positiva a negativa (o viceversa) un número finito de veces, podemos en cada trozo donde el signo de su imagen es constante, aplicar el criterio, obteniendo que la función es integrable (en caso de que cada uno de sus ``trozos'' cumpla con las hipótesis del criterio).

\subsection{Ejercicios}
\begin{ejercicio}
    En cada uno de los siguientes casos, probar que la función $f$ es integrable en el intervalo $J$ y calcular su integral:
    \begin{enumerate}
        \item $f(x)=x^2\ln x \qquad \forall x\in J=\left]0,1\right[$.
        \item $f(x)=e^{-x}\cos(2x) \qquad \forall x\in J=\bb{R}^+$.
        \item $f(x)=\dfrac{1}{x^4-1} \qquad \forall x\in J=\left]2,+\infty\right[$.
        \item $f(x)=\dfrac{1}{e^x+e^{-x}} \qquad \forall x\in J=\bb{R}$.
        \item $f(x)=\dfrac{1}{x^2 + \sqrt{x}} \qquad \forall x\in J=\left]0,1\right[$.
        \item $f(x)=\dfrac{1}{x^2\sqrt{1+x^2}} \qquad \forall x\in J=\left]1,+\infty\right[$.  
        \item $f(x)=\dfrac{1}{1+\cos x + \sen x} \qquad \forall x\in J=\left]0,\dfrac{\pi}{2}\right[$.
        \item $f(x)=\dfrac{1}{x^3\sqrt{x^2-1}} \qquad \forall x\in J=\left]1,+\infty\right[$.
    \end{enumerate}
\end{ejercicio}

\begin{ejercicio}
    En cada uno de los siguientes casos, estudiar la integrabilidad de la función $f$ en el intervalo $J$:
    \begin{enumerate}
        \item $f(x)=\dfrac{x^2}{e^x-1}\qquad \forall x\in J=\bb{R}^+.\hspace{1cm}(a\in \bb{R})$
        \item $f(x)=x^n e^{-x^2}\cos x \qquad \forall x\in J=\bb{R}.\hspace{1cm}(n\in \bb{N})$
        \item $f(x)=\dfrac{x^\rho}{1-\cos x} \qquad \forall x\in J=\left]0,\pi\right[.\hspace{1cm}(\rho\in \bb{R})$
        \item $f(x)=\dfrac{x^a{(1-x)}^b~\ln(1+x^2)}{{(\ln x)}^2} \qquad \forall x\in J=\left]0,1\right[.\hspace{1cm}(a,b\in \bb{R})$
    \end{enumerate}
\end{ejercicio}

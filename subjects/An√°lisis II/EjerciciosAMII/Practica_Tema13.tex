\section{Teorema de Fubini}

\begin{ejercicio}
    Probar que el siguiente conjunto es medible y calcular su área.
    \[ E = \left\{ (x, y) \in \mathbb{R}^2 \mid 0 \leq x \leq \min \left\{ e^y , 1 , e^{1-y} \right\} \right\} \subset \mathbb{R}^2 \]
\end{ejercicio}

\begin{ejercicio}
    En cada uno de los siguientes casos, probar que la función \( f \) es integrable en \( \Omega \) y calcular su integral.
    \begin{enumerate}
        \item \(\Omega = \left\{ (x, y) \in \mathbb{R}^2 \mid 0 \leq x \leq 2 , y^2 \leq 2x \right\} \),
        \[ f(x, y) = \frac{x}{\sqrt{1 + x^2 + y^2}} \quad \forall (x, y) \in \Omega \]

        \item \(\Omega = \left\{ (x, y) \in \mathbb{R}^2 \mid x^2 + y^2 \leq 1 , x^2 + y^2 \leq 2x \right\} \),
        \[ f(x, y) = x \quad \forall (x, y) \in \Omega \]

        \item \(\Omega = \left\{ (x, y, z) \in \mathbb{R}^3 \mid 0 < x < y < z \right\} \),
        \[ f(x, y, z) = e^{-(x+y+z)} \quad \forall (x, y, z) \in \Omega \]
    \end{enumerate}

\end{ejercicio}

\begin{ejercicio}
    En cada uno de los siguientes casos, estudiar la integrabilidad de la función \( f \) en el conjunto \( \Omega \).
    \begin{enumerate}
        \item $f(x, y) = \dfrac{\cos (x y)}{(1 + y^2) \sqrt{\sen x}} \hspace{1cm} \forall (x, y) \in \Omega= \left]0, \nicefrac{\pi}{2}\right[ \times \mathbb{R}^+$,
        \item $f(x, y) = (x - y) e^{-(x-y)^2} \hspace{1cm} \forall (x, y) \in \Omega= \mathbb{R}^+ \times \mathbb{R}^+$,
        \item $f(x, y, z) = \dfrac{\cos x + \cos y + \cos z}{(1 + x^2 + y^2 + z^2)^3} \hspace{1cm} \forall (x, y, z) \in \Omega= \mathbb{R}^3$.
    \end{enumerate}
\end{ejercicio}

\begin{ejercicio}
    Probar que el siguiente conjunto es medible y calcular su volumen.
    \[ E = \left\{ (x, y, z) \in \left(\mathbb{R}^+_0\right)^3 \mid x + y + z \leq 1 \right\} \subset \mathbb{R}^3 \]
\end{ejercicio}
    
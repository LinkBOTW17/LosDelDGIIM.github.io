\chapter{Ejercicios Voluntarios}

\begin{teo}[Aproximación de Weierstrass]
    Sea $f:[0,1]\to \bb{R}$ una función continua. Entonces, existe una sucesión de polinomios $\{P_n\}$ de manera que $\{P_n\}$ converge uniformemente a $f$ en $[0,1]$.
\end{teo}
\begin{proof}
    Definimos la sucesión de polinomios de Bernstein como:
    \begin{equation*}
        B_n(f)(x) = \sum_{k=0}^n f\left(\frac{k}{n}\right)\binom{n}{k} x^k(1-x)^{n-k}
    \end{equation*}
    Tenemos claramente que $k, n-k\in \bb{N}$, por lo que $B_n(f)(x)$ es un polinomio. Fijado $x\in [0,1]$, calculemos el límite de $B_n(f)(x)$ cuando $n\to \infty$:
    \begin{align*}
        \lim_{n\to \infty} B_n(f)(x) &= \lim_{n\to \infty} \sum_{k=0}^n f\left(\frac{k}{n}\right)\binom{n}{k} x^k(1-x)^{n-k} \\
    \end{align*}
\end{proof}

\begin{definicion}
    Un monstruo de Weierstrass es una función continua en todos sus puntos que no es derivable en ningún punto.
\end{definicion}

\begin{ejercicio*}
    Encontrar un monstruo de Weierstrass y demostrar que lo es.

    Un ejemplo es el siguiente:
    \begin{equation*}
        \sum_{n=0}^\infty \frac{1}{n!}\cos\left((n!)^2 x\right)
    \end{equation*}
\end{ejercicio*}
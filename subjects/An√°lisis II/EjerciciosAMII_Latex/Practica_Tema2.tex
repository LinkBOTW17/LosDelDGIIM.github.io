\section{Series de funciones}

\begin{ejercicio}
    Probar que la serie $\sum\limits_{n\geq 1}f_n$ converge absoluta y uniformemente en $\bb{R}$, siendo:
    \begin{equation*}
        f_n(x) = \frac{x}{n\left(1+nx^2\right)} \qquad \forall x\in\bb{R},\quad \forall n\in\bb{N}
    \end{equation*}

    Buscaremos aplicar el Test de Weierstrass. Para ello, hemos de acotar $f_n(x)$ para todo $x\in\bb{R}$ y para todo $n\in\bb{N}$.
    En primer lugar, estudiaremos su monotonía. Para cada $n\in\bb{N}$, la función $f_n$ es derivable en $\bb{R}$ con:
    \begin{equation*}
        f_n'(x) = \frac{n\left(1+nx^2\right) -xn\cdot 2xn }{n^2\left(1+nx^2\right)^2}
        = \frac{n\left(1+nx^2\right)-2x^2n^2}{n^2\left(1+nx^2\right)^2}
        = \frac{\left(1+nx^2\right)-2x^2n}{n\left(1+nx^2\right)^2}
        = \frac{1-nx^2}{n\left(1+nx^2\right)^2}
    \end{equation*}

    Tenemos por tanto que hay dos candidatos a extremos relativos, $x=\pm \frac{1}{\sqrt{n}}$. Estudiaremos la monotonía en cada uno de los intervalos:
    \begin{itemize}
        \item {Si $x\in \left]-\infty,-\dfrac{1}{\sqrt{n}}\right]$:} $f_n'(x) \leq 0$, por lo que $f_n$ es decreciente.
        \item {Si $x\in \left[-\dfrac{1}{\sqrt{n}},\dfrac{1}{\sqrt{n}}\right]$:} $f_n'(x) \geq 0$, por lo que $f_n$ es creciente.
        \item {Si $x\in \left[\dfrac{1}{\sqrt{n}},+\infty\right[$:} $f_n'(x) \leq 0$, por lo que $f_n$ es decreciente.
    \end{itemize}
    
    Para acotar, tenemos en cuenta que:
    \begin{equation*}
        f\left(\frac{1}{\sqrt{n}}\right) = \frac{1}{2n\sqrt{n}} \qquad \text{y} \qquad f\left(-\frac{1}{\sqrt{n}}\right) = -\frac{1}{2n\sqrt{n}}
    \end{equation*}

    Sabiendo eso, acotamos en cada uno de los intervalos, teniendo en cuenta la monotonía:
    \begin{itemize}
        \item {Si $x\in \left]-\infty,-\dfrac{1}{\sqrt{n}}\right]$}:
        
        \begin{equation*}
            0\geq f_n(x) \geq f_n\left(-\frac{1}{\sqrt{n}}\right) = -\frac{1}{2n\sqrt{n}}
        \end{equation*}
        
        \item {Si $x\in \left[-\dfrac{1}{\sqrt{n}},\dfrac{1}{\sqrt{n}}\right]$}:
        
        \begin{equation*}
            -\frac{1}{2n\sqrt{n}} = f_n\left(-\frac{1}{\sqrt{n}}\right) \leq f_n(x) \leq f_n\left(\frac{1}{\sqrt{n}}\right) = \frac{1}{2n\sqrt{n}}
        \end{equation*}

        \item {Si $x\in \left[\dfrac{1}{\sqrt{n}},+\infty\right[$}:
        
        \begin{equation*}
            0\leq f_n(x) \leq f_n\left(\frac{1}{\sqrt{n}}\right) = \frac{1}{2n\sqrt{n}}
        \end{equation*}
    \end{itemize}

    En cualquier caso, uniendo los tres resultados, tenemos que:
    \begin{equation*}
        |{f_n(x)}| \leq \frac{1}{2n\sqrt{n}}=\frac{1}{2n^{\nicefrac{3}{2}}} \qquad \forall x\in\bb{R},\quad \forall n\in\bb{N}
    \end{equation*}

    Por el criterio límite de comparación, la serie $\sum\limits_{n\geq 1}\dfrac{1}{2n^{\nicefrac{3}{2}}}$ es convergente por serlo la
    serie $\sum\limits_{n\geq 1}\dfrac{1}{n^{\nicefrac{3}{2}}},~\left(\nicefrac{3}{2}>1\right)$.
    Por el Test de Weierstrass, la serie $\sum\limits_{n\geq 1}f_n$ converge absoluta y uniformemente en $\bb{R}$.
\end{ejercicio}



\begin{ejercicio}
    Para cada $n\in \bb{N}$, sea $g_n:[1, +\infty[ \to\bb{R}$ la función definida por:
    \begin{equation*}
        g_n(x) = \frac{1}{n^x} \qquad \forall x\in[1, +\infty[
    \end{equation*}

    Probar las siguientes afirmaciones:
    \begin{enumerate}
        \item Para $\rho \in \bb{R},~\rho > 1$, la serie $\sum\limits_{n\geq 1}g_n$ converge uniformemente en $[\rho, +\infty[$.
        
        Fijado $n\in \bb{N}$, sabemos que $g_n$ es derivable en $[\rho, +\infty[$ con:
        \begin{equation*}
            g_n'(x) = -\frac{\ln(n)}{n^{2x}} \qquad \forall x\in[\rho, +\infty[
        \end{equation*}

        Por tanto, como la primera derivada de $g_n$ no se anula, tenemos que es estrictamente monótona. Además,
        como $n\geq 1$, tenemos que $g_n'(x) \leq 0$ para todo $x\in[\rho, +\infty[$, por lo que $g_n$ es decreciente en $[\rho, +\infty[$. Por tanto,
        \begin{equation*}
            |g_n(x)| = g_n(x) \leq g_n(\rho) = \frac{1}{n^\rho} \qquad \forall x\in[\rho, +\infty[
        \end{equation*}
        Como la serie $\sum\limits_{n\geq 1}\nicefrac{1}{n^\rho}$ es convergente $\left(\rho > 1\right)$, por el Test de Weierstrass, la serie $\sum\limits_{n\geq 1}g_n$ converge uniformemente en $[\rho, +\infty[$.
        
        \item La sucesión $\{g_n\}$ converge uniformemente a la función nula en $[1, +\infty[$.
        
        De nuevo, usando que $g_n$ es decreciente en $[1, +\infty[$, tenemos que:
        \begin{equation*}
            |g_n(x)| = g_n(x) \leq g_n(1) = \frac{1}{n} \qquad \forall x\in[1, +\infty[
        \end{equation*}
        Como $\left\{\nicefrac{1}{n}\right\}\to 0$, tenemos que $\{g_n\}$ converge uniformemente a la función nula en $[1, +\infty[$.

        \item La serie $\sum\limits_{n\geq 1}g_n$ no converge uniformemente en $]1, +\infty[$.
        
        Por reducción al absurdo, supongamos que sí.
        Entonces, la serie $\sum\limits_{n\geq 1}g_n$ converge uniformemente en $]1, +\infty[$, y por el Criterio de Cauchy tenemos que
        esto equivale a que la sucesión de sumas parciales $\{S_n\}$ sea uniformemente de Cauchy en $]1, +\infty[$, es decir, que fijado
        $\veps\in \bb{R}^+$, existe $m\in \bb{N}$ tal que, para $m\leq p < q$, se tiene que:
        \begin{equation*}
            |S_q(x) - S_p(x)| = \left|\sum_{n=p+1}^q g_n(x)\right| = \sum_{n=p+1}^q \frac{1}{n^x} < \frac{\veps}{2} \qquad \forall x\in~]1, +\infty[
        \end{equation*}

        Por tanto, tomando límite cuando $x\to 1$, tenemos que:
        \begin{equation*}
            \sum_{n=p+1}^q \frac{1}{n} = \lim_{x\to 1}\sum_{n=p+1}^q \frac{1}{n^x} \leq \frac{\veps}{2} < \veps
        \end{equation*}

        Por tanto, la serie $\sum\limits_{n\geq 1}\nicefrac{1}{n}$ es una sucesión de Cauchy, y por tanto es convergente, lo cual es absurdo, ya que sabemos que
        la serie armónica no converge. Por tanto, la serie $\sum\limits_{n\geq 1}g_n$ no converge uniformemente en $]1, +\infty[$.
    \end{enumerate}
\end{ejercicio}



\begin{ejercicio}
    Estudiar la convergencia puntual, absoluta y uniforme de la serie de potencias
    \begin{equation*}
        \sum\limits_{n\geq 0} \frac{n!}{(n+1)^n}x^n \qquad \forall x\in\bb{R}
    \end{equation*}

    Estudiamos en primer lugar su campo de convergencia. La sucesión de coeficientes es
    $\{c_n\} = \left\{\dfrac{n!}{(n+1)^n}\right\}$. Tenemos que:
    \begin{align*}
        \left\{\frac{\left|c_{n+1}\right|}{\left|c_n\right|}\right\} =& \left\{\frac{c_{n+1}}{c_n}\right\}=\left\{\frac{(n+1)!}{(n+2)^{n+1}} \cdot \frac{(n+1)^n}{n!}\right\}
        = \left\{\left(\frac{n+1}{n+2}\right)^{n+1}\right\} =\\
        =& \left\{\left[\left(\frac{n+2}{n+1}\right)^{n+1}\right]^{-1}\right\} 
        = \left\{\left[\left(1+\frac{1}{n+1}\right)^{n+1}\right]^{-1}\right\} \to \frac{1}{e}
    \end{align*}
    Por el criterio de la raíz para sucesiones, tenemos que el radio de convergencia de la serie es:
    \begin{equation*}
        \frac{1}{R} = \limsup_{n\to +\infty}\sqrt[n]{\left|c_n\right|} = \frac{1}{e} \Longrightarrow R = e
    \end{equation*}
    Por tanto, el intervalo de convergencia es $J=]-e, e[$. Por tanto, tenemos que la serie converge absolutamente en $J$
    y uniformemente en cada conjunto compacto $K\subset J$. También sabemos que la serie no converge en ningún
    punto de $\bb{R}\setminus \ol{J}=\bb{R}\setminus [-e, e]$. Falta ahora por estudiar la convergencia puntual en $x=\pm e$ y la convergencia uniforme en $J$.
    \begin{itemize}
        \item \ul{Convergencia puntual en $x=\pm e$:}

        Equivale a ver si la serie
        $\sum\limits_{n\geq 0}c_nx^n$ es convergente, con $x=\pm e$ fijo. Tenemos que:
        \begin{equation*}
            \frac{c_{n+1}x^{n+1}}{c_nx^n} = \frac{c_{n+1}}{c_n}x = x\cdot \left[\left(1+\frac{1}{n+1}\right)^{n+1}\right]^{-1}
        \end{equation*}
        donde hemos usado los cálculos ya realizados. Además, sabemos que el segundo término converge a $\nicefrac{1}{e}$ y es estrictamente
        decreciente. Por tanto, tenemos que:
        \begin{equation*}
            \frac{\left|c_{n+1}x^{n+1}\right|}{\left|c_nx^n\right|}
            = \left|\frac{c_{n+1}x^{n+1}}{c_nx^n}\right|
            = \left|x\cdot \left[\left(1+\frac{1}{n+1}\right)^{n+1}\right]^{-1}\right|
            > \left|x\cdot \frac{1}{e}\right| = \left|\frac{x}{e}\right|=1
        \end{equation*}
        Por tanto, tenemos que la sucesión $\left\{|c_nx^n|\right\}$ es estrictamente creciente, por lo que $\{c_nx^n\}$ no puede
        converger a $0$.
        Por tanto, la serie $\sum\limits_{n\geq 0}c_nx^n$ no converge en $x=\pm e$ por no converger a $0$ su término general, luego su campo
        de convergencia es $J$.
        
        
        \item \ul{Convergencia uniforme en $J$:}
        
        Supongamos que la serie $\sum\limits_{n\geq 0}c_nx^n$ converge uniformemente en $J$. Entonces, por el Criterio de Cauchy, la sucesión de sumas parciales
        $\{S_n\}$ es uniformemente de Cauchy en $J$, es decir, que fijado $\veps\in \bb{R}^+$, existe $m\in \bb{N}$ tal que, para $m\leq p < q$, se tiene que:
        \begin{equation*}
            |S_q(x) - S_p(x)| = \left|\sum_{n=p+1}^q c_nx^n\right| < \frac{\veps}{2} \qquad \forall x\in J
        \end{equation*}

        Tomando límite cuando $x\to e$, tenemos que:
        \begin{equation*}
            \lim_{x\to e}\sum_{n=p+1}^q c_nx^n \leq \frac{\veps}{2} < \veps
        \end{equation*}

        No obstante, esto es un absurdo, ya que al tomar límite con $x$ tendiendo a $e$, la serie $\sum\limits_{n\geq 0}c_nx^n$ diverge por divergir
        su término general. Por tanto, la serie $\sum\limits_{n\geq 0}c_nx^n$ no converge uniformemente en $J$.
    \end{itemize}
\end{ejercicio}


\begin{ejercicio}
    Para cada $n\in \bb{N}$, sea $f_n:~]-1,1[ \to \bb{R}$ la función definida por:
    \begin{equation*}
        f_n(x) = \frac{x^n}{1-x^n} \qquad \forall x\in]-1,1[
    \end{equation*}
    Probar que la serie $\sum\limits_{n\geq 1}f_n$ converge absolutamente en $]-1,1[$ y uniformemente
    en cada conjunto compacto $K\subset~]-1,1[~$; pero no converge uniformemente en $]-1,1[$.\\
    
    En $\bb{R}$, los conjuntos compactos son los cerrados y acotados. Por tanto, se tiene que $K\subseteq [-\rho, \rho]\subsetneq ~]-1,1[~$, con $\rho\in \bb{R}$. Tenemos que:
    \begin{equation*}
        |f_n(x)| = \left|\frac{x^n}{1-x^n}\right| = \frac{|x^n|}{|1-x^n|} \leq \frac{|x^n|}{1-|x^n|}
        = \frac{|x|^n}{1-|x|^n} \leq \frac{\rho^n}{1-\rho^n} \qquad \forall x\in K,~\forall n\in \bb{N}
    \end{equation*}

    Para ver si la serie de término general $a_n = \dfrac{\rho^n}{1-\rho^n}$ es convergente,
    usamos el criterio límite de comparación con la serie de término general $b_n = \rho^n$, que sabemos que es convergente por ser $|\rho|=\rho < 1$.
    \begin{equation*}
        \left\{\frac{a_n}{b_n}\right\}
        = \left\{\frac{1}{1-\rho^n}\right\} \to 1 \Longrightarrow \sum_{n\geq 1}a_n \text{ es convergente}
    \end{equation*}

    Por tanto, por el Test de Weierstrass, la serie $\sum\limits_{n\geq 1}f_n$ converge absoluta y uniformemente en $K$.\\

    Estudiamos ahora la convergencia absoluta en $]-1,1[$.
    \begin{description}
        \item[Opción 1.] Forma rutinaria.
        
        Fijando $x\in ]-1,1[$, para ver si
        la serie $\sum\limits_{n\geq 1}|f_n(x)|$ es convergente, usamos el criterio límite de comparación
        con la serie de término general $a_n = |x|^n$, que sabemos que es convergente por ser $|x|<1$.
        \begin{equation*}
            \left\{\frac{|f_n(x)|}{|x|^n}\right\} = \left\{\frac{1}{1-|x|^n}\right\}\to 1 \Longrightarrow \sum_{n\geq 1}|f_n(x)| \text{ es convergente}
        \end{equation*}
        Por tanto, la serie $\sum\limits_{n\geq 1}f_n$ converge absolutamente en $]-1,1[$.

        \item[Opción 2.] Usando unión de compactos.
        
        Como converge absolutamente en cada compacto $K\subset~]-1,1[~$, tenemos que converge absolutamente en $\left[-1+\nicefrac{1}{n}, 1-\nicefrac{1}{n}\right]$ para todo $n\in \bb{N}$.
        Por tanto, y por ser la convergencia absoluta una propiedad local, tenemos que converge absolutamente en la unión de todos estos conjuntos, es decir, es:
        \begin{equation*}
            \bigcup_{n\in \bb{N}}\left[-1+\nicefrac{1}{n}, 1-\nicefrac{1}{n}\right] = ]-1,1[
        \end{equation*}
        Por tanto, la serie $\sum\limits_{n\geq 1}f_n$ converge absolutamente en $]-1,1[$.
    \end{description}

    Tan solo falta por ver que no converge uniformemente en $]-1,1[$. Por reducción al absurdo, supongamos que sí.
    Entonces, por el Criterio de Cauchy tenemos que la sucesión de sumas parciales $\{S_n\}$ es uniformemente de Cauchy en $]-1,1[$, es decir, que fijado
    $\veps\in \bb{R}^+$, existe $m\in \bb{N}$ tal que, para $m\leq p < q$, se tiene que:
    \begin{equation*}
        |S_q(x) - S_p(x)| = \left|\sum_{n=p+1}^q f_n(x)\right|
        = \left|\sum_{n=p+1}^q \frac{x^n}{1-x^n}\right| < \frac{\veps}{2} \qquad \forall x\in~]-1,1[
    \end{equation*}
    
    Tomando límite cuando $x\to 1$, tenemos que:
    \begin{equation*}
        \lim_{x\to 1}\sum_{n=p+1}^q \frac{x^n}{1-x^n} \leq \lim_{x\to 1}\left|\sum_{n=p+1}^q \frac{x^n}{1-x^n}\right| \leq \frac{\veps}{2} < \veps
    \end{equation*}

    No obstante, esto es un absurdo, ya que al tomar límite con $x$ tendiendo a $1$, la serie $\sum\limits_{n\geq 1}\dfrac{x^n}{1-x^n}$ no converge por no converger a 0 su término su general, veámoslo.
    Sea $x_n=\frac{1}{\sqrt[n]{2}}$ y tenemos que $\sqrt[n]{2}>1 \Longleftrightarrow 2>1^n=1$, por lo que $x_n\in~]0,1[$.
    \begin{equation*}
        \left\{f_n(x_n)-0\right\} = \left\{\frac{\left(\frac{1}{\sqrt[n]{2}}\right)^n}{1-\left(\frac{1}{\sqrt[n]{2}}\right)^n}\right\} = \left\{\frac{\nicefrac{1}{2}}{1-\nicefrac{1}{2}}\right\} = \left\{1\right\}\to 1\neq 0
    \end{equation*}
    
    Por tanto, la serie $\sum\limits_{n\geq 1}f_n$ no converge uniformemente en $]-1,1[$.
\end{ejercicio}


\begin{ejercicio}
    Fijado $\alpha\in \bb{R}^+$, se define:
    \begin{equation*}
        g_n(x) = \frac{1}{n^\alpha} \arctg \frac{x}{n} \qquad \forall x\in\bb{R},~\forall n\in \bb{N}
    \end{equation*}
    Probar que la serie de funciones $\sum\limits_{n\geq 1}g_n$ converge absoluta y uniformemente en cada
    subconjunto acotado de $\bb{R}$ y que, si $\alpha > 1$, dicha serie converge absoluta y uniformemente en $\bb{R}$.\\

    Supongamos en primer lugar que $\alpha>1$. Entonces, tenemos que:
    \begin{equation*}
        \left|g_n(x)\right| = \left|\frac{1}{n^\alpha} \arctg \frac{x}{n}\right| \leq \frac{1}{n^\alpha}\cdot \frac{\pi}{2}
    \end{equation*}
    donde he aplicado que la $\arctan$ está acotada por $\frac{\pi}{2}$. Por tanto, como
    $\sum\limits_{n\geq 1}\dfrac{1}{n^\alpha}$ es convergente $(\alpha>1)$, tenemos que $\sum\limits_{n\geq 1}\frac{1}{n^\alpha} \cdot \frac{\pi}{2}$ es convergente;
    y por el Test de Weierstrass, la serie $\sum\limits_{n\geq 1}g_n$ converge absoluta y uniformemente en $\bb{R}$.\\

    Sin suponer ahora que $\alpha>1$, veamos que la serie $\sum\limits_{n\geq 1}g_n$ converge absoluta y uniformemente en cada
    subconjunto acotado de $\bb{R}$. Fijado $C\subset \bb{R}$ acotado, existe $M\in \bb{R}^+$ tal que $|x|\leq M$ para todo $x\in C$.
    Notemos que vamos a necesitar acotar por $\dfrac{1}{n^{\alpha+1}}$, por lo que buscamos acotar $\arctan\left(\dfrac{x}{n}\right)$ por $\dfrac{M}{n}$.
    Para ello, veremos que $\arctan x \leq x$ para todo $x\in \bb{R}^+_0$.
    \begin{description}
        \item[Opción 1:] Usando la definición mediante integrales de la arcotangente.
        
        En efecto, si $x\in \bb{R}^+_0$, tenemos que:
        \begin{equation*}
            \arctan x = \int_0^x \frac{1}{1+t^2}dt \leq \int_0^x 1dt = x \qquad \forall x\in \bb{R}^+_0
        \end{equation*}

        donde hemos usado que $\dfrac{1}{1+t^2}\leq 1$ para todo $t\in \bb{R}^+_0$.

        \item[Opción 2:] Calcular la imagen de una función auxiliar.
        
        Definimos la siguiente función auxiliar:
        \Func{f}{\bb{R}^+}{\bb{R}}{x}{\arctan x -x}
        
        Tenemos que es derivable en $\bb{R}^+_0$ con:
        \begin{equation*}
            f'(x) = \frac{1}{1+x^2} - 1 = \frac{1-1-x^2}{1+x^2} = -\frac{x^2}{1+x^2} < 0 \qquad \forall x\in \bb{R}^+
        \end{equation*}
        Por tanto, $f$ es decreciente en $\bb{R}^+_0$, por lo que se tiene que $f(x) \leq f(0) = 0$ para todo $x\in \bb{R}^+_0$.
        Como $f(x)=\arctan x -x\leq 0$, tenemos que $\arctan x \leq x$ para todo $x\in \bb{R}^+_0$. 
    \end{description}

    En cualquier caso, hemos demostrado que $\arctan x \leq x$ para todo $x\in \bb{R}^+_0$. Supongamos ahora que $x\in \bb{R}^-$. Usando que
    $\arctan$ es impar $(\arctan(-x) = -\arctan x)$, tenemos que:
    \begin{equation*}
        \arctan x = -\arctan(-x) \geq -(-x) = x \qquad \forall x\in \bb{R}^-
    \end{equation*}
    
    Como la función valor absoluto es creciente en $\bb{R}^+_0$ y decreciente en $\bb{R}^-$, tenemos que:
    \begin{equation*}
        \left|\arctan x\right| \leq |x| \qquad \forall x\in \bb{R}
    \end{equation*}

    Por tanto, para $x\in C$, tenemos que:
    \begin{equation*}
        \left|g_n(x)\right| = \left|\frac{1}{n^\alpha} \arctg \frac{x}{n}\right| \leq \frac{1}{n^\alpha} \cdot \frac{|x|}{n} \leq \frac{1}{n^\alpha} \cdot \frac{M}{n} = \frac{M}{n^{\alpha+1}} \qquad \forall x\in C,~\forall n\in \bb{N}
    \end{equation*}

    Por tanto, como $\sum\limits_{n\geq 1}\dfrac{M}{n^{\alpha+1}}$ es convergente $(\alpha+1>1)$, por el Test de Weierstrass, la serie $\sum\limits_{n\geq 1}g_n$ converge absoluta y uniformemente en $C$.    
\end{ejercicio}

\begin{ejercicio}
    Para cada $n\in \bb{N}$, sea $h_n:\bb{R}\to \bb{R}$ la función definida por:
    \begin{equation*}
        h_n(x) = \frac{1}{n}\sen(nx)\log\left(1+\frac{|x|}{n}\right) \qquad \forall x\in \bb{R}
    \end{equation*}
    Probar que la serie de funciones $\sum\limits_{n\geq 1}h_n$ converge absoluta y uniformemente en cada
    subconjunto acotado de $\bb{R}$.\\

    Para probar lo pedido, en primer lugar, demostraremos que $\log x \leq x-1$ para todo $x\in \bb{R}^+$. Para ello, hay dos opciones:
    \begin{description}
        \item[Opción 1:] Usando la definición mediante integrales del logaritmo.
        
        En efecto, si $x\geq 1$, tenemos que:
        \begin{equation*}
            \log x = \int_1^x \frac{1}{t}dt < \int_1^x 1dt = x-1 \qquad \forall x\in ]1,+\infty[
        \end{equation*}
        donde hemos usado que $\dfrac{1}{t}\leq 1$ para todo $t\geq 1$.

        Si $x\in \left]0,1\right[$, tenemos que:
        \begin{equation*}
            \log x = -\int_x^1 \frac{1}{t}dt \leq -\int_x^1 1dt = -(1-x)=x-1 \qquad \forall x\in \left]0,1\right[
        \end{equation*}
        donde hemos usado que $\dfrac{1}{t}\geq 1$ para todo $t\in \left]0,1\right[$.

        \item[Opción 2:] Calcular la imagen de una función auxiliar.
        
        Definimos la siguiente función auxiliar:
        \Func{f}{\bb{R}^+}{\bb{R}}{x}{\log x -x+1}
        
        Tenemos que es derivable en $\bb{R}^+$ con:
        \begin{equation*}
            f'(x) = \frac{1}{x} - 1 = 0 \Longleftrightarrow x=1
        \end{equation*}
        Por tanto, $f$ tiene un único punto crítico en $x=1$. Además, $f''(x) = -\dfrac{1}{x^2} < 0$ para todo $x\in \bb{R}^+$, por lo que $f$ tiene un máximo en $x=1$.
        Por tanto, $f(1)=0$ y $f(x)\leq 0$ para todo $x\in \bb{R}^+$, por lo que $\log x -x+1\leq 0$ para todo $x\in \bb{R}^+$.
    \end{description}

    En cualquier caso, tenemos que:
    \begin{equation*}
        \left|h_n(x)\right|
        = \left|\frac{1}{n}\sen(nx)\log\left(1+\frac{|x|}{n}\right)\right|
        \leq \frac{1}{n}\cdot 1\cdot \left(1+\frac{|x|}{n}-1\right) = \frac{|x|}{n^2} \qquad \forall x\in \bb{R},~\forall n\in \bb{N}
    \end{equation*}
    donde además hemos usado que $|\sen(nx)|\leq 1$ para todo $x\in \bb{R}$ y $n\in \bb{N}$. Fijado $C\subset \bb{R}$ acotado, existe $M\in \bb{R}^+$ tal que $|x|\leq M$ para todo $x\in C$. Tenemos por tanto que:
    \begin{equation*}
        \left|h_n(x)\right| \leq \frac{M}{n^2} \qquad \forall x\in C,~\forall n\in \bb{N}
    \end{equation*}

    Por tanto, como $\sum\limits_{n\geq 1}\dfrac{M}{n^2}$ es convergente, por el Test de Weierstrass, la serie $\sum\limits_{n\geq 1}h_n$ converge absoluta y uniformemente en $C$.
\end{ejercicio}


\begin{ejercicio}
    Estudiar la convergencia puntual, absoluta y uniforme de las siguientes series de funciones:
    \begin{enumerate}
        \item $\sum\limits_{n\geq 1}\dfrac{x^n}{\log(n+2)}$.
        
        Estudiamos en primer lugar su campo de convergencia. La sucesión de coeficientes es
        $\{c_n\} = \left\{\dfrac{1}{\log(n+2)}\right\}$. Tenemos que:
        \begin{equation*}
            \left\{\frac{\left|c_{n+1}\right|}{\left|c_n\right|}\right\}
            = \left\{\frac{1}{\log(n+3)}\cdot \frac{\log(n+2)}{1}\right\}
            = \left\{\frac{\log(n+2)}{\log(n+3)}\right\} \to 1
        \end{equation*}

        Por el criterio de la raíz para sucesiones, tenemos que el radio de convergencia de la serie es:
        \begin{equation*}
            \frac{1}{R} = \limsup_{n\to +\infty}\sqrt[n]{\left|c_n\right|} = 1 \Longrightarrow R = 1
        \end{equation*}

        Por tanto, el intervalo de convergencia es $J=]-1, 1[$. Por tanto, tenemos que la serie converge absolutamente en $J$
        y uniformemente en cada conjunto compacto $K\subset J$. También sabemos que la serie no converge en ningún
        punto de $\bb{R}\setminus \ol{J}=\bb{R}\setminus [-1, 1]$.
        Falta ahora por estudiar la convergencia puntual en $x=\pm 1$ y la convergencia uniforme en $J$.
        \begin{itemize}
            \item \ul{Convergencia puntual en $x=1$:}
            
            Se trata de estudiar la convergencia de la serie $\sum\limits_{n\geq 1}\dfrac{1}{\log(n+2)}$.
            Como $\log(n+2)\leq n+1$ para todo $n\in \bb{N}$ (visto en el ejercicio anterior), tenemos que:
            \begin{equation*}
                \frac{1}{\log(n+2)} \geq \frac{1}{n+1}
            \end{equation*}

            Usando el contrarrecíproco del Criterio de Comparación, como la serie $\sum\limits_{n\geq 1}\dfrac{1}{n+1}$ no converge, tenemos que la serie $\sum\limits_{n\geq 1}\dfrac{1}{\log(n+2)}$ no converge. Por tanto, la serie $\sum\limits_{n\geq 1}\dfrac{x^n}{\log(n+2)}$ no converge en $x=1$.

            \item \ul{Convergencia puntual en $x=-1$:}
            
            Se trata de estudiar la convergencia de la serie $\sum\limits_{n\geq 1}\dfrac{(-1)^n}{\log(n+2)}$. Por el Criterio
            de Leibnitz, sabemos que la serie converge si la sucesión $\left\{\dfrac{1}{\log(n+2)}\right\}$ converge a $0$ y es decreciente.
            Como $\{\log(n+2)\}$ es estrictamente creciente y diverge positivamente, tenemos que $\left\{\dfrac{1}{\log(n+2)}\right\}$ converge a $0$ y es decreciente. Por tanto, la serie $\sum\limits_{n\geq 1}\dfrac{(-1)^n}{\log(n+2)}$ converge, y por tanto la serie $\sum\limits_{n\geq 1}\dfrac{x^n}{\log(n+2)}$ converge en $x=-1$.

            \item \ul{Convergencia uniforme en $J$:}
            
            Suponemos por reducción al absurdo que la serie $\sum\limits_{n\geq 1}\dfrac{x^n}{\log(n+2)}$ converge uniformemente en $J$. Entonces, por el Criterio de Cauchy, tenemos que la sucesión de sumas parciales $\{S_n\}$ es uniformemente de Cauchy en $J$, es decir, que fijado
            $\veps\in \bb{R}^+$, existe $m\in \bb{N}$ tal que, para $m\leq p < q$, se tiene que:
            \begin{equation*}
                |S_q(x) - S_p(x)| = \left|\sum_{n=p+1}^q \frac{x^n}{\log(n+2)}\right| < \frac{\veps}{2} \qquad \forall x\in~J
            \end{equation*}

            Tomando límite cuando $x\to 1$, tenemos que:
            \begin{equation*}
                \sum_{n=p+1}^q \frac{1}{\log(n+2)} = \lim_{x\to 1}\sum_{n=p+1}^q \frac{x^n}{\log(n+2)} \leq \frac{\veps}{2} < \veps
            \end{equation*}
            Por tanto, la serie $\sum\limits_{n\geq 1}\dfrac{1}{\log(n+2)}$ es una sucesión de Cauchy, y por tanto es convergente, lo cual es absurdo, ya que se ha visto que no converge para $x=1$. Por tanto, la serie $\sum\limits_{n\geq 1}\dfrac{x^n}{\log(n+2)}$ no converge uniformemente en $J$.
        \end{itemize}


        \item $\sum\limits_{n\geq 1} \dfrac{n}{2^n}(x-1)^n$.
        
        Estudiamos en primer lugar su campo de convergencia. La sucesión de coeficientes es
        $\{c_n\} = \left\{\dfrac{n}{2^n}\right\}$. Tenemos que:
        \begin{equation*}
            \left\{\frac{\left|c_{n+1}\right|}{\left|c_n\right|}\right\}
            = \left\{\frac{n+1}{2^{n+1}}\cdot \frac{2^n}{n}\right\}
            = \left\{\frac{n+1}{2n}\right\} \to \frac{1}{2}
        \end{equation*}

        Por el criterio de la raíz para sucesiones, tenemos que el radio de convergencia de la serie es:
        \begin{equation*}
            \frac{1}{R} = \limsup_{n\to +\infty}\sqrt[n]{\left|c_n\right|} = \frac{1}{2} \Longrightarrow R = 2
        \end{equation*}

        Por tanto, el intervalo de convergencia es $J=]-1, 3[$. Por tanto, tenemos que la serie converge absolutamente en $J$
        y uniformemente en cada conjunto compacto $K\subset J$. También sabemos que la serie no converge en ningún
        punto de $\bb{R}\setminus \ol{J}=\bb{R}\setminus [-1, 3]$. Falta ahora por estudiar la convergencia puntual en $x=-1$ y $x=3$ y la convergencia uniforme en $J$.

        \begin{itemize}
            \item \ul{Convergencia puntual en $x=-1$:}
            
            Se trata de estudiar la convergencia de la serie siguiente:
            \begin{equation*}
                \sum\limits_{n\geq 1}\dfrac{n}{2^n}(-2)^n
                = \sum\limits_{n\geq 1}\dfrac{n}{2^n}(-1)^n2^n = \sum\limits_{n\geq 1}(-1)^n n
            \end{equation*}
            
            Por el criterio básico de convergencia, como el término general no converge a $0$, la serie $\sum\limits_{n\geq 1}(-1)^n n$ no converge, por lo que la serie $\sum\limits_{n\geq 1} \dfrac{n}{2^n}(x-1)^n$ no converge en $x=-1$.

            \item \ul{Convergencia puntual en $x=3$:}
            
            Se trata de estudiar la convergencia de la serie siguiente:
            \begin{equation*}
                \sum\limits_{n\geq 1}\dfrac{n}{2^n}2^n
                = \sum\limits_{n\geq 1}n
            \end{equation*}

            Por el criterio básico de convergencia, como el término general no converge a $0$, la serie $\sum\limits_{n\geq 1} n$ no converge, por lo que la serie $\sum\limits_{n\geq 1} \dfrac{n}{2^n}(x-1)^n$ no converge en $x=3$.
            \item \ul{Convergencia uniforme en $J$:}
            
            Suponemos por reducción al absurdo que la serie $\sum\limits_{n\geq 1}\dfrac{n}{2^n}(x-1)^n$ converge uniformemente en $J$. Entonces, por el Criterio de Cauchy, tenemos que la sucesión de sumas parciales $\{S_n\}$ es uniform
            mente de Cauchy en $J$, es decir, que fijado
            $\veps\in \bb{R}^+$, existe $m\in \bb{N}$ tal que, para $m\leq p < q$, se tiene que:
            \begin{equation*}
                |S_q(x) - S_p(x)| = \left|\sum_{n=p+1}^q \frac{n}{2^n}(x-1)^n\right| < \frac{\veps}{2} \qquad \forall x\in~J
            \end{equation*}

            Tomando límite cuando $x\to 3$, tenemos que:
            \begin{equation*}
                \sum_{n=p+1}^q \frac{n}{2^n}2^n = \lim_{x\to 3}\sum_{n=p+1}^q \frac{n}{2^n}(x-1)^n \leq \frac{\veps}{2} < \veps
            \end{equation*}
            Por tanto, la serie $\sum\limits_{n\geq 1}\dfrac{n}{2^n}2^n$ es una sucesión de Cauchy, y por tanto es convergente, lo cual es absurdo, ya que se ha visto que no converge para $x=3$. Por tanto, la serie $\sum\limits_{n\geq 1}\dfrac{n}{2^n}(x-1)^n$ no converge uniformemente en~$J$.
        \end{itemize}
    \end{enumerate}
\end{ejercicio}
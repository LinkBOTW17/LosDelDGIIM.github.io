\section{Sucesiones de funciones}

\begin{ejercicio}
    Para cada $n\in \bb{N}$, sea $f_n:\bb{R}^+_0\to \bb{R}$ 
    la función definida como:
    \begin{equation*}
        f_n(x) = \frac{\log(1+nx)}{1+nx} \qquad \forall x\in \bb{R}^+_0
    \end{equation*}
    Fijado un $\rho\in \bb{R}^+$, estudiar la convergencia uniforme de la sucesión
    $\{f_n\}$ en el intervalo $[0,\rho]$ y en la semirrecta $[\rho,+\infty[$.\\

    Estudiamos en primer lugar la convergencia puntual. Para $x=0$, tenemos que:
    \begin{equation*}
        f_n(0) = \frac{\log(1)}{1} = 0
    \end{equation*}
    Por tanto, es la sucesión constante $0$, por lo que $\{f_n\}$ converge puntualmente a la función nula en $0$.
    Para $x\in \bb{R}^+$, tenemos que:
    \begin{equation*}
        \lim_{n\to \infty} f_n(x) = \lim_{n\to \infty} \frac{\log(1+nx)}{1+nx} \Hop
        \lim_{n\to \infty} \frac{\dfrac{x}{1+nx}}{x} =
        \lim_{n\to \infty} \frac{1}{1+nx} = 0
    \end{equation*}

    En resumen, tenemos que $\{f_n\}$ converge puntualmente a la función nula en $\bb{R}^+_0$.\\

    Para estudiar la convergencia uniforme, estudiamos la monotonía de la función $f_n$
    para cada $n\in \bb{N}$. Para ello, como $f_n\in C^\infty(\bb{R}^+_0)$, estudiamos
    su derivada:
    \begin{equation*}
        f_n'(x) = \frac{\frac{n}{1+nx}\cdot (1+nx) - \log(1+nx)\cdot n}{(1+nx)^2} =
        \frac{n - \log(1+nx)\cdot n}{(1+nx)^2}
    \end{equation*}

    Por tanto, tenemos que los candidatos a extremos relativos de $f_n$ son:
    \begin{equation*}
        f_n'(x) = 0 \iff \log(1+nx) = 1 \iff 1+nx = e \iff x = \frac{e-1}{n}
    \end{equation*}

    Evalucando la primera derivada en cada intervalo, tenemos que:
    \begin{itemize}
        \item Si $x\in \left[0,\frac{e-1}{n}\right]$, entonces $f_n'(x) > 0$, por lo que $f_n$ es creciente para todo $n\in \bb{N}$.
        \item Si $x\in \left[\frac{e-1}{n},+\infty\right[$, entonces $f_n'(x) < 0$, por lo que $f_n$ es decreciente para todo $n\in \bb{N}$.
    \end{itemize}
    
    Estudiamos ahora la convergencia uniforme. Fijado $\rho \in \bb{R}^+$, definimos la sucesión $\{x_n\}$ de la siguiente forma:
    \begin{itemize}
        \item Si $n < \dfrac{e-1}{\rho}~\left(\rho < \dfrac{e-1}{n}\right)$, entonces $x_n = \rho \in [0,\rho]$ (podría haber tomado cualquier valor $x_n\in [0,\rho]$, ya que no afecta al límite).
        \item Si $n \geq \dfrac{e-1}{\rho}~\left(\rho \geq \dfrac{e-1}{n}\right)$, entonces $x_n = \dfrac{e-1}{n}\in [0, \rho]$.
    \end{itemize}

    De esta forma, tenemos que $\{x_n\}$ es una sucesión de puntos de $[0,\rho]$. Veamos lo siguiente:
    \begin{equation*}
        f_n\left(\frac{e-1}{n}\right) = \frac{\log\left(1+n\cdot \frac{e-1}{n}\right)}{1+n\cdot \frac{e-1}{n}} =
        \frac{\log(e)}{e} = \frac{1}{e} \qquad \forall n\in \bb{N}
    \end{equation*}

    Por tanto, como se tiene que $\{f_n(x_n)-f(x_n)\}=\{f_n(x_n)\}\to \dfrac{1}{e}$, tenemos que \ul{$\{f_n\}$ no converge uniformemente en $[0,\rho]$}.
    \begin{observacion}
        También sirve tomar $x_n = \frac{1}{n}$, y tendríamos que $f_n\left(\frac{1}{n}\right) = \frac{\log 2}{2}$.
    \end{observacion}~\\

    Para el caso de la semirrecta $[\rho,+\infty[$, tomamos $m\in \bb{N}$ tal que $\rho > \frac{e-1}{m}$. De esta forma,
    para $n\in \bb{N}$, $n\geq m$, tenemos también que $\rho > \frac{e-1}{n}$. Por tanto,
    tenemos que $[\rho ,+\infty[~\subset \left[\frac{e-1}{n},+\infty\right[$,
    por lo que $f_n$ es decreciente en $[\rho,+\infty[$. Por tanto, para $n\geq m$, tenemos que:
    \begin{equation*}
        |f_n(x)| = f_n(x) \leq f_n(\rho) \qquad \forall x\in [\rho,+\infty[
    \end{equation*}

    Además, por la convergencia puntual, tenemos que $\{f_n(p)\}\to 0$, por lo que se deduce que \ul{$\{f_n\}$ converge uniformemente en $[\rho,+\infty[$}.
\end{ejercicio}



\begin{ejercicio}
    Probar que la sucesión $\{g_n\}$ converge uniformemente en $\bb{R}$, donde $g_n:\bb{R}\to \bb{R}$ está definida como:
    \begin{equation*}
        g_n(x) = \sqrt[n]{1+x^{2n}} \qquad \forall x\in \bb{R},~\forall n\in \bb{N}
    \end{equation*}

    Estudiemos en primer lugar la convergencia puntual. Distinguimos en función
    del valor de $x$:
    \begin{itemize}
        \item Si $|x| < 1$, entonces para todo $n\in \bb{N}$, tenemos que:
        \begin{equation*}
            1 \leq 1+x^{2n} \leq 1+1 = 2 \Longrightarrow 1 \leq g_n(x) \leq \sqrt[n]{2}
        \end{equation*}

        Como $\{\sqrt[n]{2}\}\to 1$, por el Lema del Sándwich tenemos que $\{g_n(x)\}\to 1$.

        \item Si $|x| = 1$, entonces para todo $n\in \bb{N}$, tenemos que:
        \begin{equation*}
            g_n(x) = \sqrt[n]{1+x^{2n}} = \sqrt[n]{1+1} = \sqrt[n]{2}
        \end{equation*}

        Por tanto, $\{g_n(x)\}\to 1$.

        \item Si $|x| > 1$, entonces para todo $n\in \bb{N}$, tenemos que:
        \begin{equation*}
            g_n(x) = \sqrt[n]{1+x^{2n}} = x^2\sqrt[n]{\frac{1}{x^{2n}}+1}
        \end{equation*}
        Como $\left\{\frac{1}{x^{2n}}\right\}\to 0$, tenemos que $\{g_n(x)\}\to x^2$.
    \end{itemize}

    Por tanto, tenemos que $\{g_n\}$ converge puntualmente a la función:
    \begin{equation*}
        g(x) = \max\{1,x^2\}= \begin{cases}
            1 & \text{si } |x| \leq 1\\
            x^2 & \text{si } |x| > 1
        \end{cases}
    \end{equation*}

    Para la convergencia uniforme, en primer lugar tenemos en cuenta que:
    \begin{equation*}
        \sqrt[n]{1+x^{2n}}\geq \sqrt[n]{x^{2n}}=x^2, \sqrt[n]{1}=1 \Longrightarrow \sqrt[n]{1+x^{2n}}\geq \max\{1,x^2\}=g(x)
        \qquad \forall x\in \bb{R},~n\in \bb{N}
    \end{equation*}

    Por tanto, buscamos acotar $|g_n(x)-g(x)|=g_n(x)-g(x)$. Para ello, fijado $n\in \bb{N}$, usaremos la función
    \Func{\varphi_n}{\bb{R}^+}{\bb{R}^+}{t}{t^{\nicefrac{1}{n}}=\sqrt[n]{t}}
    Tenemos que es derivable en todo su dominio, y su derivada es:
    \begin{equation*}
        \varphi'(t) = \frac{1}{n}\cdot t^{\frac{1}{n}-1}
        = \frac{1}{n\cdot t^{\nicefrac{n-1}{n}}} \qquad \forall t\in \bb{R}^+
    \end{equation*}

    Por el Teorema del valor medio, tenemos que para todo $t_1,t_2\in \bb{R}^+$, con $t_1<t_2$, existe un $c\in ]t_1,t_2[$ tal que:
    \begin{equation*}
        \varphi(t_2)-\varphi(t_1) = \varphi'(c)\cdot (t_2-t_1)
    \end{equation*}

    Diferenciamos ahora si $|x|\leq 1$ o $|x|>1$:
    \begin{itemize}
        \item Si $|x|\leq 1$, aplicamos el Teorema del valor medio a la función $\varphi_n$ en el intervalo $[1,1+x^{2n}]$,
        obteniendo que existe un $c\in ]1,1+x^{2n}[$ tal que:
        \begin{equation*}
            g_n(x)-g(x)=\sqrt[n]{1+x^{2n}}-1 = \varphi_n(1+x^{2n})-\varphi_n(1) = \varphi_n'(c)\cdot (1+x^{2n}-1)
            = \frac{x^{2n}}{n\cdot c^{\nicefrac{n-1}{n}}}
        \end{equation*}
        Como $|x|\leq 1$, tenemos que $|x^{2n}|\leq 1$; y como $c>1$ y $\frac{n-1}{n}>1$, tenemos que $c^{\nicefrac{n-1}{n}}>1$, por lo que:
        \begin{equation*}
            |g_n(x)-g(x)| = \frac{x^{2n}}{n\cdot c^{\nicefrac{n-1}{n}}}< \frac{1}{n} \qquad \forall x\in [-1,1],~n\in \bb{N}
        \end{equation*}

        \item Si $|x|>1$, aplicamos el Teorema del valor medio a la función $\varphi_n$ en el intervalo $[x^{2n},1+x^{2n}]$,
        obteniendo que existe un $d\in ]x^{2n},1+x^{2n}[$ tal que:
        \begin{equation*}
            g_n(x)-g(x)=\sqrt[n]{1+x^{2n}}-x^2 = \varphi_n(1+x^{2n})-\varphi_n(x^{2n}) = \varphi_n'(d)\cdot (1+x^{2n}-x^{2n})
            = \frac{1}{n\cdot d^{\nicefrac{n-1}{n}}}
        \end{equation*}
        Como $|x|>1$, tenemos que $|x^{2n}|>1$, por tanto, $d>1$. Como también se tiene que $\frac{n-1}{n}>1$, tenemos que $d^{\nicefrac{n-1}{n}}>1$, por lo que:
        \begin{equation*}
            |g_n(x)-g(x)| = \frac{1}{n\cdot d^{\nicefrac{n-1}{n}}}< \frac{1}{n} \qquad \forall x\in \bb{R}\setminus [-1,1],~n\in \bb{N}
        \end{equation*}
    \end{itemize}

    Por tanto, uniendo ambos resultados se tiene que:
    \begin{equation*}
        |g_n(x)-g(x)|<\frac{1}{n} \qquad \forall x\in \bb{R},~n\in \bb{N}
    \end{equation*}

    Por tanto, como $\left\{\frac{1}{n}\right\}\to 0$, tenemos que \ul{$\{g_n\}$ converge uniformemente en $\bb{R}$}.
\end{ejercicio}



\begin{ejercicio}
    Sea $\{h_n\}$ la sucesión de funciones de $\bb{R}^2$ en $\bb{R}$ definida como:
    \begin{equation*}
        h_n(x,y) = \frac{xy}{n^2+x^2+y^2} \qquad \forall (x,y)\in \bb{R}^2,~\forall n\in \bb{N}
    \end{equation*}
    Probar que la sucesión $\{h_n\}$ converge uniformemente en cada subconjunto acotado de $\bb{R}^2$,
    pero no converge uniformemente en $\bb{R}^2$.\\

    Estudiemos en primer lugar la convergencia puntual. Fijado $(x,y)\in \bb{R}^2$, tenemos de forma directa que:
    \begin{equation*}
        \lim_{n\to \infty} h_n(x,y) = \lim_{n\to \infty} \frac{xy}{n^2+x^2+y^2}=0
    \end{equation*}

    Por tanto, $\{h_n\}$ converge puntualmente a la función nula en $\bb{R}^2$.\\

    Estudiemos ahora la convergencia uniforme. Fijado un subconjunto acotado $A\subset~\bb{R}^2$, como este está acotado, está acotado para la norma del máximo. Por tanto, existe un $M\in \bb{R}^+$ tal que $\max\{|x|,|y|\}<M$. De esta forma, para todo $(x,y)\in A$, tenemos que:
    \begin{equation*}
        |h_n(x,y)| = \left|\frac{xy}{n^2+x^2+y^2}\right| \leq \frac{M^2}{n^2} \qquad \forall n\in \bb{N}
    \end{equation*}

    Por tanto, como $\left\{\frac{M^2}{n^2}\right\}\to 0$, tenemos que \ul{$\{h_n\}$ converge uniformemente a $0$ en $A$}.\\

    Estudiemos ahora la convergencia uniforme en $\bb{R}^2$. Tomemos $x_n=y_n=n$ para todo $n\in \bb{N}$. De esta forma, obtenemos una sucesión de puntos de $\bb{R}^2$ de forma que:
    \begin{equation*}
        h_n(x_n,y_n) = \frac{n^2}{n^2+2n^2} = \frac{1}{3} \qquad \forall n\in \bb{N}
    \end{equation*}
    Como $\{h(x_n,y_n)\}\to \frac{1}{3}\neq 0$, tenemos que \ul{$\{h_n\}$ no converge uniformemente en $\bb{R}^2$}.
\end{ejercicio}


\begin{ejercicio}
    Se considera la sucesión de funciones $\{f_n\}$ de $\bb{R}$ en $\bb{R}$ definida como:
    \begin{equation*}
        f_n(x) = \frac{x}{n} \qquad \forall x\in \bb{R},~\forall n\in \bb{N}
    \end{equation*}
    Probar que la sucesión $\{f_n\}$ converge uniformemente en un conjunto no vacío $C\subset \bb{R}$ si y solo si $C$ está acotado.\\

    Estudiemos en primer lugar la convergencia puntual. Fijado $x\in \bb{R}$, tenemos que:
    \begin{equation*}
        \lim_{n\to \infty} f_n(x) = \lim_{n\to \infty} \frac{x}{n} = 0
    \end{equation*}

    Por tanto, $\{f_n\}$ converge puntualmente a la función nula en $\bb{R}$.\\

    Estudiemos ahora la convergencia uniforme. Fijado un conjunto no vacío $C\subset \bb{R}$, distinguimos en función de si $C$ está acotado o no:
    \begin{itemize}
        \item Si $C$ está acotado (usamos norma del máximo), entonces existe un $M\in \bb{R}^+$ tal que $|x|<M$ para todo $x\in C$. De esta forma, para todo $x\in C$, tenemos que:
        \begin{equation*}
            |f_n(x)| = \left|\frac{x}{n}\right| \leq \frac{M}{n} \qquad \forall n\in \bb{N}
        \end{equation*}

        Por tanto, como $\left\{\frac{M}{n}\right\}\to 0$, tenemos que \ul{$\{f_n\}$ converge uniformemente a $0$ en $C$}.

        \item Si $C$ no está acotado, entonces para todo $n\in \bb{N}$, existe un $x_n\in C$ tal que $|x_n|>n$. Eligiendo esta sucesión de puntos, tenemos que:
        \begin{equation*}
            |f_n(x_n)| = \left|\frac{x_n}{n}\right| > 1 \qquad \forall n\in \bb{N}
        \end{equation*}

        Por tanto, tenemos que $\{f_n(x_n)\}$ no puede converger a $0$, por lo que se tiene que \ul{$\{f_n\}$ no converge uniformemente en $C$}.
    \end{itemize}
\end{ejercicio}


\begin{ejercicio}
    Sea $\{g_n\}$ la sucesión de funciones de $\bb{R}^+_0$ en $\bb{R}$ definida como:
    \begin{equation*}
        g_n(x) = \frac{2nx^2}{1+n^2x^4} \qquad \forall x\in \bb{R}^+_0,~\forall n\in \bb{N}
    \end{equation*}

    Dado $\delta\in \bb{R}^+$, probar que la sucesión $\{g_n\}$ converge uniformemente en $[\delta,+\infty[$, pero no
    converge uniformemente en $[0,\delta]$.\\


    Estudiemos en primer lugar la convergencia puntual. Para $x=0$, tenemos que $g_n(0)=0$ para todo $n\in \bb{N}$, por lo que $\{g_n\}$ converge puntualmente a la función nula en $\bb{R}^+_0$. Para $x>0$, tenemos que:
    \begin{equation*}
        \lim_{n\to \infty} g_n(x) = \lim_{n\to \infty} \frac{2nx^2}{1+n^2x^4} = 0
    \end{equation*}

    Por tanto, $\{g_n\}$ converge puntualmente a la función nula en $\bb{R}^+_0$.\\

    Estudiemos ahora la convergencia uniforme. Fijado $\delta\in \bb{R}^+$, definimos la sucesión $\{x_n\}$ de la siguiente forma:
    \begin{itemize}
        \item Si $n < \dfrac{1}{\delta^2}~\left(\delta < \dfrac{1}{\sqrt{n}}\right)$, entonces $x_n = \delta \in [0,\delta]$.
        \item Si $n \geq \dfrac{1}{\delta^2}~\left(\delta \geq \dfrac{1}{\sqrt{n}}\right)$, entonces $x_n = \dfrac{1}{\sqrt{n}}\in [0,\delta]$.
    \end{itemize}

    De esta forma, tenemos que $\{x_n\}$ es una sucesión de puntos de $[0,\delta]$. Veamos lo siguiente:
    \begin{equation*}
        g_n\left(\frac{1}{\sqrt{n}}\right) = \frac{2n\left(\frac{1}{\sqrt{n}}\right)^2}{1+n^2\left(\frac{1}{\sqrt{n}}\right)^4} =
        \frac{2}{1+1}=1 \qquad \forall n\in \bb{N}
    \end{equation*}

    Por tanto, como se tiene que $\{g_n(x_n)-g(x_n)\}=\{g_n(x_n)\}\to 1$, tenemos que \ul{$\{g_n\}$ no converge uniformemente en $[0,\delta]$}.\\

    Para el caso de la semirrecta $[\delta,+\infty[$, estudiamos en primer lugar la monotonía de la función $g_n$ para cada $n\in \bb{N}$. Para ello, como $g_n\in C^\infty(\bb{R}^+_0)$, estudiamos
    su derivada:
    \begin{equation*}
        g_n'(x) = \frac{4nx(1+n^2x^4)-8n^3x^5}{(1+n^2x^4)^2} = \frac{-4n^3x^5+4nx}{(1+n^2x^4)^2} \qquad \forall x\in \bb{R}^+_0
    \end{equation*}

    Por tanto, tenemos que los candidatos a extremos relativos de $g_n$ son:
    \begin{equation*}
        g_n'(x) = 0 \iff -4n^3x^5+4nx = 0 \iff 4xn(-n^2x^4+1) = 0 \iff x = 0 \text{ ó } x =\pm \frac{1}{\sqrt{n}}
    \end{equation*}

    Evaluando la primera derivada en cada intervalo, tenemos que:
    \begin{itemize}
        \item Si $x\in \left[0,\frac{1}{\sqrt{n}}\right]$, entonces $g_n'(x) > 0$, por lo que $g_n$ es creciente para todo $n\in \bb{N}$.
        \item Si $x\in \left[\frac{1}{\sqrt{n}},+\infty\right[$, entonces $g_n'(x) < 0$, por lo que $g_n$ es decreciente para todo $n\in \bb{N}$.
    \end{itemize}
    
    Estudiamos ahora la convergencia uniforme. Fijado $\delta \in \bb{R}^+$,
    tomamos $m\in \bb{N}$ tal que $\delta > \frac{1}{\sqrt{m}}$. De esta forma,
    para $n\in \bb{N}$, $n\geq m$, tenemos también que $\delta > \frac{1}{\sqrt{n}}$. Por tanto,
    tenemos que $[\delta ,+\infty[~\subset \left[\frac{1}{\sqrt{n}},+\infty\right[$,
    por lo que $g_n$ es decreciente en $[\delta,+\infty[$. De esta forma, para $n\geq m$, tenemos que:
    \begin{equation*}
        |g_n(x)| = g_n(x) \leq g_n(\delta) \qquad \forall x\in [\delta,+\infty[
    \end{equation*}

    Además, por la convergencia puntual, tenemos que $\{g_n(\delta)\}\to 0$, por lo que se deduce que \ul{$\{g_n\}$ converge uniformemente en $[\delta,+\infty[$}.
\end{ejercicio}


\begin{ejercicio}
    Para cada $n\in \bb{N}$, sea $h_n:\left[0,\nicefrac{\pi}{2}\right]\to \bb{R}$ la función definida como:
    \begin{equation*}
        h_n(x) = n\cos^n x \sen x \qquad \forall x\in \left[0,\nicefrac{\pi}{2}\right]
    \end{equation*}

    Fijado un $\rho\in \left]0,\nicefrac{\pi}{2}\right[$, probar que la sucesión $\{h_n\}$ converge uniformemente en $[\rho,\nicefrac{\pi}{2}]$, pero no converge uniformemente en $\left[0,\rho\right]$.\\


    Estudiamos en primer lugar la convergencia puntual. Cabe destacar que, debido al 
    dominio de la función, tanto el seno como el coseno son positivos.
    Considerando fijo $x\in \left]0,\nicefrac{\pi}{2}\right[$, tenemos que:
    \begin{equation*}
        \lim_{n\to \infty} \frac{n}{\cos^{-n} x}
        = \lim_{n\to \infty} \frac{n}{\left(\frac{1}{\cos x}\right)^n}
        \Hop
        \lim_{n\to \infty} \frac{1}{n\cdot \left(\frac{1}{\cos x}\right)^{n-1}} = 0
    \end{equation*}
    donde he usado que $|\cos x|<1$ para todo $x\in \left]0,\nicefrac{\pi}{2}\right[$. Por tanto, tenemos que:
    \begin{equation*}
        0 \leq n\cos^n x \sen x \leq n\cos^n x = \frac{n}{\cos^{-n} x}
    \end{equation*}
    Por el Lema del Sándwich, tenemos que $\{h_n\}$ converge puntualmente a la función nula en $\left]0,\nicefrac{\pi}{2}\right[$.
    
    Sumándole que, en $x=0,\nicefrac{\pi}{2}$ se tiene que $h_n(x)=0$, se tiene que $\{h_n\}$ converge puntualmente a la función nula en $\left[0,\nicefrac{\pi}{2}\right]$.\\

    Estudiamos ahora la convergencia uniforme. Fijado $\rho\in \left]0,\nicefrac{\pi}{2}\right[$, definimos la sucesión $\{x_n\}$ de la siguiente forma:
    \begin{itemize}
        \item Si $n < \nicefrac{1}{\rho}~\left(\rho < \nicefrac{1}{n}\right)$, entonces $x_n = \rho \in [0, \rho]$.
        \item Si $n \geq \nicefrac{1}{\rho}~\left(\rho \geq \nicefrac{1}{n}\right)$, entonces $x_n = \nicefrac{1}{n}\in [0, \rho]$.
    \end{itemize}

    De esta forma, tenemos que $\{x_n\}$ es una sucesión de puntos de $[0,\rho]$. Veamos lo siguiente:
    \begin{equation*}
        \lim_{n\to \infty}h_n\left(\frac{1}{n}\right) = \lim_{n\to \infty}n\cos^n\left(\frac{1}{n}\right)\sen\left(\frac{1}{n}\right) =
        \lim_{n\to \infty} \cos^n\left(\frac{1}{n}\right)\cdot \lim_{n\to \infty} \frac{\sen\left(\frac{1}{n}\right)}{\frac{1}{n}} \AstIg e^0\cdot 1 = 1
    \end{equation*}

    Pasar estudiar el primer límite en $(\ast)$, hemos tomado en primer lugar el logaritmo neperiano, por lo que luego hemos de usar la exponencial:
    \begin{equation*}
        \lim_{n\to \infty} n \log\left(\cos\left(\frac{1}{n}\right)\right) = \lim_{n\to \infty} \frac{\log\left(\cos\left(\frac{1}{n}\right)\right)}{\frac{1}{n}} \Hop
        \lim_{n\to \infty} \frac{-\sen\left(\frac{1}{n}\right)}{\cos\left(\frac{1}{n}\right)}
        = \lim_{n\to \infty} -\tan \left(\frac{1}{n}\right) = 0
    \end{equation*}

    Por tanto, como se tiene que $\{h_n(x_n)-h(x_n)\}=\{h_n(x_n)\}\to 1$, tenemos que \ul{$\{h_n\}$ no converge uniformemente en $[0,\rho]$}.\\

    Para el caso de $[\rho,\nicefrac{\pi}{2}]$, estudiamos en primer lugar la monotonía de la función $h_n$ para cada $n\in \bb{N}$. Para ello, como $h_n\in C^\infty\left(\left]0,\nicefrac{\pi}{2}\right[\right)$, estudiamos
    su derivada:
    \begin{equation*}
        h_n'(x) = n\left(-n\cos^{n-1}x\sen^2 x + \cos^{n+1} x\right) = n\cos^{n-1}x\left(-n\sen^2 x + \cos^2 x\right)
    \end{equation*}

    Por tanto, tenemos que los candidatos a extremos relativos de $h_n$ son:
    \begin{equation*}
        h_n'(x) = 0 \iff \cos^2 x = n\sen^2 x \iff \tan^2 x = \frac{1}{n}
        \iff x = \arctan\left(\frac{1}{\sqrt{n}}\right)
    \end{equation*}

    Evaluando la primera derivada en cada intervalo, tenemos que:
    \begin{itemize}
        \item Si $x\in \left[0,\arctan\left(\frac{1}{\sqrt{n}}\right)\right]$, entonces $h_n'(x) > 0$, por lo que $h_n$ es creciente para todo $n\in \bb{N}$.
        \item Si $x\in \left[\arctan\left(\frac{1}{\sqrt{n}}\right),\nicefrac{\pi}{2}\right]$, entonces $h_n'(x) < 0$, por lo que $h_n$ es decreciente para todo $n\in \bb{N}$.
    \end{itemize}

    Estudiamos ahora la convergencia uniforme. Fijado $\rho \in \left]0,\nicefrac{\pi}{2}\right[$,
    tomamos $m\in \bb{N}$ tal que $\rho > \arctan\left(\frac{1}{\sqrt{m}}\right)$, lo cual es posible ya que
    $\left\{\arctan\left(\frac{1}{\sqrt{n}}\right)\right\}\to 0$. De esta forma,
    para $n\in \bb{N}$, $n\geq m$, tenemos también que $\rho > \arctan\left(\frac{1}{\sqrt{n}}\right)$. Por tanto,
    tenemos que $[\rho ,\nicefrac{\pi}{2}]~\subset \left[\arctan\left(\frac{1}{\sqrt{n}}\right),\nicefrac{\pi}{2}\right]$,
    por lo que $h_n$ es decreciente en $[\rho,\nicefrac{\pi}{2}]$. Por tanto, para $n\geq m$, tenemos que:
    \begin{equation*}
        |h_n(x)| = h_n(x) \leq h_n(\rho) \qquad \forall x\in [\rho,\nicefrac{\pi}{2}]
    \end{equation*}

    Además, por la convergencia puntual, tenemos que $\{h_n(\rho)\}\to 0$, por lo que se deduce que \ul{$\{h_n\}$ converge uniformemente a $0$ en $[\rho,\nicefrac{\pi}{2}]$}.
\end{ejercicio}



\begin{ejercicio}
    Sea $\{\varphi_n\}$ la sucesión de funciones de $\bb{R}$ en $\bb{R}$ definida como:
    \begin{equation*}
        \varphi_n(x) = \frac{x^2}{1+n|x|} \qquad \forall x\in \bb{R},~\forall n\in \bb{N}
    \end{equation*}

    Probar que la sucesión $\{\varphi_n\}$ converge uniformemente en cada subconjunto acotado de $\bb{R}$, pero no converge uniformemente en $\bb{R}$.\\

    Estudiemos en primer lugar la convergencia puntual. Fijado $x\in \bb{R}$, tenemos que:
    \begin{equation*}
        \lim_{n\to \infty} \varphi_n(x) = \lim_{n\to \infty} \frac{x^2}{1+n|x|} = 0
    \end{equation*}

    Por tanto, $\{\varphi_n\}$ converge puntualmente a la función nula en $\bb{R}$.\\

    Estudiemos ahora la convergencia uniforme. Fijado un conjunto no vacío $C\subset \bb{R}$ acotado (en particular, acotado para la norma del máximo), existe un $M\in \bb{R}^+$ tal que $|x|<M$ para todo $x\in C$. De esta forma, para todo $x\in C\setminus \{0\}$, tenemos que:
    \begin{equation*}
        |\varphi_n(x)| = \left|\frac{x^2}{1+n|x|}\right| \leq \frac{x^2}{n|x|} = \frac{|x|}{n} \leq \frac{M}{n} \qquad \forall n\in \bb{N}
    \end{equation*}

    Además, en el caso de que se tenga que $0\in C$, se tiene que $|\varphi_n(0)|=0 \leq \frac{M}{n}$ para todo $n\in \bb{N}$.
    En cualquier caso, como se tiene que $\left\{\frac{M}{n}\right\}\to 0$, tenemos que \ul{$\{\varphi_n\}$ converge uniformemente a $0$ en $C$}.\\

    Estudiemos ahora la convergencia uniforme en $\bb{R}$. Tomamos $x_n=n$ para todo $n\in \bb{N}$. De esta forma, obtenemos una sucesión de puntos de $\bb{R}$ de forma que:
    \begin{equation*}
        \lim_{n\to \infty} \varphi_n(n) = \lim_{n\to \infty} \frac{n^2}{1+n^2} = 1
    \end{equation*}

    Como $\{\varphi_n(n)\}\to 1\neq 0$, tenemos que \ul{$\{\varphi_n\}$ no converge uniformemente en $\bb{R}$}.
\end{ejercicio}


\begin{ejercicio}
    Se considera la sucesión de funciones $\{\varphi_n\}$ de $\bb{R}_0^+$ en $\bb{R}$ definida como:
    \begin{equation*}
        \varphi_n(x) = \frac{x^n}{1+x^n} \qquad \forall x\in \bb{R}_0^+,~\forall n\in \bb{N}
    \end{equation*}

    Dados $r,\rho\in \bb{R}$, con $0<r<1<\rho$, estudiar la convergencia uniforme de $\{\varphi_n\}$ en los intervalos $[0,r],~[r,\rho]$ y $[\rho,+\infty[$.\\

    Estudiemos en primer lugar la convergencia puntual. Fijado $x\in \bb{R}_0^+$, tenemos que:
    \begin{equation*}
        \varphi_n(x) = \frac{x^n}{1+x^n} = \frac{1}{\frac{1}{x^n}+1}
    \end{equation*}

    Por tanto, distinguimos en función de los valores de $x$:
    \begin{itemize}
        \item Si $|x|<1$:
        \begin{equation*}
            \lim_{n\to \infty} \varphi_n(x) = \lim_{n\to \infty} \frac{1}{\frac{1}{x^n}+1} = \frac{1}{\frac{1}{0}+1} = 0
        \end{equation*}
        Tenemos que $\varphi_n$ converge puntualmente a la función nula en $[0,1[$.

        \item Si $|x|>1$:
        \begin{equation*}
            \lim_{n\to \infty} \varphi_n(x) = \lim_{n\to \infty} \frac{1}{\frac{1}{x^n}+1} = \frac{1}{\frac{1}{\infty}+1} = 1
        \end{equation*}
        Tenemos que $\varphi_n$ converge puntualmente a la función constante $1$ en $]1,+\infty[$.

        \item Si $x=1$:
        \begin{equation*}
            \lim_{n\to \infty} \varphi_n(1) = \lim_{n\to \infty} \frac{1}{1+1^n} = \frac{1}{2}
        \end{equation*}
        Tenemos que $\varphi_n$ converge puntualmente a la función constante $\nicefrac{1}{2}$ en $\{1\}$.
    \end{itemize}

    Por tanto, de forma directa deducimos que \ul{$\{\varphi_n\}$ no converge uniformemente en $[r,\rho]$}, ya que
    a pesar de ser continua para todo $n\in \bb{N}$ (es racional), su función límite no lo es, por lo que no se preserva la continuidad.

    Estudiemos ahora la convergencia uniforme en $[0,r]$. Tenemos que:
    \begin{equation*}
        \left|\varphi_n(x)\right| = \left|\frac{x^n}{1+x^n}\right| \leq \left|\frac{r^n}{1}\right| = r^n \qquad \forall x\in [0,r],~\forall n\in \bb{N}
    \end{equation*}
    donde he empleado que $0\leq x\leq r<1$, y por tanto $x^n < r^n$ para todo $n\in \bb{N}$.
    Entonces, como $\{r^n\}\to 0$, tenemos que \ul{$\{\varphi_n\}$ converge uniformemente a $0$ en $[0,r]$}.

    Estudiemos por último la convergencia uniforme en $[\rho,+\infty[$. Tenemos que:
    \begin{equation*}
        \left|\varphi_n(x)-1\right| = \left|\frac{x^n}{1+x^n}-1\right| = \left|\frac{-1}{1+x^n}\right| = \frac{1}{1+x^n}
        \leq \frac{1}{x^n} \leq \frac{1}{\rho^n} \qquad \forall x\in [\rho,+\infty[,~\forall n\in \bb{N}
    \end{equation*}
    donde he empleado que $x\geq \rho>1$, y por tanto $x^n \geq \rho^n$ para todo $n\in \bb{N}$. Además,
    como $\left\{\frac{1}{\rho^n}\right\}\to 0$, tenemos que \ul{$\{\varphi_n\}$ converge uniformemente a $1$ en $[\rho,+\infty[$}.
\end{ejercicio}


\begin{ejercicio}
    Estudiar la convergencia puntual y uniforme de la sucesión de funciones $\{f_n\}$ de $[0,1]$ en $\bb{R}$ definida como:
    \begin{equation*}
        f_n(x) = x-x^n \qquad \forall x\in [0,1],~\forall n\in \bb{N}
    \end{equation*}

    Estudiemos en primer lugar la convergencia puntual. Fijado $x\in [0,1[$, tenemos:
    \begin{equation*}
        \lim_{n\to \infty} f_n(x) = \lim_{n\to \infty} x-x^n = x
    \end{equation*}

    Fijado $x=1$, tenemos que:
    \begin{equation*}
        \lim_{n\to \infty} f_n(1) = \lim_{n\to \infty} 1-1^n = 1-1=0
    \end{equation*}

    Por tanto, $\{f_n\}$ converge puntualmente a la función:
    \begin{equation*}
        f(x) = \begin{cases}
            x & \text{si } x\in [0,1[\\
            0 & \text{si } x=1
        \end{cases}
    \end{equation*}

    Para estudiar la convergencia uniforme, estudiamos la continuidad de $f$ en $[0,1]$:
    \begin{equation*}
        \lim_{x\to 1}f(x) = \lim_{x\to 1}x = 1 \neq 0 = f(1)
    \end{equation*}
    Por tanto, $f$ no es continua en $1$. No obstante, $f_n$ sí es continua en $1$ para todo $n\in \bb{N}$ (es un polinomio). Por tanto, se tiene que \ul{$\{f_n\}$ no converge uniformemente en $[0,1]$}.
\end{ejercicio}


\begin{ejercicio}
    Se considera la sucesión de funciones $\{f_n\}$ de $\bb{R}^+_0$ en $\bb{R}$ definida como:
    \begin{equation*}
        f_n(x) = \frac{x}{x+n} \qquad \forall x\in \bb{R}^+_0,~\forall n\in \bb{N}
    \end{equation*}
    Fijado un $\rho\in \bb{R}^+$, estudiar la convergencia uniforme de $\{f_n\}$ en $\bb{R}_0^+$ y en $[0,\rho]$.\\

    Estudiemos en primer lugar la convergencia puntual. Fijado $x\in \bb{R}^+$, tenemos que:
    \begin{equation*}
        \lim_{n\to \infty} f_n(x) = \lim_{n\to \infty} \frac{x}{x+n} = 0
    \end{equation*}
    Además, tenemos que $f_n(0)=0$ para todo $n\in \bb{N}$. Por tanto, $\{f_n\}$ converge puntualmente a la función nula en $\bb{R}^+_0$.\\
    
    Estudiemos ahora la convergencia uniforme en $\bb{R}^+_0$. Consideramos la sucesión $x_n=n$ para todo $n\in \bb{N}$. De esta forma, obtenemos una sucesión de puntos de $\bb{R}^+_0$ de forma que:
    \begin{equation*}
        f_n(n) = \frac{n}{n+n} = \frac{1}{2} \qquad \forall n\in \bb{N}
    \end{equation*}

    Como $\{f_n(n)\}\to \nicefrac{1}{2}\neq 0$, tenemos que \ul{$\{f_n\}$ no converge uniformemente en $\bb{R}^+_0$}.\\

    Estudiemos por último la convergencia uniforme en $[0,\rho]$. Tenemos que:
    \begin{equation*}
        \left|f_n(x)-0\right| = \left|\frac{x}{x+n}\right| \leq \frac{\rho}{n} \qquad \forall x\in [0,\rho],~\forall n\in \bb{N}
    \end{equation*}
    donde he empleado que $0\leq x\leq \rho$. Entonces, como $\left\{\frac{\rho}{n}\right\}\to 0$, tenemos que \ul{$\{f_n\}$ converge uniformemente a $0$ en $[0,\rho]$}.
\end{ejercicio}


\begin{ejercicio}
    Se considera la sucesión de funciones $\{f_n\}$ de $\bb{R}_0^+$ en $\bb{R}$ definida como:
    \begin{equation*}
        f_n(x) = \frac{\sen (nx)}{1+nx} \qquad \forall x\in \bb{R}_0^+,~\forall n\in \bb{N}
    \end{equation*}

    Fijado $\rho\in \bb{R}^+$, estudiar la convergencia uniforme de $\{f_n\}$ en $[\rho, \infty[$ y en $[0,\rho]$.\\

    Estudiemos en primer lugar la convergencia puntual. Fijado $x\in \bb{R}_0^+$, tenemos que:
    \begin{equation*}
        \lim_{n\to \infty} f_n(x) = \lim_{n\to \infty} \frac{\sen (nx)}{1+nx} = 0
    \end{equation*}

    Por tanto, $\{f_n\}$ converge puntualmente a la función nula en $\bb{R}_0^+$.\\

    Para estudiar la convergencia uniforme en $\bb{R}_0^+$, consideramos la sucesión dada por:
    \begin{itemize}
        \item Si $n < \nicefrac{1}{\rho}~\left(\rho < \nicefrac{1}{n}\right)$, entonces $x_n = \rho \in [0,\rho]$.
        \item Si $n \geq \nicefrac{1}{\rho}~\left(\rho \geq \nicefrac{1}{n}\right)$, entonces $x_n = \nicefrac{1}{n}\in [0,\rho]$.
    \end{itemize}
    Por tanto, tenemos que $\{x_n\}$ es una sucesión de puntos de $\bb{R}_0^+$. Veamos lo siguiente:
    \begin{equation*}
        f_n\left(\frac{1}{n}\right) = \frac{\sen\left(n\cdot \frac{1}{n}\right)}{1+n\cdot \frac{1}{n}} = \frac{\sen 1}{2} \qquad \forall n\in \bb{N}
    \end{equation*}

    Por tanto, como se tiene que $\{f_n(x_n)-f(x_n)\}=\{f_n(x_n)\}\to \frac{\sen 1}{2}\neq 0$, tenemos que \ul{$\{f_n\}$ no converge uniformemente en $\bb{R}_0^+$}.\\

    Estudiemos por último la convergencia uniforme en $[\rho,+\infty[$. Tenemos que:
    \begin{equation*}
        \left|f_n(x)-0\right| = \left|\frac{\sen (nx)}{1+nx}\right| \leq \frac{1}{1+nx} \leq \frac{1}{1+n\rho} \qquad \forall x\in [\rho,+\infty[,~\forall n\in \bb{N}
    \end{equation*}
    donde he empleado que $x\geq \rho$. Entonces, como sabemos que $\left\{\frac{1}{1+n\rho}\right\}\to 0$, tenemos que \ul{$\{f_n\}$ converge uniformemente a $0$ en $[\rho,+\infty[$}.
\end{ejercicio}
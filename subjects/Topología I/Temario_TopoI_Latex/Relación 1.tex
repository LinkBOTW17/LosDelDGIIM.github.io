\section{Espacios Topológicos}\label{sec:Rel1}

\begin{ejercicio}
    Sea $(X,d)$ un espacio métrico. Prueba que la siguiente aplicación $\wt{d}:X\times X\to \bb{R}$ dada por
    \begin{equation*}
        \wt{d}(x,y)=\frac{d(x,y)}{1+d(x,y)} \qquad \forall x,y\in X
    \end{equation*}
    es una distancia en $X$. Prueba que $\cc{T}_d=\cc{T}_{\wt{d}}$.\\

    Para probar que es una distancia, comprobamos las 3 condiciones sabiendo que $d$ es una distancia:
    \begin{enumerate}
        \item Como $d(x,y)\geq 0$, tenemos que el numerador es no-negativo. Además, como el denominador es la suma de no-negativos, tenemos que también es no-negativo. Por tanto, tenemos que $\wt{d}(x,y)\geq 0$. Además, tenemos que:
        \begin{equation*}
            \wt{d}(x,y) = 0 \Longleftrightarrow d(x,y)=0 \Longleftrightarrow x=y
        \end{equation*}

        \item Tenemos de forma directa que es simétrica esta distancia, ya que $d$ es también simétrica.

        \item Veamos si cumple la tercera desigualdad, es decir, :
        \begin{equation*}
            \wt{d}(x,z) = \frac{d(x,z)}{1+d(x,z)} \leq \frac{d(x,y)}{1+d(x,y)} + \frac{d(y,z)}{1+d(y,z)} = \wt{d}(x,y) + \wt{d}(y,z)
        \end{equation*}

        Definimos $f:\bb{R}^+_0\to \bb{R}$ dada por $f(x)=\dfrac{x}{1+x}$, y estudiamos su monotonía.
        $$f'(x)=\frac{1+x -x}{(1+x)^2} = \frac{1}{(1+x)^2}>0$$
        
        Por tanto, tenemos que $f$ es estrictamente creciente, por lo que si $t\leq s$, tenemos que $f(t)\leq f(s)$. Veamos además qué ocurre con la suma. Para ello, sean $a,b\in \bb{R}^+_0$,
        \begin{equation*}
            f(a+b)=\frac{a+b}{1+a+b} = \frac{a}{1+a+b} + \frac{b}{1+a+b} \leq \frac{a}{1+a} + \frac{b}{1+b} = f(a)+f(b)
        \end{equation*}
        
        
        Por tanto, como $d$ es una distancia, tenemos que $d(x,z)\leq d(x,y) + d(y,z)$. Entonces,
        \begin{equation*}
            f[d(x,z)]\leq f[d(x,y) + d(y,z)] \leq f[d(x,y)] + f[d(y,z)]
        \end{equation*}

        Por tanto, por la definición de $f$ y $\wt{d}$, tenemos:
       \begin{equation*}
           \wt{d}(x,z) = \frac{d(x,z)}{1+d(x,z)} \leq \frac{d(x,y)}{1+d(x,y)} + \frac{d(y,z)}{1+d(y,z)} = \wt{d}(x,y) + \wt{d}(y,z)
       \end{equation*} 
    \end{enumerate}

    Por tanto, como hemos probado las tres condiciones, tenemos que es una distancia. Comprobemos ahora que las topologías son iguales. Para ello, hacemos uso del Corolario \ref{coro:TopIg_Bases} y de que los abiertos básicos de un espacio métrico son sus bolas abiertas. Comprobamos ambas condiciones:
    \begin{comment}
    Para ello, simplemente comprobamos que las distancias son equivalentes, es decir, $\exists a,b\in \bb{R}^+$ tal que:
    \begin{equation*}
        a\wt{d}(x,y) \leq d(x,y)\leq b\wt{d}(x,y)
    \end{equation*}

    Suponemos $x\neq y$, ya que en ese caso se tiene para cualesquiera valores de $a,b$. Buscamos en primer lugar el valor de $a$:
    \begin{equation*}
        \frac{ad(x,y)}{1+d(x,y)} = a\wt{d}(x,y)\leq d(x,y)
        \Longleftrightarrow
        \frac{a}{1+d(x,y)} \leq 1
        \Longleftrightarrow
        a\leq 1+d(x,y)
    \end{equation*}

    Tomamos por tanto $a=1$. Busquemos el valor de $b$:
    \begin{equation*}
        d(x,y)\leq b\wt{d}(x,y) = \frac{bd(x,y)}{1+d(x,y)}
        \Longleftrightarrow
        1\leq \frac{b}{1+d(x,y)}
        \Longleftrightarrow
        1+d(x,y)\leq b
    \end{equation*}

    Por tanto, como no podemos determinar un valor de $b$, tenemos que no son equivalentes.
    \end{comment}

    \begin{enumerate}
        \item \ul{$\T_{\wt{d}}\leq \T_d$}

        Sea $\wt{x}\in X, \wt{r}\in \bb{R}^+$. Buscamos demostrar que $\forall B_{\wt{d}}(\wt{x}, \wt{r})$, $\forall x\in B_{\wt{d}}(\wt{x}, \wt{r})$, existe un $B_d(x,r)$ con $x\in B_d(x,r)\subset B_{\wt{d}}(\wt{x}, \wt{r})$.

        Como $B_{\wt{d}}(\wt{x}, \wt{r})$ es un abierto métrico, tenemos que $\forall x\in B_{\wt{d}}(\wt{x}, \wt{r}), \exists r\in \bb{R}^+$ tal que $B_{\wt{d}}(x,r)\subset B_{\wt{d}}(\wt{x}, \wt{r})$.

        Ahora, vemos que $\wt{d}(x,y)<d(x,y)$:
        \begin{multline*}
            \wt{d}(x,y) = \frac{d(x,y)}{1+d(x,y)}<d(x,y) 
            \Longleftrightarrow 
            d(x,y)<d(x,y) + d^2(x,y)
            \Longleftrightarrow\\\Longleftrightarrow  d^2(x,y)>0 \Longleftrightarrow d(x,y)>0
        \end{multline*}
        
        Con ese resultado, veamos ahora que $B_d(x,r)\subset B_{\wt{d}}(x,r)$. Sea $y\in B_d(x,r)$, es decir, $d(x,y)<r$. Como $\wt{d}(x,y)<d(x,y)<r$, tenemos que $y\in B_{\wt{d}}(x,r)$.\\

        Por tanto, hemos demostrado que $\forall B_{\wt{d}}(\wt{x}, \wt{r})$, $\forall x\in B_{\wt{d}}(\wt{x}, \wt{r})$, existe un $B_d(x,r)$ con $x\in B_d(x,r)\subset B_{\wt{d}}(x, r)\subset B_{\wt{d}}(\wt{x}, \wt{r})$.\\

        Cabe destacar el procedimiento seguido. En primer lugar, hemos demostrado que $\forall x\in B_{\wt{d}}(\wt{x}, \wt{r})$, podemos encontrar una bola con la misma distancia centrada en dicho punto. Esa bola es $B_{\wt{d}}(x, r)$. Por último, buscamos una bola centrada en dicho $x$ pero con la otra distancia, en nuestro caso $B_d(x,r)$. En este caso particular, encontrar esa segunda bola es más sencillo por la relación de las distancias.

        \item \ul{$\T_d\leq \T_{\wt{d}}$}

        Sea $x_d\in X, r\in \bb{R}^+$. Buscamos demostrar que $\forall B_{d}(x_d, r)$, $\forall x\in B_{d}(x_d, r)$, existe un $B_{\wt{d}}(x,\delta)$ con $x\in B_{\wt{d}}(x,\delta)\subset B_{d}(x_d, r)$.

        Como $B_{d}(x_d, r)$ es un abierto métrico, tenemos que $\forall x\in B_{d}(x_d, r), \exists \veps\in \bb{R}^+$ tal que $B_{d}(x,\veps)\subset B_{d}(x_d, r)$.

        Definimos de nuevo $f:\bb{R}^+_0\to \bb{R}$ dada por $f(x)=\frac{x}{1+x}$, que vimos que es estrictamente creciente. Veamos ahora que, tomando $\delta=\frac{\veps}{1+\veps}=f(\veps)$, se tiene $B_{\wt{d}}(x,\delta)\subset B_d(x,\veps)$.
        
        Sea $y\in B_{\wt{d}}(x,\delta)$, es decir, $\wt{d}(x,y)<\delta$. Por la definición de $f$, tenemos que $f(d(x,y))<f(\veps)$. Por tanto, por ser $f$ estrictamente creciente, tenemos que $d(x,y)<\veps$, por lo que $y\in B_d(x,\veps)$.

        Por tanto, hemos demostrado que $\forall B_{d}(x_d, r)$, $\forall x\in B_{d}(x_d, r)$, existe un $B_{\wt{d}}(x,\delta)$ con $x\in B_{\wt{d}}(x,\delta)\subset B_d(x,\veps)\subset B_{d}(x_d, r)$.\\

        Cabe destacar el procedimiento seguido. En primer lugar, hemos demostrado que $\forall x\in B_{d}(x_d, r)$, podemos encontrar una bola con la misma distancia centrada en dicho punto. Esa bola es $B_d(x,\veps)$. Por último, buscamos una bola centrada en dicho $x$ pero con la otra distancia, en nuestro caso $B_{\wt{d}}(x,\delta)$. En este caso particular, para encontrar esa segunda bola hemos definido un $\delta$ particular.
    \end{enumerate}
\end{ejercicio}



\begin{ejercicio}
    Sea $(X,d)$ un espacio métrico. Prueba que la siguiente aplicación $\wt{d}:X\times X\to \bb{R}$ dada por
    \begin{equation*}
        \wt{d}(x,y)=\min\{1,d(x,y)\} \qquad \forall x,y\in X
    \end{equation*}
    es una distancia en $X$. Prueba que $\cc{T}_d=\cc{T}_{\wt{d}}$.\\

    Para probar que es una distancia, comprobamos las 3 condiciones sabiendo que $d$ es una distancia:
    \begin{enumerate}
        \item Como $1>0$ y $d(x,y)\geq 0$, tenemos que $\inf\{1,d(x,y)\}\geq 0$. Además, tenemos que:
        \begin{equation*}
            \wt{d}(x,y)=\min\{1,d(x,y)\} = 0 \Longleftrightarrow d(x,y)=0 \Longleftrightarrow x=y
        \end{equation*}

        \item Tenemos de forma directa que es simétrica esta distancia, ya que $d$ es también simétrica y $\min\{a,b\}=\min\{b,a\}$.

        \item Veamos si cumple la tercera desigualdad:
        \begin{multline*}
            \wt{d}(x,z) = \min\{1,d(x,z)\} \leq \min\{1,d(x,y)+d(y,z)\} \leq \\\leq \min\{1,d(x,y)\}  + \min\{1,d(y,z)\} = \wt{d}(x,y) +  \wt{d}(y,z)
        \end{multline*}
    \end{enumerate}

    Por tanto, como hemos probado las tres condiciones, tenemos que es una distancia. Comprobemos ahora que las topologías son iguales. Para ello, hacemos uso del Corolario \ref{coro:TopIg_Bases} y de que los abiertos básicos de un espacio métrico son sus bolas abiertas. Comprobamos ambas condiciones:
    \begin{itemize}
        \item \ul{$\T_d\leq \T_{\wt{d}}$}

        Sea $x_d\in X, r_d\in \bb{R}^+$. Entonces, $\forall B_d(x_d,r_d)$, $\forall x\in B_d(x_d,r_d)$, buscamos demostrar que $\exists B_{\wt{d}}(x,r)$ tal que $x\in B_{\wt{d}}(x,r)\subset B_d(x,r)$.

        Como $B_d(x_d,r_d)$ es un abierto métrico, tenemos que $\forall x\in B_d(x_d,r_d)$, $\exists r\in \bb{R}^+$ tal que $B_d(x,r)\subset B_d(x_d,r_d)$. Tomamos $r<1$ sin pérdida de generalidad, ya que en el caso de que $r\geq 1$, tenemos que que $\exists r'<1$ tal que se tiene que $B_d(x,r')\subset B_d(x,r)$.

        Veamos ahora que $B_{\wt{d}}(x,r)\subset B_d(x,r)$. Sea $y\mid \wt{d}(x,y)<r$. Entonces, como $\wt{d}(x,y)=\inf\{1,d(x,y)\}<r<1$, tenemos que $d(x,y)=\wt{d}(x,y)<r<1$. Por tanto, $y\in B_d(x,r)$.

        Por tanto, tenemos que $\forall B_d(x_d,r_d)$, $\forall x\in B_d(x_d,r_d)$, existe $B_{\wt{d}}(x,r)$ tal que $x\in B_{\wt{d}}(x,r)\subset B_d(x,r) \subset  B_d(x_d,r_d)$.

        \item \ul{$\T_{\wt{d}}\leq \T_d$}

        Sea $\wt{x}\in X, \wt{r}\in \bb{R}^+$. Entonces, $\forall B_{\wt{d}}(\wt{x},\wt{r})$, $\forall x\in B_{\wt{d}}(\wt{x},\wt{r})$, buscamos demostrar que $\exists B_d(x,r)$ tal que $x\in B_d(x,r)\subset B_{\wt{d}}(\wt{x},\wt{r})$.

        Como $B_{\wt{d}}(\wt{x},\wt{r})$ es un abierto métrico, tenemos que $\forall x\in B_{\wt{d}}(\wt{x},\wt{r})$, $\exists r\in \bb{R}^+$ tal que $B_{\wt{d}}(x, r)\subset B_{\wt{d}}(\wt{x},\wt{r})$.

        Veamos ahora que $B_d(x,r)\subset B_{\wt{d}}(x,r)$. Sea $y\mid d(x,y)<r$. Entonces, como $\wt{d}(x,y)=\min\{1,d(x,y)\}\leq d(x,y)<r$, tenemos que $y\in B_{\wt{d}}(x,r)$.
        
        Por tanto, tenemos que $\forall B_{\wt{d}}(\wt{x},\wt{r})$, $\forall x\in B_{\wt{d}}(\wt{x},\wt{r})$, existe $B_d(x,r)$ cumpliendo que $x\in B_d(x,r)\subset B_{\wt{d}}(x,r)\subset B_{\wt{d}}(\wt{x},\wt{r})$.
        
    \end{itemize}
\end{ejercicio}

\begin{ejercicio}
    Sea $(X,d)$ un espacio métrico y $x_0\in X$. Prueba que la aplicación $d':X\times X\to \bb{R}$ dada por
    \begin{equation*}
        d'(x,y)=\left\{
            \begin{array}{ccc}
                0 & \text{si} & x=y \\
                d(x,x_0) + d(x_0,y) & \text{si} & x\neq y
            \end{array}
        \right.
    \end{equation*}
    es una distancia en $X$. Si $(X,d)$ es el espacio métrico euclídeo a $d'$ se le denomina \textbf{distancia de correos}.\\

    Para probar que es una distancia, comprobamos las 3 condiciones sabiendo que $d$ es una distancia:
    \begin{enumerate}
        \item De forma evidente, como $d$ es una distancia, tenemos que $d'(x,y)\geq 0,~\forall x,y\in X$. Además, por definición tenemos que $x=y\Longrightarrow d'(x,y)=0$. Veamos la otra implicación: 
        \begin{equation*}
            d'(x,y)=0=d(x,x_0)+d(x_0,y) \Longrightarrow x=x_0=y\Longrightarrow x=y
        \end{equation*}

        \item Tenemos de forma directa que es simétrica esta distancia, debido a que $d$ es también simétrica y a la conmutabilidad de la suma en $\bb{R}$.

        \item Veamos si cumple la tercera desigualdad. Realizamos la siguiente distinción de casos:
        \begin{itemize}
            \item Si $x=z$, entonces $d'(x,z)=0\leq d'(x,y)+d'(y,z)$.

            \item Si $x\neq z$, entonces:
            \begin{multline*}
                d'(x,z)=d(x,x_0) + d(x_0,z)
                \leq d(x,x_0) + d(x_0,y) + d(y,x_0) + d(x_0,z)=\\=
                d'(x,y) + d'(y,z)
            \end{multline*}
        \end{itemize}
    \end{enumerate}

    Por tanto, como hemos probado las tres condiciones, tenemos que es una distancia.
\end{ejercicio}


\begin{ejercicio}
    Consideramos la siguiente aplicación $d:\bb{R}^2\times \bb{R}^2\to \bb{R}$ dada por
    \begin{equation*}
        d[(x_1, x_2), (y_1, y_2)] = \left\{
        \begin{array}{lll}
            |y_2-x_2| & \text{si} & x_1=y_1, \\
            |x_2| + |y_1-x_1| + |y_2| & \text{si} & x_1\neq y_1.
        \end{array}
        \right.
    \end{equation*}
    Demuestra que $d$ es una distancia en $\bb{R}^2$ y calcula las bolas abiertas y cerradas de $(\bb{R}^2, d)$. A $d$ se le denomina la \textbf{distancia del río en la jungla} o \textbf{distancia del ascensor}.\\

    Para probar que es una distancia, comprobamos las 3 condiciones sabiendo que $d$ es una distancia:
    \begin{enumerate}
        \item De forma evidente, como al ser el valor absoluto siempre positivo o nulo, tenemos que $d(x,y)\geq 0,~\forall x,y\in X$. Veamos ahora que $d(x,y)=0 \Longleftrightarrow x=y$.
        \begin{itemize}
            \item Si $x=y$, entonces $d(x,y)=y_2-x_2=0$.
            \item Si $d(x,y)=0$, si $x_1=y_1$ tenemos que $y_2=x_2$; y si $x_1\neq y_1$, entonces ya tenemos un sumando positivo y por tanto  $d(x,y)\neq 0$, por lo que llegamos a una contradicción. Por tanto, $x=y$.
        \end{itemize}


        \item Tenemos de forma directa que es simétrica esta distancia, debido a que, en $\bb{R}$, se tiene que $|a-b|=|b-a|$ .

        \item Veamos si cumple la tercera desigualdad. Realizamos la siguiente distinción de casos:  
        \begin{itemize}
            \item Si $x_1=z_1$:

            Suponiendo $y_1=x_1=z_1$, tenemos que
            \begin{equation*}
                d(x,z)=|z_2-x_2| = |y_2-x_2+z_2-y_2|\leq |y_2-x_2| + |z_2-y_2| = d(x,y) + d(y,z)
            \end{equation*}

            Suponiendo $y_1\neq x_1=z_1$, tenemos que
            \begin{multline*}
                d(x,z)=|z_2-x_2| \leq |z_2| + |x_2|\leq |x_2| + |y_1-x_1| + |y_2| + |y_2| + |z_1-y_1| + |z_2| =\\= d(x,y) + d(y,z)
            \end{multline*}

            \item Si $x_1\neq z_1$, entonces:
            
            Suponiendo $y_1=x_1\neq z_1$, tenemos que
            \begin{equation*}
                d(x,z)=|x_2| + |z_1-x_1| + |z_2|
                \leq |y_2-x_2| + |y_2| + |z_1-y_1| + |z_2| = d(x,y) + d(y,z)
            \end{equation*}

            Suponiendo $y_1\neq x_1, y_1\neq z_1$, tenemos que
            \begin{equation*}
                \begin{split}
                    d(x,z)&
                    =|x_2| + |z_1-x_1| + |z_2|
                    \leq |x_2| + |z_1-y_1+y_1-x_1| + |z_2|
                    \leq \\&
                    \leq |x_2| + |z_1-y_1| + |y_1-x_1| + |z_2|
                    \leq \\&
                    \leq |x_2| + |y_1-x_1| + |y_2| + |y_2| + |z_1-y_1| + |z_2|
                    =\\&= d(x,y) + d(y,z)
                \end{split}
            \end{equation*}

            Por tanto, se tiene que $d(x,z)\leq d(x,y) + d(y,z)$.
        \end{itemize}
    \end{enumerate}

    Por tanto, como hemos probado las tres condiciones, tenemos que es una distancia. 
    Veamos ahora las bolas abiertas. Dado, $x=(x_1, x_2)\in \bb{R}^2$, tenemos que:
    \begin{multline*}
        B(x,r) = \{y\in X\mid d(x,y)<r\} = \{(x_1, y_2)\in \bb{R}^2 \mid  |y_2-x_2|<r\} ~\bigcup~\\ \bigcup ~ \{(y_1, y_2)\in \bb{R}^2,~x_1\neq y_1 \mid  |x_2| + |y_1-x_1| + |y_2|<r\}
    \end{multline*}
\end{ejercicio}


\begin{ejercicio}
    Consideremos la distancia discreta $d_{disc}$ en $\bb{R}^n$. Prueba que no existe ninguna norma $\|\cdot \|:\bb{R}^n\to \bb{R}^+_0$ en $\bb{R}^n$ tal que $d_{\|\cdot \|} = d_{disc}$.\\

    Para toda distancia $d_{\|\cdot \|}$ inducida por la norma $\|\cdot \|$ se cumple $\forall a\in \bb{R},~x,y\in \bb{R}^n$ que:
    \begin{equation*}
        d_{\|\cdot \|}(ax+y, y) = ||ax+y-y||=||ax||=|a|~||x|| = |a|~d_{\|\cdot \|}(x,0)
    \end{equation*}

    Por tanto, tenemos que $d_{\|\cdot \|}(ax+y, y)=|a|d_{\|\cdot \|}(x,0),~~\forall a\in \bb{R},~x,y\in \bb{R}^n$. Tomando $x\neq 0$, $a\neq \pm 1, 0$ suponemos $d_{\|\cdot \|} = d_{disc}$ y llegamos al siguiente absurdo:
    \begin{equation*}
        d_{disc}(ax+y, y)=|a|d_{disc}(x,0) \Longrightarrow d_{disc}(ax+y, y) = |a| \neq 0,1
    \end{equation*}
    que, por cómo está definida la distancia discreta, es un absurdo.
\end{ejercicio}


\begin{ejercicio}
    Consideremos la norma $\|\cdot \|_1$ en $\bb{R}^n$. Prueba que no existe ninguna forma bilineal simétrica definida positiva $g:\bb{R}^n\times \bb{R}^n\to \bb{R}$ tal que $\|\cdot \|_g=\|\cdot \|_1$.\\

    Tenemos que en todo espacio vectorial euclídeo $X$ se cumple la identidad del paralelogramo\footnote{Demostrado en Análisis Matemático I.}:
    \begin{equation*}
        2||x||^2 + 2||y||^2 = ||x+y||^2 + ||x-y||^2, \hspace{1cm} \forall x,y\in X
    \end{equation*}
    
    Busquemos contraejemplos que demuestren que eso no es cierto para $X=\bb{R}^n$ con la norma 1. Sean los valores siguientes:
    \begin{align*}
        x=(1,\dots, 1) && x+y = (0,2,\dots, 2) \\
        y=(-1,1,\dots, 1) && x-y = (2, 0, \dots, 0) 
    \end{align*}
    
    Veamos que no se cumple la identidad del paralelogramo en $\bb{R}^n$ para la norma 1:
    \begin{gather*}
        2||x||_1^2 + 2||y||_1^2 = 2n^2 + 2n^2 = 4n^2 \neq 
        [2(n-1)]^2 + 2^2 = ||x+y||_1^2 + ||x-y||_1^2
    \end{gather*}

    Por tanto, en $\bb{R}^n$ con la norma 1 no se cumple la identidad del paralelogramo. Por tanto, no existe un producto escalar asociado a dicha norma.
\end{ejercicio}

\begin{ejercicio}
    Encuentra todas las topologías de un conjunto con dos elementos.\\

    Sea $X=\{a,b\}$. Entonces tenemos que las posibles topologías son:
    \begin{equation*}
        \T_t = \{\emptyset, X\}
        \hspace{1cm} \T_{disc} = \{\emptyset, \{a\}, \{b\}, X\}
        \hspace{1cm} \T_a = \{\emptyset, \{a\}, X\}
        \hspace{1cm} \T_b = \{\emptyset, \{b\}, X\}
    \end{equation*}
\end{ejercicio}

\begin{ejercicio}
    Estudia si $(X,\cc{T})$ es un espacio topológico en los siguientes casos:
    \begin{enumerate}[label=\alph*)]
        \item $X=\bb{N}$ y $\cc{T}=\{\emptyset, X\}\cup \left\{\{1,\dots, n\}\mid n\in \bb{N}\right\}$.

        Comprobamos las tres condiciones para que sea un espacio topológico:
        \begin{enumerate}
            \item Trivialmente, $\emptyset, X\in \T$.
            \item Veamos si es cerrado para uniones arbitrarias. Sea la familia $\{U_i\}_{i\in I}$. Si $U_i=\emptyset$, no afecta a la unión, por lo que no lo tenemos en cuenta. Si $U_i=X$, tenemos que la unión será $X$ y, por tanto, también se tiene. Comprobamos por tanto el caso en que $U_i\neq \emptyset, X$ para todo $i$. Sean por tanto $U_i=\{1,\dots, n_i\mid n_i\in \bb{N}\}$.
            \begin{itemize}
                \item Si $\{n_i\}_{i\in I}$ está acotado, tenemos que $\exists \max\limits_{i\in I}\{n_i\}$. Definimos por tanto $n=\max\limits_{i\in I}\{n_i\}\in \bb{N}$. Entonces, tenemos que
                $$\bigcup_{i\in I}U_i = \{1,\dots, n\}\in \T$$

                \item Si $\{n_i\}_{i\in I}$ no está acotado, entonces $$\bigcup_{i\in I}U_i = \bb{N}=X\in \T$$

                Veámoslo. Sea $n\in \bb{N}$, por lo que como $\{n_i\}_{i\in I}$ no está acotado, tenemos que $n\in U_n$. Por tanto, $n\in \bigcup\limits_{i\in I}U_i$. La otra inclusión es trivial, ya que $U_i\subset \bb{N}~\forall i\in I$.
            \end{itemize}

            \item Veamos si es cerrado para intersecciones de dos abiertos. Si $U_1=\emptyset$, entonces la intersección es $\emptyset\in \T$. Si $U_1=X$, tenemos que la intersección será $U_2\in \T$ y, por tanto, también se tiene. Por tanto, consideramos $U_1=\{1,\dots,n_1\}$ y $U_2=\{1,\dots, n_2\}$. Sea $n=\min\{n_1, n_2\}\in \bb{N}$, y tenemos que
            $$U_1\cap U_2 = \{1,\dots, n\}\in \T$$
        \end{enumerate}

        \item $X=\bb{R}$ y $\cc{T}=\{\emptyset, X\}\cup \left\{]-\infty, b[~\mid b\in \bb{R}\right\}$.

        Comprobamos las tres condiciones para que sea un espacio topológico:
        \begin{enumerate}
            \item Trivialmente, $\emptyset, X\in \T$.
            \item Veamos si es cerrado para uniones arbitrarias. Sea la familia $\{U_i\}_{i\in I}$. Si $U_i=\emptyset$, no afecta a la unión, por lo que no lo tenemos en cuenta. Si $U_i=X$, tenemos que la unión será $X$ y, por tanto, también se tiene. Comprobamos por tanto el caso en que $U_i\neq \emptyset, X$ para todo $i$. Sean por tanto $U_i=]-\infty, b_i[$. Definimos por tanto $b=\sup\limits_{i\in I}\{b_i\}\in \bb{R}$, que existe por el axioma del supremo ya que $I\neq \emptyset$. Entonces, tenemos que
            $$\bigcup_{i\in I}U_i = ]-\infty,b[\in \T$$

            \item Veamos si es cerrado para intersecciones de dos abiertos. Si $U_1=\emptyset$, entonces la intersección es $\emptyset\in \T$. Si $U_1=X$, tenemos que la intersección será $U_2\in \T$ y, por tanto, también se tiene. Por tanto, consideramos $U_1=]-\infty, b_1[$ y $U_2=]-\infty, b_2[$. Sea $b=\min\{b_1, b_2\}\in \bb{R}$, y tenemos que
            $$U_1\cap U_2 = ]-\infty, b[\in \T$$
        \end{enumerate}

        \item $X=\bb{R}$ y $\cc{T}=\{\emptyset, X\}\cup \left\{]-\infty, b]~\mid  b\in \bb{R}\right\}$.

        No es una topología, ya que no es cerrado para uniones arbitrarias. Tenemos que
        $$\bigcup_{n\in \bb{N}}\left]-\infty, -\frac{1}{n}\right]=~]-\infty, 0[~\notin \T$$

        \item $X$ un conjunto, $\emptyset\neq A,B\subset X$ y $\cc{T}=\{\emptyset, A, B, X\}$.

        En este caso, en general no se tiene. Como contraejemplo, sea $X=\{a,b,c\}$, y consideramos $A=\{a\}$, $B=\{b\}$. Entonces, se tiene que $A\cup B\notin \T$, por lo que no es cerrado para uniones.

        Si $A\subset B$ o $X\setminus A=B$ se tiene.

        \item $X$ un conjunto, $\emptyset\neq A \subsetneq X$ y $\cc{T}=\cc{P}(A)\cup \{X\}$.

        Comprobamos las tres condiciones para que sea un espacio topológico:
        \begin{enumerate}
            \item Trivialmente, $X\in \T$. Además, como $\emptyset\in \cc{P}(A)$, tenemos que $\emptyset\in \T$.
            \item Veamos si es cerrado para uniones arbitrarias. Tenemos que $\cc{P}(A)$ es cerrado para uniones arbitrarias, por lo que comprobamos para el caso de una unión con $X$. En este caso, el resultado de la unión es $X\in \T$, por lo que se tiene.

            \item Veamos si es cerrado para intersecciones de dos abiertos. Como $\cc{P}(A)$ es cerrado para intersecciones, tenemos que se tiene. Veamos qué ocurre en el caso de que uno de los abiertos sea $X$. Es decir, $U_1=X$, $U_2\in \cc{P}(A)$. Se tiene que $U_1\cap U_2=U_2\in \cc{P}(A)\subset \T$.
        \end{enumerate}

        \item $X=\{f:[0,1]\to \bb{R}\}$ y $\cc{T}=\{\emptyset\}\cup \{A\subset X\mid \exists f\in \cc{C}^0([0,1],\bb{R})\cap A\}$.

        Es decir, tenemos que los abiertos son, además del vacío, los conjuntos que contienen al menos una función continua. Veamos que no se trata de un espacio topológico con el siguiente contraejemplo.

        Sean $f,g,h\in X$, con
        $$f(x)=x^2 \qquad g(x)=e^x \qquad h(x)=\left\{\begin{array}{ccc}
            1 & \text{si} & x<\frac{1}{2} \\
            0 & \text{si} & \geq x\frac{1}{2}
        \end{array}\right.$$

        Sean $U_1=\{f,h\}\in \T$, $U_2=\{g,h\}\in \T$. Tenemos que no es cerrado para intersecciones de abiertos, ya que
        $$U_1\cap U_2 = \{h\}\notin \T$$
    \end{enumerate}
\end{ejercicio}



\begin{ejercicio}
    Sea $X$ un conjunto infinito y $x_0\in X$. Prueba que:
    \begin{equation*}
        \cc{T}=\{U\subset X\mid x_0\notin U\}\cup \{U\subset X\mid X\setminus U \text{ es finito}\}
    \end{equation*}
    es una topología sobre $X$, a la que llamaremos \textbf{topología fuerte en un punto}.\\
        
    Comprobamos las tres condiciones para que sea un espacio topológico:
    \begin{enumerate}
        \item $\emptyset \in \T$, ya que $x_0\notin \emptyset$. Además, $X\in \T$, ya que $X\setminus X = \emptyset$ es finito.
        
        \item Veamos si es cerrado para uniones arbitrarias.

        Sea la familia $\{U_i\}_{i\in I}$.
        \begin{itemize}
            \item Si $x_0\notin U_i~\forall i\in I$, tenemos que $\bigcup\limits_{i\in I}U_i\in \T$.
            \item Si $X\setminus U$ es finito $\forall i\in I$, tenemos que $X\setminus \bigcup\limits_{i\in I}U_i = \bigcap\limits_{i\in I}X\setminus U_i$ finita por ser la intersección de conjuntos finitos. Por tanto, $\bigcup\limits_{i\in I}U_i\in \T$.
        \end{itemize}
        Por tanto, tenemos que ambos subconjuntos son cerrados para uniones arbitrarias. Veamos qué ocurre al unir un conjunto de cada tipo. Sea $U_1\subset X$ con $x_0\notin U_1$, y $U_2\subset X$ tal que $X\setminus U$ finito.
        \begin{equation*}
            U_2\subset U_1\cup U_2 \Longrightarrow X\setminus (U_1\cup U_2) \subset X\setminus  U_2 \text{ finito}
        \end{equation*}

        Por tanto, tenemos que $U_1\cup U_2\in \T$ y, por tanto, es cerrado para uniones arbitrarias.

        \item Veamos si es cerrado para intersecciones de dos abiertos.

        Sean $U_1, U_2\subset X$.
        \begin{itemize}
            \item Si $x_0\notin U_i~\forall i=1,2$, tenemos trivialmente que $x_0\notin U_1\cap U_2$, por lo que $U_1\cap U_2\in \T$.
            
            \item Si $X\setminus U$ es finito $\forall i=1,2$, tenemos $X\setminus (U_1\cap U_2)=(X\setminus U_1)\cup (X\setminus U_2)$ finito, ya que la unión finita de conjuntos finitos es finita. Por tanto, tenemos que $U_1\cap U_2\in \T$.
        \end{itemize}
        
        Por tanto, tenemos que ambos subconjuntos son cerrados para intersecciones finitas. Veamos qué ocurre al intersecar un conjunto de cada tipo. Sea $U_1\subset X$ con $x_0\notin U_1$, y $U_2\subset X$ tal que $X\setminus U$ finito.
        \begin{equation*}
            x_0\notin U_1\Longrightarrow x_0\notin U_1\cap U_2 \Longrightarrow U_1\cap U_2\in \T
        \end{equation*}

        Por tanto, tenemos que $U_1\cap U_2\in \T$ y, por tanto, es cerrado para intersecciones finitas.
    \end{enumerate}
\end{ejercicio}

\begin{ejercicio}\label{ej:3.1.10}
    Dado $n\in \bb{N}$, denotaremos $U_n$ al conjunto de los divisores de $n$. En $\bb{N}$, se considera la siguiente familia de subconjuntos $\cc{T}\subset \cc{P}(\bb{N})$ dada por $U\in \cc{T}$ si y solo si $U_n\subset U$, para todo $n\in U$. Prueba que:
    \begin{enumerate}[label=\alph*)]
        \item $\cc{T}$ es una topología en $\bb{N}$.

        Tenemos que $U\subset \bb{N}$ es un abierto si es cerrado para los divisores de sus elementos. Comprobamos las tres condiciones para que sea un espacio topológico:
        \begin{enumerate}[label=\alph*)]
            \item[A1)] $\emptyset,\bb{N} \in \T$ trivialmente.
            
            \item[A2)] Veamos si es cerrado para uniones arbitrarias.

            Sea la familia $\{U^i\}_{i\in I}$, con $U^i\in \T$, y consideramos $A=\bigcup\limits_{i\in I}U^i$. Veamos si $A\in \T$. Dado $n\in A$, tenemos que comprobar que $U_n\subset A$. Como $n\in A$, tenemos que $\exists i\in I$ tal que $n\in U^i$. Como $U^i\in \T$, tenemos que $U_n\subset U^i\subset A$, por lo que se tiene directamente.
    
            \item[A3)] Veamos si es cerrado para intersecciones de dos abiertos. Sean $A,B\in \T$, y consideramos $A\cap B$. Dado $n\in A\cap B$, tenemos que $n\in A,B$, por lo que $U_n\subset A,B$ y por tanto $U_n\subset A\cap B$, teniendo entonces que $A\cap B\in \T$.
        \end{enumerate}
    
        \item $\cc{B}=\{U_n\mid n\in \bb{N}\}$ es una base de $\cc{T}$.

        Usando la Proposición \ref{prop:Caract_Base}, tenemos que $\cc{B}$ es una base si y solo si para todo $U\in \T$, se tiene que dado $n\in U$, $\exists B_n\in \cc{B}$ con $n\in B_n\subset U$.

        Como $U\subset \bb{N}$, tenemos precisamente que $B_n=U_n\in \cc{B}$, ya que $n\in U_n$ por ser $n$ un divisor (trivial) de $n$ y, por ser $U\in \T$, tenemos que $U_n\subset U$. Por tanto, se tiene de forma directa.

        
    \end{enumerate}
\end{ejercicio}

\begin{ejercicio}
    En $H^+=\{(x,y)\in \bb{R}^2\mid y\geq 0\}$ se considera la familia
    \begin{equation*}
        \cc{B}=\{B[(x,y), \veps]\mid y>0, \veps \in~ ]0,y[\}
        \cup
        \{(x,0)\cup B[(x,y),y]\mid y>0\}
    \end{equation*}
    Prueba que existe una única topología $\cc{T}\in H^+$ tal que $\cc{B}$ es una base para $\cc{T}$. A este espacio topológico se le conoce como semiplano de Moore.\\


    Aplicamos el Teorema \ref{teo:TopoGenerada_Bases}, por lo que comprobamos ambas propiedades.
    \begin{enumerate}
        \item[B1)] Hemos de comprobar que $H^+=\bigcup\limits_{B\in \cc{B}}B$.

        Demostramos por doble inclusión:
        \begin{description}
            \item[$\subset)$] Sea $(x,y)\in H^+$. Veamos que $(x,y)$ está en uno de los abiertos básicos del segundo tipo; es decir, $(x,y)\in \{(x,0)\cup B[(x,y),y]\mid y>0\}$.

            Si $y=0$, se tiene directamente, ya que $(x,y)=(x,0)$. Si $y\neq 0$, tenemos que $y>0$ y, por tanto, $(x,y)\in B[(x,y),y]$.

            \item [$\supset)$] Sea $(a,b)\in \bigcup\limits_{B\in \cc{B}}B$. Entonces, $\exists B\in \cc{B}$ tal que $(a,b)\in B$.
            \begin{itemize}
                \item Si $B$ es del primer tipo, tenemos que $(a,b)\in B[(x,y),\veps]$, con $y>0$ y $\veps\in ~]0,y[$. Entonces,
                $$d[(x,y),(a,b)]<\veps \Longrightarrow \sqrt{(x-a)^2 + (y-b)^2}<\veps \Longrightarrow y-b<\veps <y \Longrightarrow 0<b$$

                \item Si $B$ es del segundo tipo, tenemos que $b=0$ (por lo que $(a,b)\in H^+$) o $(a,b)\in B[(x,y),y]$ con $y>0$, por lo que:
                \begin{equation*}
                    d[(x,y),(a,b)]<y \Longrightarrow \sqrt{(x-a)^2 + (y-b)^2}<y \Longrightarrow y-b<y \Longrightarrow 0<b
                \end{equation*}
            \end{itemize}
            En ambos casos, tenemos $(a,b)\in H^+$.
        \end{description}
        
        \item[B2)] Hemos de demostrar que si $B_1,B_2\in \cc{B}$ con $(a,b)\in B_1\cap B_2$, entonces $\exists B_3\in \cc{B}$ con $(a,b)\in B_3\subset B_1\cap B_2$. Realizamos la siguiente distinción de casos:
        \begin{enumerate}
            \item \ul{Si $b>0$}: Supongamos que $B_1,B_2$ son del primer tipo. Es decir, supongamos $B_1=B[(x,y),\veps]$ y $B_2=B[(x',y'),\veps']$ con $y,y'>0$ y $\veps\in ]0,y[,$ $\veps'\in ]0,y'[$. Entonces, como las bolas abiertas son base de $(\bb{R}^2,\T_u)$, tenemos que $\exists \delta >0$ tal que $B[(a,b),\delta]\subset B_1\cap B_2$. Además, de forma trivial se tiene que $\delta <b$, por lo que $\delta\in ]0,b[$.
            
            
            De forma análoga se demuestra para los otros casos, solo que $\delta \in ]0,b]$. Por tanto, se tiene que $B_3\in \cc{B}$.

            \item \ul{Si $b=0$}: Tenemos que $(a,b)=(a,0)$ y que $B_1,B_2$ son del segundo tipo y tienen la misma coordenada $x=a$. Es decir,
            \begin{gather*}
                B_1=\{(a,0)\cup B[(a,y_1),y_1]\mid y_1>0\} \\
                B_2=\{(a,0)\cup B[(a,y_2),y_2]\mid y_2>0\}
            \end{gather*}

            Supuesto $y_1\leq y_2$, tenemos que $B_1\subset B_2$, ya que dado $(u,v)\in B_1$, $(u,v)\neq (a,0)$ se tiene que $d[(a,y_1),(u,v)]<y_1$ y, por la desigualdad triangular:
            \begin{multline*}
                d[(a,y_2),(u,v)]<d[(a,y_2),(a,y_1)] + d[(a,y_1),(u,v)] = y_2-y_1+d[(a,y_1),(u,v)] <\\< y_2-y_1+y_1 = y_2 \Longrightarrow (u,v)\in B_2
            \end{multline*}
            
            Por tanto, como un abierto es subconjunto del otro abierto, tenemos que $B_1\cap B_2\in \{B_1,B_2\}$.Tomamos por tanto $B_3=B_1\cap B_2\in \{B_1,B_2\}\subset \cc{B}$, y se tiene de forma directa.
        \end{enumerate}

        Por tanto, como se tienen B1) y B2), se tiene que existe una única topología con esa base.
    \end{enumerate}
\end{ejercicio}


\begin{ejercicio}
    Sean $\cc{T}$ y $\cc{T}'$ dos topologías de un conjunto $X$. Demuéstrese que la familia $\cc{T}\cap \cc{T}'$, formada por los abiertos comunes a ambas, es también una topología de $X$. ¿Es la unión de dos topologías una topología?\\

    Para ver si $\T\cap \T'$ es una topología, aplicamos la definición:
    \begin{enumerate}
        \item[A1)] Trivialmente $\emptyset, X\in \T\cap \T'$.

        \item[A2)] Sea $\{U_i\}_{i\in I} \subset \T\cap \T'$. Entonces, $\{U_i\}_{i\in I} \subset \T$ y $\{U_i\}_{i\in I} \subset \T'$. Por ser $\T,\T'$ dos topologías, tenemos que $\bigcup\limits_{i\in I}U_i\in \T,\T'$, por lo que $\bigcup\limits_{i\in I}U_i\in \T\cap \T'$, por lo que es cerrado para uniones arbitrarias.

        \item[A3)] Sea $U_1,U_2\in \T\cap \T'$. Entonces, como $\T,\T'$ son dos topologías, tenemos que $U_1\cap U_2\in \T,\T'$, por lo que es cerrado para intersecciones finitas.
    \end{enumerate}
    Por tanto, tenemos que $\T\cap \T'$ es una topología.\\

    Veamos ahora que la unión, por norma general, no lo es. Sea $X=\{a,b,c\}$. Entonces, consideramos $\T=\{\emptyset,X,\{a\}\}, \T'=\{\emptyset,X,\{b\}\}$ son dos topologías, pero la unión $\T\cup \T'=\{\emptyset,X,\{a\},\{b\}\}$ no es una topología, ya que no es cerrada para uniones, ya que $\{a\}\cup \{b\}=\{a,b\}\notin \T\cup \T'$.
\end{ejercicio}


\begin{ejercicio}
    En $(\bb{R}, \cc{T}_u)$ los intervalos que son abiertos son los de la forma $]a,b[$, para $a\leq b$, $]a,+\infty[$, $]-\infty, b[$ y $\bb{R}$; y los que son cerrados son de la forma $[a,b]$, para $a\leq b$, $[a,+\infty[$, $]-\infty, b]$ y $\emptyset$.\\

    Veamos en primer lugar la forma de los intervalos abiertos. Consideramos cada intervalo, y tenemos en cuenta que en $(\bb{R},\T_u)$ los abiertos son abiertos métricos. Es decir $U\in \T$ si y solo si $\forall x\in U,~\exists \veps\in \bb{R}^+$ tal que $x\in B(x,\veps)\subset U$.
    \begin{enumerate}
        \item $]a,b[~$: Tenemos que $]a,b[~=B\left(a+\dfrac{b-a}{2}, \dfrac{b-a}{2}\right)$.
        \item $]a,+\infty[~$: $\forall x\in ]a,+\infty[$, se tiene que $x\in B\left(x,x-a\right)=]a,2x-a[~\subset ]a,+\infty[$.
        \item $]-\infty,b[~$: $\forall x\in ]-\infty,b[$, se tiene que $x\in B\left(x,b-x\right)=]2x-b,b[~\subset ]-\infty,b[$.
        \item $\bb{R}$: Tenemos que $\forall x\in \bb{R}$, se tiene que $\exists \veps\in \bb{R}^+$ tal que $x\in B\left(x,\veps\right)\subset \bb{R}$.
    \end{enumerate}
    Para el resto de los intervalos, tenemos que al menos uno de los extremos es cerrado, por lo que para dicho extremo $c$, se tiene que $\nexists \veps\in \bb{R}^+$ tal que $]c-\veps, c+\veps[~\subset I$, llegando a que no son abiertos métricos.\\

    Una vez se tienen los abiertos, los cerrados son triviales, ya que:
    \begin{enumerate}
        \item $[a,b]=\bb{R}\setminus \left(]-\infty,a[~\cup~ ]b,+\infty[\right)\in C_\T$.
        \item $[a,+\infty[ = \bb{R}\setminus ]-\infty, a[\in C_\T$.
        \item $]-\infty,b] = \bb{R}\setminus ]b,+\infty[\in C_\T$.
        \item $\emptyset = \bb{R}\setminus \bb{R}$.
    \end{enumerate}
    Para el resto de intervalos, tengo que no son abiertos al ser sus complementarios del estilo a estos últimos.
\end{ejercicio}

\begin{ejercicio}
    Prueba que el conjunto $U=\{(x,y)\in \bb{R}^2\mid x>0\}$ es un abierto en $(\bb{R}^2, \cc{T})$ mientras que $C=\{(x,y)\in \bb{R}^2\mid x\geq 0\}$ es un cerrado que no es abierto.\\

    Como $\T$ es la topología métrica, tenemos que $U$ es un abierto métrico si y solo si $\forall (x,y)\in U$ se tiene que $\exists \veps\in \bb{R}^+$ tal que $B[(x,y),\veps]\subset U$. Sea $\veps=\frac{x}{2}$. Entonces, sea $(a,b)\in B\left[(x,y),\frac{x}{2}\right]$. Entonces:
    \begin{multline*}
        d[(x,y),(a,b)]<\frac{x}{2}\Longrightarrow \sqrt{(a-x)^2 + (b-y)^2} < \frac{x}{2} \Longrightarrow |a-x| < \frac{x}{2} \Longrightarrow \\ \Longrightarrow x-\frac{x}{2} < a < x+\frac{x}{2} \Longrightarrow \frac{x}{2} < a < \frac{3x}{2} \Longrightarrow 0<a \Longrightarrow (a,b)\in U
    \end{multline*}
    Por tanto, tenemos que $U$ es un abierto métrico. Veamos que $C$ es un cerrado. Tenemos que su complementario es $X\setminus C = \{(x,y)\in \bb{R}^2\mid x< 0\}$, y exactamente con la misma bola llegamos a la misma desigualdad para el valor de $a$. No obstante, como ahora $x<0$, tenemos que $a<0$ y por tanto $(a,b)\in X\setminus C$, por lo que $X\setminus C\in \T$ y por tanto $C$ es un cerrado.
\end{ejercicio}


\begin{ejercicio}
    Sea $X$ un conjunto no vacío y $\{A_i\}_{i\in I}$ una partición de $X$. Demuestra que existe una única topología $\cc{T}$ en $X$ tal que $\{A_i\}_{i\in I}$ es una base de $\cc{T}$. Prueba que todo abierto de $\cc{T}$ es un cerrado.\\

    Aplicamos el Teorema \ref{teo:TopoGenerada_Bases}, por lo que comprobamos ambas propiedades.
    \begin{enumerate}
        \item[B1)] Por definición de partición, se tiene de forma directa que $X=\bigcup\limits_{i\in I}A_i$.
        \item[B2)] Sean $A_i,A_j\in \{A_i\}_{i\in I}$. Por definición de partición, tenemos que $A_i\cap A_j=\emptyset$ $\forall i\neq j$. Entonces $x\in A_i\cap A_j$ implica que $A_i=A_j$, por lo que $x\in A_i\subset A_i\cap A_j$.
    \end{enumerate}
    Por tanto, aplicando el Teorema \ref{teo:TopoGenerada_Bases} se tiene que existe una única topología $\T$ con dicha familia como base. Veamos ahora que $\T=C_\T$.
    
    Sea $U\in \T$. Entonces, por definición de base topológica tenemos que $U=\bigcup\limits_{j\in J}A_j$, con $J\subset I$. Por tanto, $X\setminus U = \bigcup\limits_{i\in I\setminus J}A_i\in \T$, ya que la unión de abiertos es un abierto.

    Por tanto, tenemos que el complementario de un abierto es un abierto, de lo que se deduce de forma directa que $\T=C_\T$.
\end{ejercicio}

\begin{ejercicio}\label{ej:3.1.16}
    Sobre $\bb{R}$, consideramos la siguiente familia de subconjuntos:
    \begin{equation*}
        \cc{T}=\{U\cup V\mid U\in \cc{T}_u,~V\subset \bb{R}\setminus \bb{Q}\}
    \end{equation*}
    Se pide:
    \begin{enumerate}[label=\alph*)]
        \item Prueba que $\cc{T}$ es una topología sobre $\bb{R}$ que contiene a la topología usual $\cc{T}_u$. El espacio topológico $(\bb{R},\cc{T})$ recibe el nombre de \textbf{recta diseminada}.

        Comprobamos en primer lugar que es una topología:
        \begin{enumerate}
            \item[A1)] Tenemos que $\emptyset\subset \bb{R}\setminus \bb{Q}$ y $\emptyset\in \T_u$, por lo que se tiene que $\emptyset\in \T$. Además, como $\bb{R}\in \T_u$, tenemos que $\bb{R}\in \T$.
            
            \item[A2)] Sea una familia de abiertos $\{W_i\}_{i\in T}$, con $W_i\in \T$ para todo $i\in I$. Entonces, se tenemos que $W_i=U_i\cup V_i$, con $U_i\in \T_u$, $V_i\in \bb{R}\setminus\bb{Q}$. Entonces:
            \begin{equation*}
                \bigcup_{i\in I}W_i = \bigcup_{i\in I} U_i\cup V_i = \left(\bigcup_{i\in I}U_i\right) \bigcup \left(\bigcup_{i\in I}V_i\right)  \AstIg U\cup V\in \T
            \end{equation*}
            donde en $(\ast)$ he aplicado $U=\bigcup\limits_{i\in I}U_i\in \T_u$ por ser la unión de abiertos un abierto, y que $V=\bigcup\limits_{i\in I}V_i\subset \bb{R}\setminus \bb{Q}$ ya que $V_i\subset \bb{R}\setminus \bb{Q}~\forall i\in I$.
            
            \item[A3)] Sean $U_1,U_2\in \T_u$, $V_1,V_2\in \bb{R}\setminus \bb{Q}$. Entonces, $U_1\cup V_1,U_2\cup V_2\in \T$. Entonces:
            \begin{equation*}\begin{split}
                (U_1\cup V_1)\cap (U_2\cup V_2) &= [(U_1\cup V_1)\cap U_2] \cup [(U_1\cup V_1)\cap V_2]
                =\\&= [(U_1\cap U_2)\cup (V_1\cap U_2)] \cup [(U_1\cup V_1)\cap V_2]
                =\\&= [(U_1\cap U_2)]\cup (V_1\cap U_2) \cup [(U_1\cup V_1)\cap V_2]
                \in \T
            \end{split}\end{equation*}
            donde he indicado que es un abierto por ser $U_1\cap U_2\in \T_u$ por ser intersección de dos abiertos y ser $(V_1\cap U_2) \cup [(U_1\cup V_1)\cap V_2]\subset V_1\cup V_2\subset \bb{R}\setminus \bb{Q}$.
        \end{enumerate}
        Por tanto, tenemos que efectivamente se tiene que $\T$ es una topología. Además, considerando $V=\emptyset$ se tiene trivialmente que $\T_u\subset \T$.

        \item Prueba que los intervalos $[a,b]$ y $[c,d[$ con $d\in \bb{R}\setminus \bb{Q}$ son cerrados en $(\bb{R},\cc{T})$.\\

        Tenemos que $\bb{R}\setminus [a,b]=]-\infty, a[~\cup~]b,+\infty[\in \T_u\subset \T$, por lo que $[a,b]\in C_\T$. Veámoslo para $[c,d[~$:
        \begin{equation*}
            \bb{R}\setminus [c,d[ ~=~ ]-\infty,c[~\cup~[d,+\infty[~=~ 
            ]-\infty,c[~\cup~]d,+\infty[~\cup~\{d\}\in \T
        \end{equation*}
        donde he indicado que es un abierto, ya que $]-\infty,c[~\cup~]d,+\infty[~\in \T$ y $\{d\}\subset \bb{R}\setminus \bb{Q}$. Por tanto, $[c,d[~\in C_\T$.

        \item Calcula una base de entornos de $x\in \bb{R}$ en $(\bb{R},\cc{T})$.

        Recordamos que una base de entornos de $x$ es una familia de entornos de $x$ tal que $\forall N\in N_x$, $\exists V\in \beta_x\mid x\in V\subset N$. Recordamos también que un entorno de $x$ es un conjunto $N\subset \bb{R}$ tal que $\exists U\in \T$ tal que $x\in U\subset N$.\\

        En el caso de que $x\in \bb{R}\setminus \bb{Q}$, tenemos que $\{x\}\in \T$, por lo que $\{x\}\in N_x$. Por tanto, tenemos que $\beta_x=\{x\}$ es una base de entornos de $x$.

        En el caso de $x\in \bb{Q}$, tenemos que una base de entornos suya coincide con una base de entornos suya en $\T_u$, es decir, $\beta_x=~]x-\veps,x+\veps[,$ con $\veps\in \bb{R}^+$.

        Por tanto, tenemos que:
        \begin{equation*}
            \beta_x = \left\{\begin{array}{ccl}
                \{x\} & \text{si} & x\in \bb{R}\setminus \bb{Q}\\
                \{]x-\veps, x+\veps[~\mid \veps\in \bb{R}^+\}& \text{si} & x\in \bb{Q}
            \end{array}\right.
        \end{equation*}

        \item Calcula el interior, la clausura y la frontera de los intervalos $A=[0,1]$ y $B=[0,\sqrt{2}[$ en $(\bb{R},\cc{T})$.\\

        Demostramos en primer lugar que $A^\circ=]0,1[$~:
        \begin{description}
            \item[$\subset)$] Tenemos que $A^\circ \subset A= [0,1]$.
        
            Veamos que $0\notin A^\circ$. Supongamos que $0\in A^\circ$. Entonces, $\exists W\in \T$ con $x\in W\subset A$. Sea $W=U\cup V$, con $U\in \T_u$, $V\subset \bb{R}\setminus \bb{Q}$. Como $0\in \bb{Q}$, tenemos que $x\in U$. Por tanto, $\exists U\in \T_u$, con $0\in U\subset [0,1]$, lo cual es trivialmente una contradicción. Análogamente, se tiene que $1\notin A^\circ$. Por tanto, deducimos que $A^\circ \subset~]0,1[$.

            \item[$\supset)$] Como $]0,1[~\in \T_u\subset \T$, tenemos que $]0,1[\subset A^\circ$.
        \end{description}

        Por tanto, tenemos que $A^\circ = ]0,1[$. Además, en el segundo apartado hemos probado que $A\in C_\T$, por lo que $A=\ol{A}=[0,1]$. Por tanto, tenemos que:
        \begin{equation*}
            A^\circ = ]0,1[,\quad \ol{A}=[0,1],\quad \partial A=\ol{A}\setminus A^\circ = \{0,1\}
        \end{equation*}

        Trabajamos ahora con $B$. De manera análoga al caso anterior, tenemos que $0\notin B^\circ$, por lo que $B^\circ \subset ]0,\sqrt{2}[$, pero como $]0,\sqrt{2}[~\in \T$, tenemos que $B^\circ = ~]0,\sqrt{2}[$. Además, en el segundo apartado hemos probado que $B\in C_\T$, ya que $\sqrt{2}\in \bb{R}\setminus \bb{Q}$. Por tanto, tenemos que $\ol{B}=B$. Por tanto,        
        \begin{equation*}
            B^\circ = ]0,\sqrt{2}[,\quad \ol{B}=[0,\sqrt{2}[,\quad \partial B=\ol{B}\setminus B^\circ = \{0\}
        \end{equation*}

        \item Calcula el interior, la clausura y la frontera de $\{x\}$ en $(\bb{R},\cc{T})$ para todo $x\in \bb{R}$.\\

        Veamos en primer lugar que $\{x\}\in C_\T$:
        \begin{equation*}
            \bb{R}\setminus \{x\} = ]-\infty,x[ ~\cup~ ]x,+\infty[\in \T_u\subset \T
        \end{equation*}
        Por tanto, tenemos que $\{x\}\in C_\T$, por lo que $\ol{\{x\}}=\{x\}$.

        Calculamos ahora su clausura. En el caso de que $x\in \bb{R}\setminus \bb{Q}$, tenemos que $\{x\}\in \T$, por lo que $[\{x\}]^\circ=\{x\}$. Supongamos $x\in \bb{Q}$, y por reducción al absurdo supongamos $\exists W\in \T$ con $x\in W\subset \{x\}$, por lo que $W=\{x\}$. No obstante, $\{x\}\notin \T$, por lo que llegamos a un absurdo y tenemos que $N_x=\emptyset$, por lo que $[\{x\}]^\circ=\emptyset$. Por tanto, tenemos que:
        \begin{equation*}
            [\{x\}]^\circ = \left\{\begin{array}{ccl}
                \{x\} & \text{si} & x\in \bb{R}\setminus \bb{Q}\\
                \emptyset & \text{si} & x\in \bb{Q}
            \end{array}\right.
        \end{equation*}

        Aplicando la definición de frontera, tenemos que:
        \begin{equation*}
            \partial \{x\} = \ol{\{x\}}\setminus [\{x\}]^\circ = \left\{\begin{array}{ccl}
                \emptyset & \text{si} & x\in \bb{R}\setminus \bb{Q}\\
                \{x\} & \text{si} & x\in \bb{Q}
            \end{array}\right.
        \end{equation*}
    \end{enumerate}
\end{ejercicio}



\begin{ejercicio}
    Sobre $\bb{R}$ consideramos la siguiente familia de subconjuntos:
    \begin{equation*}
        \cc{B}=\left\{
        \left]x-\frac{1}{n}, x+\frac{1}{n}\right[~~
        \bigcup ~~]n,+\infty[ ~\left| x\in \bb{R}, n\in \bb{N}\right.
        \right\}
    \end{equation*}

    \begin{enumerate}[label=\alph*)]
        \item Prueba que existe una única topología $\cc{T}$ en $\bb{R}$ tal que $\cc{B}$ es una base de $\cc{T}$.

        Hemos de comprobar que se dan las condiciones del Teorema \ref{teo:TopoGenerada_Bases}:
        \begin{enumerate}
            \item[B1)] Se tiene de forma directa, ya que:
            \begin{equation*}
                \bb{R}\subset \left(\bigcup\limits_{x\in \bb{R},~n\in \bb{N}} \left]x-\frac{1}{n}, x+\frac{1}{n}\right[
                \cup~]n,+\infty[\right) \subset \bigcup\limits_{x\in \bb{R},~n\in \bb{N}} \left]x-\frac{1}{n}, x+\frac{1}{n}\right[ \subset \bigcup_{x\in \bb{R}}\{x\}=\bb{R}
            \end{equation*}

            Por doble inclusión se tiene de forma directa.

            \item[B2)] Sean $B_1,B_2\in \cc{B}$, consideramos $z\in B_1\cap B_2$, por lo que $\exists x_1,x_2\in \bb{R}$ y $\exists n_1,n_2\in \bb{N}$ tal que:
            $$z\in \left(\left]x_1-\frac{1}{n_1}, x_1+\frac{1}{n_1}\right[~~ \bigcup ~~]n_1,+\infty[\right) \bigcap \left(\left]x_2-\frac{1}{n_2}, x_2+\frac{1}{n_2}\right[~~ \bigcup ~~]n_2,+\infty[\right)$$

            Calculamos la intersección:
            \begin{multline*}
                B_1\cap B_2 = \left]\max \left\{x_1-\frac{1}{n_1}, x_2-\frac{1}{n_2}\right\}, \min \left\{x_1+\frac{1}{n_1}, x_2+\frac{1}{n_2}\right\}\right[~~\bigcup~~ \\
                ~~\bigcup~~  ]\max\{n_1,n_2\}, +\infty[
            \end{multline*}

            Definimos por tanto $a,c\in \bb{R}$, $b,d,e\in \bb{N}$ de forma que:
            \begin{equation*}
                B_1\cap B_2 = \left]a+\frac{1}{b}, c+\frac{1}{d}\right[~~\bigcup~~ ]e, +\infty[
            \end{equation*}

            Distinguimos en función de dónde se encuentre $z\in B_1\cap B_2$:
            \begin{itemize}
                \item $z\geq e$.

                Entonces, definimos $n_3=e$, y sea el abierto básico $$B_3=\left](n_3+1)-\frac{1}{n_3}, (n_3+1)+\frac{1}{n_3}\right[~~ \bigcup ~~]n_3,+\infty[ = ]n_3,+\infty[$$

                Tenemos que $z\in B_3$, y $B_3\subset B_1\cap B_2$, ya que $n_3=e\geq n_1,n_2$.

                \item $z<e$.
                
                Como $z<e$, tenemos que $a-\frac{1}{b}<z<c+\frac{1}{d}$. Por tanto, elijo $n_3\in \bb{N}$ tal que:
                \begin{equation*}
                    z-\frac{1}{n_3} > a-\frac{1}{b}
                    \qquad 
                    z+\frac{1}{n_3} < c+\frac{1}{d}
                \end{equation*}
                Esto siempre es posible, ya que $\left\{\frac{1}{n}\right\}_{n\in \bb{N}}\to 0$. Veamos además que $n_3>n_1$:
                \begin{equation*}
                    \left\{
                    \begin{array}{l}
                        z-\frac{1}{n_3}>a-\frac{1}{b}\geq x_1-\frac{1}{n_1} \Longrightarrow z-\frac{1}{n_3}+\frac{2}{n_1} \geq x_1+\frac{1}{n_1}
                        \\ \\
                        z+\frac{1}{n_3}<c+\frac{1}{d}\leq x_1+\frac{1}{n_1}
                    \end{array}
                    \right.
                \end{equation*}

                Por tanto, tenemos que:
                \begin{equation*}
                    z+\frac{1}{n_3}\leq x_1+\frac{1}{n_1} \leq z-\frac{1}{n_3}+\frac{2}{n_1} \Longrightarrow
                    \cancel{z}+\frac{1}{n_3} \leq \cancel{z}-\frac{1} {n_3}+\frac{2}{n_1}
                    \Longrightarrow \frac{2}{n_3}\leq \frac{2}{n_2} \Longrightarrow n_3\geq n_1
                \end{equation*}

                Por tanto, tenemos que $n_3\geq n_1$. Análogamente tenemos que $n_3\geq n_2$, por lo que $n_3\geq \max \{n_1,n_2\}$. Por tanto, tenemos que $$]n_3,+\infty[~\subset~ ]\max\{n_1,n_2\},+\infty[ ~=~ ]e,+\infty[$$

                Por tanto, definimos $B_3=\left]z-\frac{1}{n_3}, z+\frac{1}{n_3}\right[ ~~\bigcup~~ ]n_3,+\infty[$, y tenemos que $z\in B_3$ y $B_3\subset B_1\cap B_2$.
            \end{itemize}
        \end{enumerate}

        \item Calcula una base de entornos en $x\in \bb{R}$ de $\cc{T}$ no trivial.

        En primer lugar, hemos de notar que los abiertos de $\T$ no están acotados superiormente. Además, decimos que $N\in N_x$ si $\exists U\in \T$ con $x\in U\subset N$. Como $U$ no está acotado, los entornos de $x$ tampoco lo están. Por tanto, una base de entornos de $x$ es:
        \begin{equation*}
            \beta_x=\left\{\left]x-\frac{1}{n}, +\infty \right[~,~n\in \bb{N}\right\}
        \end{equation*}
        Tenemos que $\beta_x$ es una base, y en especial numerable.

        \item Prueba que $]1,+\infty[$ es un abierto de $\cc{T}$ pero $]-\infty, 1[$ no lo es.

        Veamos en primer lugar $]1,+\infty[$. Sea $n=1$, $x\geq 2$. Entonces,
        \begin{equation*}
            ]1,+\infty[~=~]x-1, x+1[~\cup~]1,+\infty[ 
        \end{equation*}

        Por tanto, tenemos que $]1,+\infty[~\in  \T$. No obstante, $]-\infty, 1[$ no es un abierto, por no estar acotado inferiormente. Todos los abiertos están acotados inferiormente por $\min\left\{x-\frac{1}{n}, n\right\}$, pero $]-\infty, 1[$ no, por lo que no es un abierto.

        \item Prueba que $\cc{T}\subsetneq \cc{T}_u$, donde $\cc{T}_u$ es la topología usual en $\bb{R}$.

        Sabemos que una base de $\T_u$ son las bolas abiertas. Además, los intervalos del tipo $]n,+\infty[$ también son abiertos métricos, ya que para todo $x>n$, $\exists r\in \bb{R}^+$ tal que $B(x,r)\subset ]n,+\infty[$.

        Por tanto, tenemos que todo $U\in \T$ es la unión de dos abiertos métricos. Como la unión de dos abiertos es un abierto, tenemos que $U\in \T_u$.
        
        No obstante, la otra inclusión no se da, ya que $]-\infty, 1[$ es un abierto para $\T_u$ pero no para $\T$.

        \item Calcula la clausura, el interior y la frontera de los conjuntos $]-\infty, 2]$ y $[2,+\infty[$.\\

        Empezamos con el intervalo $]-\infty, 2]$. Veamos en primer lugar que es un cerrado:
        \begin{equation*}
            \bb{R}\setminus~]-\infty, 2] = ]2,+\infty[\in \T
        \end{equation*}
        Por tanto, como es un cerrado, tenemos que $\ol{]-\infty, 2]}=]-\infty, 2]$. 
        Calculemos ahora su interior. Veamos que $(]-\infty, 2])^\circ = \emptyset$.

        Por reducción al absurdo, sea $x\in (]-\infty, 2])^\circ$. Entonces, por la caracterización de los puntos interiores, tenemos que $\exists U\in \T$ con $x\in U\subset ]-\infty, 2]$, por lo que $U$ está acotado superiormente por el 2. No obstante, tenemos que $U$ es un abierto, por lo que llegamos a una contradicción ya que los abiertos en esta topología no están acotados superiormente. Por tanto, se tiene que $(]-\infty, 2])^\circ = \emptyset$.

        La frontera tenemos que es:
        \begin{equation*}
            \partial ]-\infty, 2] = ]-\infty, 2]\setminus \emptyset = ]-\infty, 2]
        \end{equation*}
        \vspace{1cm}

        Trabajamos ahora con $[2,+\infty[$. Veamos que $\ol{[2,+\infty[}=\bb{R}$:
        \begin{description}
            \item[$\subset)$] Trivial.
            \item[$\supset)$] Sea $x\in \bb{R}$. Entonces, tenemos que $[2,+\infty[~\cap B\neq \emptyset$ para todo $B\in \cc{B}$ con $x\in B$, ya que los abiertos básicos tampoco están acotados superiormente, y la intersección de dos intervalos no acotados superiormente es un intervalo no acotado superiormente.
            
            Por tanto, por la caracterización de los puntos adherentes tenemos que $x\in \ol{[2,+\infty[}$.
        \end{description}

        Veamos ahora que $([2,+\infty[)^\circ=]2,+\infty[$.
        \begin{description}
            \item[$\subset)$] Tenemos que $([2,+\infty[)^\circ\subset [2,+\infty[$. No obstante, el 2 no es un punto interior, $\nexists B\in \cc{B}$ con $2\in B\subset [2,+\infty[$.

            \item[$\supset)$] Sea $x>2$. Entonces, $]2,+\infty[\in \T$, y $x\in~]2,+\infty[~\subset [2,+\infty[$, por lo que $x$ es un punto interior.
        \end{description}

        Por tanto, tenemos que la frontera buscada es:
        \begin{equation*}
            \partial [2,+\infty[ = \bb{R}\setminus ]2,+\infty[ ~=~]-\infty, 2]
        \end{equation*}
    \end{enumerate}
\end{ejercicio}


\begin{ejercicio}\label{Ej:3.1.18}
    Sea $(X,\cc{T})$ un espacio topológico, y sea $\cc{B}$ una base de $\cc{T}$. Prueba que, para cada punto $x\in X$, la familia:
    \begin{equation*}
        \beta_x = \{B\in \cc{B}\mid x \in B\}
    \end{equation*}
    es una base de entornos abiertos del punto $x$.

    Para demostrar que $\beta_x$ es una base de entornos, tenemos que ver que dado $N\in N_x$, entonces $\exists B_x\in \cc{B}$ con $x\in B_x\subset V$.

    Como tenemos que $N\in N_x$, entonces $\exists U\in  \T$ con $x\in U\subset N$. Por ser $\cc{B}$ una base de la topología, tenemos que $U=\bigcup\limits_{B\in \cc{B}}B$, por lo que $\exists B_x\in \cc{B}$ tal que $$x\in B_x\subset \bigcup\limits_{B\in \cc{B}}B=U\subset N$$

    Por tanto, tenemos que $\beta_x$ es una base de entornos.
\end{ejercicio}


\begin{ejercicio}
    En $\bb{R}^2$ se considera para cada $z\in \bb{R}^2$, la familia de conjuntos $\beta_z=\{\{z\}\cup A_{\veps}\}_{\veps>0}$, donde $A_{\veps}$ es una bola abierta de centro $z$ y radio $\veps$ a la que se le han quitado un número finito de radios. Demuestra que $\beta_z$ es una base de entornos de $z$ para alguna topología $\cc{T}$ en $\bb{R}^2$. $(\bb{R}^2, \cc{T})$ recibe el nombre de \textbf{plano agrietado}.\\

    En primer lugar, dado $z\in \bb{R}^2, \veps\in \bb{R}^+, I\subset \bb{R}$, seguiremos la siguiente notación:
    \begin{equation*}
        \begin{split}
            R^{\veps,z}&=\{\text{Conjunto de radios de la bola } B(z,\veps)\}\\
            R^{\veps,z}_I&= \{\text{Subconjunto de radios de la bola }B(z,\veps) \text{ indexados en }I\}\subset R^{\veps,z}\\
            A_\veps^z &= B(z,\veps)\setminus R_{I}^{\veps,z},\text{ con }I\text{ finito.}
        \end{split}
    \end{equation*}

    Para ello, usamos el Teorema \ref{teo:TopGenerada_BasesEntornos}, por lo que comprobamos las siguientes 4 condiciones:
    \begin{enumerate}
        \item[V1)] Evidentemente, se tiene que $\beta_z\neq \emptyset$. De hecho, tiene una cantidad no numerable de elementos, ya que $\veps\in \bb{R}^+$.

        \item[V2)] Trivialmente, se tiene que si $V\in \beta_z$, entonces $z\in V$ por definición de $\beta_z$.

        \item[V3)] Sea $V_1=\{z\}\cup A_{\veps_1}^z$, $V_2=\{z\}\cup A_{\veps_2}^z$. Entonces, $V_1\cap V_2=\{z\}\cup (A_{\veps_1}^z\cap A_{\veps_2}^z)$.
        
        Veamos ahora el valor de $A_{\veps_1}^z\cap A_{\veps_2}^z$. Tenemos que:
        \begin{gather*}
            A_{\veps_1}^z=B(z,\veps_1)\setminus R_I^{\veps_1, z},\text{ con $I$ finito.} \\
            A_{\veps_2}^z=B(z,\veps_2)\setminus R_J^{\veps_2, z},\text{ con $J$ finito.}
        \end{gather*}

        Sea ahora $\veps_3=\min\{\veps_1, \veps_2\}$. Entonces,
        $$A_{\veps_1}^z\cap A_{\veps_2}^z
        =B(z,\veps_3)\setminus (R_I^{\veps_1, z}\cup R_J^{\veps_2, z})
        =B(z,\veps_3)\setminus (R_I^{\veps_3, z}\cup R_J^{\veps_3, z})
        =B(z,\veps_3)\setminus R_{I\cup J}^{\veps_3, z}$$

        Como $I,J$ son finitos, tenemos que $I\cup J$ es finito. Por tanto, $A_{\veps_1}^z\cap A_{\veps_2}^z=A_{\veps_3}^z$, y $V_3=V_1\cap V_2=\{z\}\cup A_{\veps_3}^z\in \beta_z$.

        \item[V4)] Sea $V=\{z\}\cup A_{\veps}^z=\{z\}\cup B(z,\veps)\setminus R_{I}^{\veps,z}$, con $I$ finito. Entonces, $\exists \delta\in \bb{R}$, con $0<\delta<\veps$, y consideramos $V'=\{z\}\cup A_{\delta}^z=\{z\}\cup B(z,\delta)\setminus R_I^{\delta,z}\subset V$, ya que $A_\delta^z\subset A_\veps^z$ por ser una bola centrada en el mismo punto, de menor radio, a la que se le han quitado los mismos radios.

        Dado $x\in V'$, consideramos $\gamma\in \bb{R}^+$. Notemos que $B(x,\gamma)\in \beta_x$, ya que $B(x,\gamma)=\{x\}\cup B(x,\gamma)\setminus R_I^{\delta,x}$, con $I=\emptyset$. Busquemos ahora $\gamma\in \bb{R}^+$ tal que $B(x,\gamma)\subset V$.

        Para ello, en primer lugar tenemos que, como $x\in V'\subset V$, entonces $d(x,z)<\veps$, por lo que $\exists \gamma'\in \bb{R}^+$ tal que $d(x,z)+\gamma'<\veps$. De esta forma, tendríamos que $B(x,\gamma')\subset B(z,\veps)$. No obstante, podríamos tener que alguno de los radios eliminados cortase a $B(x,\gamma')$, provocando que la inclusión no se dé.
        
        Buscamos ahora $\gamma\in \bb{R}^+$ de forma que $B(x,\gamma)\subset B(x,\gamma')$ pero que no esté cortada por ningún radio. Para ello, sabemos que $x\in A_{\veps}^z$, por lo que no se encuentra en ningún radio eliminado. Es decir, $d(x,R_i^{\veps,z})>0~\forall i\in I$. Consideramos $\gamma_i=d(x,R_i^{\veps,z})>0$ la distancia desde $x$ hasta cada radio medida de forma perpendicular. Entonces, sea ahora $\gamma=\min\{\gamma', \gamma_i\mid i\in I\}$, que existe por ser $I$ finito. Tenemos por tanto que $\gamma\leq\gamma'$, por lo que $B(x,\gamma)\subset B(x,\gamma')$. Además, como $\gamma\leq \gamma_i$, tenemos que $B(x,\gamma)$ no es cortada por ningún radio. 
        Tenemos por tanto que, dado $x\in V'$, $\exists V_x=B(x,\gamma)\in \beta_x$ y $V_x\subset V$.
    \end{enumerate}
\end{ejercicio}

\begin{ejercicio}
    Sea $(X,\cc{T})$ un espacio topológico, $x\in X$ y sea $\beta_x$ una base de entornos de $x$. Prueba que la familia $\wt{\beta_x}=\{V^\circ\mid V\in \beta_x\}$ es una base de entornos abiertos del punto $x$.\\

    Para ver si $\wt{\beta_x}$ es una base de entornos de $x\in X$, es necesario ver que $\beta_x\subset N_x$ y que dado $N\in N_x$, $\exists \wt{V}\in \wt{\beta_x}$ tal que $\wt{V}\subset N$.

    Veamos en primer lugar que $\wt{\beta_x}\subset N_x$. Sea $\wt{V}\in \wt{\beta_x}$. Entonces, $\wt{V}=V^\circ$, con $V\in \beta_x$. Por ser $V\in \beta_x$, tenemos que $\exists U\in \T$ con $x\in U\subset V$. Por tanto, como $U\subset V$, tenemos $U\subset V^\circ$, ya que el interior es el abierto más grande contiendo en $V$. Por tanto, tenemos que $x\in U\subset V^\circ \subset V$, por lo que $\wt{V}=V^\circ \in N_x$.

    Sea ahora $N\in N_x$. Por ser $\beta_x$ una base de entornos, tenemos que $\exists V\in \beta_x$ tal que $V\subset N$. Por definición de $\wt{\beta_x}$, tenemos que $\exists \wt{V}=V^\circ \in \wt{\beta_x}$, y se tiene que $V^\circ\subset V\subset N$. Por tanto, es efectivamente una base de entornos.
\end{ejercicio}

\begin{ejercicio}
    En el espacio topológico $(\bb{R}, \cc{T}_S)$ de la recta de Sorgenfrey, calcula la clausura de los siguientes subconjuntos:
    \begin{enumerate}
        \item $\bb{N}$:

        Veamos que los naturales son un conjunto cerrado. Para ello, vemos si su complementario es abierto:
        \begin{equation*}
            \bb{R}\setminus \bb{N} = \bb{R}^-\cup [0,1[\cup \left(\bigcup_{n\in \bb{N}}]n,n+1[\right)
        \end{equation*}

        Veamos que esa unión numerable es abierta. Para ello, vemos en primer lugar si $]n,n+1[\in \T_S$, con $n\in \bb{N}$. Esto se da si y solo si, dado $x\in ]n,n+1[$, se tiene que $\exists \veps\in \bb{R}^+$ tal que $[x,x+\veps[~\subset~ ]n,n+1[$.

        Sea $x\in \bb{R}$ tal que $n<x<n+1$. Entonces, $(n+1)-x>0$. Entonces, sea $\veps=\frac{(n+1)-x}{2}$, y se tiene que $[x,x+\veps[~\subset~]n,n+1[$. Claramente, $x>n$. Veamos ahora que $x+\veps<n+1$:
        \begin{equation*}
            x+\veps=x+\frac{(n+1)-x}{2}
            = \frac{(n+1)+x}{2}
            < n+1 \Longleftrightarrow n+1+x<2n+2 \Longleftrightarrow x<n+1
        \end{equation*}

        Por tanto, tenemos que dado $n\in \bb{N}$, el conjunto $]n,n+1[$ es un abierto, por lo que esa unión numerable también lo es. Además, de forma análoga se demuestra que $\bb{R}^-$ y $[0,1[$ también lo son. Por tanto, $\bb{R}\setminus \bb{N}$ es un abierto, por lo que, $\bb{N}\in C_{\T_S}$. Por tanto,
        $$\ol{\bb{N}}=\bb{N}$$
        
        \item $\bb{Z}$:
        
        Veamos que los enteros son un conjunto cerrado. Para ello, vemos si su complementario es abierto:
        \begin{equation*}
            \bb{R}\setminus \bb{Z} = \bigcup_{z\in \bb{Z}}]z,z+1[
        \end{equation*}
        
        Tenemos que $\bb{R}\setminus \bb{Z}$ es claramente un abierto, con una demostración análoga a la del apartado anterior. Por tanto, $\bb{Z}\in C_{\T_S}$, por lo que
        $$\ol{\bb{Z}}=\bb{Z}$$
        
        \item $\bb{Q}$:

        Veamos que $\ol{\bb{Q}}=\bb{R}$. La inclusión $\ol{\bb{Q}}\subset \bb{R}$ es evidente. Veamos la otra inclusión.

        Sea $x\in \bb{R}$, y consideramos $U\in \T_s$ con $x\in U$, por lo que $\exists \veps\in \bb{R}^+$ tal que $[x,x+\veps[~\subset U$. Por la densidad de $\bb{Q}$ en $\bb{R}$, $\exists q\in \bb{Q}$ tal que $q\in [x,x+\veps[~\subset U$. Por tanto, se tiene que $\bb{Q}\cap U\neq \emptyset$.
        
        Al ser esto para todo $U\in \T_s$ con $x\in U$, por la caracterización de los puntos adherentes tenemos que $x\in \ol{\bb{Q}}$.
        
        \item $]a,b]$:

        Veamos que $]a,b]$ no es un cerrado. Tenemos que $\bb{R}\setminus~]a,b]=~]-\infty, a]\cup ~]b,+\infty[$, y consideramos ahora el extremo superior del primer elemento de la unión, $a\in \bb{R}\setminus~]a,b]$. Tenemos que $\nexists \veps\in \bb{R}^+$ tal que $[a,a+\veps[~\subset \bb{R}\setminus~]a,b]$, por lo que $\bb{R}\setminus~]a,b]$ no es un abierto, y por tanto $]a,b]$ no es un cerrado.

        Veamos ahora que $[a,b]$ sí lo es. Su complementario $\bb{R}\setminus [a,b]=~]-\infty, a[~\cup ~]b,+\infty[$ es la unión de dos abiertos y, por tanto, es un abierto. Por tanto, $[a,b]$ es un cerrado.

        Calculemos ahora la clausura. Sea $\ol{]a,b]}=A$. Entonces, $]a,b]\subset A$. Además, como la clausura es el menor cerrado que contiene a $]a,b]$ y $[a,b]\in C_{\T_S}$, tenemos que $A\subset [a,b]$. Por tanto,
        \begin{equation*}
            ]a,b]\subset A\subset [a,b]
        \end{equation*}

        Como la clausura es un cerrado, tenemos que $\ol{]a,b]}=A=[a,b]$.
        
        \item $[a,b[$:

        Veamos que es un cerrado. Para ello, tenemos que su complementario es $\bb{R}\setminus [a,b[~=~]-\infty, a[~\cup~[b,+\infty[$, que es la unión de dos abiertos. Entonces, al ser su complementario un abierto, tenemos que $[a,b[$ es un cerrado.

        Por tanto, tenemos que $\ol{[a,b[}=[a,b[$.
        
        \item $A=\left\{\dfrac{1}{n}\mid n\in \bb{N}\right\}$:

        Veamos que $A$ no es un cerrado:
        \begin{equation*}
            \bb{R}\setminus A = \bb{R}^-\cup \{0\} \cup~ ]1,+\infty[~ \cup \left(\bigcup_{n\in \bb{N}}\left]\frac{1}{n},\frac{1}{n+1}\right[\right)
        \end{equation*}
        Este conjunto no es abierto, ya que dado $0\in \bb{R}\setminus A$, no existe $\veps\in \bb{R}^+$ tal que $[0,0+\veps[\subset\bb{R}\setminus A$, ya que $\forall\veps\in \bb{R}^+,~\exists n\in \bb{N}$ tal que $\frac{1}{n}\in~ [0,0+\veps[$ por ser $\left\{\frac{1}{n}\right\}\to 0$.

        Por tanto, $A$ no es un cerrado. No obstante, $A\cup \{0\}$ sí lo es, ya que su complementario es unión de abiertos.

        Por tanto, como $\ol{A}$ es el menor cerrado que contiene a $A$, tenemos que $A\subset \ol{A}\subset A\cup \{0\}$. Como $\ol{A}$ es un cerrado, tenemos que 
        $$\ol{A}=A\cup \{0\}=\left\{\dfrac{1}{n}\mid n\in \bb{N}\right\}\bigcup~\{0\}$$
        
        \item $B=\left\{-\dfrac{1}{n}\mid n\in \bb{N}\right\}$:

        Veamos que es un cerrado:
        \begin{equation*}\begin{split}
            \bb{R}\setminus B &= \bb{R}^+\cup \{0\} \cup~ ]-\infty,1[~ \cup \left(\bigcup_{n\in \bb{N}}\left]-\frac{1}{n},-\frac{1}{n+1}\right[\right)
            =\\&= [0,+\infty[ \cup~ ]-\infty,1[~ \cup \left(\bigcup_{n\in \bb{N}}\left]-\frac{1}{n},-\frac{1}{n+1}\right[\right)
        \end{split}\end{equation*}
        En este caso, tenemos que el complementario es la unión de tres abiertos, por lo que es un abierto. Por tanto, tenemos que $B$ es cerrado, por lo que $B=\ol{B}$.
    \end{enumerate}
\end{ejercicio}

\begin{ejercicio}
    En $(\bb{R}, \cc{T}_{CF})$ calcula la clausura, el interior y la frontera de:
    \begin{enumerate}
        \item $\bb{N}$:

            En $(\bb{R}, \cc{T}_{CF})$, tenemos que $C_{\T_{CF}}=\{\bb{R}\}\cup \{A\subset \bb{R} \mid A \text{ finito}\}$. Además, sabemos que $\bb{N}\subset \ol{\bb{N}}$, y $\ol{\bb{N}}\in C_{\T_{CF}}$. Por tanto, como $\bb{N}$ es un conjunto infinito, tenemos que $\ol{\bb{N}}=\bb{R}$.

            Análogamente, calculemos ahora los abiertos de esta topología. Tenemos que $\T_{CF}=\{\emptyset\}\cup \{A\subset \bb{R}\mid \bb{R}\setminus A\text{ finito}\}$. Como $\bb{R}$ no es numerable, para que el complementario de $A$ sea finito, $A$ tampoco puede ser numerable. Como $[\bb{N}]^\circ \subset \bb{N}$ numerable, tenemos que $[\bb{N}]^\circ$ es numerable. Además, como $[\bb{N}]^\circ \in \T_{CF}$, tenemos que es el conjunto vacío o no es numerable. Por tanto, $[\bb{N}]^\circ = \emptyset$.
            
            Por último, por definición tenemos que $\partial \bb{N}=\bb{R}$.
            
        \item $\bb{Z}$:
            
            Sabemos que $\bb{Z}\subset \ol{\bb{Z}}$, y $\ol{\bb{Z}}\in C_{\T_{CF}}$. Por tanto, como $\bb{Z}$ es un conjunto infinito, tenemos que $\ol{\bb{Z}}=\bb{R}$.

            Respecto al interior, como $[\bb{Z}]^\circ \subset \bb{Z}$ numerable, tenemos que $[\bb{Z}]^\circ$ es numerable. Además, como $[\bb{Z}]^\circ \in \T_{CF}$, tenemos que es el conjunto vacío o no es numerable. Por tanto, $[\bb{Z}]^\circ = \emptyset$.

            Por último, por definición tenemos que $\partial \bb{Z}=\bb{R}$.
            
        \item $\bb{Q}$:
        
            Sabemos que $\bb{Q}\subset \ol{\bb{Q}}$, y $\ol{\bb{Q}}\in C_{\T_{CF}}$. Por tanto, como $\bb{Q}$ es un conjunto infinito, tenemos que $\ol{\bb{Q}}=\bb{R}$.

            Respecto al interior, como $[\bb{Q}]^\circ \subset \bb{Q}$ numerable, tenemos que $[\bb{Q}]^\circ$ es numerable. Además, como $[\bb{Q}]^\circ \in \T_{CF}$, tenemos que es el conjunto vacío o no es numerable. Por tanto, $[\bb{Q}]^\circ = \emptyset$.

            Por último, por definición tenemos que $\partial \bb{Q}=\bb{R}$.
        
        \item $B=\{0,1\}$:

            Como $B$ es finito, tenemos que es cerrado y, por tanto, $\ol{B}=\{0,1\}$.

            Respecto al interior, como $B^\circ \subset B$ numerable, tenemos que $B^\circ$ es numerable. Además, como $B^\circ \in \T_{CF}$, tenemos que es el conjunto vacío o no es numerable. Por tanto, $B^\circ = \emptyset$.

            
            Por último, por definición tenemos que $\partial B=B=\{0,1\}$.
    \end{enumerate}
\end{ejercicio}

\begin{ejercicio}
    Calcula los puntos de acumulación y los puntos aislados del subconjunto $A$ en los siguientes casos:
    \begin{enumerate}
        \item $(X,\cc{T}_t)$ y $A\subset X$, con $|A|\geq 2$.

        En primer lugar, calculamos la adherencia. Como $C_{\T_t}=\{\emptyset,X\}$ y $\ol{A}\in C_{\T_t}$, tenemos que $\ol{A}=X$.

        Para ver los puntos de acumulación tenemos que, fijado $x\in X$, tenemos que el único abierto que contiene a $x$ es $X$. Por tanto, $x\in A'$, ya que:
        \begin{equation*}
            U\cap (A\setminus \{x\})
            = X\cap (A\setminus \{x\})
            = (A\setminus \{x\})
            \neq \emptyset \qquad \forall U\in \T_t,~x\in U
        \end{equation*}
        donde no es el conjunto vacío ya que $|A|\geq 2$.

        Por tanto, $A'=X$. Por consiguiente, no tiene puntos aislados.
        
        \item $(X,\cc{T}_{disc})$ y $A\subseteq X$.

        Como $C_{\T_{disc}}=\cc{P}(X)$, tenemos que $A=\ol{A}$.

        Veamos los puntos aislados. Sea $x\in A$. Entonces, $\{x\}\in \T_{disc}$, y $A\cap \{x\}=\{x\}$, por lo que tenemos que $x$ es un punto aislado de $A$. Por tanto, tenemos que todos los elementos de $A$ son puntos aislados.
        
        Sea ahora $x\in X$, y veamos si es un punto de acumulación. Como todos los puntos de $A$ son aislados, tenemos que $x\notin A$. Además, tenemos que $\{x\}$ es un abierto en $\T_{disc}$. Por tanto, tenemos que $A\cap (\{x\}\setminus \{x\})=\emptyset$, con $\{x\}\in \T$, $x\in \{x\}$. Por tanto, $x$ no es un punto de acumulación, por lo que $A'=\emptyset$.
        
        \item $(X,\cc{T}_{CF})$ y $A\subseteq X$ finito.

        Si $X$ es finito, tenemos que $\T_{CF}=\T_{disc}$, por lo que estamos en el caso anterior. Suponemos $X$ infinito.
        
        Como $A$ es finito, tenemos que $A=\ol{A}$. Dado $x\in A$, para ver los puntos aislados, sabemos que $A\setminus \{x\}$ es finito. Por tanto, consideramos el siguiente abierto:
        \begin{equation*}
            U=X\setminus (A\setminus \{x\}) = (X\setminus A)\cup \{x\} \in \T_{CF}
        \end{equation*}

        Por tanto, tenemos que $A\cap U=A\cap [(X\setminus A)\cup \{x\}] = \{x\}$. Por tanto, tenemos que $x$ es un punto aislado de $A$. Es decir, todos los puntos de $A$ son aislados.

        Sea ahora $x\in X$, y veamos si es un punto de acumulación. Como todos los puntos de $A$ son aislados, tenemos que $x\notin A$. Además, como $A$ es finito, tenemos que $X\setminus A\in \T_{CF}$. Entonces:
        \begin{equation*}
            A\cap ((X\setminus A)\setminus \{x\})
            = A\cap (X\setminus (A\cup \{x\})) = \emptyset
        \end{equation*}
        
        Entonces, $x\notin A'$, y por tanto $A'=\emptyset$.
        
        
        \item $(\bb{R}, \cc{T}_S)$ y $A=]0,1]$.

        Como $\bb{R}\setminus A = ]-\infty, 0]\cup~]1,+\infty[$, que no es un abierto. No obstante, $[0,1]$ sí es un cerrado, por lo que $\ol{A}=[0,1]$.

        Para $1\in \ol{A}$, tenemos que $U=[1,2[$. Además, $U\cap A=\{1\}$, por lo que tenemos que $1$ es un punto aislado de $A$.

        Para $x\in \ol{A}\setminus \{1\}=[0,1[$, tenemos que $A\cap (U\setminus \{x\})\neq \emptyset$ para todo $U\in \T_S$ y $x\in U$. Esto se debe a que, como $U\in \T$, entonces $\exists \veps\in \bb{R}^+$ con $x+\frac{\veps}{n}\in U$ para todo $n\in \bb{N}$. Además, $x+\frac{\veps}{n}\in A$ para cierto $n\in \bb{N}$, ya que $\frac{\veps}{n}>0$ y $\left\{\frac{\veps}{n}\right\}\to 0$. Por tanto, $A'=[0,1[$.

        
    \end{enumerate}
\end{ejercicio}

\begin{ejercicio}
    ¿Para qué espacios topológicos $(X,\cc{T})$ se cumple que $X$ es el único subconjunto denso?\\

    Nos pide que para qué espacios topológicos $(X,\cc{T})$ se cumple que, dado $A\subseteq X$, se tiene que $\ol{A}=X\Longrightarrow A=X$.

    Dado $x\in X$ veamos que $\{x\}\in \T$. Para ello, razonamos con que ${X}\setminus \{x\}\in C_\T$. Veámoslo:
    \begin{equation*}
        X\setminus \{x\} \subset \ol{X\setminus \{x\}} \subset X
    \end{equation*}
    Como $X\neq X\setminus \{x\}$, entonces, $X\setminus \{x\}$ no es denso, por lo que $\ol{X\setminus \{x\}}\neq X$. Por tanto, tenemos que $X\setminus \{x\}=\ol{X\setminus \{x\}}$, teniendo entonces que es un cerrado. 

    Por tanto, tenemos que $\{x\}\in \T$ para todo $x\in X$; y como la unión de abiertos es un abierto, se tiene que $\T=\cc{P}(X)$. Por tanto, se tiene que la topología es la discreta, $\T=\T_{disc}$.
\end{ejercicio}

\begin{ejercicio}
    Sea $(X,\T)$ un espacio topológico y $D$ un subconjunto denso de $(X,\T)$. Demuestra que para todo subconjunto abierto $A\subset X$ se tiene $\ol{D\cap A}=\ol{A}$.
    \begin{description}
        \item[$\subset)$]
            Tenemos que $\ol{D\cap A}\subset \ol{D}\cap \ol{A}$. Como $D$ es denso, tenemos que $\ol{D}=X$, por lo que:
            \begin{equation*}
                \ol{D\cap A}\subset \ol{D}\cap \ol{A} = X\cap \ol{A}=\ol{A}
            \end{equation*}
            
        \item[$\supset)$] Sea $x\in \ol{A}$. Entonces, por la caracterización de los puntos adherentes tenemos que $A\cap U\neq \emptyset,~\forall U\in \T$, con $x\in U$. Además, como $A,U$ son abiertos, tenemos que su intersección es un abierto.

        Como $D$ es denso, tenemos que $\ol{D}=X$. Por tanto, $U'\cap D\neq \emptyset$ para todo $U'\in \T$. Tomando $U'=A\cap U$ dependiente de $U$, tenemos que
        $$\emptyset \neq (A\cap U)\cap D = (A\cap D)\cap U\neq \emptyset,~~\forall U\in \T,~x\in U$$
        Habiendo llegado por tanto a que $x\in \ol{A\cap D}$.
    \end{description}
\end{ejercicio}

\begin{ejercicio}
    Prueba que en $(\bb{R}^n, \T_u)$ se verifica:
    \begin{enumerate}[label=\alph*)]
        \item $\ol{B(x,\veps)} = \ol{B}(x,\veps)$.

        \begin{description}
            \item[$\subset)$] 
            Como $B(x,\veps)\subset \ol{B}(x,\veps)\in C_{\T_u}$, y la clausura se define como el menor cerrado que contiene a $B(x,r)$, tenemos que $\ol{B(x,\veps)} \subset \ol{B}(x,\veps)$.
            
            \item[$\supset)$] Sea $y\in \ol{B}(x,\veps)$, por lo que $d(x,y)\leq \veps$. Si $d(x,y)<\veps$, tenemos que $y\in B(x,\veps)\subset \ol{B(x,\veps)}$. Supongamos por tanto que $d(x,y)=\veps$.

            Veamos que $B(x,\veps)\cap V\neq \emptyset$ para todo $V\in \beta_y$ base de entornos de $y$. Notemos $V=B(y,\delta)$, con $\delta\in \bb{R}^+$.

            Para ver que $B(x,\veps)\cap B(y,\delta)\neq \emptyset$, consideramos $z=tx+(1-t)y$, con $t\in \bb{R}$. Veamos si $\exists t\in \bb{R}$ tal que $z$ pertenezca a ambas bolas.

            Tenemos que $z\in B(x,\veps)$ si y solo si $d(x,z)<\veps$. En este punto, aplicamos que $\bb{R}^n$ es un espacio métrico:
            \begin{equation*}
                d(x,z)=\|x-z\|=\|x-tx-(1-t)y\| = |1-t|\cdot \|x-y\|=|1-t|\cdot d(x,y)=\veps\cdot |1-t|
            \end{equation*}
            Por tanto, 
            \begin{equation*}
                z\in B(x,\veps) \Longleftrightarrow d(x,z)<\veps
                \Longleftrightarrow |1-t|<1 \Longleftrightarrow |t|\in ]0,2[.
            \end{equation*}

            Análogamente, tenemos que $z\in B(y,\delta)$ si y solo si $d(y,z)<\delta$. Usando que $\bb{R}^n$ es un espacio métrico:
            \begin{equation*}
                d(y,z)=\|z-y\|=\|tx+(1-t)y-y\|=\|tx-ty\| = |t|\cdot d(x,y)=\veps\cdot |t|
            \end{equation*}
            Por tanto, 
            \begin{equation*}
                z\in B(y,\delta) \Longleftrightarrow d(y,z)<\delta
                \Longleftrightarrow \veps\cdot |t|<\delta \Longleftrightarrow |t|<\frac{\delta}{\veps}.
            \end{equation*}

            Por tanto, necesitamos $0<t<\min\left\{2,\dfrac{\delta}{\veps}\right\}$, que es posible ya que ambos números son positivos y por la densidad de $\bb{R}$ en $\bb{R}$.
            
            Por tanto, como $\forall \veps,\delta \in \bb{R}^+$ hemos encontrado que $\exists t\in \bb{R}$ que cumple que $z$ pertenece a ambas bolas, tenemos que la intersección no es nula.

            Por tanto, tenemos que $y\in \ol{B}(x,\veps)$.
        \end{description}
        
        \item $[\ol{B}(x,\veps)]^\circ = B(x,\veps)$.

        \begin{description}
            \item[$\subset)$] 
                Sea $y\in [\ol{B}(x,\veps)]^\circ\subset \ol{B}(x,\veps)$. Entonces, tenemos que $d(x,y)\leq \veps$. Si $d(x,y)<\veps$, se tiene que $y\in B(x,\veps)$. Veamos ahora que no se puede dar el caso de que $d(x,y)=\veps$. Supongamos que sí, y llegaremos a un absurdo.

                Como $y\in [\ol{B}(x,\veps)]^\circ$, tenemos que $\exists \delta\in \bb{R}^+$ tal que $B(y,\delta)\subset B(x,\veps)$. Consideramos\footnote{Intuitivamente, desde el punto $y$, nos ``alejamos'' una cantidad $\frac{\delta}{2}$, de forma que $z$ estará en la bola de radio $\delta$, pero ya no estará en la bola de radio $\veps$.} ahora $z=y+\dfrac{y-x}{\|y-x\|}\cdot \dfrac{\delta}{2}$.

                Veamos que $z\in B(y,\delta)$ usando que $\bb{R}^n$ es un espacio métrico:
                \begin{equation*}
                    d(z,y)=\|z-y\|=\left\|\dfrac{y-x}{\|y-x\|}\cdot \dfrac{\delta}{2}\right\| = \frac{\|y-x\|}{\|y-x\|}\cdot \frac{\delta}{2} = \frac{\delta}{2} < \delta  \Longrightarrow z\in B(y,\delta)
                \end{equation*}

                Veamos ahora que $z\notin B(x,\veps)$:
                \begin{equation*}
                    d(x,z)=\|z-x\|=\left\|(y-x)\left(1+\frac{\delta}{2\|y-x\|}\right)\right\| = \|y-x\|\cdot \left\|1+\frac{\delta}{2\|y-x\|}\right\| = \veps + \frac{\delta}{2}>\veps
                \end{equation*}

                Por tanto, tenemos que $z\in B(y,\delta), z\notin B(x,\veps)$. Por tanto, hemos llegado a un absurdo, y concluimos que no se puede dar que $d(x,y)=\veps$.
                
            \item[$\supset)$] Tenemos que $B(x,\veps)\subset \ol{B}(x,\veps)$, por lo que:
            \begin{equation*}
                B(x,\veps)=[B(x,\veps)]^\circ \subset [\ol{B}(x,\veps)]^\circ
            \end{equation*}
        \end{description}
        
        \item $\partial \ol{B}(x,\veps) = \partial B(x,\veps) = S(x,\veps)$.

        Usando que las bolas cerradas son cerrados métricos, y las bolas abiertas son abiertos métricos; y usando los dos apartados anteriores, tenemos que:
        \begin{gather*}
            \partial \ol{B}(x,\veps) = \ol{\ol{B}(x,\veps)} \setminus [\ol{B}(x,\veps)]^\circ = \ol{B}(x,\veps)\setminus B(x,\veps)=S(x,\veps) \\
            \partial {B}(x,\veps) = \ol{{B}(x,\veps)} \setminus [{B}(x,\veps)]^\circ = \ol{B}(x,\veps)\setminus B(x,\veps)=S(x,\veps)
        \end{gather*}
        
    \end{enumerate}
    ¿Son ciertas las igualdades anteriores en todo espacio métrico?

    No, ya que tanto en el apartado $a)$  como en el $b)$ hemos usado que $\bb{R}^n$ es un espacio métrico. Por ejemplo, en $(X,d_{disc})$ tenemos que $\T_d=\T_{disc}$. Entonces, consideramos el contraejemplo para $\veps=1$. Entonces,
    \begin{gather*}
        \ol{B}(x,1)=X\not\subset \ol{B(x,1)}=\ol{\{x\}}=\{x\}\\
        [\ol{B}(x,1)]^\circ = [X]^\circ = X \not\subset B(x,1)=\{x\}
    \end{gather*}
\end{ejercicio}

\begin{ejercicio}
    Un conjunto $A$ de un espacio topológico $(X,\T)$ se dice frontera si $A\subset \partial A$. Demuestra que:
    \begin{enumerate}
        \item $A$ es frontera $\Longleftrightarrow A^\circ =\emptyset \Longleftrightarrow \ol{X\setminus A}=X$.

        Demostramos en primer lugar la primera equivalencia:
        \begin{description}
            \item[$\Longrightarrow)$] Tenemos que $A\subset \partial A=\ol{A}\setminus A^\circ$. Además, sabemos que $A^\circ \subset A$.

            Por reducción al absurdo, sea $x\in A^\circ$. Entonces, tenemos que $x\in A$ pero $x\notin \partial A$, por lo que $A\not \subset \partial A$, llegando por tanto a una contracción. Por tanto, $A^\circ = \emptyset$.

            \item[$\Longleftarrow)$] Partimos de que $A^\circ = \emptyset$, por lo que $\partial A=\ol{A}$. Además, también sabemos que $A\subset \ol{A}$. Por tanto, tenemos que $A\subset \partial A$.
        \end{description}

        Veamos ahora la tercera equivalencia. Sabemos que $\ol{X\setminus A}=X\setminus A^\circ$. Por tanto, sabemos que:
        \begin{equation*}
            \ol{X\setminus A}=X
            \Longleftrightarrow 
            A^\circ =\emptyset 
        \end{equation*}
        
        \item En $(\bb{R}, \T_u)$, se tiene que $\bb{Q}$ y $\bb{R}\setminus \bb{Q}$ son conjuntos frontera.

        Para verlo, demostramos que el interior de ambos conjuntos es el conjunto vacío.

        Supongamos que no, y sea $x\in [\bb{Q}]^\circ$. Entonces, $\exists a,b\in \bb{R}$, $a<b$, tal que $x\in ]a,b[~\subset \bb{Q}$. No obstante, por la densidad de $\bb{R}\setminus \bb{Q}$ tenemos que $\exists y\in \bb{R}\setminus \bb{Q}$ tal que $y\in ]a,b[$, por lo que la inclusión vista no se puede dar, llegando por tanto a una contradicción. Tenemos que $[\bb{Q}]^\circ=\emptyset$.

        Análogamente, tenemos que $[\bb{R}\setminus \bb{Q}]^\circ=\emptyset$.

        Por tanto, por la equivalencia vista en el apartado anterior tenemos que ambos conjuntos son frontera.
    \end{enumerate}
\end{ejercicio}

\begin{ejercicio}
    Un conjunto $A$ de un espacio topológico $(X,\T)$ se dice enrarecido si $\left(\ol{A}\right)^\circ=\emptyset$. Demuestra que:
    \begin{enumerate}
        \item Si $A$ es enrarecido, $A$ es frontera.

        Si $A$ es enrarecido, tenemos que $\left(\ol{A}\right)^\circ=\emptyset$. Por tanto, como $A\subset \ol{A}$, tenemos que $A^\circ \subset \left(\ol{A}\right)^\circ = \emptyset$. Por tanto, $A^\circ =\emptyset$, por lo que $A$ es frontera.

        
        \item Un subconjunto frontera y cerrado es enrarecido.

        Al ser frontera, tenemos que $A^\circ = \emptyset$. Al ser cerrado, tenemos que $A=\ol{A}$. Por tanto, $\emptyset = A^\circ = \left(\ol{A}\right)^\circ$, por lo que $A$ es enrarecido.
        
        \item Si $U\in \T$, entonces $\partial U$ es enrarecido.

        Calculamos por tanto $\left[\ol{\partial U}\right]^\circ$, y veamos si es el vacío. Por definición de frontera y sabiendo que $U\in \T$, tenemos que:
        $$\left[\ol{\partial U}\right]^\circ
            = \left[\ol{\ol{U}\setminus U^\circ}\right]^\circ
            = \left[\ol{\ol{U}\setminus U}\right]^\circ$$

        Veamos ahora que ${\ol{U}\setminus U}$ es cerrado. Tenemos que ${\ol{U}\setminus U}=\ol{U}\cap (X\setminus U)$. Tenemos que $U\in \T$, por lo que $X\setminus U\in C_\T$. Por tanto, tenemos que ${\ol{U}\setminus U}$ es la composición de dos cerrados, por lo que es un cerrado. Por tanto, 
        $$\left[\ol{\partial U}\right]^\circ
            = \left[\ol{\ol{U}\setminus U^\circ}\right]^\circ
            = \left[\ol{\ol{U}\setminus U}\right]^\circ
            = \left[\ol{U}\setminus U\right]^\circ$$

        Expresando el complementario como una intersección, y sabiendo que el abierto de la intersección es la intersección de los abiertos, tenemos que:
        $$\left[\ol{\partial U}\right]^\circ
            = \left[\ol{\ol{U}\setminus U^\circ}\right]^\circ
            = \left[\ol{\ol{U}\setminus U}\right]^\circ
            = \left[\ol{U}\setminus U\right]^\circ
            = \left[\ol{U}\cap (X\setminus U)\right]^\circ
            = \left[\ol{U}\right]^\circ\cap \left[X\setminus U\right]^\circ
        $$

        Tenemos que $\left[X\setminus U\right]^\circ=X\setminus \ol{U}$. Por tanto, finalmente llegamos a que 
        $$\left[\ol{\partial U}\right]^\circ
            = \left[\ol{U}\right]^\circ\cap X\setminus\ol{U}
        $$

        No obstante, se tiene que $\left[\ol{U}\right]^\circ\subset \ol{U}$, por lo que:
        $$\left[\ol{\partial U}\right]^\circ
            = \left[\ol{U}\right]^\circ\cap X\setminus\ol{U}
            \subset 
            = \ol{U} \cap X\setminus\ol{U} = \emptyset
        $$

        Por tanto, tenemos que $\partial U$ es enrarecido.
        
        \item Todo subconjunto cerrado y enrarecido es la frontera de un abierto.

        Sea $A$ cerrado y enrarecido, y comprobemos que es la frontera de un abierto. Sea el abierto $U=X\setminus A\in \T$, que es abierto por ser $A$ cerrado. Veamos que $\partial U=A$.
        \begin{equation*}
            \partial U = \partial (X\setminus A) = \partial A = \ol{A}\setminus A^\circ
        \end{equation*}
        donde he usado que la frontera de un conjunto coincide con la frontera de su complementario. Ahora, como $A$ es cerrado, tenemos que $\ol{A}=A$. Además, al ser enrarecido, en el primer apartado hemos visto que $A^\circ = \emptyset$. Por tanto, 
        \begin{equation*}
            \partial U = \partial (X\setminus A) = \partial A = \ol{A}\setminus A^\circ = A\setminus \emptyset = A
        \end{equation*}
        De esta forma, hemos visto que dado un cerrado y enrarecido $A\in C_\T$, es la frontera de un abierto, $U\in \T$.
    \end{enumerate}
\end{ejercicio}

\begin{ejercicio}
    Demuestra que todo subconjunto cerrado de $(\bb{R}^2, \T_u)$ es la frontera de algún subconjunto de $\bb{R}^2$.\\

    Sea $C\in C_{\T_u}$ un subconjunto cerrado de $(\bb{R}^2, \T_u)$. Tenemos que encontrar $A\subset  \bb{R}^2$ tal que $\partial A=C$.

    Tenemos que $(\bb{R}^2, \T_u)$ es 2AN, por lo la topología inducida al restringir sobre $C$ también es 2AN (por ser 2AN es hereditario). Por tanto, $\exists \{B_n\}_{n\in \bb{N}}$ base de $\left(C,(\T_u)_C\right)$. Definimos $A\subset C=\{a_n\mid n\in \bb{N}\}$, donde $a_n\in B_n$ para todo $n\in \bb{N}$. Tenemos que $A$ es numerable, veamos ahora que es denso en $C$.

    Por la definición de $A$, tenemos que $A\cap U\neq \emptyset$ para todo $U\in (\T_u)_C$, ya que $A\cap B_n\neq \emptyset$ para todo $n\in \bb{N}$ y $U=\bigcup\limits_{j\in J\subset \bb{N}}B_j$. Por tanto, tenemos que es denso.

    Por tanto, hemos demostrado que $\left(C,(\T_u)_C\right)$ es separable, y el conjunto denso en $C$ y numerable que contiene es $A$. Es decir, $A\subset C$ con $\ol{A}^C=C$. Veamos que su cierre en $\bb{R}^2$ también es $C$, es decir, $\ol{A}=C$.
    \begin{description}
        \item[$\subset)$] Como $A\subset C$, entonces $\ol{A}\subset \ol{C}$. Como $C$ es cerrado, tenemos que $C=\ol{C}$, por lo que $\ol{A}\subset C$.
        \item[$\supset)$]
        Como $\ol{A}^C=\ol{A}\cap C=C$, tenemos que $C\subset \ol{A}$.
    \end{description}
    
    Por tanto, tenemos que $\ol{A}= C$. Veamos ahora que $A^\circ = \emptyset$.
    Supongamos, por reducción al absurdo, que $x\in A^\circ$. Entonces, $\exists z\in \bb{R},\veps\in \bb{R}^+$ tal que $B(z,\veps)\subset A$, pero esto no es posible ya que las bolas son no numerables y $A$ es numerable. Por tanto, $A^\circ = \emptyset$.

    Entonces, tengo que $\partial A = \ol{A}\setminus A^\circ = C\setminus \emptyset = C$.
\end{ejercicio}

\begin{ejercicio}
    Sea $(X,\T)$ un espacio topológico y $\{A_i\}_{i\in I}$ una familia de subconjuntos de $X$ tal que $\bigcup\limits_{i\in I}[A_i]^\circ=X$. Entonces, $U\in \T$ si y solo si se tiene que $U\cap A_i\in \T_{A_i},~~\forall i\in I$.
    \begin{description}
        \item[$\Longrightarrow$)] Trivial por definición de la topología inducida.

        \item[$\Longleftarrow)$] Veamos que $U\in \T$. Dado $i\in I$, tenemos que $[A_i]^\circ\in \T$, por lo que por definición de topología inducida se tiene que $[A_i]^\circ \cap A_i = [A_i]^\circ \in \T_{A_i}$. Además, tenemos que $U\cap A_i\in \T_{A_i}$. Como la intersección de dos abiertos es un abierto, tenemos que $U\cap A_i\cap [A_i]^\circ =U\cap [A_i]^\circ \in \T_{A_i}$.

        Como $U\cap [A_i]^\circ \in \T_{A_i}$, por definición de topología inducida se tiene que $\exists V_i\in \T$ (tiene el subíndice $i$ ya que depende del valor inicial escogido) tal que $U\cap [A_i]^\circ=V_i\cap A_i$.

        Veamos ahora las uniones en $I$, para las cuales lo anterior es cierto, ya que era cierto para $i$ cualquiera. Veamos primero que $\bigcup\limits_{i\in I} A_i=X$:
        \begin{equation*}
            X=\bigcup_{i\in I} [A_i]^\circ \subset \bigcup_{i\in I} A_i\subset X
        \end{equation*}
        
        Tenemos por tanto los dos siguientes resultados:
        \begin{equation*}
            \bigcup_{i\in I} (V_i\cap A_i) = \left(\bigcup_{i\in I} V_i\right) ~\bigcap~\left(\bigcup_{i\in I} A_i\right)
            = \left(\bigcup_{i\in I} V_i\right) ~\bigcap~ X = \bigcup_{i\in I} V_i
        \end{equation*}
        \begin{equation*}
            \bigcup_{i\in I} (U\cap [A_i]^\circ) = U ~\bigcap~\left(\bigcup_{i\in I} [A_i]^\circ\right)
            = U ~\bigcap~ X = U
        \end{equation*}

        Como $U\cap [A_i]^\circ=V_i\cap A_i$ para todo $i\in I$, tenemos que:
        \begin{equation*}
            U=\bigcup_{i\in I} (U\cap [A_i]^\circ) = \bigcup_{i\in I} (V_i\cap A_i)= \bigcup_{i\in I} V_i \in \T
        \end{equation*}
        donde sabemos que es un abierto por ser unión de abiertos.
    \end{description}
\end{ejercicio}

\begin{ejercicio}
    Sea $(X,\T)$ un espacio topológico y $A\subset X$ un subconjunto no vacío. Sea $a\in A$ y $\beta_a$ una base de entornos de $a$ en $(X,\T)$. Prueba que la familia
    \begin{equation*}
        (\beta_A)_a = \{B\cap A\mid B\in \beta_a\}
    \end{equation*}
    es una base de entornos de $a$ en $(A,\T_A)$.\\

    Demostrado en la Proposición \ref{prop:PropiedadesInducida}.
\end{ejercicio}

\begin{ejercicio}
    Sea $(X,\T)$ un espacio topológico, $A\subset X$ un subconjunto no vacío y $B\subset A$. Prueba que:
    \begin{enumerate}
        \item $B^\circ \cap A \subset B^{\circ A}$. Da un ejemplo de que en general no se tiene la igualdad.

        Demostrado en la Proposición \ref{prop:PropiedadesInducida}.
            
        \item $\partial_A(B)\subset A\cap \partial(B)$. Da un ejemplo de que en general no se tiene la igualdad.

        Demostrado en la Proposición \ref{prop:PropiedadesInducida}.
    \end{enumerate}
\end{ejercicio}

\begin{ejercicio}
    Consideremos el conjunto $A=[-1, 0[~\cup~ ]0,2[~\cup \{3\}$ de $\bb{R}$ con la topología $(\T_u)_A$ inducida en $A$ por $\T_u$.
    \begin{enumerate}
        \item Estudia si los conjuntos $\{3\}$ y $]0,2[$ son abiertos o cerrados en $(A, (\T_u)_A)$.

        Los tres apartados de este ejercicio los resolvemos aplicando la Proposición \ref{prop:PropiedadesInducida}.

        Tenemos que $\{3\}=A\cap~\left]\frac{5}{2}, \frac{7}{2}\right[$, siendo ese intervalo un abierto en $(\bb{R},\T_u)$. Por tanto, $\{3\}\in (\T_u)_A$. Además, $\{3\}=A\cap~\left[\frac{5}{2}, \frac{7}{2}\right]$, siendo ese intervalo cerrado un cerrado en $(\bb{R},\T_u)$. Por tanto, $\{3\}\in C_{(\T_u)_A}$. Es decir, es cerrado y abierto a la vez.
    
        Veamos ahora $]0,2[$. Tenemos que $]0,2[=A\cap~ ]0,2[$, siendo este un intervalo abierto y, por tanto, un abierto en esta topología. Por tanto, $]0,2[\in (\T_u)_A$. Además, $]0,2[=A\cap~ [0,2]$, siendo este un intervalo cerrado y, por tanto, un cerrado en esta topología. Por tanto, $]0,2[\in C_{(\T_u)_A}$.  Es decir, también es cerrado y abierto simultáneamente.
        
        \item Comprueba si $\left[-1,-\frac{1}{2}\right]$ es entorno de $-1$ en $(A,(\T_u)_A)$.

        Tenemos que $\left[-1,-\frac{1}{2}\right]=A\cap \left[-\frac{3}{2},-\frac{1}{2}\right]$, y el segundo es un entorno de $-1$ en $(\bb{R},\T_u)$. por tanto, por la Proposición \ref{prop:PropiedadesInducida}, tenemos que sí es un entorno de $-1$ en $(\T_u)_A$.

        
        \item Calcula la clausura de $[-1, 0[$ en $(A,(\T_u)_A)$.

        Tenemos que $\ol{[-1, 0[}^A = A\cap \ol{[-1, 0[} = A\cap [-1,0]=[-1,0[$.
    \end{enumerate}
\end{ejercicio}

\begin{ejercicio}
    Si $(X,d)$ es un espacio métrico y $A\subset X$ es un subconjunto no vacío, e define $d_A:A\times A\to \bb{R}$ como $d_A(x,y)=d(x,y),~~\forall x,y\in A$. Prueba que:
    \begin{enumerate}
        \item $(A,d_A)$ es un espacio métrico.

        Para ello, y como $A\neq \emptyset$, solo falta ver que, efectivamente $d_A$ es una distancia. Esto se obtiene de forma directa e inmediata, ya que $d$ lo es.
        
        \item $(\T_d)_A=\T_{d_A}$; es decir, la topología inducida en $A$ por $\T_d$ coincide con la topología asociada a $d_A$.

        Tenemos que:
        \begin{equation*}\begin{split}
            \T_{d_A}&=\{U\subset A\mid \exists B_A(x,\veps)\subset U~~ \forall x\in U\}\\
            (\T_d)_A &= \{V\cap A\mid V\in \T_d\}
            = \{V\cap A\mid \exists B(x,\veps)\subset V~~ \forall x\in V\}
        \end{split}\end{equation*}

        Demostramos la igualdad por doble inclusión:
        \begin{description}
            \item[$\subset)$]

            Sea $U\in (\T_d)_A$. Entonces, $U=V\cap A$, con $V\in \T_d$; es decir, existe una bola $B(x,\veps)\subset V$ para todo $x\in V$.

            Como $U\subset V$, tenemos que $\exists B(x,\veps)\subset V$ para todo $x\in U$. Por tanto, $B(x,\veps)\cap A\subset V\cap A=U$. Además, tenemos que $B(x,\veps)\cap A=B_A(x,\veps)$, por lo que tenemos que $\exists B_A(x,\veps)$ tal que $B_A(x,\veps)\subset U$.
            
            Por tanto, $U\in \T_{d_A}$.

            \item[$\supset)$] Sea $U\in \T_{d_A}$, por lo que $\exists B_A(x,\veps)=B(x,\veps)\cap A\subset U$ para todo $x\in U$. Veamos ahora que:
            \begin{equation*}
                U=\bigcup_{x\in U}(B(x,\veps) \cap A) = A~\bigcap~\bigcup_{x\in U}B(x,\veps)
            \end{equation*}
            donde $B(x,\veps)\cap A\subset U$, que hemos visto que existe por ser $U\in \T_{d_A}$.
            \begin{description}
                \item[$\subset)$] Sea $x\in U\subset A$, por lo que $x\in A$. Además, $x\in B(x,\veps)$. Por tanto, $x$ está en la unión descrita.
                
                \item[$\supset)$] Sabemos que $B(x,\veps)\cap A\subset U$ para todo $x\in U$. Como la unión es en $x\in U$, tenemos que la unión sigue siendo un subconjunto de $U$.
            \end{description}

            Por tanto, tenemos que $U=V\cap A$, con $V=\bigcup\limits_{x\in U}B(x,\veps)$ unión de abiertos, por lo que $V\in \T_d$.
        \end{description}
    \end{enumerate}
\end{ejercicio}

\begin{ejercicio} \label{ej:Sucesiones}
    Sea $(X,\T)$ un espacio topológico. Diremos que una sucesión $\{x_n\}_{n\in \bb{N}}$ converge a un punto $x\in X$ si para todo entorno $V\in {N}_x$, existe $n_0\in \bb{N}$ tal que $x_n\in V$ para todo $n\geq n_0$. Si la sucesión $\{x_n\}_{n\in \bb{N}}$ converge a $x$ diremos $x\in \lim\limits_{n\to \infty} x_n$ y diremos que $x$ es un límite de la sucesión $\{x_n\}_{n\in \bb{N}}$. Prueba las siguientes afirmaciones:
    \begin{enumerate}[label=\alph*)]
        \item En un espacio topológico Hausforff (T2), una sucesión convergente tiene un único límite.

        Está demostrado en la Proposición \ref{prop:T2UnicidadLimite}.

        \item Sea $(X, \T)$ un espacio topológico, $\{x_n\}_{n\in \bb{N}}$ una sucesión en $X$, $x\in X$ y $\beta_x$ una base de entornos de $x$. Entonces $x\in \lim\limits_{n\to \infty}x_n$ si y solo si para todo $B\in \beta_x$, existe $n_0\in \bb{N}$ tal que $x_n\in B$ para todo $n\geq n_0$.

        \begin{description}
            \item[$\Longrightarrow)$]
                Como $x$ es un límite de la sucesión, tenemos que se cumple la condición del enunciado para todo $N\in N_x$. Como $\beta_x\subset N_x$, tenemos que también se cumple para todo $B\in \beta_x$.
                
            \item[$\Longleftarrow)$]
                Tenemos que la condición del enunciado se cumple para todo entorno básico. Veamos que se cumple también para todo entorno.

                Sea $N\in N_X$. Entonces, por definición de base de entornos, tenemos que $\exists B\in \beta_x$ tal que $B\subset N$. Por hipótesis, tenemos que $\exists n_0\in \bb{N}$ tal que $x_0\in B\subset N$ para todo $n\geq n_0$, por lo que se tiene.
        \end{description}   

        \item En un espacio $(X,\T_t)$ con la topología trivial, cualquier sucesión en $X$ converge a todos los puntos de $X$ (una sucesión puede converger a más de un punto).

        Tenemos que $\T_t=\{\emptyset, X\}$. Dado $x\in X$, calculemos en primer lugar $N_x$. Tenemos que $N\in N_x$ si y solo si $\exists U\in \T$ con $x\in U\subset N$. Como $x\in U$, entonces $U\neq \emptyset$, por lo que $U=X$. Como $X=U\subset N$, tenemos que $N=X$. Por tanto, hemos visto que dado $x\in X$, $N_x=\{X\}$.

        Veamos ahora que, dado $x\in X$, cualquier sucesión en $X$ converge a $x$. Como $N_x=\{X\}$ y $x_n\in X$, tenemos que $\forall N\in N_x$ existe $n_0\in \bb{N}$ tal que $x_n\in N$ para todo $n\geq n_0$. Podemos tomar $n_0=1$, por ejemplo.

        \item Sea $(X,d)$ un espacio métrico, $\{x_n\}_{n\in \bb{N}}$ una sucesión en $X$ y $x\in X$. Entonces, $\{x_n\}_{n\in \bb{N}}$ converge a $x$ en $(X,\T_d)$ si y solo si, para todo $\veps>0$, existe $n_0\in \bb{N}$ tal que $d(x,x_n)<\veps$ para todo $n\geq n_0$.

        Tenemos que, dado $x\in X$, una base de entornos de $x$ en $(X,\T_d)$ son las bolas abiertas que contiene a $x$. Es decir, $\beta_x=\{B(x,\veps)\mid \veps\in \bb{R}^+\}$. Entonces, por el apartado b) de este ejercicio, tenemos que:
        \begin{equation*}\begin{split}
            x\in \lim_{n\to \infty}x_n &\Longleftrightarrow
            \forall \veps\in \bb{R}^+,~ \exists n_0\in \bb{N}\mid x_n\in B(x,\veps)~~  \forall n\geq n_0\\
            &\Longleftrightarrow
            \forall \veps\in \bb{R}^+,~ \exists n_0\in \bb{N}\mid d(x,x_n)<\veps~~  \forall n\geq n_0
        \end{split}\end{equation*}

        \item En el espacio topológico $(\bb{R}, \T_{CN})$ prueba que una sucesión $\{x_n\}_{n\in \bb{N}}$ converge a $x\in \bb{R}$ si y solo si existe $n_0\in \bb{N}$ tal que $x_n=x$ para todo $n\geq n_0$.

        \begin{description}
            \item[$\Longrightarrow)$] Supongamos que $x$ es un límite de la sucesión. Entonces, tenemos que $\forall N\in N_x$, $\exists n_0\in \bb{N}$ tal que $x_n\in N$ para todo $n\geq n_0$.

            En concreto, y como $\{U\in \T\mid x\in U\}\subset N_x$ tenemos que $\forall U\in \T$ con $x\in U$, $\exists n_0\in \bb{N}$ tal que $x_n\in U$ para todo $n\geq n_0$.

            Consideramos ahora el conjunto $U=\left(\bb{R}\setminus \{x_n\mid n\in \bb{N}\}\right) \cup \{x\}$. Veamos ahora su complementario:
            \begin{equation*}
                \bb{R}\setminus U = \bb{R}\setminus \left(\bb{R}\setminus \{x_n\mid n\in \bb{N}\}\right) \cap \bb{R}\setminus \{x\} = \{x_n\mid n\in \bb{N}\} \cap \bb{R}\setminus \{x\} = \{x_n\mid n\in \bb{N}\}\setminus \{x\}
            \end{equation*}
            Por tanto, tenemos que $\bb{R}\setminus U$ es numerable, ya que $\bb{R}$ no es numerable pero $U$ sí. Por tanto, $U\in \T_{CF}$. Además, claramente $x\in U$. Por tanto, y aplicando que la sucesión converge a $x$, tenemos que $\exists n_0\in \bb{N}$ tal que $x_m\in U$ para todo $m\geq n_0$.
            
            Como $x_m\in U=\left(\bb{R}\setminus \{x_n\mid n\in \bb{N}\}\right) \cup \{x\}$, tenemos que $x_m=x$ para todo $m\geq n_0$, quedando por tanto demostrado lo pedido.
            
            \item[$\Longleftarrow)$] Como $x\in N$ para todo $N\in N_x$, se tiene de forma directa.
        \end{description}

        Notemos que en la demostración tan solo se usa que $\bb{R}$ no es numerable, por lo que este resultado es cierto para todo $(X,\T_{CF})$ con $X$ no numerable.

        \item \label{ej:Sucesiones.LimiteEnAdherencia}
        Sea $(X,\T)$ un espacio topológico, y $A\subset X$ un subconjunto no vacío. Supongamos que existe $\{a_n\}_{n\in \bb{N}}$ una sucesión de puntos en $A$ que converge a un punto $x\in X$. Entonces $x\in \ol{A}$.

        Como $x$ es un límite, tenemos que para todo $N\in N_x$, $\exists n_0\in \bb{N}$ tal que $a_n\in N$ para todo $n\geq n_0$.

        Por tanto, tenemos que $N\cap A\neq \emptyset$, ya que $\{a_n\mid n\geq n_0\}\subset N\cap A$. Por tanto, tenemos que $x\in \ol{A}$.

        \item Sea $(X,\T)$ y $A\subset X$ un subconjunto no vacío. Si $x\in A^\circ$ entonces para cualquier sucesión de puntos $\{x_n\}_{n\in \bb{N}}$ que converge a $x$, existe $n_0\in \bb{N}$ tal que $x_n\in A$ para $n\geq n_0$.

        Sea una sucesión de puntos $\{x_n\}_{n\in \bb{N}}$ que converge a $x$. Entonces, para todo $N\in N_x$, tenemos que existe $n_0\in \bb{N}$ tal que $x_n\in N$ para $n\geq n_0$.

        Veamos por tanto si $A\in N_x$. Esto solo ocurrirá si $\exists U\in \T$ tal que $x\in U\subset A$, y ese abierto es $U=A^\circ$.

        \item Sea $(X,\T)$ un espacio topológico 1AN, $A\subset X$ un subconjunto no vacío y $x\in \ol{A}$. Entonces, existe una sucesión de puntos de $A$ que converge a $x$. Da un contraejemplo de que esto no tiene por qué ser cierto si $(X,\T)$ no es 1AN.

        Por ser 1AN, tenemos que todo punto, en concreto $x$, tiene una base de entornos numerable. Sea dicha base $\ol{\beta_x}=\{\ol{V}_n\mid n\in \bb{N}\}$.

        Definimos ahora la familia $\beta_x=\{V_n \mid n\in \bb{N}\}$ de la forma $V_n=\bigcap\limits_{k=1}^n \ol{V}_k$. Por el apartado \ref{prop:AxiomasAN_BEIntersec} de la Proposición \ref{prop:AxiomasAN}, tenemos que es una base de entornos en la que $V_i\subset V_j$ para todo $i>j$. Además, es numerable.        

        Además, como $x\in \ol{A}$, tenemos que $A\cap N\neq \emptyset$ para todo $N\in N_x$. En concreto, tenemos que $A\cap V_n\neq \emptyset$ para todo $n\in \bb{N}$.

        Definimos entonces $\{a_n\}_{n\in \bb{N}}$ tal que $a_n\in A\cap V_n$, de forma que tenemos que $a_n\in A$ para todo $n\in \bb{N}$. Tan solo falta por comprobar que converge a $x$.

        Para ello, veamos que $\forall V_n\in \beta_x$, $\exists n_0\in\bb{N}$ tal que $a_m\in V_n$ para todo $m\geq n_0$. Equivalentemente, tenemos que ver que $\forall n\in \bb{N}$, $\exists n_0\in \bb{N}$ tal que $a_m\in V_n$ para todo $m\geq n_0$. Tomamos $n_0=n$, y veamos que para $m\geq n_0$ se tiene que $a_m\in V_n$.
        \begin{itemize}
            \item Si $m=n_0=n$, entonces $a_n\in V_n$ trivialmente, por elección de $a_n$.
            \item Si $m>n_0=n$, entonces $a_m\in V_m$ trivialmente, por elección de $a_n$. Además, como $m>n$, tenemos que $V_m\subset V_n$, por lo que $a_m\in V_n$.
        \end{itemize}
        Es decir, tenemos que $a_m\in V_n$ para todo $m\geq n_0=n$. Es decir, $\{a_n\}_{n\in \bb{N}}$ converge a $x$.\\


        Para el contraejemplo, trabajamos con $(\bb{R},\T_{CN})$, que se ha visto en teoría que no es 1AN por no ser $\bb{R}$ numerable, y consideramos $A=\bb{R}\setminus \bb{Q}$. Veamos que $\ol{A} = \bb{R}$:
        \begin{description}
            \item[$\subset)$] Trivialmente, se tiene que $\ol{A}\subset \bb{R}$.
            \item[$\supset)$] Sea $x\in \bb{R}$. Entonces dado $X$ numerable, $x\notin X$, se tiene:
            \begin{equation*}
                A\cap \bb{R}\setminus X = (\bb{R}\setminus \bb{Q})\cap (\bb{R}\setminus X)
                = \bb{R}\setminus (\bb{Q}\cup X) \neq \emptyset
            \end{equation*}
            donde hemos especificado que no es vacío, ya que $\bb{R}$ es no numerable, y $\bb{Q},X$ sí lo son, y la unión de numerables es numerable. Por tanto, como $x\notin X$ y $X$ es numerable, tenemos que $x\in \bb{R}\setminus X\in \T_{CN}$. Como la intersección con $A$ es no nula, tenemos que $x\in \ol{A}$.
        \end{description}
        
        Consideramos ahora $x=1\in \ol{A}\setminus A$, y supongamos que existen puntos de $A=\bb{R}\setminus \bb{Q}$, $\{a_n\}_{n\in \bb{N}}$ que converge a $x=1$. Por el apartado e), al ser el espacio topológico $(\bb{R},\T_{CN})$, tenemos que esto solo es posible si $\exists n_0\in \bb{N}$ tal que $a_n=x$ para todo $n\neq n_0$. No obstante, $a_n=x=1\notin A$, pero $a_n\in A$, ya que es una sucesión de puntos de $A$. Por tanto, llegamos a una contradicción, y tenemos que dicha sucesión de puntos de $A$ que converge a $x=1$ no existe.

        \item Sea $(X,\T)$ un espacio topológico 1AN, $A\subset X$ un subconjunto no vacío. Supongamos que para cualquier sucesión de puntos $\{x_n\}_{n\in \bb{N}}$ que converge a $x$, existe $n_0\in \bb{N}$ tal que $x_n\in A$ para $n\geq n_0$. Entonces $x\in A^\circ$.

        Demostramos el recíproco. Supongamos que $x\notin A^\circ$, y veamos que existe una sucesión de puntos $\{x_n\}_{n\in \bb{N}}$ que converge a $x$ pero que $\forall p\in \bb{N}$, $\exists q\geq j$ tal que $x_q\notin A$.

        Por ser 1AN, tenemos que todo punto, en concreto $x$, tiene una base de entornos numerable. Sea dicha base $\ol{\beta_x}=\{\ol{V}_n\mid n\in \bb{N}\}$.

        Definimos ahora la familia $\beta_x=\{V_n \mid n\in \bb{N}\}$ de la forma $V_n=\bigcap\limits_{k=1}^n \ol{V}_k$. Por el apartado \ref{prop:AxiomasAN_BEIntersec} de la Proposición \ref{prop:AxiomasAN}, tenemos que es una base de entornos en la que $V_i\subset V_j$ para todo $i>j$. Además, es numerable.

        Habiendo definido entonces $V_n$, veamos que $V_n\setminus A\neq \emptyset$ para todo $n\in \bb{N}$. Para  ello, por reducción al absurdo, supongamos que $V_n\setminus A = V_n\cap (X\setminus A) = \emptyset$. Entonces, $V_n\subset A$, y $V_n\in N_x$, por lo que $A\in N_x$, llegando a una contracción, ya que $x\notin A^\circ$. Definimos cada $x_n\in \bb{X}$ tal que $x_n\in V_n\setminus A$ para todo $n\in \bb{N}$.

        Veamos en primer lugar que $\{x_n\}_{n\in \bb{N}}$ converge a $x$. Para esto, vemos que dado $n\in \bb{N}$, $\exists n_0\in \bb{N}$ tal que $x_m\in V_n$ para $m\geq n_0$. Tomamos $n_0=n$. Entonces:
        \begin{itemize}
            \item Si $m=n=n_0$, tenemos que $x_n\in V_n$ por elección de $x_n$.
            \item Si $m>n=n_0$, tenemos que $x_m\in V_m$. Además, como $m>n$, tenemos que $V_m\subset V_n$, por lo que $x_m\in V_n$.
        \end{itemize}
        Por tanto, hemos probado que $\{x_n\}_{n\in \bb{N}}$ converge a $x$.

        Veamos ahora que $\forall p\in \bb{N}$, $\exists q\geq j$ tal que $x_q\notin A$. De hecho, esto es siempre cierto, ya que por la elección de $x_n$ se tiene que $x_n\notin A$ para todo $n\in \bb{N}$.

        Por tanto, y tras haber demostrado el contrarrecíproco, se tiene lo pedido.
    \end{enumerate}
\end{ejercicio}




\begin{ejercicio}
    Prueba que la recta de Sorgenfrey es un espacio de Haussdorf (T2) y 1AN, pero no es 2AN.

    \begin{enumerate}
        \item Veamos en primer lugar que es T2. Para ello, vemos que, dados $x,y\in \bb{R}$, $x\neq y$, $\exists U,U'\in \T_S$ con $x\in U,~ y\in Y'$ tal que $U\cap U'=\emptyset$.
    
        Supongamos, sin pérdida de generalidad, $x<y$. Entonces, consideramos los abiertos $U=[x,y[$, $U'=[y,y+1[$, y tenemos que $x\in U,~y\in U'$ y $U\cap U'=\emptyset$.
    
        \item Veamos que es 1AN; es decir, que todo punto $x\in \bb{R}$ tiene una base de entornos numerable. Tenemos que $\beta_x=\{[x, x+\veps[~ \mid \veps\in \bb{Q}^+\}$ es numerable. Veamos ahora que, efectivamente, es una base de entornos de $x$.

        En primer lugar, vemos que $\beta_x\subset N_x$. Como $[x,x+\veps[~ \in\T_S$, tenemos que es un entorno. Además, sea $N\in N_x$. Entonces, $N=[x,x+\delta[$, con $\delta\in \bb{R}^+$. Por la densidad de $\bb{Q}$ en $\bb{R}$, $\exists \veps\in \bb{Q}$ tal que $0<\veps<\delta$. Por tanto, se tiene que $x\in [x,x+\veps[~ \subset [x,x+\delta[$.
        
        Por tanto, tenemos que es una base de entornos (numerable por ser $\veps\in\bb{Q}$).

        \item Veamos que no es 2AN. Lo hacemos por contrarrecíproco, suponiendo que sí lo es. Entonces, por la Proposición \ref{prop:2AN_ExtraerBaseNumerable}, tenemos que $\exists \cc{B}'$ base de $\T$ con $\cc{B}'\subset \cc{B}$, y $\cc{B}'$ base numerable. Como $\cc{B}=\{[x,y[\mid x<y, x,y\in \bb{R}\}$ es una base de $\T$, tenemos que $\exists \cc{B}'=\{[x_n,y_n[~ \mid n\in \bb{N}\}\subset \cc{B}$ base de $\T$. 
        
        Consideramos ahora $x\in \bb{R}\setminus \{x_n\mid n\in \bb{N}\}$, que sabemos que no es el vacío ya que $\bb{R}$ no es numerable. Consideramos $U=[x,x+1[$, y por ser $\cc{B}'$ una base, tenemos que $\exists n_0\in \bb{N}$ tal que $x\in [x_{n_0},y_{n_0}[~ \subset [x,x+1[$. Por tanto, tenemos que $x=x_{n_0}$, llegando entonces a un absurdo por la elección de $x$.
    \end{enumerate}
\end{ejercicio}

\begin{ejercicio}
    Sobre $\bb{R}$ consideramos la siguiente familia de subconjuntos:
    \begin{equation*}
        \cc{B}=\{[a,b]\mid a<b,a\in \bb{Q}, b\in \bb{R}\setminus \bb{Q}\}.
    \end{equation*}

    Se pide:
    \begin{enumerate}[label=\alph*)]
        \item Prueba que existe una única topología $\T$ en $\bb{R}$ tal que $\cc{B}$ es una base de $\T$.

        Usamos el Teorema \ref{teo:TopoGenerada_Bases}. Para ello, comprobamos las siguientes condiciones:
        \begin{enumerate}
            \item[B1)] Veamos que $\bigcup\limits_{\substack{a\in \bb{Q}\\b\in \bb{R}\setminus\bb{Q}}} [a,b]=\bb{R}$:
            \begin{description}
                \item[$\subset)$]
                    Sea $x\in \bigcup\limits_{\substack{a\in \bb{Q}\\b\in \bb{R}\setminus\bb{Q}}} [a,b]$. Entonces, $\exists a\in \bb{Q}, b\in \bb{R}\setminus\bb{Q}$ tal que $x\in [a,b]\subset \bb{R}$.

                \item[$\supset)$]
                    Sea $x\in \bb{R}$. Por la densidad de $\bb{Q}$ en $\bb{R}$, tenemos que $\exists a\in \bb{Q}$ tal que $a<x$. Análogamente, por la densidad de $\bb{R}\setminus\bb{Q}$ en $\bb{R}$, tenemos que $\exists b\in \bb{R}\setminus\bb{Q}$ tal que $x<b$. Por tanto, $x\in [a,b]$, por lo que $x\in \bigcup\limits_{\substack{a\in \bb{Q}\\b\in \bb{R}\setminus\bb{Q}}} [a,b]$.
            \end{description}

            
            \item[B2)] Sean $a,a'\in \bb{Q}$, $b,b'\in \bb{R}\setminus\bb{Q}$ de forma que $a<b, a'<b'$. Consideramos también $x\in [a,b]\cap [a',b']$. Veamos que $\exists c\in \bb{Q}, d\in \bb{R}\setminus\bb{Q}$, con $c<d$ tal que $x\in [c,d]\subset [a,b]\cap [a',b']$. Calculamos la intersección:
            \begin{equation*}
                [a,b]\cap [a',b']=[\max\{a,a'\},\max\{b,b'\}]
            \end{equation*}
            Entonces, notando $c=\max\{a,a'\}$, $d=\max\{b,b'\}$, tenemos que
            \begin{equation*}
                x\in [c',d']\subset [c',d']=[a,b]\cap [a',b']
            \end{equation*}

            Además, $[c', d']\in \cc{B}$, ya que el máximo de dos racionales es racional (igual con irracionales).

            Por tanto, se tiene.
        \end{enumerate}

        \item Calcula una base de entornos de $x\in \bb{R}$ en $\T$ no trivial.

        Supongamos $x$ racional. Entonces, una base de entornos suya es:
        \begin{equation*}
            \beta_x=\left\{\left[x,x+\frac{\pi}{n}\right]~~ \mid n\in \bb{N}\right\}
        \end{equation*}

        Supongamos $x$ irracional. Entonces, una base de entornos suya es:
        \begin{equation*}
            \beta_x=\left\{\left[a,x\right]~~ \mid a\in \bb{Q},~a<x\right\}
        \end{equation*}

        \item Prueba que $\T_u\subsetneq \T$, donde $\T_u$ es la topología usual en $\bb{R}$. ¿Es $(\bb{R},\T)$ un espacio topológico $T2$?


        Veamos en primer lugar que $\T_u\subset \T$. Usando la caracterización de inclusión con las bases topológicas, tenemos que equivale a ver que, dado $]a,b[~\in \T_u$, para todo $x\in~ ]a,b[~ (a<x<b)$ se tiene que $\exists [c,d]\in \T$ tal que $x\in [c,d]\subset ]a,b[$.

        Por la densidad de $\bb{Q}$ en $\bb{R}$, tenemos que $\exists c\in \bb{Q}$ tal que $a<c<x$. Por la densidad de $\bb{R}\setminus\bb{Q}$ en $\bb{R}$, se tiene que $\exists d\in \bb{R}\setminus\bb{Q}$ tal que $x<d<b$.
        
        Por tanto, se que $\T_u\subset \T$. No obstante, la otra inclusión no es cierta, ya que $[0,\pi]\notin \T_u$.


        Por último, falta por ver si $(\bb{R},\T)$ es T2. Para esto, vemos si dados $x,y\in \bb{R}$, $x\neq y$, se tiene que $\exists U,U'\in \T$ con $x\in U,~ y\in U'$ con $U\cap U'=\emptyset$.

        Como $\T_u$ es T2, tenemos que $\exists U,U'\in \T_u\subset \T$ tal que $x\in U,~ y\in U'$ con $U\cap U'=\emptyset$, por lo que se tiene.
        

        \item Calcula la clausura, el interior y la frontera de los conjuntos $[0,1[, [0,\sqrt{2}]$ y $\bb{Q}$.

        Veamos en primer lugar que $[0,1[$ es cerrado. Tenemos que su complementario es $\bb{R}\setminus [0,1[~=]-\infty, 0[~\cup~[1,+\infty[$. El primer intervalo es un abierto en $\T_u\subset \T$. Respecto al segundo, tenemos que:
        \begin{equation*}
            [1,+\infty[~=\bigcup_{\substack{b\in \bb{R}\setminus\bb{Q}\\b>1}}[1,b]\in \T
        \end{equation*}
        Por tanto, tenemos que el segundo intervalo es una unión (no numerable) de abiertos, por lo que es un abierto. Por tanto, $\bb{R}\setminus [0,1[~\in \T$, por lo que $[0,1[$ es un cerrado y, por tanto, $\ol{[0,1[}=[0,1[$.
        

        Además, tenemos que $[0,1[~=\bigcup\limits_{\substack{b\in \bb{R}\setminus\bb{Q}\\0<b<1}}[0,b]\in \T$, por lo que es un abierto y, por tanto, $[0,1[^\circ = [0,1[$.\\


        Trabajamos ahora con $[0,\sqrt{2}]$. Como $\T_u\subset \T$, tenemos que $C_{\T_u}\subset C_{\T}$. Por tanto, tenemos que $[0,\sqrt{2}]\in C_\T$, por lo que $\ol{[0,\sqrt{2}]}=[0,\sqrt{2}]$.

        Además, tenemos que es un abierto básico, por lo que $[0,\sqrt{2}]^\circ = [0,\sqrt{2}]$.\\

        Veamos ahora la clausura de $\bb{Q}$. Como los abiertos básicos son intervalos, tenemos que $\bb{Q}\cap B\neq \emptyset$, para todo $B\in \cc{B}$ intervalo y por la densidad de $\bb{Q}$ en $\bb{R}$. Por tanto, como esto es cierto $\forall x\in X$, tenemos que $\ol{Q}=\bb{R}$.

        De igual forma, por la densidad de $\bb{R}\setminus\bb{Q}$ en $\bb{R}$, tenemos que $\nexists B\in \cc{B}$ tal que $B\subset \bb{Q}$. Por tanto, $[\bb{Q}]^\circ = \emptyset$.

        \item Prueba que $\bb{Z}$ es un subconjunto discreto de $(\bb{R},\T)$.

        Para ello, hemos de ver que $\T_{\big| \bb{Z}}$ es la topología discreta. Sea $z\in \bb{Z}$, y veamos si $\{z\}\in \T_{\big| \bb{Z}}$. Tenemos que $\{z\}=\left[z,z+\frac{\sqrt{2}}{2}\right]\cap \bb{Z}$, ya que $\frac{\sqrt{2}}{2}<1$. Además, como $\frac{\sqrt{2}}{2}\in \bb{R}\setminus\bb{Q}$, tenemos que al sumarle $z$ sigue siendo irracional y, por tanto, $\left[z,z+\frac{\sqrt{2}}{2}\right]\in \T$.

        Por tanto, por definición de topología inducida tenemos que $\{z\}\in \T_{\big | \bb{Z}}$, y como la unión de abiertos es un abierto, tenemos que $\T_{\big | \bb{Z}}=\cc{P}(\bb{Z})={\T_{disc}}_{\big| \bb{Z}}$.

        \item Estudia si $(\bb{R}, \T)$ es un espacio topológico 1AN o 2AN.

        En el apartado b) de este ejercicio hemos demostrado que todo punto $x\in \bb{R}$ tiene una base de entornos numerable. Por tanto, $\T$ es 1AN. Veamos ahora si es o no 2AN.

        Supongamos que sí lo es, y llegaremos a una contradicción. Por la Proposición \ref{prop:2AN_ExtraerBaseNumerable}, tenemos que $\exists \cc{B}'$ base numerable de $\T$ tal que $\cc{B}'\subset \cc{B}$. Es decir, $\cc{B}'=\{[a_n,b_n]\mid n\in \bb{N},~[a_n,b_n]\in \T\}$ es una base de $\T$.

        Sea ahora $b\in \cc{R}\setminus \bb{Q}$ un irracional, y consideramos $a\in \bb{Q}$, $a<b$. Entonces, $[a,b]\in \T$ por ser un abierto básico de $\cc{B}$. Como $\cc{B}'$ es una base, tenemos que $\exists n\in \bb{N}$ tal que $b\in [a_n, b_n]\subset [a,b]$, por lo que $b=b_n$. Por tanto, hemos llegado a que $\forall b\in \bb{R}\setminus \bb{Q}$, $\exists n\in \bb{N}$ tal que $b=b_n$; es decir, a que $\bb{R}\setminus \bb{Q}$ es numerable, llegando a un claro absurdo.

        Por tanto, tenemos que $(\bb{R}, \T)$ no es 2AN.
    \end{enumerate}
\end{ejercicio}
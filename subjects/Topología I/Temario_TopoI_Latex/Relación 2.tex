\section{Aplicaciones entre Espacios Topológicos}\label{sec:Rel2}

\begin{ejercicio}\label{ej:Tema2.1}
    Sean $(X, d)$ e $(Y, d')$ espacios métricos. Diremos que una aplicación $f : (X, d) \to (Y, d')$ es lipschitziana si existe $K > 0$ tal que $d'(f(x), f(y)) \leq Kd(x, y)$ para todo $x, y \in X$. Prueba que toda aplicación lipschitziana es una aplicación continua.\\

    Sea $x\in X$, y veamos si $f$ es continua en dicho punto. Para cualquier $\veps\in \bb{R}^+$, sea $\delta=\frac{\veps}{K}$. Entonces, si $d(x,x_0)<\delta$ se tiene que:
    \begin{equation*}
        d'(f(x), f(x_0)) \leq Kd(x,y) < k\cdot \frac{\veps}{k} < \veps
    \end{equation*}
    Es decir, $f[B(x,x_0)]\subset B(f(x), \veps)$. Por tanto, $f$ es continua.
\end{ejercicio}

\begin{ejercicio}
    Sean $(X, d)$ un espacio métrico y $A \subset X$. Demuestra que la aplicación $f : (X, \T_d) \to (\bb{R},\T_u)$ dada por $f(x) = d(x, A) = \inf\{d(x, a) \mid a \in A\}$ es continua.\\

    Sea $x,y\in X$, y veamos que $f$ es lipschitziana en dicho punto. Para todo $a\in A$:
    \begin{equation*}
        f(x) = d(x,A) \leq d(x,a) \leq d(x,y) + d(y,a) \Longrightarrow f(x) - d(x,y) \leq d(y,a)
    \end{equation*}
    Por tanto, como se tiene $\forall a\in A$, se tiene que $f(x)-d(x,y)\leq f(y)$ o, equivalentemente,
    \begin{equation*}
        f(x)-f(y) \leq d(x,y) \qquad \forall x,y\in X
    \end{equation*}
    
    De forma análoga, se tiene que $f(y)-f(x) = -[f(x)-f(y)] \leq d(x,y) ~ \forall x,y\in X$. Por tanto, tenemos que:
    \begin{equation*}
        d(f(x),f(y)) = |f(x)-f(y)| \leq d(x,y)
    \end{equation*}
    
    Por tanto, tenemos que $f$ es lipschitziana con constante $K=1$. En particular, es continua.
\end{ejercicio}

\begin{ejercicio}
    Sean $(X, \T)$ un espacio topológico y $f, g:(X, \T) \to (\bb{R},\T_u)$ aplicaciones continuas. Demostrar que las siguientes aplicaciones son continuas:
    \Func{f+g}{(X,\T)}{(\bb{R},\T_u)}{x}{f(x)+g(x)}
    \Func{f\cdot g}{(X,\T)}{(\bb{R},\T_u)}{x}{f(x)\cdot g(x)}

    Escribiremos $f+g,f\cdot g$ como una composición. Sean las siguientes aplicaciones entre espacios topológicos:
    \Func{F}{(X,\T)}{\bb{R}^2,\T_u}{x}{(f(x),g(x))}
    \begin{figure}[H]
        \centering
        \begin{subfigure}{0.35\linewidth}
            \centering
            \Func{s}{(\bb{R}^2,\T_u)}{\bb{R},\T_u}{(x,y)}{x+y}
        \end{subfigure}\hspace{1cm}
        \begin{subfigure}{0.35\linewidth}
            \centering
            \Func{p}{(\bb{R}^2,\T_u)}{\bb{R},\T_u}{(x,y)}{xy}
        \end{subfigure}
    \end{figure}

    Tenemos que, para todo $x\in X$:
    \begin{gather*}
        (f+g)(x):=f(x)+g(x)=s(f(x),g(x))=s(F(x))=(s\circ F)(x)\\
        (fg)(x):=f(x)g(x)=p(f(x),g(x))=(F(x))=(p\circ F)(x)
    \end{gather*}
    Por tanto, se tiene que $f+g=s\circ F,~fg=p\circ F$. Sabemos que $s,p$ son continuas por ser polinómicas, veamos ahora que $F$ lo es.
    
    \begin{description}
        \item[Opción 1.] Usando la caracterización de la continuidad mediante bases de la topología.

        En $(\bb{R}^2,\T_u)$, consideramos la base $\cc{B}=\{]a,b[\times ]c,d[~ \mid a<b,~ c<d\}$. Entonces:
        \begin{equation*}
            \begin{split}
                F^{-1}(]a,b[\times ]c,d[)
                &= \left\{
                x\in X\mid F(x)\in~  ]a,b[\times]c,d[
                \right\}
                =\\&= \left\{
                x\in X\mid f(x)\in ]a,b[ \quad \land \quad g(x)\in ]c,d[
                \right\}
                =\\&= f^{-1}(]a,b[)~ \cap g^{-1}(]c,d[)\in \T
            \end{split}
        \end{equation*}
        donde tenemos que es un abierto por ser la intersección de dos abiertos, y estos dos lo son por ser $f,g$ continuas.

        \item[Opción 2.] Usando la topología producto.

        Como ambas componentes de $F$ son continuas, $F$ también lo es.
    \end{description}
    Por tanto, como $p,s,F$ son continuas, entonces $f+g,fg$ son continuas.
\end{ejercicio}


\begin{ejercicio}
    Sea $f : (X, \T ) \to (Y, \T')$ una aplicación. Demuestra que equivalen:
\begin{enumerate}
    \item $f$ es continua.
    \item $f^{-1}(B^\circ) \subset [f^{-1}(B)]^\circ, \forall B \subset Y$.
    \item $\partial (f^{-1}(B)) \subset f^{-1}(\partial B), \forall B \subset Y$.
\end{enumerate}
\begin{description}
    \item[$1 \Longrightarrow 2)$]

    Como $f$ es continua, y $B^\circ \in \T'$, entonces $f^{-1}(B^\circ)\in \T$. Además, como $B^\circ \subset B$, entonces $f^{-1}(B^\circ)\subset f^{-1}(B)$. Como el interior es el mayor abierto contenido en el conjunto, entonces:
    \begin{equation*}
        f^{-1}(B^\circ) \subset [f^{-1}(B)]^\circ
    \end{equation*}

    \item[$2 \Longrightarrow 1)$] Sea $U\in \T'$, por lo que $U=U^\circ$. Mediante doble inclusión, demostraremos que $f^{-1}(U)=[f^{-1}(U)]^\circ$:
    \begin{description}
        \item[$\subset)$] Como $U=U^\circ$, usando 2):
        \begin{equation*}
            f^{-1}(U^\circ) = f^{-1}(U) \subset [f^{-1}(U)]^\circ
        \end{equation*}

        \item[$\supset)$] De forma directa se tiene que $[f^{-1}(U)]^\circ \subset f^{-1}(U)$.
    \end{description}

    \item[$1\Longrightarrow 3)$] Tenemos que $\partial (f^{-1}(B)) = \ol{f^{-1}(B)}\setminus [f^{-1}(B)]\circ$.

    Veamos en primer lugar que $\ol{f^{-1}(B)}\subset f^{-1}\left(\ol{B}\right)$. Como $B\subset \ol{B}$, entonces $f^{-1}(B)\subset f^{-1}\left(\ol{B}\right)$, por lo que:
    \begin{equation*}
        \ol{f^{-1}(B)}\subset \ol{f^{-1}\left(\ol{B}\right)} \AstIg f^{-1}\left(\ol{B}\right)
    \end{equation*}
    donde en $(\ast)$ he empleado que, como $\ol{B}\in C_{\T'}$, entonces $f^{-1}\left(\ol{B}\right)\in C_\T$.

    Veamos ahora que $f^{-1}(B^\circ)\subset [f^{-1}\left(B\right)]^\circ$. Como $B^\circ \subset B$, entonces se tiene que $f^{-1}(B^\circ)\subset f^{-1}\left(B\right)$, por lo que:
    \begin{equation*}
        f^{-1}(B^\circ) \AstIg [f^{-1}(B^\circ)]^\circ \subset [f^{-1}(B)]^\circ
    \end{equation*}
    donde en $(\ast)$ he empleado que, como $B^\circ \in {\T'}$, entonces $f^{-1}\left({B}^\circ \right)\in \T$.

    Por tanto, se tiene que:
    \begin{equation*}
        \partial (f^{-1}(B)) = \ol{f^{-1}(B)}\setminus [f^{-1}(B)]^\circ \subset f^{-1}\left(\ol{B}\right)\setminus f^{-1}(B^\circ) = f^{-1}(\ol{B}\setminus B^\circ) = f^{-1}(\partial B)
    \end{equation*}

    \item[$3 \Longrightarrow 1)$] Sea $C'\in C_{\T'}$, y veremos que $f^{-1}(C')\in C_{\T}$. Para ello, conocemos la siguiente caracterización de los cerrados:
    \begin{equation*}
        C'\in \T' \Longleftrightarrow \partial C'\subset C'
    \end{equation*}

    Por tanto, tenemos que $\partial C'\subset C'$, por lo que $f^{-1}(\partial C')\subset f^{-1}(C')$. Además, por hipótesis, se tiene que:
    \begin{equation*}
        \partial (f^{-1}(C')) \subset f^{-1}(\partial C') \subset f^{-1}(C')
    \end{equation*}
    Por tanto, como $\partial (f^{-1}(C'))\subset f^{-1}(C')$, entonces $f^{-1}(C')\in C_{\T}$, por lo que $f$ es continua.
\end{description}
\end{ejercicio}

\begin{ejercicio}
    Sean $(X, \T ), (Y, \T')$, dos espacios topológicos, $f : (X, \T ) \to (Y, \T')$ una aplicación continua y sobreyectiva. Demuestra que si $D \subset X$ es un subconjunto denso, entonces $f(D)$ es denso en $Y$. Demuestra, mediante un contraejemplo, que si $f(D)$ es denso, $D$ no tiene por qué serlo.\\

    Demostremos que $f(D)$ es denso. Para ello, demostramos que $\ol{f(D)}=Y$.
    \begin{description}
        \item[$\subset)$] Tenemos que $f(D)\subset Y$. Por tanto, $\ol{f(D)}\subset \ol{Y}=Y$, ya que $Y\in C_{\T'}$.

        \item[$\supset)$] Por la caracterización de continuidad, por ser $f$ continua tenemos que $f\left(\ol{D}\right)\subset \ol{f(D)}$. Como $X$ es denso ($X=\ol{D}$), tenemos que $f(X)=f(\ol{D})$. Además, tenemos por ser $f$ sobreyectiva tenemos que $f(X)=Y$. Por tanto,
        \begin{equation*}
            Y=f(X)=f\left(\ol{D}\right)\subset \ol{f(D)}
        \end{equation*}
    \end{description}
    Por tanto, mediante doble inclusión hemos demostrado que $f(D)$ es denso.  
    
    Para ver que el recíproco no es cierto, sea $f:(X,\T_{disc})\to (Y, \T_t)$, considerando $X=\{0,1\}$ e $Y=\{y_0\}$, la aplicación constante en $y_0\in \bb{R}$. Tenemos que $f$ es continua y sobreyectiva. Tenemos que $f(\{0\})=\{y_0\}=Y$, por lo que $f(\{0\})$ es denso. No obstante, $\ol{\{0\}}=\{0\}$, por lo que $\{0\}$ no es denso.
\end{ejercicio}

\begin{ejercicio}
    Sean $(X, \T ), (Y, \T')$, dos espacios topológicos y una aplicación $f : (X, \T ) \to (Y, \T')$.
    \begin{enumerate}
        \item Demuestra que si $f$ es continua y $\{x_n\}_{n\in \bb{N}}$ es una sucesión en $X$ que converge a $x_0$ entonces $\{f(x_n)\}_{n\in \bb{N}}$ es una sucesión en $Y$ que converge a $f(x_0)$.
    
        \item \label{ej:3.2.6.2}  Demuestra que si $(X, \T )$ es un espacio topológico 1AN tal que para toda sucesión
        $\{x_n\}_{n\in \bb{N}}$ que converge a $x_0$ se tiene que $\{f(x_n)\}_{n\in \bb{N}}$ es una sucesión que converge a $f(x_0)$, entonces $f$ es continua.
    
        \item Demuestra que \ref{ej:3.2.6.2}) no es cierto en general si se elimina la condición 1AN.
    \end{enumerate}
\end{ejercicio}

\begin{ejercicio}
    Se considera en $\bb{N}$ la topología $\T$ del ejercicio \ref{ej:3.1.10} de la Relación 1. Caracteriza las aplicaciones continuas de $(\bb{N},\T)$ en sí mismo.
\end{ejercicio}

\begin{ejercicio}
    Sean $(X, \T ), (Y, \T')$, dos espacios topológicos, $f : (X, \T ) \to (Y, \T')$ una aplicación. Si $A \subset X$, entonces $f_{\big| A}$ puede ser continua sin que $f$ sea continua en los puntos de $A$.

    Esto es cierto, y ejemplo de esto es la función característica de $\bb{Q}$:
    \Func{f=\chi_{\bb{Q}}}{(\bb{R},\T_u)}{(\{0,1\},\T_{disc})}{x}{\left\{\begin{array}{ccl}
        1 & \text{si} & x\in \bb{Q} \\
        0 & \text{si} & x\notin \bb{Q}
    \end{array}\right.}

    Tenemos que $f_{\big| \bb{Q}}$ es constante en 1, por lo que es continua. No obstante, $f$ no es continua en los puntos de $\bb{Q}$. Veámoslo.
    
    Consideramos $x\in \bb{Q}$. Tenemos que $f(x)=1$, y $\{1\}\in N'_1$ por ser la topología discreta. No obstante, $f^{-1}(\{1\})=\bb{Q}\notin N_x$. Por tanto, hemos encontrado un entorno de $f(x)$ cuya preimagen no es un entorno de $x$, por lo que $f$ no es continua en $x$.
\end{ejercicio}

\begin{ejercicio}[Carácter local de la continuidad]
    Sean $(X, \T ), (Y, \T')$, dos espacios topológicos y una aplicación $f : (X, \T ) \to (Y, \T')$.
    Demuestra que $f$ es continua en $x_0$ si y solo si existe $U \in \T$ con $x_0 \in U$ tal que $f_{\big|U} : (U, \T_U ) \to (Y, \T')$ es continua en $x_0$. ¿Es cierta la equivalencia anterior si sustituimos $U$ abierto conteniendo a $x_0$ por $C$ cerrado conteniendo a $x_0$?

    \begin{description}
        \item[$\Longrightarrow)$] Sea $U=X\in \T$. Tenemos que $x_0\in X$, y $f_{\big| U}=f$, por lo que se tiene de forma directa.

        \item[$\Longleftarrow)$] Como $x_0\in U$, entonces podemos considerar $f_{\big| U}(x_0)=f(x_0)$.
        
        Sea $N'\in N_{f_{| U}(x_0)}=N_{f(x_0)}$. Entonces, por ser $f^{-1}_{\big | U}$ continua en $x_0$, se tiene que $f^{-1}_{\big | U}(N')\in N_{x_0}$ para $\T_{\big | U}$; es decir, $f^{-1}(N')\cap U \in N_{x_0}$ para $\T_{\big | U}$.

        Entonces, por ser un entorno, tenemos que $\exists O\in \T_{\big| U}$ tal que se tiene que $x_0\in O\subset f^{-1}(N')\cap U$, por lo que $\exists O'\in \T$ tal que $x_0\in O'\cap U \subset f^{-1}(N')\cap U$. Como $f^{-1}(N')\cap U\subset f^{-1}(N')$, se tiene que:        \begin{equation*}
            \exists O'\in \T \text{ tal que } x_0\in O'\cap U \subset f^{-1}(N')
        \end{equation*}
        Como la intersección de dos abiertos es un abierto, tenemos que $O'\cap U\in \T$, por lo que $f^{-1}(N')\in N_{x_0}$, por lo que $f$ es continua en $x_0$.\\

        Como un único punto es un cerrado, tenemos que al sustituir $U$ por un cerrado no es cierto. Sea $C=\{0\}$, y la función definida por:
        \Func{f}{\bb{R}}{\bb{R}}{x}{\left\{\begin{array}{lll}
            1 & \text{si} & x\neq 0\\
            0 & \text{si} & x= 0\\
        \end{array}\right.}

        Tenemos que $f_{\big| C}$ es constante en el $0$, por lo que es continua en el $0\in C$. No obstante, $f$ no es continua en el $0$, ya que:
        \begin{equation*}
            f^{-1}\left(\left]\frac{-1}{2},\frac{1}{2}\right[\right) = \{0\}\notin N_{0} \text{para la $\T_u$.}
        \end{equation*}
    \end{description}
\end{ejercicio}

\begin{ejercicio}
    Demuestra que una aplicación $f : (X, \T_{x_0}) \to (Y, \T_{y_0})$ es continua si y solo si es constante o $f(x_0) = y_0$. Deduce que $(X, \T_{x_0}) \cong (X, \T_{x_1})$ para todo par de puntos $x_0, x_1 \in X$.

    \begin{description}
        \item[$\Longrightarrow)$] Supongamos que $f$ es continua, y sea un abierto básico $\{y,y_0\}\in \T_{y_0}$. Entonces, $f^{-1}(\{y,y_0\})\in \T_{x_0}$, por lo que hay dos opciones:
        \begin{equation*}
            x_0\in f^{-1}(\{y,y_0\}) = f^{-1}(y) \cup f^{-1}(y_0) \Longrightarrow
            \left\{
            \begin{array}{l}
                f(x_0) = y \\
                \qquad \lor \\
                f(x_0)=y_0
            \end{array}
            \right.
        \end{equation*}
        \begin{equation*}
            \emptyset = f^{-1}(\{y,y_0\})  \Longrightarrow
            y,y_0\notin Im(f)
        \end{equation*}

        Es decir, tenemos que $f(x_0)=y_0$ o, para todo $y\in Y$, $f(x_0)=y$ o $y\notin f(X)$. Es decir, $f(x_0)=y_0$ o $Im(f)=\{f(x_0)\}$. Como $f$ es una aplicación, $f(x_0)$ es único, por lo que $\|Im(f)\|=1$.

        Por tanto, $f(x_0)=y_0$ o $f$ es constante en $f(x_0)$.

        \item[$\Longleftarrow)$] Supongamos $f(x_0)=y_0$, y sea $\{y,y_0\}\in \T_{y_0}$. Entonces,
        \begin{equation*}
            x_0=f^{-1}(y_0)\in f^{-1}(\{y,y_0\}) \Longrightarrow f^{-1}(\{y,y_0\})\in \T_{x_0}
        \end{equation*}

        Supongamos ahora que $f$ es constante en $k\neq y_0$.
        \begin{enumerate}
            \item Dado $y\neq k$, consideramos el abierto $\{y,y_0\}\in \T_{y_0}$. Entonces,
            \begin{equation*}
                \emptyset = f^{-1}(\{y,y_0\}) \Longrightarrow f^{-1}(\{y,y_0\})\in \T_{x_0}
            \end{equation*}
            \item Para $y=k$, tenemos que $f(x_0)=k=y$, por lo que:
            \begin{equation*}
                x_0=f^{-1}(y)\in f^{-1}(\{y,y_0\}) \Longrightarrow f^{-1}(\{y,y_0\})\in \T_{x_0}
            \end{equation*}
        \end{enumerate}
    \end{description}

    Veamos ahora que $(X, \T_{x_0}) \cong (X, \T_{x_1})$ para todo par de puntos $x_0, x_1 \in X$. Sea el homeomorfismo siguiente:
    \Func{f}{(X, \T_{x_0})}{(X, \T_{x_1})}{x}{x-x_0+x_1}

    Claramente tenemos que es continua, ya que $f(x_0)=y_0$. Además, $f^{-1}$ viene dada por:
    \Func{f^{-1}}{(X, \T_{x_1})}{(X, \T_{x_0})}{y}{y+x_0-x_1}
    Como $f^{-1}(x_1)=x_0$, también tenemos que es continua.
    Por tanto, $f$ es un homeomorfismo y ambos conjuntos son homeomorfos, como queríamos demostrar.
\end{ejercicio}

\begin{ejercicio}
    Demuestra que todo subespacio afín $S \subset \bb{R}^n$ es un cerrado de $(\bb{R}^n, \T_u)$.
    
    Si $S$ es un subespacio afín de $\bb{R}^n$, entonces $S$ viene dado por unas ecuaciones implícitas en las coordenadas usuales de $\bb{R}^n$:
    \begin{equation*}
        \left\{
            \begin{array}{ccc}
                 a_{11}x_1 + \dots + a_{1n}x_n &=& b_1 \\
                 a_{r1}x_1 + \dots + a_{rn}x_n &=& b_r
            \end{array}
        \right.
    \end{equation*}

    Definimos las siguientes aplicaciones para $i=1,\dots,r$:
    \Func{f_i}{\qquad (\bb{R}^n, \T_u)}{(\bb{R},\T_u)}{(x_i,\dots,x_n)}{a_{i1}x_1+\dots+a_{in}x_n}

    Tenemos que $S=\bigcap\limits_{i=1,\dots,r}f^{-1}(b_r)$ es cerrado en $(\bb{R}^n,\T_u)$, ya que cada función es continua y, como $\{b_r\}$ es un cerrado, entonces su imagen inversa también.
\end{ejercicio}

\begin{ejercicio}
    Consideremos el espacio $(X, \T )$ donde $X = \{a, b, c, d\}$ y
    \begin{equation*}
        \T = \{\emptyset, X, \{a\}, \{b\}, \{a, b\}, \{b, c, d\}\}.
    \end{equation*}
    Sea $f : (X, \T ) \to (X, \T )$ la aplicación dada por: $$f(a) = b\qquad f(b) = d\qquad f(c) = b\qquad f(d) = c$$
    Estudia en qué puntos la aplicación $f$ es continua. ¿Es $f$ abierta o cerrada?\\

    
    Estudiamos en primer lugar en qué puntos es continua:
    \begin{itemize}
        \item $x_0=a$:
        
        Tenemos que $f(x_0)=b$, por lo que buscamos
        demostrar que $\forall N'\in N_b,~ f^{-1}(N')\in N_{a}$. Sea $U'=\{b\}$, y tenemos que
        $U'\subset N'$, por lo que $f^{-1}(U')=\{a,c\}\subset f^{-1}(N')$.
        Como $f^(U')\in N_{a}$, entonces $f^{-1}(N')\in N_{a}$, por lo que $f$ es continua en $a$.

        \item $x_0=b$:
        
        Tenemos que $f(x_0)=d$, por lo que buscamos
        demostrar que $\forall N'\in N_d,~ f^{-1}(N')\in N_{b}$. Tenemos que:
        \begin{equation*}
            N_d = \{\{b,c,d\}, X\}
        \end{equation*}

        Sea $U'=\{b,c,d\}$, y tenemos que $U'\subset N'$, por lo que $f^{-1}(U')=X\subset f^{-1}(N')$.
        Como $f^(U')\in N_{b}$, entonces $f^{-1}(N')\in N_{b}$, por lo que $f$ es continua en $b$.

        \item $x_0=c$:
        
        Tenemos que $f(x_0)=b$, por lo que buscamos
        demostrar que $\forall N'\in N_b,~ f^{-1}(N')\in N_{c}$. Sea $N'=\{b\}\in N_b$, y tenemos que
        $f^{-1}(\{b\})=\{a,c\}$. Como $\nexists U\in \T \mid c\in U\subset f^{-1}(N')$, entonces $f^{-1}(N')\notin N_{c}$, por lo que $f$ no es continua en $c$.

        \item $x_0=d$:
        
        Tenemos que $f(x_0)=c$, por lo que buscamos
        demostrar que $\forall N'\in N_c,~ f^{-1}(N')\in N_{d}$. Tenemos que:
        \begin{equation*}
            N_c = \{\{b,c,d\}, X\}
        \end{equation*}

        Sea $U'=\{b,c,d\}$, y tenemos que $U'\subset N'$, por lo que $f^{-1}(U')=X\subset f^{-1}(N')$.
        Como $f^(U')\in N_{d}$, entonces $f^{-1}(N')\in N_{d}$, por lo que $f$ es continua en $d$.
    \end{itemize}
    
    Veamos en primer lugar que no es continua. Dado $\{b\}\in \T$, tenemos que la imagen inversa del abierto $\{b\}$ no es un abierto, es decir, $f^{-1}\{b\}=\{a,c\}\notin \T$. Por tanto, como hemos encontrado un abierto cuya preimagen no es un abierto, tenemos que $f$ no es continua.

    Veamos ahora que $f$ no es abierta. Dado $\{b\}\in \T$, tenemos que $f\{b\}=\{d\}\notin \T$. Por tanto, como hemos encontrado un abierto cuya imagen no es un abierto, tenemos que $f$ no es abierta.

    Tenemos que los cerrados son:
    \begin{equation*}
        C_\T=\{\emptyset, X, \{b,c,d\}, \{a,c,d\}, \{c,d\}, \{a\}\}
    \end{equation*}
    Veamos que $f$ no es cerrada. Dado $\{a\}\in C_\T$, tenemos que $f\{a\}=\{b\}\notin C_\T$. Por tanto, como hemos encontrado un cerrado cuya imagen no es un cerrado, tenemos que $f$ no es cerrada.
\end{ejercicio}


\begin{ejercicio}
    Se considera $f : (\bb{R},\T) \to (\bb{R},\T)$ dada por $f(x) = \sen(x)$, siendo $(\bb{R},\T)$ la recta diseminada (Ejercicio \ref{ej:3.1.16} de la Relación 1). Estudia si $f$ es continua, abierta o cerrada.

    Veamos si $f$ es continua. Tomamos como abierto en la recta diseminada el conjunto $W=\{\sen 1\}\subset \bb{R}\setminus \bb{Q}$. Tenemos que:
    \begin{equation*}
        f^{-1}(W)=\{1+2\pi k\mid k\in \bb{Z}\} \cup \{(\pi-1)+2\pi k\mid k\in \bb{Z}\}
    \end{equation*}
    Tenemos que $f^{-1}(W)\notin \T$, ya que si fuese un abierto, entonces $1\in f^{-1}(W)=U\cup V$, con $U\in \T_u$, $V\subset \bb{R}\setminus \bb{Q}$. Como $1\in \bb{Q}$, tenemos que ha de ser que $1\in U\in \T_u$. Por la definición de la topología usual, $\exists \veps\in \bb{R}^+\mid 1\in B(1,\veps)\subset U\subset f^{-1}(W)$. No obstante, esto no es posible, ya que $f^{-1}$ es discreto.\\

    Veamos que no es abierta. Tenemos que $W=\left\{\dfrac{\pi}{2}\right\}\in \T$, pero $f(W)=\{1\}\notin \T$.\\

    Veamos que no es cerrada. Para ello, usamos el conjunto $C=\left[0,\frac{3\pi}{2}\right[$. Tenemos que:
    \begin{equation*}
        X\setminus U = ]-\infty, 0[~\cup~\left]\frac{3\pi}{2},+\infty\right[~\cup~ \left\{\frac{3\pi}{2}\right\} \in \T
    \end{equation*}

    No obstante, $f(C) = ]-1,1]$, que no es un cerrado, ya que:
    \begin{equation*}
        X\setminus f(C) = ]-\infty, -1]\cup ]1,+\infty[
    \end{equation*}
    El $-1$ es racional, pero $\nexists B(x,\veps)$ tal que $1\in B(x,\veps)\subset X\setminus f(C)$.
\end{ejercicio}

\begin{ejercicio}
    Sea $f=\chi_{\left[0,\frac{1}{2}\right]}: ([0, 1],(\T_u)_{[0,1]}) \to (\{0, 1\}, \T_{disc})$ la función característica del intervalo $\left[0,\frac{1}{2}\right]$.
    Demuestra que $\chi_{\left[0,\frac{1}{2}\right]}$ es sobreyectiva, abierta, cerrada, pero no es continua.\\

    Claramente es sobreyectiva, ya para todo $y\in \{0,1\}$, $\exists x\in [0,1]$ tal que $f(x)=y$. Por ejemplo, $f(0)=1$, $f(1)=0$.

    Además, es abierta y cerrada, ya que la topología de destino es la topología discreta. No obstante, no es continua, ya que:
    \begin{equation*}
        f^{-1}(\{1\}) = \left[0,\frac{1}{2}\right]\notin \T_u)_{[0,1]}
    \end{equation*}
    Por tanto, como hay un abierto cuya imagen inversa no es un abierto, $f$ no es continua.
\end{ejercicio}


\begin{ejercicio}
    Demuestra que las proyecciones $p_i: (\bb{R}^n, \T_u) \to (\bb{R}, \T_u)$ dadas por $p_i(x_1, \dots , x_n) = x_i$, $i = 1, \dots, n$, son aplicaciones abiertas pero no cerradas.
\end{ejercicio}

\begin{ejercicio}
    Demuestra que la aplicación $f : (\bb{R}^n, \T_u) \to ([0, +\infty[,{\T_u}_{[0,+\infty[})$ dada por $f(x) = \|x\|$ es abierta, y que $g : (\bb{R}^n, \T_u) \to (\bb{R}, \T_u)$ dada por $g(x) = \|x\|$ no lo es.
\end{ejercicio}

\begin{ejercicio}
    Sea $A \subset \bb{R}^n$ con $A \in C_{\T_u}$ y $f : \left(A,{(\T_u)}_{\big|A}\right) \to (\bb{R}^m, \T_u)$ continua, y tal que $f^{-1}(B)$ es acotado en $\bb{R}^n$ para cada $B \subset \bb{R}^m$ acotado. Demuestra que entonces $f$ es cerrada.
    Deduce que la función $g$ del ejercicio anterior y las funciones polinómicas $p : (\bb{R}, \T_u) \to (\bb{R}, \T_u)$ son cerradas.
\end{ejercicio}

\begin{ejercicio}
    Demuestra que toda aplicación afín biyectiva en $\bb{R}^n$ de la forma $f : (\bb{R}^n, \T_u) \to (\bb{R}^n, \T_u)$ es un homeomorfismo.\\

    Por ser una aplicación afín, tenemos que $f$ es:
    \Func{f}{(\bb{R}^n,\T_u)}{(\bb{R}^n,\T_u)}{\left(\begin{array}{c}
        x_1\\ \vdots \\ x_n
    \end{array}\right)}{A\left(\begin{array}{c}
        x_1\\ \vdots \\ x_n
    \end{array}\right) + b}
    con $A\in \cc{M}_{n\times n}(\bb{R})$, $b\in \bb{R}^n$ fijo. Como $f$ es biyectiva, tenemos que $|A|\neq 0$.

    Es directo ver que $f$ es continua, ya que cada componente de una aplicación afín es una ecuación lineal, por lo que se trata de funciones continuas. Además, su inversa (que también es una aplicación afín) es también continua. Por tanto, $f$ es un homeomorfismo.\\
    
    
    Utiliza este resultado para construir un homeomorfismo:
    \begin{enumerate}
        \item Entre cualesquiera bolas abiertas, cualesquiera bolas cerradas y cualesquiera esferas de $(\bb{R}^n, d_u)$.

        Sean $c,c'\in \bb{R}^n$ fijos, y sean $\veps,\veps'\in \bb{R}^+$. Sea la aplicación lineal buscada una homotecia de centro $c$ y razón $\frac{\veps'}{\veps}$ compuesto con una traslación según el vector $v=\vec{cc'}$. Es decir:
        \Func{f}{(\bb{R}^n,\T_u)}{(\bb{R}^n,\T_u)}{x}{c + \dfrac{\veps'}{\veps} \vec{cx} + \vec{cc'}=c' + \dfrac{\veps'}{\veps} \vec{cx} = \dfrac{\veps'}{\veps} x + c'-\dfrac{\veps'}{\veps} c
        = \dfrac{\veps'}{\veps}(x-c) + c'}

        Veamos ahora que $f[B(c,\veps)]=B(c',\veps')$:
        \begin{description}
            \item[$\subset)$] Sea $x\in B(c,\veps)$, por lo que $\|x-c\|<\veps$.

            Tenemos que:
            \begin{equation*}
                \|f(x)-c'\| = \left\|\dfrac{\veps'}{\veps}(x-c) + \bcancel{c'}-\bcancel{c'}\right\| =  \left\|\dfrac{\veps'}{\veps}(x-c)\right\| < \veps'
            \end{equation*}
            Por tanto, $f(x)\in B(c',\veps')$.

            \item[$\supset)$] Sea $y'\in B(c', \veps')$. Veamos que $\exists x\in B(c,\veps)$ tal que $f(x)=y'$.

            Consideremos $x= \frac{\veps}{\veps'}(y'-c')+c$. Veamos que $x\in B(c,\veps)$:
            \begin{equation*}
                \|x-c\| = \left\|\frac{\veps}{\veps'}(y'-c')\right\| < \veps
            \end{equation*}
            Además, veamos que $f(x)=y'$:
            \begin{equation*}
                f\left(\frac{\veps}{\veps'}(y'-c')+c\right) = \frac{\veps'}{\veps}\left(\frac{\veps}{\veps'}(y'-c')\right)+c' = y'
            \end{equation*}
        \end{description}

        Análogamente, se demuestra que $f[\ol{B}(c,\veps)]=\ol{B}(c',\veps')$, $f[S(c,\veps)]=S(c',\veps')$.
        
        \item El cilindro circular $\bb{S}^1 \times \bb{R}$ y el cilindro elíptico $C = \{(x, y, z) \in \bb{R}^3 \mid x^2 +\frac{y^2}{4} = 1\}$ de $\bb{R}^3$.

        Tomando como sistema de referencia el usual, tenemos que la aplicación afín ha de de cumplir que $f(O)=O$, $f(e_1)=e_1$, $f(e_2)=2e_2$, $f(e_3)=e_3$. En definitiva, tenemos que $f(x,y,z)=(x,2y,z)$.

        Veamos ahora que $f(\bb{S}^1\times \bb{R})=C$.
        \begin{description}
            \item[$\subset)$]
            Sea $(x,y,z)\in \bb{S}^1\times \bb{R}$, y veamos si $f(x,y,z)\in C$. Como $(x,y,z)\in \bb{S}^1\times \bb{R}$, tenemos que $x^2+y^2=1$. Además, $f(x,y,z)=(x,2y,z)$. Por tanto, como se tiene que:
            \begin{equation*}
                x^2 + \frac{(2y)^2}{4}=x^2+y^2 = 1
            \end{equation*}
            Por tanto, tenemos que $(x,2y,z)\in C$.

            \item[$\supset)$] Sea $(u,v,w)\in C$, y veamos si $\exists (x,y,z)\in \bb{S}^1\times \bb{R}$ tal que $f(x,y,z)=(u,v,w)$.

            Como $(u,v,w)\in C$, tenemos que $u^2 + \frac{v^2}{4}=1$. Consideramos $(x,y,z)=\left(u,\dfrac{v}{2}, w\right)$. Tenemos claramente que $f\left(u,\dfrac{v}{2}, w\right)=(u,v,w)$. Veamos ahora que $\left(u,\dfrac{v}{2}, w\right)\in \bb{S}^1\times \bb{R}$.
            $$u^2 + \left(\frac{v}{2}\right)^2=u^2 + \frac{v^2}{4}=1$$
        \end{description}
    \end{enumerate}
\end{ejercicio}

\begin{ejercicio}
    Sea $X$ un conjunto. Demuestra que toda aplicación biyectiva $f : (X, \T_{CF}) \to (X, \T_{CF})$ es un homeomorfismo.

    Supongamos en primer lugar que $X$ es finito. Entonces, $\T_{CF}=\T_{disc}$, por lo que es continua y abierta y, entonces, un homeomorfismo.

    Supongamos ahora que $X$ es infinito. Veamos en primer lugar que $f$ es continua. Dado un abierto $B\in \T_{CF}$, entonces $X\setminus B$ es finito. Para ver si $f$ es continua, es necesario que $f^{-1}(B)\in \T_{CF}$ o, equivalentemente, que su complementario sea finito:
    \begin{equation*}
        X\setminus f^{-1}(B) = f^{-1}(X)\setminus f^{-1}(B) = f^{-1}(X\setminus B) \in C_{\T_{CF}}
    \end{equation*}
    que es finito porque $X\setminus B$ es finito y $f$ es biyectiva. Por tanto, $f$ es continua.
    
    Para ver si es abierta, tenemos que:
    \begin{equation*}
        X\setminus f(B) = f(X)\setminus f(B) = f(X\setminus B) \in C_{\T_{CF}}
    \end{equation*}
    que es finito porque $X\setminus B$ es finito y $f$ es biyectiva. Por tanto, $f$ es abierta.

    Como $f$ es continua, biyectiva y abierta, es un homeomorfismo.
\end{ejercicio}

\begin{ejercicio}
    Encuentra un contraejemplo que demuestre que la siguiente afirmación es falsa: Si existen aplicaciones continuas e inyectivas $f : (X,\T) \to (Y, \T')$ y $g : (Y, \T') \to (X,\T)$ entonces $(X,\T)$ e $(Y, \T')$ son homeomorfos.
\end{ejercicio}

\begin{ejercicio}
    Demuestra que toda aplicación $f : (\bb{R}, \T_u) \to (\bb{R}, \T_u)$ estrictamente creciente (decreciente) y continua es un embebimiento.\\

    Por ser estrictamente monótona, tenemos que es inyectiva. Al restringir el codominio a su imagen, tenemos que $f_{\big| Im(f)}$ es sobreyectiva; por lo que $f_{\big| Im(f)}$ es biyectiva.

    Veamos que su inversa es continua. Dado un intervalo $]a,b[~\subset X$, tenemos que:
    \begin{equation*}
        (f^{-1})_{\big| Im(f)}^{-1}(]a,b[) = f(]a,b[)= ]f(a),f(b)[~ \in \T_u
    \end{equation*}
    
\end{ejercicio}


\begin{ejercicio}
    Sea $A \subset \bb{R}^n$, $A\neq \emptyset$ y $f:(A,{(\T_u)}_A) \to (\bb{R}, \T_u)$ una función continua. Se define el \textbf{grafo} de $f$ como el como el subconjunto de $\bb{R}^{n+1}$ dado por:
    \begin{equation*}
        G(f) = \{(x, f(x)) \mid x \in A\} .
    \end{equation*}

    Demuestra que:
    \begin{enumerate}
        \item  $\left({A,(\T_u)}_A\right)$ es homeomorfo a $\left(G(f),{(\T_u)}_{G(f)}\right)$.

        Hay que probar que la siguiente aplicación es un homeomorfismo:
        \Func{H}{A}{G(f)}{x}{(x,f(x))}

        Tenemos que $H$ es continua, ya que la primera coordenada es la identidad (continua) y la segunda es $f$ (continua). Además, tenemos que $H$ es biyectiva. Su inversa es:
        \Func{H^{-1}}{G(f)}{A}{(x,y)}{x}

        Tenemos que $H^{-1}=F_{\big| G(f)}$, donde $F$ es la siguiente aplicación:
        \Func{F}{\bb{R}^{n+1}}{\bb{R}^n}{(x_0,\dots,x_n,x_{n+1})}{(x_0,\dots,x_n)}
        que es continua. Por tanto, $H$ es un homeomorfismo.
        
        \item La bola cerrada $\ol{B}(0, 1) \subset \bb{R}^n$ es homeomorfa al conjunto $\bb{S}^+ = \{(x, t) \in \bb{S}^n \mid t \geq 0\} \subset\bb{R}^{n+1}$.

        Tenemos que:
        \begin{equation*}
            \begin{split}
                \bb{S}^+ &= \{(x_1,\dots,x_n,x_{n+1})\in \bb{R}^{n+1}\mid x_1^2+\dots+x_n^2+x_{n+1}^2=1,~ x_{n+1}\geq 0\} =\\
                &= \{(x_1,\dots,x_n,x_{n+1})\in \bb{R}^{n+1}\mid x_{n+1}^2=1-x_1^2-\dots-x_n^2,~ x_{n+1}\geq 0\} =\\
                &= \left\{(x_1,\dots,x_n,x_{n+1})\in \bb{R}^{n+1}\mid x_{n+1}=\sqrt{1-x_1^2-\dots-x_n^2}\right\}
            \end{split}
        \end{equation*}

        Por tanto, sea la siguiente función:
        \Func{f}{\left(\ol{B}(0,1),{\T_u}_{\ol{B}(0,1)}\right)}{(\bb{R},\T_u)}{(x_1,\dots,x_n)}{\sqrt{1-x_1^2-\dots-x_n^2}}
        
        Tenemos que $f$ es continua por ser composición de una polinómica que toma valores en $\bb{R}^+_0$ con la raíz cuadrada, que es continua.

        Por tanto, tenemos que $\bb{S}^+=G(f)$, por lo que ambos conjuntos son homeomorfos.
        
        \item Las cuádricas $C_1,C_2,C_3$ son homeomorfas a $\bb{R}^2$:
        \begin{equation*}\begin{split}
            C_1 &= \{(x, y, z) \in \bb{R}^3 \mid z = x^2+y^2\}\\
            C_2 &= \{(x, y, z) \in \bb{R}^3 \mid z = x^2-y^2\}\\
            C_3 &= \{(x, y, z) \in \bb{R}^3 \mid x^2 + y^2 - z^2 = 0,~ z \geq 0\}
        \end{split}\end{equation*}

        Definimos $f_i:\bb{R}^2\to \bb{R}$ dadas por:
        \begin{equation*}
            f_1(x,y) = x^2+y^2\qquad f_2(x,y)=x^2-y^2\qquad f_3(x,y)=\sqrt{x^2+y^2}
        \end{equation*}
        En los tres casos, tenemos que $f_i$ es continua por ser polinómica. Además,
        $C_i=(x,G(f_i))$, por lo que $\bb{R}^2\cong C_i$ para todo $i=1,2,3$.
    \end{enumerate}
\end{ejercicio}

\begin{ejercicio}
    Demuestra que la siguiente aplicación es continua y biyectiva pero no es un homeomorfismo.
    \Func{f}{\left([0, 1[,{(\T_u)}_{[0,1[}\right)}{(\bb{S}^1,{(\T_u)}_{\bb{S}^1})}{t}{(\cos(2\pi t),\sen(2\pi t))}

    Tenemos claro que $f$ es continua, ya que cada componente lo es. Además, también es fácil ver que es biyectiva.

    No obstante, tenemos que:
    \begin{equation*}
        \begin{split}
            f\left(\left[0,\frac{1}{2}\right[\right) &= \{(x,y)\in \bb{S}^1 \mid x\in [1,-1[ ~\land~y\in [0,1] \} =\\&= \bb{S}^1 \cap \left(\{(x,y)\mid y>0\} \cup \{(1,0)\}\right)\notin {(\T_u)}_{\bb{S}^1}
        \end{split}
    \end{equation*}
    donde se especifica que no es un abierto, ya que no es un entorno del $(1,0)$. Por tanto, tenemos que no es abierta, ya que la imagen de un abierto no es un abierto.

     Por tanto, como $f$ es continua y biyectiva, como la continuidad de $f^{-1}$ equivale a ver si $f$ es abierta, tenemos que $f^{-1}$ no es continua y, por tanto, no es un homeomorfismo.
\end{ejercicio}

\begin{ejercicio}
    Sea $(\bb{R},\T_S)$ la recta de Sorgenfrey. Se define la aplicación $f : \bb{R} \to \bb{R}$ dada por
    \begin{equation*}
        f(x)=\left\{
            \begin{array}{ccc}
                e^x & \text{si} & x<0,\\
                3 & \text{si} & x\geq 0.
            \end{array}
        \right.
    \end{equation*}
    \begin{enumerate}
        \item Estudia la continuidad de $f : (\bb{R}, \T_u) \to (\bb{R}, \T_u)$, $f : (\bb{R},\T_S) \to (\bb{R},\T_S)$, $f :(\bb{R}, \T_u) \to (\bb{R},\T_S$) y $f : (\bb{R},\T_S) \to (\bb{R}, \T_u)$.
        \item Estudia si las aplicaciones anteriores son abiertas o cerradas.
    \end{enumerate}
\end{ejercicio}

\begin{ejercicio}
    Demuestra que ``ser metrizable'' es una propiedad topológica.\\

    Sea $(X,\T)$ un espacio topológico metrizable; es decir, sea $d:X\times X\to \bb{R}$ una distancia tal que $\T=\T_d$. Sea además $f:(X,\T)\to (Y,\T')$ un homeomorfismo, es decir $f$ continua, biyectiva y $f^{-1}$ continua.

    Definimos $d':Y\times Y\to \bb{R}$ de la forma:
    \begin{equation*}
        d'(y,y')=d(f^{-1}(y),f^{-1}(y')) \qquad \forall y,y'\in Y
    \end{equation*}
    Como $d$ es una distancia y $f$ es un homeomorfismo, se tiene que $d'$ es una distancia.
    
    Veamos que $(Y,\T')$ es metrizable con $d'$, es decir, $\T'=\T_{d'}$.
    \begin{description}
        \item[$\subset)$]
        Sea $U'\in \T'$. Entonces, por ser $f$ continua, tenemos que $f^{-1}(U')=U\in \T$. Por ser $(X,\T)$ metrizable, tenemos que $\forall x\in U,~\exists r\in \bb{R}^+\mid B_d(x,r)\subset U$. Entonces, veamos que $\forall y\in U'$, $B_{d'}(y,r)\subset U'$. Como $f$ es biyectiva, esto equivale a ver que $\forall x\in U$, $B_{d'}(f(x),r)\subset U'$.
        \begin{description}
            \item[$\subset)$] Sea $y'\in B_{d'}(f(x),r)\subset Y$. Como $f$ es un homeomorfismo, entonces $\exists! x'\in X$ tal que $f(x')=y'$. Entonces, $d'(y',f(x))=d(x',x)<r$. Por tanto, $x'\in B_d(x,r)\subset U$, por lo que $f(x')=y'\in f(U)=U'$.
        \end{description}
        Por tanto, $\T'\subset \T_{d'}$.

        \item[$\supset)$] Veamos que $\T_{d'}\subset \T'$. Sea $U'\in \T_{d'}$. Entonces, por ser $f$ continua, tenemos que $f^{-1}(U')=U\in \T$.
        
         Tenemos que $\forall y\in U',~\exists r\in \bb{R}^+$ tal que $B_{d'}(y,r)\subset U'$. Entonces, aplicando $f^{-1}$, tenemos que $\forall x\in f^{-1}(U'),~\exists r\in \bb{R}^+$ tal que $B_{d}(x,r)\subset U$, por lo que $f^{-1}(U')\in \T$. Por tanto, como $f^{-1}$ es continua, tenemos que $f(f^{-1}(U'))=U'\in \T$.
    \end{description}
\end{ejercicio}

\begin{ejercicio}
    Sean $(X,\T)$ e $(Y, \T')$ espacios topológicos. Demuestra que $(X,\T)$ e $(Y, \T')$ son dos espacios topológicos metrizables si y solo si $(X \times Y, \T \times \T')$ es un espacio topológico metrizable.

    \begin{description}
        \item[$\Longrightarrow)$] Supongamos que $(X,\T),(Y,\T')$ son metrizables. Entonces, veamos si la topología producto lo es. Nos definimos la siguiente distancia:
        %\Func{\ol{d}}{(X\times Y)\times (X\times Y)}{\bb{R}}{[(x,y),(x,y)]}{\max\{d(x,y), d(x',y')\}}
        \Func{\ol{d}}{(X\times Y)\times (X\times Y)}{\left[(x,y),(x,y)\right]}{\max\{d(x,y), d(x',y')\}}
        Es fácil ver que se trata de una distancia.

        Sea entonces una base para $\T_{\ol{d}}$ la dada por: $$\cc{B}_{\ol{d}} = \{B_{\ol{d}}[(x,y), r]\mid (x,y)\in X\times Y,~ r\in \bb{R}^+\}$$

        Sea la base para la topología producto:
        \begin{equation*}
            \cc{B} = \{B(x,r)\times B(y,r')\mid (x,y)\in X\times Y,~ r,r'\in \bb{R}^+\}
        \end{equation*}

        TERMINAR

        

        
    \end{description}
\end{ejercicio}

\begin{ejercicio}
    Sean $X = \{a, b, c\}$, $Y = \{u, v\}$, $\T_X = \{\emptyset, X, \{a\}, \{b, c\}\}$, $\T_Y = \{\emptyset, Y, \{u\}\}$. Halla la topología producto $\T_X\times \T_Y$.\\

    Tenemos que una base de dicha topología es:
    \begin{equation*}
        \cc{B}=\{\emptyset, X\times Y, X\times \{u\}, \{a\}\times Y, \{a\}\times \{u\}, \{b,c\}\times Y, \{b,c\}\times \{u\}\}
    \end{equation*}
    Es decir,
    \begin{equation*}
        \cc{B}=\{X\times Y, X\times \{u\}, \{a\}\times Y, \{(a,u)\}, \{b,c\}\times Y, \{(b,u), (c,u)\}\}
    \end{equation*}
\end{ejercicio}

\begin{ejercicio}
    Encuentra tres espacios topológicos $(X,\T)$, $(Y, \T')$ y $(Z, \T'')$ tales que $(X \times Y, \T \times \T') \cong (X \times Z, \T \times \T'')$ pero $(Y, \T')\not \cong (Z, \T'')$.
\end{ejercicio}


\begin{ejercicio}
    Sean$ (X,\T)$ e $(Y, \T')$ espacios topológicos y sean $A \subset X$ y $B \subset Y$. Demuestra que:
    \begin{enumerate}
        \item $[A\times B]^\circ = A^\circ \times B^\circ$.
        \begin{equation*}
            \begin{split}
                [A\times B]^\circ =& \left\{(x,y)\in X\times Y\mid (A\times B)\in N_{(x,y)}\right\} =\\
                =& \left\{(x,y)\in X\times Y\mid \exists U_\T\in \T_{X\times Y} \text{ con } (x,y)\in U_\T\subset (A\times B)\right\} \AstIg\\
                \AstIg& \left\{(x,y)\in X\times Y\mid \exists U\in \T, U'\in \T', \text{ tal que } (x,y)\in U\times U'\subset (A\times B)\right\} =\\
                = & \left\{(x,y)\in X\times Y\mid \exists U\in \T, U'\in \T', \text{ tal que } x\in U\subset A,~y\in U'\subset B\right\} =\\
                = & \left\{(x,y)\in X\times Y\mid  A\in N_x,B\in N_y\right\} =\\
                = & \left\{(x,y)\in X\times Y\mid x\in A^\circ, y\in B^\circ\right\} =\\
                =& A^\circ \times B^\circ
            \end{split}
        \end{equation*}
        donde en $(\ast)$ he aplicado que una base de la topología producto son el producto de abiertos.
        \item $\ol{A\times B}=\ol{A}\times \ol{B}$.
        \begin{equation*}
            \begin{split}
                \ol{A\times B} =& \left\{(x,y)\in X\times Y\mid (A\times B)\in N_{(x,y)}\right\} =\\
                =& \left\{(x,y)\in X\times Y\mid U\times U'\cap A\times B\neq \emptyset,\quad \forall U\in \T, U'\in \T' \text{ con } (x,y)\in U\times U'\right\} =\\
                =& \left\{(x,y)\in X\times Y\mid U\cap A\neq \emptyset, U'\cap B\neq \emptyset,\quad \forall U\in \T, U'\in \T' \text{ con } x\in U,~x'\in U'\right\} =\\
                =& \ol{A}\times \ol{B}
            \end{split}
        \end{equation*}
        
        
        \item $\partial{(A\times B)}=\left[\ol{A}\times \partial B\right]\cup \left[\partial A\times \ol{B}\right]$.
        \begin{equation*}
            \begin{split}
                \partial{(A\times B)} =& \ol{A\times B}\setminus [A\times B]^\circ = \left(\ol{A}\times \ol{B}\right)\setminus (A^\circ \times B^\circ) =\\=& \left(\ol{A}\times [\ol{B}\setminus B^\circ]\right)\cup \left([\ol{A}\setminus A^\circ]\times \ol{B}\right)
                = \left[\ol{A}\times \partial B\right]\cup \left[\partial A\times \ol{B}\right]
            \end{split}
        \end{equation*}
        
        \item $(\T \times \T')_{A\times B} = \T_A \times \T'_B$.

        Comprobemos que tienen la misma base:
        \begin{equation*}
            \begin{split}
                \cc{B}&= \{(U\times U')\cap (A\times B)\mid U\in \cc{B}_X,~U'\in \cc{B}_Y\} =\\
                &= \{(U\cap A)\times (U'\cap B)\mid U\in \cc{B}_X,~U'\in \cc{B}_Y\} =\\
                &= \cc{B}'
            \end{split}
        \end{equation*}
        donde $\cc{B}$ es una base de $(\T \times \T')_{A\times B}$, y $\cc{B}'$ es una base de $\T_A \times \T'_B$, ya que es el producto de abiertos básicos.
        
        \item $A \times B \in \T \times \T'$ si y solo si $A \in \T$ y $B \in \T'$.
        \begin{description}
            \item[$\Longleftarrow)$] Como $A\in \T$, $B\in \T'$, entonces $A\times B$ es un abierto básico de la topología producto, y en particular es un abierto básico de $\T\times \T'$.

            \item[$\Longrightarrow)$] Como $A\times B\in \T\times \T'$, entonces usando la base de la topología producto se tiene que para cada $(a,b)\in A\times B$, $\exists U\in \T,U'\in \T'$ tal que se tiene que $(a,b)\in U\times U'\subset A\times B$.

            Por un lado, se tiene entonces que para cada $a\in A$, $\exists U\in \T$ tal que $a\in U\subset A$. Por tanto, para cada $a\in A$ se tiene que $A\in N_a$, por lo que $A\in \T$. Análogamente, se tiene que $B\in \T'$.
        \end{description}
        \item $A \times B \in C_{\T \times \T'}$ si y solo si $A \in C_\T$ y $B \in C_{\T'}$.


        \item $A \times B$ es denso en $X \times Y$ si y solo si $A$ es denso en $X$ y $B$ es denso en $Y$.
    \end{enumerate}
\end{ejercicio}

\begin{ejercicio}
    Sean $(X, \T_{CF})$ e $(Y, \T_{CF})$ espacios topológicos con la topología cofinita. Demuestra que $\T_{CF}\times \T_{CF}$ no tiene por qué ser la topología cofinita en $X \times Y$.

    En el caso de que $X,Y$ sean finitos, tenemos que ambas topologías son la discreta, por lo que $\T_{CF}\times \T_{CF}=\T_{disc}$. Por tanto, buscamos ejemplos en los que alguno no sea finito.

    Sea $X=Y=\bb{R}$. Tenemos que $\bb{R}^\ast\times \bb{R}^\ast \in (\bb{R}\times \bb{R}, \T_{CF}\times \T_{CF})$. No obstante, $\bb{R}^\ast\times \bb{R}^\ast \notin (\bb{R}^2, \T_{CF})$, ya que:
    \begin{equation*}
        \bb{R}^2\setminus (\bb{R}^\ast\times \bb{R}^\ast) = \{(x,y)\in \bb{R}^2 \mid xy= 0\}
    \end{equation*}
    Es decir, se trata del plano quitándole los ejes.
\end{ejercicio}


\begin{ejercicio}
    Sean $(X, \T_{x_0})$ e $(Y, \T_{y_0})$ espacios topológicos con la topología del punto incluido. Demuestra que $\T_{x_0} \times \T_{y_0}$ no tiene por qué ser la topología $\T_{(x_0,y_0)}$ en $X \times Y$.

    Sea $U=\{(0,0), (1,1)\}$. Tenemos que $U\in (\bb{R}^2, \T_{(0,0)})$, pero $U\notin (\bb{R}\times \bb{R}, \T_0\times \T_0)$. Veámoslo.

    Una base de $\T_0$ es $\cc{B}=\{\{0,x\}\mid x\in X\}$. Por tanto, una base de $(\bb{R}\times \bb{R}, \T_0\times \T_0)$ es $\cc{B}\times \cc{B} = \{B_1\times B_2\mid B_1,B_2\in \cc{B}\} = \{\{(0,0), (0,x), (x,0), (x,x)\}\mid x\in X\}$.

    Por tanto, supongamos que $U\in (\bb{R}\times \bb{R}, \T_0\times \T_0)$. Entonces, como $(1,1)\in U$, $\exists x\in X$ tal que:
    \begin{equation*}
        (1,1)\in \{(0,0), (0,x), (x,0), (x,x)\}\subset U=\{(0,0), (1,1)\}
    \end{equation*}
    Por tanto, hemos llegado a una contradicción, por lo que $U$ no es un abierto.

    
\end{ejercicio}


\begin{ejercicio}
    En el espacio topológico producto $(\bb{R} \times \bb{R}, \T_ \times \T_S)$ calcula la clausura, el interior y la frontera del conjunto $A=[1, 2[ \times [1, 2[$. Estudia también si la aplicación $f : (\bb{R}^2, \T_u) \to(\bb{R}^2, \T_u \times \T_S)$ dada por $f(x, y) = (y, x)$ es continua.

    Tenemos que $\ol{A}=[1,2]\times [1,2[$. Además, $A^\circ = ]1,2[\times [1,2[$.

    Veamos ahora si $f$ es continua. Para ello, componemos o usamos un contraejemplo.

    Sea $]1,2[\times [1,2[$. Su preimagen es $[1,2[\times ]1,2[$.
    No es continua.
\end{ejercicio}


\begin{ejercicio}
    Sea $f : (X,\T) \to (Y, \T')$ una aplicación continua, abierta y sobreyectiva. Entonces, $(Y, \T')$ es $T2$ si y sólo si $\Delta f = \{(x, y) \in X \times X \mid f(x) = f(y)\}$ es un subconjunto cerrado de $(X \times X, \T \times \T )$. Deduce de aquí que un espacio topológico $(X,\T)$ es T2 si y sólo si el conjunto $\Delta = \{(x, x) \in X \times X \mid x \in X\}$ es cerrado en $(X \times X, \T \times \T )$.
\end{ejercicio}

\begin{ejercicio}
    Sean $(X,\T)$ e $(Y, \T')$ dos espacios topológicos con $(Y, \T')$ Hausdorff y sean $f, g :(X,\T) \to (Y, \T')$ aplicaciones continuas. Si existe un subconjunto $A \subset X$ tal que $f(x) = g(x)$ para todo $x \in A$ entonces $f(x) = g(x)$ para todo $x \in \ol{A}$. Demuestra que si $A$ es denso en $X$ entonces $f=g$.

    Demostramos por reducción al absurdo. Supongamos que $\exists x\in\ol{A}$ tal que $f(x)\neq g(x)\in Y$. Como $Y$ es T2, $\exists U,V\in \T$ tal que $f(x)\in U$, $g(x)\in V$, y $U\cap V=\emptyset$.

    Como $f,g$ son continuas, entonces $f^{-1}(U), g^{-1}(V)\in T$. Además, $x\in f^{-1}(U), g^{-1}(V)$. Entonces, $f^{-1}(U)\cap g^{-1}(V)\neq \emptyset$.

    Como $x\in \ol{A}$, entonces $f^{-1}(U)\cap A\neq \emptyset$, $g^{-1}(V)\cap A\neq \emptyset$. Por tanto, tenemos que $f^{-1}(U)\cap g^{-1}(V)\cap A \neq \emptyset$. Sea $x'\in f^{-1}(U)\cap g^{-1}(V)\cap A$. Como $x\in A$, entonces $f(x')=g(x')\in U\cap V$, por lo que llegamos a una contradicción.\\

    Veamos ahora que si $A$ es denso en $X$, entonces $f=g$. Como $A$ es denso, entonces $\ol{A}=X$. Por lo demostrado antes, $f(x)=g(x)$ para todo $x\in \ol{A}=X$, por lo que $f=g$.
\end{ejercicio}

\begin{ejercicio}
    Consideremos el espacio topológico $(\bb{R}^2, \T_{disc} \times \T )$, donde se tiene que $\T = \{\emptyset, \bb{R}, \bb{Q}, \bb{R}\setminus \bb{Q}\}$.
    \begin{enumerate}
        \item Encuentra una base de entornos, si es posible numerable, de cada punto de $\bb{R}^2$.

        Sea $(x,y)\in \bb{R}^2$, y busquemos $\beta_{(x,y)}$. Tenemos que:
        \begin{equation*}
            \beta_{(x,y)} = 
            \left\{
            \begin{array}{ccc}
                \{\{x\}\times \bb{Q}\} & \text{si} & y\in \bb{Q} \\
                \{\{x\}\times \bb{R}\setminus\bb{Q}\} & \text{si} & y\in \bb{R}\setminus\bb{Q} 
            \end{array}
            \right.
        \end{equation*}
        
        \item Encuentra un subconjunto no vacío $A \subsetneq \bb{R}^2$ que sea abierto y cerrado a la vez.

        Por ejemplo, $A=\bb{Q}\times \bb{Q}$.
        
        \item Sea $L = \{(x, y) \in \bb{R}^2\mid y = x\}$. ¿Es cerrado $L$? ¿Cuál es la topología producto $(\T_{disc} \times \T )_L$?

        Veamos que $L$ no es un cerrado. Para ello, consideramos su complementario $U=\bb{R}^2\setminus L$, y vemos si es un abierto.

        Supongamos que lo es, y lleguemos a contradicción. Como $(1,2)\in \bb{R}^2\setminus L=U$, tenemos que $U\in N_{(1,2)}$. Como $\beta_{(1,2)}=\{\{1\}\times \bb{Q}\}$ es una base de entornos, tenemos que $\{1\}\times \bb{Q}\subset U=\bb{R}^2\setminus L$. Por tanto, $(1,1)\in \bb{R}^2\setminus L$, algo que no es cierto ya que $(1,1)\in L$. Por tanto, llegamos a una contradicción, por lo que $L$ no es un cerrado.

        

        Tenemos que $(\T_{disc}\times \T)_L = (\T_{disc})_L$. Esto se debe a que $\{(x,x)\}\in (\T_{disc}\times \T)_L$, ya que:
        \begin{equation*}
            \{(x,x)\}=(\{x\}\times \bb{R})\cap L
        \end{equation*}
        como $\{x\}\in \T_{disc}$ y $\bb{R}\in \T$, entonces $\{x\}\times \bb{R}\in \T_{disc}\times \bb{R}$.
    \end{enumerate}
\end{ejercicio}


\begin{ejercicio}
    Sea $f : (\bb{R}, \T_u) \to (\bb{R}, \T_u)$ una aplicación continua que verifica la siguiente igualdad $f(x + y) = f(x)f(y),~\forall x, y \in \bb{R}$. Demuestra que $f\equiv 0$ o $f(x) = a^x$ para algún $a > 0$.
\end{ejercicio}

\begin{ejercicio}
    Sea $f : (X,\T) \to (Y, \T')$ una aplicación continua y sobreyectiva tal que para cada $y \in Y$ existe un entorno $N$ verificando que $f_{\big|f^{-1}(N)} : f^{-1}(N) \to N$ es una identificación. Demuestra que $f$ es una identificación.
\end{ejercicio}

\begin{ejercicio}
    Sean $(X,\T)$ e $(Y, \T')$ espacios topológicos. Demuestra que si $f : (X,\T) \to (Y, \T')$ es una aplicación continua, sobreyectiva y admite una inversa continua por la derecha (es decir, existe $g : (Y, \T') \to (X,\T)$ tal que $f \circ g = Id_Y$) entonces $f$ es una identificación.\\

    Como $f,g$ son continuas, tenemos que:
    \begin{equation*}
        f\circ g\text{ identificación} \Longleftrightarrow f \text{ identificación} 
    \end{equation*}
    Como $f\circ g=Id_Y$, que es una identificación, tenemos que $f$ lo es.
\end{ejercicio}

\begin{ejercicio}
    En $\bb{R}^2$ consideramos la siguiente relación de equivalencia
    \begin{equation*}
        (x, y)\cc{R}(x', y') \Longleftrightarrow x^2 + y = (x')^2 + y'
    \end{equation*}
    Demuestra que $(\bb{R}^2/\cc{R}, \T_u/\cc{R})$ es homeomorfo a $(\bb{R}, \T_u)$.\\

    Sea la siguiente aplicación:
    \Func{f}{\bb{R}^2}{\bb{R}}{(x,y)}{x^2+y}

    Es claro que $\cc{R}=\cc{R}_f$, por lo que $f$ se induce al siguiente cociente:
    \begin{equation*}
        \wt{f}:(\bb{R}^2/\cc{R}, \T_u/\cc{R})\longrightarrow (\bb{R}, \T_u)
    \end{equation*}

    Para ver si $\wt{f}$ es un homeomorfismo, basta con ver si $f$ es una identificación. Para ello, definimos la siguiente aplicación:
    \Func{g}{\bb{R}}{\bb{R}^2}{z}{(0,z)}

    Es claro que $f$ es continua por ser polinómica. Además, es sobreyectiva, ya que dado $z\in \bb{R}$, tenemos que $f(\sqrt{z-\lm}, \lm)=z$ para todo $\lm \leq z$. Además, $g$ es continua por serlo cada una de sus componentes, y:
    \begin{equation*}
        (f\circ g)(z) = f(g(z)) = f(0,z) = z = {Id}_{\bb{R}}(z) \qquad \forall z\in \bb{R}
    \end{equation*}
    Por tanto, $f\circ g = {Id}_{\bb{R}}$. Por tanto, $f$ es una identificación y se tiene que $\wt{f}$ es un homeomorfismo.
\end{ejercicio}

\begin{ejercicio}
    Demuestra que la proyección $p : X \to X/A$ es una biyección continua de $X \setminus A$ en su imagen. Demuestra también que es un homeomorfismo si $A$ es abierto o cerrado.\\

    En este caso, nos piden considerar $p_{\big| X\setminus A}:X\setminus A \to p(X\setminus A)$. Por haber restringido el codominio a la imagen de $X\setminus A$, hemos forzado que sea sobreyectiva. Veamos ahora si es inyectiva.
    
    Sean $x,y\in X\setminus A$, $x\neq y$. Entonces, como $x,y\notin A$, $x\cancel{\cc{R}}y$, por lo que las clases de equivalencia no son las mismas, es decir, $p_{\big| X\setminus A}(x)\neq p_{\big| X\setminus A}(y)$. Por tanto, $p_{\big| X\setminus A}$ es inyectiva.

    Por tanto, $p_{\big| X\setminus A}$ es biyectiva. Veamos ahora que es continua. Esto es directo, ya que por definición de topología cociente, la topología empleada es la final para su proyección, por lo que $p$ es continua. Por tanto, al restringir en el dominio, también es continua.


    TERMINAR
\end{ejercicio}

\begin{ejercicio}
    Da un ejemplo de un espacio topológico $(X,\T)$ y un subconjunto $A \subset X$ ni abierto ni cerrado tales que $X \setminus A$ no sea homeomorfo a $X/A \setminus \{[A]\}$.
\end{ejercicio}

\begin{ejercicio}
    Da un ejemplo de un espacio topológico $(X,\T)$ y una relación de equivalencia $R$ en $X$ tal que $(X,\T)$ sea Hausdorff pero $(X/\cc{R}, \T/\cc{R})$ no lo sea.
\end{ejercicio}

\begin{ejercicio}
    Da un ejemplo de un espacio topológico $(X,\T)$ y una relación de equivalencia $R$ en $X$ tal que $(X,\T)$ sea 2AN pero $(X/\cc{R}, \T /\cc{R})$ no lo sea.
\end{ejercicio}

\begin{ejercicio}
    Sea $(X,\T)$ un espacio topológico e $I = [0, 1]$. Se denomina cono de $X$ al espacio topológico cociente
    \begin{equation*}
        \left(\frac{X\times I}{X\times \{0\}},\frac{\T \times \T_u}{X\times \{0\}}\right)
    \end{equation*}
    Demuestra que que el cono de $(\bb{S}^n,{(\T_u)}_{\bb{S}^n})$ es homeomorfo a $(\ol{B^{n+1}},(\T_u)_{\ol{B^{n+1}}})$ para $n \geq 0$.
\end{ejercicio}

\begin{ejercicio}
    Sea $X = [0, 2]$ y $A = \{0, 1, 2\}$. Demuestra que $(X/A,(\T_u)_{X/A})$ es homeomorfo a $(C_1 \cup C_{-1},(\T_u)_{C_1 \cup C_{-1}})$, donde $C_1$ es la circunferencia de radio $1$ centrada en $(1, 0)$ y $C_{-1}$ es la circunferencia de radio $1$ centrada en $(-1, 0)$.
\end{ejercicio}

\begin{ejercicio}
    ¿Qué espacio se obtiene si en una banda de Möbius se identifican todos los puntos de su borde?
\end{ejercicio}

\begin{ejercicio}
    Ver que $\bb{R}\bb{P}^2$ es homeomorfo al cociente $((I \times I)/\cc{R},(\T \times \T )/\cc{R})$ donde $R$ es la menor relación que contiene a $(t, 0)R(1 - t, 1)$ y $(0, s)R(1, 1 - s)$ y $\T$ es la topología usual de I.
\end{ejercicio}

\begin{ejercicio}
    Sea $X = \{(x, y) \in \bb{R}^2\mid 1 \leq \|(x, y)\| \leq 2\}$. Se define una relación de equivalencia en $X$ de la siguiente forma: $(x, y)R(x', y')$ si y sólo si $(x, y) = (x', y')$ o $\|(x, y)\| - \|(x', y')\| = \pm 1$ y $(x, y) = \lambda(x', y')$, $\lambda > 0$. Demuestra que el espacio cociente es homeomorfo al toro.
\end{ejercicio}
\chapter{Series de términos sin restricción de signos}\label{chp:Tema11}

Hasta ahora, sólo nos habíamos preocupado de estudiar las propiedades de las series de términos no negativos. Vamos a completar nuestro estudio básico de las series de números reales eliminando la suposición que hicimos al principio del tema pasado y trabajaremos con series con no todos sus términos necesariamente no negativos. No demostraremos los resultados de este tema, así que se recomienda al lector que consulte las demostraciones en la bibliografía recomendada en la guía docente de la asignatura.

%########################################################################################################
% Resultados básicos. Convergencia conmutativa y absoluta.
%########################################################################################################

\section{Resultados básicos. Convergencia absoluta y conmutativa}
\begin{definicion}
    Se dice que una serie de números reales $\displaystyle\sum_{n \geq 1} a_n$ \textbf{converge absolutamente} si la serie $\displaystyle\sum_{n \geq 1} |a_n|$ converge.
\end{definicion}

Nótese que en el caso de series de términos no negativos la convergencia absoluta y la convergencia de la serie son equivalentes.

\begin{teo}
    Si la serie de números reales $\displaystyle\sum_{n \geq 1} a_n$ converge absolutamente, entonces es convergente, en cuyo caso se tiene que
    \begin{equation*}
        \left| \displaystyle\sum_{n = 1}^{\infty} a_n \right| \leq \displaystyle\sum_{n = 1}^{\infty} |a_n|
    \end{equation*}
\end{teo}

\begin{ejemplo}
    Algunos ejemplos son:
    \begin{enumerate}
        \item La serie $\displaystyle\sum_{n \geq 1} \frac{(-1)^{n+1}}{n^2}$ converge, ya que la serie $\displaystyle\sum_{n \geq 1} \frac{1}{n^2}$ converge, por lo que la serie $\displaystyle\sum_{n \geq 1} \frac{(-1)^{n+1}}{n^2}$ converge absolutamente.

        \item La serie $\displaystyle\sum_{n \geq 1} \frac{(-1)^{n+1}}{n}$ converge, pero no converge absolutamente, pues  la serie $\displaystyle\sum_{n \geq 1} \frac{1}{n}$ no converge como ya se probó anteriormente.
    \end{enumerate}
\end{ejemplo}

Los ejemplos anteriores ponen de manifiesto que la condición de converger absolutamente es más restrictiva que la de convergencia de la serie, pues ya hemos visto que hay series convergentes que no son absolutamente convergentes.\\

Las series que son convergentes pero no absolutamente convergentes tienen una sorprendente propiedad:
\begin{teo}[de Riemann]
    Sea $\displaystyle\sum_{n \geq 1} a_n$ una serie convergente pero no absolutamente convergente. Entonces:
    \begin{enumerate}
        \item Existe una biyección $\phi ~:~ \mathbb{N} \longrightarrow \mathbb{N}$ tal que la serie $\displaystyle\sum_{n \geq 1} a_{\phi(n)}$ no es convergente.

        \item Para todo $x \in \mathbb{R}$ existe una biyección $\phi_x ~:~ \mathbb{N} \longrightarrow \mathbb{N}$ tal que
        \begin{equation*}
            \displaystyle\sum_{n = 1}^{\infty} a_{\phi_x(n)} = x
        \end{equation*}
    \end{enumerate}
\end{teo}

Es decir, las series que son convergentes pero no convergen absolutamente cumplen que que existe una reordenación de los sumandos que hace que la serie obtenida no sea convergente
y también que podemos reordenar los sumandos para que la nueva serie sea convergente a cualquier número real.

\begin{definicion}
    Se dice que una serie de números reales $\sum_{n \geq 1} a_n$ \textbf{converge conmutativamente} si para toda biyección $\phi : \mathbb{N} \longrightarrow \mathbb{N}$ se tiene que la serie $\sum_{n \geq 1} a_{\phi(n)}$ es convergente con el mismo límite de la serie de partida.
\end{definicion}

Obviamente toda serie conmutativamente convergente es convergente, pues la aplicación $\text{Id}_{\mathbb{N}}$ (identidad de $\mathbb{N}$)  es biyectiva. El siguiente resultado establece la relación entre convergencia absoluta y conmutativa.

\begin{teo}
    Sea $\{a_n\}$ una sucesión de números reales.
    \begin{equation*}
        \displaystyle\sum_{n \geq 1} a_n ~ \text{converge conmutativamente} \Longleftrightarrow \displaystyle\sum_{n \geq 1} a_n ~ \text{converge absolutamente}
    \end{equation*}
\end{teo}

Terminamos el tema dando un último criterio:
\begin{teo}[Criterio de Leibniz]
    Sea $\{a_n\}$ una sucesión de números reales convergente a $0$ y decreciente. Entonces, $\displaystyle\sum_{n \geq 1} (-1)^{n+1} a_n$ converge.
\end{teo}

Realmente, no es necesario que la sucesión sea decreciente. Basta con exigir que exista un natural $m$ tal que
$a_{n+1} \leq a_n, ~ \forall n \geq m$, es decir, que sea decreciente a partir de un natural $m$ en adelante (¿Por qué?).

Nótese que el criterio de Leibniz nos sirve para probar la convergencia de la serie armónica alternada.
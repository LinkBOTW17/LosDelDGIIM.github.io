\chapter{Criterios de convergencia de sucesiones}\label{chp:Tema8}

En este tema veremos algunos criterios de convergencia básicos para sucesiones. Algunos de ellos los daremos sin demostrar, pues requieren de herramientas que aún no hemos introducido en estos apuntes, por lo que si el lector está interesado en conocer las demostraciones, se le recomienda buscar en la bibliografía recomendada en la guía docente de la asignatura.

%########################################################################################################
% Criterios de convergencia.
%########################################################################################################

\section{Criterios de convergencia}
\begin{teo}[Criterio de Stolz]
    Sean $\{a_n\}$ y $\{b_n\}$ dos sucesiones de números reales con $\{b_n\}$ estrictamente creciente y divergente positivamente.
    Entonces:
    \begin{equation*}
        \left\{\dfrac{a_{n+1} - a_n}{b_{n+1} - b_n}\right\} \longrightarrow L \implies \left\{\dfrac{a_n}{b_n}\right\} \longrightarrow L, \qquad L \in \mathbb{R} \cup \{+ \infty, -\infty\}.
    \end{equation*}
\end{teo}

\begin{ejemplo}
    Estudia la convergencia de la sucesión $\left\{\dfrac{1^3 + 2^3 + 3^3 + \dots + n^3}{n^4}\right\}$.
    \newline
    \newline
    Sean $\{a_n\} = \{1^3 + 2^3 + 3^3 + \dots + n^3\}$ y $\{b_n\} = \{n^4\}$. Es fácil ver que $\{b_n\}$ es estrictamente
    creciente y que diverge positivamente, por lo que podemos aplicar el criterio de Stolz.
    \begin{equation*}
        \left\{\frac{a_{n+1} - a_n}{b_{n+1} - b_n}\right\} = \left\{\frac{(n+1)^3}{(n+1)^4- n^4}\right\} = \left\{\frac{n^3 + 3n^2 + 3n + 1}{4n^3 + 6n^2 + 4n + 1}\right\}
    \end{equation*}
    
    Dividiendo numerador y denominador por $n^3$, tenemos
    \begin{equation*}
        \left\{\dfrac{a_{n+1} - a_n}{b_{n+1} - b_n}\right\} = \left\{\dfrac{\dfrac{n^3 + 3n^2 + 3n + 1}{n^3}}{\dfrac{4n^3 + 6n^2 + 4n + 1}{n^3}}\right\} =
        \left\{ \dfrac{1 + \dfrac{3}{n} + \dfrac{3}{n^2} + \dfrac{1}{n^3}}{4 + \dfrac{6}{n} + \dfrac{4}{n^2} + \dfrac{1}{n^3}} \right\} \longrightarrow \dfrac{1}{4}.
    \end{equation*}
    
    Por el criterio de Stolz, $\left\{\dfrac{a_n}{b_n}\right\} = \{\dfrac{1^3 + 2^3 + 3^3 + \dots + n^3}{n^4}\} \longrightarrow \dfrac{1}{4}$.
\end{ejemplo}

\begin{coro}[Criterio de la media aritmética]
    Sea $\{x_n\}$ una sucesión de números reales. Definimos la siguiente sucesión:
    \begin{equation*}
        \{y_n\} = \left\{\frac{x_1 + x_2 + \dots x_n}{n}\right\}.
    \end{equation*}
    Entonces:
    \begin{equation*}
        \{x_n\} \longrightarrow L \Longrightarrow \{y_n\} \longrightarrow L, \qquad L \in \mathbb{R} \cup \{+ \infty, -\infty\}.
    \end{equation*}
\end{coro}
\begin{proof}
    Basta con considerar $\{a_n\} = \{x_1 + x_2 + \dots x_n\}$ y $\{b_n\} = \{n\}$ y aplicar el criterio de Stolz para la sucesión $\{y_n\} = \{\frac{a_n}{b_n}\}$.
\end{proof}

\begin{coro}[Criterio de las medias geométricas]
    Sea $\{\alpha_n\}$ una sucesión de números reales positivos y sea $\{\beta_n\}$ tal que
    \begin{equation*}
        \beta_n = \sqrt[n]{\alpha_1 \cdot \alpha_2 \cdot \dots \alpha_n}, ~ \forall n \in \mathbb{N}.
    \end{equation*}
    Si $\{\alpha_n\} \longrightarrow L$, entonces $\{\beta_n\} \longrightarrow L$, $L \in \mathbb{R}^+_0 \cup \{+ \infty\}$.
\end{coro}

\begin{teo}[Criterio del cociente para sucesiones]
    Sea $\{a_n\}$ una sucesión de números reales tal que $a_n > 0, ~ \forall n \in \mathbb{N}$. Entonces:
    \begin{equation*}
        \left\{\frac{a_{n+1}}{a_n}\right\} \longrightarrow L \Longrightarrow \{\sqrt[n]{a_n}\} \longrightarrow L, \qquad L \in \mathbb{R}^+_0 \cup \{+ \infty\}.
    \end{equation*}
\end{teo}
\begin{proof}
    Sea $\alpha_1 = a_1, ~\alpha_2 = \frac{a_2}{a_1}, \dots, \alpha_n = \frac{a_{n}}{a_{n-1}}$. Por hipótesis, $\{\alpha_n\} \longrightarrow L$.
    Nótese que
    \begin{equation*}
        \beta_n = \sqrt[n]{\alpha_1 \cdot \alpha_2 \cdot \dots \alpha_n} = \sqrt[n]{a_n}.
    \end{equation*}
    Aplicando el criterio de las medias geométricas a $\{\alpha_n\}$, tenemos el resultado buscado.
\end{proof}

\begin{ejemplo}
    Estudia la convergencia de la sucesión $\{\sqrt[n]{n}\}$.\\
    
    Sea $a_n = n > 0 ~ \forall n \in \mathbb{N}$. Aplicamos el criterio del cociente:
    \begin{equation*}
        \left\{\frac{a_{n+1}}{a_n}\right\} = \left\{\frac{n+1}{n}\right\} \longrightarrow 1
    \end{equation*}
    y por el criterio del cociente, $\{\sqrt[n]{n}\} \longrightarrow 1$.
\end{ejemplo}

Antes de dar el último criterio de convergencia, es necesario dar la siguiente definición.
\begin{definicion}
    Definimos el \textbf{número $e$} como sigue:
    \begin{equation*}
        e = \lim \left\{ \left(1 + \frac{1}{n}\right)^n\right\}.
    \end{equation*}
\end{definicion}

Aunque no vamos a demostrarlo, el número $e$ es irracional, y tenemos que
\begin{equation*}
    e \approx 2,718281828459.
\end{equation*}

Dado $a > 0$, definimos el \textbf{logaritmo neperiano de $a$}, notado por $\ln(a)$\footnote{Otros autores utilizan la notación $\log(a)$ para el logaritmo neperiano.}, como el único número real tal que
\begin{equation*}
    e^{\ln(a)} = a.
\end{equation*}

\bigskip
Ya estamos en condiciones de dar el último criterio de convergencia de sucesiones de números reales.
\begin{teo}[Criterio de Euler (o criterio exponencial)\footnote{Aunque en estos apuntes no vamos a definir las potencias de exponente real, daremos por conocidas sus propiedades y las de los logaritmos.}]

    Sean $\{x_n\} \longrightarrow 1$, $x_n > 0, ~ \forall n \in \mathbb{N}$ e $\{y_n\}$ cualquiera.
    Entonces
    \begin{equation*}
        \left\{ y_n(x_n -1) \right\} \longrightarrow L\Longleftrightarrow \left\{ x_n^{y_n} \right\} \longrightarrow e^L
    \end{equation*}
    con $L \in \mathbb{R} \cup \{+ \infty, -\infty\}$.
\end{teo}

\begin{ejemplo}
    Estudia la convergencia de la sucesión $\{n (\sqrt[n]{2}-1)\}$.\\
    
    Sean $\{x_n\} = \{\sqrt[n]{2}\}$ e $\{y_n\} = \{n\}$. Dado que $\{x_n\} \longrightarrow 1$ y que $x_n > 0, ~ \forall n \in \mathbb{N}$, podemos aplicar el criterio de Euler.
    \begin{equation*}
        \left\{ x_n^{y_n} \right\} = \{(\sqrt[n]{2})^n\} = \{2\}
    \end{equation*}
    y por tanto, por el criterio de Euler
    \begin{equation*}
       \left\{ x_n^{y_n} \right\} \longrightarrow 2 \Longleftrightarrow \{n (\sqrt[n]{2}-1)\} \longrightarrow \ln(2)
    \end{equation*}
    ya que $2 = e^L \Longleftrightarrow L = \ln(2)$.
\end{ejemplo}


%########################################################################################################
% Ejercicios de Criterios de convergencia de sucesiones.
%########################################################################################################

\section{Ejercicios}

\begin{ejercicio}
    Estudiar la convergencia de las siguientes sucesiones:
    \begin{enumerate}
        \item $\left\{ \dfrac{n}{2^n} \right\}$
        \item $\left\{  \dfrac{2^n+n}{3^n-n} \right\}$
        \item $\left\{ \dfrac{1^3+2^3+\dots+n^3}{n^4}\right\}$
        \item $\left\{ \dfrac{1!+2!+3!+\dots+n!}{n!} \right\}$
        \item $\left\{ \dfrac{1+\dfrac{1}{2}+\dfrac{1}{3}+\dots+\dfrac{1}{n}}{n} \right\}$
        \item $\left\{ \sqrt[n]{n} \right\}$
        \item $\left\{ \sqrt[n]{n!} \right\}$
        \item $\left\{ \dfrac{n^2 \sqrt{n}}{1+2\sqrt{2}+3\sqrt{3}+\dots+n\sqrt{n}} \right\}$
        \item $\left\{ \dfrac{1}{n} \sqrt[n]{\dfrac{(2n)!}{n!}} \right\}$
        \item $\left\{ \left(\dfrac{n^2-5n+6}{n^2+2n+1}\right)^{\dfrac{n^2+5}{n+2}} \right\}$
        \item $\left\{ \sqrt[n]{\dfrac{n!}{(2n)^{n+1}}} \right\}$
    \end{enumerate}
\end{ejercicio}

\begin{ejercicio}
    Sean $a,b \in \mathbb{R}^+$ (fijos). Estudiar la convergencia de la sucesión:
    \begin{equation*}
        \left\{ (a^n + b^n)^\frac{1}{n} \right\}
    \end{equation*}
\end{ejercicio}

\begin{ejercicio}
    Estudiar la convergencia/divergencia de las siguientes sucesiones:
    \begin{enumerate}
        \item $\left\{\ln(n)\right\}$
        \item $\left\{\ln\left(\dfrac{1}{n}\right)\right\}$
        \item $\left\{\ln\left(\dfrac{n+1}{n}\right)\right\}$
        \item $\left\{\dfrac{\ln(n+1)}{\ln(n)}\right\}$
        \item $\left\{ \left(\dfrac{\ln(n+1)}{\ln(n)}\right)^n\right\}$
        \item $\left\{ \left(\dfrac{\ln(n+1)}{\ln(n)}\right)^{n \ln(n+1)}\right\}$
    \end{enumerate}
\end{ejercicio}

\begin{ejercicio}
    Estudiar la convergencia de las siguientes sucesiones:
    \begin{enumerate}
        \item $\left\{ \dfrac{n \ln(n)}{\ln(n!)} \right\}$,\quad $n >2$
        \item $\left\{ n (\sqrt[n]{a}-1) \right\}$,\quad $a \in \mathbb{R}^+$
        \item $\left\{ \dfrac{\alpha \sqrt[n]{a} + \beta \sqrt[n]{b}}{\alpha + \beta} \right\}$, 
        \quad$a,b \in \mathbb{R}^+$,\quad $\alpha, \beta \in \mathbb{R} ~:~ \alpha + \beta \neq 0$
    \end{enumerate}
\end{ejercicio}
\chapter{Funciones reales de variable real. Continuidad}\label{chp:Tema12}

%########################################################################################################
% Definiciones y propiedades básicas de las funciones.
%########################################################################################################

\section{Definiciones y propiedades básicas de las funciones}
\begin{definicion}
    Una \textbf{función real de variable real} es una aplicación $f ~:~ A \longrightarrow \mathbb{R}$, donde $A$ es un conjunto de números reales no vacío. Normalmente, nos referiremos a ellas como funciones simplemente.

    Al conjunto $A$ lo llamaremos \textbf{dominio de la función $f$}, al que notaremos por $\text{Dom}(f)$. Cuando no especifiquemos el dominio de la función (aunque en estos apuntes siempre lo especificaremos), se entiende que el dominio será el mayor\footnote{Entenderemos que con ``mayor'', nos referimos a que si un conjunto $A$ es ``mayor que'' otro conjunto $B$, entonces $B \subseteq A$.} conjunto posible para el que la función está bien definida. A dicho conjunto lo llamaremos \textbf{dominio maximal de $f$}.
\end{definicion}

A lo largo de este tema, salvo que se especifique lo contrario, supondremos que $A$ es
un conjunto no vacío de números reales.

\begin{definicion}
    Dadas dos funciones $f ~:~ A \longrightarrow \mathbb{R}$ y $g ~:~ A \longrightarrow \mathbb{R}$,
    definimos la \textbf{suma de $f$ y $g$} como la aplicación $f+g ~:~ A \longrightarrow \mathbb{R}$ dada por
    \begin{equation*}
        (f+g)(x) = f(x) + g(x), ~ \forall x \in A
    \end{equation*}
\end{definicion}

\begin{definicion}
    Dada una función $f ~:~ A \longrightarrow \mathbb{R}$ y $\alpha \in \mathbb{R}$, definimos el \textbf{producto escalar de $\alpha$ por $f$}
    como la aplicación $\alpha f ~:~ A \longrightarrow \mathbb{R}$ dada por
    \begin{equation*}
        (\alpha f)(x) = \alpha f(x), ~ \forall x \in A
    \end{equation*}
\end{definicion}

\begin{definicion}
    Dadas dos funciones $f ~:~ A \longrightarrow \mathbb{R}$ y $g ~:~ A \longrightarrow \mathbb{R}$,
    definimos el \textbf{producto de $f$ y $g$} como la aplicación $f \cdot g ~:~ A \longrightarrow \mathbb{R}$ dada por
    \begin{equation*}
        (f \cdot g)(x) = f(x) \cdot g(x), ~ \forall x \in A
    \end{equation*}
\end{definicion}

\begin{definicion}
    Dada una función $f ~:~ A \longrightarrow \mathbb{R}$, definimos la \textbf{imagen de $f$}, notaremos $\text{Im}(f)$ o $f(A)$,
    como el siguiente conjunto:
    \begin{equation*}
        \text{Im}(f):= \{f(x) ~:~ x \in A\}
    \end{equation*}
\end{definicion}

\begin{definicion}
    Sean $A,B$ dos conjuntos de números reales no vacíos. Sean $f ~:~ A \longrightarrow \mathbb{R}$ y $g ~:~ B \longrightarrow \mathbb{R}$ dos funciones,
    con $f(A) \subseteq B$. Definimos la \textbf{composición de $g$ con $f$} como
    la aplicación $g \circ f ~:~ A \longrightarrow \mathbb{R}$ definida por
    \begin{equation*}
        (g \circ f)(x) = g(f(x)), ~ \forall x \in A
    \end{equation*}
\end{definicion}

\begin{definicion}
    Dada una función $f ~:~ A \longrightarrow \mathbb{R}$, definimos la \textbf{gráfica de la función $f$}, notaremos $\text{Gr}(f)$,
    como el siguiente conjunto:
    \begin{equation*}
        \text{Gr}(f) := \{(x,f(x)) ~:~ x \in A\} \subseteq A \times \mathbb{R} \subseteq \mathbb{R}^2
    \end{equation*}
\end{definicion}

\begin{definicion}
    Sea $f ~:~ A \longrightarrow \mathbb{R}$ una función tal que $f(x) \neq 0, ~ \forall x \in A$.
    Definimos el \textbf{inverso de $f$} como la función $\dfrac{1}{f} ~:~ A \longrightarrow \mathbb{R}$ dada por
    \begin{equation*}
        \left( \frac{1}{f} \right)(x) = \frac{1}{f(x)}, ~ \forall x \in A
    \end{equation*}
\end{definicion}

\begin{definicion}
    Si $f ~:~ A \longrightarrow \mathbb{R}$ es una función inyectiva, entonces existe una única función
    $g ~:~ f(A) \longrightarrow \mathbb{R}$ tal que $(g \circ f)(x)=x, ~ \forall x \in A$.
    Dicha función $g$ recibe el nombre de \textbf{función inversa de $f$} (No confundir con la definición anterior),
    y la notaremos por $f^{-1}$.
\end{definicion}

\begin{definicion}
    Sean $f ~:~ A \longrightarrow \mathbb{R}$ una función y $B \subseteq A$ no vacío. A la función
    $f_{|_B} ~:~ B \longrightarrow \mathbb{R}$ dada por
    \begin{equation*}
        f_{|_B} (x) = f(x), ~ \forall x \in B
    \end{equation*}
    la llamaremos \textbf{restricción de $f$ a $B$}. Diremos también que $f$ es una \textbf{extensión de $f_{|_B}$ al conjunto $A$}.
\end{definicion}

%########################################################################################################
% Definición y primeras propiedades de las funciones continuas.
%########################################################################################################

\section{Definición y primeras propiedades de las funciones continuas}

\begin{definicion}
    Sean $f ~:~ A \longrightarrow \mathbb{R}$ una función y $x_0 \in A$. Diremos que $f$ es \textbf{continua en el punto $x_0$} si para toda sucesión $\{x_n\}$ de elementos de $A$ convergente a $x_0$ se tiene que $\{f(x_n)\} \longrightarrow f(x_0)$.
\end{definicion}

\begin{definicion}
    Dado $B \subseteq A$ no vacío, diremos que $f$ es \textbf{continua en (el conjunto) $B$} si $f$ es continua en todos los puntos de $B$. En particular, diremos que una función es continua si es continua en todos los puntos de su dominio.
\end{definicion}

Es importante notar que solo se puede hablar de continuidad de una función en puntos donde esta esté definida. Por ejemplo, consideremos la función $f ~:~ \mathbb{R}^* \longrightarrow \mathbb{R}$ dada por
\begin{equation*}
    f(x) = \frac{1}{x}, ~ \forall x \in \mathbb{R}^*
\end{equation*}
La afirmación ``$f$ no es continua en cero'' carece de sentido, pues la función no está definida en cero.

\begin{prop}[Álgebra de funciones continuas]\label{prop:12.2.3}
    Sean $f ~:~ A \longrightarrow \mathbb{R}$ y $g ~:~ A \longrightarrow \mathbb{R}$ dos funciones y $a \in A$ fijo.
    Supongamos que $f$ y $g$ son continuas en $a$. Entonces:
    \begin{enumerate}
        \item $f+g$ es continua en el punto $a$.
        \item $f \cdot g$ es continua en el punto $a$.
        \item Si $g(x) \neq 0, ~ \forall x \in A$, entonces $\frac{f}{g}$ es continua en el punto $a$.
    \end{enumerate}
\end{prop}
\begin{proof}
    Sea $\{x_n\} \longrightarrow a$, con $x_n \in A, ~ \forall n \in \mathbb{N}$. Por continuidad de $f$ y $g$ en $a$,
    se tiene que $\{f(x_n)\} \longrightarrow f(a)$ y $\{g(x_n)\} \longrightarrow g(a)$, por lo que
    \begin{equation*}
        \{(f+g)(x_n)\} = \{f(x_n)+g(x_n)\} \longrightarrow f(a) + g(a) = (f+g)(a)
    \end{equation*}
    y también que
    \begin{equation*}
        \{(f \cdot g)(x_n)\} = \{f(x_n) \cdot g(x_n)\} \longrightarrow f(a) \cdot g(a) = (f \cdot g)(a)
    \end{equation*}
    
    Si además, $g(x) \neq 0, ~ \forall x \in A$, razonando igual que antes vemos que
    \begin{equation*}
        \left\{ \left(\frac{f}{g}\right)(x_n) \right\} = \left\{ \frac{f(x_n)}{g(x_n)} \right\} \longrightarrow \frac{f(a)}{g(a)} = \left(\frac{f}{g}\right)(a),
    \end{equation*}
    como queríamos ver.
\end{proof}

\begin{ejemplo} Algunos ejemplos básicos de funciones continuas son:
\begin{enumerate}
    \item Dado $c \in \mathbb{R}$, la función $f ~:~ A \longrightarrow \mathbb{R}$ dada por
    \begin{equation*}
        f(x) = c, ~ \forall x \in A ~ \text{(función constantemente igual a c)}
    \end{equation*}
    es continua, pues si $x_0 \in A$ y $\{x_n\} \longrightarrow x_0$, con $x_n \in A, ~ \forall n \in \mathbb{N}$,
    entonces $\{f(x_n)\} = \{c\} \longrightarrow c = f(x_0)$.

    \item También es obvio que la función $f ~:~ A \longrightarrow \mathbb{R}$ dada por
    \begin{equation*}
        f(x) = x, ~ \forall x \in A
    \end{equation*}
    es continua.
\end{enumerate}
\end{ejemplo}

\begin{definicion}
    Una función $f ~:~ A \longrightarrow \mathbb{R}$ se dice \textbf{polinómica} si existen $p \in \mathbb{Z}$, $p \geq 0$, y
    $a_0, a_1, a_2, \dots, a_p \in \mathbb{R}$ tales que
    \begin{equation*}
        f(x) = a_0+a_1x+a_2x^2+\dots+a_px^p, ~ \forall x \in A,
    \end{equation*}
\end{definicion}

Utilizando lo probado en los ejemplos anteriores y la Proposición \ref{prop:12.2.3}, tenemos que toda función polinómica es continua.

\begin{definicion}
    Una función $f ~:~ A \longrightarrow \mathbb{R}$ se dice \textbf{racional} si existen dos funciones polinómicas $f_1$ y $f_2$ en $A$,
    con $f_2(x) \neq 0, ~ \forall x \in A$, tales que
    \begin{equation*}
        f(x) = \frac{f_1(x)}{f_2(x)}, ~ \forall x \in A
    \end{equation*}
\end{definicion}
A partir de la Proposición \ref{prop:12.2.3} y del ejemplo anterior, se deduce inmediatamente que toda función racional es continua.

\begin{prop}
    Sean $f ~:~ A \longrightarrow \mathbb{R}$ y $g ~:~ B \longrightarrow \mathbb{R}$ dos funciones, con $f(A) \subseteq B$.
    Sea $a \in A$ y $b = f(a) \in B$. Supongamos que $f$ es continua en el punto $a$ y que $g$ es continua en el punto $b$.
    Entonces $g \circ f$ es continua en el punto $a$.
\end{prop}
\begin{proof}
    Sea $\{x_n\} \longrightarrow a$, con $x_n \in A, ~ \forall n \in \mathbb{N}$ y sea $\{y_n\} = \{f(x_n)\}$.
    Por ser $y_n \in B, ~ \forall n \in \mathbb{N}$ y por continuidad de $f$, tenemos que $\{y_n\}$ es una
    sucesión de elementos de $B$ convergente a $b$, y por continuidad de $g$, tenemos que
    \begin{equation*}
        \{g(y_n)\} = \{g(f(x_n))\} \longrightarrow g(b) = g(f(a)) = (g \circ f)(a)
    \end{equation*}
    como queríamos ver.
\end{proof}

%########################################################################################################
% Carácter local de la continuidad.
%########################################################################################################

\section{Carácter local de la continuidad}
La noción de continuidad de una función en un punto involucra claramente al conjunto en el que la función está definida. Vamos que ocurre con la continuidad cuando se modifica el dominio de la función
mediante el procedimiento de restricción.

\begin{prop}\label{prop:12.3.1}
    Sea $f ~:~ A \longrightarrow \mathbb{R}$ una función real de variable real y sea $B \subseteq A$ no vacío. Entonces, $f_{|_B}$ es continua en todo punto de $B$ en el que lo sea $f$.
\end{prop}
\begin{proof}
    Sea $x_0 \in B$ tal que $f$ sea continua en $x_0$ y sea $\{x_n\} \longrightarrow x_0$, con $x_n \in B, ~ \forall n \in \mathbb{N}$.
    Al ser $B \subseteq A$, tenemos que $x_n \in A, ~ \forall n \in \mathbb{N}$ y por continuidad de $f$ tenemos que
    $\{f(x_n)\} \longrightarrow f(x_0)$, por lo que
    \begin{equation*}
        \{f_{|_B}(x_n)\} \longrightarrow f_{|_B}(x_0),
    \end{equation*}
    lo que prueba la continuidad de $f_{|_B}$ en $x_0$. 
\end{proof}

En general, el recíproco de la proposición anterior no se da. Consideramos la función
$f ~:~ \mathbb{R} \longrightarrow \mathbb{R}$ dada por
\begin{equation*}
    f(x) =
    \left\{ \begin{array}{lll}
        1 & \text{si} & x \in \mathbb{Q} \\
        -1 & \text{si} & x \in \mathbb{R} - \mathbb{Q}
    \end{array}
    \right.
\end{equation*}

Sea $B = \mathbb{Q}$. Es claro que $f_{|_{\mathbb{Q}}}$ es continua, pero $f$ no es continua en ningún punto de $\mathbb{Q}$
(de hecho no es continua en ningún punto de $\mathbb{R}$).

En general, si $B$ es un subconjunto no vacío del dominio, el hecho de que $f_{|_B}$ sea continua en todo punto de $B$ sin que lo sea $f$ se debe a que el conjunto $B$ es ``demasiado restrictivo''. Obtenemos ahora un resultado que nos permite deducir la continuidad de $f$ a partir de $f_{|_B}$ cuando el conjunto $B$ sea ``suficientemente amplio''.

\begin{prop}\label{prop:12.3.2}
    Sea $f ~:~ A \longrightarrow \mathbb{R}$ una función, $B \subseteq A$ no vacío y $x_0 \in B$. Supongamos que existe $\delta > 0$ tal que
    \begin{equation*}
        \left. \begin{array}{r}
            x \in A \\
            \left|x - x_0\right| < \delta
        \end{array}
        \right\}
        \Longrightarrow x \in B
    \end{equation*}
    Entonces, si $f_{|_B}$ es continua en $x_0$, $f$ es continua en $x_0$.
\end{prop}
\begin{proof}
    Sea $\{x_n\}$ una sucesión de puntos de $A$ convergente a $x_0$. Entonces
    \begin{equation*}
        \exists p \in \mathbb{N} ~:~ n \geq p \Longrightarrow \left|x - x_0\right| < \delta
    \end{equation*}
    con lo que para todo natural $n \geq p$ se tiene que $x_n \in B$. La sucesión $\{x_{n+p}\}$ es una sucesión de elementos de $B$ convergente a $x_0$, ya que es una parcial de $\{x_n\}$. Por ser $f_{|_B}$ continua en $x_0$ se tiene que $\{f(x_{n+p})\} \longrightarrow f(x_0)$, pero ello implica que la sucesión $\{f(x_n)\}$ converge a $f(x_0)$.
\end{proof}

Observemos que la condición que se impone a $B$, en la proposición anterior, es que contenga todos los puntos de $A$ ``suficientemente próximos'' a $x_0$.
Por tanto, lo que nos dice la proposición anterior es que la continuidad de una función en un punto sólo depende de su comportamiento en puntos ``suficientemente próximos'' a él.
Esta idea se suele expresar diciendo que la continuidad de una función en un punto es una \textbf{propiedad local}.

\begin{coro}\label{coro:12.3.3}
    Sea $f ~:~ A \longrightarrow \mathbb{R}$ una función y $x_0 \in A$. Sea $\delta > 0$, consideramos el conjunto
    \begin{equation*}
        B = \left\{x \in A ~:~ \left|x-x_0\right| < \delta \right\}
    \end{equation*}
    Entonces, $f$ es continua en $x_0$ si y sólo si $f_{|_B}$ es continua en $x_0$.
\end{coro}
\begin{proof}
    Por la Proposición \ref{prop:12.3.1}, si $f$ es continua en $x_0$, $f_{|_B}$ es continua en $x_0$.
    Evidentemente, si $x \in A$ y $\left|x-x_0\right| < \delta$, entonces $x \in B$ y basta con aplicar la Proposición \ref{prop:12.3.2}.
\end{proof}

\begin{ejemplo}
    Definimos la función \textbf{parte entera} como la función $E ~:~ \mathbb{R} \longrightarrow \mathbb{R}$ dada por
    \begin{equation*}
        E(x) = \max \left\{ p \in \mathbb{Z} ~:~ p \leq x \right\}, ~ \forall x \in \mathbb{R}
    \end{equation*}
    
    Vamos a estudiar la continuidad de la función parte entera con ayuda del corolario anterior.
    Sea $x_0$ un número real no entero. Entonces:
    \begin{equation*}
        \delta = \min \left\{x_0 - E(x_0), ~ E(x_0)+1-x_0\right\} > 0
    \end{equation*}
    
    Si $x \in \mathbb{R}$ y $\left|x-x_0\right| < \delta$, se tiene que
    \begin{equation*}
        E(x_0) < x < E(x_0)+1
    \end{equation*}
    y por tanto $E(x)=E(x_0)$, luego si
    \begin{equation*}
        B= \left\{x \in \mathbb{R} ~:~ \left|x-x_0\right| < \delta\right\}
    \end{equation*}
    entonces $E_{|_B}$ es constante y por el corolario anterior, es continua en $x_0$.

    Sea ahora $x_0 \in \mathbb{Z}$ y sea $x_n = x_0 - \frac{1}{n}, ~ \forall n \in \mathbb{N}$.
    Se tiene que $x_0-1 \leq x_n < x_0$ para todo natural $n$, luego
    \begin{equation*}
        E(x_n) = x_0-1, ~ \forall n \in \mathbb{N}
    \end{equation*}
    Así, $\{E(x_n)\}$ no converge a $E(x_0)$ y claramente $\{x_n\} \longrightarrow x_0$.
    \newline
    \newline
    En resumen, la función parte entera es continua en $\mathbb{R} - \mathbb{Z}$ y no es continua en
    ningún punto de $\mathbb{Z}$.
\end{ejemplo}

\begin{coro}\label{coro:12.3.4}
    Sea $A$ un conjunto de números reales no vacío y $x_0 \in A$. Supongamos que existe $\delta > 0$
    tal que
    \begin{equation*}
        \left\{x \in A ~:~ \left|x-x_0\right| < \delta \right\} = \{x_0\}
    \end{equation*}
    Entonces, toda función real de variable real definida en $A$ es continua en $x_0$.
\end{coro}
\begin{proof}
    Consecuencia inmediata del Corolario \ref{coro:12.3.3}, pues si $f$ es cualquier función real definida en $A$, entonces $f_{|_{\{x_0\}}}$ es, obviamente, continua.
\end{proof}


%########################################################################################################
% Caracterización de la continuidad.
%########################################################################################################

\section{Caracterización de la continuidad}
Damos a continuación un par de teoremas que nos permitirán caracterizar la continuidad de una función en un punto, y serán de suma utilidad en lo sucesivo.

\begin{teo}[Caracterización $\varepsilon$-$\delta$ de la continuidad]\label{teo:12.4.1}
    Sea $A$ un conjunto no vacío de números reales, $f ~:~ A \longrightarrow \mathbb{R}$ una función real de variable real y $x_0 \in A$ fijo. Entonces:
    \begin{equation*}
        f ~ \text{es continua en } ~ x_0 \Longleftrightarrow \forall \varepsilon > 0, ~ \exists \delta > 0 ~:~ x \in A, ~ |x-x_0| < \delta \Longrightarrow |f(x)-f(x_0)| < \varepsilon
    \end{equation*}
\end{teo}
\begin{proof} Procedemos mediante doble implicación.
\begin{description}
    \item[$\Longrightarrow)$]
        Por reducción al absurdo, supongamos que
        \begin{equation*}
            \exists \varepsilon_0 > 0 ~:~ \forall \delta > 0, ~ \exists x \in A ~ \text{con} ~ |x-x_0| < \delta ~ \text{y} ~ |f(x)-f(x_0)| \geq \varepsilon_0
        \end{equation*}
        Sea $\delta_n = \frac{1}{n}, ~ \forall n \in \mathbb{N}$. Entonces, $\exists x_n \in A ~:~ |x_n-x_0| < \delta_n$
        y $|f(x_n)-f(x_0)| \geq \varepsilon_0$. Entonces, tenemos que
        \begin{equation*}
            x_0 - \frac{1}{n} < x_n < x_0 + \frac{1}{n}, ~ \forall n \in \mathbb{N} \Longleftrightarrow \{x_n\} \longrightarrow x_0
        \end{equation*}
        y al ser $f$ continua, entonces $\{f(x_n)\} \longrightarrow f(x_0)$. Pero $|f(x_n)-f(x_0)| \geq \varepsilon_0$ para todo $n$
        natural en contra de que $\{f(x_n)\}$ converge a $f(x_0)$, lo que es una contradicción.
    \item[$\Longleftarrow)$]
        Sea $\{x_n\} \longrightarrow x_0$, con $x_n \in A, \forall n \in \mathbb{N}$.
        
        $\forall \varepsilon > 0$, por hipótesis, $\exists \delta > 0$ tal que si $x_n \in A$ con $|x_n - x_0| < \delta$
        se tiene que  $|f(x_n)-f(x_0)| < \varepsilon$. Por convergencia de $\{x_n\}$, tenemos que
        \begin{equation*}
            \exists m \in \mathbb{N} ~:~ n \geq m \Longrightarrow |x_n - x_0| < \delta
        \end{equation*}
        y aplicando la hipótesis, $|f(x_n)-f(x_0)| < \varepsilon$, y entonces $\{f(x_n)\} \longrightarrow f(x_0)$,
        por lo que $f$ es continua en $x_0$.
\end{description}
    
\end{proof}

Aunque en la definición de continuidad de una función $f$ en un punto $x_0$ de su dominio se exige que todas las sucesiones de elementos del dominio convergentes a $x_0$ verifiquen una determinada propiedad, podemos debilitar algo la condición de continuidad.

\begin{teo}[Debilitación de la condición de continuidad]\label{teo:12.4.2}
    Sea $A$ un conjunto no vacío de números reales, $f ~:~ A \longrightarrow \mathbb{R}$ una función real de variable real y $x_0 \in A$ fijo.
    
    Entonces, $f$ es continua en $x_0$ si, y sólo si, para toda sucesión $\{x_n\}$ de puntos de $A$, monótona y convergente a $x_0$, la sucesión $\{f(x_n)\}$ converge a $f(x_0)$.
\end{teo}\begin{proof} Procedemos mediante doble implicación.
\begin{description}
    \item[$\Longrightarrow)$]
        Evidente: Lo que se exige para toda sucesión de puntos de $A$ convergente a $x_0$ se exige, en particular, para aquellas que sean monótonas.
    \item[$\Longleftarrow)$]
        Por reducción al absurdo, supongamos que $f$ no es continua en $x_0$. Entonces:
        \begin{equation*}
            \exists \varepsilon_0 > 0 ~:~ \forall \delta > 0 ~ \exists x_{\delta} \in A ~ \text{con} ~ |x_{\delta}-x_0| < \delta ~ \text{y} ~ |f(x_{\delta})-f(x_0)| \geq \varepsilon_0
        \end{equation*}
        Para cada natural $n$, sea $\delta = \frac{1}{n}$ y sea $y_n=x_{\frac{1}{n}}$. Entonces, la sucesión $\{y_n\}$ de puntos de $A$ verifica que
        \begin{equation*}
            |y_n-x_0| < \delta ~ \text{y} ~ |f(y_n)-f(x_0)| \geq \varepsilon_0, ~ \forall n \in \mathbb{N}
        \end{equation*}
        Obviamente $\{y_n\} \longrightarrow x_0$. Sea $\{y_{\sigma(n)}\}$ una sucesión parcial de $\{y_n\}$ que sea monótona (Lema \ref{lema:6.2.5}).
        Aplicando la hipótesis, $\{f(y_{\sigma(n)})\} \longrightarrow f(x_0)$, lo cual es absurdo, pues
        \begin{equation*}
            |f(y_{\sigma(n)})-f(x_0)| \geq \varepsilon_0, ~ \forall n \in \mathbb{N}
        \end{equation*}
        Por lo tanto, $f$ es continua en $x_0$.
\end{description}
    
\end{proof}

El teorema anterior nos dice que para estudiar la continuidad de una función en un punto, basta con estudiar únicamente lo que ocurre con las sucesiones monótonas y convergentes a dicho punto.

%########################################################################################################
% Intervalos.
%########################################################################################################

\section{Intervalos}
Vamos a presentar una amplia familia de subconjuntos de $\mathbb{R}$, que jugarán un papel muy importante en los teoremas fundamentales de las funciones continuas.
\begin{notacion}
    Dados dos números reales $a$ y $b$, con $a \leq b$, notaremos:
    \begin{gather*}
        [a,b] = \{x \in \mathbb{R} ~:~ a \leq x \leq b\}\\
        ]a,b]= \{x \in \mathbb{R} ~:~ a < x \leq b\}\\
        [a,b[ ~ = \{x \in \mathbb{R} ~:~ a \leq x < b\}\\
        ]a,b[ ~ = \{x \in \mathbb{R} ~:~ a < x < b\}
    \end{gather*}

    que reciben, respectivamente, el nombre de \textbf{intervalo cerrado, abierto por la derecha, abierto por la izquierda y abierto, de origen $a$ y extremo $b$}.
\end{notacion}

\begin{notacion}
    Dado un número real $a$ cualquiera, notaremos:
    \begin{gather*}
        ]-\infty, a] = \{x \in \mathbb{R} ~:~ x \leq a\}\\
        ]-\infty, a[ ~ = \{x \in \mathbb{R} ~:~ x < a\}\\
        [a, +\infty[ ~ = \{x \in \mathbb{R} ~:~ x \geq a\}\\
        ]a, +\infty[ ~ = \{x \in \mathbb{R} ~:~ x > a\}
    \end{gather*}
    que reciben, respectivamente, el nombre de \textbf{semirrecta cerrada de extremo $a$, abierta de extremo $a$, cerrada de origen $a$ y abierta de origen $a$}.
\end{notacion}

\begin{definicion}
    Diremos que un conjunto $A$ de números reales es un \textbf{intervalo} si $A = \mathbb{R}$ o bien $A$ responde a una de las $8$ descripciones anteriores (Nótese que el conjunto vacío y el conjunto $\{a\}$ para todo $a \in \mathbb{R}$ son intervalos).
\end{definicion}

Es inmediato que si $A$ es un intervalo y $x,y \in A$ con $x < y$, todo real $z$ tal que
\begin{equation*}
    x \leq z \leq y
\end{equation*}
pertenece también al conjunto $A$. Veamos que esta propiedad caracteriza a los intervalos.

\begin{teo}[Caracterización de los intervalos]\label{prop:12.5.2}
    Un conjunto de números reales $A$ es un intervalo si, y sólo si, para cualesquiera $x,y \in A$ con $x < y$ se verifica que $[x,y] \subseteq A$.
\end{teo}
\begin{proof}
    El razonamiento anterior nos dice que la condición es necesaria. Veamos que también es suficiente.\\
    
    Si $A$ es vacío no hay nada que probar, pues para todo $a \in \mathbb{R}$ se verifica que $A = ~  ]a,a[$.
    Supongamos que $A$ es no vacío; distinguiremos varios casos:
    \begin{itemize}
        \item Si $A$ no está mayorado ni minorado, dado $z \in \mathbb{R}$, $z$ no es mayorante ni minorante de $A$, luego existen $x,y \in A$ con $x < z < y$, de donde $z \in [x,y] \subseteq A$ y obtenemos que $A=\mathbb{R}$.

        \item Si $A$ está mayorado pero no minorado, sea $b = \sup A$. Entonces, dado $z \in \mathbb{R}$, con $z < b$,
        $z$ no es mayorante ni minorante del conjunto $A$, luego existen $x,y \in A$. con $x < z < y$, de donde
        $z \in [x,y] \subseteq A$. Se tiene por tanto:
        \begin{equation*}
            ]-\infty, b[ ~ \subseteq A \subseteq ~ ]-\infty, b]
        \end{equation*}
        lo que deja dos posibilidades, $A = ~ ]-\infty, b[$ o $A = ~ ]-\infty, b]$.

        \item Si $A$ está minorado pero no mayorado, se demuestra de forma análoga al caso anterior que $A = [a, +\infty[$ o que $A = ~ ]a, +\infty[$, donde $a = \inf A$.

        \item Finalmente, si $A$ está acotado, sean $a = \inf A$ y $b = \sup A$. Dado $z \in \mathbb{R}$ con
        $a < z < b$, $z$ no puede ser mayorante ni minorante de $A$, luego existen $x,y \in A$ con $x < z < y$,
        de donde $z \in A$. Así tenemos que
        \begin{equation*}
        ]a,b[ ~ \subseteq A \subseteq [a,b]
        \end{equation*}
        lo que deja 4 posibilidades, según si $a$ y $b$ pertenezcan o no al conjunto A:
        \begin{equation*}
            A = ~ ]a,b[ ~~, ~~ A = ~ ]a,b] ~~, ~~ A = [a,b[ ~~, ~~ A = [a,b]
        \end{equation*}
    \end{itemize}
\end{proof}


%########################################################################################################
% Resultados fundamentales sobre funciones continuas.
%########################################################################################################

\newpage

\section{Resultados fundamentales sobre funciones continuas}
Hasta ahora, solamente hemos tratado el problema sobre si una determinada función es continua. Ahora bien, supuesto que una función sea continua, ¿qué información adicional poseemos sobre la función?. Vamos a dedicar este apartado a obtener una serie de resultados fundamentales sobre las funciones continuas, preferentemente de las funciones continuas definidas en intervalos. En lo sucesivo, cuando se hable de un intervalo, entenderemos que es no vacío.

\begin{prop}[Conservación del signo]
    Sea $A$ un conjunto no vacío de números reales, $x_0 \in A$ y $f ~:~ A \longrightarrow \mathbb{R}$ una función
    continua en $x_0$ con $f(x_0) \neq 0$.
    
    Entonces, existe $\delta > 0$ tal que si $x \in A$ con $|x-x_0| < \delta$ se tiene que $f(x) f(x_0) > 0$ ($f(x)$ tiene el mismo signo que $f(x_0)$).
\end{prop}
\begin{proof}
    Por ser $f$ continua en $x_0$, tenemos que
    \begin{equation*}
        \forall \varepsilon > 0 ~ \exists \delta > 0 ~:~ x \in A ~ \cap ~ ]x_0-\delta,x_0+\delta[ ~ \Longrightarrow |f(x)-f(x_0)| < \varepsilon
    \end{equation*}
    En particular, tomando $\varepsilon = |f(x_0)|$, obtenemos
    \begin{equation*}
        \exists \delta > 0 ~:~ x \in A ~ \cap ~ ]x_0-\delta,x_0+\delta[ ~ \Longrightarrow |f(x)-f(x_0)| < |f(x_0)|
    \end{equation*}
    y por tanto
    \begin{equation*}
        f(x_0) - |f(x_0)| < f(x) < f(x_0) + |f(x_0)|
    \end{equation*}
    de donde se deduce que $f(x)$ y $f(x_0)$ tienen el mismo signo.
\end{proof}

\begin{teo}[de los ceros de Bolzano]
    Sea $a,b \in \mathbb{R}$ con $a < b$ y $f ~:~ [a,b] \longrightarrow \mathbb{R}$ continua verificando que
    \begin{equation*}
        f(a) f(b) < 0 ~~ (\text{$f(a)$ y $f(b)$ tienen distinto signo})
    \end{equation*}
    Entonces, $\exists c \in ~ ]a,b[$ tal que $f(c) = 0$.
\end{teo}
\begin{proof}
    Supongamos que $f(a) < 0 < f(b)$. Definimos el conjunto $C$ como sigue:
    \begin{equation*}
        C= \{x \in [a,b] ~:~ f(x) < 0\}
    \end{equation*}
    Es fácil ver que $C$ es un conjunto de números reales no vacío y mayorado. Sea $c = \sup C$.
    Es claro que $c \in [a,b]$. Entonces, existe una sucesión $\{x_n\}$ de elementos de $C$ convergente
    a $c$ y por continuidad de $f$ en $c$ entonces $\{f(x_n)\} \longrightarrow f(c)$. Dado que
    \begin{equation*}
        f(x_n) < 0, ~ \forall n \in \mathbb{N}
    \end{equation*}
    entonces $f(c) \leq 0$. En particular, deducimos que $c \neq b$ y $c \leq b$, por lo que $c \in ~ ]a,b[$.\\
    
    Sea $\{z_n\} = \left\{c + \frac{b-c}{n}\right\}$. Es claro que
    \begin{equation*}
        z_n \in [a,b] ~ \text{y} ~ z_n \notin C, ~ \forall n \in \mathbb{N}
    \end{equation*}
    y por tanto ha de ser
    \begin{equation*}
        f(z_n) \geq 0, ~ \forall n \in \mathbb{N}
    \end{equation*}
    
    Evidentemente, $\{z_n\} \longrightarrow c$ y usando que $f$ es continua en $c$ y lo anterior deducimos que
    \begin{equation*}
        \{f(z_n)\} \longrightarrow f(c) \geq 0
    \end{equation*}
    y por tanto ha de ser $f(c) = 0$.\\
    
    Si fuera $f(b) < 0 < f(a)$, podemos razonar igual que antes o aplicar lo que acabamos de obtener a la función $-f$
    ($f$ es continua si y sólo si lo es $-f$).
\end{proof}

\begin{prop}[del valor intermedio]
    Sea $I$ un intervalo y $f ~:~ I \longrightarrow \mathbb{R}$ una función continua en $I$.
    
    Entonces, $f(I)$ es un intervalo.
\end{prop}
\begin{proof}
    Sean $\alpha, \beta \in f(I)$, con $\alpha \neq \beta$. Supongamos que $\alpha < \beta$.
    Sea $\lambda \in \mathbb{R}$ tal que $\alpha < \lambda <\beta$. Sean $x,y \in I$ tales que
    $f(x) = \alpha$ y $f(y) = \beta$.\\
    
    Definimos $g ~:~ I \longrightarrow \mathbb{R}$ dada por
    \begin{equation*}
        g(t) = f(t) - \lambda, ~ \forall t \in I
    \end{equation*}
    
    Es claro que $g$ es una función continua, que $g(x) < 0$ y que $g(y) > 0$, por lo que,
    aplicando el Teorema de Bolzano a la restricción de $g$ al intervalo $[x,y]$, existe $z \in I$ tal que
    \begin{equation*}
        g(z) = 0 \Longleftrightarrow f(z) - \lambda = 0 \Longleftrightarrow f(z) = \lambda
    \end{equation*}
    
    Esto prueba que $[\alpha, \beta] \subseteq f(I)$ para cualesquiera $\alpha, \beta \in f(I)$, con $\alpha < \beta$,
    por lo que $f(I)$ es un intervalo (Proposición \ref{prop:12.5.2}).\\
    
    Si fuera $\alpha > \beta$, basta con razonar igual que antes sobre la función
    $g ~:~ I \longrightarrow \mathbb{R}$ dada por
    \begin{equation*}
        g(t) = \lambda - f(t), ~ \forall t \in I
    \end{equation*}
\end{proof}

Nótese que el hecho de que $f(I)$ sea un intervalo significa que si $f$ toma dos valores cualesquiera, entonces está obligada a tomar todos los intermedios. Por tanto, el Teorema del valor intermedio incluye al Teorema de Bolzano como caso particular, pero muy poco particular, ya que la demostración se reduce fácilmente, como hemos visto, a dicho caso.

\begin{definicion}
    Sea $f ~:~ A \longrightarrow \mathbb{R}$ una función. Diremos que $f$ está \textbf{acotada} (respectivamente \textbf{mayorada}, \textbf{minorada}) si el conjunto $\{f(x) ~:~ x \in A\}$ está acotado (respectivamente mayorado, minorado).
    \begin{gather*}
        f ~ \text{está mayorada} \Longleftrightarrow \exists K \in \mathbb{R} ~:~ K \geq f(x), ~ \forall x \in A \\
        f ~ \text{está minorada} \Longleftrightarrow \exists k \in \mathbb{R} ~:~ k \leq f(x), ~ \forall x \in A \\
        f ~ \text{está acotada} \Longleftrightarrow \exists M \in \mathbb{R^+} ~:~ |f(x)| \leq M, ~ \forall x \in A
    \end{gather*}
    
    Diremos que $f$ tiene \textbf{máximo} (respectivamente \textbf{mínimo}) \textbf{absoluto} si su imagen tiene máximo (respectivamente mínimo) absoluto. Si $x_0 \in A$ con $f(x_0) = \max f(A)$ (respectivamente $f(x_0) = \min f(A)$), diremos que $f$ alcanza su máximo absoluto (respectivamente mínimo absoluto)
    en el punto $x_0$.
\end{definicion}

\begin{teo}[de Weierstrass o Propiedad de compacidad\footnote{También conocido como Teorema de Bolzano-Weierstrass para funciones continuas. No confundirlo con el que dimos para sucesiones.}]
    La imagen por una función continua de un intervalo cerrado y acotado es un intervalo cerrado y acotado. En particular, $f$ alcanza su máximo y su mínimo absolutos en dicho intervalo.
\end{teo}
\begin{proof}
    Sea $f ~:~ [a,b] \longrightarrow \mathbb{R}$ continua en $[a,b]$. Ya sabemos que $f([a,b])$ es un intervalo por el teorema del valor intermedio.\\
    
    Empezaremos probando que $f([a,b])$ está acotado. De lo contrario el conjunto
    \begin{equation*}
        \left\{ \left| f(x) \right| ~:~ x \in [a,b] \right\}
    \end{equation*}
    no está mayorado, luego dado un natural $n$ debe existir $x_n \in [a,b]$ tal que $|f(x_n)| >~n$.
    La sucesión $\{x_n\}$ así construida es acotada, luego por el teorema de Bolzano-Weierstrass admite una parcial $\{x_{\sigma(n)}\}$ convergente.\\
    
    Sea $x = \lim \{x_{\sigma(n)}\}$. Dado que $a \leq x_{\sigma(n)} \leq b, ~ \forall n \in \mathbb{N}$, tenemos que $x \in [a,b]$ y que por ser $f$ continua en $x$ tenemos que $\{f(x_{\sigma(n)})\}$ converge a $f(x)$ y 
    en particular es acotada. Pero entonces existe un número real $M$ tal que $|f(x_{\sigma(n)})| \leq M$ para todo $n$ natural, de donde deducimos que
    \begin{equation*}
        n \leq \sigma(n) < |f(x_{\sigma(n)})| \leq M, ~ \forall n \in \mathbb{N}
    \end{equation*}
    lo que es una contradicción, pues $\mathbb{N}$ no está mayorado. Así,  $f([a,b])$ está acotado.\\
    
    Sea $\alpha = \inf f([a,b])$ y $\beta = \sup f([a,b])$. Sea $\{y_n\}$ una sucesión de puntos de $f([a,b])$
    convergente a $\beta$ y para cada natural $n$, sea $t_n \in [a,b]$ tal que $f(t_n) = y_n$.\\
    
    Es claro que $\{t_n\}$ es una sucesión acotada; sea $\{t_{\sigma(n)}\}$ una sucesión parcial de $\{t_n\}$
    convergente a $t \in [a,b]$. Por ser $f$ continua en $t$ tenemos que $\{f(t_{\sigma(n)})\} \longrightarrow f(t)$, pero
    $\{f(t_{\sigma(n)})\} = \{y_{\sigma(n)}\}$ e $\{y_{\sigma(n)}\}$ converge a $\beta$, luego
    \begin{equation*}
        \beta = f(t) \in f([a,b])
    \end{equation*}
    
    El mismo razonamiento puede aplicarse para probar que $\alpha \in f([a,b])$. Por el teorema del valor intermedio, $[\alpha, \beta] \subseteq f([a,b])$, pero la inclusión contraria es trivialmente cierta, luego
    \begin{equation*}
        [\alpha, \beta] = f([a,b])
    \end{equation*}
    lo que demuestra el teorema.
\end{proof}

Óbservese que la hipótesis de que el intervalo de definición de la función, en el teorema anterior, sea cerrado y acotado es fundamental en la demostración. Si hubiésemos tenido un intervalo no acotado las sucesiones $\{x_n\}$ y $\{t_n\}$ que aparecen en la demostración no tendrían que estar acotadas, mientras que si hubiéramos tenido un intervalo acotado pero no cerrado los límites $x$ y $t$ de las parciales convergentes extraídas no tendrían por qué pertenecer al intervalo.

%########################################################################################################
% Funciones continuas e inyectivas.
%########################################################################################################

\section{Funciones continuas e inyectivas}

El último teorema que hemos visto afirmaba que toda función continua en un intervalo cerrado y acotado tiene máximo y mínimo absolutos, pero nada nos dice sobre los puntos en los que se alcanzan. Bajo la hipótesis adicional de que $f$ sea inyectiva, veremos que el máximo y el mínimo se alcanzan en los extremos del intervalo, pero esto no es más que el punto de partida para resultados más importantes\footnote{Los resultados que vamos a presentar en este apartado serán de gran utilidad cuando se estudie la derivabilidad de las funciones reales
de variable real en la asignatura de Cálculo II.}.

\begin{lema}\label{lema:12.7.1}
    Sea $f ~:~ [a,b] \longrightarrow \mathbb{R}$, con $a < b$, continua e inyectiva. Supongamos que $f(a) < f(b)$.
    Entonces, para todo $t$ verificando que $a < t < b$ se tiene que
    \begin{equation*}
        f(a) < f(t) < f(b)
    \end{equation*}
\end{lema}
\begin{proof}
    Sea $t \in ~ ]a,b[$. Si fuera $f(t) < f(a)$, si consideramos la restricción de $f$ al intervalo $[t,b]$, tenemos que $f_{|_{[t,b]}}$ es continua y al ser $f(t) < f(a)$, por el teorema del valor intermedio encontramos $z \in ~ ]t,b[$ tal que $f(z) = f(a)$. Pero $z \neq a$ en contra de que $f$ es inyectiva.\\
    
    Si fuera $f(b) < f(t)$, razonando igual que antes, encontramos $z \in ~ ]a,t[$ tal que $f(z) = f(b)$, pero $z \neq b$ en contra de que $f$ es inyectiva.\\
    
    Entonces, debe cumplirse que
    \begin{equation*}
        f(a) \leq f(t) \leq f(b), ~ \forall t \in [a,b],
    \end{equation*}
    pero al ser $f$ inyectiva, ambas desigualdades son estrictas.
\end{proof}

Notemos que el lema anterior puede aplicarse sucesivamente. Si $c \in ~]a,b[$, tenemos que $f(a) < f(c) < f(b)$ y al volver a aplicar el lema a las restricciones de $f$ a los intervalos $[a,c]$ y $[c,b]$, obtenemos que
\begin{equation*}
    a < x < c < y < b \Longrightarrow f(a) < f(x) < f(c) < f(y) < f(b)
\end{equation*}
y así sucesivamente. Vemos entonces que $f$ tiene un comportamiento muy concreto, pues crece al crecer la variable.
Antes de expresar de forma rigurosa este hecho, vamos a concretar algunos conceptos:

\begin{definicion}
    Sea $f ~:~ A \longrightarrow \mathbb{R}$ una función real de variable real.
    Diremos que $f$ es \textbf{creciente} (respectivamente \textbf{decreciente}) si
    \begin{equation*}
        \forall x,y \in A, ~ x < y \Longrightarrow f(x) \leq f(y) ~ (\text{respectivamente} ~ f(x) \geq f(y))
    \end{equation*}
    Diremos que $f$ es \textbf{estrictamente creciente} (respectivamente \textbf{estrictamente decreciente}) si
    \begin{equation*}
        \forall x,y \in A, ~ x < y \Longrightarrow f(x) < f(y) ~ (\text{respectivamente} ~ f(x) > f(y))
    \end{equation*}
    Diremos que $f$ es \textbf{monótona} si es creciente o decreciente y \textbf{estrictamente monótona}
    si es estrictamente creciente o estrictamente decreciente.
\end{definicion}

\begin{lema}
    Sea $f ~:~ [a,b] \longrightarrow \mathbb{R}$, con $a < b$, una función continua e inyectiva. Entonces, $f$ es estrictamente monótona.
\end{lema}
\begin{proof}
    Supongamos que $f(a) < f(b)$. Sean $x,y \in [a,b]$, con $x < y$. Si fuera $x = a$, usando el Lema 12.7.1 tenemos que
    $f(x) < f(y)$, e igual ocurre si $y = b$.\\
    
    Supongamos entonces que $x,y \in ~ ]a,b[$. Entonces, aplicando de nuevo el Lema \ref{lema:12.7.1}, tenemos que $f(x) < f(b)$ y aplicando de nuevo el lema a la restricción de $f$ al intervalo $[x,b]$ tenemos que $f(x) < f(y) < f(b)$. Así, hemos probado en este caso que $f$ es estrictamente creciente.\\
    
    Si fuese $f(a) > f(b)$, el razonamiento anterior, aplicado a la función $-f$ prueba que $-f$ es estrictamente creciente,
    de donde deducimos que $f$ es estrictamente decreciente, como queríamos ver.
\end{proof}

El resultado anterior se puede generalizar, de manera que es cierto para funciones continuas e inyectivas definidas en un intervalo.
\begin{teo}
    Sean $I$ un intervalo y $f ~:~ I \longrightarrow \mathbb{R}$ una función continua e inyectiva. Entonces, $f$ es estrictamente monótona.
\end{teo}
\begin{proof}
    Si $I = [a,b]$, con $a < b$ (si fuera $a = b$ no hay nada que probar), el resultado es cierto por el Lema 12.7.3.
    Sea $I$ un intervalo cualquiera y supongamos, por reducción al absurdo, que $f$ no es estrictamente monótona.
    Entonces, existen $x_1,x_2,y_1,y_2 \in I$ tales que
    \begin{equation*}
        x_1 < y_1, ~ f(x_1) < f(x_2), ~ x_2 < y_2, ~ f(x_2) > f(y_2)
    \end{equation*}
    Sean $a = \max \{x_1, x_2\}$ y $b = \max \{y_1, y_2\}$. Claramente $[a,b] \subseteq I$ y la restricción de $f$ a $[a,b]$
    es continua e inyectiva, luego es estrictamente monótona por lo ya probado. Pero ello es absurdo, pues $x_1,x_2,y_1,y_2 \in [a,b]$.
\end{proof}

Una función estrictamente monótona es siempre inyectiva. Sin embargo, una función estrictamente monótona no tiene por qué ser continua. La función $g ~:~ [0,2] \longrightarrow \mathbb{R}$ definida por
\begin{equation*}
    g(x) =
    \left\{ \begin{array}{ccl}
            x & \text{si} &0 \leq x \leq 1 \\
            1+x &\text{si} & 1 < x \leq 2
        \end{array}
    \right.
\end{equation*}
es estrictamente monótona y no es continua. Damos a continuación un teorema que garantiza la continuidad de una función monótona con una hipótesis adicional.
\begin{teo}
    Sea $A$ un conjunto de números reales no vacío y $f ~:~ A \longrightarrow \mathbb{R}$ una función monótona tal que $f(A)$ es un intervalo. Entonces, $f$ es continua.
\end{teo}
\begin{proof}
    Supongamos que $f$ es creciente. Sea $x_0$ un punto de $A$ y $\{x_n\}$ una sucesión creciente de puntos de $A$ convergente a $x_0$. Para cada natural $n$ se tiene entonces $x_n \leq x_{n+1} \leq x_0$, y por ser $f$ creciente
    \begin{equation*}
        f(x_n) \leq f(x_{n+1}) \leq f(x_0), ~ \forall n \in \mathbb{N}
    \end{equation*}
    
    Por tanto, $\{f(x_n)\}$ es creciente y mayorada, luego es convergente. Sea $L = \lim \{f(x_n)\}$, es claro que $L \leq f(x_0)$. Supongamos que $L < f(x_0)$ y sea $y = \frac{L+f(x_0)}{2}$. Entonces, tenemos que
    \begin{equation*}
        f(x_n) < y < f(x_0), ~ \forall n \in \mathbb{N}
    \end{equation*}
    
    Por ser $f(A)$ un intervalo, entonces existe $x \in A$ tal que $y = f(x)$. Si fuese $x < x_n$ para algún natural $n$, se tendría, por ser $f$ creciente, que $y = f(x) \leq f(x_n)$, cosa que no ocurre, luego
    \begin{equation*}
        x \geq x_n, ~ \forall n \in \mathbb{N}
    \end{equation*}

    Entonces $x \geq x_0$, de donde $f(x) \geq f(x_0)$, lo que es absurdo.\\

    Así, $L = f(x_0)$ y $\{f(x_n)\} \longrightarrow f(x_0)$. Un razonamiento enteramente análogo al anterior nos permite probar que si $\{x_n\}$ una sucesión decreciente de puntos de $A$ convergente a $x_0$ entonces $\{f(x_n)\} \longrightarrow f(x_0)$. Por lo tanto, para toda sucesión $\{x_n\}$ de puntos de $A$, monótona y convergente a $x_0$ se tiene que $\{f(x_n)\} \longrightarrow f(x_0)$, lo que prueba la continuidad de $f$ en $x_0$, y como $x_0$ era un punto arbitrario de $A$, $f$ es continua en $A$.\\
    
    Si $f$ fuera decreciente, $-f$ es creciente y
    \begin{equation*}
        (-f)(A) = \{-y ~:~ y \in f(A)\}
    \end{equation*}
    es, claramente, un intervalo, luego $-f$ es continua por lo ya demostrado, y en consecuencia, $f$ es continua,
    como queríamos probar.
\end{proof}
    
Si $f$ es una función inyectiva definida en un intervalo, la imagen de $f^{-1}$ es un intervalo. Este hecho evidente junto con el
siguiente lema, nos va a permitir obtener dos importantes consecuencias del teorema anterior.
\begin{lema}
    Si $f ~:~ A \longrightarrow \mathbb{R}$ es una función estrictamente creciente (respectivamente estrictamente decreciente), entonces $f^{-1} ~:~ f(A) \longrightarrow \mathbb{R}$ es estrictamente creciente (respectivamente estrictamente decreciente).
\end{lema}
\begin{proof}
    Sean $x,y \in f(A)$ con $x < y$; si fuese $f^{-1}(x) \geq f^{-1}(y)$, al ser $f$ creciente, tendríamos que
    \begin{equation*}
        f(f^{-1}(x)) \geq f(f^{-1}(y)) \Longleftrightarrow x \geq y
    \end{equation*}
    por tanto, ha de ser $f^{-1}(x) < f^{-1}(y)$ y $f^{-1}$ es estrictamente creciente.

    Análogo razonamiento para el caso en el que $f$ sea estrictamente decreciente.
\end{proof}

%########################################################################################################
% Continuidad uniforme.
%########################################################################################################

\section{Continuidad uniforme}
Sea $A$ un conjunto no vacío de números reales y $f ~:~ A \longrightarrow \mathbb{R}$ una función.
Según el Teorema \ref{teo:12.4.1}, el hecho de que $f$ sea continua en $A$ se expresa de la siguiente forma:
\begin{equation*}
    \forall x \in A, ~ \forall \varepsilon > 0, ~ \exists \delta > 0 ~:~ x \in A, ~ |x-x_0| < \delta \Longrightarrow |f(x)-f(x_0)| < \varepsilon
\end{equation*}
Es importante observar que el número $\delta$ que aparece en el enunciado anterior depende tanto del número $\varepsilon$ escogido, como del punto $x$ de $A$ prefijado. En general, para un mismo $\varepsilon$ pueden aparecer distintos $\delta$ según el punto $x$ de que se trate. Tiene especial interés el caso en que, fijado un $\varepsilon > 0$ puede encontrarse un $\delta$ común, válido para todos los puntos $x$ del conjunto $A$.

\begin{definicion}
    Sea $f ~:~ A \longrightarrow \mathbb{R}$ una función real de variable real. Diremos que $f$ es \textbf{uniformemente continua}\footnote{La continuidad uniforme no es algo que suela estudiarse en un curso de Cálculo I.
    Sin embargo, en estos apuntes vamos a ver una corta introducción, ya que es un concepto que será necesario cuando se estudie la integración de funciones reales de variable real acotadas, y en particular, de funciones continuas, definidas en intervalos cerrados y acotados en la asignatura de Cálculo II.}
    si
    \begin{equation*}
        \forall \varepsilon > 0, ~ \exists \delta > 0 ~:~ x,y \in A, ~ |x-y| < \delta \Longrightarrow |f(x)-f(y)| < \varepsilon
    \end{equation*}
    Si $B$ es un subconjunto no vacío de $A$, diremos que $f$ es uniformemente continua en $B$ cuando la restricción de $f$ a $B$
    sea uniformemente continua.
\end{definicion}

A título de ejemplo, si $A$ es cualquier conjunto no vacío de números reales, la función $f ~:~ A \longrightarrow \mathbb{R}$
dada por
\begin{equation*}
    f(x) = x, ~ \forall x \in A
\end{equation*}
es uniformemente continua (tómese $\delta = \varepsilon$). Es claro que, si $f$ es uniformemente continua, entonces $f$ es continua en $A$. El recíproco no es cierto.\\

Para verlo, sea $f ~:~ \mathbb{R} \longrightarrow \mathbb{R}$ definida por $f(x) = x^2$, $\forall x \in \mathbb{R}$.
Supongamos $f$ fuese uniformemente continua. Entonces
\begin{equation*}
    \exists \delta_1 > 0 ~:~ x,y \in A,~ |x-y| < \delta_1 \Longrightarrow |x^2 - y^2| < 1
\end{equation*}
Tomando $x = \frac{2}{\delta_1} + \frac{\delta_1}{4}$, $y = \frac{2}{\delta_1} - \frac{\delta_1}{4}$, tenemos
$|x-y| < \delta_1$ y $1 > |x^2 - y^2| = 2$, lo que es, evidentemente, absurdo.\\

El ejemplo anterior muestra que, a diferencia de lo que pasaba con la continuidad, puede ocurrir que el producto de funciones
uniformemente continuas no sea uniformemente continua.\\

El resultado más importante relativo a la continuidad uniforme de funciones reales de variable real es, sin duda, el siguiente.
\begin{teo}[de Heine]
    Toda función continua en un intervalo cerrado y acotado es uniformemente continua en dicho intervalo.
\end{teo}
\begin{proof}
    Sea $f ~:~ [a,b] \longrightarrow \mathbb{R}$ continua. Por reducción al absurdo, supongamos que $f$ no es uniformemente continua.
    Entonces:
    \begin{equation*}
        \exists \varepsilon_0 > 0 ~:~ \forall \delta > 0, ~ \exists x,y \in [a,b] ~:~ |x-y| < \delta ~ \text{y} ~ |f(x)-f(y)| \geq \varepsilon_0
    \end{equation*}
    En particular, tomando un natural $n$ arbitrario, encontramos $x_n, y_n \in [a,b]$ tales que
    \begin{equation}\label{eq:Heine}
        |x_n - y_n| < \frac{1}{n} ~ \text{y} ~ |f(x_n)-f(y_n)| \geq \varepsilon_0 ~~ (*)
    \end{equation}
    Entonces, obtenemos dos sucesiones $\{x_n\}$, $\{y_n\}$ tales que la Ec. \ref{eq:Heine} es cierta para todo $n$ natural. Aplicando el
    teorema de Bolzano-Weierstrass, la sucesión acotada $\{x_n\}$ admite una parcial $\{x_{\sigma(n)}\}$ convergente
    a un número real $x \in [a,b]$. Teniendo en cuenta que
    \begin{equation*}
        \{y_{\sigma(n)} - x_{\sigma(n)}\} \longrightarrow 0
    \end{equation*}
    y que
    \begin{equation*}
        \{y_{\sigma(n)}\} = \{y_{\sigma(n)} - x_{\sigma(n)} + x_{\sigma(n)}\}
    \end{equation*}
    deducimos que $\{y_{\sigma(n)}\} \longrightarrow x$. Por ser $f$ continua en el punto $x$, las sucesiones
    $\{f(x_{\sigma(n)})\}$ y $\{f(y_{\sigma(n)})\}$ convergen a $f(x)$, y por tanto
    \begin{gather*}
        \exists m_1 \in \mathbb{N} ~:~ n \in \mathbb{N}, ~ n \geq m_1 \Longrightarrow |f(x_{\sigma(n)})-f(x)| < \frac{\varepsilon_0}{2}\\
        \exists m_2 \in \mathbb{N} ~:~ n \in \mathbb{N}, ~ n \geq m_2 \Longrightarrow |f(y_{\sigma(n)})-f(x)| < \frac{\varepsilon_0}{2}
    \end{gather*}
    Tomando entonces $m = \max \{m_1,m_2\}$ y $n = \sigma(n)$ tenemos:
    \begin{equation*}
        |f(x_{\sigma(n)})-f(y_{\sigma(n)})| \leq |f(x_{\sigma(n)})-f(x)| + |f(x)-f(y_{\sigma(n)})| < \varepsilon_0
    \end{equation*}
    lo cual es una contradicción, pues se tenía
    \begin{equation*}
        |f(x_n)-f(y_n)| \geq \varepsilon_0, ~ \forall n \in \mathbb{N}
    \end{equation*}
\end{proof}


%########################################################################################################
% Ejercicios de Funciones reales de variable real y continuidad.
%########################################################################################################

\section{Ejercicios}
\begin{ejercicio}
    Estudiar la continuidad de la función $f ~:~ \mathbb{R} \longrightarrow \mathbb{R}$ dada por
    \begin{equation*}
        f(x) =
        \left\{ \begin{array}{ccl}
            x & \text{si} & x \in \mathbb{Q} \\
            1-x & \text{si} & x \in \mathbb{R} - \mathbb{Q}
            \end{array}
        \right.
    \end{equation*}
\end{ejercicio}

\begin{ejercicio}
    Sean $f,g ~:~ \mathbb{R} \longrightarrow \mathbb{R}$, continuas en todo $\mathbb{R}$. Supongamos que
    \begin{equation*}
        f(x) = g(x), ~ \forall x \in \mathbb{Q}
    \end{equation*}
    Probar que $f = g$. En particular, si $f ~:~ \mathbb{R} \longrightarrow \mathbb{R}$ es continua en $\mathbb{R}$ y
    $f_{|_{\mathbb{Q}}}$ constante, entonces $f$ es constante.
\end{ejercicio}

\begin{ejercicio}
    Sea $A$ un conjunto no vacío de números reales. Sea $f ~:~ \mathbb{R} \longrightarrow \mathbb{R}$
    definida por
    \begin{equation*}
        f(x) = \inf \{|x-a| ~:~ a \in A\}, ~ \forall x \in A
    \end{equation*}
    Probar que $|f(x)-f(y)| \leq |x-y|, ~ \forall x,y \in \mathbb{R}$. Deducir que $f$ es continua en todo $\mathbb{R}$.
\end{ejercicio}

\begin{ejercicio}
    Probar que toda función de $\mathbb{N}$ en $\mathbb{R}$ es continua en $\mathbb{N}$ (Las sucesiones de números reales son funciones continuas).
\end{ejercicio}

\begin{ejercicio}
    Probar que si $A$ es un conjunto finito no vacío de números reales, toda función real definida en $A$ es continua en $A$.
\end{ejercicio}

\begin{ejercicio}
    Consideremos el siguiente conjunto de números reales:
    \begin{equation*}
        A = \{0\} \cup \left\{\frac{1}{n} ~:~ n \in \mathbb{N}\right\}
    \end{equation*}
    Probar que toda función real definida en $A$ es continua en $\left\{\frac{1}{n} ~:~ n \in \mathbb{N}\right\}$.
    Dar un ejemplo de una función real definida en $A$ que no sea continua en $0$.
\end{ejercicio}

\begin{ejercicio}
    Probar, utilizando la caracterización $\varepsilon \text{-} \delta$ de la continuidad, que la función
    $f ~:~ \mathbb{R} \longrightarrow \mathbb{R}$ dada por
    \begin{equation*}
        f(x) = x^2, ~ \forall x \in \mathbb{R}
    \end{equation*}
    es continua en todo $\mathbb{R}$.
\end{ejercicio}

\begin{ejercicio}
    Dar un ejemplo de una función continua en un punto $x_0$ que no tenga signo constante en ningún intervalo abierto centrado en dicho punto (intervalo de la forma $]x_0-\delta,x_0+\delta[$ con $\delta > 0$).
\end{ejercicio}

\begin{ejercicio}
    Dar un ejemplo de una función continua cuya imagen no sea un intervalo.
\end{ejercicio}

\begin{ejercicio}
    Dar un ejemplo de una función definida en un intervalo, cuya imagen sea un intervalo y que no sea continua.
\end{ejercicio}

\begin{ejercicio}
    Dar un ejemplo de una función continua en todo $\mathbb{R}$, no constante y cuya imagen sea un conjunto (obligadamente un intervalo) acotado.
\end{ejercicio}

\begin{ejercicio}
    Pruébese que si $I$ es un intervalo y $f ~:~ I \longrightarrow \mathbb{R}$ es una función continua en $I$ tal que $f(I) \subseteq \mathbb{Q}$, entonces $f$ es constante.
\end{ejercicio}

\begin{ejercicio}
    Probar que todo polinomio de grado impar admite al menos una raíz real.
\end{ejercicio}

\begin{ejercicio}
    Dado $a > 0$, probar que existe $x \in \mathbb{R^+}$ tal que $x^2 = a$. El tal $x$ es único.
\end{ejercicio}

\begin{ejercicio}
    Sea $f ~:~ [0,1] \longrightarrow [0,1]$ una función continua en $[0,1]$. Pruébese que $f$ tiene un punto fijo (Es decir, $\exists x \in [0,1] ~:~ f(x) = x$).
\end{ejercicio}

\begin{ejercicio}
    Probar que la ecuación $e^x+x^3-6x = 2$ tiene, al menos, $3$ soluciones en el intervalo $[-3,3]$.
\end{ejercicio}

\begin{ejercicio}
    Dada $f ~:~ [0,1] \longrightarrow \mathbb{R}$ continua tal que $f(0) = f(1)$, y dado $n\in \mathbb{N}$ fijo, probar que $\exists c_n \in [0,1] ~:~ f(c_n) = f(c_n + \frac{1}{n})$.
\end{ejercicio}

\begin{ejercicio}
    Sea $f ~:~ [a,b] \longrightarrow \mathbb{R}$ continua tal que
    \begin{equation*}
        \forall x \in [a,b], ~ \exists y \in [a,b] ~:~ |f(y)| \leq \frac{1}{2} |f(x)|
    \end{equation*}
    Probar que $\exists c \in [a,b] ~:~ f(c) = 0$.
\end{ejercicio}

\begin{ejercicio}
    Suponiendo que la temperatura varía de manera continua a lo largo del Ecuador, pruébese que, en cualquier instante, existen dos puntos antípodas sobre el Ecuador que se hallan a la misma temperatura.
\end{ejercicio}

\begin{ejercicio}
    Sea $f ~:~ ]0,1[ \longrightarrow \mathbb{R}$ y $g ~:~ \mathbb{R} \longrightarrow \mathbb{R}$ dadas por:
    \begin{equation*}
        f(x) = x, ~ \forall x \in ~ ]0,1[
    \end{equation*}
    \begin{equation*}
        g(x) =
        \left\{ \begin{array}{ccl}
            \frac{x}{1+x} & \text{si} & x \in \mathbb{R}^+_0 \\
            \frac{x}{1-x} & \text{si} & x \in \mathbb{R}^-
            \end{array}
        \right.
    \end{equation*}
    Comprobar que $f$ y $g$ son continuas y acotadas pero no tienen máximo ni mínimo absolutos.
\end{ejercicio}

\begin{ejercicio}
    Sea $f ~:~ [-1,1] \longrightarrow \mathbb{R}$ definida por
    \begin{equation*}
        f(x) = \frac{x^2}{1+x^2}, ~ \forall x \in [-1,1]
    \end{equation*}
    Determinar la imagen de $f$.
\end{ejercicio}

\begin{ejercicio}
    Sea $f ~:~ [-1,1] \longrightarrow \mathbb{R}$ definida por
    \begin{equation*}
        f(x) = \frac{2x}{1+|x|}, ~ \forall x \in [-1,1]
    \end{equation*}
    Determinar la imagen de $f$.
\end{ejercicio}

\begin{ejercicio}
    Sea $I$ un intervalo y $f ~:~ I \longrightarrow \mathbb{R}$ una función inyectiva.

    Analícese la relación entre las siguientes afirmaciones:
    \begin{enumerate}
        \item $f$ es continua.
        \item $f(I)$ es un intervalo.
        \item $f$ es estrictamente monótona.
        \item $f^{-1}$ es continua.
    \end{enumerate}
\end{ejercicio}

\begin{ejercicio}
    Pruébese que si $A$ es un conjunto finito no vacío de números reales, toda función de $A$ en $\mathbb{R}$ es uniformemente continua.
\end{ejercicio}

\begin{ejercicio}
    Probar que la suma de funciones uniformemente continuas es uniformemente continua.
\end{ejercicio}

\begin{ejercicio}
    Pruébese que si $f,g$ son funciones uniformemente continuas y acotadas, entonces $fg$ es uniformemente continua.
\end{ejercicio}

\begin{ejercicio}
    Sea $f ~:~ A \longrightarrow \mathbb{R}$. Supongamos que existe $K > 0$ tal que
    \begin{equation*}
        |f(x)-f(y)| \leq K |x-y|, ~ \forall x,y \in A
    \end{equation*}
    Probar que $f$ es uniformemente continua\footnote{Las funciones que cumplen la propiedad de arriba se dicen \textbf{lipschitzianas}.
    Se verán más detalle en Cálculo II, cuando se analice su relación con la derivabilidad de una función.}.
\end{ejercicio}

\begin{ejercicio}
    Compruébese que la función $f ~:~ \mathbb{R}^+ \longrightarrow \mathbb{R}$ dada por
    \begin{equation*}
        f(x) = \frac{1}{x}, ~ \forall x \in \mathbb{R}^+
    \end{equation*}
    no es uniformemente continua.
\end{ejercicio}

\begin{ejercicio}
    Sea $A$ un conjunto no vacío de números reales y $f ~:~ A \longrightarrow \mathbb{R}$ una función uniformemente continua.
    
    Probar que si $\{x_n\}$ es una sucesión convergente de puntos de $A$ (su límite no tiene por qué pertenecer al conjunto $A$), 
    entonces la sucesión $\{f(x_n)\}$ es convergente.
    
    Probar que si $\{x_n\}$ e $\{y_n\}$ son sucesiones de elementos de $A$ convergentes al mismo límite, entonces $\{f(x_n)\}$ y $\{f(y_n)\}$ convergen y tienen el mismo límite.
\end{ejercicio}

\begin{ejercicio}
    Sea $f ~:~ \mathbb{Q} \longrightarrow \mathbb{R}$ una función uniformemente continua. Pruébese que existe una función
    $g ~:~ \mathbb{R} \longrightarrow \mathbb{R}$, uniformemente continua, cuya restricción a $\mathbb{Q}$ coincide con $f$.
\end{ejercicio}

\begin{ejercicio}
    Sean $a,b \in \mathbb{R}$ con $a < b$ y sea $f ~:~ ]a,b[ \longrightarrow \mathbb{R}$ una función.
    Probar que $f$ es uniformemente continua si, y sólo si, $f$ es la restricción al intervalo $]a,b[$ de una función continua en el intervalo $[a,b]$.
\end{ejercicio}
\chapter{Criterios de convergencia de series de términos no negativos}\label{chp:Tema10}

En este tema trataremos el estudio de las series de números reales de términos no negativos. Por ello, si $\displaystyle\sum_{n \geq 1} a_n$ es una serie, consideraremos que $a_n \geq 0$ para todo $n$ natural.

%########################################################################################################
% Criterios de convergencia.
%########################################################################################################

\section{Criterios de convergencia}
Notemos que al ser $a_n \geq 0, ~ \forall n \in \mathbb{N}$, la sucesión $\{A_n\}$ es creciente, por lo que:
\begin{prop}[Criterio básico de convergencia]
    Si $a_n \geq 0, ~ \forall n \in \mathbb{N}$, entonces
    \begin{equation*}
        \displaystyle\sum_{n \geq 1} a_n ~ \text{converge} \Longleftrightarrow  \displaystyle\sum_{n \geq 1} a_n ~ \text{está mayorada}
    \end{equation*}
\end{prop}

\begin{prop}[Criterio de comparación]
    Sean $\{a_n\}$ y $\{b_n\}$ sucesiones de números reales con $a_n \geq 0,~b_n > 0 ~ \forall n \in \mathbb{N}$.
    Supongamos que
    \begin{equation*}
        \exists m \in \mathbb{N} ~:~ n \in \mathbb{N}, ~ n \geq m \Longrightarrow a_n \leq b_n
    \end{equation*}
    Entonces
    \begin{equation*}
        \text{Si} ~\displaystyle\sum_{n \geq 1} b_n ~ \text{converge, entonces} ~ \displaystyle\sum_{n \geq 1} a_n ~ \text{converge}
    \end{equation*}
\end{prop}
\begin{proof}
    Supongamos que $\displaystyle\sum_{n \geq 1} b_n$ converge. Usando que
    \begin{equation*}
        \displaystyle\sum_{n \geq 1} a_n ~ \text{converge} \Longleftrightarrow \displaystyle\sum_{n \geq m} a_n ~ \text{converge},
    \end{equation*}
    deducimos que
    \begin{equation*}
        \displaystyle\sum_{n \geq m} a_n \leq \displaystyle\sum_{n \geq m} b_n \leq \displaystyle\sum_{n = 1}^{\infty} b_n \in \mathbb{R},
    \end{equation*}
    por lo que $\displaystyle\sum_{n \geq 1} a_n$ está mayorada, y en consecuencia, es convergente.
\end{proof}

\begin{coro}[Criterio límite de comparación]
    Sean $\{a_n\}$ y $\{b_n\}$ sucesiones de números reales con $a_n, b_n \geq 0, ~ \forall n \in \mathbb{N}$. Consideramos la sucesión $\left\{ \frac{a_n}{b_n} \right\}$.
    \begin{itemize}
        \item Si $\left\{ \dfrac{a_n}{b_n} \right\} \longrightarrow 0$, entonces
        \begin{equation*}
            \text{Si} ~\displaystyle\sum_{n \geq 1} b_n ~ \text{converge, entonces} ~ \displaystyle\sum_{n \geq 1} a_n ~ \text{converge}
        \end{equation*}

        \item Si $\left\{ \frac{a_n}{b_n} \right\} \longrightarrow +\infty$, entonces
        \begin{equation*}
            \text{Si} ~\displaystyle\sum_{n \geq 1} a_n ~ \text{converge, entonces} ~ \displaystyle\sum_{n \geq 1} b_n ~ \text{converge}
        \end{equation*}

        \item Si $\left\{ \frac{a_n}{b_n} \right\} \longrightarrow L > 0$, entonces
        \begin{equation*}
            \displaystyle\sum_{n \geq 1} a_n ~ \text{converge} \Longleftrightarrow  \displaystyle\sum_{n \geq 1} b_n ~ \text{converge}
        \end{equation*}
    \end{itemize}
\end{coro}
\begin{proof} Realizamos la distinción de casos:
    \begin{itemize}
        \item Supongamos que $\left\{ \frac{a_n}{b_n} \right\} \longrightarrow 0$. Entonces
        \begin{equation*}
            \exists m \in \mathbb{N} ~:~ n \geq m \Longrightarrow \frac{a_n}{b_n} < 1 \Longleftrightarrow 0 \leq a_n < b_n
        \end{equation*}
        y basta con aplicar el criterio de comparación.

        \item Supongamos que $\left\{ \frac{a_n}{b_n} \right\} \longrightarrow +\infty$. Entonces
        \begin{equation*}
            \exists m \in \mathbb{N} ~:~ n \geq m \Longrightarrow \frac{a_n}{b_n} > 1 \Longrightarrow a_n > b_n > 0
        \end{equation*}
        y basta con aplicar el criterio de comparación.

        \item Supongamos que $\left\{ \frac{a_n}{b_n} \right\} \longrightarrow L > 0$. Sea $\varepsilon = \frac{L}{2}$.
        
        Entonces, por convergencia de $\left\{ \frac{a_n}{b_n} \right\}$, tenemos que
        \begin{equation*}
            \exists m \in \mathbb{N} ~:~ n \geq m \Longrightarrow \frac{L}{2} < \frac{a_n}{b_n} < \frac{3L}{2} \Longleftrightarrow \frac{L}{2} b_n < a_n < \frac{3L}{2} b_n.
        \end{equation*}
        Teniendo en cuenta que (¿Por qué?)
        \begin{equation*}
            \displaystyle\sum_{n \geq 1} b_n ~ \text{converge} \Longleftrightarrow \displaystyle\sum_{n \geq 1} \frac{3L}{2} b_n ~ \text{converge} \Longleftrightarrow \displaystyle\sum_{n \geq 1} \frac{L}{2} b_n ~ \text{converge},
        \end{equation*}
        aplicando convenientemente el criterio de comparación, tenemos que
        \begin{equation*}
            \displaystyle\sum_{n \geq 1} b_n ~ \text{converge} \Longleftrightarrow \displaystyle\sum_{n \geq 1} \frac{3L}{2} b_n ~ \text{converge} \Longrightarrow \displaystyle\sum_{n \geq 1} a_n ~ \text{converge}
        \end{equation*}
        y también que
        \begin{equation*}
            \displaystyle\sum_{n \geq 1} a_n ~ \text{converge} \Longrightarrow \displaystyle\sum_{n \geq 1} \frac{L}{2} b_n ~ \text{converge} \Longleftrightarrow \displaystyle\sum_{n \geq 1} b_n ~ \text{converge}
        \end{equation*}
        por lo que
        \begin{equation*}
            \displaystyle\sum_{n \geq 1} a_n ~ \text{converge} \Longleftrightarrow  \displaystyle\sum_{n \geq 1} b_n ~ \text{converge}
        \end{equation*}
        como queríamos ver.
    \end{itemize}
\end{proof}

\begin{prop}[Criterio de condensación]
    Sea $\{a_n\}$ una sucesión de números reales decreciente con $a_n \geq 0, ~ \forall n \in \mathbb{N}$.
    \begin{equation*}
        \displaystyle\sum_{n \geq 1} a_n ~ \text{converge} \Longleftrightarrow \displaystyle\sum_{n \geq 1} 2^n a_{2^n} ~ \text{converge}
    \end{equation*}
\end{prop}
\begin{proof}
    Sean $\{A_n\}$ y $\{B_n\}$ las sucesiones de sumas parciales de $\displaystyle\sum_{n \geq 1} a_n$ y $\displaystyle\sum_{n \geq 1} 2^n a_{2^n}$.
    
    Usando que $\{a_n\}$ es decreciente, tenemos que
    \begin{equation*}
        A_n \leq A_{2^{n+1}-1}, ~ \forall n \in \mathbb{N}
    \end{equation*}
    y entonces
    \begin{multline*}
       a_1+(a_2+a_3)+(a_4+\dots+a_7)+\dots+(a_{2^n}+a_{2^n+1}+\dots+a_{2^{n+1}-1})
       \leq \\ \leq
       a_1 + 2a_2+4a_4+\dots+2^n a_{2^n} = a_1+B_n,
    \end{multline*}
    por lo que si $\{B_n\}$ está mayorada, $\{A_n\}$ también.
    
    Recíprocamente, para cada $n$ natural tenemos que
    \begin{multline*}
        \frac{1}{2} B_n = a_2 +2a_4 +4a_8+\dots+2^{n-1}a_{2^n}
        \leq \\ \leq
        (a_1+a_2)+(a_3+a_4)+(a_5+\dots+a_8)+\dots+(a_{2^{n-1}+1}+\dots+a_{2^n})
        = A_{2^n},
    \end{multline*}
    por lo que si $\{A_n\}$ está mayorada, $\{B_n\}$ también.
\end{proof}

Notemos que ya habíamos usado indirectamente el criterio de condensación para demostrar la no convergencia de la serie armónica. A continuación, veamos un ejemplo de uso del criterio de condensación para estudiar la convergencia de una importante familia de series de números reales relacionada con la serie armónica.
\begin{ejemplo}
    Sea $\alpha \in \mathbb{R}$. Consideramos la serie
    \begin{equation*}
        \displaystyle\sum_{n \geq 1} \frac{1}{n^{\alpha}}
    \end{equation*}
    
    Si $\alpha \leq 0$, dicha serie no converge, pues $\left\{\frac{1}{n^{\alpha}}\right\}$ no tiende a $0$.
    
    Si $\alpha > 0$, tenemos que $\left\{\frac{1}{n^{\alpha}}\right\} \longrightarrow  0$ y es estrictamente decreciente, por lo que podemos
    aplicar el criterio de condensación:
    \begin{equation*}
        \displaystyle\sum_{n \geq 1} 2^n a_{2^n} = \displaystyle\sum_{n \geq 1} \frac{2^n}{(2^n)^{\alpha}} = \displaystyle\sum_{n \geq 1} (2^{1-\alpha})^n
    \end{equation*}
    que es una serie geométrica con $r = 2^{1-\alpha}$, por lo que
    \begin{equation*}
        \displaystyle\sum_{n \geq 1} 2^n a_{2^n} ~ \text{converge} \Longleftrightarrow 2^{1-\alpha} < 1 \Longleftrightarrow \alpha > 1
    \end{equation*}
    y por el criterio de condensación:
    \begin{equation*}
        \displaystyle\sum_{n \geq 1} \frac{1}{n^{\alpha}} ~ \text{converge} \Longleftrightarrow \alpha > 1
    \end{equation*}
\end{ejemplo}

Las series de la forma $\displaystyle\sum_{n \geq 1} \frac{1}{n^{\alpha}}$ se llaman \textbf{series de Riemman}\footnote{En otros textos,
son llamadas series armónicas de exponente $\alpha$.}.

\begin{prop}[Criterio de la raíz $n$-ésima]
    Sea $\{a_n\}$ una sucesión de números reales tales que $a_n \geq 0, ~ \forall n \in \mathbb{N}$.
    \begin{itemize}
        \item Si $\left\{\sqrt[n]{a_n}\right\} \longrightarrow L < 1$, entonces $\displaystyle\sum_{n \geq 1} a_n$ converge.

        \item Si $\left\{\sqrt[n]{a_n}\right\} \longrightarrow L > 1$ o $\left\{\sqrt[n]{a_n}\right\} \longrightarrow + \infty$, entonces $\displaystyle\sum_{n \geq 1} a_n$ no converge.
    \end{itemize}
\end{prop}
\begin{proof} Realizamos la misma distinción de casos para la demostración:
\begin{itemize}
    \item Supongamos que $\left\{\sqrt[n]{a_n}\right\} \longrightarrow L < 1$. Sea $r \in \mathbb{R}$ con $L < r < 1$.
    Por convergencia de  $\left\{\sqrt[n]{a_n}\right\}$, sabemos que
    \begin{equation*}
        \exists m \in \mathbb{N} ~:~ n \geq m \Longrightarrow \sqrt[n]{a_n} < r \Longleftrightarrow a_n < r^n
    \end{equation*}
    Sabemos que la serie $\displaystyle\sum_{n \geq} r^n$ converge por ser $ r < 1$, por lo que aplicando el criterio de comparación,
    la serie $\displaystyle\sum_{n \geq 1} a_n$ converge.

    \item Supongamos que $\left\{\sqrt[n]{a_n}\right\} \longrightarrow L > 1$ o $\left\{\sqrt[n]{a_n}\right\} \longrightarrow + \infty$.
    Sea $r \in \mathbb{R}$ con  $1 < r < L$, en ambos casos tenemos que
    \begin{equation*}
        \exists m \in \mathbb{N} ~:~ n \geq m \Longrightarrow \sqrt[n]{a_n} > r \Longleftrightarrow a_n > r^n
    \end{equation*}
    Sabemos que la serie $\displaystyle\sum_{n \geq} r^n$ no converge por ser $ r > 1$, por lo que aplicando el criterio de comparación,
    la serie $\displaystyle\sum_{n \geq 1} a_n$ no converge. \qedhere
\end{itemize}
    
\end{proof}

El siguiente criterio de convergencia es una consecuencia inmediata de la proposición anterior.
\begin{coro}[Criterio del cociente para series]
    Sea $\{a_n\}$ una sucesión de números reales tales que $a_n > 0, ~ \forall n \in \mathbb{N}$. Supongamos que $\left\{ \dfrac{a_{n+1}}{a_n}\right\} \longrightarrow L \in \mathbb{R}^+_0 \cup \{+ \infty\}$. Entonces:
    \begin{itemize}
        \item Si $L < 1$, entonces $\displaystyle\sum_{n \geq 1} a_n$ converge.

        \item Si $L > 1$, o $L = + \infty$, entonces $\displaystyle\sum_{n \geq 1} a_n$ no converge.
    \end{itemize}
\end{coro}
\begin{proof}
    Aplicando el criterio del cociente para sucesiones, tenemos que $\left\{\sqrt[n]{a_n}\right\} \longrightarrow L$ y basta con aplicar la proposición anterior en cada caso.
\end{proof}

El último criterio de convergencia que veremos en este tema suele resultar especialmente útil cuando $L = 1$ al tratar de aplicar el criterio del cociente.

\begin{prop}[Criterio de Raabe]
    Sea $\{a_n\}$ una sucesión de números reales tal que $a_n > 0, ~ \forall n \in \mathbb{N}$.
    Supongamos que
    \begin{equation*}
        \left\{ n\left(\frac{a_{n+1}}{a_n}-1\right) \right\} \longrightarrow \alpha,
    \end{equation*}
    o equivalentemente
    \begin{equation*}
        \left\{ \left(\frac{a_{n+1}}{a_n}\right)^n \right\} \longrightarrow e^\alpha,
    \end{equation*}
    con $\alpha \in \mathbb{R} \cup \{-\infty,+\infty\}$.

    \begin{itemize}
        \item Si $\alpha < -1$ ($e^\alpha < \frac{1}{e}$, equivalentemente), entonces $\displaystyle\sum_{n \geq 1} a_n$ converge.

        \item Si $\alpha > -1$ ($e^\alpha > \frac{1}{e}$, equivalentemente), entonces $\displaystyle\sum_{n \geq 1} a_n$ no converge.
    \end{itemize}
\end{prop}

\begin{ejemplo}
    Consideramos la serie $\displaystyle\sum_{n \geq 1} \frac{1}{n^2}$.
    Si tratamos de aplicar el criterio del cociente, vemos que
    \begin{equation*}
        \left\{ \frac{\frac{1}{(n+1)^2}}{\frac{1}{n^2}} \right\} = \left\{ \frac{n^2}{(n+1)^2} \right\} \longrightarrow 1,
    \end{equation*}
    por lo que el criterio del cociente no nos dice nada. Probemos a aplicar el criterio de Raabe:
    \begin{equation*}
        \left\{ n \left(\frac{\frac{1}{(n+1)^2}}{\frac{1}{n^2}} -1\right) \right\} = \left\{ n\left(\frac{n^2}{(n+1)^2}-1\right) \right\} = \left\{ \frac{-2n^2-n}{(n+1)^2} \right\} \longrightarrow -2
    \end{equation*}
    Por lo tanto, $\displaystyle\sum_{n \geq 1} \frac{1}{n^2}$ converge en virtud del criterio de Raabe.
\end{ejemplo}

%########################################################################################################
% Ejercicios de Criterios de convergencia de series de términos no negativos.
%########################################################################################################

\section{Ejercicios}

\renewcommand{\theenumi}{\alph{enumi}} %Números romanos en minúscula para las enumeraciones

\begin{ejercicio}
    Estudiar la convergencia de las siguientes series:
    \begin{enumerate}
        \item $\displaystyle\sum_{n \geq 1} \left( \frac{n}{7n+3} \right)^{2n+1}$

        \item $\displaystyle\sum_{n \geq 1} \frac{2n^2}{2^n+3}$

        \item $\displaystyle\sum_{n \geq 1} \frac{(2n)!}{(n!)^2 (2n+1) 2^{2n}}$

        \item $\displaystyle\sum_{n \geq 1} \frac{1}{n 2^n}$

        \item $\displaystyle\sum_{n \geq 1} \frac{1}{\sqrt{n(n+1)}}$

        \item $\displaystyle\sum_{n \geq 1} \frac{1}{2^n-n}$

        \item $\displaystyle\sum_{n \geq 1} \left( \frac{n+1}{3n-1}\right)^n$

        \item $\displaystyle\sum_{n \geq 1} \frac{1}{(2n-1)2n}$

        \item $\displaystyle\sum_{n \geq 1} \left( \frac{n}{3n-2} \right)^{2n-1}$

        \item $\displaystyle\sum_{n \geq 1} \frac{1}{n} \left(\frac{2}{5}\right)^n$

        \item $\displaystyle\sum_{n \geq 1} \frac{n}{2^n}$

        \item $\displaystyle\sum_{n \geq 1} \frac{(n+1)^n}{3^n n!}$

        \item $\displaystyle\sum_{n \geq 1} \sqrt{\frac{(n+2)!}{5 \cdot 6 \cdot 7 \cdots (n+5)}}$

        \item $\displaystyle\sum_{n \geq 1} \frac{n^2}{(3n-1)^2}$

        \item $\displaystyle \sum_{n \geq 1} \frac{\sqrt[3]{n}}{(n+1) \sqrt{n}}$

        \item $\displaystyle\sum_{n \geq 1} \frac{3n-1}{(\sqrt{2})^n}$

        \item $\displaystyle\sum_{n \geq 1} \frac{2 \cdot 5 \cdot 8 \cdots (3n-1)}{1 \cdot 5 \cdot 9 \cdots (4n-3)}$

        \item $\displaystyle\sum_{n \geq 1} \frac{1}{n!}$

        \item $\displaystyle\sum_{n \geq 1} \frac{2n+1}{(n+1)^2 (n+2)^2}$

        \item $\displaystyle\sum_{n \geq 1} \left( \frac{3n}{3n+1}\right)^n$

        \item $\displaystyle\sum_{n \geq 1} \frac{1}{3^{\sqrt{n}}}$

        \item $\displaystyle\sum_{n \geq 1} \frac{n^3}{e^n}$

        \item $\displaystyle\sum_{n \geq 1} \left( \frac{2n+1}{3n+1} \right)^{\frac{n}{2}}$

        \item $\displaystyle\sum_{n \geq 1} \frac{2^n n!}{n^n}$

        \item $\displaystyle\sum_{n \geq 1} \frac{1 \cdot 3 \cdot 5 \cdots (2n-1)}{2 \cdot 4 \cdot 6 \cdots (2n+2)}$

        \item $\displaystyle\sum_{n \geq 1} \left( \frac{n+1}{n^2} \right)^{n-1}$
    \end{enumerate}
\end{ejercicio}
\renewcommand{\theenumi}{\roman{enumi}} %Números romanos en minúscula para las enumeraciones

\begin{ejercicio}[Debilitación del criterio límite de comparación]
    Sean $\{a_n\}$, $\{b_n\}$ sucesiones de números reales con $a_n,b_n \geq 0, ~ \forall n \in \mathbb{N}$.
    \begin{gather*}
        \text{Si} ~ \liminf \left\{ \frac{a_n}{b_n} \right\} = l > 0 \Longrightarrow \left( \displaystyle\sum_{n \geq 1} a_n ~ \text{converge} \Longrightarrow \displaystyle\sum_{n \geq 1} b_n ~ \text{converge} \right)\\
        \text{Si} ~ \limsup \left\{ \frac{a_n}{b_n} \right\} = L < + \infty \Longrightarrow \left( \displaystyle\sum_{n \geq 1} b_n ~ \text{converge} \Longrightarrow \displaystyle\sum_{n \geq 1} a_n ~ \text{converge} \right)
    \end{gather*}
\end{ejercicio}

\begin{ejercicio}[Debilitación del criterio de la raíz $n$-ésima]
    Sea $\{a_n\}$ una sucesión de números reales con $a_n \geq 0, ~ \forall n \in \mathbb{N}$.
    \begin{gather*}
        \limsup \left\{ \sqrt[n]{a_n} \right\} = L < 1 \Longrightarrow \displaystyle\sum_{n \geq 1} a_n ~ \text{converge}\\
        \liminf \left\{ \sqrt[n]{a_n} \right\} = l > 1 \Longrightarrow \displaystyle\sum_{n \geq 1} a_n ~ \text{no converge}
    \end{gather*}
\end{ejercicio}

\begin{ejercicio}[Debilitación del criterio del cociente para series]
    Sea $\{a_n\}$ una sucesión de números reales con $a_n \geq 0, ~ \forall n \in \mathbb{N}$.
    \begin{gather*}
        \limsup \left\{ \frac{a_{n+1}}{a_n} \right\} = L < 1 \Longrightarrow \displaystyle\sum_{n \geq 1} a_n ~ \text{converge}\\
        \liminf \left\{ \frac{a_{n+1}}{a_n} \right\} = l > 1 \Longrightarrow \displaystyle\sum_{n \geq 1} a_n ~ \text{no converge}
    \end{gather*}
\end{ejercicio}
%########################################################################################################
% Tema 13: Límite funcional.
%########################################################################################################

\chapter{Límite funcional}\label{chp:Tema13}

En este tema introduciremos el concepto de límite de una función en un punto que, cómo se verá, está muy ligado al concepto de continuidad. Analizaremos detalladamente la relación entre ambas nociones, obteniendo de paso una clasificación de las posibles discontinuidades de una función. Con todo, el concepto de límite funcional adquirirá su verdadera relevancia cuando se estudie Cálculo Diferencial en una variable en Cálculo II.\\

Además, empleando lo que ya sabemos de sucesiones divergentes, podremos analizar el comportamiento de una función cuando la variable crece o decrece indefinidamente (límites en el infinito), y por otra, considerar la noción de divergencia de una función.

%########################################################################################################
% Concepto de límite funcional.
%########################################################################################################

\section{Concepto de límite funcional}
\begin{definicion}
    Sea $A$ un conjunto no vacío de números reales y $\alpha \in \mathbb{R}$.
    
    Diremos que $\alpha$ es un \textbf{punto de acumulación de $A$} cuando exista una sucesión $\{a_n\}$ de elementos de $A$,
    distintos de $\alpha$, convergente a $\alpha$. Notaremos $A'$ al conjunto de puntos de acumulación de $A$.
    
    Diremos que un elemento $a$ de $A$ es un \textbf{punto aislado de $A$} si no es un punto de acumulación de $A$.
\end{definicion}
\begin{ejemplo}
    Si $A = \{0\} ~ \cup ~ ]1,2[$, entonces $A'= [1,2]$, mientras que el único punto aislado de $A$ es el $0$.
\end{ejemplo}

El ejemplo anterior sirve para poner de manifiesto que ninguna de las inclusiones $A' \subseteq A$, $A \subseteq A'$ tiene por qué ser cierta. De hecho, ninguna lo es en el ejemplo anterior.

\begin{prop}[Caracterización de los puntos de acumulación]\label{prop:13.1.2}
    Sea $A$ un conjunto de números reales no vacío y $\alpha \in \mathbb{R}$.
    \begin{equation*}
        \alpha \in A' \Longleftrightarrow \forall \delta > 0, ~ \exists x \in A ~:~ 0 < |x-\alpha| < \delta
    \end{equation*}
    Como consecuencia, un punto $a \in A$ es un punto aislado de $A$ si, y sólo si, existe $\delta > 0$ tal que
    $A \cap ~ ]a-\delta,a+\delta[ = \{a\}$.
\end{prop}
\begin{proof} Procedemos mediante doble implicación:
\begin{description}
    \item[$\Longrightarrow)$]
        Sea $\alpha \in A'$ y $\{a_n\}$ una sucesión de puntos de $A$, con $a_n \neq \alpha, ~ \forall n \in \mathbb{N}$, y $\{a_n\} \longrightarrow \alpha$. Dado $\delta > 0$, tenemos que
        \begin{equation*}
            \exists m \in \mathbb{N} ~:~ n \in \mathbb{N}, ~ n \geq m \Longrightarrow |a_n-\alpha| < \delta,
        \end{equation*}
        y en particular, $0 < |a_m - \alpha| < \delta$.
    \item[$\Longleftarrow)$]
        Para cada $n$ natural, sea $a_n \in A$ tal que $0 < |a_n-\alpha| < \frac{1}{n}$; es inmediato que $\{a_n\} \longrightarrow \alpha$ y $a_n \neq \alpha, ~ \forall n \in \mathbb{N}$, luego $\alpha$ es un punto de acumulación de $A$.
\end{description}
\end{proof}

\begin{definicion}
    Sea $f ~:~ A \longrightarrow \mathbb{R}$ una función real de variable real y sea $\alpha \in A'$.
    Se dice que la función $f$ tiene \textbf{límite en el punto $\alpha$} si existe $L \in \mathbb{R}$ tal que para toda sucesión $\{a_n\}$ de elementos de $A$ distintos de $\alpha$, convergente a $\alpha$, se tiene que $\{f(a_n)\} \longrightarrow  L$. En tal caso, dicho $L$ es único y diremos que $L$ es \textbf{el límite de $f$ en el punto $\alpha$} y escribiremos:
    \begin{equation*}
        \lim_{x \to \alpha} f(x) = L
    \end{equation*}
\end{definicion}

La definición anterior tiene sentido gracias a la condición de que $\alpha$ sea un punto de acumulación de $A$.
Dado que un punto de acumulación de $A$ no tiene por qué pertenecer a $A$, puede tener sentido hablar de la existencia de límite en puntos en los que la función no esté definida, a diferencia de lo que ocurría con la continuidad.\\

Por otra parte, si $\alpha$ es un punto aislado de $A$, entonces no tiene sentido hablar de la existencia de límite en el punto $\alpha$ para funciones definidas en $\alpha$.\\

Finalmente, si $\alpha \in A$ es un punto de acumulación de $A$, el valor que tome una función $f ~:~ A \longrightarrow \mathbb{R}$ en el punto $\alpha$ no afecta para nada a la existencia del límite de $f$ en el punto $\alpha$ ni al valor de dicho límite en
caso de que este exista.

\begin{teo}[Caracterización de límite funcional]\label{teo:13.1.4}
    Sea $f ~:~ A \longrightarrow \mathbb{R}$ una función real de variable real, sea $\alpha \in A'$ y $L \in \mathbb{R}$.
    Las siguientes afirmaciones son equivalentes:
    \begin{enumerate}
        \item $\displaystyle\lim_{x \to \alpha} f(x) = L$

        \item Para toda sucesión $\{a_n\}$ de elementos de $A$, monótona y convergente a $\alpha$, se tiene que $\{f(a_n)\}\longrightarrow L$.

        \item $\forall \varepsilon > 0, ~ \exists \delta > 0 ~:~ x \in A, ~ 0 < |x-\alpha| < \delta \Longrightarrow |f(x)-L| < \varepsilon$.
    \end{enumerate}
\end{teo}
\begin{proof} Demostramos por separado las equivalencias:
    \begin{description}
        \item[I. $\Longleftrightarrow$ II.] Análoga a la demostración del Teorema \ref{teo:12.4.2}.

        \item[I. $\Longleftrightarrow$ III.] Análoga a la demostración del Teorema \ref{teo:12.4.1}.
    \end{description}

    Dejamos como ejercicio la trascripción, casi literal, de las demostraciones referidas al caso presente.
\end{proof}

%########################################################################################################
% Relación entre límite funcional y continuidad.
%########################################################################################################

\section{Relación entre límite funcional y continuidad}

El siguiente enunciado permite analizar la continuidad de una función en un punto mediante el concepto de límite funcional.

\begin{prop}\label{prop:13.2.1}
    Sea $f ~:~ A \longrightarrow \mathbb{R}$ y sea $\alpha \in A$.
    \begin{enumerate}
        \item Si $\alpha$ es un punto aislado de $A$, entonces $f$ es continua en $\alpha$.

        \item Si $\alpha$ es un punto de acumulación de $A$, entonces $f$ es continua en $\alpha$ si, y sólo si,
        $\displaystyle\lim_{x \to \alpha} f(x) = f(\alpha)$.
    \end{enumerate}
\end{prop}
\begin{proof} Demostramos cada una de las afirmaciones:
    \begin{enumerate}
        \item Según la Proposición \ref{prop:13.1.2}, $\alpha$ es un punto aislado de $A$ si, y sólo si, existe un $\delta > 0$ tal que
        \begin{equation*}
            \{x \in A ~:~ |x-\alpha| < \delta\} = \{\alpha\}
        \end{equation*}
        Y esto no es más que otra forma de enunciar el Corolario \ref{coro:12.3.4}.

        \item Si $f$ es continua en $\alpha$, para toda sucesión $\{a_n\}$ de puntos de $A$, convergente a $\alpha$ la sucesión $\{f(a_n)\}$ converge a $f(\alpha)$, y esto ocurre, en particular, cuando $a_n \neq \alpha$ para todo $n$ natural, luego $\displaystyle\lim_{x \to \alpha} f(x) = f(\alpha)$.\\
        
        Recíprocamente, si $\displaystyle\lim_{x \to \alpha} f(x) = f(\alpha)$, dado $\varepsilon > 0$ existe $\delta > 0$ tal que si $x \in A$ y $0 < |x-\alpha| < \delta$, entonces $|f(x)-f(\alpha)| < \varepsilon$.
        Si fuera $|x-\alpha|=0$ tenemos $x = \alpha$ y $|f(x)-f(\alpha)| = 0 < \varepsilon$. Por tanto, hemos probado que
        \begin{equation*}
            \forall \varepsilon > 0, ~ \exists \delta > 0 ~:~ x \in A, ~ |x-\alpha| < \delta \Longrightarrow |f(x)-f(\alpha)| < \varepsilon
        \end{equation*}
        Aplicando el Teorema \ref{teo:12.4.1}, $f$ es continua en $\alpha$, como queríamos.
    \end{enumerate}
\end{proof}

Según la proposición anterior, si $f$ no es continua en $\alpha$ puede ocurrir que $f$ no tenga límite en $\alpha$ o que dicho límite sea distinto de $f(\alpha)$. Tenemos por tanto dos tipos de discontinuidades.
\begin{definicion}
    Sea $f ~:~ A \longrightarrow \mathbb{R}$ y $\alpha \in A \cap A'$. Diremos que $f$ tiene una \textbf{discontinuidad evitable} en el punto $\alpha$ cuando $f$ tenga límite en el punto $\alpha$ y dicho límite sea distinto de $f(\alpha)$.
\end{definicion}

El nombre de evitable se justifica por el hecho de que si cambiamos el valor de la función $f$ en el punto $\alpha$, dándole el valor $\displaystyle\lim_{x \to \alpha} f(x)$, obtenemos \underline{otra función} que es continua en $\alpha$ y que coincide con $f$ en $A - \{\alpha\}$.\\

Notemos que la Proposición \ref{prop:13.2.1} se aplica solamente a puntos del conjunto de definición de la función.
Para puntos de acumulación de dicho conjunto que no pertenezcan al mismo, la existencia de límite de una función equivale a la posibilidad de extender la función de forma que se obtenga una función continua.

\begin{prop}
    Sea $f ~:~ A \longrightarrow \mathbb{R}$ y $\alpha \in A'$. Las siguientes afirmaciones son equivalentes:
    \begin{enumerate}
        \item $f$ tiene límite en el punto $\alpha$.
        \item Existe una función $\hat{f} ~:~ A ~\cup~ \{\alpha\} \longrightarrow \mathbb{R}$, continua en el punto $\alpha$ y tal que $\hat{f}(x) =f(x),~ \forall x \in~A-\{\alpha\}$.
    \end{enumerate}
    Además, en caso de que se cumplan I) y II), se tiene que $\hat{f}(\alpha) = \displaystyle\lim_{x \to \alpha} f(x)$.
    Como consecuencia, la función $\hat{f}$ es única.
\end{prop}
\begin{proof} Procedemos mediante doble implicación:
\begin{description}
    \item [I. $\Longrightarrow$ II.]
    Definimos $\hat{f} ~:~ A \cup \{\alpha\} \longrightarrow \mathbb{R}$ como sigue:
    \begin{equation*}
        \hat{f}(\alpha) = \displaystyle\lim_{x \to \alpha} f(x), ~ \hat{f}(x) = f(x), ~ \forall x \in A-\{\alpha\}
    \end{equation*}
    Es evidente que $\alpha$ es un punto de acumulación de $A \cup \{\alpha\}$ y que
    \begin{equation*}
        \displaystyle\lim_{x \to \alpha} \hat{f}(x) = \displaystyle\lim_{x \to \alpha} f(x) = \hat{f}(\alpha)
    \end{equation*}
    por lo que, en virtud de la Proposición \ref{prop:13.2.1}, $\hat{f}$ es continua en $\alpha$; es obvio que $f$ y $\hat{f}$
    coinciden en $A-\{\alpha\}$.

    \item [II. $\Longrightarrow$ I.]
    Sea $\{a_n\}$ una sucesión de puntos de $A$ distintos de $\alpha$, convergente a $\alpha$.
    Por ser $\hat{f}$ continua en $\alpha$, tenemos que $\{\hat{f}(a_n)\} \longrightarrow \hat{f}(\alpha)$.
    \newline
    \newline
    Como sabemos que $a_n \in A - \{\alpha\}, ~ \forall n \in \mathbb{N}$, tenemos que
    \begin{equation*}
        \hat{f}(a_n) = f(a_n), ~ \forall n \in \mathbb{N}
    \end{equation*}
    y por tanto $\{f(a_n)\} \longrightarrow \hat{f}(\alpha)$, lo que prueba que $\displaystyle\lim_{x \to \alpha} f(x) = \hat{f}(\alpha)$.
\end{description}
    
\end{proof}

Merece la pena clarificar un poco la proposición anterior distinguiendo varios casos. Si $\alpha \in A$ y $f$ es continua en $\alpha$, entonces $\hat{f} = f$; si $\alpha \in A$ y $f$ tiene una discontinuidad evitable en $\alpha$, entonces $A \cup \{\alpha\} = A$ y $\hat{f}$ es la función que se obtiene al ``evitar'' la discontinuidad de $f$. Finalmente, si $\alpha \notin A$, según la proposición anterior, $f$ tiene límite en el punto $\alpha$ si, y sólo si, $f$ admite una extensión al conjunto $A \cup \{\alpha\}$ que sea continua en $\alpha$, que es precisamente la función $\hat{f}$.

%########################################################################################################
% Límites laterales.
%########################################################################################################


\section{Límites laterales}
Según el Teorema \ref{teo:13.1.4}, para comprobar la existencia de límite de una función en un punto, basta considerar sucesiones monótonas, es decir, considerar simultáneamente sucesiones crecientes y sucesiones decrecientes. Cuando se consideran unas y otras por separado se llega al concepto de límite lateral que introduciremos a continuación. Intuitivamente, el que una función tenga límite en un punto indica una regularidad al ``acercarnos'' a dicho punto. Cuando solamente admitimos la posibilidad de acercarnos al punto por la izquierda (respectivamente por la derecha), aparece el concepto de límite por la izquierda (respectivamente derecha).
\begin{definicion}
    Sea $f ~:~ A \longrightarrow \mathbb{R}$ una función real de variable real y $\alpha \in \mathbb{R}$.
    Se dice que $\alpha$ es un \textbf{punto de acumulación por su izquierda} (respectivamente \textbf{por su derecha}) de $A$ si existe alguna sucesión $\{x_n\}$ de puntos de $A$, menores que $\alpha$ (respectivamente mayores que $\alpha$), convergente a $\alpha$. Escribiremos que $\alpha \in A'_{-}$ (respectivamente $\alpha \in A'_{+}$).
    
    De forma equivalente, tenemos que
    \begin{gather*}
        \alpha \in A'_{-} \Longleftrightarrow \forall \delta > 0, ~ ]\alpha-\delta,\alpha[ ~ \cap ~ A \neq \emptyset \\
        \alpha \in A'_{+} \Longleftrightarrow \forall \delta > 0, ~ ]\alpha,\alpha+\delta[ ~ \cap ~ A \neq \emptyset
    \end{gather*}
\end{definicion}

\begin{definicion}
    Sea $f ~:~ A \longrightarrow \mathbb{R}$ una función real de variable real y $\alpha \in \mathbb{R}$.
    Diremos que $f$ tiene \textbf{límite por la izquierda} (respectivamente \textbf{límite por la derecha}) en el punto $\alpha$, si existe un número real $L$ verificando la siguiente propiedad:
    
    \emph{Para toda sucesión $\{a_n\}$ de elementos de $A$, distintos de $\alpha$, creciente (respectivamente decreciente) y convergente
    a $\alpha$, se tiene que la sucesión $\{f(a_n)\}$ converge a $L$.}
    
    Se suele escribir
    \begin{equation*}
        L = \lim_{x \to \alpha^{-}} f(x) ~ (\text{respectivamente} ~  L = \lim_{x \to \alpha^{+}} f(x))
    \end{equation*}
\end{definicion}

Si consideramos los conjuntos
\begin{equation*}
    A_{\alpha}^{-} = \{a \in A ~:~ a < \alpha\}; ~ A_{\alpha}^{+} = \{a \in A ~:~ a > \alpha\}
\end{equation*}

Es inmediato que para que tenga sentido hablar de límite por la izquierda (respectivamente derecha) en $\alpha$ para funciones de $A$ en $\mathbb{R}$ es condición necesaria y suficiente que $\alpha \in (A_{\alpha}^{-})'$ (respectivamente $\alpha \in (A_{\alpha}^{+})'$).

También es inmediato que $\alpha$ es un punto de acumulación de $A$ si, y sólo si, lo es de $A_{\alpha}^{-}$ o de $A_{\alpha}^{+}$.

Por tanto, si tiene sentido hablar de límite de un función en un punto $\alpha$ para funciones de $A$ en $\mathbb{R}$, también tiene sentido hablar de al menos uno de los límites laterales.

\begin{prop}
    Sea $f ~:~ A \longrightarrow \mathbb{R}$ una función real de variable real, $\alpha \in A'$ y $L$ un número real.
    \begin{enumerate}
        \item Supongamos que $\alpha \notin (A_{\alpha}^{+})'$ pero $\alpha \in (A_{\alpha}^{-})'$. Entonces
        \begin{equation*}
           \lim_{x \to \alpha} f(x) = L \Longleftrightarrow \lim_{x \to \alpha^{-}} f(x) = L
        \end{equation*}

        \item Supongamos que $\alpha \in (A_{\alpha}^{+})'$ pero $\alpha \notin (A_{\alpha}^{-})'$. Entonces
        \begin{equation*}
           \lim_{x \to \alpha} f(x) = L \Longleftrightarrow \lim_{x \to \alpha^{+}} f(x) = L
        \end{equation*}

        \item Supongamos que $\alpha \in (A_{\alpha}^{+})' \cap (A_{\alpha}^{-})'$. Entonces
        \begin{equation*}
            \lim_{x \to \alpha} f(x) = L \Longleftrightarrow \lim_{x \to \alpha^{+}} f(x) = \lim_{x \to \alpha^{-}} f(x) = L
        \end{equation*}
    \end{enumerate}
\end{prop}
\begin{proof}
    Demostramos cada uno de los apartados:
    \begin{enumerate}
        \item En este caso, el conjunto de sucesiones monótonas de puntos de $A$ distintos de $\alpha$ y convergentes a $\alpha$ coincide con el conjunto de sucesiones crecientes de puntos de $A$ distintos de $\alpha$ y convergentes a $\alpha$, luego los conceptos de límite de una función en un punto $\alpha$ y de límite por la izquierda en $\alpha$ son idénticos
        en vista del Teorema \ref{teo:13.1.4} I $\Longleftrightarrow$ II)).

        \item Análoga a I).

        \item No es más que otra forma de enunciar la equivalencia entre las afirmaciones I) y II) del Teorema \ref{teo:13.1.4}.
    \end{enumerate}
\end{proof}

\begin{definicion}
    Sea $f ~:~ A \longrightarrow \mathbb{R}$ una función real de variable real y $\alpha \in A'$.
    Si no existe $\displaystyle\lim_{x \to \alpha} f(x)$, diremos que $f$ tiene una \textbf{discontinuidad esencial}.
\end{definicion}
\begin{definicion}
    Sea $f ~:~ A \longrightarrow \mathbb{R}$ una función real de variable real y y $\alpha \in (A_{\alpha}^{+})' \cap (A_{\alpha}^{-})'$.
    Si $f$ tiene límite por la izquierda y por la derecha en el punto $\alpha$ y ocurre que
    \begin{equation*}
        \lim_{x \to \alpha^{+}} f(x) \neq \lim_{x \to \alpha^{-}} f(x)
    \end{equation*}
    diremos que $f$ tiene una \textbf{discontinuidad de salto finito} en el punto $\alpha$.
\end{definicion}

Como ejemplo, es fácil probar (¡Hágase!) que todas las discontinuidades de la función parte entera son de salto finito.

\begin{prop}
    Sea $f ~:~ A \longrightarrow \mathbb{R}$ una función real de variable real, $\alpha \in \mathbb{R}$.
    Supongamos que $\alpha$ es un punto de acumulación de $A_{\alpha}^{-}$ (respectivamente $A_{\alpha}^{+}$).
    Consideramos $g = f_{|_{A_{\alpha}^{-}}}$ (respectivamente $g = f_{|_{A_{\alpha}^{+}}}$).
    Entonces, $f$ tiene límite por la izquierda (respectivamente derecha) en el punto $\alpha$ si, y sólo si,
    $g$ tiene límite en el punto $\alpha$, en cuyo caso
    \begin{equation*}
        \displaystyle\lim_{x \to \alpha^{-}} f(x) = \displaystyle\lim_{x \to \alpha} g(x) ~ \qquad \qquad (\text{respectivamente} ~ \displaystyle\lim_{x \to \alpha^{+}} f(x) = \displaystyle\lim_{x \to \alpha} g(x))
    \end{equation*}
\end{prop}
\begin{proof}
    El conjunto de las sucesiones crecientes (respectivamente decrecientes) de puntos de $A$ distintos de $\alpha$, convergentes a $\alpha$, coincide con el conjunto de las sucesiones monótonas de puntos $A_{\alpha}^{-}$ (respectivamente $A_{\alpha}^{+}$), distintos de $\alpha$, convergentes a $\alpha$. Con esto la proposición es evidente teniendo en cuenta una vez más el Teorema \ref{teo:13.1.4}. 
\end{proof}



%########################################################################################################
% Límites en el infinito.
%########################################################################################################

\section{Límites en el infinito}

El concepto de sucesión divergente, introducido en el tema \ref{chp:Tema7}, nos va a permitir estudiar el comportamiento de una función real de variable real cuando la variable crece o decrece sin límite.

\begin{definicion}
    Sea $A$ un conjunto de números reales no mayorado (respectivamente no minorado) y $f ~:~ A \longrightarrow \mathbb{R}$ una función. Diremos que $f$ tiene \textbf{límite cuando la variable diverge positivamente} (respectivamente \textbf{negativamente}) o que tiene \textbf{límite en $+\infty$} (respectivamente que tiene \textbf{límite en $-\infty$}) cuando exista $L \in \mathbb{R}$ con la siguiente propiedad:

    \emph{Para toda sucesión $\{x_n\}$ de elementos de $A$ que diverja positivamente (respectivamente negativamente) la sucesión $\{f(x_n)\}$ converge a $L$.}

    Si existe, dicho $L$ es único y se suele escribir:
    \begin{equation*}
        L = \displaystyle\lim_{x \to + \infty} f(x) ~ (\text{respectivamente} ~ L = \displaystyle\lim_{x \to - \infty} f(x))
    \end{equation*}
\end{definicion}

\begin{prop}
    Sea $A$ un conjunto de números reales no minorado y $f ~:~ A \longrightarrow \mathbb{R}$ una función.
    Consideramos el siguiente conjunto
    \begin{equation*}
        B = \{-x ~:~ x \in A\} ~ (\text{Claramente $B$ no está mayorado})
    \end{equation*}
    Definimos $g ~:~ B \longrightarrow \mathbb{R}$ dada por
    \begin{equation*}
        g(x) = f(-x), ~ \forall x \in B
    \end{equation*}
    Si $L$ es un número real, entonces
    \begin{equation*}
        \displaystyle\lim_{x \to - \infty} f(x) = L \Longleftrightarrow \displaystyle\lim_{x \to + \infty} g(x) = L
    \end{equation*}
\end{prop}
\begin{proof}
    Probaremos solamente la implicación a la derecha, pues la otra es enteramente análoga.
    Sea $L = \displaystyle\lim_{x \to - \infty} f(x)$ y sea $\{x_n\}$ una sucesión de elementos de $B$ que diverja positivamente.
    Entonces $\{-x_n\}$ es una sucesión de puntos de $A$ que diverge negativamente, luego la sucesión $\{f(-x_n)\} = \{g(x_n)\}$
    converge a $L$.
\end{proof}

\begin{prop}\label{prop:13.4.3}
    Sea $A$ un conjunto de números reales no mayorado y $f ~:~ A \longrightarrow \mathbb{R}$ una función.
    Sea $B = \left\{ \frac{1}{x} ~:~ x \in A, ~ x > 0 \right\}$. Definimos $g ~:~ B \longrightarrow \mathbb{R}$ dada por
    \begin{equation*}
        g(x) = f\left(\frac{1}{x}\right), ~ \forall x \in B
    \end{equation*}
    Entonces, $0 \in B'$ y dado un número real $L$ se tiene que
    \begin{equation*}
        \displaystyle\lim_{x \to 0} g(x) = L \Longleftrightarrow \displaystyle\lim_{x \to + \infty} f(x) = L
    \end{equation*}
\end{prop}
\begin{proof}
    Sea $\{x_n\}$ cualquier sucesión de puntos de $A$ que diverja positivamente. Entonces
    \begin{equation*}
        \exists m \in \mathbb{N} ~:~ n \geq m \Longrightarrow x_n > 0
    \end{equation*}
    La sucesión $\{x_{n+m}\}$ diverge positivamente (¿Por qué?) y todos sus términos son positivos, luego $\left\{ \frac{1}{x_{n+m}} \right\}$ es una sucesión de puntos de $B$ no nulos que,
    por el ejercicio \ref{ej:7.2.11}, converge a $0$. Esto prueba que $0 \in B'$.\\
    
    Supongamos que $\displaystyle\lim_{x \to 0} g(x) = L$ y sea $\{x_n\}$ como antes. Entonces, la sucesión $\left\{g\left(\frac{1}{x_{n+m}}\right)\right\}$ converge a $L$, por lo que $\{f(x_{n+m})\}$ converge a $L$ y, por tanto, $\{f(x_n)\}$ converge a $L$.\\
    
    Recíprocamente, supongamos que $\displaystyle\lim_{x \to + \infty} f(x) = L$ y sea $\{x_n\}$ una sucesión cualquiera de puntos de $B$ que converja a $0$. Como $x_n \neq 0, ~ \forall n \in \mathbb{N}$, podemos aplicar de nuevo el ejercicio \ref{ej:7.2.11}, obteniendo que la sucesión $\left\{\frac{1}{|x_n|}\right\} = \left\{\frac{1}{x_n}\right\}$ diverge positivamente. Como $\frac{1}{x_n} \in A, ~ \forall n \in \mathbb{N}$, tenemos que $\left\{f\left(\frac{1}{x_n}\right)\right\} \longrightarrow L$, por lo que la sucesión $\{g(x_n)\}$ converge a $L$.
\end{proof}

%########################################################################################################
% Funciones divergentes.
%########################################################################################################

\section{Funciones divergentes}
\begin{definicion}
    Sea $f ~:~ A \longrightarrow \mathbb{R}$ una función real de variable real. Dado $\alpha \in A'$, diremos que la función $f$ \textbf{diverge positivamente en el punto $\alpha$} cuando para toda sucesión $\{a_n\}$ de puntos de $A$ distintos de $\alpha$, convergente a $\alpha$, la sucesión $\{f(a_n)\}$ diverge positivamente. Escribiremos
    \begin{equation*}
        \displaystyle\lim_{x \to \alpha} f(x) = +\infty
    \end{equation*}

    Supongamos ahora que $\alpha \in (A_{\alpha}^{-})'$ (respectivamente $\alpha \in (A_{\alpha}^{+})'$). Diremos que $f$ \textbf{diverge positivamente en el punto $\alpha$ por la izquierda} (respectivamente \textbf{por la derecha}) cuando para toda sucesión $\{a_n\}$ de puntos de $A$, distintos de $\alpha$, creciente (respectivamente decreciente) y convergente a $\alpha$, la sucesión $\{f(a_n)\}$ diverge positivamente. Escribiremos
    \begin{equation*}
        \displaystyle\lim_{x \to \alpha^{-}} f(x) = +\infty ~ (\text{respectivamente} ~ \displaystyle\lim_{x \to \alpha^{+}} f(x) = +\infty)
    \end{equation*}
    
    Finalmente, supongamos que $A$ no está mayorado (respectivamente no minorado). Diremos que $f$ \textbf{diverge positivamente en $+ \infty$} (respectivamente en $- \infty$) cuando para toda sucesión $\{a_n\}$ de puntos de $A$ que diverja positivamente (respectivamente negativamente) se tiene que $\{f(a_n)\}$ diverge positivamente. Escribiremos
    \begin{equation*}
        \displaystyle\lim_{x \to +\infty} f(x) = +\infty ~ (\text{respectivamente} ~ \displaystyle\lim_{x -\infty} f(x) = +\infty)
    \end{equation*}
    
    Análogas definiciones para la divergencia negativa de $f$ en un punto $\alpha$, en el punto $\alpha$ por la izquierda,
    en el punto $\alpha$ por la derecha, en $+ \infty$ y $- \infty$.
\end{definicion}

Las proposiciones que ya hemos probado referentes a la relación entre límites laterales y límite en un punto y a la reducción de los conceptos de límite lateral y límite en el infinito al de límite en un punto se pueden adaptar con facilidad al caso de funciones divergentes. Enunciamos la adaptación de la Proposición \ref{prop:13.4.3}, cuya demostración se deja como ejercicio.
\begin{prop}
    Sea $A$ un conjunto de números reales no mayorado y sea $f ~:~ A \longrightarrow \mathbb{R}$ una función.
    Sea $B = \left\{ \frac{1}{x} ~:~ x \in A, ~ x > 0 \right\}$. Definimos $g ~:~ B \longrightarrow \mathbb{R}$ dada por
    \begin{equation*}
        g(x) = f\left(\frac{1}{x}\right), ~ \forall x \in B
    \end{equation*}
    Entonces, $0 \in B'$ y se tiene que
    \begin{equation*}
        \displaystyle\lim_{x \to 0} g(x) = + \infty \Longleftrightarrow \displaystyle\lim_{x \to + \infty} f(x) = + \infty
    \end{equation*}
\end{prop}

En vista de lo anterior, el comportamiento de una función en un punto por la derecha o por la izquierda, en $+ \infty$ o en $- \infty$ equivale al comportamiento de otra función en un punto. En el resto del tema, todos los resultados se referirán al límite o divergencia de una función en un punto. Sin embargo, debe quedar claro que todos ellos tienen automáticamente su correspondiente enunciado para límites laterales y límites en el infinito.

\begin{definicion}
    Sea $f ~:~ A \longrightarrow \mathbb{R}$ una función real de variable real y sea $\alpha \in (A_{\alpha}^{-})' \cap A$ (respectivamente $\alpha \in (A_{\alpha}^{+})' \cap A$). Diremos
    que $f$ tiene una \textbf{discontinuidad de salto infinito por la izquierda} (respectivamente \textbf{derecha})
    \textbf{de $\alpha$} si $f$ diverge en el punto $\alpha$ por la izquierda (respectivamente derecha).
\end{definicion}

%########################################################################################################
% Álgebra de límites.
%########################################################################################################

\newpage

\section{Álgebra de límites}
En este apartado, relacionamos los conceptos de límite y de divergencia de una función en un punto con las operaciones de suma y producto.
\begin{prop}
    Sean $f,g ~:~ A \longrightarrow \mathbb{R}$ funciones reales de variable real y $\alpha \in A'$.
    \begin{enumerate}
        \item Si $\displaystyle\lim_{x \to \alpha} f(x) = L_1$ y $\displaystyle\lim_{x \to \alpha} g(x) = L_2$, entonces $\displaystyle\lim_{x \to \alpha} (f+g)(x) = L_1 + L_2$.

        \item Si $\displaystyle\lim_{x \to \alpha} f(x) = + \infty$ y $g$ verifica que
        \begin{equation*}
            \exists \delta > 0, ~ \exists M \in \mathbb{R} ~:~ x \in A, ~ |x-\alpha| < \delta \Longrightarrow g(x) \geq M
        \end{equation*}
        (esto es, $g$ está minorada en la intersección de $A$ con un cierto intervalo de centro $\alpha$),
        entonces $\displaystyle\lim_{x \to \alpha} (f+g)(x) = + \infty$.
    \end{enumerate}
\end{prop}
\begin{proof}
    Consecuencia inmediata de la Proposición \ref{prop:5.3.1} y del ejercicio \ref{ej:7.2.4}.
\end{proof}

\begin{prop}\label{prop:13.6.2}
    Sean $f,g ~:~ A \longrightarrow \mathbb{R}$ funciones reales de variable real y $\alpha \in A'$.
    \begin{enumerate}
        \item Si $\displaystyle\lim_{x \to \alpha} f(x) = 0$ y $g$ verifica que
        \begin{equation*}
            \exists \delta > 0, ~ \exists M \in \mathbb{R} ~:~ x \in A, ~ |x-\alpha| < \delta \Longrightarrow |g(x)| \leq M
        \end{equation*}
        (esto es, $g$ está acotada en la intersección de $A$ con un cierto intervalo de centro $\alpha$),
        entonces $\displaystyle\lim_{x \to \alpha} (fg)(x) = 0$.

        \item Si $\displaystyle\lim_{x \to \alpha} f(x) = + \infty$ y $g$ verifica que
        \begin{equation*}
            \exists \delta > 0, ~ \exists \lambda > 0 ~:~ x \in A, ~ 0 < |x-\alpha| < \delta \Longrightarrow g(x) \geq \lambda
        \end{equation*}
        (esto es, $g$ está minorada por un número positivo en la intersección de $A$ con un cierto intervalo de centro $\alpha$ salvo el propio punto $\alpha$), entonces se tiene $\displaystyle\lim_{x \to \alpha} (fg)(x) =+\infty$.
    \end{enumerate}
\end{prop}
\begin{proof}
    Consecuencia inmediata de la Proposición \ref{prop:5.3.2} y del Teorema \ref{teo:7.1.4}. 
\end{proof}

\begin{coro}
    Sean $f,g ~:~ A \longrightarrow \mathbb{R}$ funciones reales de variable real y $\alpha \in A'$.
    \begin{enumerate}
        \item Si $f$ y $g$ tienen límite en el punto $\alpha$, entonces $fg$ tiene límite en el punto $\alpha$ y se verifica que
        \begin{equation*}
            \displaystyle\lim_{x \to \alpha} (fg)(x) = \displaystyle\lim_{x \to \alpha} f(x) \displaystyle\lim_{x \to \alpha} g(x)
        \end{equation*}

        \item Si $f$ diverge en el punto $\alpha$ y $g$ tiene límite no nulo o diverge en $\alpha$, entonces $fg$ diverge en el punto $\alpha$.
    \end{enumerate}
\end{coro}
\begin{proof}
    Demostramos cada afirmación por separado:
    \begin{enumerate}
        \item Consecuencia de la Proposición \ref{prop:5.3.3} y también se puede deducir de la proposición anterior (proposición \ref{prop:13.6.2}).

        \item Se deduce de la segunda parte de la proposición anterior (proposición \ref{prop:13.6.2}). Véanse el Teorema \ref{teo:7.1.4} y los ejercicios \ref{ej:7.2.7}, \ref{ej:7.2.8}, \ref{ej:7.2.9}.
    \end{enumerate}
\end{proof}

\begin{prop}
    Sean $f ~:~ A \longrightarrow \mathbb{R}$ una función real de variable real y $\alpha \in A'$.
    Definimos $B = \{x \in A ~:~ f(x) \neq 0\}$. Si $\displaystyle\lim_{x \to \alpha} f(x) = L \neq 0$ (respectivamente $f$ diverge en $\alpha$), entonces $\alpha$ es un punto de acumulación de $B$ y la función $\frac{1}{f} ~:~ B \longrightarrow ~\mathbb{R}$ verifica:
    \begin{equation*}
        \displaystyle\lim_{x \to \alpha} \left(\frac{1}{f}\right)(x) = \frac{1}{L}
        \qquad \qquad
        (\text{respectivamente} ~ \displaystyle\lim_{x \to \alpha} \left(\frac{1}{f}\right)(x) = 0)
    \end{equation*}
\end{prop}
\begin{proof}
    Sea $\{a_n\}$ una sucesión de elementos de $A$, distintos de $\alpha$, convergente a $\alpha$. Entonces, o bien $\{f(a_n)\} \longrightarrow L \neq 0$ o bien $\{f(a_n)\}$ diverge. En cualquier caso
    \begin{equation*}
        \exists m \in \mathbb{N} ~:~ n \geq m \Longrightarrow f(a_n) \neq 0
    \end{equation*}
    y la sucesión $\{a_{n+m}\}$ es una sucesión de puntos de $B$ distintos de $\alpha$ convergente a $\alpha$, luego $\alpha$ es un punto de acumulación de $B$. Sea $\{b_n\}$ cualquier sucesión de elementos de $B$, distintos de $\alpha$, convergente a $\alpha$. Aplicando la Proposición \ref{prop:5.3.2} y el ejercicio \ref{ej:7.2.11}, tenemos en un caso que $\left\{ \frac{1}{f}(b_n)\right\} \longrightarrow \frac{1}{L}$ y en otro $\left\{ \frac{1}{f}(b_n)\right\} \longrightarrow 0$.
\end{proof}

%########################################################################################################
% Ejercicios de Límite funcional.
%########################################################################################################

\section{Ejercicios}
\begin{ejercicio}
    Sea $f ~:~ A \longrightarrow \mathbb{R}$ y $\alpha \in A'$. Probar que $f$ tiene límite en el punto $\alpha$ si, y sólo si, para toda sucesión $\{a_n\}$ de puntos de $A$, distintos de $\alpha$, convergente a $\alpha$, la sucesión $\{f(a_n)\}$ es convergente. ¿Es cierto el mismo enunciado pero considerando solamente sucesiones monótonas?
\end{ejercicio}

\begin{ejercicio}
    Sea $f ~:~ A \longrightarrow \mathbb{R}$ y $\alpha \in A'$. Supongamos que $\displaystyle\lim_{x \to \alpha} f(x) = L \neq 0$. Probar que existe $\delta > 0$ tal que
    \begin{equation*}
        x \in A, ~ 0 < |x-\alpha| < \delta \Longrightarrow L f(x) > 0
    \end{equation*}
\end{ejercicio}

\begin{ejercicio}
    Sea $f ~:~ ]-1,1[ \longrightarrow \mathbb{R}$ definida por
    \begin{equation*}
        f(x)=
        \left\{ \begin{array}{ccl}
            \frac{1}{1+x} & \text{si} & -1 < x < 0 \\
            a & \text{si} & x  = 0 \\
            1+x^2 & \text{si} & 0 < x < 1
        \end{array}
        \right.
    \end{equation*}
    ¿Tiene $f$ límite en $0$? ¿Existe algún valor de $a$ para el cual $f$ sea continua en $0$? ¿Admite $f$ una extensión continua
    al intervalo $]-1,1]$? ¿Y al intervalo $[-1,1]$?
\end{ejercicio}

\begin{ejercicio}
    Calcula la imagen de la función $f ~:~ ]0,1[ ~\longrightarrow \mathbb{R}$ dada por
    \begin{equation*}
        f(x) = \frac{2x-1}{x(x-1)}, ~ \forall x \in ~ ]0,1[
    \end{equation*}
\end{ejercicio}

\begin{ejercicio}
    Calcula la imagen de la función $f ~:~ ]-1,1[ ~\longrightarrow \mathbb{R}$ dada por
    \begin{equation*}
        f(x) = \sqrt{\frac{1-x}{\sqrt{1+x}}}, ~ \forall x \in ~ ]-1,1[
    \end{equation*}
\end{ejercicio}

\begin{ejercicio}
    Calcula la imagen de la función $f ~:~ ]-1,1[ ~\longrightarrow \mathbb{R}$ dada por
    \begin{equation*}
        f(x) = \frac{x}{\sqrt{1-x^2}}, ~ \forall x \in ~ ]-1,1[
    \end{equation*}
\end{ejercicio}

\begin{ejercicio}
    Sea $f ~:~ \mathbb{R} \longrightarrow \mathbb{R}$ definida por
    \begin{equation*}
        f(x)=
        \left\{ \begin{array}{ccl}
            e^{-\frac{1}{x^2}} & \text{si} & x \neq 0 \\
            0 & \text{si} & x  = 0
        \end{array} \right.
    \end{equation*}
    Probar que:
    \begin{enumerate}
        \item $f$ es continua.
        \item $f$ es decreciente en $\mathbb{R}^{-}$.
        \item $f$ es creciente en $\mathbb{R}^{+}$.
        \item Calcular la imagen de $f$.
    \end{enumerate}
\end{ejercicio}
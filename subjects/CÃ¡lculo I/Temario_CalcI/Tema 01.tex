\chapter{Números reales: Propiedades básicas}\label{chp:Tema1}
    
El primer paso para avanzar con provecho en el estudio del Análisis Matemático es establecer
sólidamente la base sobre la que se asienta todo él: \textbf{El cuerpo $\mathbb{R}$ de los números reales}.

Entre las distintas formas que tenemos de introducir el cuerpo de los números reales, elegimos hacerlo
de forma axiomática debido al bagaje de conocimientos que se presupone.

%########################################################################################################
% Axiomas de cuerpo (suma y producto en R)
%########################################################################################################

\section{Axiomas de cuerpo}
Admitimos la existencia de un conjunto $\mathbb{R}$, conjunto de los números reales, que tiene las
siguientes propiedades.
\newline
\newline
\textbf{A.}\label{Axioma_A} En $\mathbb{R}$ hay definida una operación binaria, llamada suma y denotada por ``$+$'', que
verifica los siguientes axiomas:
\newline
\newline
\hspace*{1cm} \textbf{A.1} La suma es asociativa: $(a+b)+c=a+(b+c), ~\forall a,b,c \in \mathbb{R}$.
\newline
\hspace*{1cm} \textbf{A.2} La suma es conmutativa: $a+b=b+a, ~\forall a,b \in \mathbb{R}$.
\newline
\hspace*{1cm} \textbf{A.3} Existe un elemento neutro para la suma:
\begin{equation*}
    \exists e \in \mathbb{R} ~:~a+e=a,~ \forall a \in \mathbb{R}
\end{equation*}
\hspace*{1cm} \textbf{A.4} Todo número real admite un simétrico para la suma:
\begin{equation*}
    \forall a \in \mathbb{R}~ \exists b \in \mathbb{R} ~:~a+b=e
\end{equation*}

Estos 4 axiomas se resumen diciendo que $\mathbb{R}$ con la operación suma es un grupo conmutativo o abeliano.\\

\begin{prop} Respecto a la operación suma, se cumplen las siguientes propiedades:
    \begin{enumerate}
        \item Solo hay un elemento neutro para la suma en $\bb{R}$.
        \item El simétrico para la suma de cualquier número real es único.
    \end{enumerate}
\end{prop}
\begin{proof} Demostramos cada parte por separado:
\begin{enumerate}
    \item Si $\overline{e}$ fuera otro elemento neutro, tendríamos que
    \begin{equation*}
        \overline{e} = \overline{e} + e = e
    \end{equation*}

    \item Para probar la unicidad del elemento simétrico para la suma, dado $a \in \mathbb{R}$, supongamos que $b$ y $c$ son elementos simétricos de $a$.
    Entonces, $a+b=e=a+c$, por lo que
    \begin{equation*}
        b+(a+b)=b+(a+c) \Longleftrightarrow (b+a)+b=(b+a)+c \Longleftrightarrow e+b=e+c \Longleftrightarrow b=c
    \end{equation*}
\end{enumerate}
\end{proof}

De ahora en adelante, al elemento neutro de la suma lo notaremos por $0$. Al simétrico de $a$ lo representaremos por $-a$ y lo llamaremos el \textbf{opuesto de $a$}. Escribiremos $a-b$ en lugar de $a+(-b)$

\vspace{1.3cm}
\textbf{B.}\label{Axioma_B} En $\mathbb{R}$ hay definida una operación binaria, llamada producto y denotada por yuxtaposición o ``$\cdot$'', que verifica los siguientes axiomas:
\newline
\newline
\hspace*{1cm} \textbf{B.1} El producto es asociativo: $(ab)c=a(bc), ~\forall a,b,c \in \mathbb{R}$.
\newline
\newline
\hspace*{1cm} \textbf{B.2} El producto es conmutativo: $ab=ba, ~\forall a,b \in \mathbb{R}$.
\newline
\hspace*{1cm} \textbf{B.3} Existe un elemento neutro no nulo para el producto:
\begin{equation*}
    \exists u \in \mathbb{R}, ~u \neq 0 ~:~au=a,~ \forall a \in \mathbb{R}
\end{equation*}
\hspace*{1cm} \textbf{B.4} Todo número real no nulo admite un simétrico para el producto:
\begin{equation*}
    \forall a \in \mathbb{R}-\{0\}~ \exists b \in \mathbb{R} ~:~ab=u
\end{equation*}
\hspace*{1cm} \textbf{B.5} El producto cumple la propiedad distributiva respecto  de la suma:
\begin{equation*}
    a(b+c)=ab+ac, ~\forall a,b,c \in \mathbb{R}
\end{equation*}

Los axiomas A y B se resumen diciendo que el conjunto $\mathbb{R}$ es un cuerpo conmutativo.\\

\begin{prop} Respecto a la operación producto, se cumplen las siguientes propiedades:
    \begin{enumerate}
        \item Solo hay un elemento neutro para el producto en $\bb{R}$.
        \item El simétrico para el producto de cualquier número real es único.
    \end{enumerate}
\end{prop}
\begin{proof} Demostramos cada parte por separado:
\begin{enumerate}
    \item Si $\overline{u}$ fuera otro elemento neutro, tendríamos que
    \begin{equation*}
        \overline{u} = \overline{u}u = u
    \end{equation*}

    \item Para probar la unicidad del elemento simétrico para el producto, dado $a \in \mathbb{R}$, supongamos que $b$ y $c$ son elementos simétricos de $a$.
    Entonces, $ba=1=ac$, por lo que
    \begin{equation*}
        bac=bac \Longleftrightarrow (ba)c=b(ac) \Longleftrightarrow 1c=b1 \Longleftrightarrow b=c
    \end{equation*}
\end{enumerate}
\end{proof}

Por tanto, queda probada la unicidad del elemento neutro y simétrico para el producto. Al elemento neutro del producto lo llamaremos $1$.
Al simétrico de $a$ lo notaremos $a^{-1}$ o $\frac{1}{a}$, y lo llamaremos el \textbf{inverso de $a$}.
Si $a \in \mathbb{R}$ y $b \in \mathbb{R}-\{0\}$, escribiremos $\frac{a}{b}$ en vez de $ab^{-1}$.

%########################################################################################################
% Axiomas del orden en R
%########################################################################################################

\section{Axiomas de cuerpo ordenado}
\textbf{C.}\label{Axioma_C} En el conjunto $\mathbb{R}$ hay definida una relación binaria, $\leq$, que verifica los
siguientes axiomas:
\newline
\newline
\hspace*{1cm} \textbf{C.1} Propiedad reflexiva: $a \leq a$, $\forall a \in \mathbb{R}$
\newline
\newline
\hspace*{1cm} \textbf{C.2} Propiedad antisimétrica: $a \leq b$ y $b \leq a$ $\Longrightarrow a = b$, $\forall a,b \in \mathbb{R}$
\newline
\newline
\hspace*{1cm} \textbf{C.3} Propiedad transitiva: $a \leq b$ y $b \leq c$ $\Longrightarrow a \leq c$, $\forall a,b,c \in \mathbb{R}$
\newline
\newline
Estas tres propiedades se resumen diciendo que en $\mathbb{R}$ hay definida una relación de orden.
\newline
\newline
\hspace*{1cm} \textbf{C.4} La relación de orden $\leq$ es total:
\begin{equation*}
    \forall a,b \in \mathbb{R},~a \leq b ~\text{, o bien,}~ b \leq a
\end{equation*}
Los siguientes axiomas ligan la relación de orden con la estructura de cuerpo.
\newline
\newline
\hspace*{1cm} \textbf{C.5} $\forall c \in \mathbb{R}$, $\forall a,b \in \mathbb{R} ~:~ a \leq b$ tenemos que $a+c \leq b+c$.
\newline
\newline
\hspace*{1cm} \textbf{C.6} $\forall c \in \mathbb{R} ~:~ 0 \leq c$, $\forall a,b \in \mathbb{R} ~:~ a \leq b$ tenemos que
$ac \leq bc$.\\

\begin{notacion}
    Algunas notaciones relevantes para la relación binaria descrita son:
    \begin{itemize}
        \item Diremos que $a < b$ $\Longleftrightarrow$ $a \leq b$ y $a \neq b$.
        \item Diremos que $a > b$ $\Longleftrightarrow$ $b < a$.
        \item Diremos que $a \geq b$ $\Longleftrightarrow$  $b \leq a$. 
    \end{itemize}
\end{notacion}

\vspace{0.5cm}
Las siguientes notaciones son usuales:
\begin{equation*}\begin{split}
    \mathbb{R}^{+}&=\{a \in \mathbb{R}~:~0 < a\}\\
    \mathbb{R}_0^{+}&=\{a \in \mathbb{R}~:~0 \leq a\} \\
    \mathbb{R}^{-}&=\{a \in \mathbb{R}~:~a < 0\} \\
    \mathbb{R}_0^{-}&=\{a \in \mathbb{R}~:~a \leq 0\} \\
    \mathbb{R}^{*}&=\mathbb{R}-\{0\}
\end{split}\end{equation*}

Los axiomas A, B y C se resumen diciendo que el conjunto $\mathbb{R}$ es un cuerpo conmutativo totalmente ordenado.\\

Para completar la axiomática del número real, queda por enunciar un último axioma. Este axioma, que es sin duda el
más importante, se introducirá más adelante por dos razones. La primera es la necesidad de introducir algunos conceptos
para poder dar su enunciado. Por otra parte, los axiomas del cuerpo ordenado nos permiten deducir una
amplia gama de propiedades que no necesitan el axioma que nos falta.

%########################################################################################################
% Valor absoluto
%########################################################################################################

\section{Valor absoluto}
\begin{definicion}[Valor absoluto]
Dado $a \in \mathbb{R}$, definimos el \textbf{valor absoluto de $a$}, notaremos $|a|$ como
\begin{equation*}
    |a| =
    \left\{ \begin{array}{lcc}
        ~~a ~\text{si $a \geq 0$} \\
        -a ~\text{si $a < 0$}
        \end{array}
    \right\} = \max \{a, -a\}
\end{equation*}
\end{definicion}

%########################################################################################################
% Ejercicios Axiomática de R
%########################################################################################################

\section{Ejercicios}
\begin{ejercicio}\label{ej:1.4.1}
    Probar las siguientes propiedades:
    \begin{enumerate}
        \item $\forall a \in \mathbb{R}$, $a = -(-a)$
        
        \item $\forall a,b \in \mathbb{R}$, $-(a-b)=b-a$
        
        \item $\forall a,b,c \in \mathbb{R}$, $a+b=a+c \Longleftrightarrow b=c$ (Ley de cancelación)
        
        \item $\forall a \in \mathbb{R}$, $a \cdot 0=0$. En particular, el $0$ no tiene inverso.
        
        \item Si $a,b \in \mathbb{R}$ tales que $ab=0$, entonces $a = 0 ~\lor~ b = 0$.
        
        \item \label{ej:1.4.1_6} $\forall a,b \in \mathbb{R}, ~\forall c,d \in \mathbb{R}^{*}$ tenemos que
        \begin{equation*}
            \frac{a}{b}+\frac{c}{d}=\frac{ad+bc}{bd}
        \end{equation*}
        \begin{equation*}
            \frac{a}{b} \cdot \frac{c}{d}=\frac{ac}{bd}
        \end{equation*}
        
        \item $\forall a,b \in \mathbb{R}$ tenemos que $a(-b)=-(ab)=(-a)b$ y $(-a)(-b)=ab$
    \end{enumerate}
    
    \vspace{0.5cm}
\end{ejercicio}


\begin{notacion}
    A partir de ahora, notaremos $a^{2}:=a \cdot a$ $\forall a \in \mathbb{R}$.
\end{notacion}
\begin{ejercicio}
Dados $a,b,c,d \in \mathbb{R}$, probar las siguientes propiedades:
    \begin{enumerate}
        \item $a \leq b \Longleftrightarrow -b \leq -a$

        \item $\left\{ \begin{array}{lcc}
                a < b \\
                c \leq d
            \end{array}
        \right. \Longrightarrow a+c < b+d$

        \item $\left\{ \begin{array}{lcc}
                a \leq b \\
                c \leq 0
            \end{array}
        \right. \Longrightarrow ca \geq cb$

        \item $\left\{ \begin{array}{lcc}
                a < b \\
                0 < c
            \end{array}
        \right. \Longrightarrow ac < bc$

        \item $a \cdot a > 0$ $\forall a \in \mathbb{R}^{*}$. En particular, $1 > 0$ y
        \begin{equation*}
            0 < 1 < 1+1 < 1+1+1 < \dots ~\text{($\mathbb{R}$ tiene ``muchos'' elementos)}
        \end{equation*}
        
        \item $\left\{ \begin{array}{lcc}
            a \in \mathbb{R}^{+} \Longleftrightarrow  a^{-1} \in \mathbb{R}^{+} \\
            a \in \mathbb{R}^{-} \Longleftrightarrow  a^{-1} \in \mathbb{R}^{-}
        \end{array}
        \right.$
        
        \item $0 < a < b \Longleftrightarrow 0 < \frac{1}{b} < \frac{1}{a}$
        
        \item $\left\{ \begin{array}{lcc}
                0 < a < b \\
                0 < c < d
            \end{array}
        \right. \Longrightarrow 0 < ac < bd$
        
        \item $\text{Si}~ a,b \in \mathbb{R}^{+}, ~ \text{entonces} ~ a \leq b \Longleftrightarrow a^{2} \leq b^{2}$
    \end{enumerate}

    \vspace{0.5cm}
\end{ejercicio}



\begin{ejercicio}
    Probar las siguientes propiedades del valor absoluto:
    \begin{enumerate}
        \item $\forall a \in \mathbb{R}, ~|a| \geq 0$

        \item $|a| = 0 \Longleftrightarrow a = 0$

        \item $\forall a \in \mathbb{R}, ~a \leq |a|$

        \item $\forall a \in \mathbb{R}, |-a| = |a|$

        \item $\forall a,b \in \mathbb{R}, |ab| = |a| \cdot |b|$

        \item $\forall a,b \in \mathbb{R},~b \neq 0,~|\frac{a}{b}| =\frac{|a|}{|b|}$

        \item $\text{Dados $a,b \in \mathbb{R}$}, ~|a| \leq b \Longleftrightarrow -b \leq a \leq b$

        \item $\forall a,b \in \mathbb{R},~|a+b| \leq |a| + |b| ~\text{(Desigualdad triangular)}$

        \item $\forall a,b \in \mathbb{R},~|a-b| \geq |a| - |b|$. En particular, $|a-b| \geq ||a| - |b||$ 

        \item $a^2 = (|a|)^{2} = |a^{2}| \quad \forall a \in \mathbb{R}$.


    \end{enumerate}
\end{ejercicio}
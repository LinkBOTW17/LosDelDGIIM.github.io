\chapter{Números naturales y conjuntos numerables}
Para seguir avanzando en nuestro estudio de las propiedades de los números reales, introduciremos
algunos subconjuntos importantes de $\mathbb{R}$ y algunas propiedades básicas de estos que nos permitan
caracterizarlos.

%########################################################################################################
% Conjuntos inductivos y el Principio de inducción
%########################################################################################################

\section{Conjuntos inductivos y el Principio de inducción}
Intuitivamente, el conjunto $\mathbb{N}$ de los números naturales está formado por el $1$ y por todos
los elementos que se obtienen sumando 1 consigo mismo. En un ejercicio anterior hemos justificado
que estos elementos son distintos. En particular, $\mathbb{N}$ verifica que $1 \in \mathbb{N}$ y que
dado $n \in \mathbb{N}$, se tiene que $n+1 \in \mathbb{N}$. A pesar de que no es el único conjunto de
números reales que verifica estas propiedades, notemos que $\mathbb{N}$ debe estar contenido en
todos y cada uno de los subconjuntos de $\mathbb{R}$ que las verifican. En ese sentido,
el conjunto $\mathbb{N}$ es el más pequeño de los subconjuntos de $\mathbb{R}$ que gozan de estas propiedades.

\begin{definicion}
    Se dice que un subconjunto $A \subseteq \mathbb{R}$ es \textbf{inductivo} si:
    \begin{enumerate}
        \item $1 \in A$
        \item Si $x \in A \Longrightarrow x+1 \in A$
    \end{enumerate}
\end{definicion}

\begin{definicion}
    Definimos el \textbf{el conjunto de los números naturales}, notado por $\mathbb{N}$, como el conjunto
    que resulta de la intersección de todos los subconjuntos inductivos de $\mathbb{R}$.
    \begin{equation*}
        \mathbb{N} := \{x \in \mathbb{R}~:~x \in A, ~\forall A \subseteq \mathbb{R} ~\text{inductivo}\}
    \end{equation*} 
\end{definicion}

\begin{teo}
    $\mathbb{N}$ es un conjunto inductivo y si $A \subseteq \mathbb{R}$ es inductivo, entonces $\mathbb{N} \subseteq A$.
\end{teo}
\begin{proof}
    Demostraremos únicamente la primera afirmación, ya que la segunda es evidente.
    
    Es claro que $1 \in \mathbb{N}$, pues $1 \in A, ~\forall A \subseteq \mathbb{R} ~\text{inductivo}$.
    
    Supongamos que $n \in \mathbb{N}$. Entonces $n \in A, ~\forall A ~\text{inductivo}$. Por tanto,
    $n+1 \in A, ~\forall A ~\text{inductivo}$, lo que prueba que $n+1 \in \mathbb{N}$.
\end{proof}

\begin{teo}[Principio de inducción]
    Sea $A \subseteq \mathbb{N}$. Si $A$ es inductivo, entonces $A = \mathbb{N}$.
\end{teo}
\begin{proof}
    Si $A$ es inductivo, entonces $\mathbb{N} \subseteq A$, y al darse la doble inclusión se deduce que $A = \mathbb{N} \qed$.
\end{proof}

\vspace{0.5cm}
La utilización usual del teorema anterior es la siguiente:\\

Supongamos que para cada $n \in \mathbb{N}$ se tiene una cierta afirmación $P_n$ y se quiere probar que
$P_n$ es cierta para todo $n$ natural. Para ello, basta probar que $P_1$ es cierta y que de ser cierta
$P_n$ se deduce que $P_{n+1}$ también lo es $\forall n \in \mathbb{N}$. La implicación $P_n \Longrightarrow P_{n+1}$
se conoce como \textbf{hipótesis de inducción}.\\

En efecto, definimos $A = \{n \in \mathbb{N} ~:~\text{$P_n$ es cierta}\} \subseteq \mathbb{N}$. Dado que $P_1$ es cierta y como
$P_n$ es cierta implica que $P_{n+1}$ es cierta, si $n \in A$ entonces $n+1 \in A$, luego $A$
es inductivo. Concluimos por el teorema anterior que $A = \mathbb{N}$, lo que prueba que $P_n$ es cierta para todo $n$
natural. Este razonamiento es muy usual en matemáticas y se conoce como \textbf{demostración por inducción}.

\begin{ejemplo}
    Probar que $1+2+\dots+n=\frac{n(n+1)}{2}$ para todo $n \in \mathbb{N}$.\\
    
    Sea $A = \{n \in \mathbb{N}~:~1+2+\dots+n=\frac{n(n+1)}{2}\} \subseteq \mathbb{N}$.\\
    
    Es fácil comprobar que $1 \in A$.
    
    \underline{Hipótesis de inducción:} Supongamos que para cierto $n \in \mathbb{N}$ se tiene que $n \in A$.
    Entonces:
    \begin{equation*}
        1+2+\dots+n+(n+1)=\frac{n(n+1)}{2}+(n+1)=\frac{(n+1)(n+2)}{2}
    \end{equation*}
    Por lo que $n+1 \in A$.\\
    
    Hemos probado que $A$ es inductivo, por lo que, en virtud del Principio de inducción, $A = \mathbb{N}$, o lo que
    es lo mismo, que $1+2+\dots+n=\frac{n(n+1)}{2}$ para todo $n \in \mathbb{N}$.
\end{ejemplo}

\vspace{1em}
En el siguiente resultado se resumen las propiedades más inmediatas de los números naturales.
\begin{coro}\label{coro:2.1.5} Algunas de las propiedades de $\bb{N}$ son:
    \begin{enumerate}
        \item \label{coro:2.1.5_1} $1 \leq n$, $\forall n \in \mathbb{N}$.
        \item \label{coro:2.1.5_2} Si $m,n \in \mathbb{N}$, entonces $m+n,mn \in \mathbb{N}$.
        \item \label{coro:2.1.5_3} Si $n \in \mathbb{N}$, entonces $-n$ no es natural. Si $n \in \mathbb{N}$ y $\frac{1}{n} \in \mathbb{N}$, entonces $n = 1$.
    \end{enumerate}
\end{coro}

\begin{proof}
    Demostramos cada una de las partes:
    \begin{enumerate}
        \item El conjunto $\{x \in \mathbb{R}~:~ 1 \leq x\}$ es inductivo, luego incluye a $\mathbb{N}$.
        \item Dado $m \in \mathbb{N}$ fijo y arbitrario, se demuestra fácilmente por inducción sobre $n$.
        \item Si $n \in \mathbb{N}$, entonces $0 < 1 \leq n$, luego $-n < 0$ y así $-n$ no es natural.
        \newline
        Si fuera $n \neq 1$ se tiene que $1 < n$, luego $\frac{1}{n} < 1$ y $\frac{1}{n}$ es un real no natural como
        consecuencia del apartado I).
    \end{enumerate}
\end{proof}


\begin{lema}
    Si $n \in \mathbb{N}$ y $n \neq 1$, entonces $n-1 \in \mathbb{N}$.
\end{lema}
\begin{proof}
    Sea $A=\{n \in \mathbb{N}~:~ n-1 \in \mathbb{N}\}\cup \{1\}$, probaremos que $A$ es inductivo.
    
    Que $1 \in A$ es evidente y si $x \in A$, entonces $x \in \mathbb{N}$, por lo que $x+1 \in \mathbb{N}$
    y $(x+1)-1 \in \mathbb{N}$, luego $x+1 \in A$.
\end{proof}


\begin{prop}\label{prop:2.1.7}
    Dados $m,n \in \mathbb{N}$, entonces
    \begin{equation*}
        n < m \Longleftrightarrow m-n \in \mathbb{N}
    \end{equation*}
\end{prop}
\begin{proof} Procedemos mediante doble implicación:

    \begin{description}
        \item[$\Longrightarrow)$]
            Sea $A = \{n \in \mathbb{N}~:~ m-n \in \mathbb{N}\}$.
            \newline
            \newline
            $1 \in A$ como consecuencia del lema anterior. Queremos probar que si $n \in A$ entonces
            $n+1 \in A$.
            \newline
            \newline
            Para ello, sea $n \in A$ y sea $m$ un natural tal que $n+1 < m$. Al ser $n < m$,
            deducimos que $m-n \in \mathbb{N}$ por hipótesis de inducción. Si fuera $m-n=1$ tendríamos
            que $n+1=m$ y hemos supuesto que $n+1 < m$, por lo que $m-n \neq 1$ y aplicando el lema anterior
            obtenemos que $m-(n+1)=(m-n)-1 \in \mathbb{N}$, es decir, $n+1 \in A$.
            \newline
            \newline
            Hemos probado que $A$ es inductivo, luego $A = \mathbb{N}$. Queda probado que si $m$ y $n$ son
            naturales cualesquiera tales que $n < m$, entonces $m-n \in \mathbb{N}$.
            
        \item[$\Longleftarrow)$]
            Si $m-n \in \mathbb{N}$, entonces $m-n \geq 1$, de donde se deduce que $m > n$.
    \end{description}
\end{proof}

La proposición anterior es una caracterización algebraica del orden de los naturales.\\

El siguiente resultado es una consecuencia inmediata de la anterior proposición.
\begin{coro}\label{coro.2.1.8}
    Si $m$ y $n$ son naturales cualesquiera verificando que $n < m$, entonces $n+1 \leq m$.
\end{coro}

El anterior corolario afirma que no existe ningún número natural comprendido estrictamente entre
$n$ y $n+1$. Esta propiedad se enuncia a veces diciendo que el orden de $\mathbb{N}$ es discreto.\\

Nuestro siguiente objetivo es obtener un resultado de extraordinaria importancia relativo a los
conjuntos de números naturales. Para poder enunciarlo, es preciso introducir un nuevo concepto.

\begin{definicion}[Máximo/Mínimo]
    Se dice que un conjunto de números reales $A$ tiene \textbf{máximo} cuando existe $a \in A$ tal que
    $a \geq x, ~\forall x \in A$. Análogamente, diremos que tiene \textbf{mínimo} cuando existe $a \in A$ tal que
    $a \leq x, ~\forall x \in A$. Notaremos $\max A$ y $\min A$, respectivamente. Es inmediato que, en caso de
    existir, son únicos.
\end{definicion}

Resaltamos que un conjunto de números reales puede no tener máximo y/o mínimo. El conjunto
$\{x \in \mathbb{R} ~:~ 0 < x < 1\}$ no tiene máximo ni mínimo. Sin embargo, el siguiente resultado
asegura la existencia del mínimo en determinadas circunstancias.

\begin{teo}[Principio de buena ordenación de los naturales]
    Todo conjunto no vacío $A$ de números naturales tiene mínimo.
\end{teo}
\begin{proof}
    Si $1 \in A$, tenemos que $1 = \min A$ y no hay nada que probar.
    \newline
    \newline
    De lo contrario, sea $B = \{n \in \mathbb{N} ~:~ n < a,~\forall a \in A\}$. Claramente, $1 \in B$.
    Si $B$ fuera inductivo, entonces $A = \emptyset$, luego $B$ no es inductivo. Entonces, existe
    $n \in B$ tal que $n+1 \notin B$, por lo que $n+1 \leq a,~\forall a \in A$ (Corolario \ref{coro.2.1.8}).
    Como $n+1 \in A$, pues en otro caso $n+1$ pertenecería a $B$, concluimos que $n+1 = \min A$.
\end{proof}

%########################################################################################################
% Potencias de exponente natural
%########################################################################################################

\section{Potencias de exponente natural}
Dado $x \in \mathbb{R}$, se definen las potencias de exponente natural como sigue:
\begin{gather*}
    x^1:=x\\
    x^{n+1}:=x^{n} x,~\forall n \in \mathbb{N}
\end{gather*}
Nótese que la definición dada es inductiva, pues se obtienen las sucesivas potencias de $x$ a
partir de las anteriores.\\

La siguiente proposición resume las propiedades esenciales.
\begin{prop}
    Algunas de las propiedades de las potencias de exponente natural son:
    \begin{enumerate}
        \item $x^{m+n}=x^m x^n$, $\forall x \in \mathbb{R}$, $\forall m,n \in \mathbb{N}$.

        \item $(x^m)^n=x^{mn}$, $\forall x \in \mathbb{R}$, $\forall m,n \in \mathbb{N}$.

        \item Si $1 < x$ y $m,n \in \mathbb{N}$, entonces $n < m \Leftrightarrow x^n < x^m$
    \end{enumerate}
\end{prop}
\begin{proof}
    Demostramos cada una de las propiedades:
    \begin{enumerate}
        \item Sea $m \in \mathbb{N}$ y sea $A=\{n \in \mathbb{N}~:~x^{m+n}=x^m x^n\}$.
        Por definición de potencia de exponente natural, $1 \in A$.
        
        Si $n \in A$, entonces
        \begin{equation*}
            x^{m+n+1}=x^{m+n} x=(x^m x^n) x=x^m (x^n x)=x^m x^{n+1}
        \end{equation*}
        Así, $A$ es inductivo, por lo que $A = \mathbb{N}$ como queríamos.

        \item Análogo a I).

        \item Sea $1 < x$, es fácil ver que el conjunto $B = \{n \in \mathbb{N}~:~1 < x^n\}$ es inductivo.
        Entonces, $1 < x^n$, $\forall n \in \mathbb{N}$. Si $n < m$, sabemos que $m-n \in \mathbb{N}$ y
        por tanto $1 < x^{m-n}$, luego $x^n < x^m$. Recíprocamente, supongamos que $x^n < x^m$ y $n \geq m$.
        Entonces, por lo ya demostrado, $x^n \geq x^m$ lo cual es una contradicción.
    \end{enumerate}
\end{proof}

%########################################################################################################
% Conjutos finitos e infinitos
%########################################################################################################

\section{Conjutos finitos e infinitos}
De forma intuitiva, cabría decir que dos conjuntos tienen el mismo número de elementos cuando
existe alguna aplicación biyectiva de uno sobre otro. Es igualmente intuitivo que el conjunto
$\{m \in \mathbb{N}~:~m \leq n\}$, con $n$ natural, tiene $n$ elementos. Vamos a formalizar estas
ideas intuitivas a continuación.

\begin{definicion}[Conjunto Equipotente]
    Dado un conjunto $A$, diremos que es \emph{equipotente} a otro conjunto $B$ si existe alguna biyección de $A$ sobre $B$. En tal caso, escribiremos $A \sim B$.
\end{definicion}

Es fácil ver que la relación ``ser equipotente a'' cumple las siguientes propiedades:
\begin{enumerate}
    \item \underline{Propiedad reflexiva}: $A \sim A, ~\forall A \subseteq \mathbb{R}$
    
    \item \underline{Propiedad simétrica}: $A \sim B \Longleftrightarrow B \sim A,~ \forall A,B \subseteq \mathbb{R}$
    
    \item \underline{Propiedad transitiva}:
    \begin{equation*}
        \left\{ \begin{array}{lcc}
            A \sim B \\
            B \sim C
            \end{array}
        \right.
        \Longrightarrow A \sim C, ~\forall A,B,C \subseteq \mathbb{R}
    \end{equation*}
\end{enumerate}

Estas tres propiedades se resumen diciendo que la relación de ser equipotente es una \textbf{relación de equivalencia}.\\

Dado un natural $n$, definimos el conjunto $S(n)$ como sigue:
\begin{equation*}
    S(n):=\{m \in \mathbb{N}~:~ m \leq n\}
\end{equation*}

\begin{prop}
    Sean $m,n \in \mathbb{N}$ tales que $S(m) \sim S(n)$. Entonces, $m = n$.
\end{prop}
\begin{proof}
    Para cada $n$ natural, consideramos la proposición $P_n$ siguiente: ``Si $m$ es un natural verificando $S(m) \sim S(n)$, entonces $m = n$''. Probaremos por inducción que $P_n$ es cierta para todo $n$ natural.\\
    
    Para $n=1$, si $f$ es una biyección de $S(m)$ sobre $S(1)$, entonces $f(m) = f(1) = 1$, por lo que $m=1$
    por ser $f$ inyectiva.\\
    
    Supongamos que $P_n$ es cierta y veamos que $P_{n+1}$ también lo es. Sea $h$ una biyección de $S(n+1)$ sobre $S(m)$
    y consideramos la aplicación $g:S(m) \longrightarrow S(m)$ dada por:
    \begin{equation*}
        \left\{ \begin{array}{lcc}
            g(h(n+1))=m \\ \\
            g(m)=h(n+1) \\ \\
            g(x)=x,~\forall x \in S(m)-\{m,h(n+1)\}
            \end{array}
        \right.
    \end{equation*}
    
    Es fácil ver que $g$ es biyectiva. Por tanto, la aplicación $f = g \circ h$ es una biyección de $S(n+1)$
    sobre $S(m)$ verificando
    \begin{equation*}
        f(n+1)=g(h(n+1))=m
    \end{equation*}
    
    La restricción de $f$ a $S(n)$ puede considerarse una biyección de $S(n)$ sobre $S(m-1)$
    (Notemos que $m > 1$, de lo contrario, al ser $P_1$ cierta, tendríamos $n+1=1$ lo cual es absurdo).\\
    
    Como suponemos que $P_n$ es cierta, tenemos $n=m-1 \Longleftrightarrow n+1=m$, como queríamos probar.
\end{proof}

\begin{definicion}[Conjunto finito]
    Un conjunto $A$ se dice que es \textbf{finito} si es vacío o si existe un natural $n$ tal que $A$ es equipotente a $S(n)$. Por la proposición anterior, dicho $n$ es único y diremos que $n$ es el \textbf{número de elementos del conjunto}. Convenimos que el número de elementos del conjunto vacío es $0$.
\end{definicion}

Un conjunto se dice \textbf{infinito} si no es finito.\\

La principal propiedad de los conjuntos finitos de números reales que nos interesa es la siguiente:
\begin{prop}\label{prop:2.3.4}
    Todo conjunto finito de números reales tiene máximo y mínimo.    
\end{prop}
\begin{proof}
    Hacemos inducción sobre el número de elementos del conjunto.\\
    
    Es evidente que todos los conjuntos de un elemento tienen máximo y mínimo.\\
    
    Supongamos que el resultado es cierto para conjuntos de $n$ elementos y veámoslo para conjuntos de $n+1$ elementos.
    
    Sean $A$ un conjunto con $n+1$ elementos, $f$ una biyección de $S(n+1)$ sobre $A$ tal que $a=f(n+1)$.
    Sabemos que $A \sim S(n+1)$. El conjunto $A-\{a\}$ tiene $n$ elementos (¡Compruébese!), por lo que sea $M$ su máximo y $m$ su mínimo. Para ver que $A$ tiene máximo, si $M < a$, entonces $\max A = a$, mientras que si $a < M$, entonces $\max A = M$. En cualquier caso, $A$ tiene máximo.\\
    
    El razonamiento para ver que $A$ tiene mínimo es análogo.
\end{proof}

Dado $x \in \mathbb{R}$, $x < x+1$, luego $\mathbb{R}$ no tiene máximo. El mismo razonamiento aplica para $\mathbb{N}$.
Por lo tanto:
\begin{coro}
    $\mathbb{N}$ y $\mathbb{R}$ son conjuntos infinitos.
\end{coro}

El siguiente enunciado se usa con frecuencia para decidir si un conjunto de números reales es finito.
\begin{prop}\label{prop:2.3.6} Algunas propiedades de los conjuntos finitos son:
    \begin{enumerate}
        \item\label{prop:2.3.6_1} Si $n$ es un natural, todo subconjunto de $S(n)$ es finito.

        \item Todo subconjunto de un conjunto finito es finito.

        \item Si $A$ es un conjunto no vacío y existe un natural $n$ y una aplicación inyectiva de $A$ en $S(n)$, entonces $A$ es finito.

        \item Si $A$ es un conjunto no vacío y existe un natural $n$ y una aplicación sobreyectiva de $S(n)$ sobre $A$, entonces $A$ es finito.
    \end{enumerate}
\end{prop}
\begin{proof} Demostramos cada una de las afirmaciones:
    \begin{enumerate}
        \item Hacemos inducción sobre $n$. Para $n=1$, la afirmación es evidente.
        
        Supongamos que todo subconjunto de $S(n)$ es finito y sea $B$ un subconjunto de $S(n+1)$. Si $B$ es vacío o $B = S(n+1)$,
        entonces $B$ es finito. Supongamos que $B$ es no vacío y que existe un $m \in S(n+1)$ tal que $m \notin B$.
        Definimos $f: S(n+1) \longrightarrow S(n+1)$ la aplicación dada por:
        \begin{equation*}
            \left\{ \begin{array}{lcc}
                f(n+1)=m \\ \\
                f(m)=n+1 \\ \\
                f(x)=x,~\forall x \in S(n+1)-\{m,n+1\}
                \end{array}
            \right.
        \end{equation*}
        Es fácil ver que $f$ es biyectiva luego $f(B)$ es equipotente a $B$. Dado que $m \notin B$, entonces $f(m)=n+1 \notin f(B)$,
        por lo que $f(B) \subseteq S(n)$, por lo que $f(B)$ es finito y lo mismo le ocurre a $B$.

        \item Sea $B$ un conjunto finito no vacío de $n$ elementos y sea $A$ un subconjunto no vacío de $B$.
        
        Si $f$ es una biyección de $B$ en $S(n)$, entonces $A$ y $f(A)$ son equipotentes y al ser $f(A) \subseteq S(n)$
        y aplicando I), tenemos que $A$ es finito.

        \item Sea $A$ un conjunto finito no vacío y $f:A \longrightarrow S(n)$ inyectiva. Entonces, $A$ es equipotente a $f(A) \subseteq S(n)$, luego $A$ es finito.

        \item Sea $f:S(n) \longrightarrow A$ sobreyectiva. Para cada elemento $a \in A$, el conjunto $B_a = \{m \in S(n)~:~ f(m)=a\}$ es no vacío, luego por el principio de buena ordenación de los naturales tiene mínimo. Definimos
        \begin{equation*}
            g(a) = \min B_a,~\forall a \in A
        \end{equation*}
        Claramente $g(a) \in S(n),~\forall a \in A$ luego $g$ es una aplicación de $A$ en $S(n)$ y que $f(g(a))=a,~\forall a \in A$.
        
        Veamos que $g$ es inyectiva. Sean $a,b \in A$ tales que $g(a)=g(b)$. Entonces,
        \begin{equation*}
            a=f(g(a))=f(g(b)=b
        \end{equation*}
        Por lo que $g$ es inyectiva y aplicando III), $A$ es finito.
    \end{enumerate}
\end{proof}

%########################################################################################################
% Conjuntos numerables
%########################################################################################################

\section{Conjuntos numerables}
Hasta ahora, hemos clasificado los conjuntos en infinitos y finitos, y estos últimos a su vez según el número de elementos. Pretendemos realizar ahora una clasificación de los conjuntos infinitos según su ``tamaño''. Introducimos en primer lugar a los conjuntos infinitos ``más pequeños''.

\begin{definicion}[Conjunto numerable]
    Un conjunto $A$ se dice \textbf{numerable} si es vacío o si existe alguna aplicación inyectiva de $A$ en $\mathbb{N}$.
\end{definicion}

Es fácil ver que todo conjunto finito, $\mathbb{N}$ y todo conjunto infinito equipotente a $\mathbb{N}$ son numerables. Veremos enseguida que no existen más conjuntos numerables que los citados. Para ello, veamos el siguiente lema:
\begin{lema}\label{lema:2.4.2}
    Todo conjunto infinito de números naturales es equipotente a $\mathbb{N}$. De hecho, si $A$ es un tal  conjunto, existe una biyección $f$ de $\mathbb{N}$ sobre $A$ verificando que si $n < m$ entonces $f(n) < f(m)$.
\end{lema}
\begin{proof}
    Sea $a \in A$; el conjunto $B_a=\{x \in A~:~ a < x\}$ es no vacío, pues si lo fuese entonces $A \subseteq S(a)$, por lo que $A$ sería finito por la proposición \ref{prop:2.3.6}.\ref{prop:2.3.6_1}. Teniendo en cuenta el principio de buena ordenación de $\mathbb{N}$, definimos para cada $a \in A$
    \begin{equation*}
        g(a)=\min B_a
    \end{equation*}
    Por lo que $g:A\longrightarrow A-\{\min A\}$ es una aplicación, verificando que
    \begin{equation*}
        a < g(a), ~\forall a \in A
    \end{equation*}
    Veamos que $g$ es sobreyectiva (De hecho, es biyectiva).\\
    
    Sea $a \in A-\{\min A\}$. Como el conjunto $B^a=\{x \in A~:~ x< a\}$ es finito y no vacío ($B^a \subseteq S(a)$), la proposición \ref{prop:2.3.4} nos asegura que dicho conjunto tiene máximo. Sea $b = \max B^a$, por lo que $b \in A$ y $b < a$. Por ser $g(b)=\min\{x \in A~:~ b < x\}$, tenemos que $g(b) \leq a$.\\
    
    Por otra parte, al ser $\max B^a = b < g(b)$ se tiene que $g(b) \notin B^a$ y al ser $g(b) \in A$, concluimos que $a \leq g(b)$. Así, $a = g(b)$ y $g$ es sobreyectiva.\\
    
    Definimos $f:\mathbb{N} \longrightarrow A$ como sigue:
    \begin{equation*}
        \left\{ \begin{array}{lcc}
            f(1)=\min A \\ \\
            f(n+1)=g(f(n)),~\forall n \in \mathbb{N}
            \end{array}
        \right.
    \end{equation*}
    
    Es claro que $f(n) < f(n+1)$, $\forall n \in \mathbb{N}$ y es fácil probar por inducción que
    si $n$ es un número natural, $f(n) < f(n+p)$, $\forall p \in \mathbb{N}$.
    Entonces (ver proposición \ref{prop:2.1.7}):
    \begin{equation*}
        n,m \in \mathbb{N},~n < m \Longrightarrow f(n) < f(m)
    \end{equation*}
    Y en particular, $f$ es inyectiva (¿Por qué?).\\
    
    Queda probar que $f$ es sobreyectiva. Si $A - \{f(\mathbb{N})\}$ no fuese vacío, sea $c$ su mínimo.
    Como $\min A = f(1)$, entonces $c \neq \min A$, por lo que existe $k \in A$ tal que $g(k) = c$.
    Como $k < g(k) = c$, existe un natural $n$ tal que $f(n) = k$, pero entonces
    \begin{equation*}
        f(n+1)=g(f(n))=g(k)=c
    \end{equation*}
    Lo cual es una contradicción.
\end{proof}

\begin{teo}\label{teo:2.4.3}
    Si $A$ es un conjunto numerable, entonces es finito o equipotente a $\mathbb{N}$.
\end{teo}
\begin{proof}
    Si $A$ es vacío, entonces es finito. En otro caso sea $f:A \longrightarrow \mathbb{N}$ una aplicación inyectiva. Es claro que $A$ y $f(A)$ son equipotentes. Si $f(A)$ es finito, entonces $A$ también lo es.
    Si fuera infinito, $f(A) \sim \mathbb{N}$ por el lema anterior, de donde $A \sim \mathbb{N}$.
\end{proof}

La consecuencia fundamental del anterior teorema es que todo conjunto infinito numerable es equipotente a $\mathbb{N}$. Intuitivamente, esto quiere decir que $\mathbb{N}$ es el más ``pequeño'' conjunto infinito que existe. Aunque aún no es posible probar la existencia de conjuntos no numerables, adelantamos, sin demostrar,
que $\mathbb{R}$ no es numerable.
\begin{teo}
    El conjunto $\mathbb{R}$ de los números reales es no numerable.
\end{teo}

%########################################################################################################
% Ejercicios de números naturales y conjuntos numerables
%########################################################################################################

\section{Ejercicios}
\begin{ejercicio} Pruébense las siguientes afirmaciones:
    \begin{enumerate}
        \item $\forall x,y \in \mathbb{R}$, $\forall n \in \mathbb{N}$ tenemos que
        \begin{equation*}
            (x+y)^n=\binom{n}{0}x^n+\binom{n}{1}x^{n-1}y+\dots+\binom{n}{n-1}x y^{n-1} + \binom{n}{n}y^n
        \end{equation*}

        \item Si $0 < x < 1$ y $m,n \in \mathbb{N}$, entonces
        \begin{equation*}
            n < m \Longleftrightarrow x^n > x^m
        \end{equation*}

        \item Probar que:
        \begin{equation*}
            1+1+2+2^2+2^3+\dots+2^n=2^{n+1}, ~\forall n \in \mathbb{N}
        \end{equation*}

        \item Probar que:
        \begin{equation*}
            1+3+5+\dots+(2n-1)=n^2, ~\forall n \in \mathbb{N}
        \end{equation*}

        \item La aplicación $f:\mathbb{N} \longrightarrow \mathbb{N}-\{1\}$ dada por
        \begin{equation*}
            f(n)=n+1,~\forall n \in \mathbb{N}
        \end{equation*}
        es biyectiva.
    \end{enumerate}
\end{ejercicio}


\begin{ejercicio}
    Probar que todo subconjunto de un conjunto numerable es numerable.
\end{ejercicio}

\begin{ejercicio}\label{ej:2.5.3}
    Probar que si $A$ es un conjunto no vacío, es numerable si, y sólo si, existe una aplicación sobreyectiva de $\mathbb{N}$ sobre $A$.
\end{ejercicio}

\begin{ejercicio}\label{ej:2.5.4}
    Probar que $\mathbb{N} \times \mathbb{N}$ es numerable.
\end{ejercicio}

\begin{ejercicio}
    Probar que si $A$ y $B$ son conjuntos numerables, entonces $A \cup B$ es numerable.
\end{ejercicio}
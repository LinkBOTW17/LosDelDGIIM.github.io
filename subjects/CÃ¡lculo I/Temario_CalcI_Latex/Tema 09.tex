\chapter{Series de números reales}\label{chp:Tema9}

%########################################################################################################
% Introducción a las series de números reales.
%########################################################################################################

\section{Introducción a las series de números reales}
\begin{definicion}
    Sea $\{a_n\}$ una sucesión de números reales cualquiera. Definimos la sucesión $\{A_n\}$ como sigue:
    \begin{equation*}
        A_1 = a_1, ~ A_{n+1} = A_n + a_{n+1}, ~\forall n \geq 2
    \end{equation*}
    
    A la sucesión $\{A_n\}$ la llamaremos \textbf{serie de término general $a_n$}. Usualmente, la notaremos por:
    \begin{equation*}
        \sum_{n \geq 1} a_n
    \end{equation*}
    Al término $A_n = \displaystyle\sum_{k = 1}^{n} a_k$ lo llamaremos \textbf{suma parcial}.

    Cuando la serie sea convergente, escribiremos
    \begin{equation*}
        \lim \{A_n\} = \lim_{n \to \infty} \displaystyle\sum_{k = 1}^{n} a_k = \sum_{n = 1}^{\infty} a_n 
    \end{equation*}
    y será denominado por \textbf{suma de la serie} o \textbf{límite de la serie}.

\end{definicion}

Veamos ahora algunos ejemplos de series de números reales:

%########################################################################################################
% Algunas series importantes y primeros resultados.
%########################################################################################################

%

\section{Algunas series importantes y primeros resultados}
\noindent\textbf{Series geométricas\footnote{ Ver ejercicios \ref{ej:6.5.3} , \ref{ej:6.5.4}, \ref{ej:6.5.5}, \ref{ej:6.5.6}.}:}
$\displaystyle \sum_{n \geq 1} r^n,\quad r \in \mathbb{R}$

\begin{equation*}
    A_k = \displaystyle\sum_{n = 1}^{k} r^n = \frac{1-r^{k+1}}{1-r}
\end{equation*}

La serie geométrica converge si y sólo si $|r| < 1$ y en tal caso se cumple que
\begin{equation*}
    \sum_{n = 1}^{\infty} r^n = \frac{1}{1-r}
\end{equation*}

\newpage
\vspace{0.5cm}\noindent\textbf{Serie armónica:}
$\displaystyle \sum_{n \geq 1} \frac{1}{n}$\\

La serie armónica no es convergente. Para verlo, notemos que
\begin{equation*}
    A_8 = 1+\frac{1}{2}+\left(\frac{1}{3}+\frac{1}{4}\right)+\left(\frac{1}{5}+\frac{1}{6}+\frac{1}{7}+\frac{1}{8}\right) > 1+\frac{1}{2}+\frac{1}{2}+\frac{1}{2} = \frac{5}{2}
\end{equation*}

En general, es fácil probar que $A_{2^n} \geq 1 + \frac{n}{2}$ para todo $n$ natural, por lo que
$\{A_{2^n}\} \longrightarrow +\infty$, y entonces $\{A_n\} \longrightarrow +\infty$.

\vspace{0.5cm}\noindent\textbf{Serie armónica alternada:} $\displaystyle\sum_{n \geq 1} \frac{(-1)^{n+1}}{n}$.\\

La serie armónica alternada sí es convergente. La demostración se verá más adelante.

\vspace{0.5cm}\noindent\textbf{Serie telescópica:} $\displaystyle\sum_{n \geq 1} \left(\frac{1}{n} - \frac{1}{n+1}\right)$.\\

La serie telescópica sí es convergente. Para verlo, notemos que
\begin{multline*}
    \displaystyle\sum_{n = 1}^{\infty} \left(\frac{1}{n} - \frac{1}{n+1}\right) = \lim_{n \to \infty} \displaystyle\sum_{k = 1}^{n} \left[ \left(1 - \frac{1}{2}\right) + \left(\frac{1}{2} - \frac{1}{3}\right) + \dots + \left(\frac{1}{n} - \frac{1}{n+1}\right)\right]=\\
    = \lim_{n \to \infty} \displaystyle\sum_{k = 1}^{n} \left[ 1 + \left(-\frac{1}{2} + \frac{1}{2}\right) + \left(-\frac{1}{3} + \frac{1}{3}\right) + \dots + \left(-\frac{1}{n} + \frac{1}{n}\right) - \frac{1}{n+1}\right] =\\
    =\lim_{n \to \infty} \left( 1- \frac{1}{n+1} \right) = 1
\end{multline*}

por lo que la serie telescópica es convergente con $\displaystyle\sum_{n = 1}^{\infty} \left(\frac{1}{n} - \frac{1}{n+1}\right) = 1$.

\begin{prop}
    Sean $\{a_n\}$ y $\{b_n\}$ dos sucesiones de números reales tales que $\exists m \in \mathbb{N}$, $m >1$, tal que, si $n \in \mathbb{N}$, $n \geq m$, entonces $a_n = b_n$. Entonces:
    \begin{equation*}
        \sum_{n \geq 1} a_n ~ \text{converge} \Longleftrightarrow
        \sum_{n \geq 1} b_n ~ \text{converge}
    \end{equation*}
    en cuyo caso
    \begin{equation*}
        \sum_{n = 1}^{\infty} a_n - \sum_{n = 1}^{m-1} a_n =
        \sum_{n = 1}^{\infty} b_n - \sum_{n = 1}^{m-1} b_n
    \end{equation*}
\end{prop}
\begin{proof}
    Teniendo en cuenta que para todo $n \geq m$ se verifica la igualdad
    \begin{equation*}
        \sum_{k = m}^{n} a_k
        = \sum_{k = 1}^{n} a_k - \sum_{k = 1}^{m-1} a_k 
        = \sum_{k = m}^{n} b_k
        = \sum_{k = 1}^{n} b_k - \sum_{k = 1}^{m-1} b_k
    \end{equation*}
    deducimos que
    \begin{equation*}
        \sum_{n \geq 1} a_n ~ \text{converge} \Longleftrightarrow 
        \sum_{n \geq 1} b_n ~ \text{converge}
    \end{equation*}
    en cuyo caso, tenemos que
    \begin{equation*}
        \sum_{n = 1}^{\infty} a_n - \sum_{n = 1}^{m-1} a_n
        = \sum_{n = 1}^{\infty} b_n - \sum_{n = 1}^{m-1} b_n
    \end{equation*}
    como queríamos probar.
\end{proof}


En particular, dado $m \in \mathbb{N}$, $m >1$, hemos probado que
\begin{equation*}
    \displaystyle\sum_{n \geq 1} a_n ~ \text{converge} \Longleftrightarrow \displaystyle\sum_{n \geq m} a_n ~ \text{converge}
\end{equation*}
La proposición anterior nos dice que la convergencia de una serie no se ve afectada al modificar un número finito de términos, aunque (como es lógico) la suma de dicha serie si se verá afectada.

\begin{prop}[Condición necesaria de convergencia de una serie]
    \begin{equation*}
       \text{Si} ~ \displaystyle\sum_{n \geq 1} a_n ~ \text{converge} \Longrightarrow \{a_n\} \longrightarrow 0
    \end{equation*}
\end{prop}
\begin{proof}
    Supongamos que $\{A_n\} \longrightarrow L \in \mathbb{R}$. Entonces, tenemos que
    \begin{equation*}
        \{a_n\} = \{A_{n+1} - A_n\} \longrightarrow 0
    \end{equation*}
    como queríamos demostrar.
\end{proof}
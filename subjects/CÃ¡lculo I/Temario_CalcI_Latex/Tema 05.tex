\chapter{Sucesiones de números reales}\label{chp:Tema5}

En este tema nos introduciremos al estudio de las sucesiones de números reales, siendo nuestro principal objetivo conseguir una familiaridad con el concepto de convergencia para dichas sucesiones, uno de los conceptos básicos del Análisis Matemático, que será la herramienta fundamental para el estudio posterior de las funciones reales de variable real.


%########################################################################################################
% Sucesiones convergentes.
%########################################################################################################

\section{Definición y propiedades básicas.}
\begin{definicion}
    Sea $A$ un conjunto no vacío. Una \textbf{sucesión de elementos de $A$} es una aplicación de $\mathbb{N}$ en $A$. En particular, una \textbf{sucesión de números reales} es una aplicación de $\mathbb{N}$ en $\mathbb{R}$.
\end{definicion}

Nótese que para definir una sucesión de números reales basta con asociar a cada número natural un número real. 
Así, dado un natural $n$, al número real $f(n)$, que notaremos por $x_n$, lo llamaremos \textbf{término n-ésimo} de la sucesión $f:\mathbb{N} \longrightarrow \mathbb{R}$ dada por
\begin{equation*}
    f(n)=x_n,~\forall n \in \mathbb{N}
\end{equation*}
En vez de referirnos a esta sucesión por $f$, emplearemos la notación $\{x_n\}$ para aludir a ella.
Así, por ejemplo, $\left\{\frac{1}{n}\right\}$ es la sucesión $f:\mathbb{N} \longrightarrow \mathbb{R}$ dada por
\begin{equation*}
    f(n)=\frac{1}{n},~\forall n \in \mathbb{N}
\end{equation*}

Antes de seguir, merece la pena remarcar que no debe confundirse la sucesión $\{x_n\}$ con el conjunto
de sus términos, $\{x_n~:~n \in \mathbb{N}\}$. Por ejemplo, las sucesiones $\{x_n\}$ e $\{y_n\}$ definidas
por
\begin{gather*}
    x_1 = 0;~x_n = 1,~\forall n \geq 2\\
    y_1 = 1;~y_n = 0,~\forall n \geq 2
\end{gather*}
son sucesiones diferentes\footnote{Evidentemente, diremos que dos sucesiones, $\{x_n\}$ e $\{y_n\}$,
son iguales si $x_n = y_n,~\forall n \in \mathbb{N}$.}, pero
\begin{equation*}
    \{x_n~:~n \in \mathbb{N}\} = \{y_n~:~n \in \mathbb{N}\} = \{0,1\}
\end{equation*}
por lo que los conjuntos de los términos de ambas sucesiones coinciden.

\begin{definicion}
    Se dice que una sucesión de números reales  $\{x_n\}$ es \textbf{convergente al número real $x\in \bb{R}$} si:
    \begin{equation*}
        \forall \varepsilon > 0 ~\exists m \in \mathbb{N} ~:~ \text{Si $n \in \mathbb{N}$},~n \geq m \Longrightarrow \left| x_n - x \right| < \varepsilon
    \end{equation*}
\end{definicion}

En otras palabras, dado $\varepsilon > 0$, se tiene que a partir de un término $x_m$ en adelante (en general, $m$ depende del término
$\varepsilon$ escogido) la distancia entre $x$ y un término $x_n$ cualquiera tal que $n \geq m$ es menor que $\varepsilon$.

\begin{ejemplo}
    Probar que $\{x_n\}=\left\{\dfrac{1}{n}\right\}$ converge a 0.
    \begin{equation*}
        \forall \varepsilon > 0,\quad \text{Sea } m \in \mathbb{N},~m>\frac{1}{\varepsilon} ~:~ \text{Si $n \in \mathbb{N}$},~n \geq m \Longrightarrow \left|\frac{1}{n} -0\right| = \frac{1}{n} < \varepsilon \Longleftrightarrow \frac{1}{\varepsilon} < n
    \end{equation*}
    lo cual es cierto por la elección de $m$, y sabiendo que $n\geq m$.\\
\end{ejemplo}


A continuación, veremos que si existe tal $x$ de la definición anterior, necesariamente es único. Diremos entonces que
$x$ es el \textbf{límite de la sucesión $\{x_n\}$} o que la sucesión $\{x_n\}$ \textbf{converge}
a $x$, escribiremos $x= \lim \{x_n\}$ o también $\{x_n\}\longrightarrow x$.
\begin{lema}\label{lema:5.1.3}
    Sea $\{x_n\}$ una sucesión de números reales y sea $x\in \bb{R}$. Entonces:
    $$\{x_n\} \longrightarrow  x \Longleftrightarrow \{x_n - x\} \longrightarrow 0$$
\end{lema}
\begin{proof}
    Es trivial, ya que la expresión de la convergencia de ambas sucesiones a su límite es idéntica.
    \begin{equation*}
        \forall \varepsilon > 0 ~\exists m \in \mathbb{N} ~:~ \text{Si $n \in \mathbb{N}$},~n \geq m \Longrightarrow \left| x_n - x\right| < \varepsilon
    \end{equation*}
\end{proof}

\begin{prop}[Unicidad del límite de una sucesión convergente]
    Sean $x,y \in \mathbb{R}$. Si $\{x_n\}$ es una sucesión de números reales tal que $\{x_n\}\longrightarrow x$ y $\{x_n\}\longrightarrow y$, entonces $x=y$.
\end{prop}
\begin{proof}
    Por reducción al absurdo, supongamos que $x \neq y$. Sea $\varepsilon = \frac{\left| x-y \right|}{2}$.
    Por ser $\{x_n\} \longrightarrow x$, tenemos que
    \begin{equation*}
        \exists m_1 \in \mathbb{N} ~:~ \text{Si $n \in \mathbb{N}$},~n \geq m_1 \Longrightarrow \left| x_n - x \right| < \varepsilon
    \end{equation*}
    y por ser $\{x_n\} \longrightarrow y$, tenemos que
    \begin{equation*}
        \exists m_2 \in \mathbb{N} ~:~ \text{Si $n \in \mathbb{N}$},~n \geq m_2 \Longrightarrow \left| x_n - y \right| < \varepsilon
    \end{equation*}
    Sea $m=\max \{m_1,m_2\}$. Si $n \geq m$, entonces:
    \begin{equation*}
        \left| x-y \right| = \left| x-x_n+x_n-y \right| \leq \left| x-x_n \right| + \left|x_n-y \right| < 2 \varepsilon = \left| x-y \right|
    \end{equation*}
    Lo que es una contradicción. Por tanto, ha de ser $x=y$, como queríamos probar.
\end{proof}

\begin{prop}
    Sea $x \in \mathbb{R}$ y $\{x_n\}$ una sucesión de números reales tal que $\{x_n\}\longrightarrow x$.
    Entonces $\{ \left| x_n \right| \} \longrightarrow \left| x \right|$.
\end{prop}
\begin{proof}
    Dado $\varepsilon > 0$, existe $m \in \mathbb{N}$ tal que si $n \in \mathbb{N}$ y $n \geq m$, entonces
    $\left| |x_n|-|x| \right| \leq \left| x_n-x \right| < \varepsilon$, lo que prueba que
    $\{ \left| x_n \right| \} \longrightarrow \left| x \right|$.
\end{proof}

Es fácil probar que el recíproco no es cierto. Basta con considerar la sucesión $\{(-1)^n\}$ y ver que no es convergente, pero la sucesión $\{|(-1)^n|\}$=$\{1\}$ que claramente converge a $1$ por ser constante, como podemos ver a continuación.

\begin{definicion}
    Se dice que una sucesión $\{x_n\}$  es \textbf{constante} si $\exists a \in \mathbb{R}$ tal que $x_n=a,~\forall n \in \mathbb{N}$.
\end{definicion}
\begin{lema}
    Toda sucesión constante $\{x_n\}=\{a\}$ es convergente, con $\lim \{x_n\} = a$.
\end{lema}
\begin{proof}
    Por la definición de convergencia:
    \begin{equation*}
        \forall \varepsilon > 0 ~\exists m \in \mathbb{N} ~:~ \text{Si $n \in \mathbb{N}$},~n \geq m \Longrightarrow \left| a - a \right|=0 < \varepsilon
    \end{equation*}
\end{proof}


%########################################################################################################
% Sucesiones acotadas.
%########################################################################################################

\section{Sucesiones acotadas}
\begin{definicion}
    Sean $\{x_n\}$ e $\{y_n\}$ dos sucesiones de números reales.
    \begin{itemize}
        \item Diremos que la sucesión $\{x_n\}$ está \textbf{mayorada} si el conjunto $\{x_n~:~n \in \mathbb{N}\}$ está mayorado.

        \item Diremos que la sucesión $\{x_n\}$ está \textbf{minorada} si el conjunto $\{x_n~:~n \in \mathbb{N}\}$ está minorado.

        \item Diremos que la sucesión $\{x_n\}$ está \textbf{acotada} si está mayorada y minorada.
    \end{itemize}
\end{definicion}

\begin{prop}
    Toda sucesión de números reales convergente está acotada.
\end{prop}
\begin{proof}
    Sea $\{x_n\}\longrightarrow x$. Para $\varepsilon = 1$, tenemos que
    \begin{equation*}
        \exists m \in \mathbb{N}~:~ n \geq m \Longrightarrow  \left| x_n - x \right| < 1
    \end{equation*}
    Con lo que para $n \geq m$:
    \begin{equation*}
        \left| x_n \right| = \left| x_n - x + x\right| \leq \left| x_n - x \right| + \left| x \right| < 1+ \left| x \right|
    \end{equation*}
    Lo que prueba que el conjunto $\{x_n ~:~ n \geq m \}$ está acotado. Ello implica que $\{x_n\}$ es una sucesión acotada,
    pues el conjunto $\{x_n ~:~ n < m \}$ es finito, luego está acotado (Proposición \ref{prop:2.3.4}). Por lo tanto (ejercicio \ref{ej:4.4.3}), el conjunto
    \begin{equation*}
        \{x_n ~:~ n \in \mathbb{N} \} = \{x_n ~:~ n < m \} \cup \{x_n ~:~ n \geq m \}
    \end{equation*}
    está acotado.
\end{proof}

Nótese que el recíproco de la proposición anterior no es cierto. La sucesión $\{(-1)^n\}$ no es convergente aunque está acotada.

%########################################################################################################
% Propiedades de las sucesiones convergentes.
%########################################################################################################

\section{Propiedades de las sucesiones convergentes}
A continuación, vamos a analizar la relación entre la convergencia de sucesiones de números reales con las $3$ estructuras básicas de la axiomática de $\mathbb{R}$: Suma, producto y orden. Se obtendrán las primeras reglas elementales para el cálculo de límites de sucesiones de números reales.
\begin{itemize}
    \item Definimos la sucesión suma de $\{x_n\}$ e $\{y_n\}$ como una nueva sucesión $\{z_n\}$ tal que
    \begin{equation*}
        z_n=x_n + y_n,~ \forall n \in \mathbb{N}
    \end{equation*}

    \item Definimos la sucesión opuesta de $\{x_n\}$ como una nueva sucesión $\{z_n\}=\{-x_n\}$ tal que
    \begin{equation*}
        z_n=-x_n,~ \forall n \in \mathbb{N}
    \end{equation*}

    \item Definimos la sucesión producto de $\{x_n\}$ e $\{y_n\}$ como una nueva sucesión $\{z_n\}$ tal que
    \begin{equation*}
        z_n=x_n \cdot y_n,~ \forall n \in \mathbb{N}
    \end{equation*} 

    \item Si $x_n \neq 0, ~ \forall n \in \mathbb{N}$, definimos la sucesión inversa como una nueva sucesión $\{z_n\}$ tal que
    \begin{equation*}
        z_n=\frac{1}{x_n}, \forall n \in \mathbb{N}
    \end{equation*}
\end{itemize}

\begin{prop}\label{prop:5.3.1}
    Sean $x,y \in \mathbb{R}$ y sean $\{x_n\} \longrightarrow x$ e $\{y_n\} \longrightarrow y$. Entonces, se tiene que $\{x_n + y_n\} \longrightarrow x+y$.
\end{prop}
\begin{proof}
    Fijado $\varepsilon > 0$, por ser $\{x_n\}$ e $\{y_n\}$ convergentes, se tiene que
    \begin{gather*}
        \exists m_1 \in \mathbb{N} ~:~ n \in \mathbb{N},~ n \geq m_1 \Longrightarrow \left|x_n - x \right| < \frac{\varepsilon}{2} \\
        \exists m_2 \in \mathbb{N} ~:~ n \in \mathbb{N},~ n \geq m_2 \Longrightarrow \left|y_n - y \right| < \frac{\varepsilon}{2}
    \end{gather*}
    Sea $m = \max \{m_1,m_2\}$. Si $ n \geq m$, entonces:
    \begin{equation*}
        \left|x_n + y_n - (x+y) \right| = \left|x_n - x + y_n - y \right| \leq |x_n - x| + |y_n - y| < \frac{\varepsilon}{2} + \frac{\varepsilon}{2} = \varepsilon
    \end{equation*}
    Por lo que $\{x_n + y_n\} \longrightarrow x+y$.
\end{proof}

\begin{prop}
    Sea $x \in \mathbb{R}$ y sea $\{x_n\} \longrightarrow x$. Entonces, se tiene que $\{-x_n\} \longrightarrow~-~x$.
\end{prop}
\begin{proof}
    Fijado $\varepsilon > 0$, por ser $\{x_n\}$ convergente, se tiene que
    \begin{equation*}
        \exists m_m \in \mathbb{N} ~:~ n \in \mathbb{N},~ n \geq m_1 \Longrightarrow \left|x_n - x \right| < \varepsilon
    \end{equation*}
    
    Para el mismo $m$, se tiene que:
    \begin{equation*}
        |-x_n -(-x)| = |-(x_n - x)| = |x_n - x| < \varepsilon
    \end{equation*}
    Por lo que $\{-x_n\} \longrightarrow -x$.
\end{proof}

\begin{prop}\label{prop:5.3.2}
    Sean $\{x_n\} \longrightarrow 0$ e $\{y_n\}$ una sucesión acotada. Entonces, se tiene que $\{x_n \cdot y_n\} \longrightarrow 0$.
\end{prop}
\begin{proof}
    Por ser $\{y_n\}$ acotada, vemos que
    \begin{equation*}
        \exists M > 0 ~:~ |y_n| < M, ~ \forall n \in \mathbb{N}.
    \end{equation*}
    Por otra parte, dado $\varepsilon > 0$, por ser $\{x_n\} \longrightarrow 0$,
    \begin{equation*}
        \exists m \in \mathbb{N} ~:~ n \in \mathbb{N},~ n \geq m \Longrightarrow |x_n| < \frac{\varepsilon}{M}.
    \end{equation*}
    Por tanto, si $ n \geq m$, tenemos que
    \begin{equation*}
        | x_n \cdot y_n | = |x_n| \cdot |y_n| < \frac{\varepsilon}{M} \cdot M = \varepsilon
    \end{equation*}
    Por lo que $\{x_n \cdot y_n\} \longrightarrow 0$.
\end{proof}

\begin{prop}\label{prop:5.3.3}
    Sean $x,y \in \mathbb{R}$ y sean $\{x_n\} \longrightarrow x$ e $\{y_n\} \longrightarrow y$. Entonces, se tiene que $\{x_n y_n\} \longrightarrow x y$.
\end{prop}
\begin{proof}
    \begin{equation*}
    \{x_n y_n - xy\} = \{x_n y_n -x_n y + x_n y - x y\} = \{x_n(y_n - y)\} + \{(x_n-x)y\} \longrightarrow 0 + 0 = 0
    \end{equation*}
    Donde hemos usado la proposición anterior para ver que la sucesión $\{x_n(y_n - y)\}$ converge a $0$
    y que $\{ x_n \} \longrightarrow x \Longleftrightarrow \{ x_n -x\} \longrightarrow 0$ (lema \ref{lema:5.1.3}) en la segunda sucesión.
    \newline
    \newline
    Por lo tanto, $\{x_n y_n\} \longrightarrow x y$.
\end{proof}

\begin{prop}
    Sea $x \neq 0$ y $\{x_n\}$ una sucesión de números reales tal que $\{x_n\} \longrightarrow x$ y $x_n \neq 0$ para todo natural $n$. Entonces, $\left\{\dfrac{1}{x_n}\right\} \longrightarrow \dfrac{1}{x}$.
\end{prop}
\begin{proof}
    Al ser $\{x_n\} \longrightarrow x$ y $x_n \neq 0$, tenemos que:
    \begin{equation*}
        \exists m \in \mathbb{N} ~:~ n \geq m \Longrightarrow |x_n-x| < \frac{|x|}{2}
    \end{equation*}
    De donde si $n \geq m$ se verifica
    \begin{equation*}
        |x_n| = |x-(x-x_n)| \geq |x| - |x - x_n| > \frac{|x|}{2}
    \end{equation*}
    por lo que $\frac{1}{|x_n|} < \frac{2}{|x|}$, de donde deducimos que la sucesión $\{\frac{1}{x_n}\}$ está acotada.
    Finalmente, vemos que
    \begin{equation*}
        \left\{\frac{1}{x_n} - \frac{1}{x}\right\}=\left\{\frac{x-x_n}{x} \frac{1}{x_n}\right\}
    \end{equation*}
    Y al ser $\left\{\dfrac{x-x_n}{x} \dfrac{1}{x_n}\right\} \longrightarrow 0$, tenemos que $\left\{\dfrac{1}{x_n}\right\} \longrightarrow \dfrac{1}{x}$.
\end{proof}

\begin{coro}
    Sean $\{x_n\} \longrightarrow x$ e $\{y_n\} \longrightarrow y$, con $y \neq 0$ y $y_n \neq 0$ para todo natural $n$. Entonces, $\left\{\dfrac{x_n}{y_n}\right\} \longrightarrow \dfrac{x}{y}$.
\end{coro}

\begin{prop}
    Sean $\{x_n\} \longrightarrow x$ e $\{y_n\} \longrightarrow y$. Si $x < y$, entonces existe $m \in \mathbb{N}$ tal que si $n \geq m$, entonces $x_n < y_n$.
\end{prop}
\begin{proof}
    Sean $x < y$ y sea $\varepsilon = \frac{y-x}{2}$. Notemos que $x+\varepsilon = y-\varepsilon$.
    Por convergencia de $\{x_n\}$ y $\{y_n\}$, entonces:
    \begin{equation*}
        \exists m_1 \in \mathbb{N} ~:~ n \in \mathbb{N}, ~n \geq m_1 \Longrightarrow x-\varepsilon < x_n < x+\varepsilon
    \end{equation*}
    \begin{equation*}
        \exists m_2 \in \mathbb{N} ~:~ n \in \mathbb{N}, ~n \geq m_2 \Longrightarrow y-\varepsilon < y_n < y+\varepsilon
    \end{equation*}
    Sea $m = \max \{m_1, m_2\}$. Si $n \geq m$, entonces:
    \begin{equation*}
        x-\varepsilon < x_n < x+\varepsilon = y-\varepsilon < y_n < y+\varepsilon
    \end{equation*}
    Por lo que $x_n < y_n$ para todo $n \geq m$.
\end{proof}

\begin{lema}[Lema de sucesiones encajadas o ``Lema del sandwich'']\label{lema:5.3.5}
    Sea $x \in \mathbb{R}$. Sean $\{x_n\}$, $\{y_n\}$ y $\{z_n\}$ sucesiones de números reales tales que
    $\{x_n\} \longrightarrow x$, $\{z_n\} \longrightarrow~x$ y existe $m_0 \in \mathbb{N}$ tal que
    $x_n < y_n < z_n$ para todo $n \geq m_0$. Entonces, $\{y_n\} \longrightarrow x$.
\end{lema}
\begin{proof}
    Por convergencia de $\{x_n\}$ y $\{z_n\}$, entonces:
    \begin{gather*}
        \exists m_1 \in \mathbb{N} ~:~ n \in \mathbb{N}, ~n \geq m_1 \Longrightarrow x-\varepsilon < x_n < x+\varepsilon,\\
        \exists m_2 \in \mathbb{N} ~:~ n \in \mathbb{N}, ~n \geq m_2 \Longrightarrow x-\varepsilon < z_n < x+\varepsilon.
    \end{gather*}
    Por hipótesis,
    \begin{equation*}
        \exists m_0 \in \mathbb{N} ~:~ n \in \mathbb{N}, ~n \geq m_0 \Longrightarrow x_n < y_n < z_n.
    \end{equation*}
    Sea $m = \max \{m_0, m_1, m_2\}$. Si $n \geq m$, entonces:
    \begin{equation*}
        x-\varepsilon < x_n < y_n < z_n < x + \varepsilon \Longrightarrow x-\varepsilon < y_n < x + \varepsilon
    \end{equation*}
    Por lo que $\{y_n\} \longrightarrow x$.
\end{proof}

\begin{prop}[Caracterización del supremo con sucesiones]
    Sea $A$ un conjunto de números reales no vacío y mayorado.
    Entonces, $\alpha = \sup A$ si, y sólo si, $\alpha \in M(A)$ y existe una sucesión de elementos de $A$ convergente a $\alpha$.
\end{prop}
\begin{proof} Procedemos mediante doble implicación:
\begin{description}
    \item[$\Longrightarrow)$]
        La sucesión $\left\{\alpha - \frac{1}{n}\right\}$ converge a $\alpha$. Por ser $\alpha$ el supremo de $A$,
        tenemos que
        \begin{equation*}
            \forall n \in \mathbb{N} ~ \exists a_n \in A ~:~ \alpha - \frac{1}{n} < a_n \leq \alpha
        \end{equation*}
        y por el lema anterior, $\{a_n\} \longrightarrow \alpha$.
    
    \item[$\Longleftarrow)$]
        Sea $\alpha$ un mayorante de $A$ y sea $\{a_n\}$ una sucesión de elementos de $A$
        convergente a $\alpha$. Entonces,
        \begin{equation*}
            \forall \varepsilon > 0 ~ \exists m \in \mathbb{N} ~:~ \alpha - \varepsilon < a_n < \alpha + \varepsilon.
        \end{equation*}
        Por tanto, tenemos que $\alpha \geq a, ~ \forall a \in A$ y que $\forall \varepsilon > 0$ existe $a_n \in A$ tal
        que $\alpha - \varepsilon < a_n$, lo que prueba que $\alpha = \sup A$.
\end{description}
    
\end{proof}

La proposición anterior nos dice que el único mayorante de un conjunto de números reales no vacío y mayorado que puede ser límite de una sucesión de elementos de dicho conjunto es el supremo. Es fácil deducir que el único minorante que puede ser límite de una sucesión de elementos de un conjunto de números reales no vacío y minorado es el ínfimo.

%########################################################################################################
% Ejercicios de sucesiones de números reales.
%########################################################################################################

\section{Ejercicios}
\begin{ejercicio}
    Sean $\{x_n\}$ e $\{y_n\}$ dos sucesiones de números reales acotadas. Probar que las sucesiones $\{x_n + y_n\}$ y $\{x_n y_n\}$ están acotadas.
\end{ejercicio}
\begin{ejercicio}
    Dar un ejemplo de $2$ sucesiones de números reales, $\{x_n\}$ e $\{y_n\}$, no convergentes tales que $\{x_n + y_n\}$ sea convergente.
\end{ejercicio}
\begin{ejercicio}
    Sea $x \in \mathbb{R}$ fijo. Probar que existen 4 sucesiones $\{r_n\}$, $\{q_n\}$, $\{\alpha_n\}$ y $\{\beta_n\}$, con  $\{r_n\}$, $\{q_n\}$ sucesiones en $\mathbb{Q}$ y $\{\alpha_n\}$, $\{\beta_n\}$ sucesiones en $\mathbb{R} - \mathbb{Q}$ convergentes a $x$ tales que:
    \begin{gather*}
        r_n < x < q_n, ~\forall n \in \mathbb{N}\\
        \alpha_n < x < \beta_n, ~\forall n \in \mathbb{N}
    \end{gather*}
\end{ejercicio}

\begin{ejercicio}\label{ej:5.4.4}
    Sean $\{x_n\}$ una sucesión de números reales y $p$ un número natural. Definimos la sucesión $\{z_n\}$ tal que
    \begin{equation*}
        z_n = x_{n+p}, ~ \forall n \in \mathbb{N}
    \end{equation*}
    Probar que $\{x_n\}$ converge, si y sólo si, $\{z_n\}$ converge, en cuyo caso
    \begin{equation*}
        \lim \{x_n\} = \lim \{z_n\} = \lim \{x_{n+p}\}
    \end{equation*}
\end{ejercicio}
\chapter{Números enteros y racionales}\label{chp:Tema3}

En los métodos constructivos, partiendo de los números naturales se construyen en tres etapas sucesivas los números enteros, racionales y reales. En los métodos axiomáticos, por el contrario, se admite la existencia de los números reales y los naturales, enteros y racionales aparecen como subconjuntos distinguidos de $\mathbb{R}$. Ya hemos destacado quiénes son los números naturales y en este tema destacaremos quienes son los enteros y los racionales y estudiaremos sus propiedades más importantes.

    
\section{Números enteros}
%########################################################################################################
% Números enteros
%########################################################################################################

\begin{definicion}
    Definimos \textbf{el conjunto de los números enteros}, notado por $\mathbb{Z}$, como:
    \begin{equation*}
        \mathbb{Z} := \mathbb{N} \cup \{0\} \cup \{-n~:~n \in \mathbb{N}\}
    \end{equation*}
\end{definicion}

Es fácil ver que la diferencia de dos números naturales es siempre un entero.\\

En efecto, sean $m,n \in \mathbb{N}$. Si $m > n$ entonces $m-n$ es natural por la Proposición \ref{prop:2.1.7}, y por tanto, entero.
Si $m=n$, entonces  $m-n = 0 \in \mathbb{Z}$. Finalmente, si $m < n$, entonces $n-m$ es natural y $m-n=-(n-m) \in \mathbb{Z}$.\\

Recíprocamente, todo entero $p$ puede expresarse como diferencia de dos naturales.
Basta ver que $|p|+1$ y $|p| -p + 1$ son naturales y que
\begin{equation*}
    p=|p|+1-(|p| -p + 1)
\end{equation*}
Las dos proposiciones siguientes resumen las propiedades de los números enteros que nos interesan destacar.
\begin{prop}\label{prop:3.1.2}
    La suma y producto de números enteros es un número entero.
    
    Si $a \in \mathbb{Z}-\{0\}$ y $a^{-1} \in \mathbb{Z}$, entonces $a = 1$ o $a = -1$.
    
    Por tanto, $\mathbb{Z}$ es un subanillo de $\mathbb{R}$ pero no un subcuerpo.
\end{prop}
\begin{proof}
    Sean $p_1,p_2 \in \mathbb{Z}$ y supongamos que $p_1=m_1-n_1$ y $p_2=m_2-n_2$, $m_1,m_2,n_1,n_2 \in \mathbb{N}$.
    Entonces:
    \begin{gather*}
        p_1+p_2=(m_1+m_2)-(n_1+n_2)\\
        p_1 p_2 = (m_1 m_2 + n_1 n_2) - (m_1 n_2 + n_1 m_2)
    \end{gather*}
    Como la suma y producto de naturales es un natural (Corolario \ref{coro:2.1.5}.\ref{coro:2.1.5_2}), las ecuaciones anteriores expresan $p_1+p_2$ y $p_1 p_2$ como diferencia de números naturales, por lo $p_1+p_2,p_1 p_2 \in \mathbb{Z}$.\\
    
    Sea $a \in \mathbb{Z}-\{0\}$ con $a^{-1} \in \mathbb{Z}$. Si $a > 0$, entonces
    $a^{-1} > 0$ y por tanto $a^{-1} \in \mathbb{N}$, por lo que $a=1$ (Corolario  \ref{coro:2.1.5}.\ref{coro:2.1.5_3}). Si $a < 0$, aplíquese el mismo razonamiento para $-a$, obteniéndose $a=-1$.
\end{proof}

\begin{prop}Algunas propiedades de $\bb{Z}$ son:
    \begin{enumerate}
        \item $\mathbb{Z}$ no tiene mínimo.

        \item Si $p,q \in \mathbb{Z}$ y $p < q$, entonces $p+1 \leq q$.

        \item $\mathbb{Z}$ es numerable.
    \end{enumerate}
\end{prop}
\begin{proof}
    Demostramos cada una de las propiedades:
    \begin{enumerate}
        \item Si $p$ es un entero, entonces $p-1$ es también entero y sabemos que $p-1 < p$.

        \item $q-p$ es un entero mayor estricto que $0$, luego es natural, de donde $q-p \geq 1$.

        \item Definimos $f:\mathbb{Z} \longrightarrow \mathbb{N}$ como sigue:
        \begin{equation*}
            \left\{ \begin{array}{lcc}
                f(-n)=2n+1,~\forall n \in \mathbb{N} \\ \\
                f(0)=1 \\ \\
                f(n)=2n,~\forall n \in \mathbb{N}
                \end{array}
            \right.
        \end{equation*}
        Es fácil ver que $f$ es inyectiva (de hecho es biyectiva).
    \end{enumerate}
\end{proof}

Eventualmente se utilizarán en lo sucesivo algunas propiedades algebraicas del anillo $\mathbb{Z}$, como por ejemplo, cuestiones de divisibilidad o la descomposición en factores primos de un número entero. Daremos por conocidas estas propiedades, cuya demostración no es propia de un curso de Análisis Matemático.

%########################################################################################################
% Números racionales
%########################################################################################################

\section{Números racionales}
\begin{definicion}
    Definimos \textbf{el conjunto de los números racionales}, notado por $\mathbb{Q}$, como:
    \begin{equation*}
        \mathbb{Q} := \left\{ \frac{p}{q}~:~p \in \mathbb{Z},~q \in \mathbb{N} \right\}
    \end{equation*}
    Es obvio que $\mathbb{N} \subseteq \mathbb{Z} \subseteq \mathbb{Q}$.
\end{definicion}

\begin{prop}
    $\mathbb{Q}$ es un subcuerpo de $\mathbb{R}$.
\end{prop}
\begin{proof}
    Hay que probar que la suma y el producto de racionales es racional y que el inverso de todo racional no nulo es racional. Todo ello se deduce con facilidad de la proposición \ref{prop:3.1.2} y el corolario \ref{coro:2.1.5} (Ver el ejercicio \ref{ej:1.4.1} apartado \ref{ej:1.4.1_6}).
\end{proof}

Utilizando la proposición anterior, deducimos que $\mathbb{Q}$ con las operaciones suma y producto determinadas por las de $\mathbb{R}$ es un cuerpo conmutativo.

La relación de orden de $\mathbb{R}$ determina obviamente una relación de orden total en $\mathbb{Q}$, por lo que $\mathbb{Q}$ es un cuerpo conmutativo totalmente ordenado, al igual que $\mathbb{R}$. 

A pesar de la gran cantidad de consecuencias que hemos demostrado hasta ahora, por ahora no somos capaces de probar la existencia de números reales no racionales. Veamos algunas propiedades más de $\mathbb{Q}$ antes de poner de manifiesto la necesidad de un último axioma.
\begin{prop}
    Dados $r,s \in \mathbb{Q}$ con $r < s$. Entonces, el conjunto
    \begin{equation*}
        A=\{x \in \mathbb{Q}~:~r < x < s\}
    \end{equation*}
    es un conjunto infinito.
\end{prop}
\begin{proof}
    Dado $n$ natural se tiene que $n+1 > 1 > 0$, luego $0 < \frac{1}{n+1} < 1$ y por tanto,
    $r < r + \frac{s-r}{n+1} < s$ con lo que
    \begin{equation*}
        r + \frac{s-r}{n+1} \in A,~\forall n \in \mathbb{N}.
    \end{equation*}
    La aplicación $f:\mathbb{N} \longrightarrow A$ dada por
    \begin{equation*}
        f(n) = r + \frac{s-r}{n+1},~\forall n \in \mathbb{N}
    \end{equation*}
    que claramente es inyectiva, determina una biyección de $\mathbb{N}$ sobre un subconjunto de $A$. Por tanto,
    $A$ es infinito.    
\end{proof}

Podríamos pensar que $\mathbb{Q}$ es ``mucho más grande'' que $\mathbb{N}$. Sin embargo, esto esta lejos de ser cierto.
\begin{prop}
    $\mathbb{Q}$ es numerable.    
\end{prop}
\begin{proof}
    Usando que $\mathbb{Z}$ es numerable, es fácil probar que $\mathbb{Z} \times \mathbb{N}$ es numerable (¡¡Hágase!!, similar al ejercicio \ref{ej:2.5.4}).
    Por tanto, existe una biyección $f$ de $\mathbb{N}$ sobre $\mathbb{Z} \times \mathbb{N}$ (Teorema \ref{teo:2.4.3}).
    
    Sea $g: \mathbb{Z} \times \mathbb{N} \longrightarrow \mathbb{Q}$ dada por:
    \begin{equation*}
        g((p,q))=\frac{p}{q},~ \forall p \in \mathbb{Z},~ \forall q \in \mathbb{N}
    \end{equation*}
    Claramente, $g$ es sobreyectiva, luego $\displaystyle g \circ f : \mathbb{N} \longrightarrow \mathbb{Q}$ es sobreyectiva y
    $\mathbb{Q}$ es numerable (Ver ejercicio \ref{ej:2.5.3}). 
\end{proof}

Para terminar, veamos que es necesario introducir un nuevo axioma para probar la existencia de números reales no racionales.
\begin{prop}
    No existe ningún número racional $r$ tal que $r^2 = 2$.
\end{prop}
\begin{proof}
    Supongamos que existe $r=\dfrac{p}{q} \in \mathbb{Q}$, con $mcd(p,q)=1$, tal que $r^2 = 2$.
    
    Entonces, $2=\dfrac{p^2}{q^2} \Rightarrow  2 q^2 = p^2$. Deducimos que $p^2$ es par, de donde $p$ es par.
    Sea $p = 2k,~k \in \mathbb{Z}$. Entonces:
    \begin{equation*}
        2q^2 = 4k^2 \Longrightarrow q^2 = 2 k^2
    \end{equation*}
    Y razonando igual que antes, $q$ es par. Pero entonces $p$ y $q$ son ambos pares y $mcd(p,q)=1$, lo que es
    una contradicción. Por lo tanto, no existe ningún racional $r$ tal que $r^2 = 2$.
\end{proof}

%########################################################################################################
% Ejercicios de números enteros y racionales
%########################################################################################################

\section{Ejercicios}
\begin{ejercicio}
    Probar que si $p$ es un entero no nulo, entonces:
    \begin{equation*}
        -1 \leq \frac{1}{p} \leq 1 
    \end{equation*}
\end{ejercicio}

\begin{ejercicio}
    Sean $p,q \in \mathbb{Z}$, $m,n \in \mathbb{N}$. Probar que
    \begin{equation*}
        \frac{p}{n} \cdot \frac{q}{m} \Leftrightarrow pm < qn \Leftrightarrow pm - qn \in \mathbb{N}
    \end{equation*}
    ($\mathbb{N}$ determina el orden de $\mathbb{Q}$).
\end{ejercicio}

\begin{ejercicio}
    Dados $n$ natural y $r$ racional positivo tales que $r^2=n$, probar que $r \in \mathbb{N}$.
\end{ejercicio}
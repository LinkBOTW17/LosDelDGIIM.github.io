\chapter{El último axioma}\label{chp:Tema4}
    
Ya hemos visto que existen conjuntos de números reales que no tienen máximo ni mínimo. $\mathbb{R}$ no tiene máximo ni mínimo. El conjunto
\begin{equation*}
    A=\{x \in \mathbb{R} ~:~ 0 < x < 1\}
\end{equation*}
tampoco tiene máximo ni mínimo. Sin embargo, a diferencia de $\mathbb{R}$, existen números reales mayores que todos los elementos de $A$ y menores que todos los elementos de $A$. A continuación, damos nombres a estos números reales. Merece la pena hacer notar que estas definiciones pueden hacerse en cualquier cuerpo ordenado.

%########################################################################################################
% Mayorantes, minorantes, supremo e ínfimo.
%########################################################################################################

\section{Mayorantes, minorantes, supremo e ínfimo}
\begin{definicion}
    Sea $A$ un conjunto no vacío de números reales.
    \begin{enumerate}
        \item Diremos que un número real $x$ es \textbf{mayorante} de $A$ si verifica:
        \begin{equation*}
            x \geq a,~\forall a \in A
        \end{equation*}

        \item Diremos que un número real $x$ es \textbf{minorante} de $A$ si verifica:
        \begin{equation*}
            x \leq a,~\forall a \in A
        \end{equation*}
    \end{enumerate}
\end{definicion}

Nótese que, a diferencia del máximo y del mínimo de un conjunto, un mayorante o minorante de un conjunto no tiene por qué pertencer al conjunto. De hecho, el máximo (respectivamente mínimo) de un conjunto $A$ es mayorante (respectivamente minorante) de dicho conjunto y un mayorante (respectivamente minorante) del conjunto $A$ es máximo (respectivamente mínimo) si, y sólo si, pertenece a $A$.\\

Si un conjunto admite mayorantes, diremos que está \textbf{mayorado}. Si un conjunto admite minorantes, diremos que está \textbf{minorado}. Si un conjunto de números reales está a la vez mayorado y minorado, diremos que está \textbf{acotado}.\\

Dado un conjunto de números reales $A$, notaremos $M(A)$ al conjunto de sus mayorantes y $m(A)$ al conjunto de sus minorantes. Nótese que si $A$ está mayorado, el conjunto $M(A)$ es infinito, y lo mismo le ocurre a $m(A)$ si $A$ está minorado.\\

El siguiente lema puede ser de ayuda para determinar todos los mayorantes y minorantes de un conjunto.
\begin{lema}
    Sean $a,b \in \mathbb{R}$ tales que $a < b+\varepsilon, ~\forall \varepsilon > 0$. Entonces, $a \leq b$.
\end{lema}
\begin{proof}
    Si fuera $a > b$, tomando $\varepsilon = a-b$, tenemos que:
    $$a < b+ \varepsilon = b+(a-b)=a$$ lo que es una contradicción. Por lo tanto, $a \leq b$.
\end{proof}
\begin{ejemplo}
    Utilizando el lema anterior, es fácil ver que
    \begin{gather*}
        M(\{x \in \mathbb{R}~:~ 0 < x < 1\}) = \{x \in \mathbb{R}~:~ 1 \leq x\}\\
        m(\{x \in \mathbb{R}~:~ 0 < x < 1\}) = \{x \in \mathbb{R}~:~ x \leq 0\} = \mathbb{R}_0^{-}
    \end{gather*}
\end{ejemplo}

\begin{definicion}
    Sea $A$ un conjunto no vacío de números reales. Si $A$ está mayorado y el conjunto de los mayorantes de $A$ tiene mínimo, se define el \textbf{supremo} de $A$ como el mínimo del conjunto de mayorantes de $A$. Análogamente se define el \textbf{ínfimo} de un conjunto como el máximo del conjunto de sus minorantes supuesto que exista.
\end{definicion}

Claramente el supremo y el ínfimo de un conjunto, si existen, son únicos y son respectivamente un
mayorante y un minorante del mismo. Notaremos $\sup A$ e $\inf A$, respectivamente. Volviendo al ejemplo anterior,
tenemos que
\begin{gather*}
    \sup \{x \in \mathbb{R}~:~ 0 < x < 1\} = 1\\
    \inf \{x \in \mathbb{R}~:~ 0 < x < 1\} = 0
\end{gather*}
\begin{equation*}
    \sup \mathbb{R}^{-}=\sup \mathbb{R}^{-}_0 = 0;~ \inf \mathbb{R}^{+}=\inf \mathbb{R}^{+}_0 = 0;
\end{equation*}

La relación existente entre el supremo y el máximo de un conjunto y la relación entre el mínimo y el ínfimo,
se especifican a continuación.
\begin{prop}
    Sea $A$ un conjunto no vacío de números reales.
    \begin{enumerate}
        \item Si $A$ tiene máximo, entonces tiene supremo y $\sup A = \max A$.

        \item Si $A$ tiene mínimo, entonces tiene ínfimo e $\inf A = \min A$.

        \item Supongamos que $A$ tiene supremo.
        \begin{itemize}
            \item Si $\sup A \in A$, $A$ tiene máximo y $\sup A = \max A$.

            \item Si $\sup A \notin A$, $A$ no tiene máximo.
        \end{itemize}

        \item Análogo a III) cambiando ``supremo'' por ``ínfimo'' y ``máximo'' por ``mínimo''.
    \end{enumerate}
\end{prop}
\begin{proof} Demostramos cada una de las propiedades:
    \begin{enumerate}
        \item Si $x$ es mayorante de $A$, $a \leq x,~\forall a \in A$ y cómo $\max A \in A$ y $\max A \leq x$, tenemos que $\max A$ es el mínimo de los mayorantes de $A$, es decir, $\sup A = \max A$.

        \item Análoga a I).

        \item Si $\sup A \in A$, $\sup A$ es un mayorante que pertenece a $A$, luego es el máximo de $A$. Si $\sup A \notin A$, supongamos que $A$ tuviese máximo. Por I), $\sup A = \max A \in A$, lo cual es absurdo.

        \item Análoga a IV).
    \end{enumerate}
\end{proof}

\begin{prop}[Caracterización del supremo y del ínfimo]
    Sean $A$ un conjunto no vacío de números reales y $x \in \mathbb{R}$. Entonces:
    \begin{enumerate}
        \item Respecto al supremo,
        \begin{equation*}
            x = \sup A \Longleftrightarrow
            \left\{ \begin{array}{lcc}
                x \geq a, ~\forall a \in A \\ \\
                \forall \varepsilon > 0, ~\exists a \in A ~:~ a > x-\varepsilon
                \end{array}
            \right.
        \end{equation*}

        \item Respecto al ínfimo,
        \begin{equation*}
            x = \inf A \Longleftrightarrow
            \left\{ \begin{array}{lcc}
                x \leq a, ~\forall a \in A \\ \\
                \forall \varepsilon > 0, ~\exists a \in A ~:~ a < x+\varepsilon
                \end{array}
            \right.
        \end{equation*}
    \end{enumerate}
\end{prop}
\begin{proof} Demostramos el primer apartado, ya que el segundo es análogo. Procedemos mediante doble implicación:
\begin{description}
    \item [$\Longrightarrow)$] Si $x = \sup A$, $x$ es mayorante de $A$ y dado $\varepsilon > 0$, $x - \varepsilon$ no puede ser mayorante de $A$, luego $\exists a \in A$ tal que $a > x-\varepsilon$.
    
    \item [$\Longleftarrow)$] Por hipótesis, $x$ es un mayorante de $A$. Sea $y$ un mayorante cualquiera de $A$. Si fuera $y < x$, tomando $\varepsilon = x-y$, tenemos que $\exists a \in A$ tal que $a > x-\varepsilon = x - (x-y) = y$, lo cual es absurdo, pues $y$ es un mayorante de $A$. Así, $x \leq y$, lo que prueba que $x$ es el mínimo de los mayorantes de $A$.
\end{description}
\end{proof}

%########################################################################################################
% El axioma del supremo.
%########################################################################################################

\section{El axioma del supremo.}
El siguiente axioma, en unión con los axiomas A, B y C, que se resumían diciendo que $\mathbb{R}$ era un cuerpo
conmutativo totalmente ordenado, completa la axiomática que define el cuerpo $\mathbb{R}$ de los números reales.\\

\underline{\textbf{Axioma del supremo}}

Todo conjunto de números reales no vacío y mayorado tiene supremo.\\

Nótese que para que un conjunto de números reales tenga supremo, debe ser no vacío y mayorado. El axioma del supremo nos garantiza que estas condiciones, trivialmente necesarias, son también suficientes. Cabría preguntarse porque no se exige también un ``axioma del ínfimo''. La respuesta es que no es necesario, pues puede deducirse directamente del axioma del supremo. Veámoslo:
\begin{prop}[Principio del ínfimo]
    Todo conjunto de números reales no vacío y minorado tiene ínfimo.
\end{prop}
\begin{proof}
    Sea $A$ un conjunto de números reales no vacío y minorado. Si consideramos el conjunto
    \begin{equation*}
        -A :=\{-x ~:~ x \in A\}
    \end{equation*}
    Es fácil ver que
    \begin{equation*}
        m \in m(A) \Longleftrightarrow m \leq a, ~\forall a \in A \Longleftrightarrow -m \geq -a, ~\forall a \in A \Longleftrightarrow -m \in M(-A)
    \end{equation*}
    Lo que prueba que $A$ está minorado si, y sólo si, $-A$ está mayorado. Sea $h=~\sup (-A)$; como $h$ es mayorante de $-A$, $-h$ es minorante de $A$. Dado $m$ minorante de $A$, tenemos que $-m$ es mayorante de $-A$, luego $h \leq -m \Longleftrightarrow -h \geq m$ y así, $-h = \inf A$.
\end{proof}

Nótese que hemos probado que si $A$ es un conjunto de números reales no vacío y minorado, $-A$ está mayorado y se tiene:
\begin{equation*}
    \inf A=-\sup (-A)
\end{equation*}
Cambiando $A$ por $-A$, si $A$ es un conjunto de números reales no vacío y mayorado, $-A$ está minorado
y se tiene:
\begin{equation*}
    \sup A=-\inf (-A)
\end{equation*}

\begin{coro}
    Todo conjunto de números reales no vacío y acotado tiene supremo e ínfimo.
\end{coro}

%########################################################################################################
% Primeras consecuencias del axioma del supremo
%########################################################################################################

\section{Primeras consecuencias del axioma del supremo}
\begin{teo}[Propiedad Arquimediana\footnote{También conocido como Principio de Arquímedes.} del Orden de $\mathbb{N}$]
    El conjunto de los números naturales no está mayorado.
\end{teo}
\begin{proof}
    Si lo estuviera, sea $m = \sup \mathbb{N}$. Dado $n$ natural, se tiene que $n+1$ es natural, por lo que $n+1 \leq m$. Pero entonces $n \leq m-1$ para todo $n$ natural, por lo que $m-1$ es un mayorante de $\mathbb{N}$ más pequeño que $m$, lo cuál es absurdo al ser $m$ el mínimo de los mayorantes.
\end{proof}
\begin{coro}
    Si $\beta \in \mathbb{R}^{+}_0$ tal que $\beta \leq \frac{1}{n},~ \forall n \in \mathbb{N}$, entonces $\beta = 0$.
\end{coro}
\begin{proof}
    Si fuera $\beta \neq 0$, entonces $\beta > 0$, por lo que $\beta^{-1} \geq n, ~\forall n \in \mathbb{N}$. Pero esto implica que $\beta$ es un mayorante de $\mathbb{N}$, lo que contradice la Propiedad Arquimediana. Por tanto, ha de ser $\beta = 0$.
\end{proof}

En el tema anterior aún no éramos capaces de probar la existencia de números reales no racionales. Gracias al axioma del supremo, podremos probar la existencia de dichos números.
\begin{prop}
    Existe $\alpha \in \mathbb{R}^{+}$ tal que $\alpha^2 = 2$.
\end{prop}
\begin{proof}
    Sea $A = \{x \in \mathbb{R}^{+} ~:~ x^2 \leq 2\}$. Es claro que $A$ es un conjunto de números reales no vacío ($1 \in A$)
    y mayorado ($x \leq 5 \Leftrightarrow x^2 \leq 2 \leq 25,~\forall x \in A$, luego $5 \in M(A)$, por ejemplo). Sea $\alpha = \sup A$.
    Es claro que $\alpha \geq 1$ y queda probar que $\alpha^2 = 2$.
    
    Tenemos que $\forall n \in \mathbb{N}$:
    \begin{equation}\label{ec:prop4.3.3_1}
        \alpha + \frac{1}{n} > \alpha \Longrightarrow \alpha + \frac{1}{n} \notin A \Longrightarrow \left(\alpha + \frac{1}{n}\right)^2 \geq 2
    \end{equation}
    
    Dado que $\alpha - \frac{1}{n} < \alpha$, entonces
    \begin{equation}\label{ec:prop4.3.3_2}
        \exists a_n \in A ~:~ \alpha - \frac{1}{n} < a_n
    \end{equation}
    
    Utilizando las ecuaciones \ref{ec:prop4.3.3_1} y \ref{ec:prop4.3.3_2}, vemos que:
    \begin{multline*}
        \alpha^2 - \frac{1+2 \alpha}{n} < \alpha^2 + \frac{1}{n^2} - \frac{2 \alpha}{n} = \left(\alpha - \frac{1}{n}\right)^2 < a_n^2 < 2
        \leq \\ \leq
        \left(\alpha + \frac{1}{n}\right)^2 = \alpha^2 + \frac{1}{n^2} + \frac{2 \alpha}{n} \leq \alpha^2 + \frac{1+2 \alpha}{n}
    \end{multline*}
    
    Y en consecuencia
    \begin{equation*}
        -\left(\frac{1+2 \alpha}{n}\right) \leq 2 -\alpha^2 \leq \frac{1+2 \alpha}{n}
    \end{equation*}
    
    De donde deducimos que
    \begin{equation*}
        -\frac{1}{n} \leq \frac{2 - \alpha^2}{1+2 \alpha} \leq \frac{1}{n} \Longleftrightarrow \left| \frac{2 - \alpha^2}{1+2 \alpha} \right| \leq \frac{1}{n}
    \end{equation*}
    
    Sea $\beta = \left| \dfrac{2 - \alpha^2}{1+2 \alpha} \right| = \dfrac{|2 - \alpha^2|}{1+2 \alpha}$. Por lo anterior:
    \begin{equation*}
        \beta \leq \frac{1}{n}, ~\forall n \in \mathbb{N}
    \end{equation*}
    
    Por la Propiedad Arquimediana, deducimos que
    \begin{equation*}
        \beta = 0 \Longleftrightarrow |2 - \alpha^2| = 0 \Longleftrightarrow \alpha^2 = 2
    \end{equation*}
    Como queríamos probar.
\end{proof}

Puesto que no existe ningún racional cuyo cuadrado sea $2$, deducimos que el número $\alpha$ que aparece en la anterior proposición no es racional. Por tanto, hemos probado la existencia de números
reales no racionales. Al conjunto de los números reales no racionales lo llamaremos el \textbf{conjunto de los números irracionales}, notaremos $\mathbb{R} - \mathbb{Q}$.

\begin{teo}[Densidad de $\mathbb{R} - \mathbb{Q}$ en $\mathbb{R}$]
    Si $x,y \in \mathbb{R} ~:~ x < y$, entonces $\exists \gamma \in \mathbb{R} - \mathbb{Q} ~:~ x < \gamma < y$.
\end{teo}
\begin{proof}
    Dados $x,y \in \mathbb{R}$ con $x < y$, si uno de ellos es racional y el otro es irracional, basta tomar $\gamma = \frac{x+y}{2}$.
    
    Si ambos son irracionales, sea $z=\frac{x+y}{2}$. Si $z$ es irracional, hacemos $\gamma = z$ y hemos acabado.
    De lo contrario, basta con tomar $\gamma = \frac{z+y}{2}$.
    
    Si ambos son racionales, sea $\alpha$ el irracional dado por la proposición anterior. Dado que $1 < \alpha$,
    $\frac{1}{\alpha} < 1$, y basta con tomar $\gamma = x + \frac{y-x}{\alpha}$.
\end{proof}

\begin{teo}[Densidad de $\mathbb{Q}$ en $\mathbb{R}$]
    Si $x,y \in \mathbb{R} ~:~ x < y$, entonces $\exists r \in \mathbb{Q} ~:~ x < r < y$.
\end{teo}
\begin{proof} Realizamos la siguiente distinción de casos:
    \begin{itemize}
        \item Supongamos que $0 < x < y$. Dado que $\frac{1}{y-x}$ no puede ser mayorante de $\mathbb{N}$,
        existe $n_0 \in \mathbb{N}$ tal que $\frac{1}{y-x} < n_0$, de donde $y-x > \frac{1}{n_0}$.
        Consideramos el siguiente conjunto:
        \begin{equation*}
            P=\{m \in \mathbb{N}~:~m > n_0 x\}
        \end{equation*}
        Es claro que $P$ es no vacío y al ser un subconjunto de $\mathbb{N}$, tiene mínimo.
        Sea $m_0= \min P$. Es claro que $m_0 - 1 \leq n_0 x$ y que $m_0 > n_0 x$, por lo que:
        \begin{equation*}
            x < \frac{m_0}{n_0}=\frac{m_0 -1}{n_0}+\frac{1}{n_0} \leq x +\frac{1}{n_0} < x + (y - x) = y
        \end{equation*}
        De donde se tiene que $x < \dfrac{m_0}{n_0} < y$ y basta tomar $r=\frac{m_0}{n_0}$.

        \item Si fuera $x < 0 < y$, tomamos $r=0$.

        \item Si fuera $x < y < 0$, entoces $0 < -y < -x$ y aplicando lo ya demostrado, encontramos $r' \in \mathbb{Q}$ tal que $-y < r' < -x$, de donde $x < -r' < y$. Basta con tomar $r=-r'$.
    \end{itemize}
\end{proof}

%########################################################################################################
% Ejercicios del último axioma.
%########################################################################################################

\section{Ejercicios}

\begin{ejercicio}
    Probar los siguientes enunciados:
    \begin{enumerate}
        \item Todo conjunto de números enteros no vacío y mayorado (resp. minorado) tiene máximo (resp. mínimo).
        \item Un conjunto no vacío de números enteros está acotado si, y sólo si, es finito.
    \end{enumerate}
\end{ejercicio}

\begin{ejercicio}
    Dados $A,B \subseteq \mathbb{R}$ no vacíos. Consideramos el siguiente conjunto:
    \begin{equation*}
        C = \{x+y ~:~ x \in A,~ y \in B\}
    \end{equation*}
    
    Probar que si $A$ y $B$ están mayorados, también lo está $C$ y se verifica que
    \begin{equation*}
        \sup C= \sup A + \sup B
    \end{equation*}
\end{ejercicio}

\begin{ejercicio}\label{ej:4.4.3}
    Sean $A,B \subseteq \mathbb{R}$ no vacíos.
    \begin{enumerate}
        \item Si $A \subseteq B$ y $B$ está mayorado, entonces $A$ está mayorado y $$\sup A \leq \sup B.$$

        \item Si $A \subseteq B$ y $B$ está minorado, entonces $A$ está minorado y $$\inf A \geq \inf B.$$

        \item Si $A$ y $B$ están mayorados, entonces $A \cup B$ está mayorado y $$\sup (A \cup B) = \max \{\sup A, \sup B\}.$$

        \item Si $A$ y $B$ están minorados, entonces $A \cup B$ está minorado y $$\inf (A \cup B) = \min \{\inf A, \inf B\}.$$

        \item Si $A$ y $B$ están mayorados y $A \cap B \neq \emptyset$, entonces $A \cap B$ está mayorado y $$\sup (A \cap B) \leq \min \{\sup A, \sup B\}.$$

        \item Si $A$ y $B$ están minorados y $A \cap B \neq \emptyset$, entonces $A \cap B$ está minorado y $$\inf (A \cap B) \geq \max \{\inf A, \inf B\}.$$
    \end{enumerate}
\end{ejercicio}

\begin{ejercicio}
    Dado $x \in \mathbb{R}$, probar que:
    \begin{gather*}
        x=\sup \{r \in \mathbb{Q}~:~ r < x\}=\sup \{\alpha \in \mathbb{R}-\mathbb{Q}~:~ \alpha < x\} \\
        x=\inf \{r \in \mathbb{Q}~:~ r > x\}=\inf \{\alpha \in \mathbb{R}-\mathbb{Q}~:~ \alpha > x\}
    \end{gather*}
\end{ejercicio}

\begin{ejercicio}
    Dado $x\in \bb{R}$, probar que:
    \begin{gather*}
        x=\sup \{\in \bb{Q} ~:~ r<x\} = \inf \{\in \bb{Q} ~:~ r>x\}\\
        x=\sup \{\in \bb{R}-\bb{Q} ~:~ r<x\} = \inf \{\in \bb{R}-\bb{Q} ~:~ r>x\}
    \end{gather*}
\end{ejercicio}
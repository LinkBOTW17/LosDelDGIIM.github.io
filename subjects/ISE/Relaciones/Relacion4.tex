\section{Análisis comparativo del
rendimiento}

\begin{ejercicio}\label{ej:4.1}
En la Tabla~\ref{tab:ej:4.1} se muestran los tiempos de ejecución (en segundos) y el número de instrucciones ejecutadas (en millones) en el computador Cleopatra para cinco programas distintos.
\begin{table}[h]
\centering
\begin{tabular}{@{}lrr@{}}
\toprule
Programa & Tiempo [s] & Instrucciones ($\cdot 10^6$) \\ \midrule
\verb|asterix| & 56 & 543 \\
\verb|obelix| & 59 & 346 \\
\verb|panoramix| & 113 & 415 \\
\verb|idefix| & 132 & 256 \\
\verb|abraracurcix| & 120 & 235 \\ \bottomrule
\end{tabular}
\caption{Tiempos de ejecución e instrucciones ejecutadas en Cleopatra.}
\label{tab:ej:4.1}
\end{table}
\begin{enumerate}
    \item Calcule el número medio de MIPS (millones de instrucciones por segundo) de este computador al ejecutar los 5 programas.
    \begin{align*}
        \dfrac{543 + 346 + 415 + 256 + 235}{56 + 59 + 113 + 132 + 120} &= \dfrac{1795}{480}\approx \unit[3.74]{MIPS}
    \end{align*}
    \item Determine el número medio de ciclos por instrucción (CPI) obtenidos por este computador. Considere que las instrucciones ejecutadas por los tres primeros programas duran 3 ciclos de media, mientras que las del resto duran 5 ciclos.
    \begin{align*}
        \dfrac{3 \cdot 543 + 3 \cdot 346 + 3 \cdot 415 + 5 \cdot 256 + 5 \cdot 235}{543 + 346 + 415 + 256 + 235} &= \dfrac{6367}{1795} \approx \unit[3.546]{CPI}
    \end{align*}
\end{enumerate}
\end{ejercicio}
\begin{comment}Sol:
1. El computador obtiene 3,74 MIPS.
2. El número medio de CPI es 3,55.
\end{comment}

\begin{ejercicio}\label{ej:4.2}
La Tabla~\ref{tab:ej:4.2} muestra el tipo y número de las operaciones de coma flotante ejecutadas por un programa de prueba en el computador MATES; la última columna representa el coste computacional en operaciones normalizadas.
\begin{table}[h]
\centering
\begin{tabular}{@{}lrr@{}}
\toprule
Operación & Cantidad ($\cdot 10^9$) & Operaciones normalizadas \\ \midrule
\verb|add.s, sub.s| & 456 & 1 \\
\verb|div.s, mul.s| & 340 & 3 \\
\verb|sqrt.s| & 180 & 12 \\
\verb|sqrt.d| & 70 & 15 \\
\verb|log.d| & 30 & 18 \\ \bottomrule
\end{tabular}
\caption{Operaciones de coma flotante en MATES.}
\label{tab:ej:4.2}
\end{table}
Se sabe que el programa tarda una hora en ejecutarse. Indique el rendimiento de este computador mediante el uso de MFLOPS (millones de operaciones de coma flotante por segundo) y MFLOPS normalizados. ¿Existe mucha diferencia entre ambos valores?
\begin{itemize}
    \item MFLOPS:
    \begin{align*}
        \dfrac{(456 + 340 + 180 + 70 + 30)\cdot 10^9}{3600\cdot 10^6} &= \dfrac{1076}{3600}\cdot 10^3 \approx \unit[299]{MFLOPS}
    \end{align*}
    \item MFLOPS normalizados:
    \begin{align*}
        \dfrac{(456 + 3 \cdot 340 + 12 \cdot 180 + 15 \cdot 70 + 18 \cdot 30)\cdot 10^9}{3600\cdot 10^6} &= \dfrac{5226}{3600}\cdot 10^6 \approx \unit[1451.67]{MFLOPS}
    \end{align*}
\end{itemize}
Como vemos, la diferencia es significante.
\end{ejercicio}
\begin{comment}Sol:
El programa obtiene 299 MFLOPS y 1452 MFLOPS normalizados.
\end{comment}

\begin{ejercicio}\label{ej:4.3}
Considere la información (incompleta) obtenida por la orden siguiente en un computador sin más carga que la ejecución de esta orden y sin operaciones de E/S:
\begin{minted}[linenos=false]{shell}
$ time simulador
real 0m130s
user  -----s
sys 0m5s
\end{minted}
Se sabe que el número de instrucciones ejecutadas es de $32 \cdot 10^9$; de estas últimas, el 60\% se ejecuta en dos ciclos, mientras que el resto lo hace en cinco ciclos. Calcule el número medio de ciclos por instrucción (CPI) obtenidos por el programa, la frecuencia de funcionamiento del procesador y los MIPS alcanzados por el procesador.\\

En primer lugar, tenemos que:
\begin{align*}
    \dfrac{32\cdot 10^9\cdot (2\cdot 0.6 + 5\cdot 0.4)}{32\cdot 10^9} &= \unit[3.2]{CPI}
\end{align*}

Para calcular los MIPS, puesto que el computador no tiene más carga y no hay operaciones de E/S, sabemos que el tiempo de ejecución es de $\unit[130]{s}$. Por tanto:
\begin{align*}
    \dfrac{32\cdot 10^9}{130\cdot 10^6} &= \approx \unit[246.15]{MIPS}
\end{align*}

Respecto a la frecuencia de funcionamiento del procesador, sabemos que:
\begin{align*}
    f = \dfrac{3.2\cdot 32\cdot 10^9}{130}\approx \unit[787692307.7]{Hz}\approx \unit[0.7877]{GHz}
\end{align*}

\end{ejercicio}
\begin{comment}Sol:
El programa obtiene un CPI de 3,2 y 246,2 MIPS. La frecuencia del procesador es de 0,788 GHz.
\end{comment}

\begin{ejercicio}\label{ej:4.4}
La Tabla~\ref{tab:ej:4.4} muestra los tiempos de ejecución de tres programas de un benchmark en tres máquinas diferentes: REF, A y B.
\begin{table}[h]
\centering
\begin{tabular}{@{}lrrr@{}}
\toprule
Programa & $t_{REF}$ [s] & $t_A$ [s] & $t_B$ [s] \\ \midrule
1 & 20 & 12 & 15 \\
2 & 20 & 10 & 15 \\
3 & 40 & 25 & 16 \\ \bottomrule
\end{tabular}
\caption{Tiempos de ejecución de programas en diferentes máquinas.}
\label{tab:ej:4.4}
\end{table}

Calcule, entre la máquina A y la máquina B, cuál presenta mejor rendimiento, según los siguientes criterios:
\begin{enumerate}
    \item Media aritmética.
    \begin{align*}
        \ol{t}_A &= \dfrac{12 + 10 + 25}{3} = \dfrac{47}{3} \approx \unit[15.67]{s}\\
        \ol{t}_B &= \dfrac{15 + 15 + 16}{3} = \dfrac{46}{3} \approx \unit[15.33]{s}
    \end{align*}

    Como $\ol{t}_B < \ol{t}_A$, la máquina B presenta mejor rendimiento para esta carga.
    

    \item Media aritmética ponderada, donde los pesos se escogen de forma inversamente proporcional al tiempo de ejecución de la máquina de referencia REF.

    Los pesos son:
    \begin{align*}
        C &= \dfrac{1}{\nicefrac{1}{20}+\nicefrac{1}{20}+\nicefrac{1}{40}} = 8\\
        w_1 &= \frac{C}{20} = 0.4\\
        w_2 &= \frac{C}{20} = 0.4\\
        w_3 &= \frac{C}{40} = 0.2
    \end{align*}

    Las medias ponderadas son:
    \begin{align*}
        \ol{t_W}_A &= 0.4\cdot 12 + 0.4\cdot 10 + 0.2\cdot 25 = \unit[13.8]{s}\\
        \ol{t_W}_B &= 0.4\cdot 15 + 0.4\cdot 15 + 0.2\cdot 16 = \unit[15.2]{s}
    \end{align*}
    
    Como $\ol{t_W}_A < \ol{t_W}_B$, la máquina A presenta mejor rendimiento para esta carga (según este criterio).
    \item SPEC, usando nuevamente REF como máquina de referencia.
    
    Calculamos la media geométrica de los tiempos en cada caso:
    \begin{align*}
        \ol{t_g}_A = \sqrt[3]{12 \cdot 10 \cdot 25} &\approx \unit[14.42]{s}\\
        \ol{t_g}_B = \sqrt[3]{15 \cdot 15 \cdot 16} &\approx \unit[15.33]{s}
    \end{align*}

    Aunque estos valores no son los índices SPEC, sabemos que este es inversamente proporcional a estos. Por tanto, según este criterio, la máquina A es mejor que la máquina B, puesto que la A tendrá mayor índice SPEC.
\end{enumerate}
\end{ejercicio}
\begin{comment}Sol:
a) A: 15,67s, B: 15,33s. Es más rápida la B (media aritmética menor). b) Los pesos serían: $w_1=w_2=0,4$, $w_3 = 0,2$. Las medias ponderadas según esos pesos serían: A: 13,8s, B: 15,2s. En ese caso, es la más rápida la A (media aritmética ponderada menor). c) SPEC_A = 1,75. SPEC_B = 1,64. En ese último caso, sería mejor la máquina A (SPEC mayor). También podíamos haber calculado directamente la media geométrica de los tiempos de ejecución (14,42s para A, 15,33s para la B, confirmando que la máquina A sería la máquina elegida usando el criterio SPEC por tener menor media geométrica de los tiempos de ejecución).
\end{comment}

\begin{ejercicio}\label{ej:4.5}
La Tabla~\ref{tab:ej:4.5} muestra los tiempos de ejecución, en segundos, de los 14 programas de prueba que integran un determinado benchmark empleado para el cálculo del rendimiento en aritmética de coma flotante. En particular, los tiempos corresponden a la máquina de referencia y a una máquina que denominaremos A (columnas ``Base'' y ``Peak'', con el mismo significado que usa SPEC para sus comparaciones).
\begin{table}[h]
\centering
\begin{tabular}{@{}lrrr@{}}
\toprule
Programa & Referencia & A-Base & A-Peak \\ \midrule
\verb|168.wupwise| & 1600 & 419 & 300 \\
\verb|171.swim| & 3100 & 562 & 562 \\
\verb|172.mgrid| & 1800 & 607 & 607 \\
\verb|173.applu| & 2100 & 658 & 605 \\
\verb|177.mesa| & 1400 & 273 & 242 \\
\verb|178.galgel| & 2900 & 571 & 571 \\
\verb|179.art| & 2600 & 1040 & 1038 \\
\verb|183.equake| & 1300 & 501 & 387 \\
\verb|187.facerec| & 1900 & 434 & 434 \\
\verb|188.ammp| & 2200 & 705 & 697 \\
\verb|189.lucas| & 2000 & 784 & 758 \\
\verb|191.fma3d| & 2100 & 534 & 534 \\
\verb|200.sixtrack| & 1100 & 395 & 336 \\
\verb|301.apsi| & 2600 & 866 & 866 \\ \bottomrule
\end{tabular}
\caption{Tiempos de ejecución de programas en diferentes máquinas.}
\label{tab:ej:4.5}
\end{table}
\begin{enumerate}
    \item Calcúlense los índices SPECfp\_base y SPECfp de la máquina A según el criterio de SPEC.
    
    El índice SPECfp\_base se calcula como la media geométrica de los tiempos de ejecución normalizados respecto a la máquina de referencia:
    \begin{align*}
        SPECfp\_base &= \sqrt[14]{\dfrac{1600}{419} \cdot \dfrac{3100}{562} \cdot \dfrac{1800}{607} \cdot \dots\cdot \dfrac{2100}{534} \cdot \dfrac{1100}{395} \cdot \dfrac{2600}{866}}\\
        &\approx 3.48
    \end{align*}

    El índice SPECfp se calcula de forma similar, pero usando los tiempos de ejecución en la máquina A-Peak:
    \begin{align*}
        SPECfp &= \sqrt[14]{\dfrac{1600}{300} \cdot \dfrac{3100}{562} \cdot \dfrac{1800}{607} \cdot \cdot \dots\cdot \dfrac{1100}{336} \cdot \dfrac{2600}{866}}\\
        &\approx 3.74
    \end{align*}
    \item Para la columna A-Base, si se considera el tiempo total de ejecución, ¿cuántas veces es más rápida la máquina A que la máquina de referencia?
    
    Calculemos los tiempos totales:
    \begin{align*}
        t_{R} &= 1600 + 3100 + \dots + 2600 = \unit[28700]{s}\\
        t_{A} &= 419 + 562 + \dots + 866 = \unit[8349]{s}
    \end{align*}

    Por tanto, tenemos que:
    \begin{align*}
        \dfrac{t_{R}}{t_{A}} &= \dfrac{28700}{8349} \approx 3.4375
    \end{align*}

    Por tanto, la máquina A es aproximadamente $3.44$ veces más rápida que la máquina de referencia.

    \item ¿Qué mejora del rendimiento se obtiene utilizando las opciones de optimización que ofrece el compilador?
    
    Para calcular la mejora del rendimiento, comparamos el tiempo total de ejecución de la máquina A-Base con el de la máquina A-Peak:
    \begin{align*}
        t_{A\_base} &= 419 + 562 + \dots + 866 = \unit[8349]{s}\\
        t_{A\_peak} &= 300 + 562 + \dots + 866 = \unit[7937]{s}
    \end{align*}

    La mejora del rendimiento se calcula como:
    \begin{align*}
        \dfrac{t_{A\_base}}{t_{A\_peak}} = \dfrac{8349}{7937} \approx 1.05
    \end{align*}

    Por tanto, la optimización del compilador permite mejorar el tiempo de ejecución en un $5\%$ ($1.05$ veces).
\end{enumerate}
\end{ejercicio}
\begin{comment}Sol:
1. SPECfp\_base = 3,48 y SPECfp = 3,74.
2. La máquina A es 3,44 veces más rápida que la de referencia.
3. La optimización del compilador permite mejorar 1,05 veces el tiempo de ejecución (un 5%).
\end{comment}

\begin{ejercicio}\label{ej:4.6}
Considere los tiempos de ejecución, en segundos, obtenidos en los computadores R (referencia), A y B para un conjunto de cinco programas de prueba, mostrados en la Tabla~\ref{tab:ej:4.6}.
\begin{table}[h]
\centering
\begin{tabular}{@{}lrrr@{}}
\toprule
Programa & R [s] & A [s] & B [s] \\ \midrule
\verb|tinky-winky| & 2600 & 503 & 539 \\
\verb|dipsy| & 2100 & 654 & 762 \\
\verb|laa-laa| & 9800 & 707 & 716 \\
\verb|po| & 2300 & 748 & 760 \\
\verb|noo-noo| & 1800 & 363 & 235 \\ \bottomrule
\end{tabular}
\caption{Tiempos de ejecución de programas en diferentes máquinas.}
\label{tab:ej:4.6}
\end{table}
\begin{enumerate}
    \item Compare el rendimiento de A y B utilizando el tiempo total de ejecución.
    
    Calculemos ambos tiempos totales de ejecución:
    \begin{align*}
        t_{A} &= 503 + 654 + 707 + 748 + 363 = \unit[2975]{s}\\
        t_{B} &= 539 + 762 + 716 + 760 + 235 = \unit[3012]{s}
    \end{align*}

    Calculamos cuántas veces es más rápida la máquina A que la máquina B:
    \begin{align*}
        \dfrac{t_{B}}{t_{A}} &= \dfrac{3012}{2975} \approx 1.0124
    \end{align*}

    Por tanto, la máquina A es aproximadamente $1.0124$ veces más rápida que la máquina B.
        

    \item Calcule, a la manera de SPEC, un índice de rendimiento para A y B, y compare el rendimiento de ambas máquinas con este índice. ¿Obtiene los mismos resultados que en el apartado anterior?
    
    Para calcular el índice de rendimiento, usamos la media geométrica de los tiempos de ejecución normalizados respecto a la máquina R:
    \begin{align*}
        \text{SPEC\_A} &= \sqrt[5]{\dfrac{2600}{503} \cdot \dfrac{2100}{654} \cdot \dfrac{9800}{707} \cdot \dfrac{2300}{748} \cdot \dfrac{1800}{363}} \approx 5.12\\
        \text{SPEC\_B} &= \sqrt[5]{\dfrac{2600}{539} \cdot \dfrac{2100}{762} \cdot \dfrac{9800}{716} \cdot \dfrac{2300}{760} \cdot \dfrac{1800}{235}} \approx 5.31
    \end{align*}

    Ahora, comparamos los índices:
    \begin{align*}
        \dfrac{\text{SPEC\_B}}{\text{SPEC\_A}} &= \dfrac{5.31}{5.12} \approx 1.037
    \end{align*}

    Por tanto, según el índice SPEC, la máquina B es aproximadamente $1.04$ veces mejor que la máquina A.
\end{enumerate}
\end{ejercicio}
\begin{comment}Sol:
1. Según el tiempo total de ejecución, la máquina A es 1,012 veces más rápida que B.
2. Los índices calculados a la manera de SPEC para las máquinas A y B son, respectivamente, 5,12 y 5,31; en consecuencia, según este índice la máquina B es 1,04 veces mejor que la A.
\end{comment}

\begin{ejercicio}\label{ej:4.7}
La Tabla~\ref{tab:ej:4.7} muestra los tiempos de ejecución, en minutos, de una serie de programas de prueba en dos sistemas informáticos SI1 y SI2, junto con el número de instrucciones ejecutadas por cada programa.
\begin{table}[h]
\centering
\begin{tabular}{@{}lrrl@{}}
\toprule
Programa & SI1 [min] & SI2 [min] & Instrucciones\\
\midrule
\verb|charlie brown| & 35 & 70 & $3.55 \cdot 10^8$ \\
\verb|lucy| & 101 & 78 & $7.78 \cdot 10^{13}$ \\
\verb|linus| & 57 & 55 & $9.12 \cdot 10^7$ \\
\verb|patty| & 76 & 83 & $2.94 \cdot 10^{10}$ \\ \bottomrule
\end{tabular}
\caption{Tiempos de ejecución e instrucciones ejecutadas en SI1 y SI2.}
\label{tab:ej:4.7}
\end{table}
\begin{enumerate}
    \item Suponiendo que todos los programas tienen la misma importancia en este estudio de evaluación, compárese el rendimiento de estos dos sistemas según:
    \begin{enumerate}
        \item Media aritmética de los tiempos de ejecución.
        \begin{align*}
            \ol{t}_{SI1} &= \dfrac{35 + 101 + 57 + 76}{4} = \dfrac{269}{4} = \unit[67.25]{min}\\
            \ol{t}_{SI2} &= \dfrac{70 + 78 + 55 + 83}{4} = \dfrac{286}{4} = \unit[71.5]{min}
        \end{align*}

        Como $\ol{t}_{SI1} < \ol{t}_{SI2}$, el sistema SI1 es más rápido que el SI2. Calculemos cuántas veces es más rápido:
        \begin{align*}
            \dfrac{\ol{t}_{SI2}}{\ol{t}_{SI1}} &= \dfrac{71.5}{67.25} \approx 1.0632
        \end{align*}

        Por tanto, el sistema SI1 es aproximadamente $1.06$ veces más rápido que el SI2.
        

        \item MIPS (millones de instrucciones ejecutadas por segundo).
        \begin{align*}
            \text{MPIS}_{SI1} = \dfrac{3.55 \cdot 10^8 + 7.78 \cdot 10^{13} + 9.12 \cdot 10^7 + 2.94 \cdot 10^{10}}{(35 + 101 + 57 + 76) \cdot 60\cdot 10^6} &= \approx \unit[4822.17]{MIPS}\\
            \text{MPIS}_{SI2} = \dfrac{3.55 \cdot 10^8 + 7.78 \cdot 10^{13} + 9.12 \cdot 10^7 + 2.94 \cdot 10^{10}}{(70 + 78 + 55 + 83) \cdot 60\cdot 10^6} &= \approx \unit[4535.54]{MIPS}
        \end{align*}

        Por tanto, como $\text{MPIS}_{SI1} > \text{MPIS}_{SI2}$, el sistema SI1 es más rápido que el SI2. Calculemos cuántas veces es más rápido:
        \begin{align*}
            \dfrac{\text{MPIS}_{SI1}}{\text{MPIS}_{SI2}} &= \dfrac{4822.17}{4535.54} \approx 1.0632
        \end{align*}

        Notemos que, como es de esperar, hemos obtenido el mismo resultado, aunque la primera forma es más sencilla.

    \end{enumerate}
    \item Repítase la primera parte del estudio suponiendo que los pesos atribuidos a cada programa de prueba son, respectivamente: $0.5$, $0.1$, $0.1$, y $0.3$.
    
    Calculemos la media ponderada de los tiempos de ejecución: 
    \begin{align*}
        \ol{t_W}_{SI1} &= 0.5\cdot 35 + 0.1\cdot 101 + 0.1\cdot 57 + 0.3\cdot 76= \unit[56.1]{min}\\
        \ol{t_W}_{SI2} &= 0.5\cdot 70 + 0.1\cdot 78 + 0.1\cdot 55 + 0.3\cdot 83 = \unit[73.2]{min}
    \end{align*}

    Por tanto, como $\ol{t_W}_{SI1} < \ol{t_W}_{SI2}$, el sistema SI1 es más rápido que el SI2. Calculemos cuántas veces es más rápido:
    \begin{align*}
        \dfrac{\ol{t_W}_{SI2}}{\ol{t_W}_{SI1}} &\approx 1.305
    \end{align*}

    Por tanto, el sistema SI1 es aproximadamente $1.3$ veces más rápido que el SI2.
\end{enumerate}
\end{ejercicio}
\begin{comment}Sol:
1. El sistema SI1 es 1,06 veces más rápido que SI2 atendiendo al tiempo de ejecución. Los MIPS obtenidos por ambos sistemas son, respectivamente, 4822,2 y 4535,5.
2. En este nuevo escenario, el sistema SI1 es 1,3 veces más rápido que SI2 atendiendo al tiempo de ejecución.
\end{comment}

\begin{ejercicio}\label{ej:4.8}
La Tabla~\ref{tab:ej:4.8} muestra los resultados obtenidos tras la ejecución de tres programas de prueba en un computador con un procesador que dispone de un reloj de $\unit[2]{GHz}$.
\begin{table}[h]
\centering
\begin{tabular}{@{}crrr@{}}
\toprule
Programa & Instrucciones ($\cdot 10^9$) & Ciclos por instrucción & Operaciones coma flotante ($\cdot 10^6$) \\ \midrule
1 & $150$ & $3.5$ & $50$ \\
2 & $35$ & $2.8$ & $20$ \\
3 & $250$ & $5.2$ & $175$ \\ \bottomrule
\end{tabular}
\caption{Resultados de la ejecución de programas de prueba.}
\label{tab:ej:4.8}
\end{table}
Indique, a partir de los datos anteriores, los siguientes índices de rendimiento de este computador:
\begin{enumerate}
    \item MIPS.
    
    En primer lugar, hemos de calcular el tiempo que tarda en ejecutarse cada programa:
    \begin{align*}
        t_1 &= \unit[150\cdot 10^9]{instrucciones}\cdot \dfrac{\unit[3.5]{ciclos}}{\unit[1]{instruccion}}\cdot \dfrac{\unit[1]{s}}{\unit[2\cdot 10^9]{ciclos}} = 262.5 \text{ s}\\
        t_2 &= \unit[35\cdot 10^9]{instrucciones}\cdot \dfrac{\unit[2.8]{ciclos}}{\unit[1]{instruccion}}\cdot \dfrac{\unit[1]{s}}{\unit[2\cdot 10^9]{ciclos}} = 49 \text{ s}\\
        t_3 &= \unit[250\cdot 10^9]{instrucciones}\cdot \dfrac{\unit[5.2]{ciclos}}{\unit[1]{instruccion}}\cdot \dfrac{\unit[1]{s}}{\unit[2\cdot 10^9]{ciclos}} = 650 \text{ s}
    \end{align*}

    Ahora, calculamos el número de MIPS:
    \begin{align*}
        \text{MIPS} &= \dfrac{(150 + 35 + 250)\cdot 10^9}{(262.5 + 49 + 650)\cdot 10^6} \approx 452.42
    \end{align*}
    \item MFLOPS.
    \begin{align*}
        \text{MFLOPS} &= \dfrac{(50 + 20 + 175)\cdot 10^6}{(262.5 + 49 + 650)\cdot 10^6} \approx 0.25
    \end{align*}
    \item CPI.
    
    El número medio de ciclos por instrucción (CPI) se calcula como:
    \begin{align*}
        \text{CPI} &= \dfrac{(150\cdot 3.5 + 35\cdot 2.8 + 250\cdot 5.2)\cdot 10^9}{(150 + 35 + 250)\cdot 10^9} \approx 4.42
    \end{align*}
\end{enumerate}
\end{ejercicio}
\begin{comment}Sol:
Los índices obtenidos son: 452,42 MIPS, 0,25 MFLOPS y 4,4 CPI.
\end{comment}

\begin{ejercicio}\label{ej:4.9}
La página oficial de SPEC muestra en la Tabla~\ref{tab:ej:4.9} los resultados de rendimiento para dos sistemas informáticos de la casa comercial ACER obtenidos mediante el conocido benchmark CPU2006.
\begin{table}[h]
\centering
\begin{tabular}{@{}clrr@{}}
\toprule
Sistema & Modelo & SPECint\_base2006 & SPECint2006 \\ \midrule
A & Altos G5350 (AMD Opteron 246) & 13,47 & 14,38 \\
B & Altos G5350 (AMD Opteron 254) & 17,88 & 19,18 \\ \bottomrule
\end{tabular}
\caption{Resultados de rendimiento de sistemas informáticos ACER.}
\label{tab:ej:4.9}
\end{table}
\begin{enumerate}
    \item ¿Cuál de los dos sistemas presenta mejor rendimiento? Cuantifique numéricamente la mejora.
    
    El sistema B presenta mejor rendimiento que el A atendiendo a ambos criterios. Cuantifiquemos la mejora numéricamente:
    \begin{align*}
        \text{Mejora en SPECint\_base} &= \dfrac{17.88}{13.47} \approx 1.3274\\
        \text{Mejora en SPECint} &= \dfrac{19.18}{14.38} \approx 1.3338
    \end{align*}

    Por tanto, el sistema B es aproximadamente $1.33$ veces mejor que el A en ambos índices.
    \item A la vista de los resultados anteriores, ¿afecta al rendimiento de ambos sistemas la optimización llevada a cabo por el compilador en las pruebas de evaluación?
    
    Sí, la optimización del compilador mejora el rendimiento de ambos sistemas. Calculémosla:
    \begin{align*}
        S_A = \dfrac{14.38}{13.47} \approx 1.0675\\
        S_B = \dfrac{19.18}{17.88} \approx 1.0721
    \end{align*}

    Por tanto, vemos que hay mejora en ambos casos.
    \item ¿En qué medida se reflejará en los resultados anteriores una mejora importante en la unidad de coma flotante (\verb|FPU, floating point unit|) del procesador?
    
    En teoría no afectarán, puesto que estos índices se realizan empleando programas que trabajan con aritmética entera, como muestra el elemento \verb|int| del nombre.
    \item ¿Cuál de los dos sistemas ejecutará el benchmark Whetstone más rápidamente?
    
    No se puede saber, puesto que este benchmark es de aritmética en coma flotante y los índices mostrados afectan únicamente a la aritmética entera.
\end{enumerate}
\end{ejercicio}
\begin{comment}Sol:
    1. El sistema B obtiene un mayor rendimiento. En particular, las mejoras sobre el sistema A en los índices SPECint\_base y SPECint son, respectivamente, 1,33 y 1,33 (se obtiene la misma ganancia en ambos casos).
    2. La optimización de la compilación permite obtener una mejora, en ambos casos, de 1,07.
    3. En teoría no afectará porque los índices mostrados afectan únicamente a la aritmética entera.
    4. No se puede saber porque este benchmark es de aritmética en coma flotante.
\end{comment}

\begin{ejercicio}\label{ej:4.10}
Responda brevemente a las siguientes cuestiones sobre el benchmark CPU2017 que ha desarrollado el consorcio SPEC:
\begin{enumerate}
    \item ¿Qué componentes del sistema informático evalúa?
    
    Se evalúa el procesador, pero también se evalúa el sistema de memoria y el compilador (puesto que se proporcionan tan solo los ficheros en código fuente).
    \item ¿Cuáles son los lenguajes en que están programados los diferentes programas que lo integran?
    
    Los programas están escritos en C, C++ y Fortran.
    \item ¿Cuál es la diferencia entre el índice CPU2017IntegerSpeed\_peak y el índice CPU2017IntegerSpeed\_base?
    
    El índice CPU2017IntegerSpeed\_peak se obtiene en base a la ejecución de los programas, pero empleando parámetros de optimización a la hora de compilar específicos para cada uno de ellos, buscando el menor tiempo de ejecución posible en dicha máquina para ese programa.

    Por el contrario, el índice CPU2017IntegerSpeed\_base se obtiene con los mismos programas, pero compilados con opciones de compilación genéricas y comunes a todos los programas, sin buscar la optimización específica para cada uno.
    \item Indique cómo se calcula el índice CPU2017IntegerSpeed\_peak. El método de cálculo empleado, ¿satisface todas las exigencias de un buen índice de prestaciones? Razone la respuesta.
    
    Supongamos que se emplean $n$ programas. Cada programa $i$ se ejecuta $3$ veces, y se toma como $t_i$ el valor intermedio (descartándose los valores extremos). Asísmismo, se hace lo mismo en una máquina de referencia, y sea $t_i^{\text{REF}}$ el valor de referencia. El valor del índice SPEC correspondiente es:
    \begin{align*}
        \text{SPEC} &= \sqrt[n]{\prod_{i=1}^{n} \dfrac{t_i^{\text{REF}}}{t_i}}
    \end{align*}

    Aunque es un buen índice de prestaciones, no satisface todas las exigencias de un buen índice de prestaciones, ya que no refleja de manera correcta la comparación basada en los tiempos de ejecución debido al uso de la media geométrica.
\end{enumerate}
\end{ejercicio}
\begin{comment}Sol:
1. El procesador, el sistema de memoria y el compilador.
2. C, C++ y Fortran.
3. El primero se obtiene con los programas compilados, cada uno de ellos, con parámetros que optimizan la ejecución del código en la máquina que se evalúa con el objetivo de conseguir el menor tiempo de ejecución posible (rendimiento pico). El segundo utiliza opciones de compilación genéricas y comunes a todos los programas.
4. Se usa la media geométrica de los ratios obtenidos dividiendo los tiempos de ejecución en la máquina de referencia con los de la máquina que se evalúa, es decir, la media geométrica de las ganancias en velocidad con respecto a una máquina de referencia. Este método de cálculo no satisface las exigencias de un buen índice de prestaciones ya que no refleja de manera correcta la comparación basada en los tiempos de ejecución.
\end{comment}

\begin{ejercicio}\label{ej:4.11}
En un computador se ha llevado a cabo un estudio para determinar si el tipo de memoria principal es un factor importante en su rendimiento. Para ello se ha medido el tiempo de ejecución de seis programas con dos tipos de memoria: MA (más rápida y más cara) y MB (más lenta y más barata). Las medidas de los tiempos de ejecución (en segundos) de los programas son los mostrados en la Tabla~\ref{tab:ej:4.11}.
\begin{table}[h]
\centering
\begin{tabular}{@{}lrr@{}}
\toprule
Programa & MA [s] & MB [s] \\ \midrule
\verb|lucho| & 45 & 48 \\
\verb|lupita| & 32 & 35 \\
\verb|lulila| & 51 & 56 \\
\verb|lurdo| & 43 & 49 \\
\verb|lutecio| & 48 & 51 \\ \bottomrule
\end{tabular}
\caption{Tiempos de ejecución de programas con diferentes tipos de memoria.}
\label{tab:ej:4.11}
\end{table}
Calcule si las diferencias observadas son significativas al 95\% de confianza y, en caso afirmativo, determine la mejora de velocidad conseguida debido al uso del tipo de memoria más rápida. 
\begin{observacion}
Debe emplear la tabla de la distribución T-Student.
\end{observacion}

Supongamos la hipótesis nula $H_0$, es decir, que las diferencias no son significativas ($\ol{d}_{\text{real}}=0$). Entonces, las diferencias $d_i$ siguen una distribución normal de media $0$. Calculemos su media y su desviación típica (sabiendo que tiene $5-1=4$ grados de libertad):
\begin{align*}
    \ol{d} &= \dfrac{3 + 3 + 5 + 6 + 3}{5} = \dfrac{20}{5} = 4\\
    s &= \sqrt{\dfrac{1}{4}\cdot \sum_{i=1}^5 \left(d_i-\ol{d}\right)^2} = \sqrt{\dfrac{1}{4}\cdot \left(1+1+1+4+1\right)} = \sqrt{2}
\end{align*}

Por tanto, el valor de $t_{exp}$ es:
\begin{align*}
    t_{exp} &= \dfrac{\ol{d}}{\nicefrac{s}{\sqrt{5}}} = \dfrac{4\sqrt{5}}{\sqrt{2}} = 2\sqrt{10}\approx 6.3245
\end{align*}

Ahora, tomamos $\alpha=0.05$ como nivel de significabilidad, y buscamos en la tabla de la distribución T-Student con $4$ grados de libertad, y vemos que es $2.7764$. Como $|t_{exp}| > 2.7764$, podemos rechazar la hipótesis nula $H_0$ y concluir que las diferencias son significativas al 95\% de confianza. La mejora de velocidad conseguida debido al uso del tipo de memoria más rápida es:
\begin{align*}
    S &= \dfrac{48 + 35 + 56 + 49 + 51}{45 + 32 + 51 + 43 + 48} = \dfrac{239}{219} \approx 1.091
\end{align*}

Por tanto, la memoria MA permite obtener una mejora de velocidad del $9.1\%$ con respecto a la memoria MB.
\end{ejercicio}

\begin{comment}Sol:
Las diferencias son significativas. La memoria MA permite obtener una mejora de velocidad del 9\% con respecto a la memoria MB.
\end{comment}

\begin{ejercicio}\label{ej:4.12}
La empresa \emph{Facebook} está estudiando dos grandes propuestas con el objetivo de actualizar los computadores personales de su oficina principal en Menlo Park, California. El precio de cada computador es de $\unit[1850]{\text{\euro}}$ para el Modelo A y $\unit[2200]{\text{\euro}}$ para el Modelo B. Los responsables informáticos de la empresa han ejecutado los ocho programas que utilizan habitualmente en un computador de cada propuesta, y han obtenido los tiempos de ejecución, expresados en segundos, que se muestran en la Tabla~\ref{tab:ej:4.12}.
\begin{table}[h]
\centering
\begin{tabular}{@{}lrrr@{}}
\toprule
Programa & Modelo A [s] & Modelo B [s] \\ \midrule
1 & $23.6$ & $24.0$ \\
2 & $33.7$ & $41.6$ \\
3 & $10.1$ & $8.7$ \\
4 & $12.9$ & $13.5$ \\
5 & $67.8$ & $66.4$ \\
6 & $9.3$ & $15.2$ \\
7 & $47.4$ & $50.5$ \\
8 & $54.9$ & $52.3$ \\ \bottomrule
\end{tabular}
\caption{Tiempos de ejecución de programas en los Modelos A y B.}
\label{tab:ej:4.12}
\end{table}
Determínese, para un nivel de confianza del 95\%, si existen diferencias significativas en el rendimiento de los computadores personales de las dos propuestas y qué opción sería mejor.
\begin{observacion}
Debe emplear la tabla de la distribución T-Student.
\end{observacion}

Supongamos la hipótesis nula $H_0$, es decir, que las diferencias no son significativas ($\ol{d}_{\text{real}}=0$). Entonces, las diferencias $d_i$ siguen una distribución normal de media $0$. Calculemos su media y su desviación típica (sabiendo que tiene $8-1=7$ grados de libertad):
\begin{align*}
    \ol{d} &= \dfrac{-0.4 - 7.9 + 1.4 - 0.6 + 1.4 - 5.9 - 3.1 + 2.6}{8} \approx -1.5625\\
    s &= \sqrt{\dfrac{1}{7}\cdot \sum_{i=1}^8 \left(d_i-\ol{d}\right)^2} \approx 3.751167
\end{align*}

Por tanto, el valor de $t_{exp}$ es:
\begin{align*}
    t_{exp} &= \dfrac{\ol{d}}{\nicefrac{s}{\sqrt{8}}} \approx \dfrac{-1.5625}{\nicefrac{3.751167}{\sqrt{8}}} \approx -1.178
\end{align*}

Ahora, tomamos $\alpha=0.05$ como nivel de significabilidad, y buscamos en la tabla de la distribución T-Student con $7$ grados de libertad, y vemos que es $2.3646$. Como $|t_{exp}| < 2.3646$, no podemos rechazar la hipótesis nula $H_0$ y concluimos que las diferencias no son significativas al 95\% de confianza. Por tanto, hemos de optar por la opción más barata, que es el Modelo A.
\end{ejercicio}
\begin{comment}Sol:
Como $t_{exp} = -1.18$ se encuentra dentro del intervalo $[-2.365, 2.365]$, no podemos descartar la hipótesis de que los rendimientos sean equivalentes. A la misma conclusión podríamos llegar calculando el intervalo de confianza para el valor medio real de las diferencias $[-4.696, 1.576]$. Como este intervalo incluye el 0, podemos afirmar que las diferencias observadas en los tiempos de ejecución no son significativas. En consecuencia, la mejor opción para actualizar los computadores de la empresa es la opción A, ya que resulta menos costosa.
\end{comment}

\begin{ejercicio}\label{ej:4.13}
La Tabla~\ref{tab:ej:4.13} muestra los tiempos de ejecución (en segundos) medidos en tres computadores, A, B y R, para un conjunto de cinco programas de prueba.
\begin{table}[h]
\centering
\begin{tabular}{@{}lrrr@{}}
\toprule
Programa & A [s] & B [s] & R [s] \\ \midrule
1 & $96.2$ & $95.3$ & $103.9$ \\
2 & $13.1$ & $10.2$ & $53.8$ \\
3 & $79.6$ & $67.4$ & $156.3$ \\
4 & $45.2$ & $51.8$ & $98.1$ \\
5 & $88.3$ & $89.3$ & $238.5$ \\ \bottomrule
\end{tabular}
\caption{Tiempos de ejecución de programas en diferentes máquinas.}
\label{tab:ej:4.13}
\end{table}
Calcule el índice de prestaciones de las máquinas A y B según se hace en el benchmark SPEC\_CPU, tomando como referencia la máquina R. Compárese el rendimiento de estas máquinas atendiendo tanto a este índice como al tiempo total de ejecución. ¿Hay diferencias significativas con un grado de confianza del 95\%?
\begin{observacion}
Debe usar la tabla de la distribución T-Student.
\end{observacion}

Para calcular el índice de prestaciones SPEC, tenemos que:
\begin{align*}
    \text{SPEC\_A} &= \sqrt[5]{\dfrac{103.9}{96.2} \cdot \dfrac{53.8}{13.1} \cdot \dfrac{156.3}{79.6} \cdot \dfrac{98.1}{45.2} \cdot \dfrac{238.5}{88.3}} \approx 2.196\\
    \text{SPEC\_B} &= \sqrt[5]{\dfrac{103.9}{95.3} \cdot \dfrac{53.8}{10.2} \cdot \dfrac{156.3}{67.4} \cdot \dfrac{98.1}{51.8} \cdot \dfrac{238.5}{89.3}} \approx 2.3216
\end{align*}

Ahora, comparemos los índices:
\begin{align*}
    \dfrac{\text{SPEC\_B}}{\text{SPEC\_A}} &= \dfrac{2.3216}{2.196} \approx 1.057
\end{align*}

Por tanto, según el índice SPEC, la máquina B es aproximadamente $1.06$ veces mejor que la máquina A.
Ahora, calculemos el tiempo total de ejecución:
\begin{align*}
    T_A &= 96.2 + 13.1 + 79.6 + 45.2 + 88.3 = \unit[322.4]{s}\\
    T_B &= 95.3 + 10.2 + 67.4 + 51.8 + 89.3 = \unit[314.0]{s}
\end{align*}

Comparemos ahora los tiempos totales de ejecución:
\begin{align*}
    \dfrac{T_A}{T_B} &= \dfrac{322.4}{314.0} \approx 1.0267
\end{align*}

Por tanto, la máquina B es aproximadamente $1.03$ veces más rápida que la máquina A en términos de tiempo total de ejecución. Veamos ahora si las diferencias son significativas al 95\% de confianza. Para ello, calculamos las diferencias entre los tiempos de ejecución de A y B:
\begin{align*}
    d_1 &= 96.2 - 95.3 = 0.9\\
    d_2 &= 13.1 - 10.2 = 2.9\\
    d_3 &= 79.6 - 67.4 = 12.2\\
    d_4 &= 45.2 - 51.8 = -6.6\\
    d_5 &= 88.3 - 89.3 = -1.0
\end{align*}

Calculamos la media y la desviación típica de las diferencias, sabiendo que hay $5-1=4$ grados de libertad:
\begin{align*}
    \ol{d} &= \dfrac{0.9 + 2.9 + 12.2 - 6.6 - 1.0}{5} = \dfrac{8.4}{5} = 1.68\\
    s &= \sqrt{\dfrac{1}{4}\cdot \sum_{i=1}^5 \left(d_i-\ol{d}\right)^2} \approx 6.86
\end{align*}

Por tanto, el valor de $t_{exp}$ es:
\begin{align*}
    t_{exp} &= \dfrac{\ol{d}}{\nicefrac{s}{\sqrt{5}}}  \approx 0.547216
\end{align*}

Ahora, tomamos $\alpha=0.05$ como nivel de significabilidad, y buscamos en la tabla de la distribución T-Student con $4$ grados de libertad, y vemos que es $2.7764$. Como $|t_{exp}| < 2.7764$, no podemos rechazar la hipótesis nula $H_0$ y concluimos que las diferencias no son significativas al 95\% de confianza.

\end{ejercicio}
\begin{comment}Sol:
La media geométrica de los tiempos de ejecución normalizados respecto de la máquina R son $2.20$ y $2.32$ para A y B respectivamente. En cualquier caso, estos datos permiten concluir que B rinde $1.06$ veces más que A. En cambio, la suma de los tiempos de ejecución son $322.4$s y $314.0$s respectivamente, lo que rebaja la mejora conseguida a $1.03$. Respecto a las diferencias, éstas no son significativas al 0.05 de significatividad porque el intervalo de confianza $[-6.8, 10.2]$ incluye el 0.
\end{comment}

\begin{ejercicio}\label{ej:4.14}
Una gran empresa de seguros está estudiando dos propuestas con el objetivo de actualizar los computadores de su instalación informática. El precio de cada computador es de $\unit[1300]{\text{\euro}}$ para los de tipo A y $\unit[1450]{\text{\euro}}$ para los de tipo B. Se estima que el número de computadores a reemplazar es de $75$. El ingeniero informático jefe de la empresa ha mandado ejecutar cinco de los programas que utilizan habitualmente en un computador de cada propuesta y ha obtenido los tiempos de ejecución, expresados en segundos, que se muestran en la Tabla~\ref{tab:ej:4.14}.
\begin{table}[h]
\centering
\begin{tabular}{@{}crr@{}}
\toprule
Programa & Propuesta A [s] & Propuesta B [s] \\ \midrule
1 & $23.6$ & $24.5$ \\
2 & $33.7$ & $41.6$ \\
3 & $10.1$ & $6.6$ \\
4 & $12.9$ & $13.7$ \\
5 & $67.8$ & $66.4$ \\ \bottomrule
\end{tabular}
\caption{Tiempos de ejecución de programas en las Propuestas A y B.}
\label{tab:ej:4.14}
\end{table}
\begin{enumerate}
    \item Calcule el índice de prestaciones de las máquinas A y B según se hace en el benchmark SPEC\_CPU, tomando como referencia la máquina A. Según ese índice, y suponiendo que no hay aleatoriedad en las medidas, ¿qué opción es la mejor? ¿qué opción sería la que compraría ateniéndonos a la relación prestaciones/coste?
    
    Para calcular el índice de prestaciones SPEC, tenemos que:
\begin{align*}
    \text{SPEC\_A} &= 1\\
    \text{SPEC\_B} &= \sqrt[5]{\dfrac{23.6}{24.5} \cdot \dfrac{33.7}{41.6} \cdot \dfrac{10.1}{6.6} \cdot \dfrac{12.9}{13.7} \cdot \dfrac{67.8}{66.4}} \approx 1.028
\end{align*}
Por tanto, según el índice SPEC, la máquina B es aproximadamente $1.028$ veces mejor que la máquina A. Ahora, calculemos la relación prestaciones/coste, siendo las prestaciones el valor de SPEC:
\begin{align*}
    \text{Relación prestaciones/coste}_A &= \dfrac{1}{1300} \approx 7.69\cdot 10^{-4} \text{\euro}^{-1}\\
    \text{Relación prestaciones/coste}_B &= \dfrac{1.028}{1450} \approx 7.08\cdot 10^{-4} \text{\euro}^{-1}
\end{align*}
Por tanto, la opción A es la mejor opción a comprar ateniéndonos a la relación prestaciones/coste, ya que es más barata y ofrece un mejor rendimiento por euro invertido.
    \item Suponiendo que hay aleatoriedad en las medidas, determine si existen diferencias significativas (para un nivel de confianza del 95\%) en el rendimiento de los computadores de las dos propuestas y qué opción sería la que compraría según esa información ateniéndonos a la relación prestaciones/coste. Justifique la respuesta.
\begin{observacion}
Debe usar la tabla de la distribución T-Student.
\end{observacion}

Supongamos la hipótesis nula $H_0$, es decir, que las diferencias no son significativas ($\ol{d}_{\text{real}}=0$). Entonces, las diferencias $d_i$ siguen una distribución normal de media $0$. Calculemos su media y su desviación típica (sabiendo que tiene $5-1=4$ grados de libertad):
\begin{align*}
    \ol{d} &= \dfrac{-0.9 - 7.9 + 3.5 - 0.8 + 1.4}{5} = \dfrac{-4.7}{5} = -0.94\\
    s &= \sqrt{\dfrac{1}{4}\cdot \sum_{i=1}^5 \left(d_i-\ol{d}\right)^2} \approx 4.2910371
\end{align*}

Por tanto, el valor de $t_{exp}$ es:
\begin{align*}
    t_{exp} &= \dfrac{\ol{d}}{\nicefrac{s}{\sqrt{5}}} \approx \dfrac{-0.94}{\nicefrac{4.2910371}{\sqrt{5}}} \approx -0.48983
\end{align*}

Ahora, tomamos $\alpha=0.05$ como nivel de significabilidad, y buscamos en la tabla de la distribución T-Student con $4$ grados de libertad, y vemos que es $2.7764$. Como $|t_{exp}| < 2.7764$, no podemos rechazar la hipótesis nula $H_0$ y concluimos que las diferencias no son significativas al 95\% de confianza. Por tanto, a falta de otra información, hemos de optar por la opción más barata, que es la Propuesta A.
\end{enumerate}
\end{ejercicio}
\begin{comment}Sol:
a) SPEC\_A = 1, SPEC\_B = 1.028. Según ese índice, la opción más rápida es la B. Mirando la relación prestaciones/coste, siendo las prestaciones el valor de SPEC, la opción que deberíamos comprar es la A ($7.69 \cdot 10^{-4} \text{\euro}^{-1}$ frente a $7.08 \cdot 10^{-4} \text{\euro}^{-1}$).

b) Las diferencias en los tiempos de ejecución entre la propuesta A y la B ($t_A - t_B$) están en el intervalo $[-6.3, 4.4]$ por lo que no son significativas al 95\%. A la misma conclusión llegaríamos si calculamos $t_{exp} = -0.49$, claramente dentro del intervalo $\pm 2.78$. De este modo, a falta de otra información, la opción a comprar debería ser la más barata (propuesta A).
\end{comment}

\begin{ejercicio}\label{ej:4.15}
En la empresa KANDOR GRAPHICS están intentando mejorar la técnica de distribución de carga de su servidor principal de streaming de vídeo. Para ello, han realizado cinco medidas de la productividad media del servidor durante un número determinado, pero fijo, de horas para las 2 configuraciones principales de distribución de carga: \emph{Least Connected} (LC) y \emph{Round Robin} (RR). Los resultados, expresados como número medio de MB transmitidos por segundo son los que aparecen en la Tabla~\ref{tab:ej:4.15_1}.
\begin{table}[h]
\centering
\begin{tabular}{@{}crr@{}}
\toprule
Nº Experimento & $X_{LC}$ ($\nicefrac{\text{MB}}{\text{s}}$) & $X_{RR}$ ($\nicefrac{\text{MB}}{\text{s}}$) \\ \midrule
1 & 157 & 165 \\
2 & 125 & 123 \\
3 & 172 & 185 \\
4 & 152 & 158 \\
5 & 165 & 172 \\ \bottomrule
\end{tabular}
\caption{\centering Productividad media del servidor en diferentes configuraciones de distribución de carga.}
\label{tab:ej:4.15_1}
\end{table}
Como existe aleatoriedad en los experimentos, para poder estar seguros de la decisión, los ingenieros informáticos de la empresa realizaron un test $t$ sobre estos datos, obteniéndose los resultados de la Tabla~\ref{tab:ej:4.15_2}.
\begin{table}[h]
\centering
\begin{tabular}{@{}lccccccc@{}}
\toprule
Med. 1 & Med. 2 & $t$ & $df$ & $p$ & Dif. media. & ES & IC 90\% para dif. media \\ \midrule
LC & RR & $-2.64$ & $4$ & $0.057$ & $-6.4$ & $2.42$ & $[-11.56, -1.24]$ \\ \bottomrule
\end{tabular}
\caption{\centering Resultados del test $t$ para la comparación de configuraciones de distribución de carga.}
\label{tab:ej:4.15_2}
\end{table}
\begin{enumerate}
    \item A la vista de los resultados y para un 90\% de confianza, ¿qué configuración de distribución de carga utilizaría y por qué?
    
    Sea $\alpha=0.1$ el nivel de significación. Como el valor del $p-$value es $0.057<0.1$, podemos rechazar la hipótesis nula $H_0$ de que las dos configuraciones son equivalentes y concluir que las diferencias son significativas al 90\% de confianza. Empleamos por tanto la que tenga mayor productividad media. Estas son:
    \begin{align*}
        \ol{X}_{LC} &= \dfrac{157 + 125 + 172 + 152 + 165}{5} = \dfrac{771}{5} = \unitfrac[154.2]{MB}{s}\\
        \ol{X}_{RR} &= \dfrac{165 + 123 + 185 + 158 + 172}{5} = \dfrac{803}{5} = \unitfrac[160.6]{MB}{s}
    \end{align*}

    Por tanto, la configuración de distribución de carga que utilizaría sería la \emph{Round Robin} (RR), ya que tiene una productividad media mayor. La ganancia en productividad media es de:
    \begin{align*}
        \dfrac{\ol{X}_{RR}}{\ol{X}_{LC}} = \dfrac{160.6}{154.2} \approx 1.04
    \end{align*}
    Por tanto, la productividad media de la configuración \emph{Round Robin} es aproximadamente un $4\%$ mayor que la del método \emph{Least Connected}.
    

    \item ¿Qué conclusiones podríamos extraer para un 99\% de confianza? Razone de forma general por qué y en qué medida puede afectar el \% de confianza a la decisión a tomar.

    Sea ahora $\alpha=0.01$ el nivel de significación. Como el valor del $p-$value es $0.057>0.01$, no podemos rechazar la hipótesis nula $H_0$ de que las dos configuraciones son equivalentes y concluimos que las diferencias no son significativas al 99\% de confianza. Por tanto, a este nivel de confianza, no podríamos decidir entre las dos configuraciones, ya que ambas serían equivalentes.\\

    En general, al aumentar el grado de confianza cada vez es más difícil descartar la hipótesis nula (llegando a ser imposible si se trata del 100\% de confianza).
\end{enumerate}
\end{ejercicio}
\begin{comment}Sol:
a) Al 90\% de nivel de confianza las diferencias sí son significativas. Deberíamos elegir el de mayor productividad media. En este caso, el método de \emph{Round Robin} tiene $6.4$ MB/s más de productividad media ($160.6$ MB/s frente a los $154.2$ MB/s del \emph{Least Connected}, por lo que es $1.04$ veces mayor o un $4\%$ mayor).
b) Al 99\% ya no podríamos asegurar que las diferencias son significativas por lo que consideraríamos ambos métodos iguales. Al aumentar el \% de nivel confianza lo que realmente estoy haciendo es aumentar el ancho de los intervalos de confianza de tal forma que cada vez sea más difícil descartar la hipótesis nula de que ambas alternativas tienen rendimientos equivalentes. Otra forma de verlo es que al aumentar el \% de nivel de confianza, el grado de significatividad ($\alpha$) baja por lo que es más difícil rechazar la hipótesis nula (el $p$-value tiene que ser todavía más pequeño). Para un $100\%$ de nivel de confianza nunca podríamos rechazar $H_0$.
\end{comment}


\begin{ejercicio}\label{ej:4.16}
En la empresa \emph{KINGSTON} están intentando comprobar la mejora en la latencia de los módulos de memoria RAM que introduce la nueva técnica \emph{HyperX}. Para ello, han realizado $100$ experimentos para calcular las latencias medias en múltiples diferentes contextos. Finalmente, para comprobar que las diferencias en las latencias entre el método tradicional y la técnica \emph{HyperX} no se deben a efectos aleatorios, han realizado un test $t$, cuyos resultados son los que aparecen en la Tabla~\ref{tab:ej:4.16}.
\begin{table}[h]
\centering
\begin{tabular}{@{}lccccccc@{}}
\toprule
Med. 1 & Med. 2 & $t$ & $df$ & $p$ & Dif. media. & ES & IC 95\% para dif. media \\ \midrule
NORMAL & HYPERX & $4.59$ & $99$ & $0.000$ & $0.33$ & $0.073$ & $[0.188, 0.476]$ \\ \bottomrule
\end{tabular}
\caption{\centering Resultados del test $t$ para la comparación de latencias de módulos de memoria.}
\label{tab:ej:4.16}
\end{table}
\begin{enumerate}
    \item Mirando únicamente valores medios, ¿qué técnica parece ser mejor? Justifique la respuesta.
    
    Tenemos que la diferencia media  es positiva, por lo que notando por $\ol{N}$ a la latencia media de la RAM normal y por $\ol{X}$ a la latencia media de la RAM mejorada, tenemos que:
    \begin{align*}
        \ol{N-X} = \ol{N}-\ol{X}=0.33\Longrightarrow \ol{N}>\ol{X}
    \end{align*}

    Como una latencia media mayor supone un rendimiento peor, la técnica mejorada parece ser mejor.
    \item Al 95\% de nivel de confianza, ¿son significativas esas diferencias? Justifique la respuesta.
    
    Como $0\notin [0.188, 0.476]$, tenemos que podemos rechazar la hipótesis nula, luego las diferencias son significativas al 95\% de confianza. 
\end{enumerate}
    
\end{ejercicio}
\begin{comment}Sol:
a) La diferencia entre las latencias medias de ambas técnicas es $0.33$ ns, medida como latencia\_Normal - latencia\_HyperX. Por tanto, la latencia media usando el módulo de memoria Normal es $0.33$ ns mayor que la del módulo HyperX. Como una latencia mayor supone un rendimiento peor, la técnica mejor en este caso es la HyperX.
b) Las diferencias sí son significativas al 95\%. Podemos verlo fácilmente comprobando que el $0$ no está incluido en el intervalo de confianza del 95\% de las diferencias entre las latencias de ambas técnicas. También se puede comprobar viendo que el $p$-value (celda Sig. (2-tailed)), que mide la probabilidad de que ambas técnicas sean iguales es $0.000$, claramente inferior al valor umbral que marca el nivel de significatividad ($0.05$).
\end{comment}
\begin{ejercicio}\label{ej:4.17}
En la empresa \emph{SERENDIPITY S.L.} están intentando mejorar el servidor web que alberga las páginas de la Universidad de Granada. Para ello, han ejecutado un conocido benchmark de servidores web para $5$ configuraciones distintas del S.O. actualmente en uso. Como la fuente de variabilidad es alta debido a que las pruebas han tenido que realizarlas en el equipo ya actualmente en uso (se ha elegido el intervalo entre las $4$ y las $5$ de la mañana en días sucesivos) los experimentos se han realizado $10$ veces. Los resultados del número medio de páginas servidas por segundo se muestran en la Tabla~\ref{tab:ej:4.17}, y los resultados del análisis ANOVA son los que aparecen en la Tabla~\ref{tab:ej:4.17_anova_1} y Tabla~\ref{tab:ej:4.17_anova_2}.
\begin{table}[h]
\centering
\begin{tabular}{@{}lrrrrr@{}}
\toprule
Exp. & Conf. 1 & Conf. 2 & Conf. 3 & Conf. 4 & Conf. 5 \\ \midrule
1 & 152 & 155 & 178 & 162 & 178 \\
2 & 162 & 152 & 185 & 157 & 179 \\
3 & 165 & 163 & 182 & 153 & 181 \\
4 & 159 & 162 & 189 & 158 & 182 \\
5 & 148 & 154 & 190 & 162 & 189 \\
6 & 152 & 152 & 186 & 158 & 183 \\
7 & 156 & 158 & 195 & 152 & 188 \\
8 & 160 & 160 & 185 & 149 & 178 \\
9 & 163 & 152 & 194 & 149 & 182 \\
10 & 153 & 155 & 197 & 150 & 181 \\ \bottomrule
\end{tabular}
\caption{Número medio de páginas servidas por segundo en diferentes configuraciones del S.O.}
\label{tab:ej:4.17}
\end{table}

\begin{table}[h]
\centering
\begin{tabular}{@{}lrrrrr@{}}
\toprule
Cases & Sum of Squares & $df$ & Mean Square & $F$ & $p$ \\
\midrule
Config. & $10292.600$ & 4 & $2573.150$ & $102.299$ & $< .001$ \\
Residuals & $1131.900$ & 45 & $25.153$ & & \\
\bottomrule
\end{tabular}
\caption{ANOVA - Productividad (X)}
\label{tab:ej:4.17_anova_1}
\end{table}

\begin{table}[h]
\centering
\begin{tabular}{@{}llrrrrrr@{}}
\toprule
 & & Mean Difference & Lower (95\% CI) & Upper (95\% CI) & $SE$ & $t$ & $p$ \\
\midrule
$1$ & $2$ & $0.700$ & $-5.673$ & $7.073$ & $2.243$ & $0.312$ & $0.998$ \\
    & $3$ & $-31.100$ & $-37.473$ & $-24.727$ & $2.243$ & $-13.866$ & $< .001$ \\
    & $4$ & $2.000$ & $-4.373$ & $8.373$ & $2.243$ & $0.892$ & $0.898$ \\
    & $5$ & $-25.100$ & $-31.473$ & $-18.727$ & $2.243$ & $-11.191$ & $< .001$ \\
$2$ & $3$ & $-31.800$ & $-38.173$ & $-25.427$ & $2.243$ & $-14.178$ & $< .001$ \\
    & $4$ & $1.300$ & $-5.073$ & $7.673$ & $2.243$ & $0.580$ & $0.977$ \\
    & $5$ & $-25.800$ & $-32.173$ & $-19.427$ & $2.243$ & $-11.503$ & $< .001$ \\
$3$ & $4$ & $33.100$ & $26.727$ & $39.473$ & $2.243$ & $14.758$ & $< .001$ \\
    & $5$ & $6.000$ & $-0.373$ & $12.373$ & $2.243$ & $2.675$ & $0.074$ \\
$4$ & $5$ & $-27.100$ & $-33.473$ & $-20.727$ & $2.243$ & $-12.082$ & $< .001$ \\
\bottomrule
\end{tabular}
\caption{Post Hoc Comparisons - Config.}
\label{tab:ej:4.17_anova_2}
\end{table}

\begin{enumerate}
    \item Si atendiéramos exclusivamente a la media aritmética de los resultados, ¿qué configuración parecería la mejor?
    
    Tenemos que las medias aritméticas de número medio de páginas servidas por segundo es de:
    \begin{align*}
        \ol{C}_1 = 157\qquad
        \ol{C}_2 = 156.3\qquad 
        \ol{C}_3 = 188.1\qquad 
        \ol{C}_4 = 155\qquad 
        \ol{C}_5 = 182.1
    \end{align*}

    Por tanto, la mejor configuración sería la tercera.
    \item Para un nivel de confianza del 95\%, ¿afecta la configuración del S.O. al rendimiento del equipo?
    % // TODO: Hacer
    \item Para un nivel de confianza del 95\%, agrupe las configuraciones que afectan estadísticamente por igual. ¿Cuáles serían, en ese caso, las mejores configuraciones? ¿Y para un nivel de confianza del 90\%? Explique razonadamente los resultados.
\end{enumerate}
\end{ejercicio}
\begin{comment}Sol:
a) La mejor configuración sería la número $3$ con una media de $188.1$ páginas servidas por segundo.
b) Tras un análisis de ANOVA de $1$ factor, el valor del estadístico $F$ es de $102.3$, con una probabilidad ($p$-value = $0.000...$) mucho menor de $0.05$ de que pertenezca a una distribución $F$ con $4$ y $45$ grados de libertad. Por tanto, debemos descartar que el S.O. no tenga influencia sobre el rendimiento del servidor web $\rightarrow$ sí afecta.
c) Fijándonos en el p-value o, alternativamente, en el intervalo de confianza para la media real de las diferencias, vemos que, al 95\% de nivel de confianza, la prueba de múltiples rangos o de comparaciones múltiples nos indica que las configuraciones $1$, $2$ y $4$ afectan de igual forma al rendimiento. Igualmente, nos indica que no se puede rechazar la hipótesis de que las configuraciones $3$ y $5$ tengan rendimientos equivalentes, por lo que ambas serían igualmente las mejores. Sin embargo, al 90\% de nivel confianza, el grado de significatividad sería $\alpha = 0.1$ y sí rechazaríamos la hipótesis de que los rendimientos de las configuraciones $3$ y $5$ sean equivalentes (ahora solo podríamos fijarnos en la columna del p-value y no en el intervalo de confianza para la media real de las diferencias). Por tanto, para el 90\% de nivel de confianza concluiríamos que la mejor configuración es la $3$, luego la $5$ y después vendrían, agrupadas, las configuraciones $1$, $2$ y $4$.
\end{comment}


\begin{ejercicio}\label{ej:4.18}
Un estudiante de Ingeniería de Servidores ha realizado un estudio sobre la influencia del parámetro \emph{swappiness} del Sistema Operativo Linux sobre las prestaciones de su servidor Web. Para ello, ha realizado un total de $10$ experimentos, calculando el número máximo de conexiones simultáneas que su servidor Apache es capaz de manejar, para dos valores concretos de dicho parámetro a los que ha llamado \emph{BAJO} y \emph{ALTO}. Para poder estar seguro de que la diferencia entre las medias de los valores medidos sea estadísticamente significativa, este estudiante ha realizado un test $t$, obteniéndose los resultados de la Tabla~\ref{tab:ej:4.18}.
\begin{table}[h]
\centering
\begin{tabular}{@{}lccccccc@{}}
\toprule
Med. 1 & Med. 2 & $t$ & $df$ & $p$ & Mean Dif. & SE Dif. & 95\% CI for Mean Difference \\ \midrule
BAJO & ALTO & $2.11$ & $9$ & $0.064$ & $8.3$ & $3.93$ & Lower: $-0.6$, Upper: $17.2$ \\ \bottomrule
\end{tabular}
\caption{Resultados del test $t$ para el parámetro \emph{swappiness}.}
\label{tab:ej:4.18}
\end{table}
\begin{enumerate}
    \item A la vista de los resultados y para un $90\%$ de confianza, ¿qué método utilizaría y por qué?
    
    Como el valor del $p$-value es $p=0.064<0.1$, podemos rechazar la hipótesis nula $H_0$ de que las dos configuraciones son equivalentes y concluir que las diferencias son significativas al 90\% de confianza. Empleamos por tanto la que tenga mayor productividad media. Puesto que la diferencia media es positiva, tenemos que optamos por la configuración \emph{swappiness=``BAJO''} ya que permite, de media, $8.3$ más conexiones que con el parámetro de \emph{swappiness=``ALTO''}.
    \item ¿Qué conclusiones podríamos extraer para un $95\%$ de confianza? ¿Y para un $99\%$?
    
    Como $0\in [-0.6, 17.2]$, no podemos rechazar la hipótesis nula $H_0$ de que las dos configuraciones son equivalentes y concluimos que las diferencias no son significativas al 95\% de confianza, por lo que tampoco lo serían al 99\%. Por tanto, a este nivel de confianza, no podríamos decidir entre las dos configuraciones, ya que ambas serían equivalentes.
\end{enumerate}
\end{ejercicio}
\begin{comment}Sol:
a) Hay diferencias significativas ($p$-value $< 0.1$). Debo usar \emph{swappiness="BAJO"} ya que permite, de media, $8.3$ más conexiones que con el parámetro de \emph{swappiness="ALTO"}.
b) En ambos casos las diferencias no son significativas ($p$-value $> 0.05$ y $p$-value $> 0.01$).
\end{comment}

\begin{ejercicio}\label{ej:4.19}
Determine, al $95\%$ de nivel de confianza, un intervalo en el que se debe encontrar el tiempo medio de ejecución de un determinado programa escrito en Python. En la Tabla~\ref{tab:ej:4.19} se muestran los tiempos de ejecución obtenidos en $10$ experimentos independientes.
\begin{observacion}
    Debe usar la tabla de la distribución T-Student.
\end{observacion}
\begin{table}[h]
\centering
\begin{tabular}{@{}cr@{}}
\toprule
Experimento & Tiempo ejecución (s) \\ \midrule
1 & $15.2$ \\
2 & $16.2$ \\
3 & $16.5$ \\
4 & $15.9$ \\
5 & $14.8$ \\
6 & $15.2$ \\
7 & $15.6$ \\
8 & $16.0$ \\
9 & $16.3$ \\
10 & $15.3$ \\ \bottomrule
\end{tabular}
\caption{Tiempos de ejecución de un programa en Python.}
\label{tab:ej:4.19}
\end{table}
% // TODO: Hacer
\end{ejercicio}
\begin{comment}Sol:
El intervalo de confianza al $95\%$ para el tiempo medio de ejecución del programa en Python es $[15.3, 16.1]$ segundos.
\end{comment}
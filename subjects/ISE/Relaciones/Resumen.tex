\chapter{Resumen de fórmulas}
Este apartado trata de ser un resumen de la mayoría de las fórmulas vistas en cada tema de la asignatura, está pensado para recordar de forma rápida las fórmulas vistas en la asignatura, y no debe usarse como medio de estudio.

\section{Introducción a la Ingeniería de Servidores}
Los dos conceptos más importantes del tema son:
\begin{itemize}
    \item Tiempo de respuesta (latencia).
    \item Productividad (ancho de banda).
\end{itemize}
Si queremos comparar dos dispositivos $A$ y $B$ con tiempos $t_A$ y $t_B$, la ganancia en velocidad de $A$ respecto a $B$ viene dada por:
\begin{equation*}
    S_B(A) = \dfrac{t_A}{t_B}
\end{equation*}
Si tenemos un dispositivo que tarda en ejecutar una tarea un tiempo $T_0$ y mejoramos dicho dispositivo sustituyendoo un compoente que se usa durante una fracción de tiempo $f\in [0,1]$ de forma que mejoramos dicho componentes $k$ veces, el nuevo tiempo vendrá dado por:
\begin{equation*}
    T_M = (1-f)T_0 + \dfrac{fT_0}{k}
\end{equation*}
Bajo las mismas condiciones, la Ley de Amdahl nos dice que:
\begin{equation*}
    S = \dfrac{T_0}{T_M} = \dfrac{T_0}{(1-f)T_0 + \dfrac{fT_0}{k}} = \dfrac{1}{1-f+\dfrac{f}{k}}
\end{equation*}

\setcounter{section}{3}
\section{Análisis comparativo del rendimiento}
Si sabemos las instrucciones de un programa ($NI$), cuántas de ellas son en coma flotante ($FL$) y el tiempo que tarda un dispositivo en ejecutar el programa ($t$), podremos calcular:
\begin{align*}
    \text{MIPS} &= \dfrac{NI}{t\cdot 10^6} \\
    \text{MFLOPS} &= \dfrac{FL}{t\cdot 10^6}
\end{align*}
Si para un cierto programa conocemos el número de instrucciones necesarias para su ejecución ($NI$), el número medio de ciclos por instrucción de la CPU ($CPI$) y la frecuencia del procesador ($f$), podremos calcular el tiempo que tarda la CPU en ejecutar el programa, mediante:
\begin{equation*}
    T_{CPU} = NI\cdot CPI \cdot \dfrac{1}{f}
\end{equation*}
Si al ejecutar un benchmark de $n$ programas obtenemos las puntuaciones $t_1,t_2,\ldots,t_n$ y las puntuaciones de referencia eran $t_{REF_1}, t_{REF_2}, \ldots, t_{REF_n}$, podemos calcular el índice SPEC mediante:
\begin{equation*}
    SPEC = \sqrt[n]{\dfrac{t_{REF_1}}{t_1} \cdot \dfrac{t_{REF_2}}{t_2} \cdot \ldots \cdot \dfrac{t_{REF_n}}{t_n}}
\end{equation*}

\subsection{Distribución t-Student}
Si extraemos $n$ muestras de ejecución de varios programas por dos dispositivos o programas distintos y consideramos la diferencia de los datos obtenidos: $d_1,d_2,\ldots,d_n$, estamos interesados en calcular si las dos muestras tienen diferencias significativas con un grado de significatividad mayor al $95\%$. Para ello, lo que haremos será suponer la hipótesis nula $H_0$:
\begin{center}
    Las dos máquinas/programas tienen rendimientos equivalentes
\end{center}
Definimos:
\begin{itemize}
    \item La media de las diferencias:
        \begin{equation*}
            \ol{d} = \dfrac{1}{n}\sum_{i=1}^{n}d_i
        \end{equation*}
    \item La desviación típica muestral:
        \begin{equation*}
            s = \sqrt{\dfrac{\sum\limits_{i=1}^{n}{(d_i - \ol{d})}^{2}}{n-1}}
        \end{equation*}
    \item El error estándar:
        \begin{equation*}
            \dfrac{s}{\sqrt{n}}
        \end{equation*}
\end{itemize}
Supuesta la hipótesis nula $H_0$, tras un estudio podremos rechazar la hipótesis nula (por lo que los datos no son significativamente distintos) o no podremos rechazarla. Este estudio se puede realizar de 3 formas distintas. Fijado un grado de significatividad usualmente de $\alpha=0.5$ (para el $95\%$):
\begin{enumerate}
    \item Una vez calculados $\ol{d},s$ y el error estándar, calculamos:
        \begin{equation*}
            t_{exp} = \dfrac{\ol{d}}{\nicefrac{s}{\sqrt{n}}}
        \end{equation*}
        Tras lo cual podemos calcular el $p-$value:
        \begin{equation*}
            P(|t| \geq |t_{exp}|)
        \end{equation*}
        Donde consideramos la distribución de probabilidad de $t-$Student de $n-1$ grados de libertad. Si $P(|t|\geq |t_{exp}|) < \alpha$, entonces podremos rechazar $H_0$.
    \item Conocidos $\alpha$ y $n$, si calculamos $t_{\frac{\alpha}{2},n-1}$ que cumple:
        \begin{equation*}
            P(|t| \geq |t_{\frac{\alpha}{2},n-1}|) = \alpha
        \end{equation*}
        Para una distribución de $t-$Student de $n-1$ grados de libertad, si calculamos $t_{exp}$ y:
        \begin{equation*}
            t_{exp} \notin \left[-t_{\frac{\alpha}{2},n-1}, t_{\frac{\alpha}{2},n-1}\right]
        \end{equation*}
        Entonces podremos rechazar $H_0$.
    \item Si calculamos $t_{\frac{\alpha}{2},n-1}$, si:
        \begin{equation*}
            0\notin \left[\ol{d}-\frac{s}{\sqrt{n}} \cdot t_{\frac{\alpha}{2},n-1},~\ol{d}+\frac{s}{\sqrt{n}} \cdot t_{\frac{\alpha}{2},n-1}\right]
        \end{equation*}
        Entonces podremos rechazar $H_0$.
\end{enumerate}

\section{Optimización del rendimiento}
En un servidor con $K$ dispositivos o estaciones, para cada uno registraremos los siguientes parámetros (estarán marcados con subíndice $i$, ya que harán referencia al $i-$ésimo dispositivo, con $i \in \{1,\ldots,K\}$), donde $T$ es el tiempo total durante el que observamos el servidor:
\begin{itemize}
    \item $A_i$, número de trabajos solicitados a la estación.
    \item $C_i$, número de trabajos completados por la estación.
    \item $B_i$, tiempo que el dispositivo ha estado en uso.
    \item $S_i$, tiempo medio de servicio:
        \begin{equation*}
            S_i = \dfrac{B_i}{C_i}
        \end{equation*}
    \item $W_i$, tiempo medio de espera en cola.
    \item $R_i$, tiempo medio de respuesta del dispositivo:
        \begin{equation*}
            R_i = W_i + S_i
        \end{equation*}
    \item $\lm_i$, tasa media de llegada (cuántos trabajos por segundo llegan):
        \begin{equation*}
            \lm_i = \dfrac{A_i}{T}
        \end{equation*}
    \item $X_i$, productividad media (cuántos trabajos se completan por segundo):
        \begin{equation*}
            X_i = \dfrac{C_i}{T}
        \end{equation*}
    \item $U_i$, utilización media (porcentaje de tiempo durante el cual el dispositivo ha estado en uso):
        \begin{equation*}
            U_i = \dfrac{B_i}{T}
        \end{equation*}
    \item $Q_i$, número medio de trabajos en cola.
    \item $N_i$, número medio de trabajos siendo servidos por el dispositivo:
        \begin{equation*}
            N_i = Q_i + U_i
        \end{equation*}
\end{itemize}
Si vemos el servidor como un dispositivo, podemos generalizar estos parámetros para el mismo, los cuales denotaremos con subíndice $0$, y nos importarán especialmente:
\begin{equation*}
    A_0,\quad C_0,\quad R_0,\quad \lm_0 = \dfrac{A_0}{T},\quad X_0 = \dfrac{C_0}{T},\quad N_0 = \sum_{i} N_i
\end{equation*}
Estos parámetros nos permiten considerar un par más de parámetros de dispositivos:
\begin{itemize}
    \item $V_i$, la razón media de visita (número medio de veces que un trabajo visita al dispositivo $i-$ésimo durante su visita al servidor):
        \begin{equation*}
            V_i = \dfrac{C_i}{C_0}
        \end{equation*}
    \item $D_i$, la demanda media del servicio (cantidad media de tiempo que el dispositivo dedica a cada trabajo que abandona el servidor):
        \begin{equation*}
            D_i = \dfrac{B_i}{C_0} = V_i\cdot S_i
        \end{equation*}
\end{itemize}

\subsection{Leyes operacionales}
El valor de las variables anteriores depende obviamente del tiempo de observación $T$, pero existen algunas relaciones entre variables que no dependen de este tiempo $T$, lo que se denominan leyes operacionales. En lo que sigue, supondremos que $T$ es suficientemente grande (respecto a $R_0$), y diremos que el servidor está en equilibrio de flujo si el número de trabajos que completa el servidor coincide aproximadamente por el número de trabajos solicitados al mismo, es decir, si:
\begin{equation*}
    A_0 \approx C_0
\end{equation*}
O, equivalentemente, si $X_0\approx \lm_0$.

\begin{prop}[Ley de la utilización]
    Podemos relacionar la utilización media de un dispositivo con el número de trabajos que completa por unidad de tiempo (es decir, con su productividad) y con el tiempo medio que le dedica a cada uno de ellos:
    \begin{equation*}
        U_i = X_i \cdot S_i
    \end{equation*}
    Y si está en equilibrio de flujo: $U_i = \lm_i \cdot S_i$.
    \begin{proof}
        \begin{equation*}
            S_i = \dfrac{B_i}{C_i} = \dfrac{\dfrac{B_i}{T}}{\dfrac{C_i}{T}} = \dfrac{U_i}{X_i} \Longrightarrow U_i = X_i\cdot S_i
        \end{equation*}
    \end{proof}
\end{prop}

\begin{prop}[Ley del flujo forzado]
    Los flujos de salida de cada estación de servicio han de ser proporcionales a la productividad global del servidor:
    \begin{equation*}
        X_i = X_0\cdot V_i
    \end{equation*}
    En caso de equilibrio de flujo: $X_i = X_0 \cdot V_i = \lm_0\cdot V_i = \lm_i$.
    \begin{proof}
        \begin{equation*}
            V_i = \dfrac{C_i}{C_0} = \dfrac{\dfrac{C_i}{T}}{\dfrac{C_0}{T}} = \dfrac{X_i}{X_0} \Longrightarrow X_i = X_0\cdot V_i
        \end{equation*}
    \end{proof}
\end{prop}

\begin{prop}[Relación utilización-demanda de servicio]
    Las utilizaciones de cada servicio son proporcionales a las demandas de servicio del mismo, siendo la constante de proporcionalidad la productividad del servidor:
    \begin{equation*}
        U_i = X_0 \cdot D_i
    \end{equation*}
    En caso de equilibrio de flujo: $U_i = \lm_0 \cdot D_i$.
    \begin{proof}
        \begin{equation*}
            D_i = \dfrac{B_i}{C_0} = \dfrac{\dfrac{B_i}{T}}{\dfrac{C_0}{T}} = \dfrac{U_i}{X_0} \Longrightarrow U_i = X_0\cdot D_i
        \end{equation*}
    \end{proof}
\end{prop}

\begin{prop}[Ley de Little]
    Si el sistema está en equilibrio de flujo:
    \begin{equation*}
        N_i = \lm_i \cdot R_i = X_i\cdot R_i
    \end{equation*}
    También puede aplicarse a la cola de un dispositivo:
    \begin{equation*}
        Q_i = \lm_i \cdot W_i = X_i\cdot W_i
    \end{equation*}
\end{prop}

\begin{prop}[Ley general del tiempo de respuesta]
    El tiempo medio de respuesta que experimenta una petición en un servidor en equilibrio de flujo teniendo en cuenta que cada una de ellas ha visitado $V_i$ veces el dispositivo $i-$ésimo, donde ha permanecido $R_i$ segundos de media:
    \begin{equation*}
        R_0 = \sum_{i=0}^{K} V_i \cdot R_i
    \end{equation*}
    \begin{proof}
        Tenemos que:
        \begin{equation*}
            N_0 = \sum_{i=0}^{K}N_i
        \end{equation*}
        Y si aplicamos la Ley de Little:
        \begin{equation*}
            X_0 \cdot R_0 = \sum_{i=0}^{K}X_i\cdot R_i
        \end{equation*}
        Si usamos la Ley del flujo forzado:
        \begin{equation*}
            X_0 \cdot R_0 = \sum_{i=0}^{K}X_0\cdot V_i\cdot R_i
        \end{equation*}
        De donde:
        \begin{equation*}
            R_0 = \sum_{i=0}^{K}V_i\cdot R_i
        \end{equation*}
    \end{proof}
\end{prop}

\begin{prop}[Ley del tiempo de respuesta interactivo]
    En una red cerrada:
    \begin{equation*}
        R_0 = \dfrac{N_T}{X_0} - Z
    \end{equation*}
    \begin{proof}
        Sabemos que $N_T = N_Z + N_0$ y si aplicamos la Ley de Little a ambos:
        \begin{equation*}
            N_T = N_Z + N_0 = X_0 \cdot Z + X_0\cdot R_0 = X_0(Z+R_0) \Longrightarrow R_0 = \dfrac{N_T}{X_0}-Z
        \end{equation*}
    \end{proof}
\end{prop}

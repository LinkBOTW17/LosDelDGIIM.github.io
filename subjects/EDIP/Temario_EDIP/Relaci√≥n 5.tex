\section{Variables Aleatorias Unidimensionales}


\begin{ejercicio}
    Sea $X$ una variable aleatoria con función masa de probabilidad $$P(X = i) = ki; \qquad i = 1, \dots , 20.$$
    \begin{enumerate}
        \item Determinar el valor de $k$, la función de distribución y las siguientes probabilidades:
        \begin{equation*}
            P(X = 4),\quad P(X <4),\quad  P(3 \leq X \leq 10),\quad  P(3 < X \leq 10),\quad  P(3 < X < 10).
        \end{equation*}

        Estamos trabajando con una variable aleatoria discreta, con $|Re_X|=20$. Para que $P(X=i)$ sea una función masa de probabilidad, necesitamos que:
        \begin{equation*}
            1 = \sum_{i=1}^{20} P(X=i)
            = \sum_{i=1}^{20} ki
            = k\sum_{i=1}^{20} i
            = k\cdot \frac{20(1+20)}{2} = 210k\Longrightarrow k=\frac{1}{210}
        \end{equation*}

        Por tanto, tenemos que la función masa de probabilidad es:
        \begin{equation*}
            P(X=i)=\frac{1}{210}i \qquad i=1,\dots,20
        \end{equation*}

        Para calcular la función de distribución, sabemos que:
        \begin{equation*}
            F_X(x)=P(X\leq i)=\sum_{j=1}^i \frac{1}{210}j
            =\frac{1}{210} \sum_{j=1}^i j
            = \frac{1}{210} \frac{i(i+1)}{2} = \frac{i(i+1)}{420}
        \end{equation*}

        Por tanto, la función de distribución es:
        \begin{equation*}
            F_X(x)=\left\{\begin{array}{lll}
                0 & \text{si} & x < 1\\
                \displaystyle \frac{i(i+1)}{420} & \text{si} & x\in [i, i+1[ \qquad \forall i=1,\dots,19 \\
                1 & \text{si} & 20\leq x
            \end{array}\right.
        \end{equation*}

        Por tanto, las probabilidades pedidas son:
        \begin{gather*}
            P(X=4) = \frac{4}{210} = \frac{2}{105}\approx 0.01905 
            \\
            P(X<4) = P(X\leq 3) =F_X(3) = \frac{12}{420} = \frac{1}{35} \approx 0.02857
            \\
            P(3\leq X \leq 10) = P(X\leq 10) - P(X<3 ) = F_X(10) - F_X(2) = \frac{110-6}{420}\approx 0.2476
            \\
            P(3< X \leq 10) = P(X\leq 10) - P(X\leq3 ) = F_X(10) - F_X(3) = \frac{110-12}{420}\approx 0.2\bar{3}
            \\
            P(3< X < 10) = P(X< 10) - P(X\leq 3 ) = F_X(9) - F_X(3) = \frac{90-12}{420}\approx 0.1857
        \end{gather*}

        \item Supongamos que un jugador gana 20 monedas si al observar esta variable obtiene un valor menor que 4, gana 24 monedas si obtiene el valor 4 y, en caso contrario, pierde una moneda. Calcular la ganancia esperada del jugador y decir si el juego le es favorable.\\

        Definimos una nueva variable aleatoria, $Y$, que indica el número de monedas obtenidas por el jugador en función del valor de $X$.
        \begin{equation*}
            Y=h(X) = \left\{\begin{array}{llc}
                20 & \text{si} & 1\leq  x_i < 4\\
                24 & \text{si} & x_i = 4\\
                -1 & \text{si} & 4 < x_i \leq 20 
            \end{array}\right.
        \end{equation*}

        Tenemos que $Re_y = \{-1, 20, 24\}$. Calculamos la probabilidad de cada valor de la variable $Y$:
        \begin{multline*}
            P[Y=-1] = P[4 < X \leq 20] = P[X > 4] = 1- P[X\leq 4] =\\= 1-P[X=4] - P[x<4] = 1-\frac{1}{35} - \frac{2}{105} = \frac{20}{21}
        \end{multline*}
        \begin{equation*}
            P[Y=20] = P[1\leq X < 4] = P[x<4] = \frac{1}{35}
        \end{equation*}
        \begin{equation*}
            P[Y=24] = P[X=4] = \frac{2}{105}
        \end{equation*}

        Por tanto, la ganancia esperada del jugador es:
        \begin{equation*}
            E[Y]=\sum_{y_i\in Re_y} y_iP[y_i]
            = 20\cdot P[y=20] + 24\cdot P[Y=24] -1\cdot P[Y=-1] = \frac{8}{105}
        \end{equation*}

        Como tenemos que $E[Y]= \frac{8}{105} >0$, tenemos que el juego le es favorable, ya que se espera una ganancia de $\frac{8}{105}$ monedas.
    \end{enumerate}
\end{ejercicio}

\begin{ejercicio}
    Sea $X$ el número de bolas blancas obtenidas al sacar dos de una urna con 10 bolas de las que 8 son blancas. Calcular:
    \begin{enumerate}
        \item Función masa de probabilidad y función de distribución.

        El experimento aleatorio tiene el siguiente espacio muestral: $\Omega = \{BB, B-, --\}$, donde $B$ representa sacar una bola blanca y $-$ se refiere a sacar una bola de otro color.
        
        Tenemos que $Re_X = \{0,1,2\}$. Calculamos su función de probabilidad:
        \begin{equation*}
            P[X=0] = P[--] = \frac{C_{2,2}}{C_{10,2}} = \frac{1}{\frac{10!}{2!8!}} = \frac{2}{9\cdot 10} = \frac{1}{45} = 0.0\bar{2}
        \end{equation*}
        \begin{equation*}
            P[X=1] = P[B-] = \frac{C_{2,1}\cdot C_{8,1}}{C_{10,2}} = \frac{16}{45} = 0.3\bar{5}
        \end{equation*}
        \begin{equation*}
            P[X=2] = P[BB] = \frac{C_{8,2}}{C_{10,2}} = \frac{8! \cdot 2! \cdot 8!}{10!\cdot 2! \cdot 6!} = \frac{8\cdot 7}{10\cdot 9} = \frac{28}{45} = 0.6\bar{2}
        \end{equation*}

        Por tanto, la función masa de probabilidad es:
        \begin{equation*}
            f(x) = \left\{\begin{array}{ccc}
                0.0\bar{2} & \text{si} & x=0 \\
                0.3\bar{5} & \text{si} & x=1 \\
                0.6\bar{2} & \text{si} & x=2 \\
            \end{array}\right.    
        \end{equation*}

        Por tanto, la función de distribución es:
        \begin{equation*}
            F_X(x) = \left\{\begin{array}{ccc}
                0 & \text{si} & x<0 \\
                0.0\bar{2} & \text{si} & 0\leq x < 1  \\
                0.3\bar{7} & \text{si} & 1\leq x < 2 \\
                1 & \text{si} & 2\leq x \\
            \end{array}\right.
        \end{equation*}
        
        \item Media, mediana y moda, dando la interpretación de cada una de estas medidas.

        La media de una variable aleatoria es su esperanza. Por tanto,
        \begin{equation*}
            E[X] = \sum_{x_i=0}^2 x_if(x_i) = 0\cdot 0.0\bar{2} + 1\cdot 0.3\bar{5} + 2\cdot 0.6\bar{2} = 1.6
        \end{equation*}
        Que tenga una esperanza de $1.6$ implica que se espera que tras repetir el experimento un gran número de veces, la media de bolas blancas sacadas sea $1.6$. Este es el centro de gravedad de la distribución.

        Calculamos ahora la mediana.
        \begin{equation*}
            P[X\leq 2] = 1 \geq \frac{1}{2} \hspace{2cm}
            P[X\geq 2] = 0.6\bar{2} \geq \frac{1}{2}
        \end{equation*}
        Por tanto, tenemos que $Me_X = 2$. Este valor deja por encima y por debajo la misma probabilidad.

        La moda es la abcisa del máximo de la función masa de probabilidad, que como podemos ver es $Mo_X = 2$. Esto implica que es el valor con mayor probabilidad.
        
        \item Intervalo intercuartílico, especificando su interpretación.

        Calculamos en primer lugar $Q_1$:
        \begin{equation*}
            P[X\leq 1] = 0.3\bar{7} \geq \frac{1}{4} \hspace{2cm}
            P[X\geq 1] = 0.9\bar{7} \geq \frac{3}{4} = 1-\frac{1}{4}
        \end{equation*}
        Por tanto, tenemos que $Q_1=1$. Calculamos ahora $Q_3$:
        \begin{equation*}
            P[X\leq 2] = 1 \geq \frac{3}{4} \hspace{2cm}
            P[X\geq 2] = 0.6\bar{2} \geq \frac{1}{4} = 1-\frac{3}{4}
        \end{equation*}
        Por tanto, tenemos $Q_3 = 2$. De ahí concluimos que:
        \begin{equation*}
            R_I = Q_3 - Q_1 = 1
        \end{equation*}

        Por tanto, el 50\% central de la distribución se encuentra en un intervalo de amplitud $1$.
        
    \end{enumerate}
\end{ejercicio}

\begin{ejercicio}
    El número de lanzamientos de una moneda hasta salir cara es una variable aleatoria con distribución $P(X = x) = 2^{-x};\qquad x = 1, 2, \dots$

    \begin{enumerate}
        \item Probar que la función masa de probabilidad está bien definida.

        Tenemos que $Re_X = \bb{N}-\{0\}$. Es necesario que:
        \begin{equation*}
            \sum_{x=1}^\infty P(X = x) = \sum_{x=1}^\infty 2^{-x}
            = \sum_{x=1}^\infty \left(\frac{1}{2}\right)^x
        \end{equation*}

        Tenemos que se trata de una serie geométrica. Estas cumplen que, dado $r$ tal que $ |r|<1$:
        \begin{equation*}
            \sum_{x=1}^\infty r^x = \frac{r}{1-r}
        \end{equation*}

        Por tanto,
        \begin{equation*}
            \sum_{x=1}^\infty P(X = x)
            = \sum_{x=1}^\infty \left(\frac{1}{2}\right)^x
            = \frac{\frac{1}{2}}{1-\frac{1}{2}} = 1
        \end{equation*}

        Además, también se cumple que la probabilidad es siempre positiva, ya que $2^{-x}\geq 0 \;\forall x\in \bb{N}$. Por tanto, tenemos que está bien definida.

        \item Calcular la probabilidad de que el número de lanzamientos necesarios para salir cara esté entre 4 y 10.
        \begin{equation*}\begin{split}
            P(4\leq x \leq 10) &= P(X\in [4,10]) = P[X\leq 10] - P[X<3] = P[X\leq 10] - P[X\leq 3] =\\&= \sum_{x=1}^{10} 2^{-x} - \sum_{x=1}^3 2^{-x} = \sum_{x=4}^{10} 2^{-x}
            = \frac{127}{1024}\approx 0.124
        \end{split}\end{equation*}

        \item Calcular los cuartiles y la moda de la distribución, interpretando los valores.

        Calculamos en primer lugar $Q_1$:
        \begin{equation*}
            P[X\leq 1] = \frac{1}{2} \geq \frac{1}{4}
            \hspace{2cm}
            P[X\geq 1] = 1-P[X<1] = 1 \geq \frac{3}{4} = 1-\frac{1}{4}
        \end{equation*}

        Por tanto, $Q_1 = 1$. Calculamos ahora $Q_2$:
        \begin{equation*}
            P[X\leq 1] = \frac{1}{2} \geq \frac{1}{2}
            \hspace{2cm}
            P[X\geq 1] = 1-P[X<1] = 1 \geq \frac{1}{2} = 1-\frac{1}{2}
        \end{equation*}

        Por tanto, $Q_2 = 1 =Me_X$. Calculamos ahora $Q_3$:
        \begin{equation*}
            P[X\leq 2] = \frac{3}{4} \geq \frac{3}{4}
            \hspace{2cm}
            P[X\geq 2] = 1-P[X\leq 1] = \frac{1}{2} \geq \frac{1}{4} = 1-\frac{3}{4}
        \end{equation*}
        Por tanto, $Q_3 = 2$.

        Al ser una variable discreta, tenemos que tanto el primer cuarto como la mitad de la distribución se encuentra en el valor $x=1$. El 75\% de la distribución se encuentra hasta el 2.

        Como tenemos que $P(X=x) = 2^{-x}$ es estrictamente decreciente, tenemos que el máximo se alcanza en $x=1$. Por tanto, $Mo_X = 1$. Este es el valor con mayor probabilidad.

        \item Calcular la función generatriz de momentos y, a partir de ella, el número medio de lanzamientos necesarios para salir cara y la desviación típica.

        Tenemos que la función generatriz de momentos es:
        \begin{equation*}
            M_X (t) = E[e^{tX}] = \sum_{x=1}^\infty e^{tx} P(X=x)
            = \sum_{x=1}^\infty e^{tx} \left(\frac{1}{2}\right)^x
            = \sum_{x=1}^\infty \left(\frac{e^t}{2}\right)^x
        \end{equation*}

        Veamos para qué valores de $t$ converge esa serie geométrica:
        \begin{equation*}
            \left|\frac{e^t}{2}\right|<1 \Longleftrightarrow e^t <2 \Longleftrightarrow t<\ln 2
        \end{equation*}

        Por tanto, para $t<\ln 2$ tenemos que:
        \begin{equation*}
            M_X (t) = E[e^{tX}] = \sum_{x=1}^\infty \left(\frac{e^t}{2}\right)^x
            = \frac{\frac{e^t}{2}}{1-\frac{e^t}{2}}
             = \frac{e^t}{2-e^t}
        \end{equation*}

        Para calcular la esperanza, por las propiedades de la función generatriz de momentos, tenemos que:
        \begin{equation*}
            E[X] = m_1 = M_X'(0) = 2
        \end{equation*}
        donde he hecho uso de que:
        \begin{equation*}
            M_X'(t) = \frac{e^t(2-e^t) +e^{2t}}{(2-e^t)^2}
            = \frac{2e^t}{(2-e^t)^2}
        \end{equation*}

        Para calcular la desviación típica, calculo en primer la varianza. Por las propiedades de la función generatriz de momentos, tenemos que:
        \begin{equation*}
            \Var[X] = \mu_2 = E[X^2] - E[X]^2 = m_2 - E[X]^2 = M_X''(0) - E[X]^2 = 6-4 = 2
        \end{equation*}
        donde he hecho uso de que:
        \begin{equation*}
            M_X''(t) = \frac{2e^t(2-e^t)^2 +2e^t(2-e^t)\cdot 2e^t}{(2-e^t)^4}
            = \frac{2e^t(2-e^t) +4e^{2t}}{(2-e^t)^3}
            = \frac{4e^t +2e^{2t}}{(2-e^t)^3}
        \end{equation*}

        Por tanto,
        \begin{equation*}
            \sigma_X = +\sqrt{\Var[X]} = \sqrt{2}
        \end{equation*}
    \end{enumerate}
\end{ejercicio}


\begin{ejercicio}
    Sea $X$ una variable aleatoria con función de densidad
    \begin{equation*}
        f(x) = \left\{\begin{array}{lll}
            k_1(x+1) & \text{si} & 0\leq x \leq 4 \\
            k_2x^2 & \text{si} & 4<x\leq 6
        \end{array}\right.
    \end{equation*}

    Sabiendo que $P(0 \leq X \leq 4) = 2/3$, determinar $k_1, k_2$, y deducir su función de distribución.\\

    Tenemos que se trata de una variable aleatoria continua. Por tanto, tenemos que:
    \begin{equation*}
        P(0\leq X\leq 4) = P(X\in [0,4]) = \int_0^4 f(x)\;dx = \int_0^4 k_1 (x+1)\;dx =  k_1 \left[\frac{x^2}{2}+x\right]_0^4 = 12k_1
    \end{equation*}

    Como el enunciado afirma que $P(0\leq X\leq 4) = \frac{2}{3}$, tenemos que:
    \begin{equation*}
        P(0\leq X\leq 4) = \frac{2}{3} = 12k_1\Longrightarrow k_1 = \frac{1}{18}
    \end{equation*}

    Además, se necesita que $\int_{-\infty}^{+\infty}f(x)\;dx=1$, tenemos que:
    \begin{multline*}
        1=\int_{-\infty}^{+\infty}f(x)\;dx
        = \int_{-\infty}^0 0\;dx + \int_{0}^4 k_1(x+1)\;dx + \int_4^6 k_2x^2\;dx + \int_{6}^{+\infty}0\;dx
        \Longrightarrow \\ \Longrightarrow
        1 = 0+\frac{12}{18}+ k_2\left[\frac{x^3}{3}\right]_4^6 + 0 \Longrightarrow 1-\frac{2}{3} = k_2\cdot \frac{152}{3}
        \Longrightarrow \frac{1}{3} = \frac{152k_2}{3} \Longrightarrow k_2 = \frac{1}{152}
    \end{multline*} 

    Para calcular la función de distribución, tenemos que:
    \begin{equation*}
        F_X(x) = P[x\leq x] = \int_{-\infty}^x f(t)\;dt
    \end{equation*}

    \begin{itemize}
        \item \underline{Para $x<0$}:
        \begin{equation*}
            F_X(x) = \int_{-\infty}^x 0\;dt = 0
        \end{equation*}

        \item \underline{Para $0\leq x\leq 4$}:
        \begin{equation*}
            F_X(x) = \int_{-\infty}^x f(t)\;dt = 0 +\int_{0}^x \frac{x+1}{18}\;dt = \frac{1}{18}\left[\frac{x^2}{2}+x\right]_0^x
            = \frac{1}{18}\left(\frac{x^2}{2}+x\right)
        \end{equation*}

        \item \underline{Para $4\leq x\leq 6$}:
        \begin{multline*}
            F_X(x) = \int_{-\infty}^x f(t)\;dt = 0 +\int_{0}^4 \frac{x+1}{18}\;dt + 
            \int_{4}^x \frac{x^2}{152}\;dt
            = \frac{12}{18} + \frac{1}{152} \left[\frac{x^3}{3}\right]_4^x
            =\\= \frac{12}{18} + \frac{x^3}{3\cdot 152} - \frac{4^3}{152\cdot 3}
            = \frac{2}{3} + \frac{1}{456}\left(x^3-64\right)
        \end{multline*}
    \end{itemize}
    
    Por tanto, tenemos que la función de distribución es:
    \begin{equation*}
        F_X(x) = \left\{\begin{array}{lll}
            0 & \text{si} & x<0 \\
            \displaystyle  \frac{1}{18}\left(\frac{x^2}{2}+x\right) & \text{si} & 0\leq x \leq 4 \\
            \displaystyle \frac{2}{3} + \frac{1}{456}\left(x^3-64\right) & \text{si} & 4<x\leq 6 \\
            0 & \text{si} & 6<x
        \end{array}\right.
    \end{equation*}
\end{ejercicio}


\begin{ejercicio}
    La dimensión en centímetros de los tornillos que salen de cierta fábrica es una variable aleatoria, $X$, con función de densidad
    \begin{equation*}
        f(x)=\frac{k}{x^2},\qquad 1\leq x \leq 10.
    \end{equation*}


    \begin{enumerate}
        \item Determinar el valor de $k$, y obtener la función de distribución.

        Para que $f$ sea una función de densidad, es necesario que:
        \begin{equation*}
            1=\int_{-\infty}^{\infty}f(x)\;dx = k\left[-\frac{1}{x}\right]_1^{10} = \frac{9k}{10} \Longrightarrow k=\frac{10}{9}
        \end{equation*}

        Para $x\in [1,10]$, la función de distribución es:
        \begin{equation*}
            F_X(x) = \int_{-\infty}^x \frac{k}{t^2}\;dt
            = k\left[-\frac{1}{t}\right]_1^{x}
            = k\left(1-\frac{1}{x}\right)
        \end{equation*}

        Por tanto, la función de distribución queda:
        \begin{equation*}
            F_X(x) = \left\{\begin{array}{cc}
                0 & x<1 \\
                \frac{10}{9}\left(1-\frac{1}{x}\right) & 1 \leq x < 10 \\
                1 & 10\leq x
            \end{array}\right.
        \end{equation*}

        \item Hallar la probabilidad de que la dimensión de un tornillo esté entre 2 y 5 cm.
        \begin{equation*}
            P[1\leq X \leq 5] = \int_2^5 f(x)\;dx = k\left[-\frac{1}{x}\right]_2^5 = \frac{3k}{10} = \frac{1}{3}
        \end{equation*}

        \item Determinar la dimensión máxima del 50\% de los tornillos con menor dimensión y la dimensión mínima del 5\% con mayor dimensión.

        En primer lugar, nos piden la mediana. Al ser una variable aleatoria continua, tenemos que $Me = x\in Re_X \mid F_X(x) = \frac{1}{2}$.
        \begin{equation*}
            F_X(x) = \frac{1}{2} = \frac{10}{9}\left(1-\frac{1}{x}\right) \Longrightarrow \frac{9}{20} = 1-\frac{1}{x} \Longrightarrow \frac{1}{x} = \frac{11}{20} \Longrightarrow x = \frac{20}{11} = 1.\overline{81}
        \end{equation*}

        Por tanto, la dimensión máxima del 50\% de los tornillos con menor dimensión es $Me = 1.\overline{81}\;cm$.

        En segundo lugar, se pide el percentil $95$. Por tanto, esto equivale a $x$ tal que:
        \begin{equation*}
            0.95 = F_X(x) = \frac{10}{9}\left(1-\frac{1}{x}\right) \Longrightarrow 0.855 = 1-\frac{1}{x} \Longrightarrow \frac{1}{x} = 0.145 \Longrightarrow x = \frac{1}{0.145} = 6.897
        \end{equation*}
        Por tanto, la dimensión mínima del 5\% con mayor dimensión es $P_{95}=6.897\;cm$.

        \item Si $Y$ denota la dimensión de los tornillos producidos en otra fábrica, con la misma media y desviación típica que $X$, dar un intervalo en el que tome valores la variable $Y$ con una probabilidad mínima de $0.99$.

        Es necesario emplear la desigualdad de Chebyshev, que no se ha visto en clase.
    \end{enumerate}
\end{ejercicio}


\begin{ejercicio}
    Sea $X$ una variable aleatoria con función de densidad
    \begin{equation*}
        f(x) = \left\{\begin{array}{lll}
            \displaystyle \frac{2x-1}{10} & \text{si} & 1 < x \leq 2 \\
            \\
            0.4 & \text{si} & 4<x\leq 6
        \end{array}\right.
    \end{equation*}

    \begin{enumerate}
        \item Calcular $P(1.5 < X \leq 2)$, $P(2.5 < X \leq 3.5)$, $P(4.5 \leq X <5.5)$, $P(1.2 < X \leq 5.2)$.

        Tenemos que $Re_X = ]1,2]\;\cup\; ]4,6]$. Por tanto,
        \begin{equation*}
            P(1.5 < X \leq 2) = \int_{1.5}^2 \frac{2x-1}{10}\;dx = \left[\frac{x^2-x}{10}\right]_{1.5}^2 = \frac{1}{5} - \frac{3}{40} = \frac{1}{8} = 0.125
        \end{equation*}
        \begin{equation*}
            P(2.5 < X \leq 3.5) = 0,\qquad \text{ ya que $]2.5,\;3.5]\notin Re_X$}
        \end{equation*}
        \begin{equation*}
            P(4.5 \leq X < 5.5) = \int_{4.5}^{5.5} 0.4\;dx = 0.4\left[x\right]_{4.5}^{5.5} = 0.4
        \end{equation*}
        \begin{multline*}
            P(1.2 \leq X < 5.2) = \int_{1.2}^2 \frac{2x-1}{10}\;dx +  \int_{4}^{5.2} 0.4\;dx
            = \left[\frac{x^2-x}{10}\right]_{1.2}^2
            +0.4\left[x\right]_{4}^{5.2} =\\= \frac{22}{125} + 0.4(1.2) = \frac{82}{125} = 0.656
        \end{multline*}

        \item Dar la expresión general de los momentos no centrados y deducir el valor medio de $X$.

        Se definen los momentos no centrados como:
        \begin{multline*}
            m_k = E[X^k] = \int_{-\infty}^{\infty}x^kf(x)\;dx = \int_{1}^2x^k\cdot \frac{2x-1}{10}\;dx + \int_4^6 0.4x^k\;dx =\\
            = \int_{1}^2 \frac{2x^{k+1}-x^k}{10}\;dx + 0.4\int_4^6 x^k\;dx
            = \frac{1}{10}\left[\frac{2x^{k+2}}{k+2} - \frac{x^{k+1}}{k+1}\right]_{1}^2
            +0.4\left[\frac{x^{k+1}}{k+1}\right]_{4}^{6} =\\
            = \frac{2^{k+3}-2}{10(k+2)} + \frac{-2^{k+1} + 1 +4\cdot 6^{k+1}-4^{k+2}}{10(k+1)}
        \end{multline*}

        En concreto,
        \begin{equation*}
            E[X] = m_1 = \frac{2^{4}-2}{10(3)} + \frac{-2^{2} + 1 +4\cdot 6^{2}-4^{3}}{10(2)} = \frac{259}{60} = 4.31\bar{6}
        \end{equation*}

        \item Calcular la función generatriz de momentos de $X$.
        \begin{equation*}\begin{split}
            M_X(t) &= E[e^{tX}] = \int_{-\infty}^{\infty}e^{tx}f(x)\;dx = \int_{1}^2e^{tx}\cdot \frac{2x-1}{10}\;dx + \int_4^6 0.4e^{tx}\;dx =\\
            &= \frac{1}{10}\int_{1}^2 2xe^{tx}\;dx -\frac{1}{10}\int_1^2 e^{tx}\;dx + 0.4\int_4^6 e^{tx}\;dx
            = \MetInt{u(x)=2x\quad u'(x)=2}{v'(x)=e^{tx}\quad v(x)=\frac{e^{tx}}{t}} =\\
            &= \frac{1}{10}\left[\frac{2xe^{tx}}{t}\right]_1^2 - \frac{2}{10t}\int_1^2 e^{tx}\;dx -\frac{1}{10}\int_1^2 e^{tx}\;dx +0.4\int_4^6 e^{tx}\;dx  =\\
            &= \frac{2}{10t}\left[xe^{tx}\right]_1^2 -\frac{2}{10t^2}\left[e^{tx}\right]_1^2 -\frac{1}{10t}\left[e^{tx}\right]_1^2 +\frac{0.4}{t}\left[e^{tx}\right]_4^6 =\\
            &= \frac{2}{10t}\left(2e^{2t} -e^t\right) -\frac{2}{10t^2}\left(e^{2t} -e^t\right) -\frac{1}{10t}\left(e^{2t} -e^t\right) +\frac{0.4}{t}\left(e^{6t} -e^{4t}\right) =\\
            &= \frac{1}{10t}\left[4e^{2t} -2e^t -\frac{2e^{2t}-2e^t}{t} -e^{2t} +e^t +4e^{6t} -4e^{4t}\right] =\\
            &= \frac{1}{10t}\left[3e^{2t} -e^t -\frac{2e^{2t}-2e^t}{t} +4e^{6t} -4e^{4t}\right]
        \end{split}\end{equation*}

        Por tanto, como para $t=0$ la función $M_X(t)$ no está definida, tenemos que no existe la función generatriz de momentos.
    \end{enumerate}
\end{ejercicio}


\begin{ejercicio}
    Con objeto de establecer un plan de producción, una empresa ha estimado que la demanda de sus clientes, en miles de unidades del producto, se comporta semanalmente con arreglo a una ley de probabilidad dada por la función de densidad:
    \begin{equation*}
        f(x)=\frac{3}{4}(2x-x^2),\qquad 0\leq x \leq 2.
    \end{equation*}


    \begin{enumerate}
        \item ¿Qué cantidad debería tener dispuesta a la venta al comienzo de cada semana para poder satisfacer plenamente la demanda con probabilidad $0.5$?

        Sea $X$ una variable aleatoria que determina la demanda en miles de unidades de producto. Por tanto, se pide $\hat{x}\in [0,2]$ tal que:
        \begin{equation*}
             0.5 = F_X(\hat{x}) = P[X\leq \hat{x}] = \int_0^{\hat{x}} f(x)\;dx = \frac{3}{4}\left[x^2-\frac{x^3}{3}\right]_0^{\hat{x}}
             = \frac{3}{4}\left[\hat{x}^2-\frac{\hat{x}^3}{3}\right]
        \end{equation*}

        Por tanto, se busca resolver la siguiente ecuación:
        \begin{equation*}
            3\hat{x}^2 - \hat{x}^3-2=0
        \end{equation*}

        La única solución de dicha ecuación en el intervalo $[0,2]$ es $\hat{x}=1$. Por tanto, han de tener dispuestas mil unidades del producto a la venta al comienzo de cada semana para poder satisfacer plenamente la demanda con probabilidad $0.5$.

        \item Pasado cierto tiempo, se observa que la demanda ha crecido, estimándose que en ese momento se distribuye según la función de densidad:
        \begin{equation*}
            f(y)=\frac{3}{4} (4y-y^2-3),\qquad 1\leq y \leq 3.
        \end{equation*}

        Los empresarios sospechan que este crecimiento no ha afectado a la dispersión de la demanda, ¿es cierta esta sospecha?\\

        En este caso, se ha denominado $Y=X$, con la diferencia en la función de densidad.

        El coeficiente de variación de Pearson se define como:
        \begin{equation*}
            C.V.(Z) = \frac{\sigma_Z}{E[Z]}
        \end{equation*}

        Calculamos en primer lugar los siguientes valores:
        \begin{equation*}
            E[X] = \int_0^2 xf(x)\;dx = \frac{3}{4}\int_0^2 2x^2-x^3\;dx = \frac{3}{4}\left[\frac{2x^3}{3}-\frac{x^4}{4}\right]_0^2 = 1
        \end{equation*}
        \begin{equation*}
            E[X^2] = \int_0^2 x^2f(x)\;dx = \frac{3}{4}\int_0^2 2x^3-x^4\;dx = \frac{3}{4}\left[\frac{2x^4}{4}-\frac{x^5}{5}\right]_0^2 = \frac{6}{5} = 1.2
        \end{equation*}
        \begin{equation*}
            E[Y] = \int_1^3 yf(y)\;dy = \frac{3}{4}\int_1^3 4y^2-y^3-3y\;dy = \frac{3}{4}\left[\frac{4y^3}{3} - \frac{y^4}{4} - \frac{3y^2}{2}\right]_1^3 = \frac{3}{4}\cdot \frac{8}{3} = 2
        \end{equation*}
        \begin{equation*}
            E[Y^2] = \int_1^3 y^2f(y)\;dy = \frac{3}{4}\int_1^3 4y^3-y^4-3y^2\;dy = \frac{3}{4}\left[y^4 - \frac{y^5}{5} - y^3\right]_1^3 = \frac{3}{4}\cdot \frac{28}{5} = \frac{21}{5} = 4.2
        \end{equation*}

        Por tanto, tenemos que:
        \begin{equation*}
            \Var[X] = E[X^2] -E[X]^2 = 0.2
            \qquad
            \Var[Y] = E[Y^2] -E[Y]^2 = 0.2
        \end{equation*}

        Por tanto, la varianza no varía y la desviación típica tampoco. No obstante, como $E[X]\neq E[Y]$, la dispersión varía:
        \begin{equation*}
            C.V.[X] = \frac{\sqrt{0.2}}{1} \neq \frac{\sqrt{0.2}}{2} = C.V.[Y]
        \end{equation*}

        Como podemos ver, la distribución $Y$ es más homogénea.
    \end{enumerate}
\end{ejercicio}


\begin{ejercicio}
    Calcular las funciones masa de probabilidad de las variables $Y = X + 2$ y $Z = X^2$, siendo $X$ una variable aleatoria con distribución:
    \begin{equation*}
        P(X=-2)=\frac{1}{5},\quad P(X=-1)=\frac{1}{10},\quad P(X=0)=\frac{1}{5}, \quad P(X=1)=\frac{2}{5}, \quad P(X=2)=\frac{1}{10}.
    \end{equation*}
    ¿Cómo afecta el cambio de $X$ a $Y$ en el coeficiente de variación?\\
    Tenemos que $Re_X=\{-2,-1,0,1,2\}$. Calculo en primer lugar la función masa de probabilidad de $Y=X+2$, teniendo que $Re_Y=\{0,1,2,3,4\}$. Tenemos que:
    \begin{equation*}
        P[Y=0]=P[X=-2]=\frac{1}{5}
        \qquad
        P[Y=1]=P[X=-1]=\frac{1}{10}
        \qquad
        P[Y=2]=P[X=0]=\frac{1}{5}
    \end{equation*}
    \begin{equation*}
        P[Y=3]=P[X=1]=\frac{2}{5}
        \qquad
        P[Y=4]=P[X=2]=\frac{1}{10}
    \end{equation*}

    Veamos si ha afectado la transformación al coeficiente de variación. Tenemos que $\displaystyle C.V.(X)=\frac{\sigma_x}{|E[X]|}$. Por ser una transformación afín, tenemos que:
    \begin{equation*}
       E[Y]=E[X+2]=E[X]+E[2]=E[X]+2
    \end{equation*}
    \begin{equation*}
        \Var[Y]=\Var[2X+3]=4\Var[X] \Longrightarrow \sigma_y = 2\sigma_x
    \end{equation*}
    Por tanto, tenemos que:
    \begin{multline*}
        C.V.(Y)=\frac{2\sigma_x}{|E[X]+2|} =\frac{\sigma_x}{|E[X]|}=C.V.(X)
        \Longleftrightarrow
        \sigma_x|E[X]+2| = 2\sigma_x|E[X]| \Longleftrightarrow \\ \Longleftrightarrow
        \left\{\begin{array}{c}
            \sigma_x=0 \\
             \lor \\
             \pm 2E[X]=E[X]+2 \Longleftrightarrow E[X]=2 \quad \lor \quad E[X]=-\frac{2}{3}
        \end{array}\right.
    \end{multline*}
    Por tanto, tenemos que tan solo serán iguales si $\sigma_x=0, E[X]=2$ o $E[X]=-\frac{2}{3}$.

    \vspace{1cm} Calculamos ahora la función masa de probabilidad la variable $Z$, donde $Re_Z=~\{0, 1, 4\}$.
    \begin{equation*}
        P[Z=0]=P[X=0]=\frac{1}{5}
    \end{equation*}
    \begin{equation*}
        P[Z=1]=P[X=1]+P[X=-1]=\frac{1}{2}
    \end{equation*}
    \begin{equation*}
        P[Z=4]=P[X=2]+P[X=-2]=\frac{3}{10}
    \end{equation*}
\end{ejercicio}

\begin{ejercicio}
    Calcular las funciones de densidad de las variables $Y = 2X + 3$ y $Z = |X|$, siendo $X$ una variable continua con función de densidad
    \begin{equation*}
        f_X(x)=\frac{1}{4},\qquad -2<x<2
    \end{equation*}

    \begin{enumerate}
        \item $Y=2X+3$

        Tenemos que $Re_X=]-2,2[$, $Re_Y=]-7,1[$, y sea $g(x)=2x+3$. Por el Teorema de Cambio de Variable de continua a continua, tenemos que:
        \begin{equation*}
            f_Y(y)=f_X(g^{-1}(y))\cdot |(g^{-1})'(y)|
        \end{equation*}

        Calculamos la inversa y su derivada:
        \begin{equation*}
            g^{-1}(y)=\frac{y-3}{2}
            \hspace{2cm}
            (g^{-1})'(y) = \frac{1}{2} 
        \end{equation*}

        Por tanto, tenemos que:
        \begin{equation*}
            f_Y = \frac{1}{4}\cdot \frac{1}{2} = \frac{1}{8}
        \end{equation*}

        \item $Z=|X|$

        Tenemos que $Re_{Z}=[0,2[$:
        \begin{equation*}
            h(x)=|x|=\left\{\begin{array}{cc}
                -x & x<0 \\
                 x & x\geq 0
            \end{array}\right.
        \end{equation*}

        En este caso, tenemos que $h$ no es inyectiva, y por tanto, hay más de una antiimagen por cada valor de $z$. En concreto, hay dos valores, el positivo y el negativo. Sean por tanto las dos inversas $h_1,h_2$.
        \begin{equation*}
            h_1^{-1}(z)=z
            \hspace{2cm}
            (h_1^{-1})'(z) = 1
        \end{equation*}
        \begin{equation*}
            h_2^{-1}(z)=-z
            \hspace{2cm}
            (h_2^{-1})'(z) = -1
        \end{equation*}

        Por tanto, por el teorema de cambio de variable, tenemos que:
        \begin{equation*}
            f_Z(z)=\sum_{k=1}^2 f_X(h_k^{-1}(z))\cdot |(h_k^{-1})'(z)| = \sum_{k=1}^2 \frac{1}{4}\cdot 1 = \frac{1}{4} + \frac{1}{4} = \frac{1}{2}
        \end{equation*}
    \end{enumerate}
\end{ejercicio}


\begin{ejercicio}
    Si $X$ es una variable aleatoria con función de densidad
    \begin{equation*}
        f(x)=\frac{e^{-|x|}}{2},\qquad -\infty<x<\infty,
    \end{equation*}

    hallar su función de distribución y las probabilidad de cada uno de los siguientes sucesos:
    \begin{enumerate}
        \item $\{|X|\leq 2\}$.

        En primer lugar, hallo la función de distribución:
        \begin{equation*}
            F_X(x)=\int_{-\infty}^x f(t)\;dt
            = \frac{1}{2}\int_{-\infty}^x e^t = \frac{1}{2}[e^t]_{-\infty}^x = \frac{e^x}{2} \hspace{2cm} x<0
        \end{equation*}
        \begin{multline*}
            F_X(x)=\frac{1}{2}\int_{-\infty}^0 e^x\;dx + \frac{1}{2}\int_{0}^x e^{-t}\;dt
            = \frac{1}{2}\left[[e^x]_{-\infty}^0 -\left[e^{-t}\right]_0^x\right]
            =\\= \frac{1-e^{-x}+1}{2}
            = \frac{2-e^{-x}}{2}
            \hspace{2cm} x\geq 0
        \end{multline*}

        Por tanto, tenemos que:
        \begin{equation*}
            F_X(x)=\left\{\begin{array}{cc}
                \displaystyle \frac{e^x}{2} & x<0 \\ \\
                \displaystyle \frac{2-e^{-x}}{2} & x\geq 0
            \end{array}\right.
        \end{equation*}

        Por tanto, tenemos que:
        \begin{multline*}
            P[-2\leq X\leq 2] = P[X\leq 2] - P[X<-2]
            = P[X\leq 2] - P[X\leq -2]
            =\\= F_X(2)-F_X(-2)
            = \frac{2-e^{-2}}{2} - \frac{e^{-2}}{2}
            = \frac{2-2e^{-2}}{2}
            = 1-e^{-2}\approx 0.8647
        \end{multline*}

        
        \item $\{|X|\leq 2 \quad \text{o} \quad X\geq 0\}$.
        \begin{equation*}
            P[|X|\leq 2\quad o \quad X\geq 0] = P[X\geq -2] = 1-P[X<-2] = 1-F_X(-2)=1-\frac{e^{-2}}{2} \approx 0.9323
        \end{equation*}
        
        \item $\{|X|\leq 2 \quad \text{y} \quad X\leq -1\}$.
        \begin{equation*}
            P[|X|\leq 2\quad y \quad X\leq -1] = P[-2 \leq X\leq -1] = F_X(-1) - F(-2)=\frac{e^{-1}-e^{-2}}{2} \approx 0.1163
        \end{equation*}
        
        \item $\{X^3 - X^2 - X-2 \leq 0\}$.

        Factorizamos en primer lugar el polinomio:
        \begin{equation*}
            \polyhornerscheme[x=2]{x^3-x^2-x-2}
        \end{equation*}
        Por tanto, tenemos que $X^3-X^2-X-2 = (X-2)(X^2+X+1)$.
        
        Del polinomio restante, tenemos que $\Delta=1-4<0$. Por tanto, no tiene soluciones. Como al evaluarlo en 0 da $1>0$, tenemos que es siempre positivo. Por tanto,
        \begin{multline*}
            P[X^3 - X^2 - X-2 \leq 0] = P[(X-2)(X^2+X+1)<0]
            = P[X-2\leq 0]
            =\\= P[X\leq 2] = F_X(2) = \frac{2-e^{-2}}{2}
        \end{multline*}
        
        
        \item $\{X \text{ es irracional}\}$.

        \begin{equation*}
            P[X\in \mathbb{R}-\mathbb{Q}] = P[X\in \mathbb\{R\}] - P[X\in \mathbb\{Q\}] = 1-0 = 0
        \end{equation*}

        La probabilidad de que esté en los reales es 1, ya que abarca todo el intervalo de definición de la variable aleatoria. En el caso de los racionales, al ser este un conjunto numerable, tenemos que su integral es nula.
    \end{enumerate}
\end{ejercicio}



\begin{ejercicio}
    Sea $X$ una variable aleatoria con función de densidad
    \begin{equation*}
        f(x) = 1,\qquad  0 \leq x \leq 1.
    \end{equation*}

    Encontrar la distribución de las variables:
    \begin{enumerate}
        \item $\displaystyle Y=\frac{X}{1+X}$.

        Sabemos que $Re_X=[0,1]$. Además, tenemos que $h(X)=\frac{X}{1+X}=Y$. tenemos que:
        \begin{equation*}
            h'(X) = \frac{(1+X)-X}{(1+X)^2} = \frac{1}{(1+X)^2} >0
        \end{equation*}

        Por tanto, $h$ es estrictamente creciente, y tenemos que $Re_Y=\left[0,\frac{1}{2}\right]$. Como $h$ es estrictamente monótona y derivable, podemos aplicar el teorema de cambio de variable de continua a continua. Este afirma que:
        \begin{equation*}
            g(y)=\left\{\begin{array}{cc}
                f(h^{-1}(y))|(h^{-1})'(y)| & y\in Re_Y \\
                0 & y\notin Re_Y
            \end{array}\right.
        \end{equation*}

        Tenemos que:
        \begin{equation*}
            h(X)=\frac{X}{1+X} \Longrightarrow h^{-1}(Y)=\frac{Y}{1-Y}
            \Longrightarrow (h^{-1})'(Y) = \frac{1-Y +Y}{(1-Y)^2} = \frac{1}{(1-Y)^2}
        \end{equation*}

        Por tanto, como $f(h^{-1}(y))=1$, tenemos que:
        \begin{equation*}
            g(y)=\left\{\begin{array}{cc}
                \frac{1}{(1-y)^2} & y\in Re_Y \\
                0 & y\notin Re_Y
            \end{array}\right.
        \end{equation*}

        Para obtener la distribución, resuelvo la integral siguiente:
        \begin{equation*}
            \int_0^y \frac{1}{(1-t)^2}\;dt = \left[\frac{1}{1-t}\right]_0^y = \frac{1}{1-y}-1
        \end{equation*}


        Por tanto,
        \begin{equation*}
            F_Y(y)=\left\{\begin{array}{cc}
                0 & y<0 \\
                \frac{1}{1-y}-1 & y\in Re_Y \\
                1 & y\geq \frac{1}{2}
            \end{array}\right.
        \end{equation*}

        \item $\displaystyle Z=\left\{\begin{array}{ccc}
            -1 & \text{si} & X<3/4 \\
            0 & \text{si} & X=3/4 \\
            1 & \text{si} & X>3/4 \\
        \end{array}\right.$

        En este caso, estamos ante un cambio de variable continua a discreta. Se tiene que $Re_Z=\{-1,0,1\}$, y la función masa de probabilidad de $Z$ es:
        \begin{equation*}
            g(-1) = P[Z=-1]=P[X<3/4] = \int_0^{3/4}f(x)\;dx = \left[x\right]_0^{3/4} = \frac{3}{4}
        \end{equation*}
        \begin{equation*}
            g(0) = P[Z=0]=P[X=3/4] = \int_{3/4}^{3/4}f(x)\;dx = 0
        \end{equation*}
        \begin{equation*}
            g(1) = P[Z=1]=P[X>3/4] = 1-P[X\leq 3/4] = \frac{1}{4}
        \end{equation*}

        Por tanto, se tiene que la función masa de probabilidad de $Z$ es:
        \begin{equation*}
            g(z)=\left\{\begin{array}{cl}
                \displaystyle  \frac{3}{4} & z=-1 \\ \\
                0 & z=0\\ \\
                \displaystyle  \frac{1}{4} & z=1 \\
            \end{array}\right.
        \end{equation*}

        Su función de dstribución es:
        \begin{equation*}
            F_Z(z)=\left\{\begin{array}{cc}
                0 & z<-1 \\ \\
                \displaystyle \frac{3}{4} & -1\leq z<1 \\ \\
                1 & 1 \leq z\\
            \end{array}\right.
        \end{equation*}
    \end{enumerate}
\end{ejercicio}


\begin{ejercicio}
    Sea $X$ una variable aleatoria simétrica con respecto al punto 2, y con coeficiente de variación 1. ¿Qué puede decirse acerca de las siguientes probabilidades?:
    \begin{enumerate}
        \item $P(-8 < X < 12)$
        \item $P(-6 < X < 10)$.
    \end{enumerate}

    Es necesario emplear la desigualdad de Chebyshev, que no se ha visto en clase.
\end{ejercicio}


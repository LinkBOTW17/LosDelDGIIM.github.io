\section{Probabilidad Condicionada e
Independencia de Sucesos}



\begin{ejercicio} \label{ej:4.Ejercicio1}
    En una batalla naval, tres destructores localizan y disparan simultáneamente a un submarino. La probabilidad de que el primer destructor acierte el disparo es $0.6$, la de que lo acierte el segundo es $0.3$ y la de que lo acierte el tercero es $0.1$. ¿Cuál es la probabilidad de que el submarino sea alcanzado por algún disparo?\\

    Sea $A$ el suceso de que el primer destructor acierte, $P(A)=0.6$.
    
    Sea $B$ el suceso de que el segundo destructor acierte, $P(B)=0.3$.
    
    Sea $C$ el suceso de que el tercer destructor acierte, $P(C)=0.1$.\\

    Tenemos, además, que $A,B,C$ son independientes; ya que el hecho de que uno acierte no influye en que lo hagan los otros o no.

    Por tanto, tenemos que la probabilidad de que el submarino sea alcanzado por algún disparo es:
    \begin{multline*}
        P(A\cup B \cup C)
        = P(A)+P(B\cup C) -P(A\cap (B\cup C)) =\\= P(A) +P(B) + P(C) -P(B\cap C) -P[(A\cap B) \cup (A\cap C)]
        =\\=
        P(A) +P(B) + P(C) -P(B\cap C) -P(A\cap B) -P(A\cap C) +P(A\cap B \cap C)
    \end{multline*}

    Como tenemos que son sucesos independientes:
    \begin{equation*}
        P(A\cup B \cup C) = P(A) +P(B)+P(C)-P(B)P(C)-P(A)P(C)+P(A)P(B)P(C) = 0.748
    \end{equation*}

    Por tanto, tenemos que la probabilidad de que el submarino haya sido alcanzado es de $0.748$.
\end{ejercicio}

\begin{ejercicio} \label{ej:4.Ejercicio2}
    Un estudiante debe pasar durante el curso 5 pruebas selectivas. La probabilidad de pasar la primera es $1/6$. La probabilidad de pasar la $i-$ésima, habiendo pasado las anteriores es $1/(7-i)$. Determinar la probabilidad de que el alumno apruebe el curso.

    Sea $A_i$ el suceso de aprobar la $i$-ésima prueba. El enunciado que la probabilidad de aprobar la prueba $n$-ésima condicionada a haber aprobado las $n-1$ anteriores es de $\frac{1}{7-n}$:
    \begin{equation*}
        P\left[A_n\left|\bigcap_{i=1}^{n-1} A_i\right.\right] = \frac{1}{7-n} \qquad n=1,\dots,5
    \end{equation*}

    Por tanto, por el Teorema de la Probabilidad Compuesta,
    \begin{multline*}
        P\left[\bigcap_{i=1}^5 A_i\right]
        = P(A_1)\cdot P(A_2|A_1) \cdot P(A_3|(A_1\cap A_2))\cdot P\left(A_4 | \cap_{j=1}^3 A_j\right)
        \cdot P\left(A_5 | \cap_{j=1}^4 A_j\right)
        =\\=
        \prod_{i=1}^5 \frac{1}{7-i}
        = \frac{1}{6} \cdot \frac{1}{5} \cdot \frac{1}{4} \cdot \frac{1}{3} \cdot\frac{1}{2} = \frac{1}{6!} = \frac{1}{720} = 0.0013\bar{8}
    \end{multline*}

    Por tanto, tenemos que la probabilidad de aprobar el curso de dicho estudiante es de $0.0013\bar{8}$.

\end{ejercicio}

\begin{ejercicio} \label{ej:4.Ejercicio3}
En una ciudad, el 40\% de las personas tienen pelo rubio, el 25\% tienen ojos azules y el 5\% el pelo rubio y los ojos azules. Se selecciona una persona al azar. Calcular la probabilidad de los siguientes sucesos:
\begin{enumerate}
    \item tener el pelo rubio si se tiene los ojos azules,

    Sea $A$ tener los ojos azules y $R$ tener el pelo rubio. Tenemos que:
    \begin{equation*}
        P(R)=0.4 \hspace{2cm} P(A)=0.25 \hspace{2cm} P(A\cap R)=0.05
    \end{equation*}

    Por definición de probabilibad condicionada:
    \begin{equation*}
        P(R|A) = \frac{P(A\cap R)}{P(A)} = \frac{1}{5} = 0.2
    \end{equation*}

    
    \item tener los ojos azules si se tiene el pelo rubio,
    
    Por definición de probabilibad condicionada:
    \begin{equation*}
        P(A|R) = \frac{P(A\cap R)}{P(R)} = \frac{1}{8} = 0.125
    \end{equation*}

    \item no tener pelo rubio ni ojos azules,
    \begin{equation*}
        P(\bar{R}\cap \bar{A})
        = P(\overline{A\cup R})
        = 1-P(A\cup R)
        = 1-P(A)-P(R) + P(A\cap R) = 0.4
    \end{equation*}
    
    \item tener exactamente una de estas características.
    \begin{multline*}
        P[(A\cap \bar{R}) \cup (\bar{A} \cap R)]
        = P(A\cap \bar{R}) + P(\bar{A} \cap R) -P[\cancelto{0}{(A\cap \bar{R}) \cap (\bar{A} \cap R)]}
        =\\= P(A-R) +P(R-A) = P(A) + P(R) -2\cdot P(A\cap R) = 0.55
    \end{multline*}
\end{enumerate}
    
\end{ejercicio}

\begin{ejercicio} \label{ej:4.Ejercicio4}
    En una población de moscas, el 25\% presentan mutación en los ojos, el 50\% presentan mutación en las alas, y el 40\% de las que presentan mutación en los ojos presentan mutación en las alas.
    \begin{enumerate}
        \item ¿Cuál es la probabilidad de que una mosca elegida al azar presente al menos una de las mutaciones?

        Sea $O$ tener una mutación en los ojos y $A$ tener una mutación en las alas. Tenemos que:
        \begin{equation*}
            P(O)=0.25
            \hspace{2cm}
            P(A)=0.5
            \hspace{2cm}
            P(A|O)=0.4
        \end{equation*}

        Por la definición de probabilidad condicionada, tenemos que:
        \begin{equation*}
            P(A|O) = \frac{P(A\cap O)}{P(O)} \Longrightarrow P(A\cap O) = P(A|0)\cdot P(O) = 0.1
        \end{equation*}

        Por tanto, tenemos que la probabilidad de tener al menos una de las mutaciones es:
        \begin{equation*}
            P(A\cup O) = P(A) + P(O) - P(A\cap O) = 0.65
        \end{equation*}
        

        \item ¿Cuál es la probabilidad de que presente mutación en los ojos pero no en las alas?
        \begin{equation*}
            P(O\cap \bar{A}) = P(O-A) = P(O) - P(A\cap O) = 0.15
        \end{equation*}
    \end{enumerate}
\end{ejercicio}

\begin{ejercicio} \label{ej:4.Ejercicio5}
    Una empresa utiliza dos sistemas alternativos, $A$ y $B$, en la fabricación de un artículo, fabricando por el sistema $A$ el 20\% de su producción. Cuando a un cliente se le ofrece dicho artículo, la probabilidad de que lo compre es $2/3$ si éste se fabricó por el sistema $A$ y $2/5$ si se fabricó por el sistema $B$. Calcular la probabilidad de vender el producto.

    Sea $A$ el suceso de que un producto sea fabricado usando el sistema $A$, y $B$ en el caso contrario. Se considera también el suceso $C$, que es que el cliente compra el producto. Entonces, por las condiciones del enunciado tenemos que:
    \begin{gather*}
        P(A)=0.2 = \frac{1}{5} \hspace{1cm} P(B)=0.8 = \frac{4}{5} \\
        P(C|A) = \frac{2}{3} \hspace{1cm} P(C|B) = \frac{2}{5}
    \end{gather*}

    Por el teorema de la probabilidad total, tenemos que:
    \begin{equation*}
        P(C) = P(A)P(C|A) + P(B)P(C|B) = \frac{2}{15} + \frac{8}{25} = \frac{34}{75} \approx 0.4533
    \end{equation*}
\end{ejercicio}

\begin{ejercicio} \label{ej:4.Ejercicio6}
    Se consideran dos urnas: la primera con 20 bolas, de las cuales 18 son blancas, y la segunda con 10 bolas, de las cuales 9 son blancas. Se extrae una bola de la segunda urna y se deposita en la primera; si a continuación, se extrae una bola de ésta, calcular la probabilidad de que sea blanca.\\

    Sean las bolas blancas de la primera urna notadas como $1_B$, y $1_O$ las que son de otro color. Respecto de la segunda urna, sean estas bolas $2_B,2_O$ respectivamente.

    Usando la regla de la Probabilidad total, tenemos que:
    \begin{equation*}
        P(1_B) = P(2_B)\cdot P(1_B|2_B) + P(2_O)\cdot P(1_B|2_O)
    \end{equation*}

    Por tanto, usando la Ley de Laplace, tenemos que:
    \begin{equation*}
        P(B) = \frac{9}{10}\cdot \frac{19}{21} + \frac{1}{10}\cdot \frac{18}{21} = \frac{9}{10} = 0.9
    \end{equation*}
\end{ejercicio}

\begin{ejercicio} \label{ej:4.Ejercicio7}
    Se dispone de tres urnas con la siguiente composición de bolas blancas y negras:
    \begin{center}
        $U_1$: 5B y 5N
        \qquad
        $U_2$: 6B y 4N
        \qquad
        $U_3$: 7B y 3N
    \end{center}

    Se elige una urna al azar y se sacan cuatro bolas sin reemplazamiento.
    \begin{enumerate}
        \item Calcular la probabilidad de que las cuatro sean blancas.

        Por el teorema de la Probabilidad Total, tenemos que:
        \begin{equation*}
            P(4B) = P(U_1)P(4B|U_1) + P(U_2)P(4B|U_2) + P(U_3)P(4B|U_3)
        \end{equation*}

        Usando la ley de Laplace, tenemos que:
        \begin{equation*}
            P(4B) = \frac{1}{3}\left[
            \frac{5}{10}\cdot \frac{4}{9}\cdot \frac{3}{8}\cdot \frac{2}{7}\cdot 
            +
            \frac{6}{10}\cdot \frac{5}{9}\cdot \frac{4}{8}\cdot \frac{3}{7}\cdot 
            +
            \frac{7}{10}\cdot \frac{6}{9}\cdot \frac{5}{8}\cdot \frac{4}{7}\cdot 
            \right] = \frac{1}{3}\cdot \frac{1320}{5040} = \frac{11}{126} \approx 0.0873
        \end{equation*}
        
        \item Si en las bolas extraídas sólo hay una negra, ¿cuál es la probabilidad de que la urna elegida haya sido $U_2$?

        Se pide $P(U_2|1N)$. Por la regla de Bayes, tenemos que:
        \begin{equation*}
            P(U_2|1N) = \frac{P(U_2)P(1N|U_2)}{P(U_1)P(1N|U_1) + P(U_2)P(1N|U_2) + P(U_3)P(1N|U_3)}
        \end{equation*}

        Como tenemos que $P(U_i)=\frac{1}{3}\;\forall i$, tenemos que:
        \begin{equation*}
            P(U_2|1N) = \frac{P(1N|U_2)}{P(1N|U_1) + P(1N|U_2) + P(1N|U_3)}
        \end{equation*}

        Para calcular cada probabilidad, empleamos combinatoria. Tenemos que se trata de combinaciones sin remplazamineto, por lo que:
        \begin{equation*}
            P(1N|U_1) = \frac{C_{5,1}C_{5,3}}{C_{10, 4}} = \frac{5}{21}
            \qquad
            P(1N|U_2) = \frac{C_{4,1}C_{6,3}}{C_{10, 4}} = \frac{8}{21}
        \end{equation*}
        \begin{equation*}
            P(1N|U_3) = \frac{C_{3,1}C_{7,3}}{C_{10, 4}} = \frac{1}{2}
        \end{equation*}

        Por tanto, tenemos que:
        \begin{equation*}
            P(U_2|1N) = \frac{P(1N|U_2)}{P(1N|U_1) + P(1N|U_2) + P(1N|U_3)} = \frac{16}{47}\approx 0.3404
        \end{equation*}
    \end{enumerate}
\end{ejercicio}

\begin{ejercicio} \label{ej:4.Ejercicio8}
    La probabilidad de que se olvide inyectar el suero a un enfermo durante la ausencia del doctor es $2/3$. Si se le inyecta el suero, el enfermo tiene igual probabilidad de mejorar que de empeorar, pero si no se le inyecta, la probabilidad de mejorar se reduce a 0.25. Al regreso, el doctor encuentra que el enfermo ha empeorado. ¿Cuál es la probabilidad de que no se le haya inyectado el suero?

    Tenemos los siguientes suceso:
    \begin{itemize}
        \item $A\longrightarrow $ El paciente mejora.
        \item $\bar{A}\longrightarrow $ El paciente empeora.
        \item $B\longrightarrow $ Se le ha inyectado el suero.
        \item $\bar{B}\longrightarrow $ No se le ha inyectado el suero.
    \end{itemize}

    El enunciado afirma que $P(\bar{B})=\frac{2}{3}$. Por tanto,
    \begin{equation*}
        P(B)=1-P(\bar{B}) = 1-\frac{2}{3} = \frac{1}{3}
    \end{equation*}

    Además, tenemos que $P(A|B) = P(\bar{A}|B)$. Por tanto:
    \begin{multline*}
        P(A|B) = P(\bar{A}|B) \Longleftrightarrow \frac{P(A\cap B)}{P(B)} = \frac{P(\bar{A}\cap B)}{P(B)} \Longleftrightarrow \\ \Longleftrightarrow
        P(A\cap B) = P(B-A) = P(B)-P(A\cap B) \Longleftrightarrow 2P(A\cap B)=P(B) 
        \Longleftrightarrow \\ \Longleftrightarrow
        \frac{P(A\cap B)}{P(B)} = \frac{1}{2} = P(A|B) = P(\bar{A}|B)
    \end{multline*}

    También sabemos por el enunciado que $P(A|\bar{B})=\frac{1}{4}$. Usando la regla de la probabilidad total, tenemos que:
    \begin{multline*}
        P(\Omega) = 1 = P(A|\bar{B}) + P(\bar{A}|\bar{B}) \Longrightarrow P(\bar{A}|\bar{B}) = 1-P(A|\bar{B}) = \frac{3}{4} = \frac{P(\bar{A} \cap \bar{B})}{P(\bar{B})}
        \Longrightarrow \\ \Longrightarrow
        P(\bar{A} \cap \bar{B}) = \frac{3}{4}\cdot \frac{2}{3} = \frac{1}{2}
    \end{multline*}
    
    Se pide calcular $P(\bar{B}|\bar{A})$. Usando la definición de probabilidad y la regla de Bayes, tenemos que:
    \begin{equation*}
        P(\bar{B}|\bar{A}) = \frac{P(\bar{B}\cap\bar{A})}{P(\bar{A})}
        = \frac{P(\bar{B}\cap\bar{A})}{P(B)P(\bar{A}|B) + P(\bar{B})P(\bar{A}|\bar{B})} = \frac{\frac{1}{2}}{\frac{1}{3}\cdot \frac{1}{2} + \frac{2}{3}\cdot \frac{3}{4}} = \frac{3}{4}
    \end{equation*}

    Por tanto, la probabilidad de que no se le haya inyectado el suero y por ello haya empeorado es de $0.75$.
\end{ejercicio}

\begin{ejercicio} \label{ej:4.Ejercicio9}
    $N$ urnas contienen cada una 4 bolas blancas y 6 negras, mientras otra urna contiene 5 blancas y 5 negras. De las $N + 1$ urnas se elige una al azar y se extraen dos bolas sucesivamente, sin reemplazamiento, resultando ser ambas negras. Sabiendo que la probabilidad de que queden 5 blancas y 3 negras en la urna elegida es $1/7$, encontrar $N$.\\

    En este caso, tenemos dos experimentos aleatorios. Respecto al primer experimento, que consiste en elegir una urna, sea su espacio muestral $\Omega_1 = \{U_A, U_B\}$, con:
    \begin{itemize}
        \item $U_A$: sacar una urna del primer tipo. Hay $N$ urnas de este tipo, cada una con $4$ bolas blancas y $6$ negras.
        \item $U_B$: sacar una urna del segundo tipo. Hay $1$ urna de este tipo, con $5$ bolas blancas y $5$ negras.
    \end{itemize}

    Respecto al segundo experimento aleatorio, que consiste en elegir dos bolsas sin reemplazamiento de la urna elegida, sea su espacio muestral $\Omega_2 = \{BB, NN, BN\}$, con:
    \begin{itemize}
        \item $BB$: sacar dos bolas blancas.
        \item $NN$: sacar dos bolas negras.
        \item $BN$: sacar una bola blanca y una bola negra.
    \end{itemize}

    Usando la regla de Laplace, tenemos que:
    \begin{equation*}
        P(U_A) = \frac{N}{N+1} \hspace{2cm} P(U_B) = \frac{1}{N+1}
    \end{equation*}

    Consideramos ahora el caso de que, tras extraer dos bolas negras, queden 5 blancas y 3 negras en la urna elegida, cuya probabilidad sabemos que es $1/7$. Originalmente había en la urna $5$ bolas de cada tipo, por lo que la urna es del tipo $U_B$. Por tanto, este suceso es elegir la urna $U_B$ sabiendo que hemos sacado dos bolas negras, es decir, $U_B|(NN)$. Por tanto,
    \begin{equation*}
        P(U_B | NN) = \frac{1}{7} = \frac{P(U_B\cap NN)}{P(NN)}
    \end{equation*}

    Usando la regla de la multiplicación, tenemos que:
    \begin{equation*}
        P(U_B\cap NN) = P(U_B) \cdot P(NN|U_B) = \frac{1}{N+1} \cdot \frac{5}{10} \cdot \frac{4}{9} = \frac{2}{9(N+1)}
    \end{equation*}

    Usando el teorema de la Probabilidad total, tenemos que:
    \begin{equation*}\begin{split}
        P(NN) & = P(U_A)\cdot P(NN|U_A) + P(U_B)\cdot P(NN|U_B)\\
        &= \frac{N}{N+1}\cdot \frac{6}{10}\cdot \frac{5}{9} + \frac{1}{N+1}\cdot \frac{5}{10}\cdot \frac{4}{9} \\
        & = \frac{N}{3(N+1)} + \frac{2}{9(N+1)} = \frac{3N + 2}{9(N+1)}
    \end{split}\end{equation*}

    Por tanto, como $P(U_B|NN)=\frac{1}{7}$, tenemos que:
    \begin{equation*}
        P(U_B | NN) = \frac{1}{7} = \frac{\displaystyle \frac{2}{9(N+1)}}{\displaystyle \frac{3N + 2}{9(N+1)}} = \frac{2}{3N+2} \Longrightarrow 3N+2 = 14 \Longrightarrow N=4
    \end{equation*}

    Por tanto, tenemos que hay $4$ urnas del tipo $U_A$.
\end{ejercicio}

\begin{ejercicio} \label{ej:4.Ejercicio10}
    Se dispone de 6 cajas, cada una con 12 tornillos; una caja tiene 8 buenos y 4 defectuosos; dos cajas tienen 6 buenos y 6 defectuosos y tres cajas tienen 4 buenos y 8 defectuosos. Se elige al azar una caja y se extraen 3 tornillos con reemplazamiento, de los cuales 2 son buenos y 1 es defectuoso. ¿Cuál es la probabilidad de que la caja elegida contuviera 6 buenos y 6 defectuosos?\\

    En este caso, tenemos dos experimentos aleatorios. Respecto al primer experimento, que consiste en elegir una caja, sea su espacio muestral $\Omega_1 = \{C_1, C_2, C_3\}$, con:
    \begin{itemize}
        \item $C_1$: elegir una caja del primer tipo. Solo hay una caja de este tipo, que tiene $8$ tornillos buenos y $4$ defectuosos.
        \item $C_2$: elegir una caja del segundo tipo. Hay dos cajas de este tipo, cada una con $6$ tornillos buenos y $6$ defectuosos.
        \item $C_3$: elegir una caja del tercer tipo. Hay tres cajas de este tipo, cada una con $4$ tornillos buenos y $8$ defectuosos.
    \end{itemize}

    Respecto al segundo experimento aleatorio, se repite tres veces y consiste en elegir cada vez un tornillo con reemplazamiento de la caja elegida. Sea su espacio muestral $\Omega_2 = \{B, D\}$, con:
    \begin{itemize}
        \item $B$: elegir un tornillo bueno.
        \item $D$: elegir un tornillo defectuoso.
    \end{itemize}

    El enunciado nos pide:
    \begin{equation*}
        P[C_2 | (BBD)] = \frac{P[C_2 \cap (BBD)]}{P(BBD)}
    \end{equation*}

    Por la regla de la multiplicación, tenemos que:
    \begin{equation*}
        P[C_2 \cap BBD] = P(C_2)\cdot P(B|C_2) \cdot P[B|(B\cap C_2)] \cdot P[B|(B\cap B\cap C_2)]
    \end{equation*}

    usando la regla de Laplace en cada caso, tenemos que:
    \begin{equation*}
        P[C_2 \cap BBD] = \frac{2}{6}\cdot \frac{6}{12}\cdot \frac{6}{12} \cdot \frac{6}{12} = \frac{1}{3}\cdot \frac{1}{2^3} = \frac{1}{24}
    \end{equation*}

    Usando de nuevo la regla de la multiplicación y la regla de Laplace, junto con la regla de la probabilidad total, tenemos que:
    \begin{equation*}\begin{split}
        P(BBD)& = P(C_1)P(BBD|C_1) + P(C_2)P(BBD|C_2) + P(C_3)P(BBD|C_3)=\\
        &= \frac{1}{6}\cdot \frac{8}{12}\cdot \frac{8}{12}\cdot \frac{4}{12} + \frac{2}{6}\cdot \frac{6}{12}\cdot\frac{6}{12}\cdot\frac{6}{12} + \frac{3}{6}\cdot \frac{4}{12}\cdot\frac{4}{12}\cdot\frac{8}{12} =\\
        &= \frac{2}{3^4} + \frac{1}{2^3\cdot 3} + \frac{1}{3^3} = \frac{67}{648} \approx 0.103395
    \end{split}\end{equation*}

    Por tanto, tenemos que la probabilidad de que la caja elegida contuviera $6$ buenos y $6$ defectuosos es:
    \begin{equation*}
        P[C_2 | (BBD)] = \frac{P[C_2 \cap (BBD)]}{P(BBD)}
        = \frac{\frac{1}{24}}{\frac{67}{648}} = \frac{27}{67} \approx 0.4029
    \end{equation*}
    
\end{ejercicio}

\begin{ejercicio} \label{ej:4.Ejercicio11}
    Se seleccionan $n$ dados con probabilidad $p_n=\frac{1}{2^n},\;n\in \bb{N}$. Si se lanzan estos $n$ dados y se obtiene una suma de 4 puntos, ¿Cuál es la probabilidad de haber seleccionado 4 dados?

    Sea $B$ el suceso de que sumen 4 puntos, y suponemos cada dado como un dado estándar (6 caras, equiprobables). Sea $A_n$ el suceso de haber considerado $n$ dados.

    Se pide calcular $P(A_4|B)$, que calculamos empleado el Teorema de Bayes:
    \begin{equation*}
        P(A_4|B) = \frac{P(A_4)P(B|A_4)}{\displaystyle \sum_{i=1}^{\infty}P(A_i)P(B|A_i)}
    \end{equation*}

    Tenemos que $P(A_i)=p_i=\frac{1}{2^n}$. Además, se tiene que $P(B|A_i)=0 \qquad \forall i>4$, ya que si hay más de 4 dados, la suma va a ser mayor que 4. Por tanto,
    \begin{equation*}
        P(A_4|B) = \frac{\frac{1}{2^4}\cdot P(B|A_4)}{\displaystyle \sum_{i=1}^{4}\frac{1}{2^i}\cdot P(B|A_i)}
    \end{equation*}

    Calculamos ahora la probabilidad de que sumen 4 tras haber elegido $i$ dados. Los casos totales son las combinaciones de 6 números que se pueden realizar con 2 dados. Aunque aparentemente no importa el orden, de hecho sí importa porque queremos que las combinaciones $(1,3)$ y $(3,1)$ sean distintas. También hay reemplazamiento, ya que cada dado tiene siempre 6 dados, por lo que estamos ante variaciones con repetición.
    \begin{itemize}
        \item \underline{Para $n=1$ dado}:

        Tenemos que los casos totales son: $$VR_{6,1} = 6^1 = 6$$

        Tenemos que los casos favorables son $\{(4)\}$, por lo que solo hay un caso favorable. Por tanto,
        \begin{equation*}
            P(B|A_1)=\frac{1}{6}
        \end{equation*}

        \item \underline{Para $n=2$ dados}:

        Tenemos que los casos totales son: $$VR_{6,2} = 6^2$$

        Tenemos que los casos favorables son $\{(1,3), (3,1), (2,2)\}$, por lo que hay tres casos favorables. Por tanto,
        \begin{equation*}
            P(B|A_2)=\frac{3}{6^2}
        \end{equation*}

        \item \underline{Para $n=3$ dados}:

        Tenemos que los casos totales son: $$VR_{6,3} = 6^3$$

        Tenemos que los casos favorables son $\{(1,1,2), (1,2,1), (2,1,1)\}$, por lo que hay tres casos favorables. Por tanto,
        \begin{equation*}
            P(B|A_3)=\frac{3}{6^3}
        \end{equation*}

        \item \underline{Para $n=4$ dados}:

        Tenemos que los casos totales son: $$VR_{6,4} = 6^4$$

        Tenemos que los casos favorables son $\{(1,1,1,1)\}$, por lo que hay un caso favorable. Por tanto,
        \begin{equation*}
            P(B|A_4)=\frac{1}{6^4}
        \end{equation*}
    \end{itemize}


    Por tanto, tenemos que:
    \begin{equation*}
        P(A_4|B) = \frac{\frac{1}{2^4}\cdot P(B|A_4)}{\displaystyle \sum_{i=1}^{4}\frac{1}{2^i}\cdot P(B|A_i)}=\frac{1}{2197}\approx 0.445\cdot 10^{-3}
    \end{equation*}
\end{ejercicio}

\begin{ejercicio} \label{ej:4.Ejercicio12}
    Se lanza una moneda; si sale cara, se introducen $k$ bolas blancas en una urna y si sale cruz, se introducen $2k$ bolas blancas. Se hace una segunda tirada, poniendo en la urna $h$ bolas negras si sale cara y $2h$ si sale cruz. De la urna así compuesta se toma una bola al azar. ¿Cuál es la probabilidad de que sea negra?\\

    En este caso, tenemos dos experimentos aleatorios. Respecto al primer experimento; que se repite dos veces, consiste en lanzar una moneda. Sea su espacio muestral $\Omega_1 = \{C,+\}$, con:
    \begin{itemize}
        \item $+$: obtener cruz.
        \item $C$: obtener cara.
    \end{itemize}

    Respecto al segundo experimento aleatorio, se saca una bola de la caja. Sea su espacio muestral $\Omega_2 = \{N, B\}$, con:
    \begin{itemize}
        \item $N$: obtener una bola negra.
        \item $B$: obtener una bola blanca.
    \end{itemize}

    Uniendo el teorema de la probabilidad total con la regla de la multiplicación, tenemos que:
    \begin{multline*}
        P(N) = P(C)\cdot P(C|C) \cdot P(N|(CC))
        + P(C)\cdot P(+|C) \cdot P(N|(C+))
        +\\+ P(+)\cdot P(C|+) \cdot P(N|(+C))
        + P(+)\cdot P(+|+) \cdot P(N|(++))
    \end{multline*}

    Como tenemos que las dos repeticiones del primer experimento son independientes; es decir, el resultado del primer lanzamiento no influye en el resultado del segundo lanzamiento; tenemos que:
    \begin{multline*}
        P(N) = P(C)\cdot P(C) \cdot P(N|(CC))
        + P(C)\cdot P(+) \cdot P(N|(C+))
        +\\+ P(+)\cdot P(C) \cdot P(N|(+C))
        + P(+)\cdot P(+) \cdot P(N|(++))
    \end{multline*}

    Por la ley de Laplace, tenemos que $P(+)=P(C)=\frac{1}{2}$. Por tanto,
    \begin{equation*}
        P(N) = \frac{1}{4}\cdot P(N|(CC))
        + \frac{1}{4}\cdot  P(N|(C+))
        + \frac{1}{4}\cdot  P(N|(+C))
        + \frac{1}{4}\cdot  P(N|(++))
    \end{equation*}

    Veamos ahora cuántas bolas hay en cada combinación de resultados de la moneda:
    \begin{center}
        $CC \longrightarrow $ $k$ blancas y $h$ negras. \\
        $C+ \longrightarrow $ $k$ blancas y $2h$ negras. \\
        $+C \longrightarrow $ $2k$ blancas y $h$ negras. \\
        $++ \longrightarrow $ $2k$ blancas y $2h$ negras. \\
    \end{center}

    Por la ley de Laplace y la regla de la multiplicación, tenemos que:
    \begin{gather*}
        P(N\cap C\cap C) = \frac{1}{2}\cdot \frac{1}{2}\cdot \frac{h}{k+h}
        \qquad \qquad
        P(N\cap C\cap +) = \frac{1}{2}\cdot \frac{1}{2}\cdot \frac{2h}{k+2h}
        \\
        P(N\cap +\cap C) = \frac{1}{2}\cdot \frac{1}{2}\cdot \frac{h}{2k+h}
        \qquad \qquad
        P(N\cap +\cap +) = \frac{1}{2}\cdot \frac{1}{2}\cdot \frac{2h}{2k+2h}
    \end{gather*}

    Por tanto, como $P(CC) = P(C+) = P(+C) = P(++) = \frac{1}{4}$, tenemos que:
    \begin{gather*}
        P(N|(CC)) = \frac{\frac{h}{4(k+h)}}{\frac{1}{4}} = \frac{h}{k+h}
        \qquad \qquad
        P(N\cap C\cap +) = \frac{2h}{k+2h}
        \\
        P(N\cap +\cap C) = \frac{h}{2k+h}
        \qquad \qquad \qquad
        P(N\cap +\cap +) = \frac{2h}{2k+2h} = \frac{h}{k+h}
    \end{gather*}

    Por tanto, tenemos que:
    \begin{equation*}
        P(N) = \frac{1}{4}\left[\frac{h}{k+h} + \frac{2h}{k+2h} + \frac{h}{2k+h} + \frac{h}{k+h}\right]
        = \frac{1}{4}\left[\frac{2h}{k+h} + \frac{2h}{k+2h} + \frac{h}{2k+h}\right]
    \end{equation*}
    
\end{ejercicio}
\section{Gestión de Memoria}

\begin{ejercicio}
Si un computador no posee hardware de reubicación, e implementa intercambio (\textit{swapping}), entonces el gestor de memoria necesita utilizar un cargador para recalcular las direcciones físicas de un proceso. ¿Sería posible para el sistema de intercambio reubicar los segmentos de datos y pila? Explique cómo funcionaría este sistema, o si es imposible que funcione.
\end{ejercicio}

\begin{ejercicio}
Considere un sistema con un espacio lógico de memoria de 128K páginas (máximo espacio de memoria virtual) con 8 KB cada una, una memoria física de 64 MB y direccionamiento al nivel de byte. ¿Cuántos bits hay en la dirección lógica? ¿Y en la física?\\

Calculemos en primer lugar cuántas palabras se pueden direccionar en la memoria física:
\begin{equation*}
    2^6\cdot 2^{20} \cdot \frac{1~\text{palabra}}{1~B} = 2^{26}~\text{palabras}
\end{equation*}
Por tanto, necesitamos 26 bits para direccionar la memoria física.\\

Ahora, para calcular el número de bits necesarios para direccionar la memoria virtual.
Cada página es de $8~KB=2^{13}$, por lo que el desplazamiento ocupa 13 bits.
Respecto a la página, como hay $2^{17}$ páginas, necesitamos 17 bits para direccionarlas,
por lo que en total necesitamos 30 bits para direccionar la memoria virtual.
\end{ejercicio}

\begin{ejercicio}
Considérese un sistema con memoria virtual en el que el procesador tiene una tasa de utilización del 15\% y el dispositivo de paginación está ocupado el 97\% del tiempo, ¿qué indican estas medidas? ¿Y si con el mismo porcentaje de uso del procesador el porcentaje de uso del dispositivo de paginación fuera del 15\%?
\end{ejercicio}

\begin{ejercicio}
Sea un sistema de memoria virtual paginada con direcciones lógicas de 32 bits que
proporciona un espacio virtual de \(2^{20}\) páginas y con una memoria física de
32 Mbytes
¿cuánta memoria requiere en total un proceso que tenga 453Kbytes,
incluida su tabla de páginas cuyas entradas son de 32 bits?\\

    Calculemos el tamaño de página del sistema. Como, de los 32 bits de la dirección lógica,
    20 son para la página, tenemos que el tamaño de página es de $2^{12}~B=4~KB$. Veamos ahora cuántas
    páginas ocupa el proceso:
    \begin{equation*}
        453~KB \cdot \frac{1~\text{página}}{4~KB} = 113.25~\text{páginas}
    \end{equation*}
    Es decir, necesita 114 páginas. Por tanto, el tamaño de la tabla de páginas es de $114\cdot 32~b=114\cdot 4~B=456~B$.
    Por tanto, el tamaño total del proceso es de:
    \begin{equation*}
        456~B + 114~\text{páginas} \cdot \frac{2^{12}~B}{1~\text{página}} = 456~B + 114\cdot 2^{12}~B = 456~B + 466944~B = 467400~B
    \end{equation*}
\end{ejercicio}

\begin{ejercicio}
Un ordenador tiene 4 marcos de página. En la siguiente tabla se muestran: el tiempo de carga, el tiempo del último acceso y los bits R y M para cada página (los tiempos están en tics de reloj). Responda a las siguientes cuestiones justificando su respuesta.

\begin{center}
\begin{tabular}{|c|c|c|c|c|}
\hline
Página & Tiempo de carga & Tiempo última referencia & R (Referencia) & M (Modificación) \\
\hline
0 & 126 & 279 & 1 & 0 \\
1 & 230 & 235 & 1 & 0 \\
2 & 120 & 272 & 1 & 1 \\
3 & 160 & 200 & 1 & 1 \\
\hline
\end{tabular}
\end{center}

\begin{itemize}
    \item ¿Qué página se sustituye si se usa el algoritmo FIFO?
    \item ¿Qué página se sustituye si se usa el algoritmo LRU?
\end{itemize}
\end{ejercicio}

\begin{ejercicio}
¿Depende el tamaño del conjunto de trabajo de un proceso directamente del
tamaño del programa ejecutable asociado a él? Justifique su respuesta.\\

No, ya que el conjunto de trabajo depende de las páginas que se hayan referenciado en un determinado momento.
Si, por ejemplo, en dicho momento se está ejecutando un bucle, puede ser que el conjunto de trabajo sea muy pequeño,
porque dicho bucle se encuentre entero en la misma página.
\end{ejercicio}

\begin{ejercicio}
¿Por qué una cache (o la TLB) que se accede con direcciones virtuales puede producir
incoherencias y requiere que el sistema operativo la invalide en cada cambio de
contexto y, en cambio, una que se accede con direcciones físicas no lo requiere?
\end{ejercicio}

\begin{ejercicio}
Un ordenador proporciona un espacio de direccionamiento lógico (virtual) a cada
proceso de 65536 bytes de espacio dividido en páginas de 4096 bytes.
Cierto programa tiene un tamaño de región de texto de 32768 bytes, un
tamaño de región de datos de 16386 bytes y tamaño de región de pila de 15878 bytes.
¿Cabría este programa en el espacio de direcciones? (Una página no puede ser utilizada
por regiones distintas). Si no es así, ¿cómo podríamos conseguirlo, dentro del
esquema de paginación?\\

    Calculamos cuántas páginas ocupa cada región:
    \begin{itemize}
        \item Región de texto: $32768~B \cdot \frac{1~\text{página}}{4096~B} = 8~\text{páginas}$
        \item Región de datos: $16386~B \cdot \frac{1~\text{página}}{4096~B} \approx 4.0004~\text{páginas}$
        \item Región de pila: $15878~B \cdot \frac{1~\text{página}}{4096~B} \approx 3.8764~\text{páginas}$
    \end{itemize}

    Por tanto, en total se necesitan $8+5+4=17$ páginas. Como cada página ocupa $4096~B$, el tamaño total del programa es de
    $17\cdot 4096~B = 69632~B$. Por tanto, vemos claramente que el programa no cabe en el espacio de direcciones.\\

    El espacio que en realidad necesitamos es de $32768~B + 16386~B + 15878~B = 65032~B$, por lo que
    defectivamente cabe en el espacio de direcciones. El problema con el que nos hemos encontrado 
    es el de la fragmentación interna. Para solucionarlo, la mejor solución
    es disminuir el tamaño de página, de forma que la fragmentación interna sea menor.
\end{ejercicio}

\begin{ejercicio}
Analice qué puede ocurrir en un sistema que usa paginación por demanda si se recompila un programa mientras este está ejecutando. Proponga soluciones a los problemas que pueden surgir en esta situación.
\end{ejercicio}
\begin{ejercicio}
Para cada uno de los siguientes campos de la tabla de páginas, se debe explicar si es la MMU o el sistema operativo quién los lee y escribe (en este último caso si se activa o desactiva), y en qué momentos:
\begin{enumerate}
    \item Número de marco
    \item Bit de presencia
    \item Bit de protección
    \item Bit de modificación
    \item Bit de referencia
\end{enumerate}
\end{ejercicio}

\begin{ejercicio}
Suponga que la tabla de páginas para el proceso actual se parece a la de la figura. Todos los números son decimales, la numeración comienza en todos los casos desde cero, y todas las direcciones de memoria son direcciones en bytes. El tamaño de página es de 1024 bytes.

\begin{center}
    \begin{tabular}{|c|c|c|c|c|}
    \hline
    Nº Página & Bit de validez & Bit de referencia & Bit de modificación & Nº Marco \\
    \hline
    0 & 0 & 1 & 0 & 4 \\
    1 & 1 & 1 & 0 & 7 \\
    2 & 1 & 0 & 1 & 1 \\
    3 & 0 & 0 & 0 & - \\
    4 & 1 & 0 & 0 & 2 \\
    5 & 1 & 1 & 1 & 0 \\
    \hline
    \end{tabular}
    \end{center}

    ¿Qué direcciones físicas, si existen, corresponderán con cada una de las siguientes direcciones virtuales? (no intente manejar ninguna falta de página, si las hubiese)
    \begin{enumerate}
        \item 999
        
        En primer, buscamos en qué pagina se encuentra la dirección virtual. Como $999=0\cdot 1024 + 999$, se encuentra en la página 0.
        Ahora, como el bit de validez es 0, no se encuentra en memoria, por lo que no existe dirección física para ella.
        
        \item 2121
        
        En primer, buscamos en qué pagina se encuentra la dirección virtual. Como $2121=2\cdot 1024 + 73$, se encuentra en la página 2.
        En este caso si tiene el bit de validez activado (1), y su marco es el 1.
        Por tanto, la dirección física es $1\cdot 1024 + 73 = 1097$.

        \item 5400
        
        En primer, buscamos en qué pagina se encuentra la dirección virtual. Como $5400=5\cdot 1024 + 280$, se encuentra en la página 5.
        En este caso si tiene el bit de validez activado (1), y su marco es el 0.
        Por tanto, la dirección física es $0\cdot 1024 + 280 = 280$.
    \end{enumerate}
\end{ejercicio}

\begin{ejercicio}
Sea la siguiente secuencia de números de página referenciados: 1,2,3,4,1,2,5,1,2,3,4,5. Calcule el número de faltas de página que se producen utilizando el algoritmo FIFO y considerando que el número de marcos de página de que disfruta nuestro proceso es de:
\begin{enumerate}
    \item 3 marcos
    
    \begin{table}[H]
        \centering
        \begin{tabular}{|c||c|c|c|c|c|c|c|c|c|c|c|c|}
            \hline
            Pag. Referenciada & 1 & 2 & 3 & 4 & 1 & 2 & 5 & 1 & 2 & 3 & 4 & 5 \\ \hline \hline
            Está 1            & 1 & 1 & 1 & - & 1 & 1 & 1 & 1 & 1 & - & - & - \\ \hline
            Está 2            & - & 2 & 2 & 2 & - & 2 & 2 & 2 & 2 & 2 & - & - \\ \hline
            Está 3            & - & - & 3 & 3 & 3 & - & - & - & - & 3 & 3 & 3 \\ \hline
            Está 4            & - & - & - & 4 & 4 & 4 & - & - & - & - & 4 & 4 \\ \hline
            Está 5            & - & - & - & - & - & - & 5 & 5 & 5 & 5 & 5 & 5 \\ \hline
            Falta             & $\ast$ & $\ast$ & $\ast$ & $\ast$ & $\ast$ & $\ast$ & $\ast$ &   &   & $\ast$ & $\ast$ &  \\ \hline
        \end{tabular}
        \caption{Algoritmo FIFO con 3 marcos}
    \end{table}
    En este caso, el número de faltas de página es 9.


    \item 4 marcos
    
    \begin{table}[H]
        \centering
        \begin{tabular}{|c||c|c|c|c|c|c|c|c|c|c|c|c|}
            \hline
            Pag. Referenciada & 1 & 2 & 3 & 4 & 1 & 2 & 5 & 1 & 2 & 3 & 4 & 5 \\ \hline \hline
            Está 1            & 1 & 1 & 1 & 1 & 1 & 1 & - & 1 & 1 & 1 & 1 & - \\ \hline
            Está 2            & - & 2 & 2 & 2 & 2 & 2 & 2 & - & 2 & 2 & 2 & 2 \\ \hline
            Está 3            & - & - & 3 & 3 & 3 & 3 & 3 & 3 & - & 3 & 3 & 3 \\ \hline
            Está 4            & - & - & - & 4 & 4 & 4 & 4 & 4 & 4 & - & 4 & 4 \\ \hline
            Está 5            & - & - & - & - & - & - & 5 & 5 & 5 & 5 & - & 5 \\ \hline
            Falta             & $\ast$ & $\ast$ & $\ast$ & $\ast$ &   &   & $\ast$ & $\ast$ & $\ast$ & $\ast$ & $\ast$ & $\ast$ \\ \hline
        \end{tabular}
        \caption{Algoritmo FIFO con 4 marcos}
    \end{table}
    En este caso, el número de faltas de página es 10.
\end{enumerate}
¿Se corresponde esto con el comportamiento intuitivo de que disminuirá el número de faltas de página al aumentar el tamaño de memoria de que disfruta el proceso?\\

Claramente no, ya que en el segundo caso tenemos más faltas de página que en el primero aun habiendo aumentado el número de marcos.
Esto se debe a que dicha regla no es cierta siempre, sino que depende de la secuencia de páginas referenciadas. Por norma general, sí que es cierta.
\end{ejercicio}



\begin{ejercicio}
    ¿Por qué la localidad no es un factor que se tiene en cuenta en los sistemas con
segmentación?
\end{ejercicio}


\begin{ejercicio}
    En la gestión de memoria en un sistema paginado, ¿qué estructura/s de datos necesitará
mantener el Sistema Operativo para administrar el espacio libre?

    En paginación, se usan tres estructuras de datos:
    \begin{itemize}
        \item {Tabla de páginas}
        \item {Tabla de marcos de página}
        \item {Tabla de ubicación en disco}
    \end{itemize}
\end{ejercicio}


\begin{ejercicio}
    ¿Cuánto puede avanzar como máximo la aguja del algoritmo de reemplazo de páginas
del reloj durante la selección de una página?\\

    Como máximo, puede avanzar hasta la página que se está referenciando en ese momento, es decir,
    tantos pasos como marcos de página haya. Suponiendo que todos los marcos tienen el bit de referencia
    $R=1$ activado, al avanzar la aguja se irán desactivando los bits de referencia de los marcos que se
    vayan encontrando, hasta que se encuentre el primero, que ya lo había desactivado, por lo que se
    detendrá y esa será la página seleccionada para ser sustituida.
\end{ejercicio}



\begin{ejercicio} \label{ej:Rel3_Ej16}
    Situándonos en un sistema paginado, donde cada proceso tiene asignado un número fijo
de marcos de páginas. Supongamos la siguiente situación: existe un proceso con 7
páginas y tiene asignados 5 marcos de página. Indica el contenido de la memoria
después de cada referencia a una página si como algoritmo de sustitución de página
utilizamos el LRU (la página no referenciada hace más tiempo). La secuencia de
referencias es la indicada en la Tabla \ref{tab:Rel3_Ej16}.

    \begin{table}[H]
        \centering
        \begin{tabular}{c||c|c|c|c|c|c|c|c|c|c|c|c|}
        Referencias                       & 2 & 1 & 3 & 4 & 1 & 5 & 6 & 4 & 5 & 7 & 4 & \multicolumn{1}{c|}{2} \\ \hline \hline
        \multirow{5}{*}{Marcos de página} & 2 & 2 & 2 & 2 & 2 & 2 & 6 & 6 & 6 & 6 & 6 & \multicolumn{1}{c|}{6} \\ \cline{2-13} 
                                        &   & 1 & 1 & 1 & 1 & 1 & 1 & 1 & 1 & 1 & 1 & \multicolumn{1}{c|}{2} \\ \cline{2-13} 
                                        &   &   & 3 & 3 & 3 & 3 & 3 & 3 & 3 & 7 & 7 & \multicolumn{1}{c|}{7} \\ \cline{2-13} 
                                        &   &   &   & 4 & 4 & 4 & 4 & 4 & 4 & 4 & 4 & \multicolumn{1}{c|}{4} \\ \cline{2-13} 
                                        &   &   &   &   &   & 5 & 5 & 5 & 5 & 5 & 5 & \multicolumn{1}{c|}{5} \\ \hline
        Falta de Página                   & $\ast$ & $\ast$ & $\ast$ & $\ast$ &   & $\ast$ & $\ast$ &   &   & $\ast$ &   & $\ast$                     
        \end{tabular}
        \caption{Ejercicio \ref{ej:Rel3_Ej16}}
        \label{tab:Rel3_Ej16}
    \end{table}
    
    Se producen 8 faltas de página.

\end{ejercicio}



\begin{ejercicio}
    ¿Cuál es la ventaja del algoritmo de frecuencia de faltas de página (FFP) sobre el algoritmo basado en el
modelo del conjunto de trabajo utilizando el tamaño de ventana $w$? ¿Cuál es la
desventaja?\\

    La ventaja del algoritmo de faltas de página es que, en el caso de tener muchos marcos de página que lleven tiempo sin referenciarse,
    el algoritmo liberará bastantes marcos en la siguiente falta de página; mientras que si tenemos pocos marcos de página usados,
    el algoritmo incrementará el número de marcos de página, logrando siempre un equilibrio.
    Además, es un modelo bastante más sencillo y menos costoso,
    ya que tan solo se llama cuando se produce una falta de página, mientras que en el
    modelo del conjunto de trabajo se llama cada vez que se referencia una página.


    La desventaja es que en ciertos momentos, tenemos cargadas en memoria páginas que no se usan,
    cosa que evitamos con el algoritmo basado en los conjuntos de trabajo.
\end{ejercicio}



\begin{ejercicio} \label{ej:Rel3_Ej18}
    Supongamos que tenemos un proceso ejecutándose en un sistema paginado, con
gestión de memoria basada en el algoritmo de sustitución frecuencia de faltas de
página. El proceso tiene 5 páginas ( 0, 1, 2, 3, 4 ). Represente el contenido de la
memoria real para ese proceso (es decir, indique que páginas tiene cargadas en cada
momento) y cuándo se produce una falta de página. Suponga que, inicialmente, está
cargada la página 2, el resto de páginas están en memoria secundaria y que no hay
restricciones en cuanto al número de marcos de página disponibles. La cadena de
referencias a página es: 0 3 1 1 1 3 4 4 2 2 4 0 0 0 0 3 y el parámetro es $\tau=3$.

    \begin{table}[H]
        \centering
        \begin{tabular}{|c|c|c|c|c|c|c|c|c|c|c|c|c|c|c|c|c|}
            Pag. Referenciada & 0 & 3 & 1 & 1 & 1 & 3 & 4 & 4 & 2 & 2 & 4 & 0 & 0 & 0 & 0 & 3 \\ \hline \hline
            Está 0            & 0 & 0 & 0 & 0 & 0 & 0 & - & - & - & - & - & 0 & 0 & 0 & 0 & 0 \\ \hline
            Está 1            & - & - & 1 & 1 & 1 & 1 & 1 & 1 & 1 & 1 & 1 & 1 & 1 & 1 & 1 & 1 \\ \hline
            Está 2            & 2 & 2 & 2 & 2 & 2 & 2 & - & - & 2 & 2 & 2 & 2 & 2 & 2 & 2 & 2 \\ \hline
            Está 3            & - & 3 & 3 & 3 & 3 & 3 & 3 & 3 & 3 & 3 & 3 & 3 & 3 & 3 & 3 & 3 \\ \hline
            Está 4            & - & - & - & - & - & - & 4 & 4 & 4 & 4 & 4 & 4 & 4 & 4 & 4 & 4 \\ \hline \hline
            Falta             & * & * & * &   &   &   & * &   & $\red{\ast}$ &   &   &    &   &   &   & 
        \end{tabular}
        \caption{Ejercicio \ref{ej:Rel3_Ej18}}
        \label{tab:Rel3_Ej18}
    \end{table}

    Notemos que en $\red{(\ast)}$ no se ha eliminado ninguna página, ya que la anterior falta de página había ocurrido en $t-2$,
    y la ventana es de $\tau=3$.
\end{ejercicio}



\begin{ejercicio} \label{ej:Rel3_Ej19}
    Describa el funcionamiento del algoritmo de sustitución basado en la frecuencia de
faltas de página, con los siguientes datos: 4 marcos de página, en $t = 0$ la memoria
contiene a la página 2. El tamaño de la ventana es $\tau= 3$ y se produce la secuencia de
referencias de páginas, 1 4 2 2 2 4 5 5 3 3 5 1 1 1 1 4

    \begin{table}[H]
        \centering
        \begin{tabular}{|c|c|c|c|c|c|c|c|c|c|c|c|c|c|c|c|c|}
        Pag. Referenciada & 1 & 4 & 2 & 2 & 2 & 4 & 5 & 5 & 3 & 3 & 5 & 1 & 1 & 1 & 1 & 4 \\ \hline \hline
        Marco 0           & 1 & 1 & 1 & 1 & 1 & 1 & 5 & 5 & 5 & 5 & 5 & $\red{!}$ \\ \hline
        Marco 1           & 2 & 2 & 2 & 2 & 2 & 2 & 2 & 2 & 2 & 2 & 2 & $\red{!}$  \\ \hline
        Marco 2           & - & 4 & 4 & 4 & 4 & 4 & 4 & 4 & 4 & 4 & 4 & $\red{!}$  \\ \hline
        Marco 3           & - & - & - & - & - & - & - & - & 3 & 3 & 3 & $\red{!}$\\ \hline \hline
        Falta de Página   & * &   &   &   &   &   & * &   & $\red{\ast}$ &   &   & $\red{\ast}$
        \end{tabular}
        \caption{Ejercicio \ref{ej:Rel3_Ej19}}
        \label{tab:Rel3_Ej19}
    \end{table}

    Notemos que en $\red{(\ast)}$ no se ha eliminado ninguna página, ya que no había transcurrido el tiempo suficiente
    para que se cumpliera la condición de la ventana $\tau=3$. En el último caso,
    cabe destacar que no hay marco de página disponible para la página 1, por lo que se produce una falta de página
    y el proceso se bloquea.
\end{ejercicio}


\begin{ejercicio} \label{ej:Rel3_Ej20}
    Describa el funcionamiento del algoritmo de sustitución global basado en el algoritmo
basado en el modelo del conjunto de trabajo, con los siguientes datos: 4 marcos de
página, en $t= 0$ la memoria contiene a la página 2 que se referenció en dicho instante de
tiempo. El tamaño de la ventana es $\tau=3$ y se produce la secuencia de referencias de
páginas, 1 4 4 4 2 4 1 1 3 3 5 5 5 5 1 4
    \begin{table}[H]
        \centering
        \begin{tabular}{|c|c|c|c|c|c|c|c|c|c|c|c|c|c|c|c|c|}
        Pag. Referenciada & 1 & 4 & 4 & 4 & 2 & 4 & 1 & 1 & 3 & 3 & 5 & 5 & 5 & 5 & 1 & 4 \\ \hline \hline
        Marco 0           & 1 & 1 & 1 & 1 & 2 & 2 & 2 & - & 3 & 3 & 3 & 3 & 3 & - & 1 & 1 \\ \hline
        Marco 1           & 2 & 2 & 2 & - & - & - & 1 & 1 & 1 & 1 & 1 & - & - & - & - & 4 \\ \hline
        Marco 2           & - & 4 & 4 & 4 & 4 & 4 & 4 & 4 & 4 & - & 5 & 5 & 5 & 5 & 5 & 5 \\ \hline
        Marco 3           &   &   &   &   &   &   &   &   &   &   &   &   &   &   &   &   \\ \hline \hline
        Falta de Página   & * &   &   &   & * &   & * &   & * &   & * &   &   &   & * & *
        \end{tabular}
        \caption{Ejercicio \ref{ej:Rel3_Ej20}}
        \label{tab:Rel3_Ej20}
    \end{table}
\end{ejercicio}



\begin{ejercicio}
    Una computadora con memoria virtual paginada tiene un bit U por página virtual, que se
pone automáticamente a 1 cuando se realiza un acceso a la página. Existe una
instrucción \verb|limpiar_U (dir_base_tabla)| que permite poner a 0 el conjunto de los bits U de
todas las páginas de la tabla de páginas cuya dirección de comienzo pasamos como
argumento. Explica cómo puede utilizarse este mecanismo para la implementación de un
algoritmo de sustitución basado en el modelo del conjunto de trabajo.
\end{ejercicio}


\begin{ejercicio}
    Un Sistema Operativo con memoria virtual paginada tiene el mecanismo \verb|fijar_página(np)|
cuyo efecto es proteger contra la sustitución al marco de página en que se ubica la página
virtual \verb|np|. El mecanismo \verb|des_fijar (np)| suprime esta protección.

\begin{enumerate}
    \item ¿Qué estructura/s de datos son necesarias para la realización de estos mecanismos?
    \item ¿En qué caso puede ser de utilidad estas primitivas?
    \item ¿Qué riesgos presentan y qué restricciones deben aportarse a su empleo?
\end{enumerate}
\end{ejercicio}


\begin{ejercicio}
    Disponemos de un ordenador que cuenta con las siguientes características: tiene una
memoria RAM de 4KBytes, permite usar memoria virtual paginada, las páginas son de
1KBytes de tamaño y las direcciones virtuales son de 16 bits. El primer marco de página
(marco 0) se usa únicamente por el Kernel y los demás marcos están disponibles para
su uso por los procesos que se ejecutan en el sistema. Supongamos que tenemos sólo
dos procesos, P1 y P2, y que utilizan las siguientes direcciones de memoria virtual y en
el siguiente orden:
    \begin{table}[h]
        \centering
        \begin{tabular}{|c|c|}
            \hline
            Proceso & Direcciones Virtuales \\
            \hline
            P1 & 0-99 \\
            P2 & 0-500 \\
            P1 & 100-500 \\
            P2 & 501-1500 \\
            P1 & 3500-3700 \\
            P2 & 1501-2100 \\
            P1 & 501-600 \\
            \hline
        \end{tabular}
        \caption{Direcciones virtuales utilizadas por los procesos P1 y P2}
    \end{table}
    \begin{enumerate}
        \item ¿Cuántos marcos de página tiene la memoria RAM de este ordenador?
        \item ¿Cuántos bits necesitamos para identificar los marcos de página?
        \item Describe los fallos de página que tendrán lugar para cada intervalo de ejecución de los
        procesos, si la política de sustitución de páginas utilizada es LRU. Suponga que se
        dicho algoritmo es de asignación variable y sustitución global.
    \end{enumerate}
\end{ejercicio}


\begin{ejercicio}
    Se tiene un sistema de memoria virtual con paginación a dos niveles que permite
agrupar las páginas en “directorios de páginas”. Cada tabla de páginas puede contener
hasta 1024 páginas. Los espacios de direcciones lógicas de este sistema son de
4Gbytes y el tamaño de página es de 4Kbytes. El espacio de direcciones físicas puede
tener hasta 1Gb. Describa la estructura de las direcciones lógicas y de las direcciones
físicas de este sistema de memoria virtual.
\end{ejercicio}


\begin{ejercicio}
    Suponga un sistema que utiliza paginación a dos niveles. Las direcciones son de 8 bits
con la siguiente estructura: 2 bits en la tabla de páginas de primer nivel, 2 bits en la tabla
de páginas de segundo nivel y 4 bits para el desplazamiento). El espacio de
direccionamiento virtual de un proceso tiene la estructura del dibujo. Represente
gráficamente las tablas de páginas y sus contenidos, suponiendo que cada entrada de la
tabla de páginas ocupa 8 bits y que todas las páginas están cargadas en memoria
principal (elige tú mismo la ubicación en memoria principal de dichas páginas,
suponiendo que la memoria principal es de 160 Bytes). Dada esa asignación traduce la
dirección virtual 47.
\begin{figure}[H]
    \centering
    \begin{tikzpicture}
        \def\ancho{4}
        \def\alto{0.7}

        % Definir un nuevo patrón
        \pgfdeclarepatternformonly{wide lines}{\pgfqpoint{-1pt}{-1pt}}{\pgfqpoint{10pt}{10pt}}{\pgfqpoint{9pt}{9pt}}%
        {
        \pgfsetlinewidth{0.4pt}
        \pgfsetstrokecolor{gray}
        \pgfpathmoveto{\pgfqpoint{0pt}{0pt}}
        \pgfpathlineto{\pgfqpoint{9pt}{9pt}}
        \pgfusepath{stroke}
        }

        % Zona sin usar
        \draw[pattern=wide lines] (0,-2*\alto) rectangle (\ancho, -6*\alto);


        \draw (0,0) -- (\ancho, 0) node[right] {0};
        \draw (0,-1*\alto) -- (\ancho, -1*\alto) node[right] {16 B};
        \draw (0,-2*\alto) -- (\ancho, -2*\alto) node[right] {48 B};
        \draw (0,-6*\alto) -- (\ancho, -6*\alto) node[right] {224 B};
        \draw (0,-7*\alto) -- (\ancho, -7*\alto) node[right] {256 B};

        \draw(0,0) -- (0, -7*\alto);
        \draw(\ancho,0) -- (\ancho, -7*\alto);


        % Escribe en cada casilla
        \node at (0.5*\ancho, -0.5*\alto) {Texto};
        \node at (0.5*\ancho, -1.5*\alto) {Datos};
        \node at (0.5*\ancho, -4*\alto) {\textbf{Sin usar}};
        \node at (0.5*\ancho, -6.5*\alto) {Pila};
    \end{tikzpicture}
\end{figure}
\end{ejercicio}


\begin{ejercicio}
    Considere la siguiente tabla de segmentos:
    \begin{table}[H]
        \centering
        \begin{tabular}{|c|c|c|c|}
            \hline
            Segmento & Presencia o validez & Dirección base & Longitud \\
            \hline
            0 & 0 & 219 & 600 \\
            1 & 1 & 2300 & 14 \\
            2 & 1 & 90 & 100 \\
            3 & 0 & 1327 & 580 \\
            4 & 1 & 1952 & 96 \\
            \hline
        \end{tabular}
        \caption{Tabla de segmentos}
    \end{table}
    ¿Qué direcciones físicas corresponden a las direcciones lógicas (nº segmento,
desplazamiento) siguientes? Si no puede traducir alguna dirección lógica a física, explique el
por qué.
\begin{enumerate}
    \item 0, 430
    
    Corresponde al segmento 0, pero como su bit de validez es 0, no se encuentra en memoria, por lo que no tiene dirección física.  Se produce una falta de segmento.
    \item 1, 10
    
    La dirección base es 2300, y como $10 < 14$, la dirección física es $2300 + 10 = 2310$.
    \item 3, 400
    
    Corresponde al segmento 3, pero como su bit de validez es 0, no se encuentra en memoria, por lo que no tiene dirección física. Se produce una falta de segmento.
    \item 4, 112
    
    La dirección base es 1952, y como $112 > 96$, se produce una excepción de acceso indebido.
\end{enumerate}
\end{ejercicio}


\begin{ejercicio}
    Respecto a la gestión de memoria que se hace en Linux: suponga que un proceso realiza
una llamada al sistema fork creando un proceso hijo. Represente gráficamente como
quedan las estructuras de datos relacionadas con ambos procesos.
\end{ejercicio}


\begin{ejercicio}
    ¿Qué información comparten un proceso y su hijo en un sistema Linux después de
ejecutar el siguiente código? Justifique su respuesta e indique qué hace este trozo de
código.
\begin{minted}{c}
    if ( fork() != 0 )
        wait (&status);
    else
        exec (B);
    // usamos una llamada al sistema exec genérica con un único
    // argumento, el nombre de un archivo ejecutable
\end{minted}
\end{ejercicio}
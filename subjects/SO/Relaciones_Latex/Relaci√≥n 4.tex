\section{Sistema de Archivos}

\begin{ejercicio}
Sea un Sistema Operativo que solo soporta un directorio (es decir, todos los archivos existentes estarán
al mismo nivel), pero permite que los nombres de archivos sean de longitud variable.
Apoyándonos únicamente en los servicios proporcionados por este Sistema Operativo, deseamos construir una 
``utilidad'' que ``simule'' un sistema jerárquico de archivos. ¿Es esto posible? ¿Cómo?
\end{ejercicio}

\begin{ejercicio}
En un entorno multiusuario, cada usuario tiene un directorio inicial al entrar en el sistema a
partir del cual puede crear archivos y subdirectorios. Surge, entonces, la necesidad de limitar el tamaño
de este directorio para impedir que el usuario consuma un espacio de disco excesivo.
¿De qué forma el Sistema Operativo podría implementar la limitación de tamaño de un directorio?
\end{ejercicio}

\begin{ejercicio} \label{ej:Rel4_Ej3}
En la siguiente figura se representa una tabla FAT. Al borde de sus entradas se ha escrito, como ayuda de referencia, el número correspondiente al bloque en cuestión. También se ha representado la entrada de cierto directorio. Como simplificación del ejemplo, suponemos que en cada entrada del directorio se almacena: Nombre de archivo/directorio, el tipo (F=archivo, D=directorio), la fecha de creación y el número del bloque inicial.

    \begin{figure}[H]
        \centering
        \begin{subfigure}[c]{0.5\textwidth}
            \centering
            \begin{tabular}{|c|c|c|c|}
                Nombre & Tipo & Fecha  & Nº Bloque \\ \hline \hline
                DATOS  & F    & 8-2-90 & 3         \\ \hline
                       &      &        &           \\ \hline
                       &      &        &           \\ \hline
                       &      &        &           \\ \hline
                       &      &        &           \\ \hline
            \end{tabular}
            \caption{Directorio}
        \end{subfigure}\hfill
        \begin{subfigure}[c]{0.5\textwidth}
            \centering
            \begin{tabular}{c|c|cc|c|}
                \cline{2-2} \cline{5-5}
                1 &  &\hspace{1cm}& 10 &  \\ \cline{2-2} \cline{5-5} 
                2 &  && 11 &  \\ \cline{2-2} \cline{5-5} 
                3 & 15 && 12 &  \\ \cline{2-2} \cline{5-5} 
                4 &  && 13 &  \\ \cline{2-2} \cline{5-5} 
                5 &  && 14 &  \\ \cline{2-2} \cline{5-5} 
                6 &  && 15 & $\ast$ \\ \cline{2-2} \cline{5-5} 
                7 &  && 16 &  \\ \cline{2-2} \cline{5-5} 
                8 &  && 17 &  \\ \cline{2-2} \cline{5-5} 
                9 &  && 18 &  \\ \cline{2-2} \cline{5-5} 
            \end{tabular}
            \caption{FAT}
        \end{subfigure}
    \end{figure}
    
    Tenga en cuenta que:
    \begin{itemize}
        \item El tamaño de bloque es de 512 bytes.
        \item El asterisco indica último bloque.
        \item Todo lo que está en blanco en la figura está libre.
    \end{itemize}

    Rellene la figura para representar lo siguiente:
    \begin{enumerate}
    \item Creación del archivo DATOS1 con fecha 1-3-90, y tamaño de 10 bytes.
    \item Creación del archivo DATOS2 con fecha 2-3-90, y tamaño 1200 bytes.
    \item El archivo DATOS aumenta de tamaño, necesitando 2 bloques más.
    \item Creación del directorio D, con fecha 3-3-90, y tamaño 1 bloque.
    \item Creación del archivo CARTAS con fecha 13-3-90 y tamaño 2 KBytes.
    \end{enumerate}

    \begin{figure}[H]
        \centering
        \begin{subfigure}[c]{0.5\textwidth}
            \centering
            \begin{tabular}{|c|c|c|c|}
                Nombre & Tipo & Fecha  & Nº Bloque \\ \hline \hline
                DATOS  & F    & 8-2-90 & 3         \\ \hline
                DATOS1 & F    & 1-3-90 & 1         \\ \hline
                DATOS2 & F    & 2-3-90 & 2         \\ \hline
                D      & D    & 3-3-90 & 8          \\ \hline
                CARTAS & F    & 13-3-90& 9         \\ \hline
            \end{tabular}
            \caption{Directorio}
        \end{subfigure}\hfill
        \begin{subfigure}[c]{0.5\textwidth}
            \centering
            \begin{tabular}{c|c|cc|c|}
                \cline{2-2} \cline{5-5}
                1 & $\ast$ &\hspace{1cm}& 10 & 11 \\ \cline{2-2} \cline{5-5} 
                2 & 4  && 11 & 12 \\ \cline{2-2} \cline{5-5} 
                3 & 15  && 12 & $\ast$ \\ \cline{2-2} \cline{5-5} 
                4 & 5 && 13 &  \\ \cline{2-2} \cline{5-5} 
                5 & $\ast$  && 14 &  \\ \cline{2-2} \cline{5-5} 
                6 & 7 && 15 & 6 \\ \cline{2-2} \cline{5-5} 
                7 & $\ast$ && 16 &  \\ \cline{2-2} \cline{5-5} 
                8 & $\ast$  && 17 &  \\ \cline{2-2} \cline{5-5} 
                9 & 10 && 18 &  \\ \cline{2-2} \cline{5-5} 
            \end{tabular}
            \caption{FAT}
        \end{subfigure}
        \label{fig:Rel4_Ej2}
    \end{figure}
\end{ejercicio}

\begin{ejercicio}
Si usamos un Mapa de Bits para la gestión del espacio libre, especifique la sucesión de bits que contendría
respecto a los 18 bloques del ejercicio anterior.

\begin{table}[H]
    \centering
    \begin{tabular}{ccccccccc}
        1 & 1 & 1 & 1 & 1 & 1 & 1 & 1 & 1 \\ \hline
        1 & 1 & 1 & 0 & 0 & 1 & 0 & 0 & 0
    \end{tabular}
    \caption{Mapa de bits de los bloques del Ejercicio \ref{ej:Rel4_Ej3}.}
\end{table}
\end{ejercicio}

\begin{ejercicio}
Si se pierde el primer puntero de la lista de espacio libre, ¿podría el Sistema Operativo reconstruirla? ¿Cómo?
\end{ejercicio}

\begin{ejercicio}
El espacio libre en un disco puede ser implementado usando una lista encadenada con agrupación o un mapa de bits.
La dirección en disco requiere D bits. Sea un disco con B bloques, en que F están libres.
¿En qué condición la lista usa menos espacio que el mapa de bits?\\

El mapa de bits ocupa $B$ bits, ya que contiene un bit por cada bloque.
La lista encadenada tan solo necesita almacenar un puntero, $D$ bits, ya que tan solo
se almacena la dirección del primer bloque libre; el resto de bloques libres se obtienen
a partir de este. Por tanto, la lista encadenada ocupa menos espacio que el mapa de bits
cuando $D < B$.

No obstante, hay que tener en cuenta que el acceso a la lista encadenada es más lento que
el acceso al mapa de bits, ya que para acceder al mapa de bits no es necesario
acceder a memoria.
\end{ejercicio}

\begin{ejercicio}
Entre los posibles atributos de un archivo, existe un bit que marca un archivo como temporal y por lo tanto está sujeto a destrucción automática cuando el proceso acaba ¿Cuál es la razón de esto? Después de todo un proceso siempre puede destruir sus archivos, si así lo decide.
\end{ejercicio}

\begin{ejercicio}
Algunos SO proporcionan una llamada al sistema (\verb|RENAME|) para dar un nombre nuevo a un archivo existente.
¿Existe alguna diferencia entre utilizar este mandato para renombrar un archivo y copiar el archivo a uno nuevo,
con el nuevo nombre y destruyendo el antiguo?\\

Sí, hay una gran diferencia. Al renombrar el archivo, tan solo se cambia el nombre del archivo en la entrada del directorio,
pero el puntero a su estructura de datos (i-nodo en el caso de Linux) no cambia. Por tanto, la estructura de datos que sostiene el archivo
no cambia, por lo que en esencia el archivo sigue siendo el mismo.
En cambio, si se copia el archivo a uno nuevo, con el nuevo nombre y destruyendo el antiguo, se crea una nueva entrada en el directorio,
se crea un nuevo i-nodo, y se copia el contenido del i-nodo antiguo al nuevo.
Por tanto, el archivo antiguo y el nuevo son dos archivos distintos y es mucho más costoso, ya que se copia todo el contenido del archivo.
\end{ejercicio}

\begin{ejercicio}[Ejercicio de Evaluación Continua DGIIM 2023-2024]
Un i-nodo de UNIX tiene 10 direcciones de disco para los diez primeros bloques de datos, y tres direcciones más para realizar una indexación a uno, dos y tres niveles respectivamente.
Si cada bloque índice tiene 256 direcciones de bloques de disco, ¿cuál es el tamaño del mayor archivo que puede ser manejado, suponiendo que 1 bloque de disco es de 1KByte?\\
    
    En total el número de bloques de disco que puede manejar el sistema de archivos es:
    \begin{equation*}
        10 + 256 + 256^2 + 256^3 = 10 + 2^{8} + 2^{16} + 2^{24} = 16843018
    \end{equation*}

    Por tanto, el tamaño máximo de archivo que puede manejar el sistema de archivos es:
    \begin{equation*}
        16843018 \text{ bloques} \cdot \frac{1 \text{ KiB}}{1 \text{ bloque}} = 16843018 \text{ KiB} = 16.062 \text{ GiB}
    \end{equation*}
\end{ejercicio}

\begin{ejercicio}
Sobre conversión de direcciones lógicas dentro de un archivo a direcciones físicas de disco.
Estamos utilizando la estrategia de indexación a tres niveles para asignar espacio en disco.
Tenemos que el tamaño de bloque de índices es igual a 512 bytes, y el tamaño de puntero es de 4 bytes.
Se recibe la solicitud por parte de un proceso de usuario de leer el carácter número N de determinado archivo.
Suponemos que ya hemos leído la entrada del directorio asociada a este archivo, es decir,
tenemos en memoria los datos PRIMER-BLOQUE y TAMAÑO.

Calcule la sucesión de direcciones de bloque que se leen hasta llegar al bloque de datos que posee
el citado carácter.
\begin{comment}
En primer lugar, necesitamos calcular en qué bloque se encuentra. Como el tamaño de bloque de índices es de 512 bytes, y suponemos
que el tamaño de bloque de índices es el mismo que el tamaño de bloque de datos,
entonces el bloque en el que se encuentra, sabiendo que un carácter ocupa 1 B, es:
\begin{equation*}
    \left\lfloor \frac{N}{512} \right\rfloor = \left\lfloor \frac{N}{2^9} \right\rfloor
\end{equation*}

Por tanto, necesitamos acceder al bloque de índices que contiene la dirección del bloque de datos en el que se encuentra el carácter.
Como el tamaño de puntero es de 4 bytes, entonces en cada bloque de índices hay:
\begin{equation*}
    \frac{512 \text{ B}}{4 \text{ B}} = 128=2^7 \text{ punteros}
\end{equation*}

Por tanto, el bloque de índices que contiene la dirección del bloque de datos en el que se encuentra el carácter es:
\begin{equation*}
    \left\lfloor \frac{\left\lfloor \frac{N}{2^9} \right\rfloor}{2^7} \right\rfloor = \left\lfloor \frac{N}{2^{16}} \right\rfloor
\end{equation*}
\end{comment}
\end{ejercicio}

\begin{ejercicio}
¿Qué método de asignación de espacio en un sistema de archivos elegiría para maximizar la eficiencia en términos de velocidad de acceso, uso del espacio de almacenamiento y facilidad de modificación (añadir/borrar /modificar), cuando los datos son:
\begin{enumerate}
    \item Modificados infrecuentemente, y accedidos frecuentemente de forma aleatoria.
    
    En este caso al ser accedidos de forma aleatoria, descartamos el no contiguo enlazado, ya que
    este es más eficiente cuando los datos son accedidos de forma secuencial. Por tanto, nos quedamos
    con el contiguo y el no contiguo indexado.

    Entre estos, es cierto que el contiguo puede producir fragmentación externa y los archivos no pueden crecer,
    pero como se modifican infrecuentemente, se supone que no habrá problemas, ya que el tamaño no variará prácticamente.
    Por tanto, se opta por el contiguo, ya que es más eficiente en términos de velocidad de acceso.


    \item Modificados con frecuencia, y accedidos en su totalidad con cierta frecuencia.
    
    En este caso, como es modificado con frecuencia, descartamos el contiguo, ya que este no permite crecer el archivo
    y, al ser modificado con frecuencia, se supone que crecerá. Así, al evitar este método, se evita la fragmentación externa también.

    Entre los otros dos, depende de si los accesos se realizarán de forma secuencial o aleatoria o directa.
    Como se menciona que se accede en su totalidad, se supone que se accede de forma secuencial, por lo que se opta
    por el no contiguo enlazado, ya que requerirá menos accesos a memoria (no se emplea el bloque índice).

    \item Modificados frecuentemente y accedidos aleatoriamente y frecuentemente.
    
    En este caso, descartamos el contiguo, ya que no permite crecer el archivo y, al ser modificado con frecuencia,
    se supone que crecerá. Así, al evitar este método, se evita la fragmentación externa también.
    
    Entre los otros dos, como se accede de forma aleatoria, se opta por el no contiguo indexado, ya que
    permite acceder de forma aleatoria, mientras que el no contiguo enlazado no.
\end{enumerate}
\end{ejercicio}

\begin{ejercicio}
¿Cuál es el tamaño que ocupan todas las entradas de una tabla FAT32 que son necesarias para referenciar los cluster de datos,
cuyo tamaño es de 16 KB, de una partición de 20 GB ocupada exclusivamente por la propia FAT32
y dichos cluster de datos?

\begin{observacion}
    Entendemos que ``ocupada exclusivamente por la propia FAT32
    y dichos cluster de datos'' se refiere a que no hay más datos en la partición, 
    pero sobreentendemos que el tamaño de la FAT32 no se incluye en los 20 GB, sino
    que se contabiliza aparte. De esta forma, los 20 GB son ocupados exclusivamente por
    los cluster de datos.
\end{observacion}

Veamos en primer lugar cuántos clusters de datos hay en la partición:
\begin{equation*}
    20 \text{ GB} \cdot \frac{2^{30} \text{ B}}{1 \text{ GiB}} \cdot \frac{1 \text{ cluster}}{2^{14} \text{ B}} = 1310720 \text{ clusters}
\end{equation*}

Por tanto, hay $1310720$ entradas en la tabla FAT32. Como cada entrada es de 4 bytes (32 bits), entonces
el tamaño total de la tabla FAT32 es:
\begin{equation*}
    1310720 \text{ entradas} \cdot \frac{4 \text{ B}}{1 \text{ entrada}} = 5242880 \text{ B} = 5 \text{ MiB}
\end{equation*}

\end{ejercicio}

\begin{ejercicio}
Cuando en un sistema Unix/Linux se abre el archivo \verb|/usr/ast/work/f|,
se necesitan varios accesos a disco. Calcule el número de accesos a disco requeridos
(como máximo) bajo la suposición de que el i-nodo raíz ya se encuentra en memoria y que
todos los directorios necesitan como máximo 1 bloque para almacenar los datos de sus archivos.\\

En primer lugar, como el i-nodo raíz ya se encuentra en memoria, entonces no es necesario
acceder a disco para obtenerlo. Este i-nodo nos informará de la dirección del bloque
de datos del directorio raíz. Por tanto, el primer $(\ast)$ acceso a disco será
para obtener el bloque de datos del directorio raíz.

En este bloque de datos,
se encuentra la entrada del directorio \verb|usr|, que contiene el número de i-nodo
del directorio \verb|usr|. Por tanto, el segundo $(\ast)$ acceso a disco será para
obtener el i-nodo del directorio \verb|usr|. Este i-nodo nos informará de la dirección
del bloque de datos del directorio \verb|usr|, y el tercer $(\ast)$ acceso será para
obtener dicho bloque de datos. Es decir, para cada carpeta es necesario
dos accesos a disco: uno para obtener el i-nodo y otro para obtener el bloque de datos.

Por tanto, el cuarto $(\ast)$ y quinto $(\ast)$ acceso a disco serán para obtener el i-nodo
y el bloque de datos del directorio \verb|ast|, respectivamente. El sexto $(\ast)$ y séptimo $(\ast)$
acceso a disco serán para obtener el i-nodo y el bloque de datos del directorio \verb|work|,
respectivamente.

Por última vez, una vez ya se tenga el bloque de datos del directorio \verb|work|,
se buscará la entrada del archivo \verb|f|, que contiene el número de i-nodo del archivo \verb|f|.
Por tanto, el octavo $(\ast)$ acceso a disco será para obtener el i-nodo del archivo \verb|f|.
Este i-nodo nos informará de las direcciones de los bloques de datos del archivo \verb|f|.\\

En total, se necesitan $8$ accesos a disco para obtener el i-nodo del archivo \verb|f|.
Para abrir dicho archivo, además se necesitará acceder tantas veces como bloques de datos
tenga el archivo \verb|f|, para obtener cada uno de los bloques de datos del archivo \verb|f|, pero
estos no los contabilizamos, ya que no conocemos dicho número.
\end{ejercicio}

\begin{ejercicio}
Suponiendo una ejecución correcta de las siguientes órdenes en el sistema operativo Linux:
\begin{verbatim}
/home/jgarcia/prog
$ ls -i  //lista los archivos y sus números de i-nodos del directorio prog
18020 fich1.c
18071 fich2.c
18001 pract1.c

/home/jgarcia/prog > cd ../tmp
/home/jgarcia/tmp > ln -s ../prog/pract1.c p1.c  //crea enlace simbólico
/home/jgarcia/tmp > ln ../prog/pract1.c p2.c     //crea enlace absoluto
\end{verbatim}
represente gráficamente cómo y dónde quedaría reflejada y almacenada toda la información referente a la creación anterior de un enlace simbólico y absoluto (``hard'') a un mismo archivo, \verb|pract1.c|.\\

    Al crear el enlace simbólico, se crea un nuevo i-nodo, que contiene la información del enlace simbólico,
    y se crea una nueva entrada en el directorio \verb|/home/jgarcia/tmp|, que contiene el nombre del enlace simbólico
    y el número de i-nodo del enlace simbólico. Ese i-nodo contendrá la información del enlace simbólico, que es
    el nombre del archivo al que apunta.\\

    Al crear el enlace absoluto, se crea una nueva entrada en el directorio \verb|/home/jgarcia/tmp|, que contiene
    el nombre del enlace absoluto y el número de i-nodo del archivo al que apunta, es decir, el número de i-nodo $18001$.
    Por tanto, el archivo \verb|p2.c| y el archivo \verb|pract1.c| apuntan al mismo i-nodo, y por tanto al mismo archivo.
    No se crea ningún nuevo i-nodo.
\end{ejercicio}

\begin{ejercicio}
En un sistema de archivos ext2 (Linux), ¿qué espacio total (en bytes) se
requiere para almacenar la información sobre la localización física de un archivo
que ocupa 3 Mbytes? Suponga que el tamaño de un bloque lógico es de 1 Kbytes y se utilizan
direcciones de 4 bytes. Justifique la solución detalladamente.\\

    Veamos cuántos bloques lógicos ocupa el archivo:
    \begin{equation*}
        3\cdot 2^{20} \text{ B} \cdot \frac{1 \text{ bloque}}{2^{10} \text{ B}} = 3072 \text{ bloques}
    \end{equation*}

    Por tanto, ya sabemos que las 12 primeras direcciones directas a bloque de datos se emplean.
    Veamos además cuántos bloques se pueden direccionar con un bloque índice. Como
    cada bloque índice tiene 1 KiB, y cada dirección de bloque es de 4 bytes, entonces un bloque índice contiene:
    \begin{equation*}
        2^{10} \text{ B} \cdot \frac{1 \text{ dirección}}{4 \text{ B}}= 2^8=256 \text{ direcciones}
    \end{equation*}
    Por tanto, en el primer nivel de indexación se pueden direccionar $256$ bloques de datos,
    en el segundo nivel $256^2$ bloques de datos, y en el tercer nivel $256^3$ bloques de datos.

    Como $12+256<3072$ pero $12+256+256^2>3072$, entonces sabemos que el archivo ocupa todos los bloques
    que se direccionan de forma directa y el primer nivel de indexación, mientras que el segundo nivel
    no se emplea entero. Por tanto, para almacenar la información sobre la localización física del archivo
    se necesitan:
    \begin{itemize}
        \item 12 direcciones de bloque directas.
        \item 1 dirección de bloque de índice, que contiene 256 direcciones de bloque.
        \item 1 dirección de bloque de índice,
                que contiene 256 direcciones de bloque índice,
                y cada uno de estos contiene 256 direcciones de bloque.
    \end{itemize}

    Por tanto, el número total de direcciones de bloque necesarias es:
    \begin{equation*}
        12 + 1 + 256 + 1 + 256+ 256^2 = 66062 \text{ direcciones}
    \end{equation*}

    Como cada dirección de bloque es de 4 bytes, entonces el espacio total requerido es:
    \begin{equation*}
        66062 \text{ direcciones} \cdot \frac{4 \text{ B}}{1 \text{ dirección}} = 264248 \text{ B} \approx 258.05 \text{ KiB}
    \end{equation*}
\end{ejercicio}

\begin{ejercicio}
En la mayoría de los sistemas operativos, el modelo para manejar un archivo es el siguiente:
\begin{itemize}
    \item Abrir el archivo, que nos devuelve un descriptor de archivo asociado a él.
    \item Acceder al archivo a través de ese descriptor devuelto por el sistema.
\end{itemize}
¿Cuáles son las razones de hacerlo así? ¿Por qué no, por ejemplo, se especifica el archivo a manipular en cada operación que se realice sobre él?
\end{ejercicio}

\begin{ejercicio}
Sea un directorio cualquiera en un sistema de archivos ext2 de Linux, por ejemplo, DirB. De él cuelgan algunos archivos que están en uso por uno o más procesos. ¿Es posible usar este directorio como punto de montaje? Justifíquelo.
\end{ejercicio}
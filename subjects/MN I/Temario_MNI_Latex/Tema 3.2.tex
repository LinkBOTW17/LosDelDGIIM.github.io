\chapter{Splines}
\begin{definicion}
    Dado un intervalo $[a, b]$, una partición de $n+1$ puntos de $[a, b]$ será un conjunto
    de la forma $\{x_0, x_1, \ldots, x_n\}$ tal que $x_i < x_{i+1}~~\forall i \in \{0, \ldots, n-1\}$,

    Diremos que esta partición es regular si $x_{i+1} - x_i = c~~\forall i \in \{0, \ldots, n-1\}$.

    A cada elemento $x_i$ de una partición le llamaremos nodo del spline.
\end{definicion}

Entre cada nodo daremos un polinomio de grado $m$ y con la unión de cada polinomio $p_i$ entre los nodos
$x_i$ y $x_{i+1}$ obtendremos el spline deseado $\forall i \in \{0, \ldots, n~-~1\}$.

\bigskip
\begin{definicion}[Spline]
    Sea $a=x_0<x_1 < \ldots < x_n = b$ una partición del intervalo $[a,b]$. Un spline de grado
    $m$ relativo a dicha partición es una función $s(x)$ que verifica:
    \begin{enumerate}
        \item $s(x) = p_i(x) \in \bb{P}_m~~x \in [x_i, x_{i+1}]~~~~\forall i \in \{0, \ldots, n-1\}$

        \item $s\in C^{m-1}([a,b])$
    \end{enumerate}

    La segunda condición también se puede expresar de la siguiente forma:
    \begin{equation*}
        p_i^{(k)}(x_{i+1}) = p_{i+1}^{(k)}(x_{i+1})~~\forall i \in \{1, \ldots, n-1\}~~\forall k \in
        \{0, \ldots, m-1\}
    \end{equation*}
\end{definicion}

\noindent
Por tanto, para dar un spline es necesario dar los nodos $x_i~~\forall i \in \{0, \ldots, n\}$ y
los polinomios $p_j(x)~~\forall j \in \{0, \ldots, n-1\}$

\begin{notacion}
    Notaremos por $S_m(x_0, \ldots, x_n)$ al conjunto de los splines de grado $m$ con nodos $\{x_0,
    \ldots, n_n\}$.
\end{notacion}

\begin{ejemplo} Veamos los siguientes ejemplos para distintos valores de $m$:

    \textbf{$m=1$, Splines lineales}
$$s \in S_1(x_0, \ldots, x_n)$$
$$s(x) = \left\{\begin{array}{ll}
        p_0(x) \in \bb{P}_1     & x \in [x_0, x_1[     \\
        \vdots                                     \\
        p_{n-1}(x) \in \bb{P}_1 & x \in [x_{n-1}, x_n]
    \end{array}\right.$$
$$s \in C^0([x_0, x_n])$$

\textbf{$m=2$, Splines cuadráticos}
$$s \in S_2(x_0, \ldots, x_n)$$
$$s(x) = \left\{\begin{array}{ll}
        p_0(x) \in \bb{P}_2     & x \in [x_0, x_1[     \\
        \vdots                                     \\
        p_{n-1}(x) \in \bb{P}_2 & x \in [x_{n-1}, x_n]
    \end{array}\right.$$
$$s \in C^1([x_0, x_n])$$

\textbf{$m=3$, Splines cúbicos}
$$s \in S_3(x_0, \ldots, x_n)$$
$$s(x) = \left\{\begin{array}{ll}
        p_0(x) \in \bb{P}_3     & x \in [x_0, x_1[     \\
        \vdots                                     \\
        p_{n-1}(x) \in \bb{P}_3 & x \in [x_{n-1}, x_n]
    \end{array}\right.$$
$$s \in C^2([x_0, x_n])$$
\end{ejemplo}

\bigskip
\begin{teo}
    $S_m(x_0, \ldots, x_n)$ es un espacio vectorial de dimensión $m+n$
\end{teo}
\begin{proof}
    $$S_m(x_0, \ldots, x_n) \subset \mathcal{C}^{m-1}([x_0,x_n])$$

    $\forall s_1, s_2 \in S_m(x_0, \ldots, x_n) \Rightarrow s_1 + s_2 \in S_m(x_0, \ldots, x_n)$\par
    $\forall \lambda \in \R~\forall s \in S_m(x_0, \ldots, x_n) \Rightarrow \lambda s \in S_m(x_0,
        \ldots, x_n)$\par
    $0 \in S_m(x_0, \ldots, x_n)$

    \noindent
    Queda demostrado que $S_m(x_0, \ldots, x_n)$ es un espacio vectorial. Probemos su dimensión:

    $$\forall s \in S_m(x_0, \ldots, x_n)$$
    $$s(x) = \left\{ \begin{array}{ll}
            p_0(x)     & x \in [x_0, x_1[     \\
            \vdots                            \\
            p_{n-1}(x) & x \in [x_{n-1}, x_n]
        \end{array}\right.$$
    $$p_i(x) \in \bb{P}_m \Rightarrow \mbox{ tiene } m + 1 \mbox{ coeficientes}$$
    Por lo que $s$ tiene $n(m+1)$ coeficientes.\\

    \noindent
    Además, presenta $m$ restricciones en cada nodo interior, por lo que presenta $m(n-1)$
    restricciones que además son linealmente independientes (su comprobación no es de nuestro
    interés).\\

    \noindent
    Por tanto, $S_m(x_0, \ldots, x_n)$ es un subespacio de un espacio vectorial de dimensión $n(m+1)$
    y presenta $m(n-1)$ ecuaciones linealmente independientes, por lo que $S_m(x_0, \ldots, x_n)$
    es un espacio vectorial de dimensión $n(m+1)-m(n-1) = nm + n - nm + m = n + m$
\end{proof}

\begin{ejemplo}
    Consideramos el espacio $S_1(x_0, x_1, x_2)$

Sea $s \in S_1(x_0, x_1, x_2)$:

$$s(x) = \left\{\begin{array}{ll}
        p_0(x) & x \in [x_0, x_1[ \\
        p_1(x) & x \in [x_1, x_2]
    \end{array}\right.$$

$$\begin{array}{l}
        p_0(x) = a_0 + b_0x \\
        p_1(x) = a_1 + b_1x
    \end{array}$$

$$p_0(x_1)=p_1(x_1) \Leftrightarrow a_0 + b_0x = a_1 + b_1x$$

\noindent
Necesita 4 coeficientes y presenta 1 restricción $\Rightarrow \dim S_1(x_0, x_1, x_2) = 4-1=3$
\end{ejemplo}

\begin{ejemplo}
    Dados $(x_i, y_i)_{i = 0, \ldots, n}$, se desea construir un spline lineal $s \in S_1(x_0, \ldots, x_n)$
que interpole a los $n+1$ puntos dados: $s(x_i)=y_i~\forall i \in \{0, \ldots, n\}$.\\

\noindent
Como $\dim S_1(x_0, \ldots, x_n) = n+1$ y tenemos $n+1$ restricciones, el problema tiene una única solución.
\end{ejemplo}

\begin{ejemplo}
    Dados $(x_i, y_i)_{i=0, \ldots, n}$, se desea construir un spline cuadrático $s \in S_2(x_0, \ldots, x_n)$
que interpole a los $n+1$ punteos dados: $s(x_i)=y_i~\forall i \in \{0, \ldots, n\}$.\\

\noindent
Como $\dim S_2(x_0, \ldots, x_n)=n+2$ y tenemos $n+1$ restricciones, el problema tiene $\infty$ soluciones.
Para que tenga una única solución, se añade un dato más: $s'(x_0)=d_0$, por ejemplo.
\end{ejemplo}

\begin{ejemplo}
    Calcula $s\in S_2(-1, 0, 1)$ tal que:
    \begin{gather*}
        s(-1)=1 \qquad s(0)=0 \qquad s(1)=2 \\
        s'(-1)=0
    \end{gather*}

    Tenemos que:
    \begin{equation*}
        s(x)=\left\{\begin{array}{ccl}
            p_0(x) & \text{si} & x\in [-1, 0[ \\
            p_{1}(x) & \text{si} & x\in [0,1]
        \end{array} \right.
    \end{equation*}

    Para calcular $p_0(x)\in \bb{P}_2$, podemos optar por el método de coeficientes indeterminados o por el método de Hermite. Optamos por el método de Hermite.
    \begin{equation*}
        \begin{array}{c|ccccc}
            -1 & 1 \\
            && p_0'(-1)=0 \\ 
            -1 & 1 && -1\\
            && -1\\
            0 & 0
        \end{array}
    \end{equation*}

    Por tanto, $p_0(x)=1+0(x+1)-(x+1)^2 = 1-(x+1)^2$.

    Para calcular $p_1(x)$, uso la condición de que $p_0'(0)=p_1'(0)$.
    \begin{equation*}
        p_0'(x) = -2(x+1) \qquad p'_0(0)=p'_1(0)=-2
    \end{equation*}

    Por tanto, calculo ahora $p_1(x)\in \bb{P}_2$:
    \begin{equation*}
        \begin{array}{c|ccccc}
            0 & 0 \\
            && p_1'(0)=-2 \\ 
            0 & 0 && 4\\
            && 2\\
            1 & 2
        \end{array}
    \end{equation*}

    Por tanto, $p_1(x)=-2x+4x^2$. En conclusión,
    \begin{equation*}
        s(x)=\left\{\begin{array}{ccl}
            1-(x+1)^2 & \text{si} & x\in [-1, 0[ \\
            -2x+4x^2 & \text{si} & x\in [0,1]
        \end{array} \right.
    \end{equation*}
\end{ejemplo}



\begin{ejemplo}
    Calcula $s\in S_2(-1, 0, 1)$ tal que:
    \begin{gather*}
        s(-1)=1 \qquad s(0)=0 \qquad s(1)=2 \\
    \end{gather*}

    Tenemos que:
    \begin{equation*}
        s(x)=\left\{\begin{array}{ccl}
            p_0(x) & \text{si} & x\in [-1, 0[ \\
            p_{1}(x) & \text{si} & x\in [0,1]
        \end{array} \right.
    \end{equation*}

    Como no se proporciona el valor de la derivada en ningún punto, se elige arbitrariamente. Para simplificar cálculos, \textbf{aunque se podría elegir cualquier valor real}, se suele optar por la pendiente que une los dos primeros puntos. Por tanto,
    \begin{equation*}
        s'(-1) := \frac{s(0)-s(-1)}{0-(-1)}= -1
    \end{equation*}

    Para calcular $p_0(x)\in \bb{P}_2$, podemos optar por el método de coeficientes indeterminados o por el método de Hermite. Optamos por el método de Hermite.
    \begin{equation*}
        \begin{array}{c|ccccc}
            -1 & 1 \\
            && p_0'(-1)=-1 \\ 
            -1 & 1 && 0\\
            && -1\\
            0 & 0
        \end{array}
    \end{equation*}

    Por tanto, $p_0(x)=1-(x+1)=-x$.

    Para calcular $p_1(x)$, uso la condición de que $p_0'(0)=p_1'(0)=-1$, por $p_0(x)$ ser una recta y así tener la misma derivada en todo el intervalo.

    Por tanto, calculo ahora $p_1(x)\in \bb{P}_2$:
    \begin{equation*}
        \begin{array}{c|ccccc}
            0 & 0 \\
            && p_1'(0)=-1 \\ 
            0 & 0 && 3\\
            && 2\\
            1 & 2
        \end{array}
    \end{equation*}

    Por tanto, $p_1(x)=-x+3x^2$. En conclusión,
    \begin{equation*}
        s(x)=\left\{\begin{array}{ccl}
            -x & \text{si} & x\in [-1, 0[ \\
            -x+3x^2 & \text{si} & x\in [0,1]
        \end{array} \right.
    \end{equation*}
\end{ejemplo}

\section{Unisolvencia}
\noindent
Para que las interpolaciones buscadas presenten una única solución, se dan diversas formas
de añadir datos a los problemas:

\subsection{Splines lineales}
\noindent
Para interpolar una colección de $n+1$ puntos $(x_i, y_i)_{i=0, \ldots, n}$, nos será sufciciente
con dar directamente el spline $s \in S_1(x_0, \ldots, x_n)$, ya que el problema presentará una única
solución, al ser dicho espacio de dimensión $n+1$ y presentar $n+1$ restricciones.

\subsection{Splines cuadráticos}
\noindent
Para interpolar una colección de $n+1$ puntos $(x_i, y_i)_{i=0, \ldots, n}$, no podremos dar directamente
un spline $s \in S_2(x_0, \ldots, x_n)$, ya que estamos trabajando en un espacio vectorial de dimensión
$n+2$ con $n+1$ restricciones, luego tendremos $\infty$ soluciones. Para que el problema sea
unisolvente, es necesario añadir un dato más que, generalmente, consisitirá en que el primer spline
tenga que ser la recta que une los dos primeros nodos.

\subsection{Splines cúbicos}
\noindent
Para interpolar una colección de $n+1$ puntos $(x_i, y_i)_{i=0, \ldots, n}$, no podremos dar directamente
un spline $s \in S_3(x_0, \ldots, x_n)$, ya que estamos trabajando en un espacio vectorial de dimensión
$n+3$ con $n+1$ restricciones, luego tendremos $\infty$ soluciones. Para que el problema sea
unisolvente, es necesario aportar 2 datos más, que podremos elegir aleatoriamente.\\

\noindent
Las formas más comunes de elegir estas dos nuevas restricciones son las siguientes:

\begin{itemize}
    \item \textbf{Splines ligados:} $s'(a) = \alpha~~s'(b)=\beta$
          Se fijan las pendientes en los puntos $x_0$ y $x_n$.

    \item \textbf{Spline natural:} $s''(a) = s''(b) = 0$
          Se hace que los splines exteriores continuen de forma recta.

    \item \textbf{Spline periódico:} $s'(a) = s'(b)~~s''(a)=s''(b)$
          Para que podamos ir repitiendo los splines sucesivamente, útil para la
          interpolación de funciones periódicas.
\end{itemize}

\section{Cálculo del spline cúbico}

\bigskip
\noindent
Para calcular un spline cúbico es necesario conocer $n$ polinomios de grado 3:
$$s(x) = \left\{ \begin{array}{ll}
        p_0(x)     & x \in [x_0, x_1]     \\
        p_1(x)     & x \in [x_1, x_2]     \\
        \vdots     & \vdots               \\
        p_{n-1}(x) & x \in [x_{n-1}, x_n]
    \end{array} \right.$$

\noindent
Dado el intervalo $[x_i, x_{i+1}]$, $p_i(x)$ sería el polinomio de interpolación de Hermite para
los datos:
$$\begin{array}{cc}
        p_i(x_i) = f_i  & p_i(x_{i+1}) = f_{i+1}  \\
        p_i'(x_i) = d_i & p_i'(x_{i+1}) = d_{i+1}
    \end{array}$$

\bigskip
\noindent
Por tanto, hemos de conocer las $d_0, d_1, \ldots, d_n$ derivadas para poder dar el spline cúbico.
A continuación, buscaremos una forma de calcular derivadas y poder hayar
el spline cúbico:\\

\noindent
Supongamos que conocemos las derivadas $d_0, d_1, \ldots, d_n$, definimos:\\

\textbf{Longitud del intervalo}
$$h_i = x_{i+1}-x_i~~~~i \in \{0, 1, \ldots, n-1\}$$

\textbf{Pendiente del intervalo}
$$\Delta_i = \dfrac{f_{i+1}-f_i}{x_{i+1}-x_i} = \dfrac{f_{i+1}-f_i}{h_i}~~~~i \in \{0, 1, \ldots, n-1\}$$

\noindent
Por tanto, podemos calcular el polinomio $p_i(x)$, $i \in \{0, 1, \ldots, n-1\}$ mediante la siguiente tabla:
$$\begin{array}{c|cccc}
        x_i     & f_i     &          &                               &                                        \\
                &         & d_i      &                               &                                        \\
        x_i     & f_i     &          & \dfrac{\Delta_i-d_i}{h_i}     &                                        \\
                &         & \Delta_i &                               & \dfrac{(d_{i+1}+d_i)-2\Delta_i}{h_i^2} \\
        x_{i+1} & f_{i+1} &          & \dfrac{d_{i+1}-\Delta_i}{h_i} &                                        \\
                &         & d_{i+1}  &                               &                                        \\
        x_{i+1} & f_{i+1} &          &                               &
    \end{array}$$
Luego:
$$p_i(x) = f_i + d_i(x-x_i) + \dfrac{\Delta_i - d_i}{h_i}(x-x_i)^2 +
    \dfrac{(d_{i+1}+d_i)-2\Delta_i}{h_i^2}(x-x_i)^2 (x-x_{i+1})$$

\bigskip
\noindent
Por la construcción que hemos hecho, tenemos garantizado que $s(x)$ es de clase $C^1([a,b])$.\newline
Para obtener la clase $C^2([a,b])$, imponemos que:
$$p_i''(x_{i+1}) = p_{i+1}''(x_{i+1})$$
$$p_i''(x) = 2\dfrac{\Delta_i-d_i}{h_i} + \dfrac{(d_{i+1}+d_i)-2\Delta_i}{h_i^2} \left[2(x-x_{i+1}) +
    4(x-x_i)(x-x_{i+1})\right]$$

$$p_i''(x_{i+1}) = 2\dfrac{d_i+2d_{i+1}-3\Delta_i}{h_i}$$
$$p_{i+1}''(x_{i+1}) = -2 \dfrac{2d_{i+1}+d_{i+2}-3\Delta_{i+1}}{h_{i+1}}$$

\noindent
Igualando $p_i''(x_{i+1}) = p_{i+1}''(x_{i+1})$, obtenemos que $(i \in \{0,1,\ldots, n-2\})$:
$$\dfrac{d_i}{h_i} + 2\left(\dfrac{1}{h_i}+\dfrac{1}{h_{i+1}}\right) d_{i+1} + \dfrac{d_{i+2}}{h_{i+1}}
    = 3 \left( \dfrac{\Delta_i}{h_i} + \dfrac{\Delta_{i+1}}{h_{i+1}} \right)$$

\noindent
Que es un sistema de $n-1$ ecuaciones y $n+1$ incógnitas, por lo que para obtener la unisolvencia
del spline, nos faltarían imponer dos condiciones más.

\subsection{Splines naturales}
\noindent
Si $s(x)$ es un spline cúbico natural:
$$\left\{ \begin{array}{ll}
        s''(x_0)=0 \Rightarrow & 2\dfrac{d_0}{h_0}+\dfrac{d_1}{h_0} = 3\dfrac{\Delta_0}{h_0}                       \\
        \                                                                                                          \\
        s''(x_n)=0 \Rightarrow & 2\dfrac{d_{n-1}}{h_{n-1}}+\dfrac{d_{n}}{h_{n-1}} = 3\dfrac{\Delta_{n-1}}{h_{n-1}}
    \end{array} \right.$$

\noindent
Tendríamos que el sistema de ecuaciones lineales anterior (junto con la primera y la última ecuación
que acabamos de dar) sería de forma matricial:

$$A=\left( \begin{array}{ccccc}
            \dfrac{2}{h_0} & \dfrac{1}{h_0}                                  &                    &                                                         &                    \\
            \dfrac{1}{h_0} & 2\left( \dfrac{1}{h_0} + \dfrac{1}{h_1} \right) & \dfrac{1}{h_1}     &                                                         &                    \\
                           & \ddots                                          & \ddots             & \ddots                                                  &                    \\
                           &                                                 & \dfrac{1}{h_{n-2}} & 2\left( \dfrac{1}{h_{n-2}} + \dfrac{1}{h_{n-1}} \right) & \dfrac{1}{h_{n-1}} \\
                           &                                                 &                    & \dfrac{1}{h_{n-1}}                                      & \dfrac{2}{h_{n-1}}
        \end{array} \right)$$

$$x=\left( \begin{array}{c}
            d_0     \\
            d_1     \\
            \vdots  \\
            d_{n-1} \\
            d_{n}
        \end{array} \right)$$

$$b=3\left( \begin{array}{c}
            \dfrac{\Delta_0}{h_0}                                         \\
            \dfrac{\Delta_0}{h_0} + \dfrac{\Delta_1}{h_1}                 \\
            \vdots                                                        \\
            \dfrac{\Delta_{n-2}}{h_{n-2}} + \dfrac{\Delta_{n-1}}{h_{n-1}} \\
            \dfrac{\Delta_{n-1}}{h_{n-1}}
        \end{array} \right)$$

$$Ax=b$$

\noindent
Dado que la matriz $A$ de coeficientes es tridiagonal y estrictamente diagonal dominante, podemos
afirmar qu el sistema tiene una única solución.

\bigskip
\noindent
\textbf{Ejemplo.}
Dar un Spline cúbico natural $s \in S_3(-1, 0, 1)$ que interpola a los datos:
$$\begin{array}{c|ccc}
        x_i & -1 & 0 & 1 \\
        \hline
        f_i & -1 & 0 & 1
    \end{array}$$

\bigskip
\noindent
Buscamos el valor de las derivadas, calculando las lontitudes y pendientes de cada intervalo:
$$h_i = x_{i+1}-x_i \Rightarrow h_0,h_1=1$$
$$\Delta_i = \dfrac{f_{i+1}-f_i}{h_i} \Rightarrow \Delta_0,\Delta1 = 1$$

\noindent
Resolvemos el siguiente sistema (dado por la fórmula vista en teoría):
$$\left( \begin{array}{ccc}
            2 & 1 & 0 \\
            1 & 4 & 1 \\
            0 & 1 & 2
        \end{array} \right) \left( \begin{array}{c}
            d_0 \\
            d_1 \\
            d_2
        \end{array} \right) = \left( \begin{array}{c}
            3 \\
            6 \\
            3
        \end{array} \right) \Rightarrow \left\{ \begin{array}{c}
        d_0 = 1 \\
        d_1 = 1 \\
        d_2 = 1
    \end{array} \right.$$

\noindent
Por lo que ahora tenemos los datos:
$$\begin{array}{c|ccc}
        x_i & -1 & 0 & 1 \\
        \hline
        f_i & -1 & 0 & 1 \\
        \hline
        d_i & 1  & 1 & 1
    \end{array}$$

\noindent
Con los que podemos calcular los polinomios $p_0(x), p_1(x) \in \bb{P}_3$ que forman:
$$s(x) = \left\{ \begin{array}{ll}
        p_0(x) & x \in [-1, 0] \\
        p_1(x) & x \in [0,1]
    \end{array} \right.$$

\noindent
Calculamos $p_0(x)$ en el intervalo $[-1,0]$:
$$\begin{array}{c|cccc}
        -1 & -1 &   &   &   \\
           &    & 1 &   &   \\
        -1 & -1 &   & 0 &   \\
           &    & 1 &   & 0 \\
        0  & 0  &   & 0 &   \\
           &    & 1 &   &   \\
        0  & 0  &   &   &
    \end{array}$$
$$p_0(x) = x$$

\bigskip
\noindent
Calculamos $p_1(x)$ en el intervalo $[0,1]$:
$$\begin{array}{c|cccc}
        0 & 0 &   &   &   \\
          &   & 1 &   &   \\
        0 & 0 &   & 0 &   \\
          &   & 1 &   & 0 \\
        1 & 1 &   & 0 &   \\
          &   & 1 &   &   \\
        1 & 1 &   &   &
    \end{array}$$
$$p_1(x) = x$$

Luego:
$$s(x) = \left\{ \begin{array}{ll}
        x & x \in [-1, 0] \\
        x & x \in [0,1]
    \end{array} \right. = x$$

\subsection{Splines ligados}
\noindent
En el caso de los splines de extremo sujeto o ligados, tendríamos como primera y última ecuación:
$$\left\{ \begin{array}{c}
        d_0 = s'(a) \\
        d_n = s'(b)
    \end{array}\right.$$

Por lo que:
$$A=\left( \begin{array}{ccccc}
            1              &                                                 &                    &                                                         &                    \\
            \dfrac{1}{h_0} & 2\left( \dfrac{1}{h_0} + \dfrac{1}{h_1} \right) & \dfrac{1}{h_1}     &                                                         &                    \\
                           & \ddots                                          & \ddots             & \ddots                                                  &                    \\
                           &                                                 & \dfrac{1}{h_{n-2}} & 2\left( \dfrac{1}{h_{n-2}} + \dfrac{1}{h_{n-1}} \right) & \dfrac{1}{h_{n-1}} \\
                           &                                                 &                    &                                                         & 1
        \end{array} \right)$$

$$x=\left( \begin{array}{c}
            d_0     \\
            d_1     \\
            \vdots  \\
            d_{n-1} \\
            d_{n}
        \end{array} \right)$$

$$b=3\left( \begin{array}{c}
            \frac{s'(a)}{3}                                               \\
            \dfrac{\Delta_0}{h_0} + \dfrac{\Delta_1}{h_1}                 \\
            \vdots                                                        \\
            \dfrac{\Delta_{n-2}}{h_{n-2}} + \dfrac{\Delta_{n-1}}{h_{n-1}} \\
            \frac{s'(b)}{3}
        \end{array} \right)$$

$$Ax=b$$

\noindent
Que nos da un sistema también unisolvente.

\bigskip
\noindent
\textbf{Ejemplo.}
Dar un Spline cúbico de exremo sujeto $s \in S_3(-1, 0, 1)$ que interpola a los datos:
$$\begin{array}{c|ccc}
        x_i & -1 & 0   & 1 \\
        \hline
        f_i & -1 & 0   & 1 \\
        \hline
        d_i & 0  & d_1 & 0
    \end{array}$$


\bigskip
\noindent
Buscamos el valor de las derivadas, calculando las lontitudes y pendientes de cada intervalo:
$$h_i = x_{i+1}-x_i \Rightarrow h_0,h_1=1$$
$$\Delta_i = \dfrac{f_{i+1}-f_i}{h_i} \Rightarrow \Delta_0,\Delta1 = 1$$

\noindent
Resolvemos el siguiente sistema (dado por la fórmula vista en teoría):
$$\left( \begin{array}{ccc}
            1 & 0 & 0 \\
            1 & 4 & 1 \\
            0 & 0 & 1
        \end{array} \right) \left( \begin{array}{c}
            d_0 \\
            d_1 \\
            d_2
        \end{array} \right) = \left( \begin{array}{c}
            0 \\
            6 \\
            0
        \end{array} \right) \Rightarrow d_1 = \dfrac{3}{2}$$

\noindent
Por lo que ahora tenemos los datos:
$$\begin{array}{c|ccc}
        x_i & -1 & 0           & 1 \\
        \hline
        f_i & -1 & 0           & 1 \\
        \hline
        d_i & 0  & \frac{3}{2} & 0
    \end{array}$$

\noindent
Con los que podemos calcular los polinomios $p_0(x), p_1(x) \in \bb{P}_3$ que forman:
$$s(x) = \left\{ \begin{array}{ll}
        p_0(x) & x \in [-1, 0] \\
        p_1(x) & x \in [0,1]
    \end{array} \right.$$

\noindent
Calculamos $p_0(x)$ en el intervalo $[-1,0]$:
$$\begin{array}{c|cccc}
        -1 & -1 &     &     &      \\
           &    & 0   &     &      \\
        -1 & -1 &     & 1   &      \\
           &    & 1   &     & -1/2 \\
        0  & 0  &     & 1/2 &      \\
           &    & 3/2 &     &      \\
        0  & 0  &     &     &
    \end{array}$$
$$p_0(x) = -1 + (x+1)^2 -\dfrac{1}{2}(x+1)^2x$$

\bigskip
\noindent
Calculamos $p_1(x)$ en el intervalo $[0,1]$:
$$\begin{array}{c|cccc}
        0 & 0 &     &      &      \\
          &   & 3/2 &      &      \\
        0 & 0 &     & -1/2 &      \\
          &   & 1   &      & -1/2 \\
        1 & 1 &     & -1   &      \\
          &   & 0   &      &      \\
        1 & 1 &     &      &
    \end{array}$$
$$p_1(x) = \dfrac{3}{2}x - \dfrac{1}{2}x^2-\dfrac{1}{2}x^2(x-1)$$

Luego:
$$s(x) = \left\{ \begin{array}{ll}
        -1 + (x+1)^2 -\dfrac{1}{2}(x+1)^2x                   & x \in [-1, 0] \\
        \dfrac{3}{2}x - \dfrac{1}{2}x^2-\dfrac{1}{2}x^2(x-1) & x \in [0,1]
    \end{array} \right.$$

\subsection{Splines periódicos}
\noindent
En este caso:
$$\left\{ \begin{array}{c}
        s'(a) = s'(b) \\
        s''(a) = s''(b)
    \end{array}\right.$$

\noindent
Que nos añadirían al sistema como primera y última ecuación:
$$d_0-d_n=0$$
$$\dfrac{d_1}{h_0} + 2\left(\dfrac{d_0}{h_0} + \dfrac{d_n}{h_{n-1}}\right) + \dfrac{d_{n-1}}{h_{n-1}} =
    3 \left( \dfrac{\Delta_0}{h_0} + \dfrac{\Delta_{n-1}}{h_{n-1}} \right)$$

Teniendo:
$$A=\left( \begin{array}{ccccc}
            1              &                                                 &                    &                                                         & -1                 \\
            \dfrac{1}{h_0} & 2\left( \dfrac{1}{h_0} + \dfrac{1}{h_1} \right) & \dfrac{1}{h_1}     &                                                         &                    \\
                           & \ddots                                          & \ddots             & \ddots                                                  &                    \\
                           &                                                 & \dfrac{1}{h_{n-2}} & 2\left( \dfrac{1}{h_{n-2}} + \dfrac{1}{h_{n-1}} \right) & \dfrac{1}{h_{n-1}} \\
            \dfrac{2}{h_0} & \dfrac{1}{h_0}                                  & \hdots             & \dfrac{1}{h_{n-1}}                                      & \dfrac{2}{h_{n-1}}
        \end{array} \right)$$

$$x=\left( \begin{array}{c}
            d_0     \\
            d_1     \\
            \vdots  \\
            d_{n-1} \\
            d_{n}
        \end{array} \right)$$

$$b=3\left( \begin{array}{c}
            0                                                             \\
            \dfrac{\Delta_0}{h_0} + \dfrac{\Delta_1}{h_1}                 \\
            \vdots                                                        \\
            \dfrac{\Delta_{n-2}}{h_{n-2}} + \dfrac{\Delta_{n-1}}{h_{n-1}} \\
            \dfrac{\Delta_0}{h_0} + \dfrac{\Delta_{n-1}}{h_{n-1}}
        \end{array} \right)$$

$$Ax=b$$
que es un sistema unisolvente.


\section{Ejercicios}
Los ejercicios relativos a este tema están disponbles en la sección \ref{sec:Rel3.2}.
\chapter{Interpolación Polinómica}
\section*{Problema general de interpolación}
\noindent
Sean $n+1$ datos experimentales $\{(x_0, y_0), (x_1, y_1), \ldots, (x_n,y_n)\}$ (con $x_i\neq x_j$ si $i\neq j$).\newline
El problema general de interpolación consiste en encontrar una función $g(x)$ tal que $g(x_i)=y_i \ \todoi$.\\

\noindent
Gráficamente, esta condición significa que la curva que representa a $g(x)$ pasa por los puntos previamente especificados.\\

\noindent
Buscamos funciones $g(x)$ que posean ciertas propiedades como por ejemplo:
\begin{itemize}
    \item Que sea fácil de evaluar.
    \item Simple de calcular.
    \item Suficientemente regular.
    \item Fácil de almacenar.
\end{itemize}

\noindent
Como podemos observar, los siguientes tipos de funciones cumplen las condiciones indicadas:
\begin{itemize}
    \item Interpolación polinomial: $g(x) \in \bb{P}_n$
    \item Interpolación por funciones spline: $g(x)\in S_n(x_0, x_1, \ldots, x_n)$
    \item Interpolación trigonométrica: $g(x)\in V=\langle \sin(jx), \cos(jx); j\in\{0,1,\ldots\}\rangle$
    \item Interpolación racional: $g(x)\in R_{m,n}=\{\frac{p(x)}{q(x)} \mid p(x),q(x) \in \bb{P}_n\}$
\end{itemize}

\noindent
La finalidad de encontrar una función $g(x)$ que interpole a otra $f(x)$ en los puntos $x_0, x_1, \ldots, x_n$ es la
de aproximar la función $f(x)$ en cualquier punto $x \in [a,b]$. Aplicaciones de la interpolación:
\begin{itemize}
    \item Trazado de curvas suaves que pasan por una serie de puntos.
    \item Evaluación de una función complicada $f$.
    \item Construcción de librerías de funciones matemáticas.
    \item Aproximación de la derivada (o la integral) de $f(x)$ mediante la derivada (o integral) de $g(x)$.
    \item $\ldots$
\end{itemize}

\begin{definicion}
    Supongamos que se conocen los $n+1$ valores que toma una función $f(x)$ en los puntos $\{x_0, x_1,\ldots, x_n\}$.
    Se dice que $g(x)$ interpola a $f(x)$ en $\{x_0, x_1,\ldots, x_n\}$ si:
    $$g(x_i)=f(x_i) \ \todoi$$

    Gráficamente, esta condición significa que las curvas que representan a $f(x)$ y $g(x)$ se cortan en los puntos
$\{x_0, x_1, \ldots, x_n\}$.
\end{definicion}

\section{Método de Coeficientes Indeterminados}

\begin{teo}
    Dados $n+1$ puntos $(x_i,y_i) \ \todoi$, existe un único polinomio de grado menor o igual que $n$, $p_n(x)$ que
    interpola a estos dados, es decir:
    $$p_n(x_i)=y_i \ \todoi$$
\end{teo}
\begin{proof}
    Sea $p_n(x)=a_0 + a_1x + \ldots + a_nx^n$. Aplicando que $p_n(x_i)=y_i \ \todoi$ se tiene:
    $$\left.\begin{array}{ccccccccc}
            a_0    & +      & a_1x_0 & +      & \ldots & +      & a_nx_0^n & =      & y_0    \\
            a_0    & +      & a_1x_1 & +      & \ldots & +      & a_nx_1^n & =      & y_1    \\
            \vdots & \vdots & \vdots & \vdots & \ddots & \vdots & \vdots   & \vdots & \vdots \\
            a_0    & +      & a_1x_n & +      & \ldots & +      & a_nx_n^n & =      & y_n
        \end{array}\right\}$$

    \noindent
    Se trata de ver que el sistema anterior es compatible determinado. El determinante de la matriz de coeficientes
    del sistema es el \underline{determinante de Vandermonde}:
    $$V(x_0,x_1,\ldots, x_n) = \left|\begin{array}{ccccc}
            1      & x_0    & x_0^2  & \ldots & x_0^n  \\
            1      & x_1    & x_1^2  & \ldots & x_1^n  \\
            \vdots & \vdots & \vdots & \ddots & \vdots \\
            1      & x_n    & x_n^2  & \ldots & x_n^n
        \end{array}\right|$$

    Que verifica:
    $$V(x_0,x_1,\ldots, x_n) = \prod_{i>j} (x_i-x_j)\neq 0 \Leftrightarrow x_i\neq x_j \ \forall i \neq j$$
\end{proof}

\begin{ejemplo}
    Interpolar mediante un polinomio los siguientes puntos:
    \begin{equation*}
        \begin{array}{c|ccc}
            x_i & -1 & 0 & 1 \\ \hline
            y_i & 2 & 1 & 2
        \end{array}
    \end{equation*}

    \begin{itemize}
        \item Primera solución, usando la base $\{1, x, x^2\}$.
        
        Sea el polinomio buscado $p_2(x) = a_0 + a_1x + a_2x^2$. El sistema de ecuaciones que me queda es:
        \begin{equation*}
            \left. \begin{array}{c}
                p_2(-1)=2 \\
                p_2(0)=1 \\
                p_2(1)=2 \\
            \end{array} \right\} \Longrightarrow
            \left. \begin{array}{cccc}
                a_0 & -a_1 & +a_2 & =2 \\
                a_0 &&&=1 \\
                a_0 & + a_1 &+a_2 &= 2 \\
            \end{array} \right\} \Longrightarrow
            \left. \begin{array}{c}
                a_0=1 \\
                a_1=0 \\
                a_2=1 \\
            \end{array} \right\}
        \end{equation*}
        Por tanto, el polinomio buscado es $p_2(x)=1+x^2$.

        \item Segunda solución, usando la base $\{1, x+1, (x+1)x\}$.
        
        Sea el polinomio buscado $p_2(x) = b_0 + b_1(x+1) + b_2(x+1)x$. El sistema de ecuaciones que me queda es:
        \begin{equation*}
            \left. \begin{array}{c}
                p_2(-1)=2 \\
                p_2(0)=1 \\
                p_2(1)=2 \\
            \end{array} \right\} \Longrightarrow
            \left. \begin{array}{cccc}
                b_0 &  &  & =2 \\
                b_0 & + b_1&&=1 \\
                b_0 & + 2b_1 &+2b_2 &= 2 \\
            \end{array} \right\} \Longrightarrow
            \left. \begin{array}{c}
                b_0=2 \\
                b_1=-1 \\
                b_2=1 \\
            \end{array} \right\}
        \end{equation*}
        Por tanto, el polinomio buscado es $p_2(x)=2 - (x+1) + (x+1)x$.
    \end{itemize}

    Tienen distinta expresión al anterior pero son el mismo polinomio.
\end{ejemplo}


\section{Método de Langrange}
\begin{definicion}[Polinomios de Lagrange]
    Para $0\leq k \leq n$, se definen los polinomios de Lagrange como:
    $$l_k(x)=\prod_{i=0~i\neq k}^n \dfrac{x-x_i}{x_k-x_i}$$
    que son polinomios de grado $n$ y verifican:
    $$l_k(x_j) = \delta_{kj} = \left\{
        \begin{array}{cc}
            1 & k=j     \\
            0 & k\neq j
        \end{array}\right.
    $$

    \noindent
    Notemos que $\{l_k(x); k\in \{0, 1, \ldots, n\}\}$ constituyen una base de $\bb{P}_n$.
\end{definicion}

\begin{teo}
    El polinomio $p_n(x)$ que interpola los datos $\{(x_0,y_0), \ldots, (x_n,y_n)\}$ se escribe como:
    $$p_n(x) = \sum_{k=0}^n y_kl_k(x)$$
    y se denomina \underline{fórmula de Lagrange}.
\end{teo}
\begin{proof}
    Demostramos, en primer lugar, que $p_n(x)\in \bb{P}_n$
    $$l_i(x)\in \bb{P}_i \; \forall i=0, \dots, n \Longrightarrow \sum_{i=0}^n y_il_i(x) \in \bb{P}_n $$

    Demostramos ahora que cumple con las condiciones de interpolación, es decir, que pasa por los puntos indicados.
    $$p_n(x_j) = \cancel{y_0l_0(x_j)} + \dots + y_jl_j(x_j) + \dots + \cancel{y_nl_n(x_j)} = y_jl_j(x_j) \;\forall j=0, \dots, n$$
\end{proof}
\noindent
Ventajas
\begin{itemize}
    \item No hay que resolver un sistema de ecuaciones.
    \item Los polinomios básicos de Lagrange no depende de nada más que de las abscisas.
\end{itemize}
Inconvenientes
\begin{itemize}
    \item No es recursiva: si añadimos un nuevo punto hemos de rehacer los cálculos.
    \item Es inestable numéricamente.
\end{itemize}

\begin{ejemplo}
    Interpolar mediante el método de Lagrange los siguientes puntos:
    \begin{equation*}
        \begin{array}{c|ccc}
            x_i & -1 & 0 & 1 \\ \hline
            y_i & 2 & 1 & 2
        \end{array}
    \end{equation*}

    \begin{equation*}
        l_0(x) = \frac{(x-x_1)(x-x_2)}{(x_0-x_1)(x_0-x_2)} = \frac{(x-0)(x-1)}{(-1-0)(-1-1)} = \frac{x(x-1)}{2}
    \end{equation*}
    \begin{equation*}
        l_1(x) = \frac{(x-x_0)(x-x_2)}{(x_1-x_0)(x_1-x_2)} = \frac{(x+1)(x-1)}{(0+1)(0-1)} = \frac{(x+1)(x-1)}{-1}
    \end{equation*}
    \begin{equation*}
        l_2(x) = \frac{(x-x_0)(x-x_1)}{(x_2-x_0)(x_2-x_1)} = \frac{(x+1)(x-0)}{(1+1)(1-0)} = \frac{x(x+1)}{2}
    \end{equation*}

    Por tanto, el polinomio queda:
    \begin{equation*}
        p_2(x) = 2\frac{x(x-1)}{2} + 1\frac{(x+1)(x-1)}{-1} + 2\frac{(x+1)x}{2}
    \end{equation*}
\end{ejemplo}

\section{Método de Newton}
\noindent
Dados $n$ puntos, calculamos su polinomio de interpolación y dados $n+1$ volvemos a calcular su polinomio:
$$\{(x_0,y_0),\ldots, (x_{n-1},y_{n-1})\} \mapsto p_{n-1}(x)$$
$$\{(x_0,y_0),\ldots, (x_n,y_n)\} \mapsto p_n(x)$$

\noindent
Definimos una nueva función como la diferencia de ambas:
$$h(x)=p_n(x)-p_{n-1}(x) \in \bb{P}_n$$
$$h(x_i) = p_n(x_i) - p_{n-1}(x_i) = y_i - y_i = 0 \ \forall i \in \{0, \ldots, n-1\}$$

\ \\ \noindent
Por tanto, cada $x_i$ es una raíz de $h(x) \ \forall i \in \{0, \ldots, n-1\}$:
$$h(x)=D_n(x-x_0)\ldots (x-x_{n-1})$$

$$\left.\begin{array}{l}
        h(x_n)=p_n(x_n) - p_{n-1}(x_n) = y_n - p_{n-1}(x_n) \\
        h(x_n)=D_n(x_n-x_0) \ldots (x_n - x_{n-1})
    \end{array}\right\} \Rightarrow
    D_n = \dfrac{y_n - p_{n-1}(x_n)}{(x_n-x_0)\ldots(x_n-x_{n-1})}$$

\begin{teo}
    Sea $p_{n-1}(x)$ el polinomio que interpola los datos $(x_i, y_i)$, $i = 0, \dots, n-1$ y  sea $p_n(x)$ el polinomio que interpola a los mismos datos y además $(x_n, y_n)$. Entonces
    $$p_{n}(x) = p_{n-1}(x) + D_n(x-x_0)\dots(x-x_{n-1})$$
    donde
    $$D_n = \frac{y_n - p_{n-1}(x_n)}{(x_n-x_0)\dots(x_n-x_{n-1})}$$
\end{teo}
\begin{proof}
    Sea $h(x) = p_n(x) - p_{n-1}(x) \in \bb{P}_n$. Tenemos que $$h(x_i)=p_n(x_i) - p_{n-1}(x_i) = y_i - y_i = 0 \;\forall i=0, \dots, n-1$$

    Como el grado máximo de $h$ es $n$, y ya hemos encontrado $n$ raíces distintas, entonces el polinomio queda como:
    $$h(x)=D_n(x-x_0)\dots(x-x_{n-1})$$

    Calculemos ahora $D_n$. Como el polinomio pasa por $(x_n, y_n)$:
    \begin{equation*}
        \left.\begin{array}{l}
            h(x_n) = p_n(x_n) - p_{n-1}(x_n) = y_n - p_{n-1}(x_n)  \\
            h(x_n) = D_n(x_n-x_0)\dots(x_n-x_{n-1}) 
        \end{array}\right\} \Longrightarrow D_n = \frac{y_n - p_{n-1}(x_n)}{(x_n-x_0)\dots(x_n-x_{n-1})}
    \end{equation*}
\end{proof}
\begin{coro}
    Como consecuencia se tiene:
    $$p_n(x)=D_0 + D_1(x-x_0) + \ldots + D_n(x-x_0)\ldots (x-x_{n-1})$$
    Donde cada $D_k$ sólo depende de los puntos $\{(x_0,y_0),\ldots, (x_k,y_k)\}$.
\end{coro}

\bigskip
\begin{ejemplo}
Buscar el polinomio de grado 2 que interpola los puntos:
$$\begin{array}{c|ccc}
        x_i & -1 & 0 & 1 \\
        \hline
        y_i & 2  & 1 & 2
    \end{array}$$

$$p_2(x)=D_0 + D_1(x-x_0) + D_2(x-x_0)(x-x_1) = D_0 + D_1(x+1)+D_2(x+1)x$$

Aplicamos las condiciones de interpolación y tenemos que:
$$\left.\begin{array}{l}
        p_2(-1)=2 \\
        p_2(0)=1  \\
        p_2(1)=2
    \end{array}\right\} \Rightarrow
    \left. \begin{array}{ccccccc}
        D_0 &   &      &   &      & = & 2 \\
        D_0 & + & D_1  &   &      & = & 1 \\
        D_0 & + & 2D_1 & + & 2D_2 & = & 2
    \end{array}\right\}$$

$$p_2(x)=2-(x+1)+(x+1)x$$
\end{ejemplo}

Notemos que la expresión cambia si alteramos el orden de los puntos (el polinomio no).

\begin{definicion}[Diferencias divididas]
    Al coeficiente $D_k = f[x_0,\ldots, x_k]$ se le llama la \underline{diferencia dividida de orden $k$}.
\end{definicion}

\noindent
Por tanto, el polinomio $p_n(x)$ que interpola los datos $\{(x_0,y_0),\ldots, (x_n,y_n)\}$ se escribe como:
$$p_n(x) = f[x_0] + f[x_0,x_1](x-x_0) + f[x_0,x_1,x_2](x-x_0)(x-x_1) + \ldots + $$
$$ + f[x_0,x_1, \ldots, x_n](x-x_0)(x-x_1)\ldots(x-x_{n-1}) = $$
$$= f[x_0] + \sum_{k=1}^n f[x_0, \ldots, x_k] \prod_{i=0}^{k-1}(x-x_i)$$

\noindent
Propiedades de las diferencias divididas:
\begin{itemize}
    \item \textbf{Simetría:} Si notamos $\pi(x)=(x-x_0)\ldots(x-x_n)$, se tiene que:
          $$f[x_0, \ldots, x_n] = \sum_{i=0}^n \dfrac{f(x_i)}{\pi'(x_i)}$$
    \item \textbf{Ley de recurrencia:}
          $$f[x_0,\ldots, x_n] = \dfrac{f[x_1,\ldots,x_n]-f[x_0,\ldots,x_{n-1}]}{x_n-x_0} ~~~x_0 \neq x_n$$
          $$f[x_i] = y_i~\todoi$$
\end{itemize}

\noindent
Gracias a esta última ley, podemos construir la tabla de las diferencias dividas:
$$\begin{array}{c|cccc}
        x_0 & f[x_0] &             &                &                    \\
            &        & f[x_0,x_1]  &                &                    \\
        x_1 & f[x_1] &             & f[x_0,x_1,x_2] &                    \\
            &        & f[x_1,x_2]  &                & f[x_0,x_1,x_2,x_3] \\
        x_2 & f[x_2] &             & f[x_1,x_2,x_3] &                    \\
            &        & f[x_2, x_3] &                &                    \\
        x_3 & f[x_3] &             &                &
    \end{array}$$

\begin{ejemplo}
    Interpolar mediante el método de Newton los siguientes puntos:
    \begin{equation*}
        \begin{array}{c|ccc}
            x_i & -1 & 0 & 1 \\ \hline
            y_i & 2 & 1 & 2
        \end{array}
    \end{equation*}

    Usando la Ley de Recurrencia:
    \begin{equation*}
        \begin{array}{c|ccc}
            x_i & f[x_i] \\
            \\
            -1 & \textbf{2} \\
            && f[-1, 0] = \textbf{-1}\\
            0 & 1 & & f[-1, 0, 1] = \frac{1-(-1)}{1-(-1)} = \textbf{1}\\
            && f[0, 1] = 1\\
            1 & 2 \\
        \end{array}
    \end{equation*}

    Por tanto, el polinomio queda:
    \begin{equation*}
        p_2(x) = 2-(x+1)+(x+1)x
    \end{equation*}
\end{ejemplo}

\begin{ejemplo}
    Interpolar mediante el método de Newton los siguientes puntos:
    \begin{equation*}
        \begin{array}{c|cccc}
            x_i & -1 & 0 & 1 & 2 \\ \hline
            y_i & 0 & 1 & 2 & 9
        \end{array}
    \end{equation*}

    Usando la Ley de Recurrencia:
    \begin{equation*}
        \begin{array}{c|cccc}
            x_i & f[x_i] \\
            \\
            -1 & \textbf{0} \\
            && \textbf{1}\\
            0 & 1 & & \textbf{0}\\
            && 1&&\textbf{1}\\
            1 & 2 && 3\\
            & & 7\\
            2 & 9
        \end{array}
    \end{equation*}

    Por tanto, el polinomio queda:
    \begin{equation*}\begin{split}
        p_3(x) &= 0 + 1(x-(-1)) + 0(x-(-1))(x-0) + 1(x-(-1))(x-0)(x-1) \\
        &= (x+1)+(x+1)x(x-1)
    \end{split}\end{equation*}
\end{ejemplo}

\begin{lema}[Aitken] Sean los datos $(x_0,f(x_0)),\dots,(x_n,f(x_n))$. Denotemos por $p_{x_0,\dots,x_{n-1}}(x)$ el polinomio que interpola los $n$ primeros, y sea $p_{x_1,\dots,x_n}(x)$ el polinomio que interpola los $n$ últimos. Entonces, el polinomio que interpola todos los datos $p_n(x)$ se puede escribir de la forma
\begin{equation*}
    \frac{(x-x_0)p_{x_1,\dots,x_n}(x) - (x-x_n)p_{x_0,\dots,x_{n-1}}(x)}{x_n-x_0}
\end{equation*}
\end{lema}
\begin{proof}
    Definimos $$h(x)=\frac{(x-x_0)p_{x_1,\dots,x_n}(x) - (x-x_n)p_{x_0,\dots,x_{n-1}}(x)}{x_n-x_0}\in \bb{P}_n$$

    Verificamos ahora que se cumplen las condiciones de interpolación.\\
    Para $i\neq 0,n$,
    \begin{multline*}
        h(x_i)=\frac{(x_i-x_0)p_{x_1,\dots,x_n}(x_i) - (x_i-x_n)p_{x_0,\dots,x_{n-1}}(x_i)}{x_n-x_0} =\\=
        \frac{(x_i-x_0)f(x_i)-(x_i-x_n)f(x_i)}{x_n-x_0} = \frac{(x_n-x_0)f(x_i)}{x_n-x_0} = f(x_i)
    \end{multline*}
    Para $i=0$,
    \begin{multline*}
        h(x_0)=\frac{(x_0-x_0)p_{x_1,\dots,x_n}(x_0) - (x_0-x_n)p_{x_0,\dots,x_{n-1}}(x_0)}{x_n-x_0} =\\=
        \frac{- (x_0-x_n)p_{x_0,\dots,x_{n-1}}(x_0)}{x_n-x_0} = p_{x_0,\dots,x_{n-1}}(x_0) = f(x_0)
    \end{multline*}
    Para $i=n$,
    \begin{multline*}
        h(x_n)=\frac{(x_n-x_0)p_{x_1,\dots,x_n}(x_n) - (x_n-x_n)p_{x_0,\dots,x_{n-1}}(x_n)}{x_n-x_0} =\\=
        \frac{(x_n-x_0)p_{x_1,\dots,x_n}(x_n)}{x_n-x_0} = p_{x_1,\dots,x_n}(x_n) = f(x_n)
    \end{multline*}
\end{proof}

Notemos además que:
$$p_n(x)=f[x_0] + f[x_0,x_1](x-x_0) + \ldots + f[x_0, \ldots, x_n](x-x_0)\ldots (x-x_{n-1})$$
El término del coeficiente líder de $p_n(x)$ es $f[x_0, \ldots, x_n]$ para cada $n$:
$$f[x_0, \ldots, x_n] = \dfrac{f[x_1, \ldots, x_n]-f[x_0, \ldots, x_{n-1}]}{x_n-x_0}$$
Por tanto, los coeficientes del polinomio de interpolación serán las diferencias dividas.

\bigskip \noindent
Ventajas de la fórmula de Newton:
\begin{itemize}
    \item Los coeficientes de la fórmula, las diferencias divididas, se pueden calcular recurrentemente
          a partir de la tabla anterior.
    \item La fórmula es recurrente, si añadimos un punto sólo hay que calcular una nueva diagonal
          en la tabla de las diferencias divididas.
    \item La fórmula se puede evaluar mediante un algoritmo análogo al esquema de Horner.
\end{itemize}



\begin{ejemplo}
    Interpolar mediante el lema de Aitken los siguientes puntos:
    \begin{equation*}
        \begin{array}{c|ccc}
            x_i & -1 & 0 & 1 \\ \hline
            y_i & 2 & 1 & 2
        \end{array}
    \end{equation*}

    Usando el Lema de Aitken:
    \begin{equation*}
        \begin{array}{c|ccc}
            x_0=-1 & p_{x_0}(x)=f(x_0)=2 \\
            && p_{x_0,x_1} = \frac{(x-x_0)P_{x_1}(x)-(x-x_1)P_{x_0}(x)}{x_1-x_0} = \frac{(x+1)1 - x\cdot 2}{1} = -x+1\\
            x_1=0 & p_{x_1}(x)=f(x_1)=1 & & \Longrightarrow \\
            && p_{x_1,x_2} = \frac{(x-x_1)P_{x_2}(x)-(x-x_2)P_{x_1}(x)}{x_2-x_1} = \frac{x\cdot 2 - (x-1)1)}{1} = x+1\\
            x_2=1 & p_{x_2}(x)=f(x_2)=2 \\
        \end{array}
    \end{equation*}
    \begin{multline*}
        \Longrightarrow P_{x_0, x_1, x_2}(x) = \frac{(x-x_0)P_{x_1,x_2}(x) - (x-x_2)P_{x_0,x_1}(x)}{x_2-x_0} =\\= \frac{(x+1)(x+1)-(x-1)(-x+1)}{2} = x^2+1
    \end{multline*}
\end{ejemplo}
\subsection{Forma de evaluar el polinomio de Newton}
\textbf{Newton Horner}\par
Para realizar el menor número de operaciones:

$$p_n(x) = f[x_0] + f[x_0,x_1](x-x_0) + \ldots + f[x_0,\ldots ,x_n](x-x_0)\ldots(x-x_{n-1}) = $$
$$=f[x_0] + (x-x_0) [f[x_0,x_1] + (x-x_1)[\ldots (x-x_{n-1}) f[x_0, \ldots, x_n] \ldots ]]$$

\bigskip
Para evaluar nuestro polinomio $p_n(x)$ en el punto $x$:

$$\left.\begin{array}{l}
        b_n = f[x_0, \ldots, x_n]                            \\
        b_{n-1} = b_n(x - x_{n-1}) + f[x_0, \ldots, x_{n-1}] \\
        \vdots                                               \\
        b_1 = b_2 (x-x_1) + f[x_0, x_1]                      \\
        p_n(x) = b_0 = b_1 (x-x_0) + f[x_0]
    \end{array}\right\}$$

\section{El error de interpolación}
A la hora de interpolar un polinomio, podemos calcular el error que cometemos si conocemos el polinomio
mediante la fórmula:
$$E_r (f; x) = |f(x) - p_n(x)|$$

\begin{ejemplo}
    Calcular el error al aproximar la función $f(x)=e^x$ sabiendo que:
    \begin{equation*}
        \begin{array}{c|cc}
            x_i & 0 & 1 \\ \hline
            f(x_i) & 1 & e
        \end{array}
    \end{equation*}

    Aproximando mediante el método de Newton,
    \begin{equation*}\begin{split}
        P_2(x)&
        = f[x_0] + f[x_0,x_1](x-x_0)
        = f(x_0) + \frac{f(x_1) - f(x_0)}{x_1 - x_0} (x-x_0) \\
        &= 1 + \frac{e - 1}{1-0} (x-0)
        = 1+(e-1)x
    \end{split}\end{equation*}

    Por tanto, el error dado por la función $E(x)$ es:
    \begin{equation*}
        E(x) = |e^x - P_2(x)| = |e^x - 1 - (e-1)x| = |f(x)-f(0) - (e-1)x|
    \end{equation*}
    aplicando el Teorema de Lagrange en el intervalo $[\min\{0,x\},\max\{0,x\}]$ (ya que $f(x)$ es continua y derivable en $\bb{R}$) tenemos que:
    \begin{equation*}
        \exists \xi \in ]\min\{0,x\},\max\{0,x\}[ \mid f(x)-f(0) = f'(\xi)(x-0) = e^\xi x
    \end{equation*}
    
    Por tanto,
    \begin{equation*}
        E(x)=|e^\xi x - (e-1)x| = |x(e^\xi - (e-1))| = |x|\cdot |e^\xi -e+1| \stackrel{(\ast)}{\leq}
        |x|\cdot |e -e+1| = |x|
    \end{equation*}
    donde en $(\ast)$ he aplicado que, como estamos interpolando entre $x_0=0$y $x_1=1$, entonces $x\in[0,1]$. Por tanto, $\xi \in ]\min\{0,x\},\max\{0,x\}[ \subseteq ]0,1[$. Por tanto, $e^\xi \in1,e[$.
    
    Por tanto, hemos visto que $E(x)\leq |x|\quad \forall x\in ]0,1[$.
\end{ejemplo}
\begin{teo}
    Sea $f\in C^{n+1}([a,b])$ y sean $x_0,x_1,\dots,x_n\in[a,b]$. Entonces,
    \begin{equation*}
        \exists \xi \mid \min\{x,x_0,\dots,x_n \} \leq \xi \leq \max\{x,x_0,\dots,x_n \} 
    \end{equation*}
    con:
    \begin{equation*}
        f(x)-p_n(x) = \frac{f^{n+1)}(\xi)}{(n+1)!}\prod_{k=0}^n (x-x_k)
    \end{equation*}
\end{teo}
\begin{proof}
    Supongamos $x=x_j\;(j=0,\dots,n)$. Tenemos que se cumple, ya que el miembro de la izquiera se anula por las condiciones de $p_n$ y el término de la derecha se anula al anularse el productorio.

    Supongamos por tanto $x\neq x_j\;(j=0,\dots,n)$. Tomamos $x\in[a,b]$ fijo. Definimos $$\psi (x) = \prod_{k=0}^n (x-x_k) \in \bb{P}_{n+1}(x)$$

    En el intervalo $[a,b]$, definimos también
    \begin{equation*}
        g(t) = [f(t)-p_n(t)]\psi(x) - [f(x)-p_n(x)]\psi(t)
    \end{equation*}
    \begin{itemize}
        \item \underline{Para $t=x$}: tenemos que $g(t)=g(x)=0$.
        
        \item \underline{Para $t=x_j\;(j=0,\dots,n)$}: tenemos que $$g(t)=g(x_j)=\cancelto{0}{[f(t)-p_n(t)]}\psi(x) - [f(x)-p_n(x)]\cancelto{0}{\psi(t)} = 0$$
    \end{itemize}

    Por tanto, tenemos que $g$ se anula en $t=x\neq x_j$ y en $t=x_j$. Por tanto, se anula en $n+2$ puntos. Por el teorema de Rolle, $g'$ se anula en $n+1$ puntos distintos.

    Siguiendo un razonamiento idéntico, y derivando sucesivamente ya que $g\in~C^{n+1}[a,b]$, entonces tenemos que:
    \begin{itemize}
        \item $g'$ se anula en $n+1$ puntos distintos.
        \item $g''$ se anula en $n$ puntos distintos.
        \item $\vdots$
        \item $g^{n)}$ se anula en dos puntos distintos.
        \item $g^{n+1)}$ se anula en un único punto. Sea $\xi \in \bb{R}\mid g^{n+1)}(\xi) = 0$.
    \end{itemize}


    Veamos el valor de $g^{n+1)}$
    \begin{equation*}
        g^{n+1)}(t) = [f^{n+1)}(t)-{p_n^{n+1)}}(t)]\psi(x) - [f(x)-p_n(x)]\psi^{n+1)}(t)
    \end{equation*}
    donde tenemos que 
    \begin{equation*}
        {p_n^{n+1)}}(t) = 0\text{ por ser la derivada de orden $n+1$ de un polinomio de grado $n$.}
    \end{equation*}
    \begin{equation*}
        \psi^{n+1)}(t) = (n+1)!\text{ ya que $\psi (x) \in \bb{P}_{n+1}(x)$ y su coeficiente líder es $1$.}
    \end{equation*}

    Por tanto, tenemos que
    \begin{equation*}
        g^{n+1)}(t) = f^{n+1)}(t)\psi(x) - [f(x)-p_n(x)](n+1)!
    \end{equation*}

    Tomando $t=\xi$ (único), 
    \begin{multline*}
        g^{n+1)}(\xi) = 0 = f^{n+1)}(\xi)\psi(x) - [f(x)-p_n(x)](n+1)!
        \Longrightarrow\\ \Longrightarrow
        f(x)-p_n(x) = \frac{f^{n+1)}(\xi)}{(n+1)!}\psi(x) = \frac{f^{n+1)}(\xi)}{(n+1)!}\prod_{k=0}^n (x-x_k)
    \end{multline*}
\end{proof}

\begin{ejemplo}
    Calcular el error al aproximar la función $f(x)=e^x$ sabiendo que:
    \begin{equation*}
        \begin{array}{c|cc}
            x_i & 0 & 1 \\ \hline
            f(x_i) & 1 & e
        \end{array}
    \end{equation*}

    Aproximando mediante el método de Newton,
    \begin{equation*}\begin{split}
        P_2(x)&
        = f[x_0] + f[x_0,x_1](x-x_0)
        = f(x_0) + \frac{f(x_1) - f(x_0)}{x_1 - x_0} (x-x_0) \\
        &= 1 + \frac{e - 1}{1-0} (x-0)
        = 1+(e-1)x
    \end{split}\end{equation*}

    Por tanto, el error dado por la función $E(x)$ es:
    \begin{multline*}
        E(x) = |f(x) - P_1(x)| = \left| \frac{f''(\xi)}{2!} (x-x_0) (x-x_1)\right|
        = \left |\frac{e^\xi}{2} (x) (x-1)\right|
        =\\=
        \frac{e^\xi}{2} \left |(x) (x-1)\right|
        \stackrel{(1)}{\leq} \frac{e}{2} \left |(x) (x-1)\right|
        \stackrel{(2)}{\leq} \frac{e}{2}\cdot  \frac{1}{4} = \frac{e}{8}
    \end{multline*}

    donde en $(1)$ he empleado que $\xi \in [0,1]$ y en $(2)$ he aplicado que $x\in[0,1]$, y representando $|x^2-x|$ lo deduzco.
\end{ejemplo}

\section{Método de Hermite}

\begin{ejemplo}
    Estudie para que valores de $a\in \bb{R}$ es unisolvente el siguiente problema de interpolación:\\
    Encontrar $p\in \bb{P}_2$ tal que:
    \begin{equation*}
        \left.\begin{array}{ccc}
            p(0) &=& \omega_0  \\
            p'(a) &=& \omega_1  \\
            p(1) &=& \omega_2  \\
        \end{array}\right\}
    \end{equation*}

    Resolvemos mediante el método de coeficientes indeterminados. Sea $p_2(x)=b_0 + b_1x + b_2x^2$.
    \begin{equation*}
        \left.\begin{array}{ccc}
            p(0) &=& \omega_0  \\
            p'(a) &=& \omega_1  \\
            p(1) &=& \omega_2  \\
        \end{array}\right\}
        \Longrightarrow
        \left.\begin{array}{rrrrrcc}
            b_0&&&& &=& \omega_0  \\
            &&b_1&+&2b_2a &=& \omega_1  \\
            b_0&+&b_1&+&b_2 &=& \omega_2  \\
        \end{array}\right\}
    \end{equation*}

    La solución será única si:
    \begin{equation*}
        \left|\begin{array}{ccc}
            1 & 0 & 0 \\
            0 & 1 & 2a \\
            1 & 1 & 1
        \end{array}\right| \neq 0 \Longleftrightarrow 1-2a \neq 0 \Longleftrightarrow a=\frac{1}{2}
    \end{equation*}

    Para $a=\frac{1}{2}$, tenemos que para $\omega_1=0$ hay infinitas soluciones, mientras que para $\omega_1\neq 0$ no hay solución.
\end{ejemplo}

\noindent
En algunos casos, se proporcionarán datos sobre el valor de la función y sobre derivadas sucesivas
en los punto de esta.\newline
Se interpondrá la condición de que si se da como dato $f^{(j)}(x_i)$, entonces se proporcionarán
en el punto $x_i$ las derivadas sucesivas de orden inferior a $j$.\\

\textbf{Problema clásico}
$$\begin{array}{c|cccc}
         & x_0     & x_1     & \ldots & x_n     \\
        \hline
         & f(x_0)  & f(x_1)  & \ldots & f(x_n)  \\
        \hline
         & f'(x_0) & f'(x_1) & \ldots & f'(x_n)
    \end{array}$$

\textbf{Problema generalizado}\\

\noindent
Se nos dan $n+1$ puntos tales que en el punto $x_i$ conocemos: $f(x_i)$, $f'(x_i)$, \ldots, $f^{(k_i)}(x_i)
    ~~\forall i \in \{0, \ldots, n\}$\\

\noindent
Por lo que en verdad conocemos $m = k_0 + k_1 + \ldots + k_n + n$ datos.

\begin{observacion}
Como caso particular, si solo hay un punto, tenemos el polinomio de Taylor.  
\end{observacion}

\begin{prop}
    Existe un único polinomio $p(x)$ de grado menor o igual que $m-1$ que interpola los $m$ datos de interpolación
    de Hermite.
\end{prop}

\bigskip
\noindent
Notemos que:
$$f[x_0,x_1] = \dfrac{f[x_1]-f[x_0]}{x_1-x_0}$$
$$\lim_{x_1 \to x_0}f[x_0,x_1] = \lim_{x_1 \to x_0} \dfrac{f[x_1]-f[x_0]}{x_1-x_0} = f'(x_0)$$
Por lo que tiene sentido hacer la siguiente definición:

\bigskip
\newtheorem*{defDifDiv}{Definición}
\begin{defDifDiv}
    Sea $f \in C^{k+1}([a,b])$ y $x_0 \in [a,b]$. Entonces, definimos:
    $$f[\underbrace{x_0, x_0, \ldots, x_0}_{(k+1)}] = \dfrac{f^{(k)}(x_0)}{k!}$$

    Notemos que $f[x_0,x_0] = f'(x_0)$
\end{defDifDiv}

\bigskip
\begin{teo}
    Dados los datos de interpolación:

    $$\begin{array}{c|cccc}
             & x_0     & x_1     & \ldots & x_n     \\
            \hline
             & f(x_0)  & f(x_1)  & \ldots & f(x_n)  \\
            \hline
             & f'(x_0) & f'(x_1) & \ldots & f'(x_n)
        \end{array}$$

    Entonces, se tiene que:

    $$p_{2n+1}(x) = f[x_0]+f[x_0,x_0](x-x_0) + f[x_0,x_0,x_1](x-x_0)^2 + $$
    $$ + f[x_0,x_0,x_1,x_1](x-x_0)^2(x-x_1) + \ldots + $$
    $$ + f[x_0,x_0,\ldots, x_n,x_n](x-x_0)^2\ldots (x-x_{n-1})^2(x-x_n)$$

    \smallskip

    $$\begin{array}{c|cccc}
            x_0    & f(x_0) &            &        &                               \\
                   &        & f'(x_0)    &        &                               \\
            x_0    & f(x_0) &            &        &                               \\
                   &        & f[x_0,x_1] &        &                               \\
            x_1    & f(x_1) &            &        &                               \\
                   &        & f'(x_1)    &        &                               \\
            x_1    & f(x_1) &            &        &                               \\
            \vdots & \vdots & \vdots     &        &                               \\
            \vdots & \vdots & \vdots     & \ldots & f[x_0, x_0, \ldots, x_n, x_n] \\
            \vdots & \vdots & \vdots     &        &                               \\
            x_n    & f(x_n) &            &        &                               \\
                   &        & f'(x_n)    &        &                               \\
            x_n    & f(x_n) &            &        &
        \end{array}$$
\end{teo}

\begin{ejemplo}
    Interpolar mediante el método de Hermite clásico los siguientes datos:
    \begin{equation*}
        \begin{array}{c|cc}
            x_i & 0 & 1 \\ \hline
            f(x_i) & 1 & 2 \\ \hline
            f'(x_i) & 0 & 3
        \end{array}
    \end{equation*}

    Usando la Ley de Recurrencia:
    \begin{equation*}
        \begin{array}{c|ccccc}
            0 & 1 \\
            && f[0,0]=f'(0)=0 \\ 
            0 & 1 && 1\\
            && f[0,1]=1 & & 1\\
            1 & 2 & & 2 \\
            && f[1,1]=f'(1)=3\\
            1 & 2 \\
        \end{array}
    \end{equation*}

    Por tanto, el polinomio queda:
    \begin{equation*}
        p_3(x) = 1+0(x-0) +1(x-0)^2 +1(x-0)^2(x-1) = 1+x^3
    \end{equation*}
\end{ejemplo}


\begin{teo}
    Sea $f\in C^{2n+2}(I)$ y sean $x_0,x_1,\dots,x_n\in I$.
    
    Sea $p_{2n+1}(x)\in \bb{P}_{2n+1} \left|\begin{array}{c}
        f(x_i) = p_{2n+1}(x_i) \\
        f'(x_i) = p_{2n+1}'(x_i) 
    \end{array} \quad \forall i=0,\dots,n \right.$
    
    Entonces $\exists \xi \in I \mid$
    \begin{equation*}
        f(x)-p_{2n+1}(x) = \frac{f^{2n+2)}(\xi)}{(2n+2)!}\prod_{i=0}^n (x-x_i)^2
    \end{equation*}
\end{teo}
\begin{proof}
    Supongamos $x=x_j\;(j=0,\dots,n)$. Tenemos que se cumple, ya que el miembro de la izquierda se anula por las condiciones de $p_{2n+1}$ y el término de la derecha se anula al anularse el productorio.

    Supongamos por tanto $x\neq x_j\;(j=0,\dots,n)$. Tomamos $x\in I$ fijo. Definimos $$\psi (x) = \prod_{i=0}^n (x-x_i)^2 \in \bb{P}_{2n+2}(x)$$

    En el intervalo $I$, definimos también
    \begin{equation*}
        g(t) = [f(t)-p_{2n+1}(t)]\psi(x) - [f(x)-p_{2n+1}(x)]\psi(t)
    \end{equation*}
    \begin{itemize}
        \item \underline{Para $t=x$}: tenemos que $g(t)=g(x)=0$.
        
        \item \underline{Para $t=x_j\;(j=0,\dots,n)$}: tenemos que $$g(t)=g(x_j)=\cancelto{0}{[f(t)-p_{2n+1}}(t)]\psi(x) - [f(x)-p_{2n+1}(x)]\cancelto{0}{\psi(t)} = 0$$
    \end{itemize}

    Por tanto, tenemos que $g$ se anula en $t=x\neq x_j$ y en $t=x_j$; es decir, se anula en $n+2$ puntos. Por el teorema de Rolle, $g'$ se anula en $n+1$ puntos distintos (y todos ellos distintos a $x_i\;\forall i$)\footnote{Ya que Rolle afirma que $\exists x\in ]a,b[ \mid f'(x)=0$. El intervalo al que se refiere Rolle es, por tanto, abierto.}.

    Consideramos ahora $g'(t)$.
    \begin{equation*}
        g'(t) = [f'(t)-p'_{2n+1}(t)]\psi(x) - [f(x)-p_{2n+1}(x)]\psi'(t)
    \end{equation*}
    $$\psi' (t) = 2\sum_{i=0}^n\left[ (t-x_i)\prod_{\substack{j=0\\i\neq j}}^n (t-x_j)^2\right] \in \bb{P}_{2n+1}$$
    
    \begin{itemize}
        \item \underline{Para $t=x_j\;(j=0,\dots,n)$}: tenemos que $$g'(t)=g'(x_j)=\cancelto{0}{[f'(t)-p'_{2n+1}}(t)]\psi(x) - [f(x)-p_{2n+1}(x)]\cancelto{0}{\psi'(t)} = 0$$
    \end{itemize}
    Por tanto, tenemos que $g'$ se anula en $t=x_j$; es decir, se anula en $n+1$ puntos, todos ellos distintos a los ya calculados.

    Por tanto, tenemos que $g'(x)$ se anula en $2n+2$ puntos distintos. Por el Teorema de Rolle, podemos concluir que $g''(x)$ se anula en $2n+1$ puntos distintos, y así sucesivamente. Por tanto, tenemos:
    \begin{itemize}
        \item $g'$ se anula en $2n+2$ puntos distintos.
        \item $g''$ se anula en $2n+1$ puntos distintos.
        \item $\vdots$
        \item $g^{2n+1)}$ se anula en dos puntos distintos.
        \item $g^{2n+2)}$ se anula en un único punto. Sea $\xi \in \bb{R}\mid g^{2n+2)}(\xi) = 0$.
    \end{itemize}


    Veamos el valor de $g^{2n+2)}$
    \begin{equation*}
        g^{2n+2)}(t) = [f^{2n+2)}(t)-{p_{2n+1}^{2n+2)}}(t)]\psi(x) - [f(x)-p_{2n+1}(x)]\psi^{2n+2)}(t)
    \end{equation*}
    donde tenemos que 
    \begin{equation*}
        {p_{n+1}^{2n+2)}}(t) = 0\text{ por ser la derivada de orden $2n+2$ de un polinomio de grado $2n+1$.}
    \end{equation*}
    \begin{equation*}
        \psi^{2n+2)}(t) = (2n+2)!\text{ ya que $\psi (x) \in \bb{P}_{2n+2}(x)$ y su coeficiente líder es $1$.}
    \end{equation*}

    Por tanto, tenemos que
    \begin{equation*}
        g^{2n+2)}(t) = f^{2n+2)}(t)\psi(x) - [f(x)-p_{2n+1}(x)](2n+2)!
    \end{equation*}

    Tomando $t=\xi$ (único), 
    \begin{multline*}
        g^{2n+2)}(\xi) = f^{2n+2)}(\xi)\psi(x) - [f(x)-p_{2n+1}(x)](2n+2)!
        \Longrightarrow\\ \Longrightarrow
        f(x)-p_{2n+1}(x) = \frac{f^{2n+2)}(\xi)}{(2n+2)!}\psi(x) = \frac{f^{n+1)}(\xi)}{(n+1)!}\prod_{k=0}^n (x-x_k)^2
    \end{multline*}
\end{proof}

\section{Polinomio de Chebyshev}
\noindent
Partiendo de la fórmula del coseno para el ángulo suma, tenemos que para $\theta \in [0, \pi]$:
$$\cos((n+m)\theta) = \cos(n\theta)\cos(m\theta) - \sin(n\theta)\sin(m\theta)$$
$$\cos((n-m)\theta) = \cos(n\theta)\cos(m\theta) + \sin(n\theta)\sin(m\theta)$$
De la suma de ambas tenemos que:
$$\cos((n+m)\theta) + \cos((n-m)\theta) = 2\cos(m\theta)\cos(n\theta)$$
Haciendo $m=1$:
$$\cos((n+1)\theta) + \cos((n-1)\theta) = 2\cos(\theta)\cos(n\theta)$$
Sea $x = \cos\theta \in [-1,1]$, definimos:
$$T_n(x) = \cos(n\theta) = \cos(n \arccos(x)) \in [-1, 1]$$
Cada $T_n(x)$ satisface la relación $(n \geq 1)$:
$$T_{n+1}(x) + T_{n-1}(x) = 2xT_n(x) \Rightarrow T_{n+1}(x) = 2xT_n(x) - T_{n-1}(x)$$

$$T_0(x) = \cos0 = 1$$
$$T_1(x) = x$$
$$T_2(x) = 2xT_1(x) - T_0(x) = 2x^2-1$$
$$T_3(x) = 2xT_2(x) - T_1(x) = 2^2x^3-3x$$

\ \newline
Notemos que $T_n(x) \in \bb{P}_n$ y que $T_n(x) = 2^{n-1}x^n + \ldots ~~~~\forall n \in \N$\\

\noindent
Estos son los polinomios de Chebyshev de primera especie.

\bigskip
\noindent
Notemos que $T_n(x)$ tiene $n$ ceros distintos en $\left]-1,1\right[$:
$$\left.\begin{array}{c}
        T_n(x) = \cos\theta \\
        \cos\theta = x
    \end{array}\right\} \Rightarrow \cos(n\theta) = 0 \Leftrightarrow n\theta = \frac{\pi}{2}+k\pi~~\forall
    k \in \Z$$

$$n\theta = \dfrac{(2k+1)\pi}{2} \Rightarrow \theta = \dfrac{(2k+1)\pi}{2n}~~k \in \Z $$

$k=0 \Rightarrow \theta_0 = \dfrac{\pi}{2n}$\par
$\theta_1 = \dfrac{3\pi}{2n}$\par
$\theta_2 = \dfrac{5\pi}{2n}$\par
\ldots \par
$\theta_n = \dfrac{(2n+1)\pi}{2n} = \pi + \dfrac{\pi}{2n}$\\

\bigskip
\noindent
$T_{n+1}(x)$ tiene $n+1$ raíces. Sean $x_0, x_1, \ldots, x_n$ dichas raíces:
$$x_k = \cos \dfrac{(2k+1)\pi}{2(n+1)}~~~\forall k \in \{0, \ldots, n\}$$
$$f(x) - p_n(x) = \dfrac{f^{(n+1)}(\xi)}{(n+1!)} \prod_{i=0}^n (x-x_i)$$
$$\prod_{k=0}^n (x-x_k) = \dfrac{T_{n+1}(x)}{2^n}$$
Luego:
$$|f(x) - p_n(x)| = \dfrac{|f^{(n+1)}(\xi)|}{(2n+1)!} \left| \dfrac{T_{n+1}(x)}{2^n} \right| \leq
    \dfrac{|f^{(n+1)}(\xi)|}{(n+1)!} \dfrac{1}{2^n}$$


\section{Ejercicios}
Los ejercicios relativos a este tema están disponbles en la sección \ref{sec:Rel3.1}.
\section{Introducción a los Problemas del Análisis Numérico} \label{sec:Rel1}


\begin{ejercicio}
    En aritmética de tres dígitos por redondeo, calcula el valor de $\sqrt{543}-\sqrt{540}$.

    \begin{equation*}
        \sqrt{543}-\sqrt{540} \approx 23.3-23.2 = 0.1
    \end{equation*}
\end{ejercicio}

\begin{ejercicio}\label{Ejercicio2}
    Sea $p(x) = a_n x^n + a_{n-1}x^{n-1} + \dots + a_1x+a_0$ y $c\in\bb{R}$. Demuestra que en el algoritmo de Horner para evaluar $p(x)$ en el punto $x = c$, el polinomio $q(x) = b_n x^{n-1} + b_{n-1} x^{n-2} + \dots + b_2x + b_1$ es el cociente de la división de $p(x)$ entre el polinomio $x-c$, y que $p(c)$ es el resto.
\end{ejercicio}

\begin{ejercicio}
    Demuestra que al aplicar el método de Horner dos veces consecutivas se obtiene el valor del polinomio y el de su derivada.
    \begin{proof}
        Usando el resultado del ejercicio \ref{Ejercicio2}, al aplicar por primera vez el Algoritmo de Horner, se obtiene $p(c)$, demostrando así el primer resultado.
        $$p(x) = q(x)(x-c)+p(c)$$
    
        Al aplicar por primera vez el Algoritmo de Horner, se obtiene $q(c)$.
        $$q(x) = t(x)(x-c)+q(c)$$
    
        Como se pide demostrar que al aplicarlo la segunda vez se obtiene $p'(c)$, comprobemos que $q(c)=p'(c)$:
        $$p'(x) = q'(x)(x-c) + q(x) \Longrightarrow p'(c) = q'(c)\cdot 0 + q(c) \Longrightarrow p'(c) = q(c)$$
    
        Por tanto, se comprueba que al aplicar Horner por segunda vez se obtiene el valor de la derivada evaluada en el punto.
    \end{proof}
\end{ejercicio}

\begin{ejercicio}
    Calcula $p(1.1)$ para el polinomio $p(x)=x^3-3x^2+3x$ utilizando la expresión explícita y el algoritmo de Horner en aritmética de redondeo a dos dígitos. ¿Cambian los resultados? ¿Qué valor es más exacto? ¿Por qué?
    \begin{itemize}
    \item\underline{Exp. explícita}\\
    \begin{equation*}
        p(1.1) = (1.1)^3 - 3(1.1)^2+3(1.1) = 1.3 -3.6 +3.3 = 1
    \end{equation*}
    \item\underline{Algoritmo de Horner}\\
    \begin{equation*}
        \begin{array}{c|ccccc}
                & 1 & -3 & 3 & 0 \\
            1.1 &   & 1.1 & -2.1 & 0.99\\          \hline
            & 1 & -1.9 & 0.9 & 0.99
        \end{array} \Longrightarrow p(1.1) = 0.99
    \end{equation*}
    \end{itemize}
\end{ejercicio}

\begin{ejercicio}
    Calcule $p(1.1)$ para el polinomio $p(x) = x^3 -3 x^2 + 3 x$ utilizando la expresión explícita y el algoritmo de Horner en aritmética de redondeo a dos dígitos. ¿Cambian los resultados? ¿Qué valor es más exacto? ¿Porqué?\\

    En el caso de el algoritmo de Horner, el valor es $p(1.1)=0.99$.
    \begin{equation*}
        \begin{array}{c|cccc}
             & 1 & -3 & 3 & 0 \\
            1.1  & & 1.1 & -2.1 & 0.99\\ \hline
            & 1 & -1.9 & 0.9 & 0.99
        \end{array}
    \end{equation*}

    Evaluando en la expresión analítica:
    \begin{equation*}
        p(1.1)=(1.1)^3 -3 (1.1)^2 + 3 (1.1) = 1.3 -3(1.2) + 3(1.1) = 1.3 -3.6 + 3.3 = 1
    \end{equation*}

    El valor exacto es $p(1.1)=1.001$. Aunque por norma general es más exacto el algoritmo de Horner, en este caso es más exacto evaular en la expresión analítica. Esto se debe a que al evaluar al cubo se produce un error por defecto, mientras que al evaluar en el monomio de grado dos se produce un error por exceso. Por tanto, los errores, aunque son mayores, se compensan.
\end{ejercicio}

\begin{ejercicio}
    Determina el número de operaciones necesario para evaluar un polinomios de grado $n$ mediante los siguientes métodos:
    \begin{enumerate}
        \item Evaluación directa de las potencias.
        
        En cada monomio de grado $n$ se realizan $n-1$ multiplicaciones para calcular la potencia, y una última multiplicación de la potencia por el coeficiente. Por tanto, en cada monomio de grado $n$ se realizan $n$ multipicaciones. Por ello, en un polinomio de grado $n$ el número de multiplicaciones será:
        \begin{equation*}
            \sum_{i=0}^n n-i = n + (n-1) + \dots + 1 = \frac{n(n+1)}{2}
        \end{equation*}

        Respecto a las sumas, como hay $n+1$ monomios, se realizan $n$ sumas.

        \item El esquema de Horner.
        
        En este caso, un polinomio de grado $n$, al representarlo en la forma de Ruffini-Horner, tendrá $n+1$ coeficientes y por tanto $n+1$ columnas.

        Exceptuando la primera, en cada columna se realiza una suma, por lo que se realizarán $n$ sumas.

        Respecto a las multiplicaciones, se multiplica el valor en el que se evalúa por cada $b_i$ exceptuando el último, $b_0$. Por tanto, se realizarán $n$ multiplicaciones.

        Por tanto, en total se realizan $2n$ operaciones.
    \end{enumerate}
\end{ejercicio}

\section{Cálculo Integral Teórico}

\begin{ejercicio} Calcular usando el Teorema de Cauchy para integrales que
\begin{equation*}
    \int_0^1 x^pdx = \frac{1}{p+1} \quad (\forall p\in \bb{N}\cup\{0\}).
\end{equation*}

Tenemos que:
\begin{equation*}
    \int_0^1 f(x)\;dx = \lim_{n\to \infty} \frac{1}{n}\sum_{k=1}^n f\left(\frac{k}{n}\right)
\end{equation*}

Por tanto, tomando $f(x)=x^p$, tenemos:
\begin{equation*}
    \int_0^1 x^p\;dx
    = \lim_{n\to \infty} \frac{1}{n}\sum_{k=1}^n f\left(\frac{k}{n}\right)
    = \lim_{n\to \infty} \frac{1}{n} \sum_{k=1}^n \frac{k^p}{n^p}
    = \lim_{n\to \infty} \frac{1}{n}\cdot \frac{1}{n^p} \sum_{k=1}^n k^p
    = \lim_{n\to \infty} \frac{1}{n^{p+1}} \sum_{k=1}^n k^p
\end{equation*}

Como tengo que $\{n^{p+1}\}\nearrow\nearrow +\infty$, uso el Criterio de Stolz para sucesiones:
\begin{equation*}
    \int_0^1 x^p\;dx
    = \lim_{n\to \infty} \frac{1}{n^{p+1}} \sum_{k=1}^n k^p
    \stackrel{Stolz}{=} \lim_{n\to \infty} \frac{(n+1)^p}{(n+1)^{p+1} -n^{p+1}}
\end{equation*}

Demuestro ahora por inducción sobre $p$:
\begin{itemize}
    \item \underline{Para $p=0$}:
    \begin{equation*}
        \int_0^1 x^0\;dx
        = \lim_{n\to \infty} \frac{(n+1)^p}{(n+1)^{p+1} -n^{p+1}}
        = \lim_{n\to \infty} \frac{1}{n+1 -n}
        = \lim_{n\to \infty} 1 = 1
    \end{equation*}
    Por tanto, para $p=0$ es cierto.

    \item \underline{Supongo cierto para $p$ y demuestro para $p+1$}:

    Por hipótesis de inducción, tengo que:
    \begin{equation*}
       \int_0^1 x^{p}\;dx = \lim_{n\to \infty} \frac{(n+1)^p}{(n+1)^{p+1} -n^{p+1}}
        = \frac{1}{p+1}
    \end{equation*}

    Comprobamos para $p+1$:
    \begin{multline*}
        \int_0^1 x^{p+1}\;dx
        = \lim_{n\to \infty} \frac{(n+1)^{p+1}}{(n+1)^{p+2} -n^{p+2}}
        = \lim_{n\to \infty} \frac{(n+1)^{p}(n+1)}{(n+1)^{p+1}(n+1) -n^{p+1}n}
        =\\
        = \lim_{n\to \infty} \frac{(n+1)^{p}(n+1)}{(n+1)^{p+1}\cdot n +(n+1)^{p+1} -n^{p+1}n}
        = \lim_{n\to \infty} \frac{(n+1)^{p}(n+1)}{n[(n+1)^{p+1}-n^{p+1}] +(n+1)^{p+1}}
        =\\
        = \lim_{n\to \infty} \frac{(n+1)^{p}(n+1)}{[(n+1)^{p+1}-n^{p+1}]\cdot \left[n+\frac{(n+1)^{p+1}}{(n+1)^{p+1}-n^{p+1}}\right]}
        =\\
        =\lim_{n\to \infty}  \frac{(n+1)^{p}}{(n+1)^{p+1}-n^{p+1}} \cdot \frac{n+1}{n+\frac{(n+1)^{p}}{(n+1)^{p+1}-n^{p+1}}\cdot (n+1)}
        =\\
        =\lim_{n\to \infty}  \frac{(n+1)^{p}}{(n+1)^{p+1}-n^{p+1}} \cdot \frac{1}{\frac{n}{n+1}+\frac{(n+1)^{p}}{(n+1)^{p+1}-n^{p+1}}}
    \end{multline*}

    Haciendo uso de la hipótesis de inducción, tenemos que:
    \begin{equation*}
        \int_0^1 x^{p+1}\;dx
        = \lim_{n\to \infty}  \frac{(n+1)^{p}}{(n+1)^{p+1}-n^{p+1}} \cdot \frac{1}{\frac{n}{n+1}+\frac{(n+1)^{p}}{(n+1)^{p+1}-n^{p+1}}} = \frac{1}{p+1}\cdot \frac{1}{1+\frac{1}{p+1}} = \frac{1}{p+2}
    \end{equation*}

    Por tanto, se verifica para $p+1$.
\end{itemize}

Por tanto, hemos demostrado que:
\begin{equation*}
    \int_0^1 x^pdx = \frac{1}{p+1} \quad (\forall p\in \bb{N}\cup\{0\}).
\end{equation*}
\end{ejercicio}

\begin{ejercicio}
    Justificar las siguientes desigualdades:
    \begin{enumerate}
        \item $\displaystyle \frac{1}{2}< \int^1_0 \frac{dx}{1+x} < 1$\\

        Definimos $f:\left[0,1\right]\to \bb{R}$ dada por $f(x) = \frac{1}{1+x}$. Sabemos que $f(x)$ es estrictamente decreciente en $]-1, +\infty[$, por lo que $Im(f)=\left[\frac{1}{2},1\right]$. Por tanto, tenemos que:
        \begin{equation*}
            \frac{1}{2} \leq f(x) \leq 1 \qquad \forall x\in [0,1]
        \end{equation*}

        Por la conservación del orden, como tenemos las desigualdades no estrictas y, además, $f(x)\neq 1 \land f(x)\neq \frac{1}{2}$, tenemos que:
        \begin{equation*}
            \int_0^1\frac{1}{2}dx < \int_0^1f(x)d(x) < \int^1_0dx
        \end{equation*}

        Resolviendo las integrales de las constantes, tenemos que:
        \begin{equation*}
            \frac{1}{2} < \int_0^1f(x)d(x) < 1
        \end{equation*}

        \item $\displaystyle \frac{1}{4\sqrt{2}} < \int^1_0 \frac{x^3dx}{\sqrt{1+x}} < \frac{1}{4}$

        Trivialmente, tenemos que:
        \begin{equation*}
            \frac{x^3}{\sqrt{2}} < \frac{x^3}{\sqrt{1+x}} < x^3 \qquad \forall x\in ]0,1[
        \end{equation*}

        Por la conservación del orden, tenemos que:
        \begin{equation*}
            \int_0^1 \frac{x^3}{\sqrt{2}}\;dx < \int_0^1\frac{x^3}{\sqrt{1+x}}\;dx < \int_0^1 x^3\;dx
        \end{equation*}

        Resuelvo la siguiente integral:
        \begin{equation*}
            \int_0^1 x^3\;dx = \left[\frac{x^4}{4}\right]_0^1 = \frac{1}{4}
        \end{equation*}

        Por tanto,
        \begin{equation*}
            \frac{1}{4\sqrt{2}} = \int_0^1 \frac{x^3}{\sqrt{2}}\;dx < \int_0^1\frac{x^3}{\sqrt{1+x}}\;dx < \int_0^1 x^3\;dx = \frac{1}{4}
        \end{equation*}
    \end{enumerate}
\end{ejercicio}

\begin{ejercicio}\label{Ejercicio3}
    Sea $f:[a,b]\to \bb{R}$, una función continua y tal que $f(x)\geq 0\;\forall x\in [a,b]$. Demostrar que:
    \begin{equation*}
        \int_a^bf(x)dx = 0 \Longrightarrow f=0
    \end{equation*}

    Sea $x_0\in ]a,b[$ arbitrario. Por ser $f$ continua, tenemos que $\exists \delta>0$ tal que se cumple que:
    \begin{equation*}
        A=[x_0-\delta, x_0+\delta] \subseteq ]a,b[
        \qquad \land \qquad
        f(A) \geq \frac{f(x_0)}{2}
    \end{equation*}

    Por tanto, por las acotaciones inferiores en cada intervalo,
    \begin{multline*}
        0 = \int_a^bf(x)dx =
        \int_a^{x_0-\delta} f(x)\;dx + \int_{x_0-\delta}^{x_0+\delta}f(x)\;dx +\int_{x_0+\delta}^b f(x)\;dx
        \geq \\ \geq
        0(x_0-\delta-a) +\frac{f(x_0)}{2}(x_0+\delta -x_0 +\delta) + 0(b-x_0-\delta) = \delta f(x_0) \Longrightarrow f(x_0)=0
    \end{multline*}

    Por tanto, llegamos a que $f(x_0)=0 \; \forall x_o \in ]a,b[$. Además, por ser $f$ continua, tenemos que $f(a)=0=f(b)$. Por tanto, $f=0$.
\end{ejercicio}

\begin{ejercicio} Sea $f:[0,1]\to [0,1]$ una función continua verificando que $$\int_0^xf(t)dt = \int_x^1f(t)dt \quad \forall x\in [0,1]$$ Demostrar que $f=0$.\\

    En la desigualdad dada, como es cierto $\forall x\in [0,1]$, tomamos límite con $x\longrightarrow 0$:
    \begin{equation*}
        0=\lim_{x\to 0}\int_0^xf(t)dt = \lim_{x\to 0}\int_x^1f(t)dt = \int_0^1f(t)dt
    \end{equation*}

    Al ser $f(x)\geq 0\quad \forall x\in [0,1]$, por el ejercicio anterior tenemos que $f=0$. No obstante, se puede razonar de otra manera como sigue:

    Además, por ser $f$ integrable tenemos que,  $\forall x\in [0,1]$ tenemos que:
    \begin{equation*}
        0= \int_0^1f(t)dt = \int_0^x f(t)dt+ \int_x^1f(t)dt
    \end{equation*}

    Además, por el la condición del enunciado, tenemos que:
    \begin{equation*}
        \int_0^x f(t)dt = \int_x^1f(t)dt \Longrightarrow
        \int_0^x f(t)dt - \int_x^1f(t)dt = 0
    \end{equation*}

    Usando ambas ecuaciones, tengo que:
    \begin{equation*}
        0=\cancel{\int_0^x f(t)dt} - \int_x^1f(t)dt = \cancel{\int_0^x f(t)dt}+ \int_x^1f(t)dt
        \Longrightarrow \int_x^1 -f(t)dt = \int_x^1f(t)dt
    \end{equation*}

    Por tanto, como es cierto $\forall x\in [0,1]$, tenemos que $f(x)=-f(x)\quad \forall x\in [0,1]$, entonces $ f=0$.
\end{ejercicio}

\begin{ejercicio}
    Sean $f,g:[a,b]\to \bb{R}$ funciones continuas tales que
    \begin{equation*}
        \int_a^b f(x)dx = \int_a^b g(x)dx
    \end{equation*}
    Demostrar que $\exists c\in [a,b]\mid f(c)=g(c)$.\\

    Sea $h:[a,b]\to \bb{R}$ dada por $h(x)=f(x)-g(x)$. Entonces, por la hipótesis del enunciado:
    \begin{equation*}
        \int_a^b h(x)\;dx = \int_a^b f(x)dx - \int_a^b g(x)dx = 0
    \end{equation*}

    Además, como $f,g$ son continuas tenemos que $h$ también. Por el Teorema del Valor Medio para las integrales, tenemos:
    \begin{equation*}
        \exists c\in [a,b] \mid \int_a^b h(x)\;dx = h(c)(b-a)
    \end{equation*}

    Por tanto, igualando tenemos que $\exists c\in [a,b]$ tal que:
    \begin{equation*}
        h(c)(b-a) = 0 \Longrightarrow h(c) = 0 \Longrightarrow f(c)-g(c) = 0 \Longrightarrow f(c)=g(c)
    \end{equation*}

    Por tanto, queda demostrado que $\exists c\in [a,b]\mid f(c)=g(c)$.
\end{ejercicio}

\begin{ejercicio} Sea $f[a,b]\to \bb{R}$ una función continua y sea $\alpha\in \bb{R}$. Demostrar que $\alpha=\int_a^bf(x)dx$ si, y solo si, para cada partición $P\in \mathscr{P}[a,b]$ existe al menos una suma de Riemann $\sigma(f,P)\mid \alpha=\sigma(f,P)$.\\

    Procedemos mediante doble implicación:
    \begin{description}
        \item [$\Longrightarrow$)] Suponemos que $\alpha=\int_a^bf(x)dx$, y comprobemos que $\forall P\in \mathscr{P}[a,b]$ existe al menos una suma de Riemann $\sigma(f,P)\mid \alpha=\sigma(f,P)$.

        Sea $P=\{x_0=a,x_1,\dots,x_n=b\}$ una partición cualquiera. Por la aditividad respecto del intervalo de integración, tenemos:
        \begin{equation*}
            \int_a^b f(x)dx = \int_{x_0}^{x_1} f(x)dx + \int_{x_1}^{x_2} f(x)dx + \dots + \int_{x_{n-1}}^{x_n} f(x)dx = \sum_{i=1}^n \int_{x_{i-1}}^{x_i} f(x)\;dx
        \end{equation*}
    
        Aplicando el Teorema del valor medio integral en cada una de las integrales,
        \begin{equation*}
            \int_a^b f(x)dx = f(c_1)(x_1-x_0) + \dots + f(c_n)(x_n-x_{n-1}) = \sum_{i=1}^n f(c_i)(x_i - x_{i-1})
        \end{equation*}
    
        con $c_i\in [x_{i-1},x_i]$. Por tanto, como $\sum_{i=1}^n f(c_i)(x_i - x_{i-1})=\sigma(f,P)$ con etiquetas $c_i$, tenemos que:
        \begin{equation*}
            \alpha = \int_a^bf(x)dx = \sigma(f,P)
        \end{equation*}


        \item [$\Longleftarrow$)] Suponemos que $\forall P\in \mathscr{P}[a,b]$ existe al menos una suma de Riemann $\sigma(f,P)\mid \alpha=\sigma(f,P)$, y veamos que $\alpha=\int_a^bf(x)dx$.


        Sea $P_n\in \mathscr{P}[a,b]$ una sucesión de particiones con $\displaystyle \lim_{n\to\infty}\Delta P_n = 0$ (algo que siempre es posible, usando particiones encajadas, por ejemplo).

        Por tanto, tengo que:
        \begin{equation*}
            \int_a^b f(x)dx = \lim_{n\to \infty}  \sigma(P_n,f) = \alpha
        \end{equation*}

        
        
    \end{description}
    
    

    
\end{ejercicio}

\begin{ejercicio} Sea $r>0$ y sea $f[-r,r]\to\bb{R}$ una función continua. Demostrar que:
\begin{enumerate}
    \item Si $f$ es par, entonces $\int_{-r}^r f(x)dx = 2\int_0^r f(x)dx$

    Suponemos $f$ par, es decir, $f(x)=f(-x)$. Entonces:
    \begin{equation*}
        \int_{-r}^r f(x)dx = 
        \int_{-r}^0 f(x)dx + \int_0^r f(x)dx 
    \end{equation*}

    Para resolver la primera integral, aplico el cambio de variable $-x=t$:
    \begin{equation*}
        \int_{-r}^0 f(x)dx
        = \MetInt{-x=t}{-dx = dt}
        = -\int_{r}^{0} f(-t)dt
        = \int_{0}^{r} f(-t)dt
    \end{equation*}

    Como $f$ es par, tenemos que:
    \begin{equation*}
        \int_{-r}^0 f(x)dx
        = \int_{0}^{r} f(t)dt
    \end{equation*}

    Por tanto, como la variable es muda,
    \begin{equation*}
        \int_{-r}^r f(x)dx = 
        \int_{-r}^0 f(x)dx + \int_0^r f(x)dx 
        = \int_{0}^{r} f(t)dt + \int_0^r f(x)dx 
        = 2\int_0^r f(x)dx 
    \end{equation*}
    
    \item Si $f$ es impar, entonces $\int_{-r}^r f(x)dx = 0$

    Suponemos $f$ impar, es decir, $-f(x)=f(-x)$. Entonces:
    \begin{equation*}
        \int_{-r}^r f(x)dx = 
        \int_{-r}^0 f(x)dx + \int_0^r f(x)dx 
    \end{equation*}

    Para resolver la primera integral, aplico el cambio de variable $-x=t$:
    \begin{equation*}
        \int_{-r}^0 f(x)dx
        = \MetInt{-x=t}{-dx = dt}
        = -\int_{r}^{0} f(-t)dt
        = \int_{0}^{r} f(-t)dt
    \end{equation*}

    Como $f$ es impar, tenemos que:
    \begin{equation*}
        \int_{-r}^0 f(x)dx
        = -\int_{0}^{r} f(t)dt
    \end{equation*}

    Por tanto, como la variable es muda,
    \begin{equation*}
        \int_{-r}^r f(x)dx = 
        \int_{-r}^0 f(x)dx + \int_0^r f(x)dx 
        = -\int_{0}^{r} f(t)dt + \int_0^r f(x)dx 
        = 0
    \end{equation*}
    
\end{enumerate}
\end{ejercicio}


\begin{ejercicio} Demostrar que si $f:\bb{R}\to \bb{R}$ es una función continua y periódica de periodo $T$, entonces, para cada $x\in \bb{R}$ se cumple que:
\begin{equation*}
    \int_x^{x+T}f(t)dt = \int_0^T f(t+x)dt
\end{equation*}

    Notando que $x\in \bb{R}$ es fijo, tenemos que:
    \begin{equation*}
        \int_x^{x+T}f(t)dt
        = \MetInt{t-x=z}{dt=dz}
        = \int_0^{T}f(z+x)dz
    \end{equation*}

    Como las variables $t,z$ son mudas, tenemos demostrado lo pedido.

    \begin{comment}
    Consideremos $x\in \bb{R}$ fijo. Sea $P\in \mathscr{P}[0,T]$. Se tiene que $P$ tiene asociada una única partición $\hat{P}\in \hat{\mathscr{P}}[x,x+~T]$ con $\hat{x_k} = x_k + x$. Sean $P,\hat{P}$ las siguientes particiones:
    \begin{gather*}
        P=\{x_0=0,x_1,\dots,x_n=T\} \\
        \hat{P}=\{\hat{x_0}=x,\hat{x_1}=x_1+x,\dots,\hat{x_n}=x_n+x=T+x\}
    \end{gather*}

    Análogamente, cada partición $\hat{P}\in \hat{\mathscr{P}}[x,x+T]$ tiene asociada una única partición $P\in \mathscr{P}[0,T]$ con $x_k = \hat{x_k}-x$.
    \begin{gather*}
        \hat{P}=\{\hat{x_0}=x,\hat{x_1},\dots,\hat{x_n}=T+x\}\\
        P=\{x_0=0,{x_1}=\hat{x_1}-x,\dots,x_n = \hat{x_n}-x=T\}
    \end{gather*}

    Es decir, tenemos la siguiente biyección:
    \begin{equation*}\begin{split}
        \mathscr{P}[0,T]& \Longleftrightarrow \hat{\mathscr{P}}[x,x+T] \\
        P &\;\;\overset{-x}{\underset{+x}{\leftrightarrows}}\;\; \hat{P}
    \end{split}\end{equation*}

    Por tanto, se tiene que:
    \begin{equation*}
        \Delta P_n \to 0 \Longleftrightarrow \Delta \hat{P}_n \to 0 \qquad\qquad (\Delta P = \Delta \hat{P} \quad\forall P,\hat{P})
    \end{equation*}


    Notemos $f(t+x) = h(t)$. Veamos ahora que $\sigma (h, {P_n}) = \sigma (f, \hat{P_n})$:
    \begin{equation*}
        \sigma (h, {P_n})
        = \sum_{k=1}^n h(t_k)({x_k}-{x_{k-1}})
        = \sum_{k=1}^n f(t_k+x)({x_k}-{x_{k-1}})
        = \sum_{k=1}^n f(\hat{t_k})(\hat{x_k}-\hat{x_{k-1}})
        = \sigma (f, \hat{P_n})
    \end{equation*}
    donde $t_k\in [x_k, x_{k-1}]$.

    Por tanto, seleccionando $P_n$ tal que $\Delta P_n\to 0$, tenemos que:
    \begin{equation*}
        \int_x^{x+T}f(t)dt = \sigma (f, \hat{P_n}) = \sigma (h, {P_n}) = \int_0^T f(t+x)dt
    \end{equation*}
    \end{comment}
\end{ejercicio}

\begin{ejercicio} Calcular los siguientes límites:
\begin{enumerate}
    \item $\displaystyle \lim_{n\to\infty} \left(\frac{1}{n+1} + \frac{1}{n+2} + \dots + \frac{1}{2n} \right)$

    \begin{multline*}
        \lim_{n\to\infty} \left(\frac{1}{n+1} + \frac{1}{n+2} + \dots + \frac{1}{2n} \right)
        = \lim_{n\to\infty} \frac{1}{n}\left(\frac{1}{1+\frac{1}{n}} + \frac{1}{1+\frac{2}{n}} + \dots + \frac{1}{1+\frac{n}{n}} \right)
        =\\= \lim_{n\to\infty} \frac{1}{n} \sum_{k=1}^n \frac{1}{1+\frac{k}{n}}
    \end{multline*}

    Definiendo $f(x)=\frac{1}{1+x}$, tenemos que el límite pedido es:
    \begin{equation*}
        \lim_{n\to\infty} \left(\frac{1}{n+1} + \frac{1}{n+2} + \dots + \frac{1}{2n} \right)
        = \lim_{n\to\infty} \frac{1}{n} \sum_{k=1}^n f\left(\frac{k}{n}\right)
        = \int_0^1 f(x)dx = \left[\ln |1+x|\right]_0^1 = \ln 2
    \end{equation*}

    \item $\displaystyle \lim_{n\to\infty} \left[n\left(\frac{1}{n^2 +1} + \frac{1}{n^2+4} + \frac{1}{n^2+ 9} + \dots + \frac{1}{2n^2} \right)\right]$
    \begin{multline*}
        \lim_{n\to\infty} \left[n\left(\frac{1}{n^2 +1} + \frac{1}{n^2+4} + \frac{1}{n^2+ 9} + \dots + \frac{1}{2n^2} \right)\right]
        =\\= \lim_{n\to\infty} \frac{1}{n}\left(\frac{1}{1+\frac{1}{n^2}} + \frac{1}{1+\frac{4}{n^2}} + \frac{1}{1+\frac{9}{n^2}} + \dots + \frac{1}{1+\frac{n^2}{n^2}} \right)
        =  \lim_{n\to\infty} \frac{1}{n}\sum_{k=1}^n \frac{1}{1+\left(\frac{k}{n}\right)^2}
    \end{multline*}

    Definiendo $f(x)=\frac{1}{1+x^2}$, tenemos que el límite pedido es:
    \begin{multline*}
        \lim_{n\to\infty} \left[n\left(\frac{1}{n^2 +1} + \frac{1}{n^2+4} + \frac{1}{n^2+ 9} + \dots + \frac{1}{2n^2} \right)\right]
        = \lim_{n\to\infty} \frac{1}{n}\sum_{k=1}^n f\left(\frac{k}{n}\right)
        =\\= \int_0^1 f(x)\;dx = \left[\arctan x\right]_0^1 = \frac{\pi}{4}
    \end{multline*}

    \item $\displaystyle \lim_{n\to\infty} \left[ \frac{1}{n} \left(\sen \frac{\pi}{n} +\sen \frac{2\pi}{n} +\dots+\sen \frac{n\pi}{n} \right)\right]$
    
    Definiendo $f(x)=\sen (\pi x)$, tenemos que el límite pedido es:
    \begin{equation*}
        \lim_{n\to\infty} \frac{1}{n} \sum_{k=1}^n \sen \left(\frac{k}{n}\right) = \int_0^1 f(x)\;dx = \left[-\frac{1}{\pi}\cos (\pi x)\right]_0^1 = \frac{2}{\pi}
    \end{equation*}
\end{enumerate}
\end{ejercicio}

\begin{ejercicio}
    Demostrar que la función $f(x)=\frac{e^x\sen x}{x}$ es integrable en $[0,1]$ verificándose que $0\leq \int_0^1 f(x)dx \leq e-1$.\\

    Tenemos que $f$ no está definida en $x=0$. 
    \begin{equation*}
        \lim_{x\to 0}f(x) = \lim_{x\to 0}\frac{e^x\sen x}{x}
        \Hop
        =\lim_{x\to 0}\frac{e^x(\sen x + \cos x)}{1} = e^0 = 1
    \end{equation*}

    Sabemos que $f$ es integrable en $]0,1]$ por ser acotada. Además, definiendo $f(0)=~1$ tenemos que $f$ es continua y, por tanto, es integrable en $[0,1]$.
    \begin{equation*}
        \int_0^1 f(x)\;dx = \lim_{c\to 0^+}\int_c^1 f(x)\;dx
    \end{equation*}
    
    Además, podemos definir $f(0)=k\in \bb{R}$, que como $f$ solo tendría una discontinuidad de salto finito, tenemos que no varía el valor de la integral. Por tanto, podemos suponer $f(0)=0$, lo cual no es restrictivo. Por tanto, tenemos que:
    \begin{equation*}
        0\leq f(x)=0 \leq e^x=1 \qquad x=0
    \end{equation*}

    Veamos ahora para $x\neq 0$:
    \begin{equation*}
        0 < f(x)=\frac{e^x\sen x}{x} < e^x \qquad \forall x\in ]0,1]
    \end{equation*}

    La primera desigualdad es trivial, ya que todos los términos son positivos. Para la segunda desigualdad, es necesario ver lo siguiente:
    \begin{equation*}
        \frac{e^x\sen x}{x} < e^x \Longleftrightarrow \sen x < x \Longleftrightarrow \sen x -x < 0 \qquad \forall x\in ]0,1]
    \end{equation*}

    Para ver que $\sen x - x < 0$, buscamos que la imagen de $h(x)=\sen x -x$ sea negativa.
    \begin{equation*}
        h(0)=0 \qquad\qquad h'(x)=\cos x -1 < 0 \Longrightarrow h\text{ estr. decreciente}
    \end{equation*}
    Por tanto, $Im(h)\subset \bb{R}^-$, por lo que se da lo buscado. En conclusión, tenemos que:
    \begin{equation*}
        0\leq f(x) \leq e^x \qquad \forall x\in [0,1]
    \end{equation*}

    Aplicamos el operador integral sabiendo que mantiene el orden:
    \begin{equation*}
        0 = \int_0^1 0\cdot dx \leq \int_0^1 f(x)\;dx \leq \int_0^1e^x\;dx = \left[e^x\right]_0^1 = e-1
    \end{equation*}

    Por tanto, hemos demostrado que 
    \begin{equation*}
        0 \leq \int_0^1 f(x)\;dx \leq e-1
    \end{equation*}
\end{ejercicio}

\begin{ejercicio}
    Sea $f:[a,b]\to \bb{R}$ una función integrable. Demostrar que si para cada $c,d\in [a,b]$ tales que $a<c<d<b$, $\exists x\in ]c,d[$ verificando que $f(x)=0$, entonces $\int_a^b f(x)dx = 0$.

    Sea una sucesión de particiones $P_n\in \mathscr{P}[a,b]$ con $\displaystyle \lim_{n\to \infty}\Delta P_n = 0$, algo que siempre es posible, por ejemplo, tomando sucesiones de puntos encajados. Denotemos dicha partición como:
    \begin{equation*}
        P_n = \{x_0=a,x_1\dots, x_n=b\}
    \end{equation*}

    Consideremos para cada $k=0,\dots,n-1$ el intervalo $[x_k, x_{k+1}]$, y denotemos $\xi_k\in [x_k, x_{k+1}]$ como el valor que verifica que $f(\xi_k)=0$. Este siempre existe por la hipótesis del enunciado. Entonces,
    \begin{equation*}
        \sum_{k=0}^{n-1}f(\xi_k)(x_{k+1}-x_k) = \sum_{k=0}^{n-1}0\cdot (x_{k+1}-x_k) = 0
    \end{equation*}

    No obstante, tenemos que dicha sumatoria es una suma intermedia $\sigma(f,P_n)$ en las etiquetas $\xi_k$. Por tanto, como por hipótesis tenemos que $f$ es integrable y hemos encontrado una sucesión de particiones cuyo diámetro tiende a $0$ y una suma parcial suya es nula, tenemos que:
    \begin{equation*}
        \int_a^b f(x) = \lim_{n\to \infty} \sigma(f, P_n) = 0
    \end{equation*}
\end{ejercicio}

\begin{ejercicio}
    Demostrar que la composición de dos funciones integrables puede no ser una función integrable.

    Sea $f_1:[1,0]\to \bb{R}$ la función de las Palomitas, que se ha demostrado que es integrable.
    \begin{equation*}
        f_1(x) = \left\{\begin{array}{ccc}
            0 & \text{si} & x\notin \bb{Q}\\
            \frac{1}{q} & \text{si} & x=\frac{p}{q},\;mcd(p,q)=1\\
            1 & \text{si} & x\in \{0,1\}
        \end{array} \right.
    \end{equation*}


    Sea $f_2:[1,0]\to \bb{R}$ la siguiente función, que es integrable por ser continua en todos los puntos excepto en $x=0$, donde presenta una discontinuidad de salto finito.
    \begin{equation*}
        f_2(x) = \left\{\begin{array}{ccc}
            1 & \text{si} & x\neq 0\\
            0 & \text{si} & x=0
        \end{array} \right.
    \end{equation*}

    Veamos el valor de la composición:
    \begin{equation*}
        \begin{array}{ccccc}
            x\notin \bb{Q} & \stackrel{f_1}{\longrightarrow} & 0 &\stackrel{f_2}{\longrightarrow} & 0\\
            x=\frac{p}{q} & \longrightarrow  & \frac{1}{q} & \longrightarrow  & 1 \\
            x \in \{0,1\} & \longrightarrow & 1& \longrightarrow  & 1
        \end{array}
    \end{equation*}


    Por tanto, tenemos que $f_2\circ f_1$ es la función de Dirichlet, que se ha visto que no es integrable.

    Por tanto, hemos visto que no se cumple que la composición de funciones integrables sea integrable. Para que se cumpliese, $f_2$ debía haber sido continua.

    

\end{ejercicio}

\begin{ejercicio}
    Sea $f:[a,b]\to \bb{R}$ una función acotada que es integrable. Sea $c\in \bb{R}$. Si $g:[a+c,b+c]\to \bb{R}$ está dada por $g(x)=f(x-c)$, para $x\in [a+c,b+c]$, demostrar que $g$ es integrable, siendo
    \begin{equation*}
        \int_a^b f(x)dx = \int_{a+c}^{b+c}g(x)dx
    \end{equation*}
    Dedúzcase que, para cada $h\in \bb{R}$,
    \begin{equation*}
        \int_a^b f(x)dx = \int_{a-h}^{b-h}f(x+h)dx
    \end{equation*}


    Para demostrar lo primero, realizo el cambio de variable $t=x+c$:
    \begin{equation*}
        \int_a^b f(x)dx = \MetInt{t=x+c}{dt=dx} =
        \int_{a+c}^{b+c} f(t-c)dt
    \end{equation*}

    Por tanto, la primera igualdad queda demostrada. Como se ha demostrado $\forall c\in~\bb{R}$, tomando $h=-c$ queda demostrada la segunda.
\end{ejercicio}

\begin{ejercicio}
    Sea $f:\bb{R}^+_0\to \bb{R}$ una función positiva y estrictamente decreciente. Demuéstrese que para cada $n,p\in \bb{N}$ se verifica que
    \begin{equation*}
        f(n+p) + \int_n^{n+p} f(x)dx < f(n)+f(n+1)+\dots+f(n+p)< f(n)+ \int_n^{n+p} f(x)dx
    \end{equation*}

    Como consecuencia demostrar que $1+\frac{1}{\sqrt{2}} + \dots + \frac{1}{\sqrt{p}} > \sqrt{p}\qquad p\geq 2$.\\

    Al ser monótona y acotada en $[n,n+p]$ tenemos que es Riemman Integrable. Por tanto,
    \begin{equation*}
        \int_{n}^{n+p}f(x)\;dx = S(f) = I(f)
    \end{equation*}
    Además, sabemos que $S(f)<S(f,P)$ para todo $P\in \mathscr{P}[n,n+p]$. Por tanto, suponiendo $P=\{n,n+1,\dots, n+p\}$, tenemos que:
    \begin{equation*}
        \int_n^{n+p}f(x)\;dx = S(f) < S(f,p) = \sum_{i=n}^{n+p-1}f(i)
    \end{equation*}
    Por tanto, y simplificando $f(n+p)$, tenemos la primera desigualdad. Demostremos ahora la segunda:
    \begin{equation*}
        \int_n^{n+p}f(x)\;dx = I(f) > I(f,p) = \sum_{i=n+1}^{n+p}f(i)
    \end{equation*}
    Simplificando en este caso $f(n)$, tenemos probada la segunda desigualdad.

    \vspace{1cm}
    Para demostrar que $1+\frac{1}{\sqrt{2}} + \dots + \frac{1}{\sqrt{p}} > \sqrt{p}\qquad p\geq 2$, tomamos $f(x)=\frac{1}{\sqrt{x}}$ para $n=1$. Entonces:
    \begin{multline*}
        \int_{1}^{p+1}f(x)\;dx = \left[2\sqrt{x}\right]_1^{p+1} = 2(\sqrt{p+1}-1)>\sqrt{p} \Longleftrightarrow
        2\sqrt{p+1}>\sqrt{p}+2
        \Longleftrightarrow \\ \Longleftrightarrow
        4p+4 > p + 4 + 4\sqrt{p}
        \Longleftrightarrow 3p>4\sqrt{p} \Longleftrightarrow 9p > 16 \Longleftrightarrow p>\frac{16}{9} \quad \text{¡Cierto!, ya que $p\geq 2$}
    \end{multline*}

    Por tanto, tenemos que $\int_{1}^{p+1}f(x)\;dx>\sqrt{p}$. Por tanto, por lo demostrado previamente, tenemos que:
    \begin{equation*}
        \sum_{i=1}^{p-1}f(i) = 1+\frac{1}{\sqrt{2}} + \dots + \frac{1}{\sqrt{p}} > \int_{1}^{p+1}f(x)\;dx>\sqrt{p}
    \end{equation*}

    Por tanto, queda demostrado que:
    \begin{equation*}
        1+\frac{1}{\sqrt{2}} + \dots + \frac{1}{\sqrt{p}} > \sqrt{p}
    \end{equation*}
\end{ejercicio}


\begin{ejercicio}
    Sea $f:\bb{R}\to \bb{R}$ la función dada por $f(x)=e^{-x^2}$, para cada $x\in \bb{R}$. Sea $h:\bb{R}\to \bb{R}$ la función dada por $h(x)=\int_0^{\sen x}f(t)dt$, para cada $x\in \bb{R}$.

    Demostrar que $h$ es derivable en $\bb{R}$ y calcular su derivada.\\
    
    Tenemos que $f$ es continua y acotada en $\bb{R}$, por lo que sabemos que es Riemman Integrable. Por el TFC, tenemos que $\int_0^{x}f(t)dt$ es derivable en $\bb{R}$. Además, como el seno también es derivable, tenemos que $h$ lo es, con
    \begin{equation*}
        h'(x) = f(\sen x)\cdot \cos x = e^{-\sen^2 x}\cos x
    \end{equation*}
\end{ejercicio}

\begin{ejercicio}
    Sea $f:\bb{R}\to \bb{R}$ la función dada por $f(x)=e^{-x^2}$, para cada $x\in \bb{R}$. Sea $g:\bb{R}^+_0\to \bb{R}$ la función dada por $g(x)=\int_0^{\sqrt{x}}f(t)dt$, para cada $x\in \bb{R}^+_0$.

    Estudiar la continuidad uniforme de la función $g$ y su derivabilidad.\\

    Tenemos que $f$ es continua y acotada en $\bb{R}$, por lo que sabemos que es Riemman Integrable. Por el TFC, tenemos que $\int_0^{x}f(t)dt$ es derivable en $\bb{R}^+_0$. Además, $\sqrt{x}$ es derivable en $\bb{R}^+$, por lo que $g$ es derivable en $\bb{R}^+$ con
    \begin{equation*}
        g'(x) = f(\sqrt{x})\cdot \frac{1}{2\sqrt{x}} = \frac{e^{-x}}{2\sqrt{x}}
    \end{equation*}

    Veamos si es derivable en $x=0$:
    \begin{equation*}
        g'(0)=\lim_{x\to 0}g'(x) = \frac{1}{0} = +\infty \Longrightarrow \nexists g'(0)
    \end{equation*}
    Por tanto, tenemos que $g$ es derivable en $\bb{R}^+$.

    \vspace{1cm} Veamos ahora si es uniformemente continua en $\bb{R}^+_0$. 

    Por el teorema de Heine, tenemos que $g(x)$ es uniformemente continua en $[0,r]\quad \forall r>~0$. Supongamos $r>0$ y veamos que es lipschitziana. Sabemos que $g'(x)$ es estrictamente decreciente en $\bb{R}^+$, y tenemos que:
    \begin{equation*}
        g'(r)=\frac{e^{-r}}{2\sqrt{r}} = \frac{1}{e^r \sqrt{r}}\in \bb{R}
        \qquad \lim_{x\to \infty}g'(x)=0
    \end{equation*}

    Por tanto, como $g'(x)$ está acotada $\forall \;r>0$, tenemos que $g$ es lipschitziana y por tanto uniformemente continua en $[r,+\infty[$. Como además hemos visto que es uniformemente continua en $[0,r[$ y sabemos que es continua en $r$, tenemos que $g$ es uniformemente continua en $\bb{R}^+_0$.
    
\end{ejercicio}

\begin{ejercicio}
    Estudiar la derivabilidad de la función $F:\bb{R}^+_0 \to \bb{R}$ dada por $F(x)=~\int_{\sqrt[3]{x}}^{x^2} \frac{t}{\sqrt{1+t^3}} dt$ y calcular $F'(1)$.\\

    Sea $\displaystyle f(t)=\frac{t}{\sqrt{1+t^3}}$. Tenemos que $f$ es continua, por lo que es Riemman Integrable. Por tanto, dado $c\in [\sqrt[3]{x}, x^2]$ tenemos que:
    \begin{equation*}
        F(x)=\int_{\sqrt[3]{x}}^{x^2} \frac{t}{\sqrt{1+t^3}} dt
        = \int_{\sqrt[3]{x}}^{c} \frac{t}{\sqrt{1+t^3}} dt
        + \int_{c}^{x^2} \frac{t}{\sqrt{1+t^3}} dt
        = \int_{c}^{x^2} \frac{t}{\sqrt{1+t^3}}
        - \int_c^{\sqrt[3]{x}} \frac{t}{\sqrt{1+t^3}} dt
    \end{equation*}

    Por tanto, por el TFC tenemos que $F$ es derivable en $\bb{R}^+_0$, con:
    \begin{equation*}
        F'(x) = \frac{2x^2}{\sqrt{1+x^3}} - \frac{x}{\sqrt{1+x^3}}\cdot \frac{1}{3\sqrt[3]{x^2}}
    \end{equation*}

    Por tanto,
    \begin{equation*}
        F'(1) = \frac{2}{\sqrt{2}} - \frac{1}{3\sqrt{2}} = \frac{5\sqrt{2}}{6}
    \end{equation*}
\end{ejercicio}

\begin{ejercicio}
    Probar que todas las funciones continuas $f:\bb{R}_0^+ \to \bb{R}$ que verifican la igualdad $\int_0^x f(t)dt = \frac{x}{3}f(x)$ son derivables en $\bb{R}_0^+$. Determinar el conjunto de dichas funciones.

    Sea $F(x)=\int_0^x f(t)dt = \frac{x}{3}f(x)$. Por tanto, por el TFC, como $f$ es continua entonces $F$ es derivable en $\bb{R}^+_0$, con $F'=f$. Por tanto,
    \begin{equation}\label{Ej18.DerivadaF'}
        F'(x)=f(x) = \left[\frac{x}{3}f(x)\right]' = \frac{1}{3}[f(x)+xf'(x)]
    \end{equation}
    donde, como $f$ es continua, tenemos que es derivable.

    \vspace{1cm}Calculamos ahora el conjunto de dichas funciones $f$.

    \begin{itemize}
        \item \underline{Supuesto $f = 0$}:
        
        Se tiene, ya que $f$ es una recta constante en 0. Se cumple la igualdad dada.

        \item \underline{Supuesto $f(x) \neq 0$ en $[a,b]$, con $0<a<b$}:
    
        De la ecuación \ref{Ej18.DerivadaF'}, tenemos que:
        \begin{equation*}
            3f(x)=f(x)+xf'(x) \Longrightarrow 2f(x)=xf'(x)
            \Longrightarrow
            \frac{f'(x)}{f(x)} = \frac{2}{x} \qquad \forall x\in [a,b]
        \end{equation*}

        Por tanto, tenemos que:
        \begin{multline*}
            \int_a^x \frac{f'(t)}{f(t)}\;dt
            = \int_a^x \frac{2}{t}\;dt
            \Longrightarrow \ln |f(x)| - \ln|f(a)| = 2\ln x - 2\ln a
            \Longrightarrow \\ \Longrightarrow
            \ln|f(x)| = \ln(x^2) +\ln|f(a)| -\ln(a^2)
        \end{multline*}

        Como $a\in \bb{R}^+$ es fijo, definimos:
        \begin{equation*}
            K:=\frac{|f(a)|}{a^2}>0 \qquad \qquad
            \ln K = \ln \left(\frac{|f(a)|}{a^2}\right) = \ln |f(a)|-\ln(a^2)
        \end{equation*}

        Por tanto, tenemos que:
        \begin{equation*}
            \ln|f(x)| = \ln(x^2) + \ln K = \ln (Kx^2)
        \end{equation*}

        Como el $\ln(x)$ es una función inyectiva, tenemos que:
        \begin{equation*}
            |f(x)|=Kx^2 \Longrightarrow f(x)=\pm Kx^2 \qquad K\in \bb{R}^+
        \end{equation*}
        Es decir, $f$ es una parábola.

        Veamos ahora cómo de grande es ese intervalo $[a,b]$.
        
        Supongamos que $\inf\{0\leq \hat{a} \leq a \mid f(x)\neq 0 \;\forall x\in [\hat{a},b]\}>0$. Entonces:
        \begin{equation*}
            \exists a_n\to \hat{a} \quad \mid \quad 0\leq a_n\leq \hat{a} \;\land\;f(a_n)=0\;\forall n\in \bb{N}
        \end{equation*}

        Como tenemos que $f$ es continua, por la elección de $a_n$ y $\hat{a}$, tenemos:
        \begin{equation*}
            \lim_{x\to \hat{a}^-} f(x) = \lim_{x\to \hat{a}^-} 0 = 0
        \end{equation*}

        No obstante, tenemos que $f$ es continua y es del tipo $f(x)=kx^2\quad \forall x\in [\hat{a},b]$, $k\neq 0$. Por tanto,
        \begin{equation*}
            \lim_{x\to \hat{a}^+} f(x)= \lim_{x\to \hat{a}^+} kx^2 = k\hat{a}^2 \neq 0 \qquad \text{ya que }k,\hat{a}\neq 0
        \end{equation*}
        

        Por tanto, hemos llegado a que en $\hat{a}$ la función no es continua, por lo que es una contradicción. Tenemos entonces que $\hat{a}=0$.


        Supongamos que $\max \{b\leq \hat{b} \mid f(x)\neq 0 \;\forall x\in [0,\hat{b}]\}\in \bb{R}$.
        Entonces:
        \begin{equation*}
            \exists b_n\to \hat{b} \quad \mid \quad b_n> \hat{b} \;\land\;f(b_n)=0\;\forall n\in \bb{N}
        \end{equation*}

        Como tenemos que $f$ es continua, por la elección de $b_n$ y $\hat{b}$, tenemos:
        \begin{equation*}
            \lim_{x\to \hat{b}^+} f(x) = \lim_{x\to \hat{b}^+} 0 = 0
        \end{equation*}

        No obstante, tenemos que $f$ es continua y es del tipo $f(x)=kx^2\quad \forall x\in [0,\hat{b}]$, $k,\hat{b}\neq 0$. Por tanto,
        \begin{equation*}
            \lim_{x\to \hat{b}^-} f(x)= \lim_{x\to \hat{b}^-} kx^2 = k\hat{b}^2 \neq 0 \qquad \text{ya que }k,\hat{b}\neq 0
        \end{equation*}
        

        Por tanto, hemos llegado a que en $\hat{b}$ la función no es continua, por lo que es una contradicción. Tenemos entonces que $\hat{b}\notin \bb{R}$.

        Por tanto, hemos llegado a que el conjunto pedido es:
        \begin{equation*}
            \{f=0\}\cup \{f\mid f(x)=kx^2,\;\;k\neq 0\} = \{f\mid f(x)=kx^2,\;\;k\in \bb{R}\}
        \end{equation*}

        Por tanto, el conjunto pedido es el de las parábolas con vértice en el $(0,0)$.

        
    \end{itemize}
    
\end{ejercicio}

\begin{ejercicio}
    Sea $f:\bb{R}\to \bb{R}$ la función continua. Justificar que la función dada por $H(x)=\int_{x^2}^{x^3}f(t)dt$, para cada $x\in \bb{R}$, es derivable y calcular su derivada.

    Como $f$ es continua, tenemos que es Riemman Integrable. Por tanto, dado $c~\in~[x^2,x^3]$, tenemos que:
    \begin{equation*}
        H(x)=\int_{x^2}^{x^3}f(t)dt =\int_{x^2}^{c}f(t)dt + \int_{c}^{x^3}f(t)dt = \int_{c}^{x^3}f(t)dt - \int_{c}^{x^2}f(t)dt
    \end{equation*}

    Por el TFC, como $f$ es Riemman Integrable y continua, tenemos que $H'$ es derivable, com:
    \begin{equation*}
            H'(x) = 3x^2f(x^3) -2xf(x^2)
    \end{equation*}
    
\end{ejercicio}

\begin{ejercicio}
    Calcular la derivada de cada una de las siguientes funciones:
    \begin{enumerate}
        \item $F_1(x) = \displaystyle \int_0^x \sen^3 t dt$

        Como tenemos que el integrando es continuo y acotado en $\bb{R}$, por el TFC tenemos que $F_1$ es derivable en $\bb{R}$ con:
        \begin{equation*}
            F_1'(x) = \sen^3 x
        \end{equation*}
        
        \item $F_2(x) = \int_x^b \frac{1}{1+t^2+\sen^2 t} dt$

        Reescribo $F_2$ para que se le pueda aplicar el TFC:
        \begin{equation*}
            F_2(x) = -\int_b^x \frac{1}{1+t^2+\sen^2 t} dt
        \end{equation*}
        
        Como tenemos que el integrando es continuo y acotado en $\bb{R}$, por el TFC tenemos que $F_2$ es derivable en $\bb{R}$ con:
        \begin{equation*}
            F_2'(x) = -\frac{1}{1+x^2+\sen^2 x}
        \end{equation*}
        
        \item $F_3(x) = \int_a^b \frac{x}{1+t^2+\sen^2 t} dt$

        Como estamos integrando respecto a $t$, tenemos que $x$ es una constante. Por tanto,
        \begin{equation*}
            F_3(x) = x\cdot \int_a^b \frac{1}{1+t^2+\sen^2 t} dt
        \end{equation*}

        Además, como la integral es propia y el integrando es Riemman Integrable, tenemos que tomará un valor real. Es decir, tomará un valor constante y fijo, que no depende en ningún momento de $x$. Por tanto,
        \begin{equation*}
            F_3'(x) = \int_a^b \frac{1}{1+t^2+\sen^2 t} dt
        \end{equation*}
        
    \end{enumerate}
\end{ejercicio}

\begin{ejercicio}
    Sea $f:\bb{R}^+\to \bb{R}$ la función definida por $f(x)=\int_0^{x^3-x^2}e^{-t^2} dt$. Estudiar los intervalos de crecimiento y decrecimiento y determinar los extremos relativos de dicha función. Calcular $\displaystyle \lim_{x\to 0}\frac{f(x)}{\sen(x^3-x^2)}$.

    Como el integrando es continuo y acotado en $\bb{R}$, por el TFC tenemos que $f$ es derivable en $\bb{R}^+$ con:
    \begin{equation*}
        f'(x)=e^{-(x^3-x^2)^2} (3x^2-2x) = 0 \Longleftrightarrow x(3x-2)=0 \Longleftrightarrow x=\left\{0,\frac{2}{3}\right\}
    \end{equation*}

    Estudiamos la monotonía:
    \begin{itemize}
        \item \underline{Para $x\in \left]0, \frac{2}{3}\right[$}: $f'(x)<0\Longrightarrow f(x)$ estrictamente decreciente.
        \item \underline{Para $x\in \left]\frac{2}{3},+\infty\right[$}: $f'(x)>0\Longrightarrow f(x)$ estrictamente creciente.
    \end{itemize}

    Por tanto, tenemos que $x=\frac{2}{3}$ es un mínimo relativo y absoluto, con $f\left(\frac{2}{3}\right)=0$.
    
    Además, como $f$ es derivable en $\bb{R}^+$, tenemos que una condición necesaria de ser extremo relativo es que se anule la primera derivada. Por tanto, tenemos que no hay más posibles extremos relativos. El 0 no lo consideramos, ya que $0\notin \bb{R}^+$.

    \vspace{1cm}Resolvemos ahora el límite:
    \begin{equation*}
        \lim_{x\to 0}\frac{f(x)}{\sen(x^3-x^2)} = \left[\frac{0}{0}\right] \Hop
        \lim_{x\to 0}\frac{e^{-(x^3-x^2)^2} \cancel{(3x^2-2x)}}{\cos(x^3-x^2)\cdot \cancel{(3x^2-2x)}} = \frac{e^0}{\cos 0} = 1
    \end{equation*}
    
\end{ejercicio}


\begin{ejercicio}
    Sea $f:[1,\infty[ \to\bb{R}$ la función definida por $f(x)=\int_0^{x-1}(e^{-t^2}-e^{-2t})dt$. Calcular su máximo absoluto. Sabiendo que $\displaystyle \lim_{x\to+\infty}f(x) = \frac{1}{2}(\sqrt{\pi}-1)$, calcular el mínimo absoluto de $f$.

    Como tenemos que el integrando es continuo y acotado en $\bb{R}$, tenemos que $f(x)$ es derivable en $[1,\infty[$ por el TFC. Por tanto,
    \begin{equation*}
        f'(x) = e^{-(x-1)^2} -e^{-2(x-1)} = 0 \Longleftrightarrow (x-1)^2 = 2(x-1) \Longleftrightarrow x^2-4x+3=0
    \end{equation*}

    Por tanto, resolviendo la ecuación, tenemos que los puntos que anulan la primera derivada son $x=\{1,3\}$. Estudiamos la monotonía:
    \begin{itemize}
        \item \underline{Para $x\in ]1, 3[$}: $f'(2)=e^{-1}-e^{-2}>0\Longrightarrow f(x)$ estrictamente creciente.
        \item \underline{Para $x\in ]1,+\infty[$}: $f'(4)=e^{-9}-e^{-6}<0\Longrightarrow f(x)$ estrictamente decreciente.
    \end{itemize}


    Por tanto, tenemos que $x=3$ es un máximo relativo que, además, es máximo absoluto.

    Para estudiar el mínimo absoluto, calculamos la imagen de $x=1$ y el límite en $+\infty$:
    \begin{equation*}
        f(1)=\int_0^{0}(e^{-t^2}-e^{-2t})dt=0 \qquad \lim_{x\to+\infty}f(x) = \frac{1}{2}(\sqrt{\pi}-1)>0
    \end{equation*}

    Como $f(1)$ es menor, tenemos que $x=1$ es el mínimo absoluto.
    
\end{ejercicio}

\begin{ejercicio}
    Sea $H:[-1, 1]\to \bb{R}$ la función dada por $H(x) = \int_0^{\pi x^2} e^{2t}\sen t dt$. Estudiar los extremos absolutos y relativos de la función $H$ y determinar su imagen.\\

    Por el TFC, como el integrando es una funcion continua y acotada, tenemos que $F$ es derivable en $[-1,1]$, con:
    \begin{multline*}
        H'(x)=e^{2\pi x^2}\sen(\pi x^2)\cdot 2\pi x = 0 \Longleftrightarrow \\ \Longleftrightarrow \left\{\begin{array}{l}
            2\pi x = 0 \Longleftrightarrow x=0 \\
            \lor \\
            \sen(\pi x^2)=0 \Longleftrightarrow \pi x^2 =\pi k,\;k\in \bb{Z} \Longleftrightarrow x^2=k,\;k\in \bb{Z} \Longleftrightarrow x=\{-1,0,1\}
        \end{array}\right.
    \end{multline*}

    Estudiamos, por tanto, la monotonía de $H(x)$:
    \begin{itemize}
        \item \underline{Para $x\in ]-1, 0[$}: $H'(x)<0\Longrightarrow H(x)$ estrictamente decreciente.
        \item \underline{Para $x\in ]0,1[$}: $H'(x)>0\Longrightarrow H(x)$ estrictamente creciente.
    \end{itemize}

    Por tanto, en $x=0$ tenemos un mínimo relativo.
    \begin{equation*}
        H(0)=\int_0^0 e^{2t}\sen t dt = 0
    \end{equation*}
    \begin{equation*}\begin{split}
        H(-1)&=H(1)=\int_0^{\pi} e^{2t}\sen t dt = \MetInt{u(t)=e^{2t} \quad u'(t)=2e^{2t}}{v'(t)=\sen t \quad v(t)=-\cos t} 
        =\\&= \left[-e^{2t}\cos t\right]_0^\pi +2\int_0^\pi e^{2t}\cos t\;dt = \MetInt{u(t)=e^{2t} \quad u'(t)=2e^{2t}}{v'(t)=\cos t \quad v(t)=\sen t} 
        =\\&= \left[-e^{2t}\cos t +2e^{2t}\sen t\right]_0^\pi -4\int_0^\pi e^{2t}\sen t\;dt
    \end{split}\end{equation*}

    Por tanto, resolviendo la integral cíclica tenemos:
    \begin{multline*}
        5H(-1) = 5\int_0^{\pi} e^{2t}\sen t dt = \left[-e^{2t}\cos t +2e^{2t}\sen t\right]_0^\pi 
        \Longrightarrow \\ \Longrightarrow H(-1)=H(1) = \frac{\left[-e^{2t}\cos t +2e^{2t}\sen t\right]_0^\pi}{5} = \frac{e^{2\pi} +1}{5}
    \end{multline*}

    Por tanto, tenemos que:
    \begin{equation*}
        Im(H) = \left[0,\frac{e^{2\pi} +1}{5}\right]
    \end{equation*}

    Los máximos absolutos son:
    \begin{equation*}
        \left(-1, \frac{e^{2\pi} +1}{5}\right) \qquad \left(1, \frac{e^{2\pi} +1}{5}\right)
    \end{equation*}

    El mínimo absoluto, que además es mínimo relativo, es $$(0,0)$$
\end{ejercicio}

\begin{ejercicio}
    Probar que la función $f:[1,2]\to \bb{R}$ dada por $f(y)=\int_0^1 \frac{dx}{\sqrt{x^4+y^4}}$ es lipschitziana.

    Ver si es Lipschiziana desde la definición es complejo, por lo que veamos si la derivada $f'(y)$ es acotada. Para poder calcular la derivada mediante el TFC, buscados la variable $y$ en los límites de integración.
    \begin{equation*}
        f(y)=
        \int_0^1 \frac{dx}{\sqrt{x^4+y^4}} =
        \int_0^1 \frac{dx}{y^2\sqrt{\frac{x^4}{y^4}+1}} =
        \frac{1}{y^2} \int_0^1 \frac{dx}{\sqrt{\left(\frac{x}{y}\right)^4+1}}
    \end{equation*}
    donde la constante $\frac{1}{y^2}$ no se ve afectada por la integral ya que se integra respecto a $x$. Llegados a este punto, aplicamos el siguiente cambio de variable:
    \begin{equation*}
        \MetInt{\frac{x}{y}=u(x)}{\frac{dx}{y}=du}
    \end{equation*}
    Entonces:
    \begin{equation*}
        f(y)=
        \frac{1}{y^2} \int_0^1 \frac{dx}{\sqrt{\left(\frac{x}{y}\right)^4+1}}
        = \frac{1}{y^2} \int_{u(0)}^{u(1)} \frac{y\;du}{\sqrt{u^4+1}}
        = \frac{1}{y} \int_{0}^{1/y} \frac{du}{\sqrt{u^4+1}}
    \end{equation*}

    Por tanto, calculamos la derivada empleando la regla del producto:
    \begin{equation*}\begin{split}
        f'(y)&=-\frac{1}{y^2} \int_0^{1/y} \frac{1}{\sqrt{u^4+1}}\;du + \frac{1}{y} \left[\int_{0}^{1/y} \frac{du}{\sqrt{u^4+1}}\right]'
        =-\frac{1}{y} f(y) - \frac{1}{y^3} \frac{1}{\sqrt{\left(\frac{1}{y}\right)^4+1}} =
        \\&= -\frac{1}{y} f(y) - \frac{1}{y} \frac{1}{\sqrt{1+y^4}}
        = -\frac{1}{y}\left(f(y)+\frac{1}{\sqrt{1+y^4}}\right)
    \end{split}\end{equation*}

    donde he empleado que, por el TFC:
    \begin{equation*}
        \left[\int_{0}^{\frac{1}{y}} \frac{du}{\sqrt{u^4+1}}\right]' = \frac{1}{\sqrt{\left(\frac{1}{y}\right)^4+1}}\cdot \left(\frac{1}{y}\right)' = \frac{1}{\sqrt{\left(\frac{1}{y}\right)^4+1}}\cdot \left(-\frac{1}{y^2}\right)
    \end{equation*}

    Tenemos que la derivada es acotada, ya que:
    \begin{multline*}
        |f'(y)| = \left|-\frac{1}{y}\left(f(y)+\frac{1}{\sqrt{1+y^4}}\right)\right|
        = \frac{1}{y}\left(f(y)+\frac{1}{\sqrt{1+y^4}}\right)
        \stackrel{y\in[1,2]}{\leq}\\\leq \frac{1}{1} \left(f(y)+\frac{1}{\sqrt{1+1^4}}\right)
        \stackrel{(\ast)}{\leq} M+\frac{1}{\sqrt{2}}\in \bb{R}
    \end{multline*}

    donde en $(\ast)$ he usado que, por el TFC, $f(y)$ es continua y, por el Teorema de Bolzano-Weirstrass, $f(y)$ es acotada.
    \begin{equation*}
        0\leq f(y) \leq M\in \bb{R}
    \end{equation*}

    Por tanto, como su derivada es acotada, tenemos que $f(y)$ es lipschitziana.
\end{ejercicio}

\begin{ejercicio}
    Dado $a>0$, calcular la imagen de la función $G:[-a,a]\to\bb{R}$ dada por $G(x) = \int_{-x}^x \sqrt{a^2-t^2}dt$.\\

    En primer lugar, aplicamos que el integrando es una función par en la variable $t$, por lo que:
    \begin{equation*}
        G(x) = \int_{-x}^x \sqrt{a^2-t^2}dt = G(x) = 2\int_{0}^x \sqrt{a^2-t^2}dt
    \end{equation*}

    Para calcular la imagen, es necesario calcular $G'(x)$. Aplicando el TFC, tenemos que:
    \begin{equation*}
        G'(x)=2\sqrt{a^2-x^2} \geq 0 \qquad \forall x\in [-a, a]
    \end{equation*}

    Por tanto, tenemos que $G$ es creciente. Además, por el Teorema de Bolzano-Weierstrass tenemos que $Im(G)$ es cerrado y acotado, por lo que:
    \begin{equation*}
        Im(G)=[G(-a), G(a)]
    \end{equation*}
    
    \begin{description}
        \item [Opción 1: Analíticamente.]
        
        Para calcular las imágenes, calculo en primer lugar la integral indefinida.
        \begin{multline*}
            \int \sqrt{a^2-t^2}dt = \MetInt{t=a\sen u \qquad u\in \left[0,\frac{\pi}{2}\right]}{dt = a\cos u\;du} = \int \sqrt{a^2-a^2\sen^2 u}\;a\cos u\;du
            =\\=\int a\sqrt{1-\sen^2 u}\;a\cos u\;du
            =\int a^2\cos^2 u\;du
        \end{multline*}
    
        Para resolver la integral del coseno al cuadrado, aplico las siguientes identidades trigonométricas:
        \begin{equation*}
            \left.\begin{array}{c}
                \sen^2 u + \cos^2 u =1 \\
                \cos^2 u - \sen^2 u = \cos(2u) 
            \end{array}\right\} \Longrightarrow
            \cos^2 u = \frac{1+\cos(2u)}{2}
        \end{equation*}
    
        Por tanto, 
        \begin{equation*}
            \int \sqrt{a^2-t^2}dt = \int a^2\cos^2 u\;du = a^2\int \frac{1+\cos(2u)}{2}\;du = a^2 \left(\frac{u}{2} + \frac{1}{4}\sen(2u)\right) +C
        \end{equation*}
    
        Deshaciendo el cambio de variable $(u=\arcsen\left(\frac{t}{a}\right))$, llegamos a que:
        \begin{equation*}
            \int \sqrt{a^2-t^2}dt = a^2 \left(\frac{u}{2} + \frac{1}{4}\sen(2u)\right) +C = a^2 \left(\frac{\arcsen\left(\frac{t}{a}\right)}{2} + \frac{t}{4a} \cos\left(\arcsen\frac{t}{a}\right)\right) +C
        \end{equation*}
    
        Por tanto, las imágenes buscadas son:
        \begin{equation*}
            G(a) = 2\int_0^a \sqrt{a^2-t^2}dt = 2a^2\left(\frac{\pi}{4}+0-0\right) = a^2\cdot \frac{\pi}{2}
        \end{equation*}
        \begin{equation*}
            G(-a) = 2\int_0^{-a} \sqrt{a^2-t^2}dt = 2a^2\left(-\frac{\pi}{4}+0-0\right) = -a^2\cdot \frac{\pi}{2}
        \end{equation*}
    
        Por tanto, \begin{equation*}
            Im(G)=[G(-a), G(a)] = \left[-a^2\cdot \frac{\pi}{2}, a^2\cdot \frac{\pi}{2}\right]
        \end{equation*}

        \item [Opción 2: Geométricamente.] Sabemos que la ecuación de la circunferencia es $x^2+y^2=r^2$. Por tanto, la función que determina el semicírculo superior de una circunferencia de radio $r$ es:
        \begin{equation*}
            f(x)=\sqrt{r^2-x^2}
        \end{equation*}

        Por tanto, la función dada $G(x)=\int_{-x}^x\sqrt{a^2-t^2}\;dt$ determina el área encerrada por una circunferencia de radio $a$ con el eje $X$ entre $x$ y $-x$. Por tanto,
        \begin{equation*}
            G(a) = \int_{-a}^a\sqrt{a^2-t^2}\;dt = \frac{\pi a^2}{2}
        \end{equation*}
        \begin{equation*}
            G(-a) = \int_{a}^{-a}\sqrt{a^2-t^2}\;dt= -\int_{-a}^a\sqrt{a^2-t^2}\;dt = -\frac{\pi a^2}{2}
        \end{equation*}

        Como vemos, esta opción facilita mucho los cálculos, pero es necesario caer en el área del círculo.
    \end{description}

\end{ejercicio}

\begin{ejercicio}
    Sea $f:[a,b]\to \bb{R}$ una función continua tal que $f(a)=0$ y $\int_a^bf(t)dt=~0$. Definimos la función $F:[a,b]\to\bb{R}$ como $F(a)=0$ y 
    \begin{equation*}
        F(x) = \frac{\int_a^x f(t)dt}{x-a} \qquad \forall x\neq a
    \end{equation*}

    Probar que $F$ es continua en $[a,b]$ y derivable en $]a,b]$. Demostrar que si $f$ es derivable en $a$ entonces $F$ es derivable en $[a,b]$ y existe $c\in]a,b[$ tal que $F'(c)=0$.\\

    Estudio en primer lugar la continuidad de $F$. Como $f$ es continua en $[a,b]$, tenemos por el TFC que $\int_a^x f(t)dt$ es continua en $[a,b]$. Por tanto, tenemos la continuidad de $F(x)$ en $x\in [a,b]-\{a\}$. Comprobemos para $x=a$:
    \begin{equation*}
        \lim_{x\to a}F(x) = \lim_{x\to a}\frac{\int_a^x f(t)dt}{x-a} \Hop = \lim_{x\to a}f(x) = f(a) = 0 = F(a)
    \end{equation*}

    Por tanto, tenemos la continuidad de $F$ en $[a,b]$.
    
    \vspace{1cm}Respecto a la derivabilidad, por el TFC tenemos que $\int_a^x f(t)dt$ es derivable en $[a,b]$, con derivada $f(x)$. Por el carácter local de la derivabilidad, tenemos que:
    \begin{equation*}
        F'(x)=\frac{f(x)(x-a)-\int_a^xf(t)dt}{(x-a)^2} \qquad \forall x\neq a
    \end{equation*}
    Por tanto, tenemos que $F$ es derivable en $[a,b]-\{a\}$. Supongamos ahora $f$ derivable en $a$:
    \begin{equation*}\begin{split}
        F'(a)&=\lim_{x\to a}F'(x) = \lim_{x\to a}\frac{f(x)(x-a)-\int_a^xf(t)dt}{(x-a)^2}
        \Hop
        \\&= \lim_{x\to a}\frac{f'(x)(x-a) + \cancel{f(x)}-\cancel{f(x)}}{2(x-a)} =
        \\&= \lim_{x\to a}\frac{f'(x)}{2} = \frac{f'(a)}{2}
    \end{split}\end{equation*}

    Por tanto, suponiendo $f$ derivable en $a$ tenemos $F'$ derivable en $a$, con $F'(a)=~\frac{f'(a)}{2}$.

    \vspace{1cm}Demostramos ahora que existe $c\in]a,b[$ tal que $F'(c)=0$.
    \begin{equation*}
        F(a)=0 \qquad F(b)=\frac{\int_a^b f(t)dt}{b-a} = \frac{0}{b-a} = 0
    \end{equation*}

    Como tenemos $F$ continua en $[a,b]$, $F$ derivable en $]a,b[$ y $F(a)=F(b)$; por el Teorema de Rolle se tiene que:
    \begin{equation*}
        \exists c\in ]a,b[ \mid F'(c)=0
    \end{equation*}
\end{ejercicio}

\begin{ejercicio}
    Calcular $\displaystyle \lim_{x\to 0} \frac{x\int_0^x\sen (t^2) dt}{\sen (x^4)}$.

    Por el TFC, tenemos que:
    \begin{equation*}
        \left[\int_0^x\sen (t^2) dt\right]' = \sen(x^2)
    \end{equation*}

    Por tanto,
    \begin{multline*}
        \lim_{x\to 0} \frac{x\int_0^x\sen (t^2) dt}{\sen (x^4)} \Hop
        \lim_{x\to 0} \frac{\int_0^x\sen (t^2) dt +\sen(x^2)x}{4x^3\cos (x^4)} \Hop \\=
        \lim_{x\to 0} \frac{\sen(x^2) +\sen(x^2) +2x^2\cos(x^2)}{12x^2\cos (x^4) -16x^6\sen(x^4)} =
        \lim_{x\to 0} \frac{\sen(x^2) +x^2\cos(x^2)}{6x^2\cos (x^4) -8x^6\sen(x^4)} = \\=
        \lim_{x\to 0} \frac{\sen(x^2)}{6x^2\cos (x^4) -8x^6\sen(x^4)} + \frac{\cos(x^2)}{6\cos (x^4) -8x^4\sen(x^4)} \stackrel{Ec.\;\ref{Ej27.Ind}}{=} \frac{1}{6} + \frac{1}{6} = \frac{1}{3}
    \end{multline*}

    \begin{multline}\label{Ej27.Ind}
        \lim_{x\to 0} \frac{\sen(x^2)}{6x^2\cos (x^4) -8x^6\sen(x^4)} \Hop \\=
        \lim_{x\to 0} \frac{2x\cos(x^2)}{12x\cos(x^4) -24x^5\sen (x^4) -48x^5\sen(x^4) -32x^9\cos(x^4)} =\\=
        \lim_{x\to 0} \frac{2\cos(x^2)}{12\cos(x^4) -24x^4\sen (x^4) -48x^4\sen(x^4) -32x^8\cos(x^4)} = \frac{1}{6}
    \end{multline}
\end{ejercicio}

\begin{ejercicio}
    Calcular $\displaystyle \lim_{x\to +\infty} \frac{\int_1^{(x+1)e^x}\ln (t) \arctan (t) dt}{x^2e^x}$.

    En primer lugar, hemos de demostrar que $\int_1^{+\infty}\ln (x) \arctan (x) dt$ diverge positivamente. Sea $f(x)=\ln (x) \arctan (x)$.
    \begin{equation*}
        \lim_{x\to +\infty} \ln (x) \arctan (x) = +\infty \Longrightarrow \int_1^{+\infty}\ln (x) \arctan (x) dt = +\infty
    \end{equation*}
    
    
    Por tanto, procedemos a calcular el límite aplicando el TFC:
     \begin{equation*}
         \begin{split}
             \lim_{x\to +\infty}& \frac{\int_1^{(x+1)e^x}\ln (t) \arctan (t) dt}{x^2e^x} =\left[\frac{\infty}{\infty}\right]
             \Hop \lim_{x\to +\infty} \frac{\ln[(x+1)e^x] \arctan [(x+1)e^x] \cdot (e^x(x+2))}{2xe^x +x^2e^x} =
             \\&= \lim_{x\to +\infty} \frac{\ln[(x+1)e^x] \arctan [(x+1)e^x] \cdot (x+2)}{x(2+x)} =
             \\&= \lim_{x\to +\infty} \frac{[ln(x+1) +x] \arctan [(x+1)e^x]}{x}
             = \lim_{x\to +\infty} \left(\frac{ln(x+1)}{x} +1\right) \arctan [(x+1)e^x] =\\&\stackrel{Ec.\;\ref{Ej28.Ind}}{=} \frac{\pi}{2}
         \end{split}
     \end{equation*}
     donde he hecho uso del siguiente resultado:
     \begin{equation}\label{Ej28.Ind}
         \lim_{x\to +\infty} \frac{ln(x+1)}{x}
         \Hop \lim_{x\to +\infty} \frac{1}{x+1} = 0
     \end{equation}
\end{ejercicio}

\begin{ejercicio}
    Sea $f:[a,b]\to \bb{R}$ una función de clase $C^1$ en el intervalo $[a,b]$. Para cada $n\in \bb{N}$, sean $I_n:=\int_a^bf(x)\cos(nx)dx$ y $J_n:=\int_a^bf(x)\sen(nx)dx$. Demostrar que las sucesiones $\{I_n\}, \{J_n\}$ convergen a 0.

    Demuestro en primer lugar para $I_n$:
    \begin{equation*}
        \lim_{n\to \infty} I_n = \lim_{n\to \infty} \int_a^bf(x)\cos(nx)dx
    \end{equation*}

    Aplico el método de integración por partes:
    \begin{equation*}
        \MetInt{u(x)=f(x) \qquad u'(x)=f'(x)}{v'(x)=\cos(nx) \qquad v(x)=\frac{1}{n}\sen x}
    \end{equation*}

    Por tanto,
    \begin{multline*}
         \lim_{n\to \infty} I_n =  \lim_{n\to \infty} \left(\frac{1}{n}f(x)\sen x -\int_a^b \frac{1}{n}\sen x f'(x)\;dx\right)
         =\\= \lim_{n\to \infty} \frac{1}{n} \left(f(x)\sen x -\int_a^b \sen x f'(x)\;dx\right) = 0
    \end{multline*}
    donde esto último se da ya que $x\in [a,b]$ fijo, lo que tiene a $+\infty$ es el valor de $n$.


    \vspace{1cm}Demuestro ahora para $J_n$:
    \begin{equation*}
        \lim_{n\to \infty} J_n = \lim_{n\to \infty} \int_a^bf(x)\sen(nx)dx
    \end{equation*}

    Aplico el método de integración por partes:
    \begin{equation*}
        \MetInt{u(x)=f(x) \qquad u'(x)=f'(x)}{v'(x)=\sen(nx) \qquad v(x)=-\frac{1}{n}\cos x}
    \end{equation*}

    Por tanto,
    \begin{multline*}
         \lim_{n\to \infty} J_n =  \lim_{n\to \infty} \left(-\frac{1}{n}f(x)\cos x +\int_a^b \frac{1}{n}\cos x f'(x)\;dx\right)
         =\\= \lim_{n\to \infty} \frac{1}{n} \left(-f(x)\cos x +\int_a^b \cos x f'(x)\;dx\right) = 0
    \end{multline*}
\end{ejercicio}

\begin{ejercicio}
    Demostrar que para cada $x\in \left[0, \frac{\pi}{2} \right]$ se verifica que:
    \begin{equation*}
        \int_0^{\cos^2 x} \arccos{\sqrt{t}}\; dt + \int_0^{\sen^2 x} \arcsen\sqrt{t}\;dt =\frac{\pi}{4}
    \end{equation*}

    Sea $F:\left[0,\frac{\pi}{2}\right]\to \bb{R}$ definida por:
    \begin{equation*}
        F(x)=\int_0^{\cos^2 x} \arccos{\sqrt{t}}\; dt + \int_0^{\sen^2 x} \arcsen\sqrt{t}\;dt
    \end{equation*}

    Como los integrandos son acotados, ya que las funciones trigonométricas inversas dadas lo son, por el TFC tenemos que $F'$ es derivable, con:
    \begin{multline*}
        F'(x)=\arccos[\sqrt{\cos^2 x}] \cdot (-2\cos x\sen x) + \arcsen[\sqrt{\sen^2 x}] \cdot (2\cos x\sen x) =\\
        = -x\sen(2x) +x\sen(2x) = 0 \qquad \qquad \forall x\in \left[0,\frac{\pi}{2}\right]
    \end{multline*}
    donde hemos hecho uso de que $x\in \left[0,\frac{\pi}{2}\right]$, por lo que el seno y el coseno son positivos.

    Por tanto, tenemos que $F(x)$ es constante. 
    
    
    Para ver que es igual a $\frac{\pi}{2}$, demostramos en primer lugar que: $$h(x)=\arccos x + \arcsen x = \frac{\pi}{2} \qquad \forall x\in \bb{R}$$
    Demostramos en primer lugar que es constante y luego mostramos un valor.
    \begin{equation*}
        h'(x)=\frac{-1}{\sqrt{1-x^2}} + \frac{1}{\sqrt{1-x^2}} = 0  \hspace{2cm} h\left(\frac{\sqrt{2}}{2}\right)=2\cdot \frac{\pi}{4} = \frac{\pi}{2}
    \end{equation*}

    Por tanto, sabiendo esto, calculamos el valor de $F$ para cierto $x$. En este caso, nos es conveniente usar $x=\frac{\pi}{4}$, ya que su seno y coseno coinciden:
    \begin{equation*}
        F\left(\frac{\pi}{4}\right) = \int_0^{1/2} \arccos{\sqrt{t}} + \arcsen\sqrt{t}\;dt = \int_0^{1/2} \frac{\pi}{2}\;dt = \frac{\pi}{2}\left[t\right]_0^{1/2} = \frac{\pi}{4}
    \end{equation*}

    Por tanto, como $F$ es constante y tenemos un valor de $x$ que vertifica la condicion del enunciado, esta queda demostrada.
\end{ejercicio}
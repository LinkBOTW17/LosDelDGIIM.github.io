\section{Cálculo Diferencial}

\begin{ejercicio}\label{Ejercicio:1}
Estudiar la derivabilidad de la función $f:A\to \bb{R}$, en cada uno de los siguientes casos:
\begin{enumerate}
    \item $A=[-1,1]$ y $f(x)=\sqrt{1-x^2}$
    
    $$f'(x)=\frac{-2x}{2\sqrt{1-x^2}} = \frac{-x}{\sqrt{1-x^2}} \qquad \forall x \in ]-1, 1[$$
    \begin{itemize}
        \item Para $x=-1$:
        \begin{equation*}
            f'(-1)=\lim_{x\to -1^+} f'(x) = \lim_{x\to -1^+} \frac{-x}{\sqrt{1-x^2}} = \frac{1}{0^+} = +\infty \Longrightarrow \nexists f'(-1)
        \end{equation*}

        \item Para $x=1$:
        \begin{equation*}
            f'(1)=\lim_{x\to 1^-} f'(x) = \lim_{x\to 1^-} \frac{-x}{\sqrt{1-x^2}} = \frac{-1}{0^+} = -\infty \Longrightarrow \nexists f'(1)
        \end{equation*}

        Por tanto, $f$ es derivable en $]-1,1[$
    \end{itemize}

    \item $A=\bb{R}$ y $f(x)=\sqrt[3]{|x|}$
    $$f(x)=\left\{
    \begin{array}{ccc}
    \displaystyle \sqrt[3]{x} & \text{si} & x\geq 0 \\
    \displaystyle \sqrt[3]{-x} & \text{si} & x<0
    \end{array}\right.$$

    Por el carácter local de la derivabilidad,
    $$f'(x)=\left\{
    \begin{array}{ccc}
    \frac{1}{3}x^{-2/3} & \text{si} & x > 0 \\
    -\frac{1}{3}(-x)^{-2/3} & \text{si} & x < 0 \\
    \end{array}\right.$$

    Veamos para $x=0$:
    \begin{equation*}
        \lim_{x\to 0^+} f'(x) = \lim_{x\to 0^+} \frac{1}{3}x^{-2/3} = \frac{1}{0^+} = +\infty
    \end{equation*}
    \begin{equation*}
        \lim_{x\to 0^-} f'(x) = \lim_{x\to 0^-} -\frac{1}{3}(-x)^{-2/3} = -\frac{1}{0^+} = -\infty
    \end{equation*}
    Por tanto, $\nexists f'(0)$    

    \item $A=\bb{R}$ y $f(x)=\frac{2x}{1+|x|}$
    \label{Ejercicio:1.3}

    $$f(x)=\left\{
    \begin{array}{ccc}
    \displaystyle \frac{2x}{1+x} & \text{si} & x\geq 0 \\
    \displaystyle \frac{2x}{1-x} & \text{si} & x<0
    \end{array}\right.$$
    Veamos para $x=0$:
    \begin{equation*}
        f'_-(0) = \lim_{x\to 0^-} \frac{f(x)-f(0)}{x-0} = \lim_{x\to 0^-} \frac{\frac{2x}{1-x}-0}{x} = \lim_{x\to 0^-} \frac{2}{1-x} = 2
    \end{equation*}
    \begin{equation*}
        f'_+(0) = \lim_{x\to 0^+} \frac{f(x)-f(0)}{x-0} = \lim_{x\to 0^+} \frac{\frac{2x}{1+x}-0}{x} = \lim_{x\to 0^+} \frac{2}{1+x} = 2
    \end{equation*}
    Por tanto, como las derivadas laterales coinciden, $f'(0)=2$.

    Además, Por el carácter local de la derivabilidad, f también es derivable en $\bb{R}-\{0\}$.
    $$f'(x)=\left\{
    \begin{array}{ccc}
    \displaystyle \frac{2}{(1+x)^2} & \text{si} & x> 0 \\
    2 & \text{si} & x= 0 \\
    \displaystyle \frac{2}{(1-x)^2} & \text{si} & x<0
    \end{array}\right.$$

    \item $A=R^+_0$ y $f(x)=\left\{
    \begin{array}{ccc}
    x^x & \text{si} & x\in \bb{R^+} \\
    0 & \text{si} & x=0
    \end{array}\right.$
    
    Estudiemos la continuidad en $x=0$:
    \begin{equation*}
        \lim_{x\to 0^+}f(x)=\lim_{x\to 0^+}x^x = \lim_{x\to 0^+}e^{\ln(x^x)} = \lim_{x\to 0^+} e^{x\ln{x}} \stackrel{(\ast)}{=} e^0 = 1
    \end{equation*}
    \begin{equation*}
        \lim_{x\to 0^+}x \ln{x} = \lim_{x\to 0^+} \frac{\ln{x}}{\frac{1}{x}} = \frac{\infty}{\infty} \stackrel{L'H\hat{o}pital}{=} \lim_{x\to 0^+} \frac{\frac{1}{x}}{-\frac{1}{x^2}} = \lim_{x\to 0^+} - x = 0
    \end{equation*}
    Como $L=1 \neq 0 = f(0) \Longrightarrow f$ no es continua en $x=0 \Longrightarrow f$ no es derivable en $x=0$ .

    Por el carácter local de la derivabilidad,
    $$f'(x) = e^{x\ln{x}} \left(\ln(x)+\frac{x}{x}\right) = x^x \left(\ln(x)+1\right) \qquad \forall x\in \bb{R}^+$$
\end{enumerate}
\end{ejercicio}

\begin{ejercicio}
    Sean $\alpha, \beta \in \bb{R}$ $f:\bb{R} \to \bb{R}$ la función $f(x)=x^2+\alpha x + \beta$. Determinar los valores de $\alpha$ y $\beta$ que hacen que el punto $(2,4)$ pertenezca a la gráfica de $f$ y que la recta tangente a la misma en dicho punto sea la recta de ecuación $2x-y=0$.\\

    Por el carácter local de la derivabilidad, $f'(x) = 2x+\alpha \qquad \forall x \in \bb{R}$

    Como la recta tangente en $x=2$ es $y=2x$ y dicha recta tiene pendiente $m=2$; por la interpretación geométrica de la derivada:
    \begin{equation*}
        f'(2) = 2 \Longrightarrow 4+\alpha = 2 \Longrightarrow \alpha = -2
    \end{equation*}
    
    Como el punto $(2,4)$ pertenece a la gráfica,
    \begin{equation*}
        f(2)=4 \Longrightarrow 2^2 + 2\alpha + \beta = 4 \Longrightarrow \beta = -2\alpha = 4
    \end{equation*}

    Por tanto, $f(x) = x^2 -2x +4$
    
\end{ejercicio}

\begin{ejercicio}
    Sea $f$ una función tal que $f(x+h)=f(x)+3xh+h^2-2h$,\quad para cada $h,r \in \bb{R}$, calcular $f'(0)$ y $f'(2)$.
    \begin{multline*}
        f'(x) = \lim_{h\to 0} \frac{f(x+h)-f(x)}{h} = \lim_{h\to 0} \frac{f(x)+3xh+h^2-2h-f(x)}{h} \\ 
        = \lim_{h\to 0} \frac{3xh+h^2-2h}{h} = \lim_{h\to 0} 3x+h-2 = 3x-2
    \end{multline*}
    Por tanto, $f'(0) = -2$ y $f'(2) = 4$.
\end{ejercicio}

\begin{ejercicio}
    Estudiar la derivabilidad de la función $f:\bb{R} \to \bb{R}$ definida por
    $$f(x)=\left\{
    \begin{array}{ccc}
    e^{-\frac{1}{x^2}} & \text{si} & x\neq0 \\
    0 & \text{si} & x=0
    \end{array}\right.$$
    y determinar su imagen.\\
    
    Por el carácter local de la derivabilidad,
    \begin{equation*}
        f'(x) = \frac{2}{x^3}e^{-\frac{1}{x^2}}  \qquad \forall x \in \bb{R}^\ast
    \end{equation*}
    
    Estudiemos la derivabilidad en $x=0$:
    \begin{equation*}
        f'(0) = \lim_{x\to 0} f'(x) = \lim_{x\to 0} \frac{2}{x^3}e^{-\frac{1}{x^2}}
        = \lim_{x\to 0} 2x\frac{1}{x^4}e^{-\frac{1}{x^2}}
        = \lim_{x\to 0} 2x \cdot \frac{\frac{1}{x^4}}{e^{\frac{1}{x^2}}}
        \stackrel{Ec.\;\ref{Ej4.Ind}}{=} 0
    \end{equation*}
    
    donde he hecho uso del siguiente resultado:
    \begin{multline}\label{Ej4.Ind}
        \lim_{x\to 0} \frac{\frac{1}{x^4}}{e^{\frac{1}{x^2}}} \stackrel{L'H\hat{o}pital}{=}
        \lim_{x\to 0} \frac{-\frac{4x^3}{x^8}}{e^{\frac{1}{x^2}} \cdot - \frac{2}{x^3}} = \lim_{x\to 0} \frac{\frac{2}{x^2}}{e^{\frac{1}{x^2}}} \stackrel{L'H\hat{o}pital}{=}\\=
        \lim_{x\to 0} \frac{-\frac{4}{x^3}}{e^{\frac{1}{x^2}} \cdot -\frac{2}{x^3}} = \lim_{x\to 0} \frac{2}{e^{\frac{1}{x^2}}} = \frac{2}{\infty} = 0
    \end{multline}

    Por tanto, $f$ es derivable en $\bb{R}$ y
    $$f'(x)=\left\{
    \begin{array}{ccc}
    \frac{2}{x^3}e^{-\frac{1}{x^2}} & \text{si} & x\neq0 \\
    0 & \text{si} & x=0
    \end{array}\right.$$

    Estudiemos ahora su imagen:
    \begin{itemize}
        \item Si $x<0$:\\
            $f'(x)<0 \Longrightarrow f(x)$ estrictamente decreciente en $]-\infty, 0[$

        \item Si $x>0$:\\
            $f'(x)>0 \Longrightarrow f(x)$ estrictamente creciente en $]0, +\infty[$
    \end{itemize}
    Además, veamos su comportamiento en $\pm \infty$:
    \begin{equation*}
        \lim_{x\to\infty}f(x)=\lim_{x\to\infty}e^{-\frac{1}{x^2}} = e^0 = 1
    \end{equation*}
    \begin{equation*}
        \lim_{x\to -\infty}f(x)=\lim_{x\to -\infty}e^{-\frac{1}{x^2}} = e^0 = 1
    \end{equation*}

    Como $f(0)=0$ y $f$ es continua en $\bb{R}$, su imagen es:
    \begin{equation*}
        Im(f)=[0, 1[
    \end{equation*}
\end{ejercicio}

\begin{ejercicio}
    Estudiar la derivabilidad y el comportamiento en $\pm \infty$ de la función
    $$f(x)=\left\{
    \begin{array}{ccc}
    \frac{e^x}{x} & \text{si} & x<0 \\
    x & \text{si} & 0\leq x < 1\\
    \sqrt[5]{x} & \text{si} & 1 \leq x
    \end{array}\right.$$

    Por el carácter local de la derivabilidad,
    $$f'(x)=\left\{
    \begin{array}{ccc}
    \frac{e^x x - e^x}{x^2} & \text{si} & x<0 \\
    1 & \text{si} & 0 < x < 1\\
    \frac{1}{5}x^{-\frac{4}{5}} & \text{si} & 1 < x
    \end{array}\right.$$

    Veamos para los puntos frontera:
    \begin{itemize}
        \item Para $x=0$:
        \begin{equation*}
            \lim_{x\to 0^-}f'(x) = \lim_{x\to 0^-} \frac{e^x x - e^x}{x^2} = \frac{0-1}{0^-} = -\infty
        \end{equation*}
        \begin{equation*}
            \lim_{x\to 0^+}f'(x) = \lim_{x\to 0^+} 1 = 1
        \end{equation*}
        Por tanto, $\nexists f'(0)$.

        \item Para $x=1$:
        \begin{equation*}
            \lim_{x\to 1^+}f'(x) = \lim_{x\to 1^+} \frac{1}{5}x^{-\frac{4}{5}} = \frac{1}{5}
        \end{equation*}
        \begin{equation*}
            \lim_{x\to 1^-}f'(x) = \lim_{x\to 1^-} 1 = 1
        \end{equation*}
        Como los límites laterales no coinciden, $\nexists f'(1)$.
    \end{itemize}
\end{ejercicio}

\begin{ejercicio}
    Calcular la derivada de las siguientes funciones en los puntos indicados:
    \begin{enumerate}
        \item $(f^{-1})'(9)$, siendo $f(x)=x^3+1$.
        $$f'(x) = 3x^2 \qquad \forall x\in \bb{R}$$
        
        Como $f$ es continua, inyectiva y derivable en $\bb{R}$,
        \begin{equation*}
            (f^{-1})'(f(a)) = \frac{1}{f'(a)}
        \end{equation*}

        Veamos el valor de $a$:
        $$f(a)=9 \Longleftrightarrow a^3+1=9 \Longleftrightarrow a^3=8 \Longleftrightarrow a=2$$

        Por tanto,
        \begin{equation*}
            (f^{-1})'(9) = (f^{-1})'(f(2)) = \frac{1}{f'(2)} = \frac{1}{3\cdot 2^2} = \frac{1}{12}
        \end{equation*}
        

        \item $(f^{-1})'(16)$, siendo $f(x)=x^3+2x^2+3x+10$.
        $$f'(x) = 3x^2+4x+3 \qquad \forall x\in \bb{R}$$

        Además, $\Delta=b^2-4ac = 16-4\cdot 9 < 0 \Longrightarrow f'(x)\neq 0 \qquad \forall x \in \bb{R}$. Como $f$ es también continua y derivable en $\bb{R}$, entonces
        \begin{equation*}
            (f^{-1})'(f(a)) = \frac{1}{f'(a)}
        \end{equation*}

        Veamos el valor de $a$:
        \begin{equation*}
            f(a)=16 \Longrightarrow a^3+2a^2+3a+10 = 16 \Longrightarrow a^3+2a^2+3a-6 = 0 \\
        \end{equation*}

        \begin{figure}[H]
            \centering
            \polyhornerscheme[x=1]{x^3+2x^2+3x-6}
            \caption{División mediante Ruffini donde se ve que $x=1$ es una solución.}
            \label{Fig:DivRuffini6.2}
        \end{figure}        

        Por tanto, al dividir con Ruffini (figura \ref{Fig:DivRuffini6.2}), $f(a)=16 \Longrightarrow a=1$.
        \begin{equation*}
            (f^{-1})'(16) = (f^{-1})'(f(1)) = \frac{1}{f'(1)} = \frac{1}{3+4+3} = \frac{1}{10}
        \end{equation*}

        \item $(f^{-1})'(2)$, siendo $f(x)=\sqrt[3]{x^3+5x+2}$.
        $$f'(x) = \frac{3x^2+5}{3 \sqrt[3]{(x^3+5x+2)^2}} \qquad \forall x \in \bb{R}$$

        Veamos ahora si $f(x)$ es inyectiva.\\Sean $x_1, x_2 \in \bb{R}$ y supongamos $f(x_1)=f(x_2)$.
        \begin{multline*}
            f(x_1)=f(x_2) \Longrightarrow \sqrt[3]{x_1^3+5x_1+2} = \sqrt[3]{x_2^3+5x_2+2} \\
            \Longrightarrow x_1^3+5x_1+2 = x_2^3+5x_2 +2 \Longrightarrow x_1^3+5x_1 = x_2^3+5x_2 \Longrightarrow x_1^3-x_2^3 + 5(x_1-x_2)=0 \\
            \Longrightarrow (x_1-x_2)((x_1+x_2)^2 + 5)=0 \Longrightarrow x_1-x_2=0 \Longrightarrow x_1=x_2
        \end{multline*}

        Por tanto, $f(x_1)=f(x_2)\Longrightarrow x_1=x_2$, demostrando que $f$ es inyectiva. $\hfill \qed$

        Por tanto, como $f$ es continua, inyectiva y derivable,
        \begin{equation*}
            (f^{-1})'(f(a)) = \frac{1}{f'(a)}
        \end{equation*}

        Veamos el valor de $a$:
        \begin{multline*}
            f(a)=2 \Longrightarrow \sqrt[3]{a^3+5a+2} = 2 \Longrightarrow a^3+5a+2 = 8 \Longrightarrow a^3+5a-6=0
        \end{multline*}
        
        \begin{figure}[H]
            \centering
            \polyhornerscheme[x=1]{x^3+5x-6}
            \caption{División mediante Ruffini donde se ve que $x=1$ es una solución.}
            \label{Fig:DivRuffini6.3}
        \end{figure}
        Por tanto, $f(a)=2 \Longrightarrow a=1$
        \begin{equation*}
            (f^{-1})'(2) = (f^{-1})'(f(1)) = \frac{1}{f'(1)} = \frac{3\sqrt[3]{(1+5+2)^2}}{3+5} = \frac{12}{8}=\frac{3}{2}
        \end{equation*}

        \item $(g^{-1})'(9)$, siendo $f(x-2)=x^3+1$ y $g(x)=f(\arctg(x))$.

        En primer lugar, vemos que $f(x)=(x+2)^3 +1$.
        
        $g$ es continua en $\bb{R}$, ya que es la composición de dos funciones continuas. Por el carácter local de la derivabilidad:
        $$g'(x) = 3(\arctan(x)+2)^2 \cdot \frac{1}{1+x^2} \qquad \forall x \in \bb{R}$$

        Como $g$ es continua, inyectiva y derivable,
        \begin{equation*}
            (g^{-1})'(g(a)) = \frac{1}{g'(a)}
        \end{equation*}

        Veamos el valor de $a$:
        \begin{multline*}
            g(a)=9 \Longrightarrow f(\arctan(a))=9 \Longrightarrow (\arctan(a)+2)^3+1=9 \\
            \Longrightarrow \arctan(a) +2 = 2 \Longrightarrow \arctan{a} = 0 \Longrightarrow a=0
        \end{multline*}

        Por tanto, $g(a)=9 \Longrightarrow a=0$.
        \begin{equation*}
            (g^{-1})'(9) = (g^{-1})'(g(0)) = \frac{1}{g'(0)} = \frac{1+0^2}{3(\arctan(0)+2)^2} = \frac{1}{12}
        \end{equation*}

        \item $(g\circ f^{-1})'(6)$, siendo $f(x)=x^3+2x^2+3x$ y $g(x)=\frac{x^3+6x^2+9x+5}{x^4+1}$.

        Por la regla de la cadena,$$(g\circ f^{-1})'(6) = g'(f^{-1}(6)) \cdot (f^{-1})'(6)$$

        Calculo en primer lugar $f^{-1}(6)$.
        \begin{equation*}
            f^{-1}(6) = a \Longleftrightarrow f(a)=6 \Longleftrightarrow a^3+2a^2+3a = 6 \Longleftrightarrow a^3+2a^2+3a - 6 = 0 \Longleftrightarrow a=1
        \end{equation*}
        \begin{figure}[H]
            \centering
            \polyhornerscheme[x=1]{x^3+2x^2+3x-6}
            \caption{División mediante Ruffini donde se ve que $a=1$ es una solución.}
            \label{Fig:DivRuffini6.5}
        \end{figure}

        Por tanto, $f^{-1}(6)=1$. Calculemos ahora $(f^{-1})'(6)$. Por el carácter local de la derivabilidad, $f'(x)=3x^2+4x+3 \quad \forall x \in \bb{R}$.
        
        Por tanto, como $f$ es continua, inyectiva y derivable,
        \begin{equation*}
            (f^{-1})'(f(a)) = \frac{1}{f'(a)}
        \end{equation*}

        Como hemos visto anteriormente, $f(a)=6 \Longrightarrow a=1$. Por tanto,
        \begin{equation*}
            (f^{-1})'(6) = \frac{1}{f'(1)} = \frac{1}{3+4+3} = \frac{1}{10}
        \end{equation*}

        Respecto a $g$, por el carácter local de la derivabilidad,
        $$g'(x)=\frac{(3x^2+12x+9)(x^4+1)-4x^3(x^3+6x^2+9x+5)}{(x^4+1)^2} \quad \forall x \in \bb{R}$$

        Por tanto,
        \begin{multline*}
            (g\circ f^{-1})'(6) = g'(f^{-1}(6)) \cdot (f^{-1})'(6) =  g'(1) \frac{1}{10}\\
            = \frac{(3+12+9)(1+1)-4(1+6+9+5)}{10(1+1)^2} =\frac{48-84}{40} = \frac{-9}{10}
        \end{multline*}
        
    \end{enumerate}
\end{ejercicio}

\begin{ejercicio}
    Calcular la imagen de las siguientes funciones:
    \begin{enumerate}
        \item $f:[0,1]\to \bb{R}, f(x)=x+\arctg(x)$, para todo $x\in[0,1]$,

        Por el carácter local de la derivabilidad:
        $$f'(x)=1+\frac{1}{1+x^2} \qquad \forall x \in ]0,1[$$

        Como $f'(x)>0 \quad \forall x \in ]0,1[$, $f(x)$ es estrictamente creciente en $]0,1[$. Como $f(0)=0$ y $f(1)=1+\frac{\pi}{4}$,
        $$Im(f)=\left[0,1+\frac{\pi}{4}\right]$$

        \item $f:\bb{R}\to \bb{R}, f(x)=x+\arctg(x)$, para todo $x\in \bb{R}$,

        Por el carácter local de la derivabilidad:
        $$f'(x)=1+\frac{1}{1+x^2} \qquad \forall x \in \bb{R}$$

        Como $f'(x)>0 \quad \forall x \in \bb{R}$, $f(x)$ es estrictamente creciente en $\bb{R}$. Veamos el comportamiento de la función en $\pm \infty$:
        \begin{equation*}
            \lim_{x\to \infty} f(x) = \lim_{x\to \infty} x+\arctg(x) = \infty
        \end{equation*}
        \begin{equation*}
            \lim_{x\to -\infty} f(x) = \lim_{x\to -\infty} x+\arctg(x) = -\infty
        \end{equation*}
        
        Por tanto, el conjunto imagen no está mayorado ni minorado, es decir:
        $$Im(f)=\bb{R}$$

        \item $f:]0,1[\to \bb{R}, f(x)=\frac{2x-1}{x(x+1)}$, para todo $x\in]0,1[$,

        Por el carácter local de la derivabilidad:
        $$f'(x)=\frac{2x(x+1)-(2x-1)(2x+1)}{(x^2+x)^2} = \frac{2x^2+2x-4x^2+1}{(x^2+x)^2} = \frac{-2x^2+2x+1}{(x^2+x)^2} \qquad \forall x \in ]0,1[$$

        Veamos en qué puntos se anula la primera derivada:
        \begin{equation*}
            f'(x)=0 \Longrightarrow -2x^2+2x+1=0 \Longrightarrow \nexists \;sol \in ]0,1[
        \end{equation*}
        
        Como no hay puntos críticos y $f'(0)>0 \Longrightarrow f(x)$ estrictamente creciente en $]0,1[$. Veamos su comportamiento en $x=0,1$.
        \begin{equation*}
            \lim_{x \to 0^+} f(x) = \lim_{x \to 0^+} \frac{2x-1}{x(x+1)} = \frac{-1}{0^+} = -\infty
        \end{equation*}
        \begin{equation*}
            \lim_{x \to 1^-} f(x) = \lim_{x \to 1^-} \frac{2x-1}{x(x+1)} = \frac{1}{2}
        \end{equation*}

        Por tanto, $Im(f)=]-\infty, \frac{1}{2}[$.

        \item $f:[-1,1]\to \bb{R}, f(x)=\frac{x^2}{1+x^2}$, para todo $x\in[-1,1]$,

        Por el carácter local de la derivabilidad:
        $$f'(x)=\frac{2x(1+x^2)-2x^3}{(1+x^2)^2} = \frac{2x}{(1+x^2)^2}\qquad \forall x \in ]-1,1[$$

        Veamos en qué puntos se anula la primera derivada:
        \begin{equation*}
            f'(x)=0 \Longrightarrow 2x=0 \Longrightarrow x=0
        \end{equation*}

        \begin{itemize}
            \item \underline{Para $x<0$}\\$f'(x)<0 \Longrightarrow f$ estrictamente decreciente en $]-1,0[$.
            \item \underline{Para $x>0$}\\$f'(x)>0 \Longrightarrow f$ estrictamente creciente en $]0,1[$.
        \end{itemize}

        Como $f(-1)=f(1)=\frac{1}{2}$ y $f(0)=0$,
        $$Im(f)=\left[0,\frac{1}{2}\right]$$

        \item $f:[-1,1]\to \bb{R}, f(x)=\frac{2x}{1+|x|}$, para todo $x\in[-1,1]$,
        $$f(x)=\left\{
        \begin{array}{ccc}
        \frac{2x}{1+x} & \text{si} & x\geq 0 \\
        \frac{2x}{1-x} & \text{si} & x<0
        \end{array}\right.$$

        Como se ha visto en el ejercicio \ref{Ejercicio:1}.\ref{Ejercicio:1.3},        
        $$f'(x)=\left\{
        \begin{array}{ccc}
        \displaystyle \frac{2}{(1+x)^2} & \text{si} & x> 0 \\
        2 & \text{si} & x= 0 \\
        \displaystyle \frac{2}{(1-x)^2} & \text{si} & x<0
        \end{array}\right.$$

        Por tanto, como $f'(x)>0 \quad \forall x\in ]-1,1[, f$ es estrictamente creciente en $]-1,1[$.

        Como $f(-1)=-1$ y $f(1)=1$, $Im(f)=[-1,1]$.

        \item $f:]-1,1[\to \bb{R}, f(x)=x(1-x^2)^{-1/2}$, para todo $x\in]-1,1[$.
        $$f(x) = \frac{x}{\sqrt{1-x^2}}$$

        Por el carácter local de la derivabilidad,
        $$f'(x)=\frac{\sqrt{1-x^2}-x\frac{-2x}{2\sqrt{1-x^2}}}{1-x^2} = \frac{\sqrt{1-x^2}+\frac{x^2}{\sqrt{1-x^2}}}{1-x^2} = \frac{1}{(1-x^2)\sqrt{1-x^2}} \qquad \forall x \in ]-1,1[$$

        Como $f'(x)>0\quad \forall x \in ]-1,1[, f$ es estrictamente creciente en $]-1,1[$.Veamos su comportamiento en $x=\pm1$.
        \begin{equation*}
            \lim_{x \to -1^+} f(x) = \lim_{x \to -1^+} \frac{x}{\sqrt{1-x^2}} = \frac{-1}{0^+} = -\infty
        \end{equation*}
        \begin{equation*}
            \lim_{x \to 1^-} f(x) = \lim_{x \to 1^-} \frac{x}{\sqrt{1-x^2}} = \frac{1}{0^+} = +\infty
        \end{equation*}

        Por tanto, $Im(f)=\bb{R}$.
    \end{enumerate}
\end{ejercicio}

\begin{ejercicio}
    Demostrar las siguientes desigualdades para los valores de $x$ indicados en cada caso:
    \begin{enumerate}
        \item $\frac{x}{1+x} < \ln(1+x)<x$ \quad para todo $x>0$,

        Comprobemos en primer lugar la primera desigualdad, es decir, que $\frac{x}{1+x} < \ln(1+x) \quad \forall x>0$.
        
        Sea $f:\bb{R}^+\to \bb{R}$ dada por $f(x)=\frac{x}{1+x} - \ln(1+x)$. Calculemos su imagen.

        Por el carácter local de la derivabilidad, $$f'(x)=\frac{1+x-x}{(1+x)^2} -\frac{1}{1+x}= \frac{1}{(1+x)^2} - \frac{1}{1+x} = -\frac{x}{(1+x)^2}\quad \forall x>0$$

        Veamos en qué puntos se anula la primera derivada:
        $$f'(x)=0 \Longrightarrow x=0 \notin \bb{R}^+ \Longrightarrow \nexists \;sol$$

        Como $f'(x)<0 \; \forall x\in \bb{R}^+$, $f$ es estrictamente decreciente en $\bb{R}^+$. Veamos ahora su comportamiento en $x=0$ y en $+\infty$.
        \begin{equation*}
            \lim_{x\to0^+}f(x)=\lim_{x\to0^+} \frac{x}{1+x} - \ln(1+x) = 0
        \end{equation*}
        \begin{equation*}
            \lim_{x\to\infty}f(x)=\lim_{x\to\infty} \frac{x}{1+x} - \ln(1+x) = -\infty
        \end{equation*}

        Por tanto, $Im(f)=\bb{R}^-$, es decir,
        $$\frac{x}{1+x} - \ln(1+x) <0 \;\forall x\in \bb{R}^+ \Longrightarrow \frac{x}{1+x} < \ln(1+x) \;\forall x\in \bb{R}^+ $$
        demostrando así la primera desigualdad.

        Para la segunda desigualdad, sea $g:\bb{R}^+\to \bb{R}$ dada por $g(x)=\ln(1+x) - x$. Veamos cuál es su imagen.

        Por el carácter local de la derivabilidad:
        $$g'(x)=\frac{1}{1+x}-1 \quad \forall x \in \bb{R}^+$$

        Veamos en qué puntos se anula la primera derivada:
        $$g'(x)=0 \Longrightarrow \frac{1}{1+x}=1 \Longrightarrow x=0 \notin \bb{R}^+ \Longrightarrow \nexists \;sol$$

        Como $g'(x)<0 \; \forall x\in \bb{R}^+$, $g$ es estrictamente decreciente en $\bb{R}^+$. Veamos ahora su comportamiento en $x=0$ y en $+\infty$.
        \begin{equation*}
            \lim_{x\to0^+}g(x)=\lim_{x\to0^+} \ln(1+x) - x = 0
        \end{equation*}
        \begin{equation*}
            \lim_{x\to\infty}g(x)=\lim_{x\to\infty} \ln(1+x) - x = -\infty
        \end{equation*}

        Por tanto, $Im(g)=\bb{R}^-$, es decir,
        $$\ln(1+x) - x <0 \;\forall x\in \bb{R}^+ \Longrightarrow \ln(1+x) < x \;\forall x\in \bb{R}^+ $$
        demostrando así la segunda desigualdad.

        Por tanto, se han demostrado las dos desigualdades. $\hfill \qed$

        \item $\cos(x) > 1-\frac{x^2}{2}$ \quad para todo $x \in \left]0, \frac{\pi}{2} \right[$,
        
        Sea $f:\left]0, \frac{\pi}{2} \right[ \to \bb{R}$ dada por $f(x)=\cos(x) - 1+\frac{x^2}{2}$. Veamos cuál es su imagen.

        Por el carácter local de la derivabilidad:
        $$f'(x)=-\sen(x)+x \quad \forall x \in \left]0, \frac{\pi}{2} \right[$$

        Veamos en qué puntos se anula la primera derivada:
        $$f'(x)=0 \Longrightarrow \sen{x}=x \Longrightarrow x=0 \notin \left]0, \frac{\pi}{2} \right[ \Longrightarrow \nexists \;sol$$

        Como $f'(x)>0 \; \forall x\in \bb{R}^+$, $g$ es estrictamente creciente en $\left]0, \frac{\pi}{2} \right[$. Veamos ahora su comportamiento en $x=0$ y en $x=\frac{\pi}{2}$.
        \begin{equation*}
            \lim_{x\to0^+}f(x)=\lim_{x\to0^+} \cos(x) - 1+\frac{x^2}{2} = 0
        \end{equation*}
        \begin{equation*}
            \lim_{x\to\frac{\pi}{2}}f(x)=\lim_{x\to\frac{\pi}{2}} \cos(x) - 1+\frac{x^2}{2} = -1+\frac{\pi^2}{8}
        \end{equation*}

        Por tanto, $Im(f)=]0,-1+\frac{\pi}{8}[ \subset \bb{R}^+$, es decir,
        $$\cos(x) - 1+\frac{x^2}{2} >0 \;\forall x\in \left]0, \frac{\pi}{2} \right[ \Longrightarrow \cos{x} > 1-\frac{x^2}{2} \;\forall x\in \left]0, \frac{\pi}{2} \right[ $$
        demostrando así la desigualdad. $\hfill \qed$

        \item $\frac{2x}{\pi} < \sen{x} < x < \tg (x)$ \quad para todo $x \in \left]0, \frac{\pi}{2} \right[$.

        Comprobemos en primer lugar la primera desigualdad, es decir, que $\frac{2x}{\pi} < \sen(x) \quad \forall x \in \left]0, \frac{\pi}{2} \right[$.
        
        Sea $f:\left]0, \frac{\pi}{2} \right[\to \bb{R}$ dada por $f(x)=\frac{2x}{\pi} - \sen{x}$. Calculemos su imagen.

        Por el carácter local de la derivabilidad, $f'(x)=\frac{2}{\pi} - \cos{x}\quad \forall x \in \left]0, \frac{\pi}{2} \right[$. Veamos en qué puntos se anula la primera derivada:
        $$f'(x)=0 \Longrightarrow \cos{x}=\frac{2}{\pi} \Longrightarrow x=\arccos{\frac{2}{\pi}} \approx 0.88$$
        \begin{itemize}
            \item \underline{Para $x\in \left]0,\arccos{\frac{2}{\pi}}\right[$}:

            $f'(x)<0$, por lo que $f$ es estrictamente decreciente en $\left]0,\arccos{\frac{2}{\pi}}\right[$.

            \item \underline{Para $x\in \left]\arccos{\frac{2}{\pi}}, \frac{\pi}{2}\right[$}:

            $f'(x)>0$, por lo que $f$ es estrictamente creciente en $\left]\arccos{\frac{2}{\pi}}, \frac{\pi}{2}\right[$.
        \end{itemize}
        
        Veamos ahora su comportamiento en $x=0$ y en $x=\frac{\pi}{2}$.
        \begin{equation*}
            \lim_{x\to0^+}f(x)=\lim_{x\to0^+} \frac{2x}{\pi} - \sen{x} = 0
        \end{equation*}
        \begin{equation*}
            \lim_{x\to\frac{\pi}{2}}f(x)=\lim_{x\to\frac{\pi}{2}} \frac{2x}{\pi} - \sen{x} = 0
        \end{equation*}

        Como, además, $f\left(\arccos{\frac{2}{pi}}\right) \approx -0.21 <0$, $Im(f)=\left]f\left(\arccos{\frac{2}{pi}}\right) ,0\right[ \subset \bb{R}^-$, es decir,
        $$\frac{2x}{\pi} - \sen{x} <0 \;\forall x\in \left]0, \frac{\pi}{2} \right[ \Longrightarrow \frac{2x}{\pi} < \sen{x} \;\forall x\in \left]0, \frac{\pi}{2} \right[ $$
        demostrando así la primera desigualdad.

        Para la segunda desigualdad, sea $g:\left]0, \frac{\pi}{2} \right[ \to \bb{R}$ dada por $g(x)=\sen{x} -x$. Veamos cuál es su imagen.

        Por el carácter local de la derivabilidad:
        $$g'(x)=\cos{x}-1 \quad \forall x \in \left]0, \frac{\pi}{2} \right[$$

        Veamos en qué puntos se anula la primera derivada:
        $$g'(x)=0 \Longrightarrow \cos{x}=1 \Longrightarrow x=0 \notin \left]0, \frac{\pi}{2} \right[ \Longrightarrow \nexists \;sol$$

        Como $g'(x)<0 \; \forall x\in \left]0, \frac{\pi}{2} \right[$, $g$ es estrictamente decreciente en $\left]0, \frac{\pi}{2} \right[$. Veamos ahora su comportamiento en $x=0$ y en $x=\frac{\pi}{2}$.
        \begin{equation*}
            \lim_{x\to0^+}g(x)=\lim_{x\to0^+} \sen{x} -x = 0
        \end{equation*}
        \begin{equation*}
            \lim_{x\to\frac{\pi}{2}}g(x)=\lim_{x\to\frac{\pi}{2}} \sen{x} -x = 1-\frac{\pi}{2} <0
        \end{equation*}

        Por tanto, $Im(g)=\left]1-\frac{\pi}{2},0\right[ \subset \bb{R}^-$, es decir,
        $$\sen{x}-x <0 \;\forall x\in \left]0, \frac{\pi}{2} \right[ \Longrightarrow \sen{x} < x \;\forall x\in \left]0, \frac{\pi}{2} \right[ $$
        demostrando así la segunda desigualdad.

        Para la tercera desigualdad, sea $h:\left]0, \frac{\pi}{2} \right[ \to \bb{R}$ dada por $h(x)=x-\tg{x}$. Veamos cuál es su imagen.

        Por el carácter local de la derivabilidad:
        $$h'(x)=1-1- \tg^2{x} = -\tg^2{x} \quad \forall x \in \left]0, \frac{\pi}{2} \right[$$

        Veamos en qué puntos se anula la primera derivada:
        $$h'(x)=0 \Longrightarrow -\tg^2{x}=0 \Longrightarrow \tg{x}=0 \Longrightarrow x=0 \notin \left]0, \frac{\pi}{2} \right[ \Longrightarrow \nexists \;sol$$

        Como $h'(x)>0 \; \forall x\in \left]0, \frac{\pi}{2} \right[$, $g$ es estrictamente decreciente en $\left]0, \frac{\pi}{2} \right[$. Veamos ahora su comportamiento en $x=0$ y en $x=\frac{\pi}{2}$.
        \begin{equation*}
            \lim_{x\to0^+}h(x)=\lim_{x\to0^+} x-\tg{x} = 0
        \end{equation*}
        \begin{equation*}
           \lim_{x\to\frac{\pi}{2}}h(x)=\lim_{x\to\frac{\pi}{2}} x-\tg{x} = -\infty
        \end{equation*}

        Por tanto, $Im(h)=\bb{R}^-$, es decir,
        $$x-\tg{x} < 0 \;\forall x\in \left]0, \frac{\pi}{2} \right[ \Longrightarrow x < \tg{x}\;\forall x\in \left]0, \frac{\pi}{2} \right[ $$
        demostrando así la tercera desigualdad.

        Por tanto, se han demostrado las tres desigualdades. $\hfill \qed$
    \end{enumerate}
\end{ejercicio}

\begin{ejercicio}
    Determinar el número de ceros y la imagen de la función $f:\bb{R} \to \bb{R}$ definida por $f(x)=x^6-3x^2+2$, para cada $x\in \bb{R}$.

    $f$ es derivable, con $f'(x)=6x^5-6x = 6x(x^4-1) \quad \forall x\in \bb{R}$.

    $$f'(x)=0 \Longrightarrow x=\{-1,0,1\}$$
    \begin{itemize}
        \item \underline{Para $x \in \;]-\infty, -1[$:}\\
        $f'(x)<0 \Longrightarrow f$ estrictamente decreciente en $]-\infty, -1[$.

        \item \underline{Para $x \in \;]-1, 0[$:}\\
        $f'(x)>0 \Longrightarrow f$ estrictamente creciente en $]-1, 0[$.

        \item \underline{Para $x \in \;]0, 1[$:}\\
        $f'(x)<0 \Longrightarrow f$ estrictamente decreciente en $]0, 1[$.

        \item \underline{Para $x \in \;]1, +\infty[$:}\\
        $f'(x)>0 \Longrightarrow f$ estrictamente creciente en $]1, +\infty[$.

        Por tanto, por el cambio de crecimiento, $x=\pm 1$ son mínimos relativos, y $x=0$ es un máximo relativo.

        Veamos ahora el valor en los extremos relativos y su comportamiento en $\pm \infty$.
        $$f(-1)=0 \qquad f(0)=2 \qquad f(1)=0$$
        \begin{equation*}
            \lim_{x\to\infty}f(x) =\lim_{x\to-\infty}f(x) = +\infty
        \end{equation*}

        Por tanto, y debido a la continuidad de $f$, esta tiene exactamente dos raíces en $\bb{R}$, que son $x=\{-1,1\}$. Además, $Im(f)=\bb{R}^+_0$.
        
    \end{itemize}
\end{ejercicio}

\begin{ejercicio}
    Calcular el número de soluciones de la ecuación $3\ln(x)-x=0$.\\

    Sea la función $f:\bb{R}^+ \to \bb{R}$ dada por $f(x)=3\ln(x)-x \qquad \forall x\in \bb{R}^+$. Veamos cuántos ceros tiene $f$.
    
    Por el carácter local de la derivabilidad, $f'(x)=\frac{3}{x}-1 \qquad \forall x\in \bb{R}^+$.

    \begin{equation*}
        f'(x)=0 \Longrightarrow \frac{3}{x}=1 \Longrightarrow x=3 
    \end{equation*}

    Como $f'(x)=0$ tiene 1 solución, $f$ tiene, a lo sumo, dos soluciones. Calculemos la imagen de $f(x)$.
    \begin{itemize}
        \item \underline{Para $x<3$}\\
        $f'(x)>0 \Longrightarrow f$ es estrictamente creciente en $]0, 3[$.

        \item \underline{Para $x>3$}\\
        $f'(x)<0 \Longrightarrow f$ es estrictamente decreciente en $]3, +\infty[$.
    \end{itemize}

    Veamos el comportamiento de la función en $x=0$ y en $+\infty$.
    \begin{equation*}
        \lim_{x\to0^+}f(x) = \lim_{x\to0^+} 3\ln(x)-x = -\infty
    \end{equation*}
    \begin{equation*}
        \lim_{x\to\infty}f(x) = \lim_{x\to\infty} 3\ln(x)-x = \lim_{x\to\infty}x(3\frac{\ln(x)}{x}-1)  = \infty(0-1) = -\infty
    \end{equation*}

    Como $f(3)=3\ln(3)-3$, $$Im(f)=]-\infty, 3\ln(3)-3]$$

    Como $f(3)=3\ln(3)-3> 0 \Longleftrightarrow \ln{3}>1 \Longleftrightarrow 3 > e$, y esto es trivialmente cierto, $f(x)$ tiene \textbf{2 soluciones}.
\end{ejercicio}

\begin{ejercicio}
    Dado $a>1$, probar que la ecuación $x+e^{-x}=a$ tiene, al menos, una solución positiva y otra negativa.\\
    Sea $f:\bb{R} \to \bb{R}$ dado por $f(x)=x+e^{-x}-a$. Por el carácter local de la continuidad, $f$ es continua en $\bb{R}$.
    \begin{equation*}
        \lim_{x\to -\infty} f(x) = \lim_{x\to -\infty}x+e^{-x}-a = +\infty
    \end{equation*}
    \begin{equation*}
        \lim_{x\to \infty} f(x) = \lim_{x\to \infty}x+e^{-x}-a = +\infty
    \end{equation*}
    Como $f(0) = 0+e^0 -a = 1-a < 0$ y ambos límites en $\pm\infty$ son positivos, por el Teorema de Bolzano, tiene al menos una solución en $\bb{R}^+$ y otra en $\bb{R}^-$.
\end{ejercicio}

\begin{ejercicio}
    Sea $f:\bb{R}^+ \to \bb{R}$ la función dada por $f(x)=x + \ln(x) + \arctg(x)$. Demostrar que la ecuación $f(x)=0$ tiene una única solución.

    Por el carácter local de la derivabilidad,
    $$f'(x) = 1+\frac{1}{x} + \frac{1}{1+x^2} \qquad \forall x\in \bb{R}^+$$

    Como $f'(x)>0 \quad \forall x\in \bb{R}^+$, $f(x)$ es estrictamente creciente en $\bb{R}^+$.
    \begin{equation*}
        \lim_{x\to0^+}f(x) = \lim_{x\to0^+} x + \ln(x) + \arctg(x) = -\infty
    \end{equation*}
    \begin{equation*}
        \lim_{x\to +\infty}f(x) = \lim_{x\to + \infty} x + \ln(x) + \arctg(x) = +\infty
    \end{equation*}

    Por tanto, al menos hay una solución en $\bb{R}^+$. Además, no puede haber más de una solución ya que $f(x)$ es estrictamente creciente.\footnote{También se puede razonar mediante el Teorema de Rolle.}
\end{ejercicio}

\begin{ejercicio}
    Probar que la ecuación $x + e^x + \arctg(x)=0 $ tiene una única raíz real y determinar un intervalo de longitud uno en el que se encuentre dicha raíz.\\
    
    Sea $f:\bb{R}\to \bb{R}$ dado por $f(x)=x + e^x + \arctg(x)$. Por el carácter local de la continuidad, $f$ es continua en $\bb{R}$.
    
    Por el carácter local de la derivabilidad,
    $$f'(x) = 1+e^x + \frac{1}{1+x^2} \qquad \forall x \in \bb{R}$$

    Como $f'(x)>0 \quad \forall x \in \bb{R}$, $f$ es estrictamente creciente en $\bb{R}$.

    Para calcular el intervalo de longitud 1, uso el T. de Bolzano.
    $$f(0)=1>0 \qquad f(-1)=-1+e^{-1}+\arctg(-1) < 0$$
    Por el T. de Bolzano, $\exists c \in ]-1,0[ \;\; \mid f(c) = 0$. Además, como $f$ es estrictamente creciente, es la única solución que tiene.
\end{ejercicio}

\begin{ejercicio}
    Probar que la ecuación $\tg(x) = x$ tiene infinitas soluciones.\\

    Sea la función $f:\bb{R}-\left\{\frac{\pi}{2}+\pi k \quad \forall k \in \bb{Z}\right\} \to \bb{R}$ dada por $f(x)=\tg(x)-x$.

    Por el carácter local de la derivabilidad,
    $$ f'(x)=1+\tg^2(x) - 1 = \tg^2(x) \qquad \forall x \in \bb{R}-\left\{\frac{\pi}{2}+\pi k \quad \forall k \in \bb{Z}\right\}$$

    Como $f'(x)=0$ tiene $\infty$ soluciones en $\bb{R}$, $f(x)$ también puede tener, como mucho, $\infty$ soluciones.
    
    Además, como la función es continua y en cada intervalo de amplitud $\pi$ tiene como imagen $\bb{R}$, en cada intervalo de amplitud $\pi$ tiene una solución. Es decir, hay una solución en cada
    $$\left] \frac{\pi}{2}+k\pi,\frac{\pi}{2}+(k+1)\pi \right[\qquad k\in \bb{Z}$$

    Como $\bb{Z}$ tiene $\infty$ elementos, hay $\infty$ soluciones.
\end{ejercicio}

\begin{ejercicio}
    Calcular la imagen de la función $f:\bb{R}^+ \to \bb{R}$, dada por $f(x)=x^{1/x}$.
    $$f(x)=e^{\frac{\ln{x}}{x}} \qquad \forall x\in \bb{R}^+$$

    Por el carácter local de la derivabilidad, $$f'(x)=e^{\frac{\ln{x}}{x}} \frac{\frac{1}{x}x-\ln{x}}{x^2} = e^{\frac{\ln{x}}{x}} \frac{1-\ln{x}}{x^2} \qquad \forall x \in \bb{R}^+$$

    Veamos los puntos que anulan la primera derivada:
    $$f'(x)=0 \Longrightarrow 1-\ln{x}=0 \Longrightarrow x=e$$
    \begin{itemize}
        \item \underline{Para $x<e$}\\
        $f'(x)>0 \Longrightarrow f$ es estrictamente creciente en $]0,e[$.

        \item \underline{Para $x>e$}\\
        $f'(x)<0 \Longrightarrow f$ es estrictamente decreciente en $]e, +\infty[$.

    \end{itemize}
    
    Veamos su comportamiento en $x=\{0,+\infty\}$.
    \begin{equation*}
        \lim_{x\to0^+}f(x) = \lim_{x\to0^+} e^{\frac{\ln{x}}{x}} = e^\frac{-\infty}{0^+} = e^{-\infty} = 0
    \end{equation*}
    \begin{equation*}
        \lim_{x\to\infty}f(x) = \lim_{x\to\infty} e^{\frac{\ln{x}}{x}} = e^0 = 1
    \end{equation*}
    Además, como $f(e)=e^\frac{1}{e}>1 \Longleftrightarrow \frac{1}{e} > 0$, y esto último es trivialmente cierto, $Im(f)=\left]0,e^\frac{1}{e}\right]$.
\end{ejercicio}

\begin{ejercicio}
    Sean $a,b,c \in \bb{R}$ tales que $a^2<3b$. Probar que la ecuación dada por $x^3+ax^2+bx+c=0$ tiene una solución real única.\\

    Sea $f:\bb{R} \to \bb{R}$ dada por $f(x)=x^3+ax^2+bx+c$.
    
    Por el carácter local de la derivabilidad, $f'(x)=3x^2+2ax+b \quad \forall x\in \bb{R}$. Veamos el número de soluciones de la ecuación $f'(x)=0$.
    $$\Delta = 4a^2 -12b = 4(a^2-3b) < 0 \Longleftrightarrow a^2-3b < 0 \Longleftrightarrow a^2 < 3b$$
    Por tanto, como $\Delta < 0$, $f'(x)\neq0 \quad \forall x \in \bb{R}$. $f(x)$ es estrictamente monótona.
    \begin{equation*}
        \lim_{x\to +\infty} f(x) = +\infty \qquad \qquad \lim_{x\to -\infty} f(x) = -\infty
    \end{equation*}
    Sabiendo el valor de los límites en $\pm \infty$ y sabiendo que es continua, concluimos que tiene al menos una solución. Además, como es estrictamente monótona, esta solución es única.
\end{ejercicio}

\begin{ejercicio}
    Estudiar la derivabilidad de la función $f:\left[-\frac{1}{2}, +\infty \right[ \to \bb{R}$ definida por
    $$f(x)=\left\{
    \begin{array}{ccl}
    (x+e^x)^{1/x} & \text{si} & x\neq0 \\
    e^2 & \text{si} & x=0
    \end{array}\right.$$

    Para simplificar los cálculos y evitar indeterminaciones, tomamos
    $$f(x)=\left\{
    \begin{array}{ccl}
    e^{\frac{\ln(x+e^x)}{x}} & \text{si} & x\neq0 \\
    e^2 & \text{si} & x=0
    \end{array}\right.$$

    Por el carácter local de la derivabilidad,
    $$f'(x) = e^{\frac{\ln(x+e^x)}{x}} \frac{\frac{1+e^x}{x+e^x}x - \ln(x+e^x)}{x^2} \qquad \forall x \in \left]-\frac{1}{2}, +\infty \right[ - \{0\}$$

    Veamos en el caso de $x=0$:
    \begin{multline*}\
        f'(0) = \lim_{x\to 0} f'(x) = \lim_{x\to 0} e^{\frac{\ln(x+e^x)}{x}} \frac{\frac{1+e^x}{x+e^x}x - \ln(x+e^x)}{x^2} \stackrel{Ec. \ref{IndeterminacionEj.17.1}, \ref{IndeterminacionEj.17.2}}{=} e^2 \cdot \frac{-3}{2} = -\frac{3e^2}{2}
    \end{multline*}
    \begin{multline}\label{IndeterminacionEj.17.1}
        \lim_{x\to 0} \frac{\frac{1+e^x}{x+e^x}x - \ln(x+e^x)}{x^2} \stackrel{L'H\hat{o}pital}{=}
        \lim_{x\to 0} \frac{\frac{e^x(x+e^x)-(1+e^x)^2}{(x+e^x)^2}x + \cancel{\frac{1+e^x}{x+e^x}} - \cancel{\frac{1+e^x}{x+e^x}}}{2x} =
        \lim_{x\to 0} \frac{\frac{e^x(x+e^x)-(1+e^x)^2}{(x+e^x)^2}}{2} =\\
        = \lim_{x\to 0} \frac{e^x(x+e^x)-(1+e^x)^2}{2(x+e^x)^2} = \frac{1-2^2}{2\cdot 1} = \frac{-3}{2}
    \end{multline}
    \begin{equation}\label{IndeterminacionEj.17.2}
        \lim_{x\to 0} \frac{\ln(x+e^x)}{x} \stackrel{L'H\hat{o}pital}{=} \lim_{x\to 0} \frac{1+e^x}{x+e^x} = \frac{2}{1} = 2 
    \end{equation}

    Por tanto, $f$ es derivable en $\left]-\frac{1}{2}, +\infty \right[$ con:
    $$f'(x)=\left\{
    \begin{array}{ccl}
    e^{\frac{\ln(x+e^x)}{x}} \frac{\frac{1+e^x}{x+e^x}x - \ln(x+e^x)}{x^2} & \text{si} & x\neq0 \\
    -\frac{3e^2}{2} & \text{si} & x=0
    \end{array}\right.$$

\end{ejercicio}

\begin{ejercicio}
    Estudiar el comportamiento de la función $f:A\to \bb{R}$ en el punto $\alpha$, en cada uno de los siguientes casos:
    \begin{enumerate}
        \item $A = ]2, +\infty[\;,\quad f(x)=\frac{\sqrt{x}-\sqrt{2}+\sqrt{x-2}}{\sqrt{x^2-4}}\quad (x\in A),\quad \alpha = 2$.
        \begin{multline*}
            \lim_{x\to2^+}f(x) = \lim_{x\to2^+}\frac{\sqrt{x}-\sqrt{2}+\sqrt{x-2}}{\sqrt{x^2-4}} = \lim_{x\to2^+}\frac{\sqrt{x}-\sqrt{2}+\sqrt{x-2}}{\sqrt{x+2}\sqrt{x-2}} = \\ = \lim_{x\to2^+}\frac{\frac{\sqrt{x}-\sqrt{2}}{\sqrt{x-2}}+1}{\sqrt{x+2}} \stackrel{Ec.  \ref{Ej18.a.Indet}}{=} \frac{1}{\sqrt{4}} = \frac{1}{2}
        \end{multline*}
        
        donde he tenido que resolver en primer lugar la siguiente indeterminación: \begin{equation}\label{Ej18.a.Indet}
            \lim_{x\to2^+} \frac{\sqrt{x}-\sqrt{2}}{\sqrt{x-2}} \stackrel{L'H\hat{o}pital}{=} \lim_{x\to2^+} \frac{\frac{1}{2\sqrt{x}}}{\frac{1}{2\sqrt{x-2}}} = \lim_{x\to2^+} \frac{\sqrt{x-2}}{\sqrt{x}} = 0
        \end{equation}

        \item $A = \bb{R}^+\backslash\{1\},\quad f(x)=\frac{1}{\ln{x}}-\frac{1}{x-1}\quad (x\in A),\quad \alpha = 1$.
        \begin{multline*}
            \lim_{x\to1}f(x)=\lim_{x\to1}\frac{1}{\ln{x}}-\frac{1}{x-1} = \lim_{x\to1} \frac{x-1-\ln{x}}{\ln{x}(x-1)}
            \stackrel{L'H\hat{o}pital}{=}
            \lim_{x\to1}\frac{1-\frac{1}{x}}{\frac{x-1}{x} + \ln{x}} \stackrel{L'H\hat{o}pital}{=}\\
            = \lim_{x\to1}\frac{\frac{1}{x^2}}{\frac{x-x+1}{x^2}+\frac{1}{x}}
            = \lim_{x\to1}\frac{\frac{1}{x^2}}{\frac{1+x}{x^2}}
            = \lim_{x\to1}\frac{1}{1+x} = \frac{1}{2}
        \end{multline*}

        \item $A = ]1, +\infty[\;,\quad f(x)=\frac{x^x-x}{1-x-\ln{x}}\quad (x\in A),\quad \alpha = 1$.
        \begin{multline*}
            \lim_{x\to1^+}f(x)
            =\lim_{x\to1^+}\frac{x^x-x}{1-x-\ln{x}}
            \stackrel{L'H\hat{o}pital}{=} \lim_{x\to1^+}\frac{e^{x\ln(x)}(\ln(x) + 1) - 1}{-1-\frac{1}{x}} = \frac{e^0(1) - 1}{-1-1} = 0
        \end{multline*}

        \item $A = \bb{R}^\ast,\quad f(x)=\frac{1}{x^4}-\frac{1}{6x^2}-\frac{\sen{x}}{x^5}\quad (x\in A),\quad \alpha = 0$.
        \begin{multline*}
            \lim_{x\to0^+}f(x)
            =\lim_{x\to0^+}\frac{1}{x^4}-\frac{1}{6x^2}-\frac{\sen{x}}{x^5}
            = \lim_{x\to0^+} \frac{6x-x^3-6\sen{x}}{6x^5}
            \stackrel{L'H\hat{o}pital}{=}\\=
            \lim_{x\to0^+} \frac{6-3x^2-6\cos{x}}{30x^4}
            \stackrel{L'H\hat{o}pital}{=}
            \lim_{x\to0^+} \frac{-6x+6\sen{x}}{120x^3}
            \stackrel{L'H\hat{o}pital}{=}
            \lim_{x\to0^+} \frac{-6+6\cos{x}}{360x^2}
            = \\ \stackrel{L'H\hat{o}pital}{=}
            \lim_{x\to0^+} \frac{-6\sen{x}}{720x}
            \stackrel{L'H\hat{o}pital}{=}
            \lim_{x\to0^+} \frac{-6\cos{x}}{720} = -\frac{6}{720} = -\frac{1}{120}
        \end{multline*}
        

        \item $A = \left] 0, \frac{\pi}{2}\right[\;,\quad f(x)=\left(\frac{1}{\tg(x)}\right)^{\sen{x}}\quad (x\in A),\quad \alpha = \frac{\pi}{2}$.

        \begin{equation*}
            \lim_{x\to \frac{\pi}{2}^-}f(x)
            = \lim_{x\to \frac{\pi}{2}^-}\left(\frac{1}{\tg(x)}\right)^{\sen{x}} = \left( \frac{1}{\infty} \right)^1 = 0
        \end{equation*}       

        \item $A = \left] 0, \frac{\pi}{2}\right[\;,\quad f(x)=\left(1+\sen{x}\right)^{\cotg{x}}\quad (x\in A),\quad \alpha = 0$.
        \begin{multline*}
            \lim_{x\to 0^+}f(x)
            = \lim_{x\to 0^+}\left(1+\sen{x}\right)^{\cotg{x}} = 1^\infty =
            \lim_{x\to 0^+}e^{\ln\left(\left(1+\sen{x}\right)^{\cotg{x}}\right)} =\\
            = \lim_{x\to 0^+}e^{\frac{\ln\left(1+\sen{x}\right)}{\tg(x)}} \stackrel{Ec. \ref{Ej18.6.Ind}}{=} e^1 = e
        \end{multline*}
        donde he tenido que resolver en primer lugar la siguiente indeterminación:
        \begin{equation}\label{Ej18.6.Ind}
            \lim_{x\to 0^+}\frac{\ln\left(1+\sen{x}\right)}{\tg(x)} = \frac{0}{0} \stackrel{L'H\hat{opital}}{=}
            \lim_{x\to 0^+}\frac{\frac{\cos{x}}{1+\sen{x}}}{1+\tg^2(x)} = 1
        \end{equation}

        \item $A = \bb{R}^+\backslash\{e\},\quad f(x)=x^{\frac{1}{\ln(x)-1}}\quad (x\in A),\quad \alpha = e$.
        \begin{equation*}
            \lim_{x\to e^+}f(x)
            = \lim_{x\to e^+}x^{\frac{1}{\ln(x)-1}}
            = e^{\frac{1}{0^+}} = e^{+\infty} = \infty
        \end{equation*}
        \begin{equation*}
            \lim_{x\to e^-}f(x)
            = \lim_{x\to e^-}x^{\frac{1}{\ln(x)-1}}
            = e^{\frac{1}{0^-}} = e^{-\infty} = 0
        \end{equation*}
        

        \item $A = \bb{R}^+,\quad f(x)=\frac{e-(1+x)^{\frac{1}{x}}}{x}\quad (x\in A),\quad \alpha = 0$.
        \begin{multline*}
            \lim_{x\to 0^+}f(x)
            = \lim_{x\to 0^+}\frac{e-(1+x)^{\frac{1}{x}}}{x}
            = \lim_{x\to 0^+}\frac{e-e^{\frac{\ln(1+x)}{x}}}{x}
            \stackrel{Ec. \ref{Ej18.8.Ind_1}}{=} \frac{0}{0}
            \stackrel{L'H\hat{o}pital}{=}\\=
            \lim_{x\to 0^+} -e^{\frac{\ln(1+x)}{x}} \frac{\frac{x}{1+x} - \ln(1+x)}{x^2}
            \stackrel{Ec. \ref{Ej18.8.Ind_1} y \ref{Ej18.8.Ind_2}}{=} -e^1\cdot -\frac{1}{2} = \frac{e}{2}
        \end{multline*}
        donde he tenido que resolver en primer lugar las siguientes indeterminaciones:
        \begin{equation}\label{Ej18.8.Ind_1}
            \lim_{x\to 0^+}{\frac{\ln(1+x)}{x}} \stackrel{L'H\hat{o}pital}{=}
            \lim_{x\to 0^+}{\frac{\frac{1}{1+x}}{1}} = 1
        \end{equation}
        \begin{multline}\label{Ej18.8.Ind_2}
            \lim_{x\to 0^+}\frac{\frac{x}{1+x} - \ln(1+x)}{x^2}
            \stackrel{L'H\hat{o}pital}{=}
            \lim_{x\to 0^+} \frac{\frac{1+x-x}{(1+x)^2} - \frac{1}{1+x}}{2x}
            =\\= \lim_{x\to 0^+} \frac{\frac{-x}{(1+x)^2}}{2x}
            = \lim_{x\to 0^+} \frac{-1}{2(1+x)^2}=-\frac{1}{2}
        \end{multline}
        
    \end{enumerate}
\end{ejercicio}

\begin{ejercicio}
    Estudiar el comportamiento en cero de la función $f:A\to \bb{R}$, en cada uno de los siguientes casos:
    \begin{enumerate}        
        \item $A = \left] 0, \frac{\pi}{2}\right[\;,\quad f(x)=\left(\sen(x)+\cos(x)\right)^{\frac{1}{x}}\quad (x\in A)$,

        Como la función solo está definida a la derecha del 0, estudiar su comportamiento en el 0 es tomar límite cuando $x$ tiene a 0 por la derecha.
        \begin{equation*}
            \lim_{x\to 0^+}f(x)
            = \lim_{x\to 0^+} \left(\sen(x)+\cos(x)\right)^{\frac{1}{x}}
            = \left[1^\infty \right]
            = \lim_{x\to 0^+} e^{\frac{\ln(\sen(x) + \cos(x))}{x}}
            \stackrel{Ec. \ref{Ej19.1.Ind}}{=} e^1 = e
        \end{equation*}
        donde he tenido que resolver en primer lugar la siguiente indeterminación:
        \begin{equation}\label{Ej19.1.Ind}
            \lim_{x\to 0^+}{\frac{\ln(\sen(x) + \cos(x))}{x}}
            \stackrel{L'H\hat{o}pital}{=}
            \lim_{x\to 0^+} \frac{\cos(x)-\sen(x)}{\sen(x)+\cos(x)} = 1
        \end{equation}

        Alternativamente, usando la regla del Zapato o Teorema de Euler,
        \begin{equation*}
            \lim_{x\to 0^+}f(x)
            = \lim_{x\to 0^+} \left(\sen(x)+\cos(x)\right)^{\frac{1}{x}}
            = \left[1^\infty \right]
            = \lim_{x\to 0^+} e^{\frac{1}{x}(\sen x + \cos x -1)}
            \stackrel{Ec. \ref{Ej19.1.Ind2}}{=} e^1 = e
        \end{equation*}
        donde he tenido que resolver en primer lugar la siguiente indeterminación:
        \begin{equation}\label{Ej19.1.Ind2}
            \lim_{x\to 0^+} \frac{1}{x}(\sen x + \cos x -1)
            \stackrel{L'H\hat{o}pital}{=}
            \lim_{x\to 0^+} \cos x -\sen x = 1
        \end{equation}
        
        \item $A = \left] 0, \frac{\pi}{2}\right[\;,\quad f(x)=\left(\cos(x) + \frac{x^2}{2}\right)^{\frac{1}{x^2}}\quad (x\in A)$,
        \begin{equation*}
            \lim_{x\to 0^+}f(x)
            = \lim_{x\to 0^+} \left(\cos(x) + \frac{x^2}{2}\right)^{\frac{1}{x^2}}
            = \lim_{x\to 0^+} e^{\frac{\ln\left(\cos(x) + \frac{x^2}{2}\right)}{x^2}}
            \stackrel{Ec. \ref{Ej19.2.Ind}}{=} e^0 = 1
        \end{equation*}
        donde he tenido que resolver en primer lugar la siguiente indeterminación:
        \begin{multline}\label{Ej19.2.Ind}
            \lim_{x\to 0^+}\frac{\ln\left(\cos(x) + \frac{x^2}{2}\right)}{x^2}
            \stackrel{L'H\hat{o}pital}{=}
            \lim_{x\to 0^+} \frac{\frac{-sen(x)+x}{cos(x)+\frac{x^2}{2}}}{2x}
            = \lim_{x\to 0^+} \frac{x-sen(x)}{2x\cos(x)+x^3}
            \stackrel{L'H\hat{o}pital}{=}\\=
            \lim_{x\to 0^+} \frac{1-\cos(x)}{2\cos(x) -2x\sen(x) +3x^2} = 0
        \end{multline}

        \item $A = \left] 0, \frac{\pi}{2}\right[\;,\quad f(x)=\frac{x-\arctg(x)}{\sen^3(x)}\quad (x\in A)$,
        \begin{equation*}\begin{split}
            \lim_{x\to 0^+}f(x) &
            = \lim_{x\to 0^+} \frac{x-\arctg(x)}{\sen^3(x)}
            \stackrel{L'H\hat{o}pital}{=}
            \lim_{x\to 0^+} \frac{1-\frac{1}{1+x^2}}{3\sen^2(x)\cos(x)}
            \stackrel{L'H\hat{o}pital}{=} \\
            &= \lim_{x\to 0^+} \frac{\frac{2x}{(1+x^2)^2}}{6\sen(x)\cos^2(x)-3\sen^3(x)}
            \stackrel{L'H\hat{o}pital}{=}\\
            &= \lim_{x\to 0^+} \frac{\frac{2(1+x^2)^2 -2(2x)^2(1+x^2)}{(1+x^2)^4}}{6\cos^3(x) -12\sen^2(x)\cos(x) -9\sen(x)^2\cos(x)}=\\
            &= \lim_{x\to 0^+} \frac{\frac{2+2x^2 -4x^2}{(1+x^2)^3}}{6\cos^3(x) -21\sen^2(x)\cos(x)} = \frac{2}{6} = \frac{1}{3}
        \end{split}\end{equation*}

        \item $A = \left] 0, \frac{\pi}{2}\right[\;,\quad f(x)=(1-\tg{x})^\frac{1}{x^2}\quad (x\in A)$,
        \begin{equation*}
            \lim_{x\to 0^+}f(x)
            = \lim_{x\to 0^+} (1-\tg{x})^\frac{1}{x^2}
            = \lim_{x\to 0^+} e^{\frac{\ln\left(1-\tg{x}\right)}{x^2}}
            \stackrel{Ec. \ref{Ej19.4.Ind}}{=} e^{-\infty} = 0
        \end{equation*}
        donde he tenido que resolver en primer lugar la siguiente indeterminación:
        \begin{equation}\label{Ej19.4.Ind}
            \lim_{x\to 0^+}\frac{\ln\left(1-\tg{x}\right)}{x^2}
            \stackrel{L'H\hat{o}pital}{=}
            \lim_{x\to 0^+} \frac{\frac{-1-\tg^2{x}}{1-\tg{x}}}{2x} = \frac{-1}{0^+} = -\infty 
        \end{equation}

        \item $A = \bb{R}^+\;,\quad f(x)=x^{\sen(x)}\quad (x\in A)$,
        \begin{equation*}
            \lim_{x\to 0^+}f(x)
            = \lim_{x\to 0^+} x^{\sen{x}}
            = \lim_{x\to 0^+} e^{\sen{x}\ln{x}}
            \stackrel{Ec. \ref{Ej19.5.Ind}}{=} e^0=1
        \end{equation*}
        donde he tenido que resolver en primer lugar la siguiente indeterminación:
        \begin{multline}\label{Ej19.5.Ind}
            \lim_{x\to 0^+}\sen{x}\ln{x}
            = \lim_{x\to 0^+}\frac{\ln{x}}{\frac{1}{\sen{x}}}
            \stackrel{L'H\hat{o}pital}{=}
            \lim_{x\to 0^+} \frac{\frac{1}{x}}{-\frac{\cos{x}}{\sen^2{x}}}
            = \lim_{x\to 0^+} \frac{-\sen^2{x}}{x\cos{x}}
            \stackrel{L'H\hat{o}pital}{=}\\=
            \lim_{x\to 0^+} \frac{-2\sen{x}\cos{x}}{\cos{x}-x\sen{x}} = \frac{0}{1} = 0
        \end{multline}
        
    \end{enumerate}
\end{ejercicio}

\begin{ejercicio}
    Sea $a\in \bb{R}$, y sea $f:\left]-\frac{\pi}{2},\frac{\pi}{2} \right[   \to \bb{R}$ la función dada por:
    $$f(x)=\left\{
    \begin{array}{ccc}
    \displaystyle \frac{\ln(1-\sen{x}) -2\ln(\cos{x})}{\sen{x}} & \text{si} & x\in \displaystyle \left]-\frac{\pi}{2},\frac{\pi}{2} \right[ \backslash \{0\} \\
    a & \text{si} & x=0
    \end{array}\right.$$
    Estudiar la continuidad y la derivabilidad de $f$ en función del valor del parámetro $a$.\\

    Por el carácter local de la continuidad, $f$ es continua en $\displaystyle \left]-\frac{\pi}{2},\frac{\pi}{2} \right[ \backslash \{0\}$. Veamos para $x=0$.
    \begin{equation*}
        \lim_{x\to0}f(x) = \frac{\ln(1-\sen{x}) -2\ln(\cos{x})}{\sen{x}} = \frac{0}{0} \stackrel{L'H\hat{o}pital}{=}
        \lim_{x\to0} \frac{\frac{-\cos(x)}{1-\sen(x)} +2\frac{\sen(x)}{\cos(x)}}{\cos(x)} = \frac{\frac{-1}{1} + 2\tg(0)}{1} = -1
    \end{equation*}
    Por tanto, para que sea continua en $x=0$ es necesario que $f(0)=a=-1 =L$.
    \begin{itemize}
        \item \underline{Si $a\neq -1$}: $f$ no es continua en $x=0$.
        \item \underline{Si $a=-1$}: $f$ es continua en $x=0$, con $f(0)=-1$.
    \end{itemize}

    Veamos ahora la derivabilidad. Por el carácter local de la derivabilidad, $f$ es derivable en $\displaystyle \left]-\frac{\pi}{2},\frac{\pi}{2} \right[ \backslash \{0\}$, con:
    \begin{equation*}
        f'(x) = \frac{\left(\frac{-\cos(x)}{1-\sen(x)} +2\frac{\sen(x)}{\cos(x)}\right)\sen(x) -\cos(x)[\ln(1-\sen{x}) -2\ln(\cos{x})]}{\sen^2(x)} \qquad
        \forall x\in \displaystyle \left]-\frac{\pi}{2},\frac{\pi}{2} \right[ \backslash \{0\}
    \end{equation*}
    Veamos si es derivable en $x=0$:
    \begin{multline*}
        f'(0)= \lim_{x\to 0}\frac{f(x)-f(0)}{x-0}
        = \lim_{x\to 0}\frac{\frac{\ln(1-\sen{x}) -2\ln(\cos{x})}{\sen{x}}-a}{x}
        =\\= \lim_{x\to 0}\frac{\ln(1-\sen{x}) -2\ln(\cos{x})-a\sen{x}}{x\sen{x}}
        \stackrel{L'H\hat{o}pital}{=}\\=
        \lim_{x\to 0}\frac{\frac{-\cos{x}}{1-\sen{x}}+2\frac{\sen{x}}{\cos{x}} -a\cos{x}}{\sen{x}+x\cos{x}}=\frac{-1-a}{0} = 
        \left\{
        \begin{array}{cc}
            \frac{0}{0} & \text{ si } a=-1   \\
            \frac{k}{0} = \infty & \text{ si } a\neq-1 
        \end{array}
        \right.
    \end{multline*}
    Por tanto, para $a=-1$, y sabiendo que $f$ es continua, aplico L'Hôpital.
    \begin{equation*}
        f'(0)=\lim_{x\to 0} \frac{\frac{\sen{x}(1-\sen{x}) -\cos^2{x}}{(1-\sen{x})^2} +2(1+\tg^2{x}) -\sen{x}}{\cos{x}+\cos{x} -x\sen{x}} = \frac{-1+2-0}{2}=\frac{1}{2}
    \end{equation*}
    Por tanto, para $a=-1$, $f$ es continua y derivable en $\displaystyle \left]-\frac{\pi}{2},\frac{\pi}{2} \right[$ con
    $$f(x)=\left\{
    \begin{array}{ccc}
    \displaystyle \frac{\left(\frac{-\cos(x)}{1-\sen(x)} +2\frac{\sen(x)}{\cos(x)}\right)\sen(x) -\cos(x)[\ln(1-\sen{x}) -2\ln(\cos{x})]}{\sen^2(x)} & \text{si} & x\in \displaystyle \left]-\frac{\pi}{2},\frac{\pi}{2} \right[ \backslash \{0\} \\
    0 & \text{si} & x=0
    \end{array}\right.$$
    En el caso de $a\neq-1$, $f$ no es continua en $x=0$ y por tanto tampoco es derivable en dicho punto.
    
\end{ejercicio}


\begin{ejercicio}
    Estudiar el comportamiento en $+\infty$ de la función $f:A\to \bb{R}$, en cada uno de los siguientes casos:
    \begin{enumerate}
    
        \item $A = \bb{R}^+,\quad f(x)=(a^x + x)^{\frac{1}{x}},\quad a\in \bb{R}^+$.
        \begin{equation*}
            \lim_{x\to\infty}f(x)
            = \lim_{x\to\infty}(a^x + x)^{\frac{1}{x}}
            = \lim_{x\to\infty}e^{\frac{\ln(a^x+x)}{x}}
            \stackrel{Ec. \ref{Ej21.1.Ind}}{=} \left\{ \begin{array}{cc}
                e^0=1 & \text{ si } a\leq1 \\
                e^{\ln{a}}=a & \text{ si } a>1
            \end{array}\right.
        \end{equation*}
        donde he tenido que resolver en primer lugar la siguiente indeterminación:
        \begin{multline}\label{Ej21.1.Ind}
            \lim_{x\to \infty}\frac{\ln(a^x+x)}{x}
            \stackrel{L'H\hat{o}pital}{=}
            \lim_{x\to \infty}\frac{e^{x\ln(a)}\ln(a)}{a^x+x} = \left\{ \begin{array}{cc}
                \frac{0}{\infty}=0 & \text{ si } a\leq1 \\
                \frac{\infty}{\infty} & \text{ si } a>1
            \end{array}\right.
        \end{multline}
        En el caso de que $a>1$, aplicamos la Regla de L'Hôpital.
        \begin{equation*}
            \lim_{x\to \infty}\frac{e^{x\ln(a)}\ln^2(a)}{e^{x\ln(a)}\ln(a)} = \ln a
        \end{equation*}
    
        \item $A = ]1, +\infty[\;,\quad f(x)=\frac{x(x^\frac{1}{x}-1)}{\ln{x}}$.
        \begin{multline*}
            \lim_{x\to \infty}f(x)
            = \lim_{x\to \infty}\frac{x(x^\frac{1}{x}-1)}{\ln{x}}
            = \lim_{x\to \infty}\frac{x^\frac{1}{x}-1}{\frac{\ln{x}}{x}}
            \stackrel{Ec.\;\ref{Ej21.2.Ind1}\; y\;\ref{Ej21.2.Ind2}}{=}\frac{0}{0} 
            \stackrel{L'H\hat{o}pital}{=}\\=
            \lim_{x\to \infty}\frac{e^{\frac{\ln{x}}{x}}\cancel{\frac{1-\ln{x}}{x^2}}}{\cancel{\frac{1-\ln{x}}{x^2}}}
            = \lim_{x\to \infty}e^{\frac{\ln{x}}{x}} \stackrel{Ec.\;\ref{Ej21.2.Ind1}}{=} e^0 = 1
        \end{multline*}
        donde he hecho uso de que:
        \begin{equation}\label{Ej21.2.Ind1}
            \lim_{x\to \infty} \frac{\ln{x}}{x} \stackrel{L'H\hat{o}pital}{=}
            \lim_{x\to \infty} \frac{\frac{1}{x}}{1} = 0 
        \end{equation}
        \begin{equation}\label{Ej21.2.Ind2}
            \lim_{x\to \infty} x^{\frac{1}{x}}
            = \lim_{x\to \infty} e^{\frac{\ln{x}}{x}} \stackrel{Ec.\;\ref{Ej21.2.Ind1}}{=} e^0 = 1
        \end{equation}
        
        \item $A = \bb{R}^+,\quad f(x)=x^a\sen\left(\frac{1}{x}\right),\quad a\in \bb{R}$.
        \begin{itemize}
            \item \underline{Si $a=0$}
            \begin{equation*}
                \lim_{x\to \infty} f(x)
                = \lim_{x\to \infty} x^a\sen\left(\frac{1}{x}\right) = 1\cdot \sen(0) = 0
            \end{equation*}
            \item \underline{Si $a<0$}
            \begin{equation*}
                \lim_{x\to \infty} f(x)
                = \lim_{x\to \infty} x^a\sen\left(\frac{1}{x}\right) = 0\cdot \sen(0) = 0
            \end{equation*}
            \item \underline{Si $a>0$}
            \begin{equation*}
                \lim_{x\to \infty} f(x)
                = \lim_{x\to \infty} x^a\sen\left(\frac{1}{x}\right)
                = \lim_{x\to \infty} \frac{\sen\left(\frac{1}{x}\right)}{x^{-a}}
                \stackrel{L'H\hat{o}pital}{=}
                \lim_{x\to \infty} \frac{-\cos\left(\frac{1}{x}\right)\frac{1}{x^2}}{-ax^{-a-1}}
                = \lim_{x\to \infty} \frac{-\cos\left(\frac{1}{x}\right)}{-ax^{-a+1}}
            \end{equation*}
            \begin{itemize}
                \item \underline{Si $a<1$}
                $$\lim_{x\to \infty} f(x)
                = \frac{1}{-\infty} = 0$$
                \item \underline{Si $a=1$}
                $$\lim_{x\to \infty} f(x)
                = \frac{1}{1} = 1$$
                \item \underline{Si $a>1$}
                $$\lim_{x\to \infty} f(x)
                = \frac{1}{0^+} = +\infty
                $$
            \end{itemize}
        \end{itemize}

        Por tanto,
        \begin{equation*}
            \lim_{x\to \infty} f(x) =
            \left\{ \begin{array}{cc}
                0 & \text{ si } a<1 \\
                1 & \text{ si } a=1 \\
                +\infty & \text{ si } a>1 \\
            \end{array}\right.
        \end{equation*}
        
        \item $A = \bb{R}^+,\quad f(x)=\frac{x^2\sen\frac{1}{x}}{\ln{x}}.$
        \begin{multline*}
            \lim_{x\to \infty} f(x)
            = \lim_{x\to \infty}\frac{x^2\sen\frac{1}{x}}{\ln{x}}
            = \lim_{x\to \infty}\frac{\sen\frac{1}{x}}{\frac{\ln{x}}{x^2}} = \left[\frac{0}{0} \right]
            \stackrel{L'H\hat{o}pital}{=}
            \lim_{x\to \infty} \frac{-\cos\left(\frac{1}{x}\right)\frac{1}{x^2}}{\frac{x-2x\ln{x}}{x^4}}
            =\\=
            \lim_{x\to \infty} \frac{-\cos\left(\frac{1}{x}\right)x}{1-2\ln{x}}
            \stackrel{L'H\hat{o}pital}{=}
            \frac{\frac{-\sen\left(\frac{1}{x}\right)}{x}-\cos\left(\frac{1}{x}\right)}{-\frac{2}{x}}
            = \frac{\sen\left(\frac{1}{x}\right) +x\cos\left(\frac{1}{x}\right)}{2} = \frac{0+\infty\cdot 1}{2} =\infty
        \end{multline*}
    \end{enumerate}
\end{ejercicio}

\begin{ejercicio}
    Dadas las funciones $f,g:A\to \bb{R}$ y $a\in A$, demostrar que si $f$ es derivable en $a$, siendo $f(a)=0$, y si $g$ es continua en $a$ entonces $fg$ \footnote{$fg$ no se refiere a la composición, sino a la multiplicación. Esto se debe a que no se pueden componer por los dominios.} es derivable en $a$.
    \begin{proof}
        Aplicamos la definición formal de derivada en un punto:
        \begin{multline*}
            (fg)'(a)=\lim_{x\to a}\frac{f(x)g(x) - \cancelto{0}{f(a)g(a)}}{x-a}
            = \lim_{x\to a}\frac{f(x)g(x)}{x-a}
            = \lim_{x\to a}\frac{f(x)}{x-a} g(x)
            =\\=
            \lim_{x\to a}\frac{f(x)-f(a)}{x-a} g(x) \stackrel{(\ast)}{=} f'(a)g(a)
        \end{multline*}
        donde en $(\ast)$ he aplicado que el límite del producto converge al producto de los límites. El primer límite es la definición formal de $f'(a)$ y, en segundo lugar, he aplicado que $g$ es continua en $x=a$.
    
        Por tanto, se ha demostrado que $fg$ es derivable en $x=a$, con $(fg)'(a) = f'(a)g(a)$.
    \end{proof}
\end{ejercicio}

\begin{ejercicio}
    Sea $r>1$. Si $f$ es una función real de variable real tal que $|f(x)|\leq |x|^r$ en algún intervalo abierto que contenga al cero, demostrar que entonces $f$ es derivable en cero.
    \begin{proof}
        Veamos en primer lugar el valor de $f(0)$.
        $$0 \leq |f(0)| \leq |0|^r = 0 \Longrightarrow |f(0)| = 0 \Longrightarrow f(0) = 0$$
    
        Usando la definición de la derivada de una función en un punto,
        \begin{equation*}
            f'(0)=\lim_{x\to 0} \frac{f(x)-f(0)}{x-0} = \lim_{x\to 0} \frac{f(x)}{x} \stackrel{Ec.\;\ref{Ej23.Sand}}{=} 0
        \end{equation*}
        donde he usado este resultado:
        \begin{equation}\label{Ej23.Sand}
            0 \leq \left|\frac{f(x)}{x}\right| \leq \frac{|x|^r}{|x|}
            = |x|^{r-1} \stackrel{(\ast)}{\Longrightarrow} \lim_{x\to 0} \left|\frac{f(x)}{x}\right| = 0 \Longrightarrow \lim_{x\to 0} \frac{f(x)}{x} = 0
        \end{equation}
        donde en $(\ast)$ he usado que, como el exponente es $r-1>0$, la sucesión $\{|x|^{r-1}\}\longrightarrow 0$. Por el Lema del sándwich, el límite buscado es 0.
    
        Por tanto, por lo demostrado anteriormente, $f$ es derivable en $x=0$, con $f'(0)=~0$.
    \end{proof}
\end{ejercicio}


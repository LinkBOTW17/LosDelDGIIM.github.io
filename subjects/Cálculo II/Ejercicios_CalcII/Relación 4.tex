\section{Continuidad Uniforme}

\begin{prop*}
    Si $f:A\to \bb{R}$ uniformemente continua en $A$ y $\emptyset \neq B \subseteq A$ entonces $f$ es uniformemente continua en $B$.
\end{prop*}

\begin{prop*}\label{Prop:Pegado}
    Sea $f:I\to \bb{R}$, y dado $a\in I$. Si $f$ es uniformemente continua en $I \cap ]-\infty, a[$ y en $I\cap ]a, +\infty[$ y $f$ continua en $a$, entonces $f$ uniformemente continua en $I$.
\end{prop*}
\begin{proof}
    Dado $\varepsilon >0$. $\exists \delta_1, \delta_2, \delta_3$ tales que:
    \begin{itemize}
        \item Si $x,y\in I\cap ]-\infty, 
        a[$ con $|x-y|<\delta_1 \Longrightarrow |f(x)-f(y)| < \varepsilon$
        \item Si $|x-a|<\delta_2 \Longrightarrow |f(x)-f(a)|<\frac{\varepsilon}{2}$
        \item Si $x,y\in I\cap ]a, +\infty[$ con $|x-y|<\delta_3 \Longrightarrow |f(x)-f(y)| < \varepsilon$
    \end{itemize}

    Sea $\delta = \min \{\delta_1, \delta_2, \delta_2\}$. Si $|x-y| < \delta$, hay tres posibilidades:
    \begin{enumerate}
        \item $x,y < a \Longrightarrow x,y\in ]-\infty, a[ \Longrightarrow $ $f$ uniformemente continua en dicho intervalo.

        \item $x,y > a \Longrightarrow x,y\in ]a, +\infty[ \Longrightarrow $ $f$ uniformemente continua en dicho intervalo.

        \item $x \leq a \leq y \Longrightarrow x\in  I\cap ]-\infty, 
        a[, y\in I\cap ]a, +\infty[$\\
        Como $|x-y| = y-x < \delta \Longrightarrow
        \left\{\begin{array}{l}
            |x-a| = a-x \leq y-x < \delta \Longrightarrow |f(x) - f(a)| < \frac{\varepsilon}{2} \\
            |a-y| = y-a \leq y-x < \delta \Longrightarrow |f(a) - f(y)| < \frac{\varepsilon}{2}
        \end{array} \right.
        $

        Por tanto, tenemos el siguiente resultado:
        \begin{multline*}
            |f(x)-f(y)| = |f(x)-f(a)+f(a) - f(y)| 
            \leq \\ \leq 
            |f(x)-f(a)| + |f(a) - f(y)| < \frac{\varepsilon}{2} + \frac{\varepsilon}{2} = \varepsilon
        \end{multline*}

        Por tanto, si $|x-y|<\delta \Longrightarrow |f(x)-f(y)| < \varepsilon \Longrightarrow f$ es uniformemente continua en este caso.
    \end{enumerate}
\end{proof}

\begin{ejercicio} Sean $f,g:A\longrightarrow \bb{R}$ funciones uniformemente continuas en $A$. Probar que $f+g$ también es uniformemente continua en $A$. Si adicionalmente las funciones $f$ y $g$ están acotadas en $A$, demostrar que entonces $fg$ también es uniformemente continua en $A$.\\

Demostramos en primer lugar que $f+g$ es uniformemente continua.

Dado $\varepsilon>0$, tomamos $\frac{\varepsilon}{2}$. Entonces, como $f,g$ son uniformemente continuas, $\exists \delta_1,\delta_2>~0$ tal que si:
\begin{equation*}
    \begin{array}{l}
        |x-y| < \delta_1 \Longrightarrow |f(x) - f(y)| < \frac{\varepsilon}{2} \\
        |x-y| < \delta_2 \Longrightarrow |g(x) - g(y)| < \frac{\varepsilon}{2} 
    \end{array}
\end{equation*}

Por tanto,
\begin{multline*}
    Si\; |x-y| < \delta = \min\{\delta_1, \delta_2\} \Longrightarrow |f(x)+g(x) - (f(y) + g(y))| \leq \\
    \leq |f(x)-f(y)| + |g(x)-g(y)| <  \frac{\varepsilon}{2} + \frac{\varepsilon}{2} = \varepsilon
\end{multline*}
Por tanto, $f+g$ es uniformemente continua.

Supuestas $f,g$ acotadas en $A$, veamos ahora que $fg$ es uniformemente continua.

Como $f$ está acotada, sea $M_1$ tal que $|f(x)| \leq M_1\; \forall x\in A$.

Como $g$ está acotada, sea $M_2$ tal que $|g(x)| \leq M_2\; \forall x\in A$.\\


Sea $\delta_1$ el asociado a $\frac{\varepsilon}{2M_1}$ para la continuidad uniforme de $f$ en $A$.
$$\text{Si } |x-y|<\delta_1 \Longrightarrow |f(x)-f(y)|<\frac{\varepsilon}{2M_1}$$

Sea $\delta_2$ el asociado a $\frac{\varepsilon}{2M_2}$ para la continuidad uniforme de $g$ en $A$.
$$\text{Si } |x-y|<\delta_2 \Longrightarrow |g(x)-g(y)|<\frac{\varepsilon}{2M_2}$$


Sea $\delta = \min \{\delta_1, \delta_2\}$. Entonces, por la Ec. \ref{Ec_Ej1},
$$\text{Si } |x-y|<\delta \Longrightarrow |f(x)g(x)-f(y)g(y)| \leq \varepsilon$$
\begin{multline}\label{Ec_Ej1}
    |f(x)g(x) - f(y)g(y)| = |f(x)g(x) -f(x)g(y) + f(x)g(y) - f(y)g(y)|
    \leq \\ \leq
    |f(x)| |g(x) - g(y)| + |f(x) - f(y)||g(y)|
    \leq \\ \leq
    M_1 |g(x) - g(y)| + M_2|f(x) - f(y)| \leq
     M_1 \frac{\varepsilon}{2M_1} + M_2\frac{\varepsilon}{2M_2} = \varepsilon
\end{multline}

\begin{observacion}
    $f(x)=x$ sí es uniformemente continua en $\bb{R}$, pero no podemos asegurar que $g(x)=x^2$ lo sea, ya que $\bb{R}$ no está acotado. De hecho, no lo es.
\end{observacion}
\end{ejercicio}

\begin{ejercicio}
    Probar que una función $f:A\longrightarrow \bb{R}$ uniformemente continua transforma conjuntos acotados en conjuntos acotados.\\

    Se pide demostrar lo siguiente:
    
    Supuesto $B \subseteq A$ acotado, es decir, $|x|\leq M \; \forall x\in B$, entonces $f(B)$ está acotado, es decir, $\exists \hat{M}$ con $|f(x)| \leq \hat{M} \; \forall x\in B$. En particular, si $A$ está acotado y $f$ es uniformemente continua entonces $f(A)$ está acotado.

    \begin{proof} Demostramos mediante reducción al absurdo.
    
    Supongamos $B \subseteq A$ acotado pero $f(B)$ no mayorado.
    
    Entonces consideramos la sucesión $\{b_n\}\subseteq B$ acotado tal que
    \begin{equation*}
        f(b_{n+1}) = f(b_n) + \varepsilon_0\qquad \varepsilon_0>0 \text{ fijo}
    \end{equation*}
    Definir de esta forma $\{b_n\}$ siempre es posible, ya que $f(B)$ no está mayorado. Por tanto, siempre podremos encontrar un elemento de $B$ con una imagen mayor, definiendo así $\{b_n\}$.
    
    Además, como $\{b_n\}\subset B$ acotado, entonces $\{b_n\}$ admite una parcial convergente. Es decir, $\exists \sigma$ parcial tal que $\{b_{\sigma(n)}\} \longrightarrow L \in \bb{R}$. Además, tenemos también que $\{b_{\sigma(n+1)}\} \longrightarrow L$.

    Por tanto, tenemos que
    \begin{equation*}
        \{b_{\sigma(n+1)}\} - \{b_{\sigma(n)}\} \longrightarrow L-L=0
    \end{equation*}
    No obstante,
    \begin{equation*}
        |f(b_{\sigma(n+1)}) - f(b_{\sigma(n)})|
         = |f(b_{\sigma(n)}) + \varepsilon_0 - f(b_{\sigma(n)})| = \varepsilon_0>0
    \end{equation*}
    Por tanto, tenemos que $f$ no es uniformemente continua, llegando por tanto a una contradicción, por lo que $f(B)$ mayorado.

    Si hubiésemos supuesto que $f(B)$ no minorado, el procedimiento es análogo pero 
    \begin{equation*}
        f(b_{n+1}) = f(b_n) - \varepsilon_0\qquad \varepsilon_0>0
    \end{equation*}
    demostrando así que $f(B)$ minorado.
\end{proof}

    \begin{observacion}
    Las funciones continuas transforman cerrados y acotados en cerrados y acotados. Las uniformemente continuas transforman acotados en acotados (algo que no ocurre con las continuas).

    Esto prueba que la función $f:]0, 1]\longrightarrow \bb{R}$ dada por $f(x)=\frac{1}{x}$ no es uniformemente continua, ya que $Im(f)=[1, +\infty[$.

    De no disponer de esta herramienta, habría que encontrar $x_n,\;y_n$ con $\{x_n-~y_n\} \to~0$ y $|f(x_n) - f(y_n)| \geq \varepsilon_0$. En este caso, un ejemplo sería
    $$x_n = \frac{1}{n} \qquad y_n = \frac{1}{n+1} \qquad |f(x_n) - f(y_n)| = |-1|=1$$
    \end{observacion}
\end{ejercicio}

\begin{ejercicio}
    Sea $f:\bb{R}^+\longrightarrow \bb{R}$ la función $f(x)=\frac{1}{x}$, para cada $x\in \bb{R}^+$. Dado $r > 0$, probar que la restricción de $f$ a $[r, +\infty[$ es lipschitziana, mientras que la restricción de $f$ a $]0, r]$ no es uniformemente continua. ¿Sucede lo mismo con $g(x)=\ln x$?\\

    Su derivada es $f'(x)=-\frac{1}{x^2}$.
    
    Sé que $f'([r, +\infty[)=]0,-\frac{1}{r^2}]$ está acotado, por lo que es lipschitziana en $[r, +\infty[$.
    
    Sin embargo, no es uniformemente continua en $]0, r]$ ya que $f(]0, r])$ no está acotadado y $]0, r]$ acotado. Como transforma conjuntos acotados en no acotados, no es uniformemente continua.\\


    Veamos ahora el caso de $g(x)=\ln x$. Si fuese $\ln(x)$ uniformemente continua en $\bb{R}^+$, también lo sería en $]0, r]$ y en consecuencia, tendría que $\ln(]0, r])$ acotado y no lo es. Como conclusión, el $\ln x$ no es uniformemente continua en $]0, r]$ y por tanto no lo puede ser en $\bb{R}^+$. Sin embargo sí es lipschitziana en $[r, +\infty[ \; \forall r>0$. Por tanto, sí sucede lo mismo con $g(x)=\ln x$.
\end{ejercicio}

\begin{ejercicio}
    Sea $I$ un intervalo no trivial ($I\neq \emptyset,\;I\neq \{a\},\;I\neq \pm \bb{R}$). Probar que si todas las funciones continuas en $I$ son uniformemente continuas en $I$ entonces $I$ es un intervalo cerrado y acotado.\\

    Esto es, en el Teorema de Heine no podemos reemplazar $[a,b]$ por otro intervalo~$I$. 
    
    Dado $a,b\in \bb{R}\mid a<b$, los tipos de intervalos no triviales que hay definidos son: $[a,b], [a,b[, ]a,b], ]a,b[, [a, +\infty[, ]a,+\infty[, ]-\infty, b], ]-\infty, b[$.
    \begin{enumerate}
        \item $I=[a, b[ \quad I'=]a, b[ \quad I''=]-\infty,b[$\\
        Sea $f(x)=\frac{1}{b-x}$. Las imágenes son:
        \begin{equation*}
            f(I)=[f(a),+\infty[ \quad f(I')=]f(a),+\infty[
        \end{equation*}

        $f$ es continua en $I,I',I''$.
        
        $f$ no es uniformemente continua en $I,I'$, ya que el dominio es acotado pero la imagen no, es decir, no transforma conjuntos acotados en acotados.

        Además, tampoco es uniformemente continua en $I''=]-\infty,b[$, pues de serlo también lo sería en $I,I'\subset I''$.

        Por tanto, como $\exists f$ continua que no es uniformemente continua, estos intervalos quedan descartados.

        \item $I=]a, b] \quad I'=]a, +\infty[$\\
        Sea $f(x)=\frac{1}{a-x}$. Las imágenes son:
        \begin{equation*}
            f(I)=]-\infty, f(b)]
        \end{equation*}
        $f$ es continua en $I$.
        
        $f$ no es uniformemente continua en $I$, ya que el dominio es acotado pero la imagen no, es decir, no transforma conjuntos acotados en acotados.

        Además, tampoco es uniformemente continua en $I'=]a,+\infty[$, pues de serlo también lo sería en $I\subset I'$.

        Por tanto, como $\exists f$ continua que no es uniformemente continua, estos intervalos quedan descartados.

        \item $I=[a,+\infty[ \quad I'=]a, +\infty[$

        Este contraejemplo es válido para los conjuntos no mayorados. La función $f(x)=~x^2$ es continua en los intervalos $I,I'$.
        
        No obstante, sean las sucesiones
        \begin{equation*}
            \{x_n\} = n+\frac{1}{n}
            \qquad
            \{y_n\} = n
        \end{equation*}

        Tenemos que:
        \begin{equation*}
            x_n-y_n = n+\frac{1}{n} -n = \frac{1}{n}  \to 0
        \end{equation*}
        \begin{equation*}
            |f(x_n)-f(y_n)| = \left| n^2 +2\frac{n}{n} + \frac{1}{n^2} -n^2 \right| = \left| 2+\frac{1}{n^2}\right| > 2
        \end{equation*}

        Por tanto, como $\exists f$ continua que no es uniformemente continua, estos intervalos quedan descartados.

        \item $I=]-\infty,b] \quad I'=]-\infty,b[$

        Este contraejemplo es válido para los conjuntos no minorados. La función $f(x)=~x^2$ es continua en los intervalos $I,I'$.
        
        No obstante, sean las sucesiones
        \begin{equation*}
            \{x_n\} = -n-\frac{1}{n}
            \qquad
            \{y_n\} = -n
        \end{equation*}

        Tenemos que:
        \begin{equation*}
            x_n-y_n = -n-\frac{1}{n} +n = -\frac{1}{n}  \to 0
        \end{equation*}
        \begin{equation*}
            |f(x_n)-f(y_n)| = \left| n^2 +2\frac{n}{n} + \frac{1}{n^2} -n^2 \right| = \left| 2+\frac{1}{n^2}\right| > 2
        \end{equation*}

        Por tanto, como $\exists f$ continua que no es uniformemente continua, estos intervalos quedan descartados.        

        \item $I=[a,b]$\\
        Por el Teorema de Heine, en este caso continuidad uniforme y continuidad son equivalentes, por lo que es el único caso en el que se cumple.
    \end{enumerate}
\end{ejercicio}

\begin{ejercicio}
    Sea $f:\bb{R}\longrightarrow \bb{R}$ una función continua y sea $r > 0$. Probar que si la restricción de $f$ al conjunto $\{x \in \bb{R} \mid |x|\geq r\}$ es uniformemente continua, entonces $f$ es uniformemente continua.\\

    Sabemos que $f$ es uniformemente continua en $[-r, r]$ por Heine. También, el enunciado nos afirma que $f$ es uniformemente continua en $]-\infty, -r]$ y en $[r, +\infty[$.\\

    Por la proposición \ref{Prop:Pegado} (la herramienta de ``pegado''), $f$ es uniformemente continua en $\bb{R}$. Veamos el proceso.

    En primer lugar, demuestro que es uniformemente continua en $]-\infty, r]$. Para ello, es necesario usar que $f$ es continua en $-r$ y que es uniformemente continua en $]-\infty,-r],[-r,r]$.

    Una vez tenemos que es uniformemente continua en $]-\infty, r]$, vemos que lo es en $\bb{R}$. Para ello, es necesario usar que $f$ es continua en $r$ y que es uniformemente continua en $]-\infty,r],[r,+\infty]$.

    Usando dos veces la proposición \ref{Prop:Pegado}, tenemos que $f$ es continua en $\bb{R}$.
    \begin{comment}
    Dado $\varepsilon > 0$:
    \begin{itemize}
        \item Por la continuidad uniforme en $]-\infty,r]$ dada por el enunciado,
        $$\exists \delta_1 \mid \text{ si }|x-y| < \delta_1 \text{ con } x,y \leq r \Longrightarrow |f(x) - f(y)| < \varepsilon$$
        \item Por la continuidad uniforme en $[-r,r]$ dada por Heine:
        $$\exists \delta_2 \mid \text{ si }|x-y| < \delta_1\text{ con }x,y \in [-r, r] \Longrightarrow |f(x) - f(y)| < \varepsilon$$
        \item Por la continuidad uniforme en $[r,+\infty[$ dada por el enunciado, $$\exists \delta_3 \mid \text{ si }|x-y| < \delta_1 \text{ con } x,y \geq r \Longrightarrow |f(x) - f(y)| < \varepsilon$$
        \item Por la continuidad en $r$:\\
        $$\exists \delta_4 \mid \text{ si }|x-r| < \delta_4 \Longrightarrow |f(x) - f(r)| < \frac{\varepsilon}{2}$$
        \item Por la continuidad en $-r$:\\
        $$\exists \delta_5 \mid \text{ si }|x-(-r)|=|x+r| < \delta_5 \Longrightarrow |f(x) - f(-r)| < \frac{\varepsilon}{2}$$
    \end{itemize}
    \end{comment}
\end{ejercicio}

\begin{ejercicio} \label{Ej6}
    Sea $a<b$ y $f:]a,b[ \longrightarrow \bb{R}$ una función. Demostrar que las siguientes afirmaciones son equivalentes:
    \begin{enumerate}
        \item $\exists \hat{f}:[a,b]\longrightarrow \bb{R}$ continua tal que $\hat{f}_{\big|]a,b[}=f$.
        \item $f$ es uniformemente continua en $]a,b[$.
    \end{enumerate}

    \begin{description}
    \item [$1 \Longrightarrow 2$)]
     $\hat{f}$ continua en $[a,b] \stackrel{Heine}{\Longrightarrow} \hat{f}$ uniformemente continua en $[a,b]$
     
     Por tanto, $\hat{f}_{\left|]a, b[\right.}=f$ es uniformemente continua en $]a,b[$, por ser restricción de una uniformemente continua.
    
    \item [$2 \Longrightarrow 1$)] Suponemos $f$ uniformemente continua en $]a,b[$
    
    Veamos ahora que $\exists \hat{f}$ continua en los puntos $a,b$.
    Sea $\hat{f}$ definida como:
    \begin{equation*}
        \hat{f}(x)=\left\{
        \begin{array}{ccc}
            \hat{f}(a) & si &x=a\\
            f(x) & si &x\in ]a,b[ \\
            \hat{f}(b) & si &x=b
        \end{array}
        \right.
    \end{equation*}
    
    Para la continuidad en el punto $a$, es necesario que:
    \begin{equation*}
        \forall x_n\to a,\;x_n\in [a,b] \Longrightarrow \hat{f}(x_n)\to \hat{f}(a)
    \end{equation*}

    Sea $x_n\longrightarrow a,\;x_n\in]a,b[$. Como $]a,b[\subset \bb{R}$ y $x_n$ convergente $\Longrightarrow x_n$ de Cauchy. Por la continuidad uniforme en $]a,b[\Longrightarrow f(x_n)\subseteq\bb{R}$ de Cauchy, y por ende $f(x_n)$ convergente $\Longrightarrow \exists L\in \bb{R}\mid f(x_n)\to L$.

    Veamos ahora que, dada otra sucesión $y_n\to a,\;y_n\in]a,b[$, se cumple que $f(y_n)\to L$. En primer lugar, $f(y_n)$ ha de converger por el mismo razonamiento anterior. Supongamos $f(x_n)\to L'\neq L$. Entonces, tenemos $x_n,y_n\in ]a,b[$ con:
    \begin{equation*}
        x_n-y_n\to a-a=0 \qquad f(x_n)-f(y_n)\to L-L'\neq 0
    \end{equation*}
    En contradicción con que $f$ es uniformemente continua en $]a,b[$. Por tanto, $L=L'$. Es decir,
    \begin{equation*}
        \forall x_n\to a\;x_n\in ]a,b[ \Longrightarrow f(x_n)\to L
    \end{equation*}
    Definiendo $\hat{f}(a)=L$, obtenemos la definición de continuidad en el punto $a$. Por tanto, $\hat{f}$ es continua en el punto $a$.

    Análogamente, se demuestra que, debido a la continuidad uniforme de $f$ en $]a,b[$, $\hat{f}$ es continua en $b$.
    
    Por tanto, $\exists \hat{f}:[a,b]\longrightarrow \bb{R}$ continua tal que $\hat{f}_{\left|]a,b[ \right.}=f$.

    \begin{comment}
        Como $f$ es uniformemente continua en $]a,b[ \text{ (acotado)}\Longrightarrow f(]a,b[) \text{ acotado}$. Por tanto, $\exists M>0\mid f(x)\in [-M,M]\;\forall x\in]a,b[$.

    Sea $x_n\to a,\;x_n\in ]a,b[$. Como $f(x_n)\in[-M,M]$ acotado, por el Teorema de Bolzano-Weierstrass, $\exists f(x_{\sigma(n)})\to L\in \bb{R}$.

    Supongamos ahora que $\exists y_n\to a$ con $f(y_n)\nrightarrow L\Longrightarrow \exists \varepsilon_0 \mid |f(y_n)-L|\geq \varepsilon\;\forall n$.
    \end{comment}
    \end{description}

    \begin{observacion}
        Esto también es válido para $]a,b]$ y $[a, b[$
    \end{observacion}
\end{ejercicio}

\begin{ejercicio}
    Sea $f:\bb{R}\longrightarrow \bb{R}$ una función continua y periódica. Probar que:
    \begin{enumerate}
        \item $f$ está acotada y alcanza (en $\bb{R}$) su máximo y su mínimo absoluto.

        Supongamos que el periodo de la función es $T$. Demostramos en primer lugar que $Im(f) = Im(f\left|_{[0,T]} \right.)$.
        \begin{equation*}
            \forall f(x) \in  Im(f) \qquad \exists k!\in \bb{Z} \mid f(x)=f(x-kT)\text{ con }x-kT\in [0,T[
        \end{equation*}
        De hecho, $k$ es el cociente entero de dividir $x$ entre $T$. El resto, $x-kT$, pertenece a $[0,T[$. Por tanto, $Im(f) \subseteq Im(f\left|_{[0,T]} \right.)$

        El hecho de que $Im(f\left|_{[0,T]} \right.) \subseteq Im(f)$ es trivial, por lo que $Im(f) = Im(f\left|_{[0,T]} \right.)$.

        Como, por el teorema de Weierstarss, la imagen, por una función continua, de un intervalo cerrado y acotado es un intervalo cerrado y acotado, tengo que $Im(f\left|_{[0,T]} \right.)$ es un intervalo cerrado y acotado.

        Por tanto, como $Im(f) = Im(f\left|_{[0,T]} \right.)$, tenemos que $Im(f)$ es un intervalo cerrado y acotado. En particular, tenemos que $f$ está acotada y alcanza en $\bb{R}$ su máximo y su mínimo absolutos.
        
        \item $f$ es uniformemente continua en $\bb{R}$.

        Por el Teorema de Heine, tenemos que $f$ es uniformemente continua en $[0,2T]$. Es decir,
        \begin{equation*}
            \forall {\varepsilon}>0,\;\exists \hat{\delta}>0 \mid  \text{ si }x,y\in [0,2T] \text{ con } |x-y|<\hat{\delta} \Longrightarrow |f(x)-f(y)| < {\varepsilon}
        \end{equation*}

        Sea ahora $\delta = \min \{\hat{\delta}, T\}$. Sea $x,y\in \bb{R} \mid |x-y|<\delta$. Supongamos $x<y$ sin pérdida de generalidad. Entonces, como $|x-y| < \delta \leq T$:
        \begin{equation*}
            \exists k! \in \bb{Z} \mid \left\{ \begin{array}{l}
                f(x)=f(x-kT) \text{ con }x-kT\in [0,T]  \\
                f(y)=f(y-kT) \text{ con }y-kT\in [0,2T]
            \end{array} \right.
        \end{equation*}

        Por tanto, como $x-kT, y-kT \in [0,2T]$ y $f$ es uniformemente continua en dicho intervalo,
        \begin{equation*}
            |f(x)-f(y)| = |f(x-kT) - f(y-kT)| < {\varepsilon}
        \end{equation*}

        En conclusión, tenemos que
        \begin{equation*}
            \forall \varepsilon>0,\;\exists \delta>0 \mid  \text{ si } |x-y|<\delta \Longrightarrow |f(x)-f(y)| < \varepsilon
        \end{equation*}

        Por lo que $f$ es uniformemente continua en $\bb{R}$.
    \end{enumerate}
\end{ejercicio}

\begin{ejercicio}
    Se dice que dos sucesiones $x_n$ e $y_n$ son \emph{paralelas} si, para cada $\varepsilon > 0$, existe $n_0\in \bb{N}$ tal que $|x_n - y_n| < \varepsilon$, para cada $n>n_0$. Demostrar que $f:A\to \bb{R}$ es uniformemente continua en $A$ si, y solo si, transforma sucesiones paralelas de $A$ en sucesiones paralelas de $\bb{R}$.

    \begin{description}
        \item [$\Longrightarrow$)] Supongamos $f$ uniformemente continua y veamos que transforma sucesiones paralelas de $A$ en sucesiones paralelas de $\bb{R}$.

        Sean $x_n,y_n\;(x_n,y_n\in A)$ sucesiones paralelas de $A$. Es decir,
        \begin{equation}\label{Ec:Ej8.Paralelas}
            \forall \hat{\varepsilon} >0\; \exists n_0\in \bb{N}\mid \text{ Si } n>n_0 \Longrightarrow |x_n - y_n| < \hat{\varepsilon}
        \end{equation}

        Veamos que $f(x_n),f(y_n)$ son sucesiones paralelas de $\bb{R}$. Para ello, usamos que $f$ es uniformemente continua, es decir, que:
        \begin{equation}\label{Ec:Ej8.Unif}
            \forall \varepsilon>0,\;\exists \delta>0 \mid  \text{ si } |x-y|<\delta \Longrightarrow |f(x)-f(y)| < \varepsilon
        \end{equation}

        Tomando $\hat{\varepsilon} = \delta$, vemos que, dado $\varepsilon>0$,
        \begin{equation*}
            \exists \delta>0,n_0\in \bb{N} \mid \text{ Si } n>n_0
            \stackrel{Ec.\;\ref{Ec:Ej8.Paralelas}}{\Longrightarrow}
            |x_n-y_n| < \delta
            \stackrel{Ec.\;\ref{Ec:Ej8.Unif}}{\Longrightarrow}
            |f(x_n) - f(y_n)| < \varepsilon
        \end{equation*}

        Por tanto, tenemos que:
        \begin{equation*}
            \forall \varepsilon >0\; \exists n_0\in \bb{N}\mid \text{ Si } n>n_0 \Longrightarrow |f(x_n) - f(y_n)| < \varepsilon
        \end{equation*}

        Es decir, que $f(x_n),f(y_n)$ son paralelas. Además, podemos ver que $n_0$ es el mismo.

        \item [$\Longleftarrow$)] Suponemos que $f$ transforma sucesiones paralelas de $A$ en sucesiones paralelas de $\bb{R}$, y veamos que $f$ es uniformemente continua.

        Por reducción al absurdo, supongamos que $f$ no es uniformemente continua. Es decir, dado $\varepsilon_0>0$,
        \begin{equation*}
            \exists x_n,y_n\subseteq A \mid x_n-y_n \to 0 \land |f(x_n)-f(y_n)| \geq \varepsilon_0 \qquad \forall n\in \bb{N}
        \end{equation*}

        Veamos ahora que $x_n,y_n$ son paralelas:
        \begin{equation*}
            x_n-y_n \to 0 \Longrightarrow 
            \forall {\varepsilon} >0\quad \exists n_0\in \bb{N}\mid \text{ Si } n>n_0 \Longrightarrow |x_n - y_n - 0| < {\varepsilon}
        \end{equation*}

        No obstante, $f(x_n),f(y_n)$ no son paralelas, ya que dado $\varepsilon_0$ tenemos que:
        \begin{equation*}
            \nexists n_0\in \bb{N} \mid \text{ Si } n>n_0 \Longrightarrow |f(x_n)-f(y_n)| < \varepsilon_0
        \end{equation*}

        Por tanto, $f$ transforma un par de sucesiones paralelas en un par de sucesiones no paralelas. Esto contradice nuestra hipótesis, por lo que $f$ sí es uniformemente continua. 
    \end{description}
\end{ejercicio}

\begin{ejercicio}
    Sea $A\subseteq \bb{R}$ un conjunto acotado. Demostrar que $f:A\longrightarrow \bb{R}$ es uniformemente continua en $A$ si, y solo si, preserva las sucesiones de Cauchy. ¿Sería cierto el resultado si $A$ no fuese un conjunto acotado?

    \begin{description}
        \item [$\Longrightarrow$)] Supongamos $f$ uniformemente continua.
        
        Sea $x_n$ sucesión de Cauchy en $A$, es decir,
        $$\forall \hat{\varepsilon}>0\;\exists n_0 \mid si\; m,n\geq n_0 \Longrightarrow |x_n - x_m| < \hat{\varepsilon}$$
        Veamos que $f(x_n)$ es de Cauchy también. Como $f$ es uniformemente continua en $A$,
        \begin{equation}\label{Ec:Ej9.ContUnif}
            \forall \varepsilon>0,\exists \delta>0\mid |x-y|<\delta\Longrightarrow |f(x)-f(y)|<\varepsilon
        \end{equation}

        Uniendo ambos resultados, obtenemos que para $\hat{\varepsilon}=\delta$:
        \begin{equation*}
            \exists n_0 \mid si\; m,n\geq n_0 \Longrightarrow |x_n - x_m| < \delta \stackrel{Ec.\;\ref{Ec:Ej9.ContUnif}}{\Longrightarrow} |f(x_n)-f(x_m)| < \varepsilon 
        \end{equation*}

        Por tanto, resumiendo:
        \begin{equation*}
            \forall \varepsilon>0, \exists n_0\mid si\; m,n\geq n_0 \Longrightarrow |f(x_n)-f(x_m)| < \varepsilon 
        \end{equation*}

        Por tanto, $f(x_n)$ es una sucesión de Cauchy.
        
        \item [$\Longleftarrow$)] Supongamos que $f$ transforma sucesiones de Cauchy en sucesiones de Cauchy. Veamos que $f$ es uniformemente continua.

        Demostramos mediante reducción al absurdo. Supongamos que $f$ no es uniformemente continua y llegaremos a una contradicción.
        
        Como $f$ no es uniformemente continua, dado $\varepsilon_0 > 0$,
        \begin{equation}\label{Ej9:Absurdo}
            \exists x_n,y_n \subseteq A \text{ con } x_n - y_n \to 0 \quad t.q. \quad |f(x_n) - f(y_n)| \geq \varepsilon_0\quad \forall n
        \end{equation}

        \begin{itemize}
            \item Como $x_n \subseteq A$ acotado $\Longrightarrow \exists x_{\sigma (n)} \to L\in \bb{R}$ por el Teorema de Bolzano-Weierstrass.

            \item Como $y_n \subseteq A$ acotado $\Longrightarrow \exists y_{\sigma' (n)} \to L'\in \bb{R}$ por el Teorema de Bolzano-Weierstrass.
            
            Pasando a la parcial de la parcial, no es restrictivo decir que la parcial es la misma. Por tanto, tenemos:
            \begin{equation*}
                \begin{array}{l}
                    x_{\sigma (n)} \to L\in \bb{R} \\
                    y_{\sigma (n)} \to L'\in \bb{R}
                \end{array}
            \end{equation*}

            Como $\{x_n - y_n\}\to 0$ y toda parcial de una sucesión convergente converge al mismo límite, tenemos que
            $$\{x_n - y_n\}_{\sigma (n)}=\{x_{\sigma (n)} - y_{\sigma (n)}\} \longrightarrow 0 = L-L' \Longrightarrow L=L'$$

            Sea ahora la sucesión $z_n=\left\{ \begin{array}{cc}
                x_{\sigma (n)} & si\;n\;par \\
                y_{\sigma (n)} & si\;n\;impar
            \end{array} \right.$
            
            Es fácil ver que $z_n$ es convergente, ya que $z_n\to L=L'$.
            
            Por tanto, como $z_n\subseteq A\subseteq \bb{R}$, por el Teorema de Complitud de $\bb{R}$ tenemos que ser una sucesión convergente es equivalente a ser de Cauchy. Por tanto, $z_n$ es de Cauchy.

            Veamos ahora si $f(z_n)$ es de Cauchy.
            $$f(z_n)=\left\{ \begin{array}{cc}
                f(x_{\sigma (n)}) & si\;n\;par \\
                f(y_{\sigma (n)}) & si\;n\;impar
            \end{array} \right.$$

            Para que sea de Cauchy, es necesario que
            \begin{equation*}
                \forall \hat{\varepsilon}>0\;\exists n_0 \mid si\; m,n\geq n_0 \Longrightarrow |x_n - x_m| < \hat{\varepsilon}
            \end{equation*}
            es decir, es necesario que a partir de un término $n_0$ en adelante, todos los elementos disten tan poco entre ellos como delimite $\hat{\varepsilon}$. Veamos si esto ocurre para $f(z_n)$.
            \begin{equation*}
                |f(z_{n+1})-f(z_{n})| = |f(x_{\sigma (n)}) - f(y_{\sigma (n)})| \stackrel{(1)}{\geq} \varepsilon_0
            \end{equation*}
            donde en $(1)$ he aplicado la ecuación de hipótesis de inducción, la Ec. \ref{Ej9:Absurdo}, ya que es cierta $\forall n$, por lo que también lo es para la parcial.

            Por tanto, podemos ver que dos términos de la sucesión siempre distan al menos $\varepsilon_0$, por lo que esta sucesión no es de Cauchy.

            Para terminar, vemos que $z_n$ es de Cauchy y $f(z_n)$ no lo es, pero nuestra hipótesis es que $f$ transforma sucesiones de Cauchy en sucesiones de Cauchy, por lo que llegamos a una contradicción. Se demuestra así que nuestra suposición era falsa, y que $f$ sí es uniformemente continua.
        \end{itemize}
    \end{description}

    Lo segundo no es cierto, y ejemplo de ello es $f(x)=x^2$ en $\bb{R}$. Sabemos que transforma Cauchy en Cauchy porque, como $Dom(f),Im(f)\subseteq \bb{R}$ y por el Teorema de Complitud de $\bb{R}$, esto es equivalente a que transforme sucesiones convergentes en sucesiones convergentes. Esto es cierto, ya que la función es continua en $\bb{R}$. Por tanto, hemos encontrado una sucesión que transforma Cauchy en Cauchy, pero que no es uniformemente continua. Esto se debe a que el dominio no está acotado.
\end{ejercicio}

\begin{ejercicio}
    Estudiar la continuidad uniforme de las funciones:
    \begin{enumerate}
        \item $f(x)=e^x$ en $\bb{R}^+_0$.

        Buscamos dos sucesiones $x_n,y_n$ tal que $|x_n-y_n|\to 0$ pero que $|f(x_n) - f(y_n)| = |e^{x_n} - e^{y_n}| \geq \varepsilon_0$, para algún $\varepsilon_0>0$.
        
        Sean esas sucesiones$\{x_n\} = \{\ln (n)\}$ e $\{y_n\} = \{\ln (n+1)\}$.
        \begin{equation*}
            |x_n-y_n| = |\ln n -\ln(n+1)| = \left| \ln\left( \frac{n}{n+1}\right)\right| \longrightarrow \ln 1 = 0
        \end{equation*}
        \begin{equation*}
            |f(x_n) - f(y_n)| = |e^{x_n} - e^{y_n}| = |n - n -1| = 1 = \varepsilon_0
        \end{equation*}

        Por tanto, hemos demostrado que $f$ no es uniformemente continua.
        
        
        \item $g(x)=\sen \left( \frac{1}{x} \right)$ en $]0,1[$.

        Podemos ver que $\nexists \lim_{x\to 0} g(x)$, por lo que no se puede ampliar el dominio de definición. Por el ejercicio \ref{Ej6}, al no poder ampliarse a $[0,1]$, la función $g$ no es uniformemente continua.
    \end{enumerate}
\end{ejercicio}
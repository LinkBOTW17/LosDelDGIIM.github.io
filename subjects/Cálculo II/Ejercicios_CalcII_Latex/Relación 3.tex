\section{Polinomios de Taylor y Concavidad}

\begin{teo*}
    Sean $f,g:I\longrightarrow \bb{R}$ funciones derivables en $a\in I^\circ$ hasta el orden $n$. Entonces,
    \begin{equation*}
        P_{n,a}^{(f+g)} = P_{n,a}^{f}(x) + P_{n,a}^{g}(x)
    \end{equation*}
\end{teo*}

\begin{proof}
    Sabemos que $P_{n,a}^{(f+g)}$ es el único polinomio $Q(x)$ de grado $n$ que pasa por $(a,f(a))$ y tiene la propiedad de $\lim_{x\to a} \frac{(f+g)(x) - Q(x)}{(x-a)^n} = 0$
    \begin{equation*}
        \lim_{x\to a} \frac{(f+g)(x) - (P_{n,a}^{f}(x) + P_{n,a}^{g}(x))}{(x-a)^n} = \lim_{x\to a} \frac{f(x) - P_{n,a}^{f}(x)}{(x-a)^n} + \lim_{x\to a} \frac{g(x) - P_{n,a}^{g}(x)}{(x-a)^n} = 0
    \end{equation*}

    Como conclusión,
    $$Q(x) = P_{n,a}^{f}(x) + P_{n,a}^{g}(x) = P_{n,a}^{(f+g)}(x)$$
\end{proof}



\begin{ejercicio}
    Sea $p(x)=\alpha_0 + \alpha_1x + \dots + \alpha_nx^n$. Dado $a\in \bb{R}$, expresar el polinomio $p(x)$ en potencias de $(x-a)$. Como aplicación, expresar en potencias de $(x-2)$ el polinomio $p(x)=6+7x - 3x^2-5x^3+x^4$.\\

    Usando el polinomio de Taylor,
    \begin{equation*}
        P_{n, a}^p(x) = p(a) + p'(a)(x-a) + \frac{p''(a)}{2}(x-a)^2 + \dots + \frac{p^{n)}(a)}{n!}(x-a)^n
    \end{equation*}

    Como, además, el polinomio de Taylor de un polinomio es dicho polinomio, $ \Longrightarrow P_{n, a}^p(x) = p(x)$. Por tanto,
    \begin{equation*}
        p(x) = p(a) + p'(a)(x-a) + \frac{p''(a)}{2}(x-a)^2 + \dots + \frac{p^{n)}(a)}{n!}(x-a)^n
    \end{equation*}


    Para expresar el polinomio $p(x)=6+7x - 3x^2-5x^3+x^4$ como potencias de $(x-2)$, tenemos que:
    \begin{equation*}
        p(x) = p(2) + p'(2)(x-2) + \frac{p''(2)}{2}(x-2)^2 + \frac{p^{3)}(2)}{6}(x-2)^3 + \frac{p^{4)}(2)}{4!}(x-2)^4
    \end{equation*}

    Como:
    \begin{equation*}
        p'(x) = 7-6x-15x^2+4x^3 \quad
        p''(x) = -5-30x+12x^2
        \quad
        p'''(x) = -30+24x
        \quad
        p''''(x) = 24
    \end{equation*}

    Por tanto,
    \begin{equation*}\begin{split}
        p(x) &= -16 -33(x-2) - \frac{18}{2}(x-2)^2 + \frac{18}{6}(x-2)^3 + \frac{24}{14}(x-2)^4
        =\\&=
        -16 -33(x-2) - 9(x-2)^2 + 3(x-2)^3 + (x-2)^4
    \end{split}\end{equation*}
\end{ejercicio}

\begin{ejercicio}
    Sea $f(x)$ una función cuyo polinomio de Taylor de grado $3$ centrado en el origen es $P_{3,0}^f (x) = 1 + x + \frac{x^2}{2} + \frac{x^3}{3}$. Calcular el polinomio de Taylor de grado $3$ centrado en el origen de la función $g(x) = xf(x)$.\\

    Sabiendo que $P_{n,0}^{x}(x) = x$, tenemos que:
    $$P_{3,0}^{g}(x) = \left[P_{3,0}^{x}(x) P_{3,0}^{f}(x)\right]_{\text{truncado en } n=3} = \left[ x + x^2 + \frac{x^3}{2} + \frac{x^4}{3} \right]_{\text{truncado en } n=3} = x + x^2 + \frac{x^3}{2}$$
\end{ejercicio}

\begin{ejercicio} Si el polinomio de Taylor de grado 3, centrado en el origen, de la función $f(x) $ es $ P_{3,0}^f(x) = 1-x+x^2 - x^3$, calcular el polinomio de Taylor del mismo orden y centro de la función $g(x) = e^{f(x)}$. \\

    En primer lugar, sabemos que:
    \begin{equation*}
        P_{3,0}^f(x) = f(0) + f'(0)x + \frac{f''(0)}{2} x^2 + \frac{f'''(0)}{6}x^3 = 1 - x + x^2 - x^3
    \end{equation*}

    de lo que deducimos:
    \begin{equation*}
        f(0)=1 \qquad f'(0)=-1 \qquad f''(0)=2\cdot 1 = 2 \qquad f'''(0) = -1 \cdot 6 = -6
    \end{equation*}

    \begin{equation*}
        P_{3,0}^g(x) = g(0) + g'(0)x + \frac{g''(0)}{2} x^2 + \frac{g'''(0)}{3!}x^3
    \end{equation*}
    donde:
    \begin{align*}
        g(0)& = e^{f(0)} = e^{P_{3,0}^f(0)} = e^{1} = e\\
        g'(0)& = f'(0)e^{f(0)} = ef'(0) = -e\\
        g''(0)& = f''(0)e^{f(0)} + (f'(0))^2e^{f(0)} = 2e + e = 3e \\
        g'''(0)& = f'''(0)e^{f(0)} + f''(0)f'(0)e^{f(0)} + 2f'(0)f''(0)e^{f(0)} + (f'(0))^3e^{f(0)} = \\
         & \qquad = -6e + -2e -4e -e = -13e
    \end{align*}
    
    Por tanto, la solución queda como:
    \begin{equation*}
        P_{3,0}^g(x) = e -ex + 3e\frac{x^2}{2} - 13e\frac{x^3}{6}
    \end{equation*}
\end{ejercicio}

\begin{ejercicio}
    Calcular el polinomio de Taylor de orden $n$, en el punto $x=0$ (desarrollo de Maclaurin) de las siguientes funciones:
    \begin{enumerate}
        \item $f(x) = e^x$
        \begin{equation*}
            P_{n, 0}^{e^x} = 1 + x + \frac{x^2}{2!} + \frac{x^3}{3!} + \dots + \frac{x^n}{n!} = \sum_{k=0}^n \frac{x^k}{k!}
        \end{equation*}

        \item $f(x) = (1+x)^\alpha \quad (\alpha \in \bb{R})$
        \begin{equation*}
            P_{n, 0}^{(1+x)^\alpha} = 1 + \alpha x + \dots + \frac{\alpha!}{(\alpha-n)!} \frac{x^n}{n!} = \sum_{k=0}^n \left(\begin{array}{c}
                \alpha \\ k \end{array} \right) x^k
        \end{equation*}

        \item $f(x) = \cos(x)$
        \begin{equation*}
            P_{2n, 0}^{\cos(x)} = P_{2n+1, 0}^{\cos(x)} = 1 + 0x -\frac{x^2}{2} + 0 + \frac{x^4}{4!} + \dots + (-1)^n \frac{x^{2n}}{(2n)!}
            = \sum_{k=0}^n (-1)^k \frac{x^{2k}}{(2k)!}
        \end{equation*}

        \item $f(x) = \sen(x)$
        
        Las derivadas del seno son:
        \begin{equation*}
            f^{(4k)}(x) = \sen x
            \quad\;\;
            f^{(4k+1)}(x) = \cos x
            \quad\;\;
            f^{(4k+2)}(x) = -\sen x
            \quad\;\;
            f^{(4k+3)}(x) = -\cos x
        \end{equation*}
        Como $\sen 0 = -\sen 0 = 0$, tenemos que:
        \begin{equation*}
            f^{(2k)}(0) = 0
            \qquad
            f^{(4k+1)}(0) = 1
            \qquad
            f^{(4k+3)}(x) = -1
        \end{equation*}

        Por tanto, de la definición del polinomio de Taylor:        
        \begin{equation*}\begin{split}
            P_{2n, 0}^{\sen(x)} &= \cancelto{0}{f^{(0)}(0)} + \cancelto{1}{ f^{(1)}(0)}x + \cancelto{0}{f^{(2)}}(0)\frac{x^2}{2} + \cancelto{\text{-}1}{f^{(3)}}(0)\frac{x^3}{3!} + \cancelto{0}{f^{(4)}}(0)\frac{x^4}{4!} + \cancelto{1}{f^{(5)}}(0)\frac{x^5}{5!} + \dots \\
            &= x -\frac{x^3}{3!} + \frac{x^5}{5!} - \frac{x^7}{7!}+ \dots + (-1)^{n-1} \frac{x^{2n-1}}{(2n-1)!}
            = \sum_{k=1}^n (-1)^{k-1} \frac{x^{2k-1}}{(2k-1)!}
        \end{split}\end{equation*}

        \item $f(x) = \tan(x)$

        En primer lugar, vemos que la tangente es una función impar. Es decir, que $f(x)=-f(-x)$.
        \begin{equation*}
            f(x) = \frac{\sen x}{\cos x} = -\frac{\sen(-x)}{\cos (-x)} = -f(-x)
        \end{equation*}
        donde he hecho uso de que el seno es una función impar y el coseno es par.
        
        Demostremos ahora mediante inducción que toda derivada de orden par de una función impar es una función impar. Es decir, que dado $f$ impar, se da que:
        \begin{equation*}
            f^{(2k)}(x) = - f^{(2k)}(-x) \qquad \forall k\in \bb{N}
        \end{equation*}
        \begin{itemize}
            \item \underline{Para $k=1$}:\\
            Veamos que $f^{(2)}(x)=f''(x) = -f''(-x)$:
            \begin{multline*}
                f(x) = -f(-x)
                \Longrightarrow \\ \Longrightarrow
                f'(x) = -f'(-x)\cdot (-1) = f'(-x)
                \Longrightarrow \\ \Longrightarrow
                f''(x) = -f''(-x)
            \end{multline*}
            Por tanto, es cierto para $k=1$.

            \item \underline{Supuesto cierto para $k-1$, demostrar para $k$}:\\
            Partiendo de la hipótesis de inducción,
            \begin{multline*}
                f^{(2k-2)}(x) = -f^{(2k-2)}(-x)
                \Longrightarrow \\ \Longrightarrow
                f^{(2k-1)}(x) = -f^{(2k-1)}(-x)\cdot (-1) = f^{(2k-1)}(-x)
                \Longrightarrow \\ \Longrightarrow
                f^{(2k)}(x) = -f^{(2k)}(-x)
            \end{multline*}
        \end{itemize}
        Por tanto, ya hemos visto que $f^{(2k)}(x) = - f^{(2k)}(-x)$ . Obtenemos de manera directa lo siguiente:
        \begin{equation*}
            f^{(2k)}(x) = - f^{(2k)}(-x) \Longrightarrow  f^{(2k)}(0) = - f^{(2k)}(0) \Longrightarrow f^{(2k)}(0) = 0 \qquad \forall k\in \bb{N}
        \end{equation*}

        Haciendo uso de la definición de Polinomio de Taylor, tenemos que el polinomio de Taylor de la tangente queda como:
        \begin{equation*}\begin{split}
            P_{n,0}^f(x) &= f(0) + f'(0)x + f''(0)\frac{x^2}{2} + \dots \\
            &= a_0 + a_1x + \dots + a_nx^n = \sum_{i=0}^n a_ix^i \quad a_i\in \bb{R}
        \end{split}\end{equation*}

        El objetivo es calcular los valores de $a_i$. Haciendo uso de que $f^{(2k)}(0) = 0$, sabemos que $a_{2k}=0\;\forall k$. Por tanto, queda como:
        \begin{equation*}\begin{split}
            P_{2n+1,0}^f(x) &=  f'(0)x + f'''(0)\frac{x^3}{3} + \dots \\
            &= a_1x + a_3x^3 + a_5x^5 + \dots + a_{2n+1}x^{2n+1} = \sum_{k=0}^n a_{2k+1}x^{2k+1}
        \end{split}\end{equation*}

        Haciendo uso del Resto de Lagrange:
        \begin{equation*}
            f(x) = P_{2n+1,0}^f(x) + R_{2n+1,0}(x) = a_1x + a_3x^3 + a_5x^5 + \dots + a_{2n+1}x^{2n+1} + R_{2n+1,0}(x)
        \end{equation*}

        Por tanto,
        \begin{equation*}\begin{split}
            f'(x) &= 1+f^2(x)\\
            &= a_1 + 3a_3x^2 + 5a_5x^4 + \dots + (2n+1)a_{2n+1}x^{2n} + R'_{2n+1,0}(x)
        \end{split}\end{equation*}

        Evaluando la derivada en $x=0$, obtenemos que $$f'(0)=1 = a_1 +\cancel{R'_{2n+1,0}(0)} \Longrightarrow a_1=1$$

        Por tanto, hasta ahora tenemos que $a_1=1$ y que $a_{2k}=0\;\forall k\in \bb{N}\cup \{0\}$.
        
        Para calcular el resto de coeficientes $a_i$, calculamos $P_{2n,0}^{f'}(x)$ de dos formas distintas. Por un lado, tenemos que:
        \begin{equation}\label{Ec:Ej4.5.Derivada1}
        \begin{split}
            P_{2n,0}^{f'}(x) &= \left[P_{2n+1,0}^f(x)\right]' \\
            &= a_1 + 3a_3x^2 + 5a_5x^4 + \dots + (2n+1)a_{2n+1}x^{2n} \\
            &= 1 + 3a_3x^2 + 5a_5x^4 + \dots + (2n+1)a_{2n+1}x^{2n}
        \end{split}\end{equation}

        Por el otro lado, como $f'(x)=1+f^2(x)$, entonces:
        \begin{equation}\label{Ec:Ej4.5.Derivada2}
            P_{2n,0}^{f'}(x) =
            P_{2n,0}^{1+f^2}(x) = P_{2n,0}^{1}(x) + P_{2n,0}^{f^2}(x) = 1+P_{2n,0}^{f^2}(x)
        \end{equation}

        Calculamos por tanto $P_{2n,0}^{f^2}(x)$:
        \begin{equation*}\begin{split}
            P_{2n,0}^{f^2}
            & = \left[\left(P_{2n}^f(x)\right)^2\right]_{\text{truncado en } 2n} \\
            &= \left[\left(a_1x + a_3x^3 + a_5x^5 + \dots + a_{2n-1}x^{2n-1}\right)^2\right]_{\text{truncado en } 2n}\\
            & = \beta_2x^2 + \beta_3x^3 + \dots + \beta_{2n}x^{2n}          
        \end{split}\end{equation*}
        donde $\displaystyle \beta_k = \sum_{\substack{
            i,j\in \bb{N} \\
            i+j=k}} a_ia_j$.
        Por tanto, como $a_1=1$ y $a_{2k}=0\;\forall k$, obtenemos:
        \begin{equation*}
            \beta_2 = a_1^2 = 1
            \qquad \beta_3 = a_1a_2 + a_2a_1 = 0
            \qquad \beta_4 = a_1a_3 + a_2^2 + a_3a_1 = 2a_1a_3
        \end{equation*}
        \begin{equation*}
            \beta_5 = 2a_1a_4 + 2a_2a_3 = 0 \qquad \beta_6 = 2a_1a_5 + 2a_2a_4 + a_3^2 = 2a_1a_5 + a_3^2
        \end{equation*}

        Igualando las dos expresiones de $P_{2n}^{f'}(x)$ (ecuaciones \ref{Ec:Ej4.5.Derivada1} y \ref{Ec:Ej4.5.Derivada2}), tenemos que:
        \begin{equation*}
            \cancel{1} + 3a_3x^2 + 5a_5x^4 + \dots + (2n+1)a_{2n+1}x^{2n} = \cancel{1} + \beta_2x^2 + \beta_3x^3 + \dots + \beta_{2n}x^{2n}
        \end{equation*}
        Igualando los términos del mismo grado, obtengo que $\beta_{2k+1}=0\;\forall k\in \bb{N}\cup \{0\}$. Además, podemos calcular cada coeficiente $a_i$.
        \begin{equation*}
            3a_3 = \beta_2 = 1 \Longrightarrow a_3=\frac{1}{3}
            \qquad
            5a_5 = \beta_4 = 2a_1a_3 = \frac{2}{3} \Longrightarrow a_5 = \frac{2}{15}
        \end{equation*}
        \begin{equation*}
            7a_7 = \beta_6 = 2a_1a_5 + a_3^2 = \frac{4}{15} + \frac{1}{9} = \frac{17}{45} \Longrightarrow a_7 = \frac{17}{315}
        \end{equation*}
        De forma general, obtenemos cada coeficiente $a_k$ se puede obtener sabiendo los anteriores de la siguiente forma:
        \begin{equation*}
            a_k = \frac{\beta_{k-1}}{k} = \frac{\displaystyle \sum_{\substack{
            i,j\in \bb{N} \\
            i+j=k-1}} a_ia_j}{k}
        \end{equation*}

        Como conclusión, obtenemos que:
        \begin{equation*}\begin{split}
            P_{2n+1,0}^f(x) &= x + a_3x^3 + a_5x^5 + \dots + a_{2n+1}x^{2n+1} = \sum_{k=0}^n a_{2k+1}x^{2k+1} \\
            &= 1 + \frac{1}{3}x^3 + \frac{2}{15}x^5 + \frac{17}{315}x_7 + \dots
        \end{split}\end{equation*}
        
        
        

        \item $f(x) = \arcsen(x)$\\
        Para calcular su desarrollo de Taylor, hago uso de que:
        \begin{equation*}
            \left[ P_{n+1, a}^f (x) \right]' =P_{n,a}^{f'} (x)\Longrightarrow P_{n+1, a}^f (x) = \int P_{n,a}^{f'}(x) dx
        \end{equation*}

        Sabemos que $f'(x) = \frac{1}{\sqrt{1-x^2}} = (1-x^2)^{-\frac{1}{2}}$. Además, como $f'(x)$ está definida en $]-1,1[$, tenemos que $|x|<1$.
        \begin{equation*}
            P_{n, 0}^{(1-x)^\alpha} = 1 - \alpha x + \dots + (-1)^n\frac{\alpha!}{(\alpha-n)!} \frac{x^n}{n!} = \sum_{k=0}^n \left(\begin{array}{c}
                \alpha \\ k \end{array} \right) (-1)^k x^k
        \end{equation*}
        \begin{equation*}
            P_{2n,0}^{f'}(x) =
            \sum_{k=0}^n \left(\begin{array}{c}
                -\frac{1}{2} \\ k \end{array} \right) (-1)^k x^{2k}
        \end{equation*}

        Por tanto,
        \begin{equation*}
            P_{2n+1, 0}^f (x) = \int P_{2n,0}^{f'}(x) dx = \int \sum_{k=0}^n \left(\begin{array}{c}
            -\frac{1}{2} \\ k \end{array} \right) (-1)^k x^{2k}
            = \sum_{k=0}^n \left(\begin{array}{c}
            -\frac{1}{2} \\ k \end{array} \right) (-1)^k \frac{x^{2k+1}}{2k+1}
        \end{equation*}

        \item $f(x) = \ln(1+x)$
        \begin{equation*}
            P_{n, 0}^{\ln(1+x)} = 0 + x -\frac{x^2}{2!} + 2! \frac{x^3}{3!} + \dots + (-1)^{n-1} \frac{(n-1)!}{n!}x^n
            = \sum_{k=1}^n (-1)^{k-1} \frac{x^k}{k}
        \end{equation*}

        \item $f(x) = \arctg(x)$
        Para calcular su desarrollo de Taylor, hago uso de que:
        \begin{equation*}
            \left[ P_{n+1, a}^f (x) \right]' =P_{n,a}^{f'} (x)\Longrightarrow P_{n+1, a}^f (x) = \int P_{n,a}^{f'}(x) dx
        \end{equation*}

        Sabemos que $f'(x) = \frac{1}{1+x^2} = (1+x^2)^{-1}$.
        \begin{equation*}
            P_{n, 0}^{(1+x)^\alpha} = \sum_{k=0}^n \left(\begin{array}{c}
                \alpha \\ k \end{array} \right) x^k
        \end{equation*}
        \begin{equation*}
            P_{n, 0}^{(1+x)^{-1}} = \sum_{k=0}^n \left(\begin{array}{c}
                -1 \\ k \end{array} \right) x^k = (-1)^k x^k
        \end{equation*}
        donde esto último se da, ya que:
        \begin{equation*}
            \left(\begin{array}{c}
                -1 \\ k \end{array} \right)
                := \frac{(-1)(-2) \dots (-k)}{k!} = (-1)^k \frac{k!}{k!} = (-1)^k
        \end{equation*}

        Por tanto,
        \begin{equation*}
            P_{2n,0}^{f'}(x) =
            \sum_{k=0}^n (-1)^k x^{2k}
        \end{equation*}

        Por tanto, el desarrollo pedido es:
        \begin{equation*}
            P_{2n+1, 0}^f (x)
            = \int P_{2n,0}^{f'}(x) dx
            = \int \sum_{k=0}^n (-1)^k x^{2k}
            = \sum_{k=0}^n (-1)^k \frac{x^{2k+1}}{2k+1}
        \end{equation*}
    \end{enumerate}
\end{ejercicio}

\begin{ejercicio}
    Sean $f,g:I\to \bb{R}$ funciones $(n+1)$ veces derivables en el punto $a\in I$. Sea $h(x)=f(x)g(x)$ para cada $x\in I$. Demostrar que $P_{n,a}^h(x)$ se obtiene del polinomio producto $P_{n,a}^f(x)P_{n,a}^g(x)$ eliminando los términos de orden mayor estricto que $n$.

    Se pide demostrar:
    $$P_{n,a}^{fg}(x) = \left[P_{n,a}^{f}(x) P_{n,a}^{g}(x) \right]_{\text{truncado en } n}$$

    \begin{proof}
        \begin{multline*}
            \lim_{x\to a} \frac{f(x)g(x) - P_{n,a}^{f}(x) P_{n,a}^{g}(x)}{(x-a)^n}
            =\\
            = \lim_{x\to a} \frac{f(x)g(x) - P_{n,a}^{f}(x)g(x)  + P_{n,a}^{f}(x)g(x) - P_{n,a}^{f}(x)P_{n,a}^{g}(x)}{(x-a)^n}
            =\\=
            \lim_{x\to a}\frac{f(x) - P_{n,a}^f(x)}{(x-a)^n}g(x) + \lim_{x\to a}P_{n,a}^f(x) \frac{g(x) - P_{n,a}^g(x)}{(x-a)^n} = 0
        \end{multline*}
    \end{proof}

    Por tanto, multiplicando ambos 
\end{ejercicio}

\begin{ejercicio}
    Sea $f:I\to \bb{R}$ una función $(n+1)$ veces derivable en el punto $x=0$ (que es interior a $I$). Sea $k\in \bb{N}$ y $h(x) = f(x^k)$. Demostrar que $P_{nk, 0}^h(x) = P_{n,0}^f(x^k)$.

    \begin{proof}
        \begin{equation*}
            \lim_{x\to 0} \frac{f(x) - P_{n,0}^f(x)}{x^n} = 0 \Longrightarrow
            \lim_{x\to 0} \frac{f(x^k) - P_{n,0}^f(x^k)}{(x^k)^n} = 0
        \end{equation*}
        Esto se da ya que $\{x\}\longrightarrow 0 \Longleftrightarrow \{x^k\} \longrightarrow 0 \qquad \forall k \in \bb{N}$
    \end{proof}
    
    \begin{ejemplo} Aplicaciones del último ejercicio son:
        \begin{equation*}
            P_{3,0}^{e^x} (x) = 1 + x + \frac{x^2}{2!} + \frac{x^3}{3!}
        \end{equation*}
        \begin{equation*}
            P_{6,0}^{e^{x^2}} (x) = 1 + x^2 + \frac{x^4}{2!} + \frac{x^6}{3!}
        \end{equation*}
    \end{ejemplo}
\end{ejercicio}

\begin{ejercicio}
    Calcular los siguientes límites (utilizando el desarrollo de Taylor):
    \begin{enumerate}
        \item $\displaystyle \lim_{x\to 0} \frac{1}{x^4} \left(2x\sqrt[3]{1+x^3} + 2\sqrt{1+x^2}-2-2x-x^2\right)
        = \lim_{x\to 0} \cancelto{0}{\frac{f(x) - P_{4,0}^f(x)}{x^4}} + \frac{P_{4,0}^f(x)}{x^4}
        = \lim_{x\to 0} \frac{P_{4,0}^f(x)}{x^4} \stackrel{Ec.\;\ref{Ej.7_Taylor_1}}{=} \frac{5}{12}$

        siendo $f(x) = 2x(1+x^3)^\frac{1}{3} + 2(1+x^2)^{\frac{1}{2}} - 2- 2x - x^2$.

        \begin{equation*}
            P_{4,0}^{(1+x^3)^\frac{1}{3}}(x) = 1 + 2\frac{x^3}{3!} = 1 + \frac{x^3}{3}
        \end{equation*}

        \begin{equation*}
            P_{4,0}^{(1+x^2)^\frac{1}{2}}(x) = 1 + \frac{x^2}{2!} -3\frac{x^4}{4!}= 1 + \frac{x^2}{2} - \frac{x^4}{8}
        \end{equation*}

        \begin{multline}\label{Ej.7_Taylor_1}
            P_{4,0}^f(x) = 2\left( x + \frac{x^4}{3}\right) + 2\left( 1 + \frac{x^2}{2} - \frac{x^4}{8}\right) - 2-2x-x^2 =
            \\= 2x + \frac{2}{3}x^4 + 2 + x^2 - \frac{x^4}{4} - 2-2x-x^2= \frac{5}{12}x^4
        \end{multline}

        \item $\displaystyle \lim_{x\to 0} \frac{(\tan x - \sen x)\sen x - \frac{x^4}{2}}{x^6} = \lim_{x\to 0} \frac{P_{6,0}^f(x)}{x^6} \stackrel{Ec.\;\ref{Ec:Ej7.2}}{=} \frac{1}{24}$
        
        siendo $f(x) = (\tan x - \sen x)\sen x - \frac{x^4}{2}$.

        \begin{equation*}
            P_{6,0}^{\tan x}(x) = x + \frac{1}{3}x^3 + \frac{2}{15}x^5
            \qquad \qquad
            P_{6,0}^{\sen x}(x) = x - \frac{x^3}{3!} + \frac{x^5}{5!}
        \end{equation*}
        
        Por tanto,
        \begin{equation}\label{Ec:Ej7.2}
        \begin{split}
            P_{6,0}^{f}(x) &= \left[\left(\frac{1}{2}x^3 + \frac{1}{8}x^5\right)\left(x - \frac{x^3}{3!} + \frac{x^5}{5!}\right) \right]_{n=6} - \frac{1}{2}x^4 \\&= \cancel{\frac{1}{2}x^4} - \frac{1}{12}x^6 + \frac{1}{8}x^6 - \cancel{\frac{1}{2}x^4} = \frac{x^6}{24}
        \end{split}\end{equation}

        \item $\displaystyle \lim_{x\to 0} \frac{\ln(1+x)\sen x - x^2+x^3}{x^3} = \lim_{x\to 0} \frac{P_{3,0}^f(x)}{x^3} \stackrel{Ec.\;\ref{Ej.7_Taylor_3}}{=}\frac{1}{2}$
        
        siendo $f(x) = \ln(1+x)\sen x - x^2+x^3$.
        \begin{equation*}
            P_{3,0}^{\ln (1+x)}(x) = x - \frac{x^2}{2} + \frac{x^3}{3}
        \end{equation*}
        \begin{equation*}
            P_{3,0}^{\sen x}(x) = x - \frac{x^3}{3!}
        \end{equation*}
        \begin{multline}\label{Ej.7_Taylor_3}
            P_{3,0}^{f}(x) = \left[\left( x - \frac{x^2}{2} + \frac{x^3}{3} \right)\left(x - \frac{x^3}{3!} \right) - x^2 + x^3\right]_{n=3}
            = x^2 - \frac{x^3}{2} - x^2 + x^3 = \frac{x^3}{2}
        \end{multline}

        \item $\displaystyle \lim_{x\to 0} \frac{\ln^2(1+x)-\sen^2(x)}{1-e^{-x^2}}$

        Sea $f(x) = \ln^2(1+x)-\sen^2(x)$ y $g(x) = 1-e^{-x^2}$. Entonces:
        \begin{equation*}
            \lim_{x\to 0} \frac{f(x)}{g(x)} = \lim_{x\to 0} \frac{\frac{f(x)}{x^2}}{\frac{g(x)}{x^2}}
            = \lim_{x\to 0} \frac{\frac{P_{2, 0}^f(x)}{x^2}}{\frac{P_{2, 0}^g(x)}{x^2}}
            = \lim_{x\to 0} \frac{\frac{0}{x^2}}{\frac{x^2}{x^2}}
            = \lim_{x\to 0} \frac{0}{1} = 0
        \end{equation*}

        donde he usado que:
        \begin{equation*}
            P_{2, 0}^f(x) = \left[\left(x-\frac{x^2}{2}\right)^2 - x^2 \right]_{\text{truncado en } n=2} = 0
        \end{equation*}
        \begin{equation*}
            P_{2, 0}^g(x) = \left[1-\left( 1 - x^2 + \frac{x^4}{2!} \right) \right]_{\text{truncado en } n=2} = x^2
        \end{equation*}

        \item $\displaystyle \lim_{x\to 0} \frac{e^x - \sum_{k=0}^n \frac{x^k}{k!}}{x^n} = \lim_{x\to 0} \frac{P_{n,0}^f(x)}{x^n} \stackrel{Ec.\;\ref{Ej.7_Taylor_5}}{=} \lim_{x\to 0} \frac{0}{x^n} = 0$
        
        siendo $f(x) = e^x - \sum_{k=0}^n \frac{x^k}{k!}$.
        \begin{equation} \label{Ej.7_Taylor_5}
            P_{n,0}^f(x) = P_{n,0}^{e^x}(x) - \sum_{k=0}^n \frac{x^k}{k!} = 0
        \end{equation}

        \item $\displaystyle \lim_{x\to 0} \frac{e^x - \sum_{k=0}^{n-1} \frac{x^k}{k!}}{x^n} = \lim_{x\to 0} \frac{P_{n,0}^f(x)}{x^n} \stackrel{Ec.\;\ref{Ej.7_Taylor_6}}{=} \lim_{x\to 0} \frac{\frac{x^n}{n!}}{x^n} = \frac{1}{n!}$
        
        siendo $f(x) = e^x - \sum_{k=0}^{n-1} \frac{x^k}{k!}$.
        \begin{equation} \label{Ej.7_Taylor_6}
            P_{n,0}^f(x) = P_{n,0}^{e^x}(x) - \sum_{k=0}^{n-1} \frac{x^k}{k!} = \sum_{k=0}^{n} \frac{x^k}{k!} - \sum_{k=0}^{n-1} \frac{x^k}{k!} = \frac{x^n}{n!}
        \end{equation}

        \item $\displaystyle \lim_{x\to 0} \frac{e^x-1-x-\frac{x^2}{2}-\frac{x^3}{6}}{x^4} = \lim_{x\to 0} \frac{P_{4,0}^f(x)}{x^4} \stackrel{Ec.\;\ref{Ej.7_Taylor_7}}{=} \frac{1}{4!}$
        
        siendo $f(x) = e^x-1-x-\frac{x^2}{2}-\frac{x^3}{6}$.
        \begin{equation}\label{Ej.7_Taylor_7}
            P_{4,0}^{f}(x) = 1 + x + \frac{x^2}{2!} + \frac{x^3}{3!} + \frac{x^4}{4!} -1 - x - \frac{x^2}{2} - \frac{x^3}{6} = \frac{x^4}{4!}
        \end{equation}

        \item $\displaystyle \lim_{x\to 0} \frac{e^x- \sen x}{e^x - 1 -x-x^2}$\\
        
        Sea $f(x) = e^x- \sen x$ y $g(x) = e^x - 1 -x-x^2$. Entonces:
        \begin{equation*}
            \lim_{x\to 0} \frac{f(x)}{g(x)} = \lim_{x\to 0} \frac{\frac{f(x)}{x^2}}{\frac{g(x)}{x^2}}
            = \lim_{x\to 0} \frac{\frac{P_{2, 0}^f(x)}{x^2}}{\frac{P_{2, 0}^g(x)}{x^2}}
            = \lim_{x\to 0} \frac{\frac{1+\frac{x^2}{2}}{x^2}}{\frac{-\frac{x^2}{2}}{x^2}}
            = \lim_{x\to 0} \frac{1+\frac{x^2}{2}}{-\frac{x^2}{2}} = \frac{1}{0^-} = -\infty
        \end{equation*}

        donde he usado que:
        \begin{equation*}
            P_{2, 0}^f(x) = \left[1+x+\frac{x^2}{2} - x \right]_{\text{truncado en } n=2} = 1+\frac{x^2}{2}
        \end{equation*}
        \begin{equation*}
            P_{2, 0}^g(x) = \left[1+x+x\frac{x^2}{2} - 1 - x - x^2 \right]_{\text{truncado en } n=2} = -\frac{x^2}{2}
        \end{equation*}
        
    \end{enumerate}
\end{ejercicio}

\begin{ejercicio}
    Estudiar el comportamiento en $0$ y en $\pm \infty$ de la función $f:\bb{R}^\ast \longrightarrow \bb{R}$ dada por $f(x) = \frac{x-\sen x}{x^6} \left( e^x - 1 - x - \frac{x^2}{2} \right)$, para cada $x \in \bb{R}^\ast$

    \begin{equation*}\begin{split}
        \lim_{x\to 0}f(x) & = \lim_{x\to 0} \frac{P_{6, 0}}{x^6} =\\
        & = \lim_{x\to 0} \frac{\left[ \left(x - x + \frac{x^3}{3!} - \frac{x^5}{5!}\right)\left( 1 + x + \frac{x^2}{2} + \frac{x^3}{3!} + \frac{x^4}{4!} + \frac{x^5}{5!} + \frac{x^6}{6!} -1 -x - \frac{x^2}{2} \right) \right]_{n=6}}{x^6} = \\
        & = \lim_{x\to 0} \frac{\left[ \left(\frac{x^3}{3!} - \frac{x^5}{5!}\right)\left( \frac{x^3}{3!} + \frac{x^4}{4!} + \frac{x^5}{5!} + \frac{x^6}{6!}\right) \right]_{n=6}}{x^6} = \lim_{x\to 0} \frac{\frac{x^6}{3! \cdot 3!}}{x^6} = \frac{1}{3! \cdot 3!} = \frac{1}{36}
    \end{split}\end{equation*}
    \begin{multline*}
        \lim_{x\to \infty} f(x)
        = \lim_{x\to \infty} \frac{x-\sen x}{x^6} \left( e^x - 1 - x - \frac{x^2}{2} \right)
        = \lim_{x\to \infty} \frac{x-\sen x}{x} \left( \frac{e^x - 1 - x - \frac{x^2}{2}}{x^5} \right)
        =\\= \;\stackrel{L'H\hat{o}pital}{\dots}\;=  1\cdot \infty = \infty
    \end{multline*}
    \begin{multline*}
        \lim_{x\to -\infty} f(x)
        = \lim_{x\to -\infty} \frac{x-\sen x}{x^6} \left( e^x - 1 - x - \frac{x^2}{2} \right)
        = \lim_{x\to -\infty} \frac{x-\sen x}{x} \left( \frac{e^x - 1 - x - \frac{x^2}{2}}{x^5} \right)
        =\\= 1\cdot 0=0
    \end{multline*}
\end{ejercicio}

\begin{ejercicio}
    Probar que $\displaystyle 1 - \frac{x^2}{2} \leq \cos x \leq 1 - \frac{x^2}{2} + \frac{x^4}{24}$ para todo $x \in [0, \pi]$\\

    Demostramos haciendo uso del resto de Taylor. Sabemos que:
    \begin{equation}\label{Ec.RestoTaylor}
        \cos x = P_{n, 0}^{\cos x} (x) + R_{n, 0}^{\cos x} (x)
    \end{equation}
    
    Demuestro en primer lugar la primera desigualdad. La ecuación \ref{Ec.RestoTaylor} para $n=2$ queda como:
    \begin{equation*}
        \cos x = P_{2, 0}^{\cos x} (x) + R_{2, 0}^{\cos x} (x) = 1 - \frac{x^2}{2} + \frac{\cos'''(c)}{3!}x^3 = 1 - \frac{x^2}{2} + \frac{\sen(c)}{3!}x^3 \quad \text{para algún } c \in [0, x]
    \end{equation*}

    Por tanto,
    \begin{equation*}
        \cos x - \left( 1-\frac{x^2}{2}\right) = \cancel{1 - \frac{x^2}{2}} + \frac{\sen(c)}{3!}x^3 - \cancel{\left( 1-\frac{x^2}{2}\right)} = \frac{\sen(c)}{3!}x^3 \geq 0
    \end{equation*}
    ya que $c\in  [0,x] \subseteq [0,\pi]$, y el seno en dicho intervalo es $\geq 0$. Además, como $x\in [0, \pi] \subset \bb{R}^+_0$, también es $\geq 0$. Por tanto, como es $\geq 0$, se demuestra la primera desigualdad.

    Demuestro ahora la segunda desigualdad. La ecuación \ref{Ec.RestoTaylor} para $n=4$ queda como:
    \begin{multline*}
        \cos x = P_{4, 0}^{\cos x} (x) + R_{4, 0}^{\cos x} (x) = 1 - \frac{x^2}{2} + \frac{x^4}{4!} + \frac{\cos(x)^{5)}(c)}{5!}x^5
        =\\=
        1 - \frac{x^2}{2} + \frac{x^4}{4!} - \frac{\sen(c)}{5!}x^5 \quad \text{para algún } c \in [0, x]
    \end{multline*}

    Por tanto,
    \begin{equation*}
        \cos x - \left( 1-\frac{x^2}{2} + \frac{x^4}{24}\right) = \cancel{1 - \frac{x^2}{2} + \frac{x^4}{4!}} - \frac{\sen(c)}{5!}x^5 - \cancel{\left( 1-\frac{x^2}{2} + \frac{x^4}{4!}\right)} = -\frac{\sen(c)}{5!}x^5 \leq 0
    \end{equation*}
    ya que $c\in  [0,x] \subseteq [0,\pi]$, y el seno en dicho intervalo es $\geq 0$. Además, como $x\in [0, \pi] \subset \bb{R}^+_0$, no cambia el signo. Por tanto, como es $\leq 0$, se demuestra la segunda desigualdad.
    
    
\end{ejercicio}

\begin{ejercicio}
    Calcular el polinomio de Taylor de orden $8$ en el punto $x=0$ de la función $\ln(1+x^4)$.\\

    Obtengo en primer lugar el polinomio de Taylor de orden 2 en el punto $x=0$ de la función $f(x)=\ln(1+x)$.
    \begin{equation*}
        P_{2, 0}^{f}(x) = x - \frac{x^2}{2}
    \end{equation*}

    Por tanto, sabiendo que la función pedida es $f(x^4) = \ln(1+x^4)$
    \begin{equation*}
        P_{8, 0}^{\ln(1+x^4)}(x) = P_{2\cdot 4, 0}^{f}(x^4) = x^4 - \frac{x^{2\cdot 4}}{2} = x^4 - \frac{x^8}{2}
    \end{equation*}
\end{ejercicio}

\begin{ejercicio}
    Calcular un valor aproximado, con un error menor que $10^{-2}$ de los siguientes números reales:
    \begin{enumerate}
        \item $\sqrt[3]{7}$\\
        Vamos a aproximar el valor mediante el desarrollo de Taylor de la exponencial de base 7.
        $$\sqrt[3]{7}\approx P_{n, 0}^{7^x}\left(\frac{1}{3}\right)$$

        Veamos el valor de la derivada n-ésima de la exponencial de base 7:
        \begin{equation*}
            7^x = e^{x\ln(7)} \Longrightarrow (7^x)' = 7^x \ln(7)
        \end{equation*}
        Por inducción se demuestra fácilmente que $\frac{d^n}{dx^n} 7^x = (7^x)^{n)}=  7^x \ln^n(7)$

        Para saber de qué orden debe ser la aproximación, establecemos la condición de que el error debe ser menor que $10^{-2}$. El error viene dado por:
        \begin{equation*}
            R_{n,0}^{7^x}\left(x\right) = \frac{(7^x)^{n+1)}(c)}{(n+1)!}x^{n+1} = \frac{7^c \cdot \ln^{n+1} (7)}{(n+1)!}x^{n+1} \quad \text{para algún } c\in ]0, x[
        \end{equation*}

        Como estamos evaluando en $x=\frac{1}{3}$, y sabiendo que la exponencial de base 7 es estrictamente creciente, tenemos el siguiente resultado:
        \begin{equation*}
            0 < c < \frac{1}{3} \Longrightarrow 1 = 7^0 < 7^c < 7^{1/3} < 8^{1/3} = 2  \Longrightarrow 7^c < 2
        \end{equation*}

        Como nos piden que el error sea menor a $10^{-2}$, la inecuación a resolver es:
        \begin{equation*}
            R_{n,0}^{7^x}\left(\frac{1}{3}\right) = \frac{7^c}{(n+1)!}\frac{1}{3^{n+1}} < 10^{-2}
        \end{equation*}
        Usando la condición de que $7^c < 2$:
        \begin{equation*}
            R_{n,0}^{7^x}\left(\frac{1}{3}\right) = \frac{7^c \cdot \ln^{n+1} (7)}{(n+1)!}\frac{1}{3^{n+1}} < \frac{2 \cdot \ln^{n+1} (7)}{(n+1)!}\frac{1}{3^{n+1}}  < 10^{-2}
        \end{equation*}
        Por tanto, resuelvo la siguiente inecuación para hallar el valor de $n$:
        \begin{equation*}
            \frac{2\cdot \ln^{n+1} (7)}{(n+1)!}\frac{1}{3^{n+1}}  < 10^{-2} \Longrightarrow 200 < \frac{(n+1)! \cdot 3^{n+1}}{\ln^{n+1} (7)}
        \end{equation*}

        Como $n\in \bb{N}$, podemos ver que $n=4$ satisface la inecuación. Para valores mayores de $n$ la aproximación será mejor, pero para obtener error menor al pedido no es necesario más.
        
        Por tanto, con $n=4$, el resultado queda:
        $$\sqrt[3]{7}\approx P_{4, 0}^{7^x}\left(\frac{1}{3}\right) = 1 + \frac{\ln 7}{3} + \frac{\ln^2 7}{3^2 \cdot 2!} + \frac{\ln^3 7}{3^3 \cdot 3!} + \frac{\ln^4 7}{3^4 \cdot 4!}$$

        
        \item $\sen\left( \frac{1}{2} \right)$\\
        Vamos a aproximar el valor mediante el desarrollo de Taylor del seno.
        $$\sen\left( \frac{1}{2} \right) \approx P_{n, 0}^{\sen x}\left(\frac{1}{2}\right)$$

        El error viene dado por:
        \begin{equation*}
            R_{n,0}^{\sen x}\left(x\right) = \frac{\sen^{n+1)} (c)}{(n+1)!}x^{n+1} \quad \text{para algún } c\in ]0, x[
        \end{equation*}
        Como estamos evaluando en $x=\frac{1}{2}$, y sabiendo que el seno y el coseno están acotados entre $-1$ y $1$, tenemos el siguiente resultado:
        \begin{equation*}
            -1 < \sen^{n+1)} (c) < 1
        \end{equation*}

        Como nos piden que el error sea menor a $10^{-2}$, la inecuación a resolver es:
        \begin{equation*}
            R_{n,0}^{\sen x}\left(\frac{1}{2}\right) = \frac{\sen^{n+1)} (c)}{(n+1)!}\frac{1}{2^{n+1}} < 10^{-2}
        \end{equation*}
        Usando la condición de que $\sen c < 1$:
        \begin{equation*}
            R_{n,0}^{\sen x}\left(\frac{1}{2}\right) = \frac{\sen^{n+1)} (c)}{(n+1)!}\frac{1}{2^{n+1}} < \frac{1}{(n+1)!}\frac{1}{2^{n+1}}  < 10^{-2}
        \end{equation*}
        Por tanto, resuelvo la siguiente inecuación para hallar el valor de $n$:
        \begin{equation*}
            \frac{1}{(n+1)!}\frac{1}{2^{n+1}}  < 10^{-2} \Longrightarrow \frac{1}{(n+1)!2^{n+1}} < \frac{1}{100} \Longrightarrow 100 < (n+1)!2^{n+1}
        \end{equation*}
        Como $n\in \bb{N}$, podemos ver que $n=3$ satisface la inecuación. Para valores mayores de $n$ la aproximación será mejor, pero para obtener error menor al pedido no es necesario más. Por tanto, con $n=3$, el resultado queda:
        $$\sen\left( \frac{1}{2} \right) \approx P_{3, 0}^{\sen x}\left(\frac{1}{2}\right) = \frac{1}{2} - \frac{1}{2^3 \cdot 3!} = \frac{23}{48}$$
        
        
        \item $\ln 3$

        Vamos a aproximar el valor mediante el desarrollo de Taylor del $\ln(x)$ centrado en $a=e$.
        $$\ln 3\approx P_{n, e}^{\ln}\left(3\right)$$

        Veamos el valor de la derivada n-ésima del $\ln x$. Por inducción se demuestra fácilmente que: $$f^{(n)}(x) = (-1)^{n+1} \frac{(n-1)!}{x^n}$$

        Para saber de qué orden debe ser la aproximación, establecemos la condición de que el error debe ser menor que $10^{-2}$. El error viene dado por:
        \begin{equation*}
            R_{n,e}^{\ln x}\left(x\right) = \frac{(-1)^n \frac{n!}{c^{n+1}}}{(n+1)!}(x-e)^{n+1} = \frac{(-1)^n}{n+1}\cdot \left(\frac{x-e}{c} \right)^{n+1}
            \quad \text{para algún } c\in ]e, x[
        \end{equation*}

        Por tanto, como $e<c<3$
        \begin{equation*}
            R_{n,e}^{\ln x}\left(3\right)
            = \frac{(-1)^n}{n+1}\cdot \left(\frac{3-e}{c} \right)^{n+1}
            < \frac{1}{n+1}\cdot \left(\frac{3-e}{e} \right)^{n+1}
        \end{equation*}

        Como nos piden que el error sea menor a $10^{-2}$, la inecuación a resolver es:
        \begin{equation*}
            R_{n,e}^{\ln x}\left(3\right)
            = \frac{(-1)^n}{n+1}\cdot \left(\frac{3-e}{c} \right)^{n+1}
            < \frac{1}{n+1}\cdot \left(\frac{3-e}{e} \right)^{n+1}
            < 10^{-2}
        \end{equation*}
        
        Por tanto, resuelvo la siguiente inecuación para hallar el valor de $n$:
        \begin{equation*}
            \frac{1}{n+1}\cdot \left(\frac{3-e}{e} \right)^{n+1}
            < 10^{-2}
        \end{equation*}

        Como $n\in \bb{N}$, podemos ver que $n=1$ satisface la inecuación. Para valores mayores de $n$ la aproximación será mejor, pero para obtener error menor al pedido no es necesario más.
        
        Por tanto, con $n=1$, el resultado queda:
        $$\ln (3) \approx P_{1, e}^{\ln x}\left(3\right)
        = \ln e + \frac{1}{3}(3-e) = 1+\frac{3-e}{3}$$

        
        \item $\sen\left( \frac{1}{2} \right)$\\
        Vamos a aproximar el valor mediante el desarrollo de Taylor del seno.
        $$\sen\left( \frac{1}{2} \right) \approx P_{n, 0}^{\sen x}\left(\frac{1}{2}\right)$$

        El error viene dado por:
        \begin{equation*}
            R_{n,0}^{\sen x}\left(x\right) = \frac{\sen^{n+1)} (c)}{(n+1)!}x^{n+1} \quad \text{para algún } c\in ]0, x[
        \end{equation*}
        Como estamos evaluando en $x=\frac{1}{2}$, y sabiendo que el seno y el coseno están acotados entre $-1$ y $1$, tenemos el siguiente resultado:
        \begin{equation*}
            -1 < \sen^{n+1)} (c) < 1
        \end{equation*}

        Como nos piden que el error sea menor a $10^{-2}$, la inecuación a resolver es:
        \begin{equation*}
            R_{n,0}^{\sen x}\left(\frac{1}{2}\right) = \frac{\sen^{n+1)} (c)}{(n+1)!}\frac{1}{2^{n+1}} < 10^{-2}
        \end{equation*}
        Usando la condición de que $\sen c < 1$:
        \begin{equation*}
            R_{n,0}^{\sen x}\left(\frac{1}{2}\right) = \frac{\sen^{n+1)} (c)}{(n+1)!}\frac{1}{2^{n+1}} < \frac{1}{(n+1)!}\frac{1}{2^{n+1}}  < 10^{-2}
        \end{equation*}
        Por tanto, resuelvo la siguiente inecuación para hallar el valor de $n$:
        \begin{equation*}
            \frac{1}{(n+1)!}\frac{1}{2^{n+1}}  < 10^{-2} \Longrightarrow \frac{1}{(n+1)!2^{n+1}} < \frac{1}{100} \Longrightarrow 100 < (n+1)!2^{n+1}
        \end{equation*}
        Como $n\in \bb{N}$, podemos ver que $n=3$ satisface la inecuación. Para valores mayores de $n$ la aproximación será mejor, pero para obtener error menor al pedido no es necesario más. Por tanto, con $n=3$, el resultado queda:
        $$\sen\left( \frac{1}{2} \right) \approx P_{3, 0}^{\sen x}\left(\frac{1}{2}\right) = \frac{1}{2} - \frac{1}{2^3 \cdot 3!} = \frac{23}{48}$$

        \begin{comment}
        Vamos a aproximar el valor mediante el desarrollo de Taylor de $f(x)=\ln \left(\frac{1+x}{1-x}\right) = \ln (1+x)-\ln(1-x)$. Tomando $g(x)=\ln (1+x),\;h(x)=\ln(1-x)$, entonces $f(x)=g(x)-h(x)$.
        $$\ln{3}\approx P_{n, 0}^{f}\left(\frac{1}{2}\right)$$

        Para saber de qué orden debe ser la aproximación, establecemos la condición de que el error debe ser menor que $10^{-2}$. El error viene dado por:
        \begin{equation*}
            R_{n,0}^{f}\left(x\right) = \frac{f^{n+1)}(c)}{(n+1)!}x^{n+1}
            = \frac{x^{n+1}}{(n+1)!}\left(g^{n+1)}(c) - h^{n+1)}(c) \right)
            \quad \text{para algún } c\in ]0, x[
        \end{equation*}

        Calculo en primer lugar la derivada de orden $n+1$ de $g(x)=\ln(1+x)$. Demuestro por inducción sobre $n$ que:
        \begin{equation*}
            g^{n+1)}(x) = (-1)^n \frac{n!}{(1+x)^{n+1}}
        \end{equation*}
        \begin{itemize}
            \item \underline{Para $n=0$}: Vemos que es cierto
            \begin{equation*}
                g'(x) = \frac{1}{1+x}
            \end{equation*}

            \item \underline{Supuesto cierto para $n-1$, demuestro para $n$}:
            \begin{multline*}
                g^{n)}(x) = (-1)^{n-1} \frac{(n-1)!}{(1+x)^{n}}
                \Longrightarrow\\\Longrightarrow
                g^{n+1)}(x) = (-1)^{n-1}\cdot (-1) \frac{(n-1)! \cdot n}{(1+x)^{n+1}} = (-1)^{n} \frac{n!}{(1+x)^{n+1}}
            \end{multline*}
        \end{itemize}

        Calculo ahora la derivada de orden $n+1$ de $h(x)=\ln(1-x)$. Demuestro por inducción sobre $n$ que:
        \begin{equation*}
            h^{n+1)}(x) = - \frac{n!}{(1-x)^{n+1}}
        \end{equation*}
        \begin{itemize}
            \item \underline{Para $n=0$}: Vemos que es cierto
            \begin{equation*}
                h'(x) = -\frac{1}{1-x}
            \end{equation*}

            \item \underline{Supuesto cierto para $n-1$, demuestro para $n$}:
            \begin{multline*}
                h^{n)}(x) = - \frac{(n-1)!}{(1-x)^{n}}
                \Longrightarrow\\\Longrightarrow
                h^{n+1)}(x) = - \frac{(n-1)! \cdot n}{(1-x)^{n+1}} = - \frac{n!}{(1-x)^{n+1}}
            \end{multline*}
        \end{itemize}

        Por tanto, $f^{n+1)}(x) = (-1)^n \frac{n!}{(1+x)^{n+1}} + \frac{n!}{(1-x)^{n+1}} = n! \left(\frac{(-1)^n }{(1+x)^{n+1}} + \frac{1}{(1-x)^{n+1}} \right)$
        

        Por tanto, el error cometido viene dado por:
        \begin{multline*}
            R_{n,0}^{f}\left(\frac{1}{2}\right) = \frac{ f^{n+1)}(c)}{(n+1)!}{\frac{1}{2^{n+1}}} = n! \left(\frac{(-1)^n }{(1+c)^{n+1}} + \frac{1}{(1-c)^{n+1}} \right) \cdot \frac{1}{(n+1)!} \cdot \frac{1}{2^{n+1}}
            =\\=
            \left(\frac{(-1)^n }{(1+c)^{n+1}} + \frac{1}{(1-c)^{n+1}} \right) \cdot \frac{1}{(n+1)2^{n+1}}
        \end{multline*}

        Sabiendo que $0<c<\frac{1}{2}$, acotamos por tanto lo siguiente:
        \begin{multline*}
            \frac{(-1)^n }{(1+c)^{n+1}} + \frac{1}{(1-c)^{n+1}}
            \leq
            \frac{1}{(1+c)^{n+1}} + \frac{1}{(1-c)^{n+1}}
            <\\<
            \frac{1}{(1+0)^{n+1}} + \frac{1}{\left(1-\frac{1}{2}\right)^{n+1}}
            =
            1 + \frac{1}{\left(\frac{1}{2}\right)^{n+1}}
        \end{multline*}

        Por tanto,
        \begin{multline*}
            R_{n,0}^{f}\left(\frac{1}{2}\right) = 
            \left(\frac{(-1)^n }{(1+c)^{n+1}} + \frac{1}{(1-c)^{n+1}} \right) \cdot \frac{1}{(n+1)2^{n+1}}
            \leq \\ \leq 
            \left(1 + \frac{1}{\left(\frac{1}{2}\right)^{n+1}}\right) \cdot \frac{1}{(n+1)2^{n+1}}
            = \frac{1}{(n+1)2^{n+1}} + \frac{1}{n+1}
        \end{multline*}
        \end{comment}
        \begin{comment}
        Estudio ahora la monotonía de $f^{n+1)}(x)$ si $x\in \left[0,\frac{1}{2}\right]$. Como $f^{n+1)}(x)$ es derivable en $\left[0,\frac{1}{2}\right]$, entonces el punto crítico de $f^{n+1)}(x)$ es:
        \begin{multline*}
            f^{n+2)}(x)=0 \Longleftrightarrow
            (-1)^{n+1} \frac{(n+1)!}{(1+x)^{n+2}} = - \frac{(n+1)!}{(1-x)^{n+2}}
            \Longleftrightarrow\\\Longleftrightarrow
            \frac{(-1)^{n+1}}{(1+x)^{n+2}} = - \frac{1}{(1-x)^{n+2}}
            \Longleftrightarrow
            (-1)^{n}(1-x)^{n+2} = (1+x)^{n+2}
            \Longleftrightarrow\\\Longleftrightarrow
            (-1)^{n}(1-x)^{n} = (1+x)^{n}
            \Longleftrightarrow
            x=0\text{ y $n$ par}
        \end{multline*}
        
        Por tanto, $f^{n+1)}(x)$ no cambia su de monotonía en $\left[0,\frac{1}{2}\right]$. Veamos si es creciente o decreciente.
        \begin{itemize}
            \item \underline{Si $n$ es impar}:
            \begin{equation*}
                f^{n+2)}(x) = \frac{(n+1)!}{(1+x)^{n+2}} + \frac{(n+1)!}{(1-x)^{n+2}} > 0
            \end{equation*}
            \item \underline{Si $n$ es par}:
            \begin{equation*}
                f^{n+2)}(x) = -\frac{(n+1)!}{(1+x)^{n+2}} + \frac{(n+1)!}{(1-x)^{n+2}} > 0
            \end{equation*}
        \end{itemize}
        
        Por tanto, $f^{n+1)}(x)$ es creciente en $\left[0,\frac{1}{2}\right]$.

        Supongo $n$ par. Esto es posible, ya que lo único que podría ser es que el menor $n$ que acota el error fuese impar, por lo que al considerar un grado de $n$ mayor tendremos menos error aún. Por tanto,
        \begin{equation*}
            f^{n+1)}(x) =n!\left( \frac{1}{(1+x)^{n+1}} + \frac{1}{(1-x)^{n+1}}\right)
        \end{equation*}

        Como estamos evaluando en $x=\frac{1}{2}$, y sabiendo que $f^{n+1)}(x)$ es estrictamente creciente en $\left[0,\frac{1}{2}\right]$, tenemos el siguiente resultado:
        \begin{equation*}
            0 < c < \frac{1}{2} \Longrightarrow 2(n!) < f^{n+1)}(c) <  n!\left( \frac{1}{(1.5)^{n+1}} + \frac{1}{(0.5)^{n+1}}\right)
        \end{equation*}

        Como nos piden que el error sea menor a $10^{-2}$, la inecuación a resolver es:
        \begin{equation*}
            R_{n,0}^{f}\left(\frac{1}{2}\right) = \frac{ f^{n+1)}(c)}{(n+1)!}{\frac{1}{2^{n+1}}} < 10^{-2}
        \end{equation*}
        Usando la condición de que $f^{n+1)}(c) < n!\left( \frac{1}{(1.5)^{n+1}} + \frac{1}{(0.5)^{n+1}}\right)$ para $n$ par, tenemos que:
        \begin{equation*}
            R_{n,0}^{\ln (x+1)}\left(\frac{1}{2}\right) = \frac{n!\left( \frac{1}{(1.5)^{n+1}} + \frac{1}{(0.5)^{n+1}}\right)}{(n+1)!} \frac{1}{2^{n+1}}
            = \frac{\frac{1}{(1.5)^{n+1}} + \frac{1}{(0.5)^{n+1}}}{2^{n+1}(n+1)} < 10^{-2}
        \end{equation*}
        Por tanto, resuelvo la siguiente inecuación para hallar el valor de $n$, sabiendo que $n$ es par:
        \begin{equation*}
            \frac{\frac{1}{(1.5)^{n+1}} + \frac{1}{(0.5)^{n+1}}}{2^{n+1}(n+1)} < 10^{-2}
            \Longleftrightarrow
            \frac{2^{n+1}(n+1)}{\frac{1}{(1.5)^{n+1}} + \frac{1}{(0.5)^{n+1}}} > 100
        \end{equation*}
        \end{comment}
        
        \begin{comment}
        Supongo $n$ impar. Esto es posible, ya que lo único que podría ser es que el menor $n$ que acota el error fuese par, por lo que al considerar un grado de $n$ mayor tendremos menos error aún. Por tanto,
        \begin{equation*}
            f^{n+1)}(x) =n!\left( -\frac{1}{(1+x)^{n+1}} + \frac{1}{(1-x)^{n+1}}\right)
        \end{equation*}

        Como estamos evaluando en $x=\frac{1}{2}$, y sabiendo que $f^{n+1)}(x)$ es estrictamente creciente en $\left[0,\frac{1}{2}\right]$, tenemos el siguiente resultado:
        \begin{equation*}
            0 < c < \frac{1}{2} \Longrightarrow 0 < f^{n+1)}(c) <  n!\left( -\frac{1}{(1.5)^{n+1}} + \frac{1}{(0.5)^{n+1}}\right)
        \end{equation*}

        Como nos piden que el error sea menor a $10^{-2}$, la inecuación a resolver es:
        \begin{equation*}
            R_{n,0}^{f}\left(\frac{1}{2}\right) = \frac{ f^{n+1)}(c)}{(n+1)!}{\frac{1}{2^{n+1}}} < 10^{-2}
        \end{equation*}
        Usando la condición de que $f^{n+1)}(c) < n!\left( -\frac{1}{(1.5)^{n+1}} + \frac{1}{(0.5)^{n+1}}\right)$ para $n$ impar, tenemos que:
        \begin{equation*}
            R_{n,0}^{\ln (x+1)}\left(\frac{1}{2}\right) = \frac{n!\left( \frac{1}{(1.5)^{n+1}} + \frac{1}{(0.5)^{n+1}}\right)}{(n+1)!} \frac{1}{2^{n+1}}
            = \frac{\frac{1}{(1.5)^{n+1}} + \frac{1}{(0.5)^{n+1}}}{2^{n+1}(n+1)} < 10^{-2}
        \end{equation*}
        Por tanto, resuelvo la siguiente inecuación para hallar el valor de $n$, sabiendo que $n$ es par:
        \begin{equation*}
            \frac{\frac{1}{(1.5)^{n+1}} + \frac{1}{(0.5)^{n+1}}}{2^{n+1}(n+1)} < 10^{-2}
            \Longleftrightarrow
            \frac{2^{n+1}(n+1)}{\frac{1}{(1.5)^{n+1}} + \frac{1}{(0.5)^{n+1}}} > 100
        \end{equation*}
        \end{comment}

        Como $n\in \bb{N}$ y hemos supuesto que es par, podemos ver que $n=3$ satisface la inecuación. Para valores mayores de $n$ la aproximación será mejor, pero para obtener error menor al pedido no es necesario más. Por tanto, con $n=3$, el resultado queda:
        $$\sqrt{e}\approx P_{3, 0}^{e^x}\left(\frac{1}{2}\right) = 1 + \frac{1}{2} + \frac{1}{2^2 \cdot 2!} + \frac{1}{2^3 \cdot 3!} = \frac{79}{48}$$
        
        \item $\sqrt{e}$\\
        Vamos a aproximar el valor mediante el desarrollo de Taylor de la exponencial.
        $$\sqrt{e}\approx P_{n, 0}^{e^x}\left(\frac{1}{2}\right)$$

        Para saber de qué orden debe ser la aproximación, establecemos la condición de que el error debe ser menor que $10^{-2}$. El error viene dado por:
        \begin{equation*}
            R_{n,0}^{e^x}\left(x\right) = \frac{e^c}{(n+1)!}x^{n+1} \quad \text{para algún } c\in ]0, x[
        \end{equation*}

        Como estamos evaluando en $x=\frac{1}{2}$, y sabiendo que la exponencial es estrictamente creciente, tenemos el siguiente resultado:
        \begin{equation*}
            0 < c < \frac{1}{2} \Longrightarrow 1 = e^0 < e^c < e^{1/2} < 2  \Longrightarrow e^c < 2
        \end{equation*}

        Como nos piden que el error sea menor a $10^{-2}$, la inecuación a resolver es:
        \begin{equation*}
            R_{n,0}^{e^x}\left(\frac{1}{2}\right) = \frac{e^c}{(n+1)!}\frac{1}{2^{n+1}} < 10^{-2}
        \end{equation*}
        Usando la condición de que $e^c < 2$:
        \begin{equation*}
            R_{n,0}^{e^x}\left(\frac{1}{2}\right) = \frac{e^c}{(n+1)!}\frac{1}{2^{n+1}} < \frac{2}{(n+1)!}\frac{1}{2^{n+1}}  < 10^{-2}
        \end{equation*}
        Por tanto, resuelvo la siguiente inecuación para hallar el valor de $n$:
        \begin{equation*}
            \frac{2}{(n+1)!}\frac{1}{2^{n+1}}  < 10^{-2} \Longrightarrow \frac{1}{(n+1)!2^n} < \frac{1}{100} \Longrightarrow 100 < (n+1)!2^n
        \end{equation*}

        Como $n\in \bb{N}$, podemos ver que $n=3$ satisface la inecuación. Para valores mayores de $n$ la aproximación será mejor, pero para obtener error menor al pedido no es necesario más. Por tanto, con $n=3$, el resultado queda:
        $$\sqrt{e}\approx P_{3, 0}^{e^x}\left(\frac{1}{2}\right) = 1 + \frac{1}{2} + \frac{1}{2^2 \cdot 2!} + \frac{1}{2^3 \cdot 3!} = \frac{79}{48}$$
    \end{enumerate}
\end{ejercicio}

\begin{ejercicio}
    Probar que la función $\ln x$ es cóncava hacia abajo. Deducir la Desigualdad de Young: si $\frac{1}{p} + \frac{1}{q} = 1$, siendo $p>1$, entonces $ab \leq \frac{a^p}{p} + \frac{b^q}{q}$, para cada $a,b \in \bb{R}^+$.

    Demuestro en primer lugar que $f(x)=\ln x$ es cóncava hacia abajo.
    \begin{equation*}
        f'(x)=\frac{1}{x} \qquad f''(x)=-\frac{1}{x^2}
    \end{equation*}
    Como $f''(x)<0 \quad \forall x\in \bb{R}^+$, tenemos que $f$ es cóncava hacia abajo en $\bb{R}^+$.

    Por tanto, si $x<y$ tenemos que
    \begin{equation*}
        f(tx+(1-t)y) \leq tf(x)+(1-t)f(y) \qquad \forall t\in [0,1]
    \end{equation*}

    Demostramos ahora la Desigualdad de Young. Sea el cambio de variable $x=a^p$ y $y=b^q$.

    \begin{itemize}
        \item \underline{Suponemos $a^p<b^q$}.
        
        Sea el cambio de variable $t=\frac{1}{p}$. Por tanto, como $t\in [0,1]$, necesitamos que $p\geq 1$ (se tiene, ya que $p>1$). Además, como $\frac{1}{p} + \frac{1}{q} = 1$, tenemos que $(1-t) = 1-\frac{1}{p} = \frac{1}{q}$. Por tanto:
        \begin{equation*}
            \ln \left(\frac{1}{p}a^p+\frac{1}{q}b^q\right) \geq \frac{1}{p}\ln (a^p)+\frac{1}{q}\ln (b^q) = \ln (a^p)^{\frac{1}{p}} + \ln (b^q)^{\frac{1}{q}} = \ln a + \ln b = \ln (ab)
        \end{equation*}
    
        Por tanto, como $\ln$ es una función creciente, tenemos que
        \begin{equation*}
            \frac{1}{p}x+\frac{1}{q}y = \frac{a^p}{p} + \frac{b^q}{q} \geq ab
        \end{equation*}
    
        Es decir, para $a^p<b^q$ se cumple la desigualdad.
        
        \item \underline{Suponemos $a^p>b^q$}.
        
        Sea el cambio de variable $t=\frac{1}{q}$. Por tanto, como $t\in [0,1]$, necesitamos que $q\geq 1$. Veamos que se da. Como $\frac{1}{p} + \frac{1}{q} = 1$, tenemos que:
        \begin{equation*}
            q = \frac{1}{1-\frac{1}{p}} = \frac{p}{p-1} \geq 1 \Longleftrightarrow p\geq p-1 \text{ cierto}
        \end{equation*}
        
        Por tanto, tenemos que $q\geq 1$. Sabiendo que $t=\frac{1}{q}$, tenemos que $1-t = 1-\frac{1}{q} = \frac{1}{p}$. Por tanto:
        \begin{equation*}
            \ln \left(\frac{1}{q}b^q+\frac{1}{p}a^p\right) \geq \frac{1}{q}\ln (b^q)+\frac{1}{p}\ln (a^p) = \ln (b^q)^{\frac{1}{q}} + \ln (a^p)^{\frac{1}{p}} = \ln b + \ln a = \ln (ab)
        \end{equation*}
    
        Por tanto, como $\ln$ es una función creciente, tenemos que
        \begin{equation*}
            \frac{1}{q}b^q+\frac{1}{p}a^p = \frac{a^p}{p} + \frac{b^q}{q} \geq ab
        \end{equation*}
    
        Es decir, para $a^p > b^q$ se cumple la desigualdad.
        
        \item \underline{Suponemos $a^p=b^q$}.
        
        \begin{equation*}
            \frac{a^p}{p} + \frac{b^q}{q} = a^p\left( \frac{1}{p} + \frac{1}{q}\right) = a^p
        \end{equation*}
        
        Por tanto, para $a^p=b^q$ hemos de comprobar que $a^p \geq ab$. Sabemos, lo siguiente:
        \begin{equation*}
            a^p=b^q \Longrightarrow p = \log_a(b^q)
            \qquad
            \frac{1}{p} + \frac{1}{q} = 1 \Longrightarrow q = \frac{p}{p-1}
        \end{equation*}
        \begin{equation*}
            a^p = b^q \Longrightarrow a^p = b^{\frac{p}{p-1}} \Longrightarrow p = \frac{p}{p-1} \log_a b \Longrightarrow \log_a b = {p-1}
        \end{equation*}
        
        Por tanto, usando las tres ecuaciones anteriores,
        \begin{equation*}
            a^p \geq ab \Longleftrightarrow p \geq \log_a(ab) = 1+\log_a(b) = 1+p-1 = p \Longleftrightarrow p \geq p
        \end{equation*}
    
        Por tanto, para $a^p = b^q$ se da la desigualdad.
    \end{itemize}
\end{ejercicio}

\begin{ejercicio}
    Sean $I$, $J$ intervalos, y $f:I\longrightarrow \bb{R}$ y $g:J\longrightarrow \bb{R}$ funciones cóncavas hacia arriba tales que $f(I)\subseteq J$. Probar que si $g$ es creciente, entonces $g\circ f$ es cóncava hacia arriba. Deducir que la función $h:I\longrightarrow \bb{R}$ dada por $h(x)=e^{f(x)}$ es cóncava hacia arriba.

    Como la función $f$ es cóncava hacia arriba, dados $a,b\in I\mid a<b$, tenemos que:
    \begin{equation*}
        f(ta + (1-t)b) \leq tf(a) +(1-t)f(b) \qquad \forall t\in [0,1]
    \end{equation*}

    Como $f(I)\subset J$, entonces puedo aplicar $g$ al primer lado de la desigualdad. Veamos ahora que el segundo lado de la desigualdad también pertenece a $J$.

    Supongamos, sin perder generalidad, que $f(a)\leq f(b)$. Entonces, es fácil ver que $[f(a),f(b)] \subset J$. Entonces,
    \begin{equation*}
        f(a)
        = tf(a) +(1-t)f(a)
        \leq tf(a) +(1-t)f(b)
        \leq tf(b) +(1-t)f(b)
        = f(b)
    \end{equation*}

    Por tanto, $tf(a) +(1-t)f(b) \in [f(a),f(b)] \subset J$, por lo que puedo aplicar $g$ también al segundo lado de la desigualdad. Además, como sé que $g$ es una función creciente, el sentido de la desigualdad se conserva. Por tanto,
    \begin{multline*}
        (g\circ f)(ta + (1-t)b) = g(f(ta + (1-t)b))
        \leq g(tf(a) +(1-t)f(b))
        \qquad \forall t\in [0,1]
    \end{multline*}

    Aplicamos ahora que $g$ es cóncava hacia arriba.
    \begin{itemize}
        \item \underline{Supuesto $f(a)<f(b)$}
        \begin{multline*}
            (g\circ f)(ta + (1-t)b) 
            \leq g(tf(a) +(1-t)f(b))
            \stackrel{(1)}{\leq}
            tg(f(a)) +(1-t)g(f(b)) =\\= t(g\circ f)(a)+(1-t)(g\circ f)(b)
            \qquad \forall t\in [0,1]
        \end{multline*}
        donde en $(1)$ he aplicado que $g$ es cóncava hacia arriba. Por tanto, tenemos que $g\circ f$ es cóncava hacia arriba.

        \item \underline{Supuesto $f(a)>f(b)$}

        Realizo el cambio de variable $s=1-t$. Tenemos que $s\in [0,1]$.
        \begin{multline*}
            (g\circ f)(ta + (1-t)b) 
            \leq g(tf(a) +(1-t)f(b)) 
            = g(sf(b) +(1-s)f(a))
            \stackrel{(1)}{\leq} \\ \stackrel{(1)}{\leq}
            sg(f(b)) +(1-s)g(f(a)) = s(g\circ f)(b)+(1-s)(g\circ f)(a)
            = \\ =
            t(g\circ f)(a)+(1-t)(g\circ f)(b)
            \qquad \forall t\in [0,1]
        \end{multline*}
        donde en $(1)$ he aplicado que $g$ es cóncava hacia arriba. Por tanto, tenemos que $g\circ f$ es cóncava hacia arriba.

        \item \underline{Supuesto $f(a)=f(b)$}
        Tenemos que $f$ es constante, por lo que $g\circ f$ también lo es. Por tanto, como las funciones constantes son cóncavas hacia arriba y hacia abajo, se tiene.
    \end{itemize}
    

    Respecto a la función $h(x)=e^{f(x)}$, estamos en las condiciones del enunciado con $g(x)=e^x$. Por tanto, tenemos que la composición también es cóncava hacia arriba.    
\end{ejercicio}

\begin{ejercicio}
    Dar un ejemplo que muestre que la composición de dos funciones cóncavas hacia arriba puede no ser cóncava hacia arriba.\\

    Sean $f(x)=x^2-4$ y $g(x)=|x|$ definidas en $\bb{R}$. Ambas son funciones cóncavas hacia arriba. Sin embargo, $(g\circ f)(x) = |x^2-4|$ no es cóncava hacia arriba, ya que su restricción a $[-2,2]$ es cóncava hacia abajo.
\end{ejercicio}

\begin{ejercicio}
    En cada uno de los siguientes casos, determinar los intervalos en los que la función $f:\bb{R}\to \bb{R}$ es cóncava hacia arriba o cóncava hacia abajo.
    \begin{enumerate}
        \item $f(x)=x^5-5x^4+5x^3 + 10$
        \begin{equation*}
            f'(x) = 5x^4 -20x^3+15x^2
        \end{equation*}
        \begin{equation*}
            f''(x)=20x^3 -60x^2 +30x = 10x(2x^2-6x+3) = 10x\left(\frac{3+\sqrt{3}}{2}-x\right)\left(\frac{3-\sqrt{3}}{2}-x\right)
        \end{equation*}

        Sabemos que $f$ derivable en $\bb{R}$, por lo que los únicos candidatos a puntos de inflexión son los que anulan la segunda derivada, es decir, $\left\{0, \frac{3+\sqrt{3}}{2}, \frac{3-\sqrt{3}}{2}  \right\}$.

        Veamos si efectivamente son puntos de inflexión:
        \begin{itemize}
            \item \underline{Para $x<0$}:
            $f''(x)<0\Longrightarrow f$ es cóncava hacia abajo en este intervalo.

            \item \underline{Para $0<x< \frac{3-\sqrt{3}}{2}$}:
            $f''(x)>0\Longrightarrow f$ es cóncava hacia arriba en este intervalo.

            \item \underline{Para $ \frac{3-\sqrt{3}}{2} < x <  \frac{3+\sqrt{3}}{2}$}:
            $f''(x)<0\Longrightarrow f$ es cóncava hacia abajo en este intervalo.
            
            \item \underline{Para $x> \frac{3+\sqrt{3}}{2}$}:
            $f''(x)>0\Longrightarrow f$ es cóncava hacia arriba en este intervalo.
        \end{itemize}
        Como en los 3 candidatos a punto de inflexión se produce un cambio de concavidad, tenemos que efectivamente son todos los puntos de inflexión.
        
        \item $f(x)=\frac{x^2+3x+1}{x^2+1}$
        \begin{multline*}
            f'(x) = \frac{(2x+3)(x^2+1)-2x(x^2+3x+1)}{(x^2+1)^2} = \frac{\cancel{2x^3}+\cancel{2x}+3x^2+3-\cancel{2x^3}-6x^2-\cancel{2x}}{(x^2+1)^2} =\\= \frac{3(1-x^2)}{(1+x^2)^2}
        \end{multline*}
        \begin{multline*}
            f''(x) = 3\cdot \frac{-2x(1+x^2)^{\cancel{2}} -4x(1-x^2)\cancel{(1+x^2)}}{(1+x^2)^{\cancelto{3}{4}}}
            = 3\cdot \frac{-2x-2x^3-4x+4x^3}{(1+x^2)^3}
            =\\=
            3\cdot \frac{-6x+2x^3}{(1+x^2)^3}
            = 6x\cdot \frac{x^2-3}{(1+x^2)^3}
        \end{multline*}

        Además, sabemos que $f$ es derivable en $\bb{R}$. Por tanto, los candidatos a puntos de inflexión son aquellos que anulan la segunda derivada, es decir, $\{0, \pm \sqrt{3}\}$.
        \begin{itemize}
            \item \underline{Para $x<-\sqrt{3}$}:
            $f''(x)<0\Longrightarrow f$ es cóncava hacia abajo en este intervalo.

            \item \underline{Para $-\sqrt{3}<x<0$}:
            $f''(x)>0\Longrightarrow f$ es cóncava hacia arriba en este intervalo.

            \item \underline{Para $0<x<\sqrt{3}$}:
            $f''(x)<0\Longrightarrow f$ es cóncava hacia abajo en este intervalo.
            
            \item \underline{Para $x>\sqrt{3}$}:
            $f''(x)>0\Longrightarrow f$ es cóncava hacia arriba en este intervalo.
        \end{itemize}

        Como en los 3 candidatos a punto de inflexión se produce un cambio de concavidad, tenemos que efectivamente son todos los puntos de inflexión.
        
        \item $f(x)=\ln (1+x^2)$
        \begin{equation*}
            f'(x) = \frac{2x}{1+x^2}
            \qquad
            f''(x) = \frac{2(1+x^2)-4x^2}{(1+x^2)^2}
            = \frac{-2x^2+2}{(1+x^2)^2} 
        \end{equation*}
        Además, sabemos que $f$ es derivable en $\bb{R}$. Por tanto, los únicos candidados a puntos de inflexión son los puntos que anulan la segunda derivada, es decir, $\{\pm 1\}$.
        \begin{itemize}
            \item \underline{Para $x<-1$}:
            $f''(x)<0\Longrightarrow f$ es cóncava hacia abajo en este intervalo.

            \item \underline{Para $-1<x<1$}:
            $f''(x)>0\Longrightarrow f$ es cóncava hacia arriba en este intervalo.

            \item \underline{Para $x>1$}:
            $f''(x)<0\Longrightarrow f$ es cóncava hacia abajo en este intervalo.
        \end{itemize}
        Como en los dos candidatos a punto de inflexión se produce un cambio de concavidad, tenemos que efectivamente son todos los puntos de inflexión.
        
        \item $f(x)=\sen x$
        \begin{equation*}
            f''(x) = -\sen x = 0 \Longleftrightarrow x={\pi}k, \;k\in \bb{Z}.
        \end{equation*}

        Además, sabemos que $\sen x \in C^{\infty}$, por lo que los dados son todos los candidatos a puntos de inflexión. Para ver los intervalos, hago uso de que tiene periodicidad con periodo $T=2\pi$. Estudio, por tanto, en $[0,2\pi]$:
        \begin{itemize}
            \item \underline{Para $0<x<\pi$}:
            $f''(x)<0\Longrightarrow f$ es cóncava hacia abajo en este intervalo.

            \item \underline{Para $\pi<x<2\pi$}:
            $f''(x)>0\Longrightarrow f$ es cóncava hacia arriba en este intervalo.
        \end{itemize}

        Generalizando para todo el dominio, para $k\in \bb{Z}$,
        \begin{itemize}
            \item \underline{Para $2k\pi<x<(2k+1)\pi$}:
            $f''(x)<0\Longrightarrow f$ es cóncava hacia abajo en este intervalo.

            \item \underline{Para $(2k+1)\pi<x<2(k+1)\pi$}:
            $f''(x)>0\Longrightarrow f$ es cóncava hacia arriba en este intervalo.
        \end{itemize}
        
        
        Por tanto, como en todos los candidatos se produce un cambio de concavidad, efectivamente los puntos de inflexión son:
        \begin{equation*}
            x=\{\pi k, \;k\in \bb{Z}\}
        \end{equation*}
    \end{enumerate}
\end{ejercicio}

\begin{ejercicio}
    Demostrar que toda función $f:\bb{R}\to \bb{R}$ cóncava hacia abajo y acotada es constante.\\

    %\begin{comment}
    Demostramos mediante reducción al absurdo. Suponemos que $f$ no es constante. Por tanto, $\exists x,y\in \bb{R}\mid x\neq y \land f(x)<f(y)$.
    \begin{itemize}
        \item \underline{Para $x<y$}:
        \begin{equation*}
            f(x) = f\left( t\cdot \frac{x-(1-t)y}{t} + (1-t)y\right) \geq tf\left(\frac{x-(1-t)y}{t}\right) + (1-t)f(y) \qquad \forall t\in ]0,1]
        \end{equation*}

        Despejando,
        \begin{equation*}
            \frac{f(x)-(1-t)f(y)}{t} = \frac{f(x)-f(y)}{t} + f(y) \geq f\left(\frac{x-(1-t)y}{t}\right)
        \end{equation*}
        Tomando límite con $t\to 0^+$, tenemos que
        \begin{equation*}
             \frac{f(x)-f(y)}{t} + f(y) \longrightarrow -\infty
        \end{equation*}
    
        Por tanto, para $t\approx 0$, tenemos que:
        \begin{equation*}
            -\infty \geq f\left(\frac{x-(1-t)y}{t}\right)
        \end{equation*}
    
        Por lo que $f$ no tiene cota inferior, en contradicción con que $f$ sea acotada.

        \item \underline{Para $x>y$}:
        \begin{equation*}
            f(x) = f\left( (1-t')\cdot \frac{x-(t')y}{1-t'} + t'y\right) \geq (1-t')f\left(\frac{x-t'y}{1-t'}\right) + t'f(y) \qquad \forall t'\in [0,1[
        \end{equation*}
        
        Despejando,
        \begin{equation*}
            \frac{f(x)-t'f(y)}{1-t'} \geq f\left(\frac{x-t'y}{1-t'}\right)
        \end{equation*}
        Tomando límite con $t'\to 1^-$, tenemos que
        \begin{equation*}
             \frac{f(x)-t'f(y)}{1-t'} \longrightarrow \frac{f(x)-f(y)}{0^+} = -\infty
        \end{equation*}
    
        Por tanto, para $t'\approx 1$, tenemos que:
        \begin{equation*}
            -\infty \geq f\left(\frac{x-t'y}{1-t'}\right)
        \end{equation*}
    
        Por lo que $f$ no tiene cota inferior, en contradicción con que $f$ sea acotada.
    \end{itemize}

    Por tanto, concluimos que $f$ es constante.

    
    %\end{comment}

    \begin{comment}
    Demostramos mediante reducción al absurdo. Suponemos que $f$ no es constante. Elegimos $b_0\in \bb{R}$ fijo, y cualquier $x\in \bb{R}$ es de la forma $x=ta + (1-t)b_0 \qquad t\in[0,1]$, para cierto $a\in \bb{R}$.

    \begin{itemize}
        \item \underline{Supuesto $a=b_0$}:
        
        \begin{equation*}
            f(x) = f(ta +(1-t)b_0) = f(ta+(1-t)b_0)  = f(b_0)
        \end{equation*}
        
        \item \underline{Supuesto $a<b_0$}:
        \begin{equation*}
            f(x) = f(ta+(1-t)b_0) \geq tf(a)+(1-t)f(b_0) 
        \end{equation*}

        Con $t\to 0$, tenemos que $f(x)\geq f(b_0)$.

        \item \underline{Supuesto $a>b_0$}:

        Realizamos el cambio de variable $s=1-t$:
        \begin{equation*}
            f(x) = f(sb_0 +(1-s)a) = f(sb_0+(1-s)a) \geq sf(b_0)+(1-s)f(a) 
        \end{equation*}

        Con $s\to 1$, tenemos que $f(x)\geq f(b_0)$.
    \end{itemize}
    
    Por tanto, tenemos que $f(x)\geq f(b_0)$.
    \end{comment}
    \begin{comment}
    \begin{itemize}
        \item \underline{Supuesto $a<b$}:
        
        Como $f$ es cóncava hacia abajo, supuesto $a<b$, tenemos que:
        \begin{equation*}
            f(ta+(1-t)b) \geq tf(a) + (1-t)f(b) \qquad \forall t\in [0,1]
        \end{equation*}
    
        Sea $x=ta+(1-t)b$. Entonces, $b=\frac{x-ta}{1-t}$. Por tanto,
        \begin{equation*}
            f(x) \geq tf(a) + (1-t)f\left(\frac{x-ta}{1-t}\right) \qquad \forall t\in [0,1]
        \end{equation*}
    
        Despejando,
        \begin{equation*}
            \frac{f(x)-tf(a)}{1-t} \geq f\left(\frac{x-ta}{1-t}\right) \qquad \forall t\in [0,1[
        \end{equation*}
    
        Sabemos que $f$ es acotada, por lo que $\exists M\in \bb{R}^+ \mid |f(x)|\leq M\;\forall x\in\bb{R}$. Por tanto,
        \begin{equation*}
            \frac{M}{1-t} \geq \frac{f(x)-tf(a)}{1-t} \geq f\left(\frac{x-ta}{1-t}\right) \geq -M
        \end{equation*}
        
    \end{itemize}
    \end{comment}
\end{ejercicio}

\begin{ejercicio}
    Calcular los puntos de inflexión (si los hay) de las funciones:
    \begin{enumerate}
        \item $f(x)=\sqrt[3]{x+2}$
        \begin{equation*}
            f'(x) = \frac{1}{3}(x+2)^{-\frac{2}{3}}
        \end{equation*}
        \begin{equation*}
            f''(x) = -\frac{2}{9}(x+2)^{-\frac{5}{3}} = -\frac{2}{9\sqrt[5]{(x+2)^3}}
        \end{equation*}

        Sabemos que $f$ es derivable en $\bb{R}-\{-2\}$. Por tanto, como no hay puntos que anulen la segunda derivada, y una condición necesaria para que $x\in\bb{R}-\{0\}$ sea punto de inflexión es que $f''(c)=0$, tenemos que no hay puntos de inflexión en $\bb{R}-\{-2\}$.

        Estudiamos para $x=-2$, ya que $f$ no es derivable en $x=-2$.
        \begin{itemize}
            \item \underline{Para $x<-2$:} $f''(x)>0 \Longrightarrow f$ es cóncava hacia arriba en ese intervalo.
            \item \underline{Para $x>-2$:} $f''(x)<0 \Longrightarrow f$ es cóncava hacia abajo en ese intervalo.
        \end{itemize}
        Por tanto, como en $x=-2$ se produce un cambio de concavidad, tenemos que $x=-2$ es un punto de inflexión.
        
        \item $f(x)=2x^3 + 9x^2 + 2x + 1$
        \begin{equation*}
            f'(x) = 6x^2 + 18x + 2
        \end{equation*}
        \begin{equation*}
            f''(x) = 12x + 18 = 0 \Longleftrightarrow x=-\frac{18}{12} = -\frac{3}{2}
        \end{equation*}

        Además, como $f'''(x) = 12 \neq 0 \;\forall x$, el candidato a punto de inflexión efectivamente lo es. Además, no hay más puntos de inflexión por ser $f$ derivable en $\bb{R}$.
    \end{enumerate}
\end{ejercicio}

\begin{ejercicio}
    Sea $f:I\to \bb{R}$ una función cóncava hacia arriba. Probar que:
    \begin{enumerate}
        \item Si $f$ tiene un mínimo relativo en $x_0\in I$ entonces $f$ tiene un mínimo absoluto en $x_0$.

        Sea $a\in I$ el mínimo absoluto; es decir, $f(a)<f(x)\quad \forall x\in I$. Dividimos la demostración en dos partes.
        
        Supuesto $x_0 < a$, por ser $f$ cóncava hacia arriba, tenemos que
        \begin{equation*}
            f(tx_0 + (1-t)a) \leq tf(x_0) + (1-t)f(a) \qquad \forall t\in [0,1] 
        \end{equation*}

        Entonces, como $f(a)<f(x_0)$, tenemos:
        \begin{equation*}
            f(tx_0 + (1-t)a) \leq tf(x_0) + (1-t)f(a) < tf(x_0) + (1-t)f(x_0) = f(x_0)\qquad \forall t\in [0,1] 
        \end{equation*}

        Por tanto,
        \begin{equation*}
            f(tx_0 + (1-t)a) < f(x_0)\qquad \forall t\in [0,1] 
        \end{equation*}

        Para $t\to 1$, tenemos que $tx_0 + (1-t)a \to x_0$. Por tanto, para $t\approx 1$, tenemos que
        \begin{equation*}
            \exists r>0 \mid tx_0+(1-t)a \in ]x_0-r, x_0+r[, \text{ con }
            f(tx_0 + (1-t)a) < f(x_0)
        \end{equation*}
        Esto, sin embargo, contradice que $x_0$ sea un mínimo relativo, por lo que $x_0\geq a$.

        Supongamos ahora que $a<x_0$. Por ser $f$ cóncava hacia arriba, tenemos que
        \begin{equation*}
            f(ta + (1-t)x_0) \leq tf(a) + (1-t)f(x_0) \qquad \forall t\in [0,1] 
        \end{equation*}

        Entonces, como $f(a)<f(x_0)$, tenemos:
        \begin{equation*}
            f(ta + (1-t)x_0) \leq tf(a) + (1-t)f(x_0) < tf(x_0) + (1-t)f(x_0) = f(x_0) \qquad \forall t\in [0,1] 
        \end{equation*}

        Por tanto,
        \begin{equation*}
            f(ta + (1-t)x_0) < f(x_0)\qquad \forall t\in [0,1] 
        \end{equation*}

        Para $t\to 0$, tenemos que $ta + (1-t)x_0 \to x_0$. Por tanto, para $t\approx 0$, tenemos que
        \begin{equation*}
            \exists r>0 \mid ta+(1-t)x_0 \in ]x_0-r, x_0+r[, \text{ con }
            f(ta + (1-t)x_0) < f(x_0)
        \end{equation*}
        Esto, sin embargo, contradice que $x_0$ sea un mínimo relativo. Por tanto, $a\geq x_0$.

        Uniendo ambos resultados, tenemos que $x_0=a$. Por tanto, $x_0$ es el mínimo absoluto.

        \item Si $f$ es derivable en $I$ y $x_0 \in I$ es un punto crítico de $f$ entonces $f$ alcanza un mínimo absoluto en $x_0$.

        Sabiendo $f$ es derivable en $I$ y $x_0$ es un punto crítico; por la condición necesaria de punto crítico tenemos que $f'(x_0)=0$.

        Como $f$ es cóncava hacia arriba y es derivable, tenemos que $f''(x)>0 \quad \forall x\in I$. Por tanto, $f'$ es creciente.

        \begin{itemize}
            \item Supuesto $x<x_0 \Longrightarrow f'(x) \leq f'(x_0)=0 \Longrightarrow f'(x)\leq 0 \Longrightarrow f$ decreciente.

            \item Supuesto $x>x_0 \Longrightarrow f'(x) \geq f'(x_0)=0 \Longrightarrow f'(x)\geq 0 \Longrightarrow f$ creciente.
        \end{itemize}

        Por tanto, como en $x_0$ se produce un cambio de crecimiento, es un mínimo relativo. Como es el único extremo relativo, es el extremo absoluto.
        
    \end{enumerate}
\end{ejercicio}
\section{Aplicaciones del Cálculo Integral}

\begin{ejercicio}
    Calcular el área limitada por las gráficas de las funciones dadas por $f(x)=3x$ y por $g(x)=x^2$.
\end{ejercicio}

\begin{ejercicio}
    Calcular mediante integración, el área de un triángulo y de un trapecio.
\end{ejercicio}

\begin{ejercicio}
    Calcular el área del recinto limitado por la gráfica de la función $f(x)=~x^3$, el eje de abcisas y las rectas $x=-1$ y $x=1$.
\end{ejercicio}

\begin{ejercicio}
    Calcular el área del recinto limitado por la curva que tiene por ecuación $f(x)=\frac{x-2}{(x-4)(x+2)}$, el eje de abcisas y las rectas $x=-1$ y $x=3$.
\end{ejercicio}

\begin{ejercicio}
    Calcular el área del recinto limitado por la parábola que tiene por ecuación $y^2-2x=0$ y la recta que une los puntos $(2,-2)$ y $(4,2\sqrt{2})$.
\end{ejercicio}

\begin{ejercicio}
    Calcular el área del recinto limitado por la parábola que tiene por ecuación $f(x)=4x-x^2$ y el eje de abcisas.
\end{ejercicio}

\begin{ejercicio}
    Calcular el área de los recintos limitados por la circunferencia de ecuación $x^2+y^2=1$ y la recta de ecuación $x+y=1$.
\end{ejercicio}

\begin{ejercicio}
    Calcular el área comprendida entre las curvas $x^2+y^2=2$ y $2y=x^2+1$.
\end{ejercicio}

\begin{ejercicio}
    Calcular el área de la región acotada delimitada por la gráfica de la función $f(x)=x^2\ln x$, la recta $x=e$, y el eje $OX$.

    Para saber qué área me piden, representamos la gráfica de $f(x)$. Como no indican dominio, tomamos el dominio maximal, es decir, $Dom(f)=\bb{R}^+$. Tenemos que $f$ es derivable en $\bb{R}^+$, con
    \begin{equation*}
        f'(x)=2x\ln x +\frac{x^2}{x} = 2x\ln x +x = x(2\ln x +1)=0\Longleftrightarrow \ln x = -\frac{1}{2} \Longleftrightarrow x = \frac{1}{\sqrt{e}}
    \end{equation*}
\end{ejercicio}

\begin{ejercicio}
    Calcular el área limitada por la curva $y(x)=\frac{a}{2}\left(e^{\frac{x}{a}} + e^{-\frac{x}{a}}\right)$, los ejes coordenados y la recta $x=a$, siendo $a>0$.
\end{ejercicio}

\begin{ejercicio}
    Calcular el área de cada una de las regiones del plano que delimitan conjuntamente las funciones $f(x)=\frac{1}{1+x}$ y $g(x)=\frac{1}{1+x^2}$ en el primer cuadrante.
\end{ejercicio}

\begin{ejercicio}
    Calcular el área de las dos partes en que la parábola $y^2=4x$ divide al círculo $x^2+y^2=8$.
\end{ejercicio}

\begin{ejercicio}
    Calcular el valor de $\lambda$ para el cual la curva $y=\lambda \cos x$ divide en dos partes de igual área a la región limitada por la curva $y=\sen x$ y el eje de abscisas cuando $0\leq x \leq \frac{\pi}{2}$.
\end{ejercicio}

\begin{ejercicio}
    Calcular el área de una elipse de semiejes $a$ y $b$.
\end{ejercicio}

\begin{ejercicio}
    Calcular el área comprendida entre las elipses $\frac{x^2}{1} + \frac{y^2}{4}=1$ y $\frac{x^2}{4} + \frac{y^2}{1}=1$
\end{ejercicio}

\begin{ejercicio}
    Calcular el área encerrada por el bucle de la curva que tiene por ecuación $y^2=x(x-1)^2$.
\end{ejercicio}

\begin{ejercicio}
    Dados $a,b>0$, calcular el área encerrada por la curva que tiene por ecuación $x^4-ax^3+by^2=0$.
\end{ejercicio}

\begin{ejercicio}
    Sea $a>0$. Calcular el área comprendida entre el esferoide que tiene por ecuación $y^2(a+x)=x^2(a-x)$, y su asíntota vertical.
\end{ejercicio}

\begin{ejercicio}
    Calcular la longitud de la curva $y(x)=\ln \frac{e^x+1}{e^x-1}$ entre $x=1$ y $x=2$.

    Por lo visto en clase, notando $l$ como la longitud pedida, sabemos que:
    \begin{equation*}
        l=\int_1^2 \sqrt{1+[y'(x)]^2}\;dx
    \end{equation*}


    Calculamos por tanto dicha derivada, que sabemos que es derivable $[1,2]$ por ser continua en dicho intervalo:
    \begin{equation*}
        y'(x) = \frac{e^x-1}{e^x+1}\cdot \frac{e^x(e^x-1) -e^x(e^x+1)}{(e^x-1)^2} = \frac{e^x[e^x-1-(e^x+1)]}{(e^x+1)(e^x-1)}
        = \frac{-2e^x}{e^{2x}-1}
    \end{equation*}

    Por tanto,
    \begin{equation*}\begin{split}
        l&=\int_1^2 \sqrt{1+[y'(x)]^2}\;dx
        = \int_1^2 \sqrt{1+\left[\frac{-2e^x}{e^{2x}-1}\right]^2}\;dx
        = \int_1^2 \sqrt{\frac{(e^{2x}-1)^2 + 4e^{2x}}{(e^{2x}-1)^2}}\;dx =\\
        &= \int_1^2 \sqrt{\frac{(e^{2x}-1)^2 + 4e^{2x}}{(e^{2x}-1)^2}}\;dx
        = \int_1^2 \sqrt{\frac{(e^{2x}+1)^2}{(e^{2x}-1)^2}}\;dx = \ln \frac{e^2-\frac{1}{e^2}}{e-\frac{1}{e}}
    \end{split}\end{equation*}
    
\end{ejercicio}

\begin{ejercicio}
    Calcular la longitud de arco de la curva $y(x)=\sqrt{8}\ln x$ entre los puntos $(1,0)$ y $(8,\sqrt{8} \ln8)$.
\end{ejercicio}

\begin{ejercicio}
    Hallar la longitud de arco de la curva $y(x)=2\sqrt{x}$, entre los puntos $x=1$ y $x=2$.
\end{ejercicio}

\begin{ejercicio}
    Hallar la longitud de arco de la curva $y(x)=\ln{x}$, entre los puntos $x=\sqrt{3}$ y $x=\sqrt{8}$.
\end{ejercicio}

\begin{ejercicio}
    Sea $a>0$. Calcular la longitud del arco de la parábola semicúbica de ecuación $ay^2=x^3$ comprendido entre el origen de coordenadas y el punto $x=5a$.
\end{ejercicio}

\begin{ejercicio}
    Dado $a\in \bb{R}$, calcular la longitud de arco de la curva dada por la ecuación $8a^2y^2=x^2(a^2-2x^2)$.
\end{ejercicio}

\begin{ejercicio}
    Sea $a>0$. Hallar la longitud de arco de la catenaria de ecuación $y(x)=\frac{a}{2}\left(e^{\frac{x}{a}} + e^{-\frac{x}{a}}\right)$ entre el origen de coordenadas y el punto $(x_0,y_0)$.
\end{ejercicio}

\begin{ejercicio}
    Hallar el volumen del cuerpo engendrado al girar sobre el eje $OX$ la superficie limitada por parábola $y=ax-x^2$ (siendo $a>0$) y el eje de abcisas.
\end{ejercicio}

\begin{ejercicio}
    Calcular el volumen del sólido obtenido al girar sobre el eje $OX$, la región limitada por la gráfica de curva $f(x)= \sen x+\cos x$, el eje de abcisas en el intervalo $[0, \pi]$.
\end{ejercicio}

\begin{ejercicio}
    Para cada $a>0$, sea $V(x)$ el volumen del sólido obtenido al girar sobre el eje $OX$ la superficie determinada por la gráfica de la función $y(t)=\frac{\sqrt{t}}{1+t^2}$ cuando $t$ recorre el intervalo $[0,x]$. Determinar el valor de $a>0$ tal que $\displaystyle V(a)=\frac{1}{2}\lim_{x\to \infty}V(x)$.
\end{ejercicio}

\begin{ejercicio}
    Un sólido de revolución está generado por la rotación alrededor del eje $OX$ de la superficie determinada por gráfica de la función positiva $y=f(x)$ sobre el intervalo $[0,a]$. Si para $a>0$ el volumen de dicho sólido es $a^3+a$, ¿Quién es la función $f$?
\end{ejercicio}

\begin{ejercicio}
    Determinar el volumen del sólido que se obtiene al girar alrededor del eje $OY$ la superficie de la región acotada determinada por las parábolas $y=ax^2$ e $y=b-cx^2$, siendo $a,b,c>0$.
\end{ejercicio}

\begin{ejercicio}
    Determinar el volumen del sólido que se obtiene al girar la curva $y^2=8x$, para valores de $x$ en el intervalo $[0,2]$ cuando la curva gira sobre:
    \begin{enumerate}
        \item El eje $OX$,
        \item El eje $OY$,
        \item La recta $x=2$,
    \end{enumerate}
\end{ejercicio}

\begin{ejercicio}
    Determinar el volumen del sólido obtenido al girar alrededor del eje $OX$ la región acotada determinada por las parábolas $y^2=2px$, y $x^2=2py$, siendo $p>0$.
\end{ejercicio}

\begin{ejercicio}
    Calcular el volumen del sólido obtenido al girar alrededor del eje $OX$ la región acotada limitada por las gráficas de las funciones $f(x)=\sen x$ y $g(x)=\cos x$ en el intervalo $\left[0,\frac{\pi}{2}\right]$.
\end{ejercicio}

\begin{ejercicio}
    Determinar el volumen del sólido obtenido al girar la región acotada limitada por las gráficas de las funciones $f(x)=x^2-4x+4$, y $g(x)=4-x$, cuando dicha región gira alrededor de la recta $y=-1$.
\end{ejercicio}

\begin{ejercicio}
    Determinar el volumen del sólido que se obtiene al girar alrededor del eje $OX$ la región que en el primer cuadrante delimitan las curvas $y=\frac{1}{x^2}$ e $y=\sen \left(\frac{\pi x}{2}\right)$, y las rectas $x=0$ e $y=e$.
\end{ejercicio}

\begin{ejercicio}
    Calcular el volumen generado cuando la superficie acotada limitada por la parábola de ecuación $y=4x-x^2$ y el eje $OX$ se hace girar alrededor de la recta $y=6$.
\end{ejercicio}

\begin{ejercicio}
    Calcular el volumen del toro. Esto es el sólido de revolución generado por un círculo de radio $r$ que gira alrededor de un eje situado en el mismo plano del círculo, a una distancia $a$ del centro del círculo, con $a>r$.
\end{ejercicio}

\begin{ejercicio}
    Calcular la superficie de la esfera de radio $r$.
\end{ejercicio}

\begin{ejercicio}
    Calcular el área de la superficie de revolución engendrada al girar la curva $y=\sqrt{1-x^2}$ alrededor del eje $OX$ entre $x=0$ y $x=1$.
\end{ejercicio}

\begin{ejercicio}
    Calcular el área de la superficie de revolución engendrada al girar la curva $8y^2=x^2-x^4$ alrededor del eje $OX$ en el intervalo $[-1,1]$.
\end{ejercicio}

\begin{ejercicio}
    Calcular el área lateral de la superficie de revolución engendrada al girar la curva $y=\frac{e^x + e^{-1}}{2}$ alrededor del eje $OX$ entre $x=-1$ y $x=1$.
\end{ejercicio}

\begin{ejercicio}
    Calcular el área lateral de superficie de revolución engendrada al girar la curva $(x-4)^2 +y^2=1$ alrededor del eje $OX$
\end{ejercicio}

\begin{ejercicio}
    Calcular el área lateral de superficie la superficie de revolución engendrada por la curva $y^2=4x$ cuando gira alrededor del alrededor del eje $OX$ en el intervalo $[0,3]$.
\end{ejercicio}
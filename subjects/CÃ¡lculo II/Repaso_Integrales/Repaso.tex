Este documento ha sido creado con el objetivo de ser un manual de rápida lectura que sirva de recordatorio de cómo se integra una función. En ningún momento pretende ser un manual riguroso de cómo se define la integral, o un recurso relevante de ningún tipo.

\noindent
Recordamos que disponemos de 5 métodos de cálculo de integrales:
\begin{itemize}
    \item Integrales inmediatas.
    \item Integración por partes.
    \item Integración de funciones racionales.
    \item Integración mediante cambio de variable.
    \item Ideas felices para el cambio de variable.
\end{itemize}
Los cuales desarrollaremos a continuación.

\section{Propiedades básicas de la integral}
\begin{prop}[Linealidad de la Integral]
    Sean $f, g$ funciones integrables y $\alpha, \beta \in \mathbb{R}$:
    \begin{equation*}
        \int (\alpha f(x) + \beta g(x))~dx = \alpha \int f(x)~dx  + \beta \int g(x)~dx 
    \end{equation*}
\end{prop}

\begin{prop}[Aditividad de la integral]
    Sea un intervalo $[a, b] \subseteq \mathbb{R}$, $f$ una función integrable y $a < c < b$, entonces:
    \begin{equation*}
        \int_{a}^{b} f(x)~dx = \int_{a}^{c} f(x)~dx + \int_{c}^{d} f(x)~dx 
    \end{equation*}
\end{prop}

\begin{prop}[Regla de Barrow]
    Sea $f$ una función integrable y $F$ una primitiva suya, entonces:
    \begin{equation*}
        \int_{a}^{b} f(x)~dx = F(b) - F(a)
    \end{equation*}
    Que notaremos por:
    \begin{equation*}
        \int_{a}^{b} f(x)~dx = F(b) - F(a) = {[F(x)]}^b_a
    \end{equation*}
\end{prop}

\begin{prop}[Cálculo de primitivas]
    Sea $f$ una función integrable, podemos calcular una primitiva suya de la forma:
    \begin{equation*}
        F(x) = \int f(x)~dx + c \qquad c\in \mathbb{R}
    \end{equation*}
\end{prop}

\section{Integrales inmediatas}
Este método se resume con recordar la siguiente lista
\begin{align*}
        \displaystyle \int x^n~dx &= \frac{x^{n+1}}{n+1} + c \qquad \forall n\in \mathbb{R}\\\\
        \displaystyle \int \frac{1}{x}~dx &= \ln |x| + c \\\\
        \displaystyle\int a^x~dx  &= \dfrac{a^x}{\ln a} + c \\\\
        \displaystyle\int \sen x~dx  &= -\cos x + c \\\\
        \displaystyle\int \tan x~dx  &= -\ln |\cos x| + c \\\\
        \displaystyle\int \cot x~dx  &= \ln|\sen x| + c \\\\
        \displaystyle\int \frac{1}{\cos^2 x}~dx  &= \tan x + c \\\\
        \displaystyle\int \frac{1}{\sen^2 x}~dx &= -\cot x + c \\\\
        \displaystyle \int \frac{1}{\sqrt{1-x^2}}~dx  &= \arcsen x + c \\\\
        \displaystyle \int \frac{-1}{\sqrt{1-x^2}}~dx  &= \arccos x + c \\\\
        \displaystyle \int \frac{1}{1+x^2}~dx  &= \arctan x + c
\end{align*}
Recordando que podemos aplicar la Regla de la cadena al revés donde en vez de $x$ aparezca $f(x)$ y se multiplique por $f'(x)$:
\begin{equation*}
    \displaystyle\int {(f(x))}^n \cdot f'(x) = \frac{{(f(x))}^{n+1}}{n+1}~dx \qquad \forall n\in \mathbb{R}
\end{equation*}

\section{Integración por partes}
\begin{prop}[Integración por partes]
    Dadas dos funciones $u$ y $v$ integrables y derivables, se verifica:
    \begin{equation*}
        \displaystyle\int u(x) v'(x)~dx  = u(x)v(x) - \displaystyle\int v(x)u'(x)~dx 
    \end{equation*}
    Si notamos $du(x) = u'(x)$ y $dv(x) = v'(x)$, llegamos a:
    \begin{equation*}
        \displaystyle\int u(x)dv(x)~dx  = u(x)v(x) - \displaystyle\int v(x)du(x)~dx 
    \end{equation*}
\end{prop}
Que puede recordarse fácilmente mediante la regla mnemotécnica ``Un día vi una vaca sin rabo vestida de uniforme''.\\

\noindent
Este método sirve para resolver integrales de un producto de funciones.\\

\noindent
Destacamos además la regla ALPES\@, que sirve como criterio de selección de la $u$:
\begin{itemize}
    \item Arcosenos, arcocosenos o arcotangentes.
    \item Logaritmos.
    \item Polinomios.
    \item Exponenciales.
    \item Senos, cosenos o tangentes.
\end{itemize}
Trataremos de asignar $u$ a dichas fórmulas, comenzando por encima.

\section{Integración de funciones racionales}
Si queremos calcular la integral indefinida de una función racional (un cociente de polinomios), realizamos el siguiente procedimiento:\\

\noindent
Sean $p(x)$, $q(x)$ polinomios, queremos cacular
\begin{equation*}
    \displaystyle\int \frac{p(x)}{q(x)}~dx 
\end{equation*}
Para ello, si $\deg(p) \geq \deg(q)$, aplicamos la fórmula de la división, obteniendo dos polinomios $t(x)$ y $r(x)$ de forma que:
\begin{equation*}
    p(x) = t(x)q(x) + r(x)
\end{equation*}
de donde:
\begin{equation*}
    \displaystyle\int \frac{p(x)}{q(x)}~dx  = \displaystyle\int \frac{t(x)q(x)+r(x)}{q(x)}~dx = \displaystyle\int t(x)~dx  + \displaystyle\int \frac{r(x)}{q(x)}~dx 
\end{equation*}
A continuación, descompondremos $q(x)$ en raíces simples, para realizar la siguiente descomposición de $\frac{r(x)}{q(x)}$. Debido a la complejidad de este desarrollo, supondremos el caso concreto en el que $q(x) = (x-1){(x-2)}^2$, para ver cómo proceder.

Sean $A,B,C\in \mathbb{R}$, buscamos calcular sus valores exigiendo:
\begin{equation*}
    \dfrac{r(x)}{q(x)} = \dfrac{A}{x-1} + \dfrac{B}{x-2} + \dfrac{C}{{(x-2)}^{2}}
\end{equation*}
Notemos que si tiene una raíz con multiplicidad mayor que 1, repetimos dicho denominador tantas veces como multiplicidad tenga, aumentando en 1 el exponente.

Para resolverlo, usamos que:
\begin{equation*}
    r(x) = A{(x-2)}^{2} + B(x-1)(x-2) + C(x-1)
\end{equation*}
e igualamos los coeficientes del mismo grado o le damos el valor de las raíces.\\

\noindent
De esta forma, hemos llegado a que:
\begin{align*}
    \displaystyle\int \frac{p(x)}{q(x)}~dx  = \displaystyle\int t(x)~dx  &+ \displaystyle\int \frac{r(x)}{q(x)}~dx  \\
                                            &+ A \displaystyle\int \dfrac{1}{x-1}~dx  + B\displaystyle\int \dfrac{1}{x-2}~dx  + C\displaystyle\int \dfrac{1}{{(x-2)}^{2}}~dx \\
                                            &+ A\ln|x-1| + B\ln|x-2| -\frac{C}{x-2} + c
\end{align*}

\subsection{Raíces complejas}
Si el polinomio denominador $q(x)$ tiene alguna raíz compleja, la forma en la que expresamos $\nicefrac{r(x)}{q(x)}$ cambia. Incluimos el siguiente ejemplo como ilustración de qué procedimiento seguir en dicho caso:
\begin{ejemplo}
    Calcular
    \begin{equation*}
        \displaystyle\int \dfrac{x+1}{x^4-1}~dx 
    \end{equation*}
    Como no podemos dividir el numerador entre el denominador, factorizamos el denominador:
    \begin{equation*}
        x^4 -1 = (x+1)(x-1)(x^2+1)
    \end{equation*}
    de donde:
    \begin{equation*}
        \displaystyle\int \dfrac{x+1}{x^4-1}~dx = \displaystyle\int \dfrac{\cancel{x+1}}{(\cancel{x+1})(x-1)(x^2+1)}~dx = \displaystyle\int \dfrac{1}{(x-1)(x^2+1)}~dx 
    \end{equation*}
    Sean $A,B,C\in \mathbb{R}$, buscamos:
    \begin{equation*}
        \dfrac{r(x)}{q(x)} = \dfrac{A}{x-1}+ \dfrac{Bx+C}{x^2+1} \Longrightarrow r(x) = A(x^2+1) + (Bx+C)(x-1)
    \end{equation*}
    \begin{align*}
        1 &= A(x^2+1) + (Bx+C)(x-1) = A(x^2 + 1)+Bx^2-Bx + Cx - C \\
          &= x^2 (A + B) + x(C-B) + A - C
    \end{align*}
    de donde:
    \begin{align*}
        1 &= A - C \Longrightarrow A = 1 + C \\
        0 &= C - B \Longrightarrow C = B \\
        0 &= A+B \Longrightarrow A = -B = -C
    \end{align*}
    Luego $A = 1 - A \Longrightarrow 2A = 1 \Longrightarrow A = \nicefrac{1}{2}$.
    \begin{equation*}
        B = C = A - 1 = -\nicefrac{1}{2} \\
    \end{equation*}
    por lo que tenemos:
    \begin{equation*}
        \dfrac{r(x)}{q(x)} = \dfrac{1}{2(x-1)} -\dfrac{1}{2}\dfrac{x+1}{x^2+1} = \dfrac{1}{2(x-1)} - \dfrac{1}{2}\dfrac{x}{x^2+1}-\dfrac{1}{2}\dfrac{1}{x^2+1}
    \end{equation*}
    \begin{align*}
        \displaystyle\int \dfrac{x+1}{x^4-1}~dx &= \displaystyle\int \dfrac{1}{(x-1)(x^2+1)}~dx \\
        &= \dfrac{1}{2}\displaystyle\int \dfrac{1}{x-1}~dx -\dfrac{1}{2}\displaystyle\int \dfrac{x}{x^2+1}~dx -\dfrac{1}{2}\displaystyle\int \dfrac{1}{x^2+1}~dx \\
        &= \dfrac{1}{2}\ln|x-1| -\dfrac{\ln|x^2+1|}{4} -\dfrac{\arctan x}{2} + c
    \end{align*}
\end{ejemplo}

\section{Integración mediante cambio de variable}
El método de cambio de variable nos dice que si queremos resolver una primitiva:
\begin{equation*}
    \displaystyle\int f(x)~dx  
\end{equation*}
Podemos cambiar la variable $x$ por cualquier otra, que esté en función de otra llamada por ejemplo $t$: $x = g(t)$. De esta forma, para realizar el cambio debemos expresar: $dx = g'(t)dt$ y obtememos que:
\begin{equation*}
    \displaystyle\int f(x)~dx  = \displaystyle\int f(g(t))g'(t)dt = F(t)
\end{equation*}
Que probablemente sea una integral más fácil de resolver (si hemos hecho bien el cambio). Una vez calculada la integral, aplicamos $g^{-1}$ (en caso de existir, luego hemos tenido que escoger bien la $g$) en ambos lados, deshaciendo el cambio y obteniendo $g^{-1}(x) = t$, con lo que:
\begin{equation*}
    \displaystyle\int f(x)~dx  = \displaystyle\int f(g(t))g'(t)dt = F(t) = F(g^{-1}(x))
\end{equation*}

\begin{ejemplo}
    Por ejemplo, queremos calcular:
    \begin{equation*}
        \displaystyle\int \dfrac{\tan(\ln x)}{x}~dx 
    \end{equation*}
    Para ello, hacemos $\ln x = t$ y derivando en ambos lados: $(\nicefrac{1}{x})dx = dt \Longrightarrow dx = xdt$
    \begin{equation*}
        \displaystyle\int \dfrac{\tan(\ln x)}{x}~dx = \displaystyle\int \dfrac{\tan(t)\cancel{x}}{\cancel{x}}~dt = \displaystyle\int \tan(t)~dt = -\ln |\cos t| + c
    \end{equation*}
    Deshaciendo el cambio de variable, $\ln x = t \Longrightarrow x = e^t$.
    \begin{equation*}
        \displaystyle\int \dfrac{\tan(\ln x)}{x}~dx = -\ln |\cos e^x| + c
    \end{equation*}
\end{ejemplo}

\subsection{Límites de integración}
Además de para el cálculo de primitivas, podemos aplicar el método de cambio de variable cuando nos dispongamos a calcular integrales definidas, entendiendo que si el cambio que vamos a realizar es: $x = g(t)$, se debe cumplir la siguiente fórmula:
\begin{equation*}
    \displaystyle\int_{g(a)}^{g(b)} f(x)~dx  = \displaystyle\int_{a}^{b} f(g(t))g'(t)~dt  \qquad a,b\in \bb{R}
\end{equation*}
Cambiando los límites de integración no nos es necesario luego deshacer el cambio de variable realizado.
\begin{ejemplo}
    Calcule:
    \begin{equation*}
        \displaystyle\int_{0}^{4} 2xe^{x^2}~dx 
    \end{equation*}
    \begin{align*}
        \displaystyle\int_{0}^{4} 2xe^{x^2}~dx &= \MetInt{t=x^2\Longrightarrow x=\sqrt{t}\qquad   dt = 2xdx}{\sqrt{t}=0\text{\ para\ }t=0\qquad \sqrt{t}=4\text{\ para\ }t=16} \\
                                               &= \displaystyle\int_{0}^{16} e^t~dt  = {\left[e^t\right]}_0^{16} = e^{16} - e^0 = e^{16}- 1
    \end{align*}

\end{ejemplo}

\section{Ideas felices para los cambios de variable}
Hay situaciones en las que usualmente es fácil aplicar un cambio de variable ingenioso para calcular integrales. Ciertas personas son capaces de averiguar estos cambios por su cuenta, mientras que otros no. Es el segundo motivo por el que existe esta sección, para traernos al resto de mortales dichos cambios de forma mecánica.

\subsection{Función impar en el seno}
\begin{definicion}[Función impar en el seno]
    Una función $f$ es impar en el seno si al sustituir todas las apariencias de $\sen x$ en su expresión por $-\sen x$, obtenemos $-f$.
\end{definicion}
Si queremos calcular una primitiva de una función impar en el seno, realizaremos el cambio de variable:
\begin{equation*}
    t = \cos x
\end{equation*}

\begin{ejemplo}
    Calcular
    \begin{equation*}
        \displaystyle\int \sen^3 x \cos^4 x~dx 
    \end{equation*}
    Como la función a integrar es impar en el seno, realizamos el cambio $t = \cos x$
    \begin{equation*}
        dt = -\sen x dx \Longrightarrow dx = \dfrac{dt}{-\sen x}
    \end{equation*}
    \begin{align*}
        \displaystyle\int \sen^3x\cos^4x~dx  &= \displaystyle\int \sen^3x\cdot t^4 \dfrac{dx}{-\sen x}~dt = \displaystyle\int -\sen^2 x \cdot t^4~dt  \\
                                             &= \displaystyle\int (\cos^2 x - 1)t^4~dt = \displaystyle\int (t^2-1)t^4~dt = \displaystyle\int (t^6 - t^4)~dt \\
                                             &= \dfrac{t^7}{7}-\dfrac{t^5}{5} + c =  \dfrac{\cos^7 x}{7}-\dfrac{\cos^5x}{5} + c
    \end{align*}
\end{ejemplo}

\subsection{Función impar en el coseno}
\begin{definicion}[Función impar en el coseno]
    Una función $f$ es impar en el seno si al sustituir todas las apariencias de $\cos x$ en su expresión por $-\cos x$, obtenemos $-f$.
\end{definicion}
Si queremos calcular una primitiva de una función impar en el seno, realizaremos el cambio de variable:
\begin{equation*}
    t = \sen x
\end{equation*}

\begin{ejemplo}
    Calcular
    \begin{equation*}
        \displaystyle\int \sen^4x\cos^5x~dx 
    \end{equation*}
    Como la función a integrar es impar en el coseno, realizamos el cambio $t = \sen x$
    \begin{equation*}
        dt = \cos x dx \Longrightarrow dx = \dfrac{dt}{\cos x}
    \end{equation*}
    \begin{align*}
        \displaystyle\int \sen^4x\cos^5x~dx &= \displaystyle\int t^4 \dfrac{\cos^5x}{\cos x}~dt  = \displaystyle\int t^4 \cos^4 x~dt = \displaystyle\int t^4 {(\cos^2 x)}^{2}~dt \\
                                            &= \displaystyle\int t^4 {(1-\sen^2x)}^{2}~dt = \displaystyle\int t^4 {(1-t^2)}^{2}~dt  = \displaystyle\int t^4(1+t^4-2t^2)~dt \\
                                            &= \displaystyle\int (t^4 + t^8 -2t^6)~dt = \dfrac{t^5}{5}+\dfrac{t^9}{9}-\dfrac{2t^7}{7} +c\\
                                            &= \dfrac{\sen^5x}{5}+\dfrac{\sen^9x}{9}-\dfrac{2\cdot \sen^7x}{7}+c
    \end{align*}
\end{ejemplo}

\subsection{Función par en el producto seno por coseno}
\begin{definicion}[Función par en el producto seno por coseno]
    Una función $f$ es par en el producto seno por coseno si al sustituir todas las apariencias de $\sen x$ por $-\sen x$ y $\cos x$ por $-\cos x$, obtenemos $f$.
\end{definicion}
Si queremos calcular una primitiva de una función impar en el seno, realizaremos el cambio de variable:
\begin{equation*}
    t = \tan x
\end{equation*}
Obteniendo por tanto:
\begin{equation*}
    t = \tan x \Longrightarrow \arctan t = x \Longrightarrow \dfrac{dt}{1+t^2} = dx
\end{equation*}
Además, obtendremos:
\begin{gather*}
    \sen x = \dfrac{t}{\sqrt{1+t^2}} \\
    \cos x = \dfrac{1}{\sqrt{1+t^2}}
\end{gather*}

\begin{ejemplo}
    Calcular
    \begin{equation*}
        \displaystyle\int \sen x\cos^3x~dx 
    \end{equation*}
    Como la función a integrar es par en el producto seno por coseno, realizamos el cambio $t=\tan x$:
    \begin{align*}
        \displaystyle\int \sen x\cos^3x~dx &= \displaystyle\int \dfrac{t}{\sqrt{1+t^2}}{\left(\dfrac{1}{\sqrt{1+t^2}}\right)}^{3}\dfrac{dt}{1+t^2} \\
                                           &= \displaystyle\int \dfrac{t}{{(1+t^2)}^{3}}~dt  = \dfrac{-1}{4}\dfrac{1}{{(1+t^2)}^{2}} + c = \dfrac{-1}{4{(1+\tan^2x)}^{2}} + c
    \end{align*}
Notemos que la función también era impar en el seno y en el coseno, luego podríamos haber aplicado en su lugar alguno de los cambios de variable ya vistos.
\end{ejemplo}

\subsection{Función con varias raíces del mismo radicando}
Si queremos calcular la primitiva de una función $f$ que tiene varias raíces de un mismo radicando (por ejemplo $x$) de distintos índices (por ejemplo, 2, 3, \ldots), podemos realizar el siguiente cambio de variable:
\begin{equation*}
    radicando = t^m
\end{equation*}
siendo $m$ el mínimo común múltiplo de los índices de las raíces.

\begin{ejemplo}
    Calcular
    \begin{equation*}
        \displaystyle\int \dfrac{1-\sqrt{x}}{\sqrt[3]{x}}~dx 
    \end{equation*}
    \begin{itemize}
        \item Radicando: $x$.
        \item Índices de las raíces: 2 y 3.
        \item mcm$(2,3)=6$.
    \end{itemize}
    Realizamos el cambio de variable $x = t^6 \Longrightarrow dx = 6t^5dt$:
    \begin{align*}
        \displaystyle\int \dfrac{1-\sqrt{x}}{\sqrt[3]{x}}~dx &= \displaystyle\int \dfrac{1-t^{\frac{6}{2}}}{t^{\frac{6}{3}}}6t^5~dt = 6\displaystyle\int \dfrac{1-t^3}{t^2}t^5~dt \\
                                                             &= 6\displaystyle\int (1-t^3)t^3~dt = 6\displaystyle\int (t^3 - t^6)~dt \\
                                                             &= \dfrac{6}{4}t^4 - \dfrac{6}{7}t^7 + c = \dfrac{3}{2}\sqrt[3]{x^2}- \dfrac{6}{7}x\sqrt[6]{x} + c
    \end{align*}
\end{ejemplo}

\subsection{Función con exponenciales}
Si queremos calcular la primitiva de una función $f$ en la que aparece una o varias veces $e^x$, suele ser de utilidad realizar el cambio de variable $t=e^x$.

\subsection{Funciones con cocientes}
Los siguientes cambios de variable suelen ser útiles:
\begin{itemize}
    \item Si $f$ es de la forma:
        \begin{equation*}
            f(x) = \dfrac{x}{\sqrt{a^2 - x^2}}\qquad a\in \mathbb{R}
        \end{equation*}
        Aplicar el cambio $x = a\cdot \sen t$.
    \item Si $f$ es de la forma:
        \begin{equation*}
            f(x) = \dfrac{x}{\sqrt{a^2 + x^2}}\qquad a\in \mathbb{R}
        \end{equation*}
        Aplicar el cambio $x = a\cdot \tg t$.
    \item Si $f$ es de la forma:
        \begin{equation*}
            f(x) = \dfrac{x}{\sqrt{x^2 - a^2}}\qquad a\in \mathbb{R}
        \end{equation*}
        Aplicar el cambio $x=a\cdot \sec t$.
    \item Si $f$ es de la forma:
        \begin{equation*}
            f(x) = \dfrac{x}{\sqrt[n]{\dfrac{ax+b}{cx+d}}}\qquad a,b,c,d,n\in \mathbb{R}
        \end{equation*}
        Aplicar el cambio:
        \begin{equation*}
            \dfrac{ax+b}{cx+d} = t^n
        \end{equation*}
\end{itemize}
En estos casos superiores, no es necesario que $f$ sea estrictamente lo que aparece, sino algo similar. Por ejemplo, ante la integral:
\begin{equation*}
    \displaystyle\int \dfrac{x^4}{\sqrt{{\left(1-x^2\right)}^{3}}}~dx 
\end{equation*}
Podríamos aplicar el cambio $x = \sen t$.

\subsection{Funciones con funciones trigonométricas}
Si no podemos aplicar ninguno de los cambios de variable ya vistos para funciones trigonométricas, podemos aplicar como último recurso el cambio:
\begin{equation*}
    t = \tg\left(\dfrac{x}{2}\right)
\end{equation*}
\begin{equation*}
    \arctan t = \dfrac{x}{2} \Longrightarrow dx = \dfrac{2}{1+t^2}dt
\end{equation*}

\begin{ejemplo}
    Calcular
    \begin{equation*}
        \displaystyle\int \dfrac{1}{1+\sen x}~dx 
    \end{equation*}
    Aplicamos el cambio $t = \tg\left(\nicefrac{x}{2}\right)$:
    \begin{gather*}
        dx = \dfrac{2}{1+t^2}dt\\
        \sen x = \dfrac{2t}{1+t^2}
    \end{gather*}
    \begin{align*}
        \displaystyle\int \dfrac{1}{1+\sen x}~dx &= \displaystyle\int \dfrac{1}{1+\dfrac{2t}{1+t^2}}\dfrac{2}{1+t^2}~dt = 2\displaystyle\int \dfrac{1}{{(1+t)}^{2}}~dt = \dfrac{-2}{1+t}+c \\
        &= \dfrac{-2}{1+\tg\left(\nicefrac{x}{2}\right)} + c
    \end{align*}
\end{ejemplo}

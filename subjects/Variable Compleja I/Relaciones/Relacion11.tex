\section{Comportamiento local de una función holomorfa}

\begin{ejercicio}
    Sea $R\in \bb{R}^+$ y $f\in \cc{H}(D(0,R))$, no constante. Probar que la función $M$ definida por
    \Func{M}{\left]0,R\right[}{\bb{R}}{r}{\max\left\{ |f(z)| : z\in C(0,r)^* \right\}}
    es estrictamente creciente.
\end{ejercicio}

\begin{ejercicio}
    Sea $W = \{z\in \bb{C} : |z| > 1\}$ y $f : \ol{\Omega} \to \bb{C}$ una función continua en $\ol{\Omega}$ y holomorfa en $\Omega$, que tenga límite en $\infty$. Probar que $|f|$ tiene un máximo absoluto en un punto de $\bb{T}$. Suponiendo que $f$ no es constante, probar que la función $M$ definida por
    \Func{M}{\left[1,+\infty\right[}{\bb{R}}{r}{\max\left\{ |f(z)| : z\in C(0,r)^* \right\}}
    es estrictamente decreciente.
\end{ejercicio}

\begin{ejercicio}
    Sea $P$ un polinomio de grado $n\in \bb{N}$ y $M = \max\left\{ |P(z)| : z\in \bb{T} \right\}$. Probar que:
    \begin{equation*}
        z\in \bb{C}, |z| > 1 \implies |P(z)| \leq M |z|^n.
    \end{equation*}
\end{ejercicio}

\begin{ejercicio}
    Sea $f : \ol{D}(0,1) \to \bb{C}$ una función continua en $\ol{D}(0,1)$ y holomorfa en $D(0,1)$, que verifica la siguiente condición:
    \begin{equation*}
        z\in \bb{T} \implies |f(z)| \leq
        \begin{cases}
            1 & \text{si } \Re z \leq 0,\\
            2 & \text{si } \Re z > 0.
        \end{cases}
    \end{equation*}
    Probar que $|f(0)| \leq \sqrt{2}$.
\end{ejercicio}

\begin{ejercicio}
    Sea $f\in \cc{H}(D(0,1))$ y supongamos que existe $n\in \bb{N}$ tal que, para todo $r\in \left]0,1\right[$ se tiene
    \begin{equation*}
        \max\left\{ |f(z)| : |z| = r \right\} = r^n.
    \end{equation*}
    Probar que existe $a\in \bb{T}$ tal que $f(z) = az^n$ para todo $z\in D(0,1)$.
\end{ejercicio}

\begin{ejercicio}
    Sea $f\in \cc{H}(D(0,1))$ verificando que
    \begin{equation*}
        |f(z)| \leq |f(z^2)| \quad \forall z\in D(0,1).
    \end{equation*}
    Probar que $f$ es constante.
\end{ejercicio}

\begin{ejercicio}
    Mostrar que el teorema fundamental del Álgebra se deduce directamente del principio del módulo mínimo.
\end{ejercicio}

\begin{ejercicio}
    Sea $\Omega$ un dominio y $f\in \cc{H}(\Omega)$. Probar que, si la función $\Re f$ tiene un extremo relativo en un punto de $\Omega$, entonces $f$ es constante.
\end{ejercicio}

\begin{ejercicio}
    Sea $f : \ol{D}(0,1) \to \bb{C}$ una función continua en $\ol{D}(0,1)$ y holomorfa en $D(0,1)$, tal que $\Im f(z) = 0$ para todo $z\in \bb{T}$. Probar que $f$ es constante.
\end{ejercicio}

\begin{ejercicio}
    Sea $\Omega$ un abierto de $\bb{C}$ y $f\in \cc{H}(\Omega)$ una función inyectiva. Dados $a\in \Omega$ y $r\in \bb{R}^+$ tales que $\ol{D}(a,r) \subseteq \Omega$, calcular, para cada $z\in \ol{D}(a,r)$, la integral
    \begin{equation*}
        \int_{C(a,r)} \frac{w f'(w)}{f(w) - f(z)} \, dw.
    \end{equation*}
\end{ejercicio}
\section{Comportamiento local de una función holomorfa}

\begin{ejercicio}
    Sea $R\in \bb{R}^+$ y $f\in \cc{H}(D(0,R))$, no constante. Probar que la función $M$ definida por
    \Func{M}{\left]0,R\right[}{\bb{R}}{r}{\max\left\{ |f(z)| : z\in C(0,r)^* \right\}}
    es estrictamente creciente.\\

    Supongamos que $M$ no es estrictamente creciente en $\left]0,R\right[$. Por tanto, existen $0 < r_1 < r_2 < R$ tales que $M(r_2)\leq M(r_1)$. Esto es:
    \begin{equation*}
        \max\left\{ |f(z)| : z\in C(0,r_2)^* \right\} \leq \max\left\{ |f(z)| : z\in C(0,r_1)^* \right\}.
    \end{equation*}

    Para cada $i\in \{1,2\}$, como $D(0,r_i)$ es un dominio acotado, $f$ es continua en $\ol{D}(0,r_i)$ y holomorfa en $D(0,r_i)$, y $\Fr(D(0,r_i)) = C(0,r_i)^*$, por el corolario del principio del módulo máximo, se tiene que:
    \begin{align*}
        \max\left\{ |f(z)| : z\in C(0,r_i)^* \right\} &= \max\left\{ |f(z)| : z\in \ol{D}(0,r_i) \right\}
    \end{align*}

    Por tanto, se tiene que:
    \begin{equation*}
        \max\left\{ |f(z)| : z\in \ol{D}(0,r_2) \right\} \leq \max\left\{ |f(z)| : z\in \ol{D}(0,r_1) \right\}.
    \end{equation*}

    Por otro lado, como $\ol{D}(0,r_1) \subseteq \ol{D}(0,r_2)$, se tiene trivialmente que:
    \begin{equation*}
        \max\left\{ |f(z)| : z\in \ol{D}(0,r_2) \right\} \geq \max\left\{ |f(z)| : z\in \ol{D}(0,r_1) \right\}.
    \end{equation*}

    Uniendo ambas desigualdades, se obtiene que:
    \begin{equation*}
        \max\left\{ |f(z)| : z\in \ol{D}(0,r_2) \right\} = \max\left\{ |f(z)| : z\in \ol{D}(0,r_1) \right\}.
    \end{equation*}

    Por tanto, $|f|$ alcanza su máximo absoluto en un punto $z_0\in \ol{D}(0,r_1)$. Por tanto, considerando la restricción de $f$ a $D(0,r_2)$, por el principio del módulo máximo, se tiene que $f_{\big|D(0,r_2)}$ es constante. Como este conjunto no es numerable y $D(0,R)$ es un dominio, por el Principio de Identidad, se tiene que $f$ es constante en $D(0,R)$. No obstante, como $f$ no es constante, se llega a una contradicción. Por tanto, $M$ es estrictamente creciente en $\left]0,R\right[$.
\end{ejercicio}

\begin{ejercicio}
    Sea $\Omega = \{z\in \bb{C} : |z| > 1\}$ y $f : \ol{\Omega} \to \bb{C}$ una función continua en $\ol{\Omega}$ y holomorfa en $\Omega$, que tenga límite en $\infty$. Probar que $|f|$ tiene un máximo absoluto en un punto de $\bb{T}$. Suponiendo que $f$ no es constante, probar que la función $M$ definida por
    \Func{M}{\left[1,+\infty\right[}{\bb{R}}{r}{\max\left\{ |f(z)| : z\in C(0,r)^* \right\}}
    es estrictamente decreciente.\\

    Sea $\Omega^*$ el siguiente conjunto:
    \begin{equation*}
        \Omega^* = \left\{ z\in \bb{C} : |z| < 1 \right\} = D(0,1).
    \end{equation*}

    Calculamos previamente el siguiente límite:
    \begin{equation*}
        \lim_{z\to 0} f\left(\frac{1}{z}\right) = \lim_{w\to \infty} f(w) = L \in \bb{C}.
    \end{equation*}

    Definimos por tanto la siguiente función:
    \Func{g}{\ol{\Omega^*}}{\bb{C}}{z}{\begin{cases}
        f\left(\frac{1}{z}\right) & \text{si } z\neq 0,\\
        L & \text{si } z = 0.
    \end{cases}}

    Veamos en primer lugar que $g$ está bien definida; es decir, que para cada $z\in \ol{\Omega^*}\setminus \{0\}$, $\nicefrac{1}{z}\in \ol{\Omega}$. Sea $z\in \ol{\Omega^*}$. Entonces, $|\nicefrac{1}{z}| = \frac{1}{|z|}\geq 1$, luego $\nicefrac{1}{z}\in \ol{\Omega}$. Por tanto, $g$ está bien definida.\\

    Sabemos directamente que $g$ es continua en $\ol{\Omega^*}$ y holomorfa en $\Omega^*\setminus \{0\}$. Por el Teorema de Extensión de Riemman, $g\in \cc{H}(\Omega^*)$. Por tanto, $\Omega^*$ es un dominio y $g$ es continua en $\ol{\Omega^*}$ y holomorfa en $\Omega^*$. Por tanto, por el Teorema del Módulo Máximo, $|g|$ alcanza su máximo absoluto en un punto de $\bb{T}$. De esta forma:
    \begin{align*}
        \max\left\{ |g(z)| : z\in \ol{D}(0,1) \right\} &= \max\left\{ |g(z)| : z\in \bb{T} \right\}
    \end{align*}

    Traduzcamos ambos conjuntos a la función $f$. Para ello, para cada $z\in \ol{\Omega}$, consideramos el punto $w = \nicefrac{1}{z}\in \ol{\Omega^*}$. Entonces, se tiene que:
    \begin{align*}
        \max\left\{ |f(z)| : z\in \bb{C},\ |z|\geq 1 \right\} &= \max\left\{ |g(w)| : w=\nicefrac{1}{z},\ z\in \bb{C},\ |z|\geq 1 \right\} \\
        &= \max\left\{ |g(w)| : w\in \ol{D}(0,1)\setminus \{0\} \right\} \\
        &\AstIg \max\left\{ |g(w)| : w\in \bb{T} \right\}\\
        &= \max\left\{ |f(z)| : z=\nicefrac{1}{w},\ w\in \bb{T} \right\} \\
        &= \max\left\{ |f(z)| : z\in \bb{T} \right\}.
    \end{align*}
    donde en $(\ast)$ hemos empleado lo previamente demostrado, que $|g|$ alcanza su máximo absoluto en un punto de $\bb{T}$. Por tanto, se ha probado que $|f|$ alcanza su máximo absoluto en un punto de $\bb{T}$.\\

    Supongamos ahora que $f$ no es estrictamente decreciente. Entonces, existe $r\in \bb{R}$ tal que $1 < r$ con $M(1) \leq M(r)$. Esto es:
    \begin{equation*}
        \max\left\{ |f(z)| : z\in C(0,1)^* \right\} \leq \max\left\{ |f(z)| : z\in C(0,r)^* \right\}.
    \end{equation*}

    Por lo demostrado en la primera parte, se tiene que:
    \begin{equation*}
        \max\left\{ |f(z)| : z\in C(0,1)^* \right\} = \max\left\{ |f(z)| : z\in \bb{C}, |z|\geq 1 \right\}.
    \end{equation*}

    Por tanto, se tiene que:
    \begin{equation*}
        \max\left\{ |f(z)| : z\in \bb{C}, |z|\geq 1 \right\} \leq \max\left\{ |f(z)| : z\in C(0,r)^* \right\}.
    \end{equation*}
    
    No obstante, como $C(0,r)^* \subseteq \{z\in \bb{C} : |z|\geq 1\}$, se tiene que:
    \begin{equation*}
        \max\left\{ |f(z)| : z\in C(0,r)^* \right\} \leq \max\left\{ |f(z)| : z\in \bb{C}\ |z|\geq 1 \right\}.
    \end{equation*}

    Uniendo ambas desigualdades, se obtiene que:
    \begin{equation*}
        \max\left\{ |f(z)| : z\in \bb{C}\ |z|\geq 1 \right\} = \max\left\{ |f(z)| : z\in C(0,r)^* \right\}.
    \end{equation*}

    Por tanto, $|f|$ alcanza su máximo absoluto en un punto de $C(0,r)^*$. Por el principio del módulo máximo, $f_{\big|\Omega}$ es constante. Por continuidad, $f$ es constante en $\ol{\Omega}$. No obstante, como $f$ no es constante, se llega a una contradicción. Por tanto, $M$ es estrictamente decreciente en $\left[1,+\infty\right[$.
\end{ejercicio}

\begin{ejercicio}
    Sea $P$ un polinomio de grado $n\in \bb{N}$ y $M = \max\left\{ |P(z)| : z\in \bb{T} \right\}$. Probar que:
    \begin{equation*}
        z\in \bb{C}, |z| > 1 \implies |P(z)| \leq M |z|^n.
    \end{equation*}

    \begin{comment}
    Sea $P$ un polinomio de grado $n\in \bb{N}$, no constante. Entonces, se puede escribir como:
    \begin{equation*}
        P(z) = a_n z^n + a_{n-1} z^{n-1} + \ldots + a_0 = \sum_{k=0}^n a_k z^k.
    \end{equation*}

    Aplicamos la propiedad de la media. Para todo $r\in \bb{R}^+$, se tiene que:
    \begin{align*}
        P(z) &= \frac{1}{2\pi} \int_{-\pi}^{\pi} P\left( r e^{i\theta} \right) \, d\theta \\
        &= \frac{1}{2\pi} \int_{-\pi}^{\pi} \sum_{k=0}^n a_k \left( r e^{i\theta} \right)^k \, d\theta \\
        &= \frac{1}{2\pi} \int_{-\pi}^{\pi} \sum_{k=0}^n a_k r^k \left(e^{i\theta} \right)^k \, d\theta
    \end{align*}

    Por tanto:
    \begin{align*}
        |P(z)| & \leq \frac{1}{2\pi} \int_{-\pi}^{\pi} \sum_{k=0}^n |a_k| |r^k| \left|e^{i\theta} \right|^k  \, d\theta \\
        &\leq \frac{1}{2\pi} \int_{-\pi}^{\pi} \sum_{k=0}^n |a_k| |r^k| \left|e^{i\theta} \right|^k  \, d\theta \\
    \end{align*}
    \end{comment}
\end{ejercicio}

\begin{ejercicio}
    Sea $f : \ol{D}(0,1) \to \bb{C}$ una función continua en $\ol{D}(0,1)$ y holomorfa en $D(0,1)$, que verifica la siguiente condición:
    \begin{equation*}
        z\in \bb{T} \implies |f(z)| \leq
        \begin{cases}
            1 & \text{si } \Re z \leq 0,\\
            2 & \text{si } \Re z > 0.
        \end{cases}
    \end{equation*}
    Probar que $|f(0)| \leq \sqrt{2}$.
\end{ejercicio}

\begin{ejercicio}
    Sea $f\in \cc{H}(D(0,1))$ y supongamos que existe $n\in \bb{N}$ tal que, para todo $r\in \left]0,1\right[$ se tiene
    \begin{equation*}
        \max\left\{ |f(z)| : |z| = r \right\} = r^n.
    \end{equation*}
    Probar que existe $a\in \bb{T}$ tal que $f(z) = az^n$ para todo $z\in D(0,1)$.\\

    Definimos la siguiente función:
    \Func{g}{D(0,1)\setminus \{0\}}{\bb{C}}{z}{\frac{f(z)}{z^n}}.

    Para cada $r\in \left]0,1\right[$, se tiene que:
    \begin{equation*}
        \max\left\{ |g(z)| : |z| = r \right\} = \max\left\{ \left| \frac{f(z)}{z^n} \right| : |z| = r \right\} = \dfrac{\max\left\{ |f(z)| : |z| = r \right\}}{r^n} = \dfrac{r^n}{r^n} = 1.
    \end{equation*}

    Por tanto, haciendo ahora el máximo en $r\in \left]0,1\right[$, se tiene que:
    \begin{align*}
        \max\left\{ |g(z)| : z\in D(0,1)\setminus \{0\} \right\} &= \max\left\{ |g(z)| : |z| = r,\ r\in \left]0,1\right[ \right\} = 1
    \end{align*}

    Por tanto, el máximo absoluto se alcanza en toda circunferecia de radio $r\in \left]0,1\right[$. En particular, se alcanza en un punto $z_0\in D(0,1)\setminus \{0\}$. Por el principio del módulo máximo, $g$ es constante en $D(0,1)\setminus \{0\}$. Por tanto, $\exists \alpha\in \bb{C}$ tal que $g(z) = \alpha$ para todo $z\in D(0,1)\setminus \{0\}$. Por tanto, se tiene que:
    \begin{equation*}
        f(z) = \alpha z^n \quad \forall z\in D(0,1)\setminus \{0\}.
    \end{equation*}

    Evaluando en $z = 0$ también se tiene la igualdad, luego:
    \begin{equation*}
        f(z) = \alpha z^n \quad \forall z\in D(0,1).
    \end{equation*}
\end{ejercicio}

\begin{ejercicio}
    Sea $f\in \cc{H}(D(0,1))$ verificando que
    \begin{equation*}
        |f(z)| \leq |f(z^2)| \quad \forall z\in D(0,1).
    \end{equation*}
    Probar que $f$ es constante.
    \begin{description}
        \item[Opción 1:] Sea $r\in \left]0,1\right[$. Entonces, tenemos que:
        \begin{align*}
            \max\left\{ |f(z)| : |z| = r \right\} &= \max\left\{ |f(z)| : |z|\leq r \right\}
        \end{align*}

        Sea $z\in C(0,r)^*$ tal que $|f(z)| = \max\left\{ |f(w)| : |w| = r \right\}$. Consideramos el punto $z^2$. Como $|z|^2 = r^2 < r$, se tiene que:
        \begin{align*}
            |f(z^2)| \geq |f(z)| = \max\left\{ |f(w)| : |w| = r \right\}
            = \max\left\{ |f(w)| : |w|\leq r \right\}.
        \end{align*}

        Por tanto, en $z^2$ se alcanza un máximo relativo de $|f|$ en $D(0,1)$. Por el principio del módulo máximo, $f$ es constante en $D(0,1)$.

        \item[Opción 2:] Fijado $z\in D(0,1)$, consideramos la sucesión de puntos $\{z_n\}=\{z^{2^n}\}$. Tomando límites con $n\to \infty$, se tiene que:
        \begin{equation*}
            \{|z_n|\} = \{ |z|^{2^n} \} \to 0
            \Longrightarrow
            \{z_n\} \to 0.
        \end{equation*}

        La condición dada implica que:
        \begin{align*}
            |f(z_n)| = |f(z^{2^n})| \leq |f(z^{2^{n+1}})| = |f(z_{n+1})| \quad \forall n\in \bb{N}.
        \end{align*}

        Por tanto, se tiene que:
        \begin{equation*}
            |f(z)| = |f(z_0)| \leq |f(z_n)| \qquad \forall n\in \bb{N}.
        \end{equation*}

        Tomando el límite con $n\to \infty$, se tiene que:
        \begin{equation*}
            |f(z)| \leq \lim_{n\to \infty} |f(z_n)| = |f(0)|.
        \end{equation*}

        Como $z$ era un punto arbitrario de $D(0,1)$, se tiene que:
        \begin{equation*}
            |f(z)| \leq |f(0)| \quad \forall z\in D(0,1).
        \end{equation*}

        Por el principio del módulo máximo, $f$ es constante en $D(0,1)$.
    \end{description}
\end{ejercicio}

\begin{ejercicio}
    Mostrar que el teorema fundamental del Álgebra se deduce directamente del principio del módulo mínimo.\\

    Veamos que el teorema fundamental del Álgebra se deduce directamente del principio del módulo mínimo. Sea $P\in \bb{C}[z]$ un polinomio de grado $n\in \bb{N}$, no constante. 
    \begin{align*}
        P(z) &= a_n z^n + a_{n-1} z^{n-1} + \ldots + a_0 = \sum_{k=0}^n a_k z^k
    \end{align*}
    
    
    Como $P\to \infty$ cuando $z \to \infty$, sabemos que el ínfimo de $|P|$ en $\bb{C}$ se encuentra en un conjunto acotado. Considerando el conjunto acotado $\ol{D}(0,R)$, donde $R\in \bb{R}^+$ es suficientemente grande, se tiene que:
    \begin{equation*}
        m = \min\left\{ |P(z)| : z\in \ol{D}(0,R) \right\} = \min\left\{ |P(z)| : z\in \bb{C} \right\}.
    \end{equation*}

    Por tanto, $\exists z_0\in D(0,R+1)$ tal que $|P(z_0)| = m$. Por el principio del módulo mínimo, hay dos opciones:
    \begin{enumerate}
        \item $P(z)$ es constante en $D(0,R+1)$, y por el Principio de Identidad, $P$ es constante en $\bb{C}$, lo cual es una contradicción.
        \item $z_0$ es una raíz de $P$, y por tanto, $P(z_0) = 0$, quedando así demostrado el Teorema Fundamental del Álgebra.
    \end{enumerate}
\end{ejercicio}

\begin{ejercicio}
    Sea $\Omega$ un dominio y $f\in \cc{H}(\Omega)$. Probar que, si la función $\Re f$ tiene un extremo relativo en un punto de $\Omega$, entonces $f$ es constante.\\

    Definimos la siguiente función:
    \Func{g}{\Omega}{\bb{R}}{z}{e^{f(z)}}.

    Como $f$ es holomorfa, $g$ es holomorfa en $\Omega$. Calculemos su módulo:
    \begin{align*}
        |g(z)| &= |e^{f(z)}| = e^{\Re f(z)}.
    \end{align*}

    Como $\Re f$ tiene un extremo relativo en un punto $z_0\in \Omega$, como la exponencial real es estrictamente creciente, entonces $|g|$ tiene un extremo relativo en $z_0$.
    \begin{itemize}
        \item Si $\Re f$ tiene un máximo relativo en $z_0$, entonces $|g|$ tiene un máximo relativo en $z_0$. Por el principio del módulo máximo, $g$ es constante en $\Omega$.
        \item Si $\Re f$ tiene un mínimo relativo en $z_0$, entonces $|g|$ tiene un mínimo relativo en $z_0$. Por el principio del módulo mínimo, como la exponencial compleja no se anula, $g$ es constante en $\Omega$.
    \end{itemize}

    En cualquier caso, $g$ es constante en $\Omega$. Sea por tanto $\alpha\in \bb{C}^*$ tal que:
    \begin{equation*}
        g(z) = e^{f(z)} = \alpha \quad \forall z\in \Omega.
    \end{equation*}

    Por tanto, se tiene que:
    \begin{equation*}
        f(z) \in \Log(\alpha)\qquad \forall z\in \Omega.
    \end{equation*}

    Como además $f$ es continua, se tiene que $\exists \beta\in \Log(\alpha)$ tal que:
    \begin{equation*}
        f(z) = \beta \quad \forall z\in \Omega.
    \end{equation*}
    Por tanto, $f$ es constante en $\Omega$.
\end{ejercicio}

\begin{ejercicio}
    Sea $f : \ol{D}(0,1) \to \bb{C}$ una función continua en $\ol{D}(0,1)$ y holomorfa en $D(0,1)$, tal que $\Im f(z) = 0$ para todo $z\in \bb{T}$. Probar que $f$ es constante.\\

    Definimos la siguiente función:
    \Func{g}{\ol{D}(0,1)}{\bb{C}}{z}{e^{-i f(z)}}.

    De esta forma, tenemos que $g$ es continua en $\ol{D}(0,1)$ y holomorfa en $D(0,1)$. Calculemos su módulo:
    \begin{align*}
        |g(z)| &= |e^{-i f(z)}| = e^{\Im f(z)}\qquad \forall z\in \ol{D}(0,1).
    \end{align*}

    Como $\Im f(z) = 0$ para todo $z\in \bb{T}$, se tiene que:
    \begin{align*}
        |g(z)| &= e^{\Im f(z)} = e^{0} = 1\qquad \forall z\in \bb{T}.
    \end{align*}

    Como $|g|$ es constante en $\bb{T}$ y sabemos que $g(z) \neq 0$ para todo $z\in \ol{D}(0,1)$, por recíproco del corolario que une el principio del módulo máximo y el principio del módulo mínimo, se tiene que $g$ es constante en $\ol{D}(0,1)$. Por tanto, existe $\alpha\in \bb{C}^*$ tal que:
    \begin{equation*}
        g(z) = e^{-i f(z)} = \alpha \qquad \forall z\in \ol{D}(0,1).
    \end{equation*}
    Por tanto, se tiene que:
    \begin{equation*}
        f(z) \in -i \Log(\alpha) \qquad \forall z\in \ol{D}(0,1).
    \end{equation*}

    Como además $f$ es continua, se tiene que $\exists \beta\in \Log(\alpha)$ tal que:
    \begin{equation*}
        f(z) = -i\beta \qquad \forall z\in \ol{D}(0,1).
    \end{equation*}

    Por tanto, $f$ es constante en $\ol{D}(0,1)$.
\end{ejercicio}

\begin{ejercicio}
    Sea $\Omega$ un abierto de $\bb{C}$ y $f\in \cc{H}(\Omega)$ una función inyectiva. Dados $a\in \Omega$ y $r\in \bb{R}^+$ tales que $\ol{D}(a,r) \subseteq \Omega$, calcular, para cada $z\in \ol{D}(a,r)$, la integral
    \begin{equation*}
        \int_{C(a,r)} \frac{w f'(w)}{f(w) - f(z)} \, dw.
    \end{equation*}
\end{ejercicio}

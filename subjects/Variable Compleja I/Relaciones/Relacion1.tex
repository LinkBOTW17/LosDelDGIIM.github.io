\section{Números complejos}

\begin{ejercicio}
    Probar que el conjunto de matrices
    \[
        M = \left\{ \begin{pmatrix} a & -b \\ b & a \end{pmatrix} \mid a,b \in \mathbb{R} \right\}
    \]
    con las operaciones de suma y producto de matrices, es un cuerpo isomorfo a $\mathbb{C}$.\\

    % // TODO: Comprobar que M es un cuerpo.

    Para comprobar ahora que $M$ es isomorfo a $\bb{C}$, se debe probar que existe un isomorfismo entre ambos cuerpos.
    Sea la siguiente aplicación:
    \Func{f}{\bb{C}}{M}{z}{\begin{pmatrix} \Re z & -\Im z \\ \Im z & \Re z \end{pmatrix}}

    Para probar que $f$ es un isomorfismo, hemos de probar que es un homomorfismo (entre anillos, puesto que los cuerpos son un caso particular), y que es biyectivo. En primer lugar, comprobamos que es un homomorfismo:
    \begin{enumerate}
        \item $f(z_1+z_2)=f(z_1)+f(z_2)$.
        \begin{align*}
            f(z_1+z_2) &= \begin{pmatrix}
                \Re z_1 + \Re z_2 & -(\Im z_1 + \Im z_2) \\
                \Im z_1 + \Im z_2 & \Re z_1 + \Re z_2
            \end{pmatrix}
            = \begin{pmatrix} \Re z_1 & -\Im z_1 \\ \Im z_1 & \Re z_1 \end{pmatrix} + \begin{pmatrix} \Re z_2 & -\Im z_2 \\ \Im z_2 & \Re z_2 \end{pmatrix} =\\&= f(z_1) + f(z_2).
        \end{align*}

        \item $f(z_1\cdot z_2)=f(z_1)\cdot f(z_2)$.
        \begin{align*}
            f(z_1\cdot z_2) &= \begin{pmatrix}
                \Re z_1\cdot \Re z_2 - \Im z_1\cdot \Im z_2 & -(\Re z_1\cdot \Im z_2 + \Im z_1\cdot \Re z_2) \\
                \Im z_1\cdot \Re z_2 + \Re z_1\cdot \Im z_2 & \Re z_1\cdot \Re z_2 - \Im z_1\cdot \Im z_2
            \end{pmatrix} =\\&=
            \begin{pmatrix} \Re z_1 & -\Im z_1 \\ \Im z_1 & \Re z_1 \end{pmatrix} \cdot \begin{pmatrix} \Re z_2 & -\Im z_2 \\ \Im z_2 & \Re z_2 \end{pmatrix} = f(z_1)\cdot f(z_2).
        \end{align*}

        \item $f(1)=1$.
        
        Tenemos que $f(1)=\begin{pmatrix} 1 & 0 \\ 0 & 1 \end{pmatrix}=Id_2=1$.
    \end{enumerate}

    Por tanto, $f$ es un homomorfismo. Ahora, comprobamos que es biyectivo. Para ello, comprobamos que es inyectivo y sobreyectivo.
    \begin{itemize}
        \item $f$ es inyectiva.
        
        Sean $z_1,z_2\in \bb{C}$ de forma que $f(z_1)=f(z_2)$. Entonces, igualando componente a componente, tenemos que $\Re z_1=\Re z_2$ y $\Im z_1=\Im z_2$. Por lo tanto, $z_1=z_2$ y $f$ es inyectiva.

        \item $f$ es sobreyectiva.
        
        Sea $A=\begin{pmatrix} a & -b \\ b & a \end{pmatrix}\in M$. Entonces, sea $z=a+bi\in \bb{C}$, y tenemos que $f(z)=A$. Por tanto, $f$ es sobreyectiva.
    \end{itemize}

    Por tanto, $f$ también es biyectiva, y por tanto es un isomorfismo. Por tanto, $M$ es isomorfo a $\bb{C}$.
\end{ejercicio}

\begin{ejercicio}
    Calcular la parte real, la parte imaginaria y el módulo de los siguientes números complejos:
    \begin{enumerate}
        \item $z_1 = \dfrac{i-\sqrt{3}}{1+i}$.
        
        Tenemos que:
        \begin{align*}
            z_1 = (-\sqrt{3} + i)\cdot \dfrac{1}{1+i}=
            (-\sqrt{3} + i)\cdot \dfrac{1-i}{1+1}=
            \dfrac{-\sqrt{3}+i\sqrt{3}+i-i^2}{2}
            = \dfrac{1-\sqrt{3}+(1+\sqrt{3})i}{2}
        \end{align*}

        Por tanto, tenemos que:
        \begin{align*}
            \Re z_1 &= \dfrac{1-\sqrt{3}}{2}, \\
            \Im z_1 &= \dfrac{1+\sqrt{3}}{2}, \\
            |z_1| &= \sqrt{\left(\dfrac{1-\sqrt{3}}{2}\right)^2 + \left(\dfrac{1+\sqrt{3}}{2}\right)^2}
            = \sqrt{\dfrac{1+1+3+3}{4}} = \sqrt{2}.
        \end{align*}
        \item $z_2 = \dfrac{1}{i\sqrt{3}-1}$.
        
        Tenemos que:
        \begin{align*}
            z_2 = \dfrac{1}{i\sqrt{3}-1} = \dfrac{-1-\sqrt{3}i}{1+3}
        \end{align*}

        Por tanto, tenemos que:
        \begin{align*}
            \Re z_2 &= -\dfrac{1}{4}, \\
            \Im z_2 &= -\dfrac{\sqrt{3}}{4}, \\
            |z_2| &= \sqrt{\left(-\dfrac{1}{4}\right)^2 + \left(-\dfrac{\sqrt{3}}{4}\right)^2}
            = \sqrt{\dfrac{1+3}{16}} = \dfrac{1}{2}.
        \end{align*}
    \end{enumerate}
\end{ejercicio}

\begin{ejercicio}
    Sea $U = \{z \in \mathbb{C} \mid |z| < 1\}$. Fijado $a \in U$, se considera la función $f : U \to \mathbb{C}$ dada por
    \[
        f(z) = \dfrac{z-a}{1-\ol{a}z} \qquad \forall z \in U.
    \]
    Probar que $f$ es una biyección de $U$ sobre sí mismo y calcular su inversa.\\

    En primer lugar, comprobamos que $f$ es una aplicación de $U$ sobre $U$. Dado $z \in U$, tenemos que:
    \begin{align*}
        |f(z)| &= \left|\dfrac{z-a}{1-\ol{a}z}\right| = \dfrac{|z-a|}{|1-\ol{a}z|} < 1 \Longleftrightarrow
        |z-a| < |1-\ol{a}z| \Longleftrightarrow
        |z-a|^2 < |1-\ol{a}z|^2 \Longleftrightarrow \\&\Longleftrightarrow
        (z-a)(\ol{z}-\ol{a}) < (1-\ol{a}z)(1-a\ol{z}) \Longleftrightarrow
        z\ol{z}-a\ol{z}-z\ol{a}+a\ol{a} < 1-a\ol{z}-z\ol{a}+a\ol{a} z\ol{z}
        \Longleftrightarrow \\&\Longleftrightarrow
        |z|^2 +|a|^2 < 1+|a|^2|z|^2 \Longleftrightarrow
        |z|^2 - |a|^2|z|^2 < 1-|a|^2 \Longleftrightarrow \\&\Longleftrightarrow
        |z|^2(1-|a|^2) < 1-|a|^2 \Longleftrightarrow
        |z|^2 < 1.
    \end{align*}
    donde hemos usado que, como $|a| < 1$, entonces $|a|^2 < 1$ y por tanto $1-|a|^2 > 0$. Por tanto, $f$ es una aplicación de $U$ sobre $U$. A partir de ahora por tanto consideramos $f : U \to U$. Veamos que es biyectiva. Para ello, vamos a probar que es inyectiva y sobreyectiva.
    \begin{itemize}
        \item \ul{Inyectividad}:
        
        Sean $z_1, z_2 \in U$ tales que $f(z_1) = f(z_2)$. Entonces, tenemos que:
        \begin{align*}
            \dfrac{z_1-a}{1-\ol{a}z_1} &= \dfrac{z_2-a}{1-\ol{a}z_2} \Longrightarrow
            (z_1-a)(1-\ol{a}z_2) = (z_2-a)(1-\ol{a}z_1) \Longrightarrow \\&\Longrightarrow
            z_1 - \cancel{a} - \cancel{\ol{a}z_1z_2} + |a|^2z_2 = z_2 - \cancel{a} - \cancel{\ol{a}z_2z_1} + |a|^2z_1 \Longrightarrow\\&\Longrightarrow
            z_1 - |a|^2z_1 = z_2 - |a|^2z_2 \Longrightarrow 
            (1-|a|^2)z_1 = (1-|a|^2)z_2 \Longrightarrow
            z_1 = z_2.
        \end{align*}

        \item \ul{Sobreyectividad}:
        
        Sea $w \in U$. Vamos a buscar $z \in U$ tal que $f(z) = w$. Para ello, vamos a despejar $z$ de la ecuación $f(z) = w$:
        \begin{align*}
            \dfrac{z-a}{1-\ol{a}z} = w &\Longrightarrow
            z-a = w(1-\ol{a}z) \Longrightarrow
            z-a = w - w\ol{a}z \Longrightarrow
            z + w\ol{a}z = a + w \Longrightarrow\\&\Longrightarrow
            z(1+w\ol{a}) = a + w \Longrightarrow
            z = \dfrac{a+w}{1+w\ol{a}}.
        \end{align*}
        donde, en el último paso, hemos hecho uso de que $1+w\ol{a}\neq 0$, ya que $|wa|=|w||a|<1$ y:
        \begin{equation*}
            |1+w\ol{a}| \geq
            \left|1-|w||a|\right|=1-|w||a|>0
            \iff 1>|w||a|
        \end{equation*}
        y por tanto $1+w\ol{a}\neq 0$. Por tanto, dado $w \in U$, consideramos $z = \dfrac{a+w}{1+w\ol{a}}$. Vamos a comprobar que $z \in U$:
        \begin{align*}
            |z| &= \left|\dfrac{a+w}{1+w\ol{a}}\right| = \dfrac{|a+w|}{|1+w\ol{a}|} < 1 \Longleftrightarrow
            |a+w| < |1+w\ol{a}| \Longleftrightarrow
            |a+w|^2 < |1+w\ol{a}|^2 \Longleftrightarrow \\&\Longleftrightarrow
            (a+w)(\ol{a}+\ol{w}) < (1+w\ol{a})(1+\ol{w}a) \Longleftrightarrow \\&\Longleftrightarrow
            a\ol{a} + a\ol{w} + w\ol{a} + w\ol{w} < 1 + w\ol{a} + \ol{w}a + a\ol{a}w\ol{w} \Longleftrightarrow \\&\Longleftrightarrow
            |a|^2 + |w|^2 < 1 + |w|^2|a|^2 \Longleftrightarrow
            |a|^2 - |w|^2|a|^2 < 1 - |w|^2 \Longleftrightarrow \\&\Longleftrightarrow
            |w|^2(1-|a|^2) < 1 - |a|^2 \Longleftrightarrow
            |w|^2 < 1.
        \end{align*}

        Por tanto, $z \in U$ y $f(z) = w$. Por tanto, $f$ es sobreyectiva.
    \end{itemize}

    Por tanto, $f$ es biyectiva. Además, hemos comprobado que su inversa es:
    \Func{f^{-1}}{U}{U}{w}{\dfrac{a+w}{1+w\ol{a}}}
\end{ejercicio}

\begin{ejercicio}
    Dados $z_1, z_2, \ldots, z_n \in \mathbb{C}^*$, encontrar una condición necesaria y suficiente para que se verifique la siguiente igualdad:
    \[
        \left|\sum_{k=1}^n z_k\right| = \sum_{k=1}^n |z_k|.
    \]

    Veamos que dicha condición es que, para cada $k\in \Delta_n$, se tenga que $\exists \lm_k\in \mathbb{R}^+$ tal que $z_k = \lm_k~z_1$. Comprobaremos que dicha condición es necesaria y suficiente.
    \begin{description}
        \item[$\Longrightarrow)$] Veamos que es una condición necesaria. Demostramos por inducción sobre $n$.
        \begin{itemize}
            \item \ul{$n=1$}: La igualdad es trivialmente cierta, tomando $\lm_1 = 1$.
            
            \item \ul{$n=2$}: Hay dos opciones:
            \begin{description}
                \item[Opción Rutinaria]
                
                Supongamos que se cumple para $n=2$. Entonces, tenemos que:
            \begin{align*}
                |z_1+z_2| &= |z_1| + |z_2| \Longrightarrow
                |z_1+z_2|^2 = (|z_1| + |z_2|)^2 \Longrightarrow\\&\Longrightarrow
                z_1\ol{z_1} + z_1\ol{z_2} + z_2\ol{z_1} + z_2\ol{z_2} = |z_1|^2 + 2|z_1||z_2| + |z_2|^2 \Longrightarrow\\&\Longrightarrow
                z_1\ol{z_2} + z_2\ol{z_1} = 2|z_1||z_2|
                \Longrightarrow (z_1\ol{z_2})^2 + 2|z_1||z_2| + (z_2\ol{z_1})^2 = 4|z_1|^2|z_2|^2
                \Longrightarrow\\&\Longrightarrow
                (z_1\ol{z_2})^2 - 2|z_1||z_2| + (z_2\ol{z_1})^2 = 0
                \Longrightarrow
                (z_1\ol{z_2} - z_2\ol{z_1})^2 = 0
                \Longrightarrow
                z_1\ol{z_2} = z_2\ol{z_1}
            \end{align*}

            Tenemos ahora dos opciones:
            \begin{description}
                \item[Opción 1]
                
                Tenemos que:
                \begin{equation*}
                    z_1\ol{z_2} = z_2\ol{z_1}=\ol{z_1\ol{z_2}} \Longrightarrow
                    z_1\ol{z_2}\in \bb{R}^*
                \end{equation*}

                Tomamos ahora $\lm_2=\dfrac{z_2\ol{z_2}}{z_1\ol{z_2}}\in \bb{R}$, por lo que:
                \begin{equation*}
                    \lm_2~z_1 = \frac{z_2\ol{z_2}}{z_1\ol{z_2}}~z_1 = z_2
                \end{equation*}


                \item[Opción 2]

                Sea ahora $z_1=a+bi$ y $z_2=c+di$. Entonces, tenemos que:
                \begin{align*}
                    z_1\ol{z_2} = (a+bi)(c-di) = ac + bd + (bc-ad)i, \\
                    z_2\ol{z_1} = (c+di)(a-bi) = ac + bd + (ad-bc)i.
                \end{align*}

                Por tanto, tenemos que:
                \begin{align*}
                    z_1\ol{z_2} = z_2\ol{z_1} \Longrightarrow
                    bc-ad = ad-bc \Longrightarrow
                    ad=bc.
                \end{align*}

                Distinguimos en función del valor de $b$:
                \begin{itemize}
                    \item Si $b=0$, entonces $ad=0$.
                    \begin{itemize}
                        \item Si $a=b=0$, entonces $z_1=0\notin \mathbb{C}^*$, por lo que no es posible.
                        \item Si $a\neq 0$, entonces $d=b=0$, por lo que $z_1=a$, $z_2=c$, con $z_1,z_2\in \mathbb{R}^*$. Por tanto, tomando $\lm_2=\nicefrac{c}{a}$, se tiene que $z_2=\lm_2~z_1$.
                    \end{itemize}

                    \item Si $b\neq 0$, entonces $c=\nicefrac{ad}{b}$. Por tanto, tomando $\lm_2=\nicefrac{d}{b}$, se tiene que $z_2=\lm_2~z_1$.
                    \begin{equation*}
                        \lm_2~z_1 = \frac{d}{b}(a+bi) = \frac{ad}{b} + di = c+di = z_2.
                    \end{equation*}
                \end{itemize}
            \end{description}

            Por tanto, tenemos que $z_2=\lm_2~z_1$, con $\lm_2\in \bb{R}$. Para ver que $\lm_2\in \bb{R}^+$, tenemos que:
            \begin{align*}
                |z_1+z_2| &= |z_1(1+\lm_2)| = |z_1||1+\lm_2|\\
                |z_1| + |z_2| &= |z_1| + |\lm_2~z_1| = |z_1| + |\lm_2||z_1| = |z_1|(1+|\lm_2|).
            \end{align*}

            Igualando, y como $|z_1|\neq 0$, tenemos que $|1+\lm_2|=1+|\lm_2|$. Por tanto, como la igualdad de la desigualdad triangular en $\bb{R}$ se da si los dos números tienen el mismo signo, tenemos que $\lm_2\in \bb{R}^+$.

                \item[Otra Opción]
                
                Vemos ahora los elementos de $\bb{C}$ como elementos de $\bb{R}^2$, con el producto escalar de $\bb{R}^2$ y la norma euclídea. En Análisis Matemático I se provó que, en $\bb{R}^2$, se cumple la igualdad si y solo si:
                \begin{enumerate}
                    \item $z_1$ y $z_2$ son linealmente dependientes. Es decir, $\exists \lm\in \bb{R}$ tal que $z_2=\lm~z_1$.
                    \item Su producto escalar es positivo. Es decir, $\langle z_1,z_2\rangle > 0$. Esto se da si y solo si:
                    \begin{align*}
                        \langle z_1,z_2\rangle &= \langle z_1,\lm~z_1\rangle = \lm\langle z_1,z_1\rangle = \lm \|z_1\|^2 > 0 \Longleftrightarrow \lm > 0.
                    \end{align*}
                \end{enumerate}
            \end{description}
            
            
            En cualquier caso se cumple para $n=2$.

            \item \ul{Supongamos que se cumple para $n$, demostrémolo para $n+1$}.
            
            Por hipótesis (no de inducción, sino por trabajar en esta implicación), tenemos que:
            \begin{equation*}
                \left|\sum_{k=1}^{n+1} z_k\right| = \sum_{k=1}^{n+1} |z_k|.
            \end{equation*}

            Por tanto:
            \begin{align*}
                \left|\left(\sum_{k=1}^{n} z_k\right) + z_{n+1}\right| &= \left(\sum_{k=1}^{n} |z_k|\right) + |z_{n+1}|
            \end{align*}

            Usando ahora la hipótesis de inducción, tenemos que:
            \begin{align*}
                \left|\left(\sum_{k=1}^{n} \lm_k\right)z_1 + z_{n+1}\right| &= \left(\sum_{k=1}^{n} |\lm_k~z_1|\right) + |z_{n+1}|
                \Longrightarrow
                \\&\Longrightarrow
                \left|\left(\sum_{k=1}^{n} \lm_k\right)z_1 + z_{n+1}\right| = \left(\sum_{k=1}^{n} \lm_k\right)|z_1| + |z_{n+1}|
                \Longrightarrow
                \\&\Longrightarrow
                \left|\left(\sum_{k=1}^{n} \lm_k\right)z_1 + z_{n+1}\right| = \left|\left(\sum_{k=1}^{n} \lm_k\right)z_1\right| + |z_{n+1}|
            \end{align*}

            Notando por $w=\left(\sum\limits_{k=1}^{n} \lm_k\right)z_1\in \bb{C}^*$, y aplicando lo ya demostrado para $n=2$, vemos que $\exists \rho \in \bb{R}^+$ tal que $z_{n+1}=\rho~w$. Por tanto:
            \begin{equation*}
                z_{n+1} = \rho~w = \rho\left(\sum_{k=1}^{n} \lm_k\right)z_1 
            \end{equation*}

            Tomando $\lm_{n+1}=\rho\left(\sum\limits_{k=1}^{n} \lm_k\right)\in \bb{R}^+$, se tiene que $z_{n+1}=\lm_{n+1}~z_1$. Por tanto, se cumple para $n+1$.
        \end{itemize}

        Por tanto, por inducción se cumple para todo $n\in \bb{N}$.
        
        \item[$\Longleftarrow)$] Veamos que es una condición suficiente. Supongamos que, para cada $k\in \Delta_n$, se tiene que $\exists \lm_k\in \mathbb{R}^+$ tal que $z_k = \lm_k~z_1$. Entonces, tenemos que:
        \begin{align*}
            \left|\sum_{k=1}^n z_k\right| &= \left|\sum_{k=1}^n \lm_k~z_1\right| = \left|\left(\sum_{k=1}^n \lm_k\right)z_1\right|
            = \left(\sum_{k=1}^n \lm_k\right)|z_1| = \sum_{k=1}^n \lm_k|z_1| = \sum_{k=1}^n |\lm_k~z_1| = \sum_{k=1}^n |z_k|.
        \end{align*}
    \end{description}
\end{ejercicio}

\begin{ejercicio}
    Describir geométricamente los subconjuntos del plano dados por
    \begin{enumerate}
        \item $A = \{z \in \mathbb{C} \mid |z+i| = 2|z-i|\}$.
        
        Sea $z=x+iy\in A\subset \bb{C}$. Entonces, tenemos que:
        \begin{align*}
            |x+iy+i| &= 2|x+iy-i| \Longrightarrow
            |x+(y+1)i| = 2|x+(y-1)i| \Longrightarrow\\&\Longrightarrow
            \sqrt{x^2+(y+1)^2} = 2\sqrt{x^2+(y-1)^2} \Longrightarrow
            x^2+(y+1)^2 = 4(x^2+(y-1)^2) \Longrightarrow\\&\Longrightarrow
            x^2+y^2+2y+1 = 4x^2+4y^2-8y+4 \Longrightarrow
            3x^2+3y^2-10y+3 = 0 \Longrightarrow\\&\Longrightarrow
            x^2+y^2-\dfrac{10}{3}y+1 = 0
            \Longrightarrow
            x^2 + \left(y-\dfrac{5}{3}\right)^2 -\dfrac{25}{9}+1 = 0
            \Longrightarrow\\&\Longrightarrow
            x^2 + \left(y-\dfrac{5}{3}\right)^2 = \dfrac{16}{9}
        \end{align*}

        Por tanto, $A$ es la circunferencia de centro $\left(0,\dfrac{5}{3}\right)$ y radio $\dfrac{4}{3}$.
        \item $B = \{z \in \mathbb{C} \mid |z-i| + |z+i| = 4\}$.
        
        Sea $z=x+iy\in B\subset \bb{C}$. Entonces, tenemos que:
        \begin{align*}
            |x+iy-i| &+ |x+iy+i| = 4 \Longrightarrow
            |x+(y-1)i| + |x+(y+1)i| = 4 \Longrightarrow\\&\Longrightarrow
            \sqrt{x^2+(y-1)^2} + \sqrt{x^2+(y+1)^2} = 4
            \Longrightarrow\\&\Longrightarrow
            x^2+(y-1)^2 = 16 + x^2+(y+1)^2 - 8\sqrt{x^2+(y+1)^2}
            \Longrightarrow\\&\Longrightarrow
            -2y = 16 + 2y - 8\sqrt{x^2+(y+1)^2}
            \Longrightarrow\\&\Longrightarrow
            4+y=2\sqrt{x^2+(y+1)^2}
            \Longrightarrow
            16+8y+y^2=4x^2+4y^2+4+8y
            \Longrightarrow\\&\Longrightarrow
            4x^2 + 3y^2 = 12
            \Longrightarrow
            \dfrac{x^2}{3} + \dfrac{y^2}{4} = 1
        \end{align*}

        Por tanto, se trata de una elipse con centro en el origen. El semieje menor mide $\sqrt{3}$ y el semieje mayor mide $2$. Por tanto, la distancia focal es $\sqrt{4-3}=1$. Es decir, se trata de una elipse con ejes en los puntos $(0,i)$, $(0,-i)$ y eje mayor de longitud $4$. Esto se podría haber interpretado de forma directa al ver que la suma de las distancias de un punto a dos puntos fijos es constante.
    \end{enumerate}
\end{ejercicio}

\begin{ejercicio}\label{ej:1.6}
    Probar que se cumple la siguiente igualdad para todo $z \in \mathbb{C}^\ast \setminus \mathbb{R}^-$:
    \[
        \arg z = 2\arctan\left(\dfrac{\Im z}{\Re z + |z|}\right)\qquad z\in \mathbb{C}^\ast \setminus \mathbb{R}^-.
    \]

    Fijado $z\in \bb{C}^*\setminus \bb{R}^-$, consideramos $\arg z\in \left]-\pi,\pi\right[$. Entonces, como en particular se tiene $\arg z\in\Arg z$, tenemos que:
    \begin{equation*}
        \cos (\arg z) = \dfrac{\Re z}{|z|} \qquad \land \qquad \sen(\arg z) = \dfrac{\Im z}{|z|}.
    \end{equation*}
    \begin{comment}
    \begin{equation*}
        \left\{
            \begin{array}{c}
                \cos\theta = \dfrac{\Re z}{|z|} \\ \land \\ \sen\theta = \dfrac{\Im z}{|z|}.
            \end{array}
        \right\}
        \Longleftrightarrow \theta\in \Arg z
    \end{equation*}
    \end{comment}

    De esta forma, como $z\notin \bb{R}^-$ (y por tanto $|z|\neq -\Re z$), tenemos que:
    \begin{align*}
        \dfrac{\Im z}{\Re z + |z|}
        = \dfrac{\sen(\arg z)\cdot |z|}{\cos(\arg z)\cdot |z| + |z|}
        = \dfrac{\sen(\arg z)}{\cos(\arg z) + 1}
    \end{align*}

    Por ser ambas expresiones iguales, tenemos que:
    \begin{equation*}
        2\arctan\left(\dfrac{\Im z}{\Re z + |z|}\right)=2\arctan\left(\dfrac{\sen(\arg z)}{\cos(\arg z) + 1}\right)
    \end{equation*}

    La demostración se terminaría si vemos que las expresiones anteriores valen $\arg z$. Para ello, Definimos ahora la función auxiliar siguiente:
    \Func{f}{\left]-\pi,\pi\right[}{\bb{R}}{\alpha}{\alpha - 2\arctan\left(\dfrac{\sen\alpha}{\cos\alpha + 1}\right)}

    En primer lugar, tenemos que $f(0)=0-2\arctan(0)=0$. Por otro lado, como $f\in C^1\left(\left]-\pi,\pi\right[,\bb{R}\right)$, consideramos la derivada de $f$:
    \begin{align*}
        f'(\alpha) &= 1 - 2\cdot \dfrac{1}{1+\left(\dfrac{\sen\alpha}{\cos\alpha + 1}\right)^2}\cdot \dfrac{\cos\alpha(\cos\alpha + 1) + \sen\alpha\sen\alpha}{(\cos\alpha + 1)^2} =\\&=
        1-2\cdot \dfrac{\cos^2\alpha+\cos\alpha+\sen^2\alpha}{(\cos\alpha + 1)^2+\sen^2\alpha}=
        1-2\cdot \dfrac{1+\cos\alpha}{\cos^2\alpha + 1+2\cos\alpha+\sen^2\alpha}
        =\\&=
        1-2\cdot \dfrac{1+\cos\alpha}{2+2\cos\alpha}
        = 1- \dfrac{1+\cos\alpha}{1+\cos\alpha} = 0\qquad \forall \alpha\in \left]-\pi,\pi\right[.
    \end{align*}



    Por tanto, $f$ es constante, por lo que $f(\alpha)=0$ para todo $\alpha\in \left]-\pi,\pi\right[$. Tomando como ángulo $\alpha=\arg z$, que por la elección hecha sabemos que $\arg z \in \left]-\pi,\pi\right[$, tenemos que: 
    \begin{equation*}
        \arg z = 2\arctan\left(\dfrac{\sen(\arg z)}{\cos(\arg z) + 1}\right)
    \end{equation*}

    Por tanto, por lo anteriormente visto tenemos que:
    \begin{align*}
        \arg z = 2\arctan\left(\dfrac{\Im z}{\Re z + |z|}\right)\qquad z\in \mathbb{C}^\ast \setminus \mathbb{R}^-.
    \end{align*}
    como queríamos demostrar.
\end{ejercicio}

\begin{ejercicio}
    Probar que, si $z = x+iy \in \mathbb{C}^*$, con $x,y \in \mathbb{R}$, se tiene:
    \[
        \arg z = \begin{cases}
            \arctan\left(\nicefrac{y}{x}\right) & \text{si } x > 0, \\
            \arctan\left(\nicefrac{y}{x}\right) + \pi & \text{si } x < 0, y > 0, \\
            \arctan\left(\nicefrac{y}{x}\right) - \pi & \text{si } x < 0, y < 0, \\
            \nicefrac{\pi}{2} & \text{si } x = 0, y > 0, \\
            \nicefrac{-\pi}{2} & \text{si } x = 0, y < 0.
        \end{cases}
    \]
    
    Como $\arg z\in\Arg z$, tenemos que:
    \begin{equation*}
        \cos (\arg z) = \dfrac{\Re z}{|z|} \qquad \land \qquad \sen(\arg z) = \dfrac{\Im z}{|z|}.
    \end{equation*}

    Por tanto, tenemos que $x=\Re z = |z|\cos(\arg z)$ e $y=\Im z = |z|\sen(\arg z)$. Por tanto, distinguimos en función de los valores de $x$ e $y$, usando además que $\arg z\in \left]-\pi,\pi\right[$:
    \begin{itemize}
        \item \ul{Si $x>0$}:
        
        En este caso, $x=|z|\cos(\arg z)>0\Longrightarrow \arg z\in \left]\nicefrac{-\pi}{2},\nicefrac{\pi}{2}\right[$.
        \begin{align*}
            \arctan\left(\dfrac{y}{x}\right) = \arctan\left(\dfrac{\sen(\arg z)}{\cos(\arg z)}\right) = \arctan\left(\tan(\arg z)\right) = \arg z
        \end{align*}
        donde, en la última igualdad, hemos usado que la arcotangente es la inversa de la tangente en el intervalo $\left]-\nicefrac{\pi}{2},\nicefrac{\pi}{2}\right[$.

        \item \ul{Si $x<0, y>0$}:
        
        En este caso, $y=|z|\sen(\arg z)>0\Longrightarrow \arg z\in \left]0,\pi\right[$. Además, se tiene que $x=|z|\cos(\arg z)<0$. Por tanto, $\arg z\in \left]\nicefrac{\pi}{2},\pi\right[$. No obstante, como no pertenece a la rama principal, hemos de considerar $\theta = \arg z - \pi\in \left]\nicefrac{-\pi}{2},0\right[$, que por la periodicidad de la tangente sabemos que $\tan(\theta)=\tan(\arg z)$. Por tanto, tenemos que:
        \begin{align*}
            \arctan\left(\dfrac{y}{x}\right) &= \arctan\left(\tan(\arg z)\right) = \arctan\left(\tan(\theta)\right) = \theta = \arg z - \pi
            \Longrightarrow\\&\Longrightarrow
            \arg z = \arctan\left(\dfrac{y}{x}\right) + \pi
        \end{align*}
        
        \item \ul{Si $x<0, y<0$}:
        
        En este caso, $y=|z|\sen(\arg z)<0\Longrightarrow \arg z\in \left]-\pi,0\right[$. Además, se tiene que $x=|z|\cos(\arg z)<0$. Por tanto, $\arg z\in \left]-\pi,-\nicefrac{\pi}{2}\right[$. No obstante, como no pertenece a la rama principal, hemos de considerar $\theta = \arg z + \pi\in \left]0,\nicefrac{\pi}{2}\right[$, que por la periodicidad de la tangente sabemos que $\tan(\theta)=\tan(\arg z)$. Por tanto, tenemos que:
        \begin{align*}
            \arctan\left(\dfrac{y}{x}\right) &= \arctan\left(\tan(\arg z)\right) = \arctan\left(\tan(\theta)\right) = \theta = \arg z + \pi
            \Longrightarrow\\&\Longrightarrow
            \arg z = \arctan\left(\dfrac{y}{x}\right) - \pi
        \end{align*}

        \item \ul{Si $x=0, y>0$}:
        
        En este caso, $y=|z|\sen(\arg z)>0\Longrightarrow \arg z\in \left]0,\pi\right[$. Además, se tiene que $x=|z|\cos(\arg z)=0$. Por tanto, $\arg z=\nicefrac{\pi}{2}$.

        \item \ul{Si $x=0, y<0$}:
        
        En este caso, $y=|z|\sen(\arg z)<0\Longrightarrow \arg z\in \left]-\pi,0\right[$. Además, se tiene que $x=|z|\cos(\arg z)=0$. Por tanto, $\arg z=\nicefrac{-\pi}{2}$.
    \end{itemize}
\end{ejercicio}

\begin{ejercicio}
    Probar las \emph{fórmulas de De Moivre}:
    \[
        \cos(n\theta) + i\sen(n\theta) = (\cos\theta + i\sen\theta)^n \qquad \forall \theta \in \mathbb{R}, \forall n \in \mathbb{N}.
    \]

    Demostraremos las fórmulas de De Moivre por inducción sobre $n$.
    \begin{itemize}
        \item \ul{$n=1$}: La igualdad es trivialmente cierta.
        \item \ul{Supongamos que se cumple para $n$, demostrémoslo para $n+1$}:
        \begin{align*}
            (\cos\theta + i\sen\theta)^{n+1}
            &= (\cos\theta + i\sen\theta)^n(\cos\theta + i\sen\theta)
            =\\&= (\cos(n\theta) + i\sen(n\theta))(\cos\theta + i\sen\theta)
            =\\&= \cos(n\theta)\cos\theta - \sen(n\theta)\sen\theta + i(\cos(n\theta)\sen\theta + \sen(n\theta)\cos\theta) = \\&=
            \cos((n+1)\theta) + i\sen((n+1)\theta).
        \end{align*}
    \end{itemize}

    Por tanto, por inducción se cumple para todo $n\in \mathbb{N}$. Como no hemos impuesto restricciones sobre $\theta$, se cumple para todo $\theta \in \mathbb{R}$.
\end{ejercicio}

\begin{ejercicio}
    Calcular las partes real e imaginaria del número complejo
    \[
        z=\left(\dfrac{1+i\sqrt{3}}{2}\right)^8.
    \]

    Sea $z'=\dfrac{1+i\sqrt{3}}{2}$. Entonces, tenemos que:
    \begin{align*}
        |z'| &= \sqrt{\left(\dfrac{1}{2}\right)^2+\left(\dfrac{\sqrt{3}}{2}\right)^2} = \sqrt{\dfrac{1}{4}+\dfrac{3}{4}} = 1, \\
        \arg (z') &= \arctan\left(\dfrac{\sqrt{3}}{1}\right) = \dfrac{\pi}{3}
    \end{align*}
    donde, para calcular el argumento, hemos empleado que $\Re z'>0$. Por tanto, tenemos que:
    \begin{align*}
        z' &= \cos\left(\dfrac{\pi}{3}\right) + i\sen\left(\dfrac{\pi}{3}\right) \\
        z &= (z')^8 = \left[\cos\left(\dfrac{\pi}{3}\right) + i\sen\left(\dfrac{\pi}{3}\right)\right]^8 \AstIg \cos\left(\dfrac{8\pi}{3}\right) + i\sen\left(\dfrac{8\pi}{3}\right) = \cos\left(\dfrac{2\pi}{3}\right) + i\sen\left(\dfrac{2\pi}{3}\right). 
    \end{align*}
    donde en $(\ast)$ hemos usado las fórmulas de De Moivre.    Por tanto, tenemos que:
    \begin{align*}
        \Re z &= \cos\left(\dfrac{2\pi}{3}\right) = -\dfrac{1}{2}, \\
        \Im z &= \sen\left(\dfrac{2\pi}{3}\right) = \dfrac{\sqrt{3}}{2}.
    \end{align*}
\end{ejercicio}

\begin{ejercicio}
    Probar que, para todo $x \in \mathbb{R}$, se tiene:
    \begin{align}
        \sen\left(\frac{x}{2}\right)\sum_{k=0}^n \cos(kx) &= \cos \left(\dfrac{nx}{2}\right) \sen\left(\dfrac{(n+1)x}{2}\right), \\
        \sen\left(\frac{x}{2}\right)\sum_{k=0}^n \sen(kx) &= \sen \left(\dfrac{nx}{2}\right) \sen\left(\dfrac{(n+1)x}{2}\right)
    \end{align}

    Demostraremos ambas igualdades de forma simultánea. Para ello, multiplicaremos la segunda igualdad por $i$ y sumaremos ambas:
    \begin{align*}
        \sen\left(\frac{x}{2}\right)\sum_{k=0}^n \left(\cos(kx) + i\sen\left(kx\right)\right) &\AstIg \sen \left(\dfrac{x}{2}\right)\sum_{k=0}^n \left(\cos(x) + i\sen(x)\right)^k
    \end{align*}
    donde en $(\ast)$ hemos usado la fórmula de De Moivre. Considerando el número complejo $z=\cos(x)+i\sen(x)$, definimos $u=\cos\left(\dfrac{x}{2}\right)+i\sen\left(\dfrac{x}{2}\right)$, por lo que $u^2=z$. Además, tenemos que:
    \begin{equation*}
        1-z^k = u^k\ol{u}^k-u^{2k} = u^k(\ol{u}^k-u^k) = -2i\sen\left(k\cdot \dfrac{x}{2}\right)\cdot u^k\qquad \forall k\in \bb{N}.
    \end{equation*}

    Por tanto, usando dicho valor de $z$, tenemos que:
    \begin{align*}
        \sen\left(\frac{x}{2}\right)\sum_{k=0}^n \left(\cos(kx) + i\sen\left(kx\right)\right) &= \sen \left(\dfrac{x}{2}\right)\sum_{k=0}^n z^k
    \end{align*}
    
    La suma de la derecha es la suma de una progresión geométrica, cuya suma parcial se calcula de igual forma que en $\bb{R}$:
    \begin{align*}
        \sen\left(\frac{x}{2}\right)\sum_{k=0}^n \left(\cos(kx) + i\sen\left(kx\right)\right) &\AstIg \sen \left(\dfrac{x}{2}\right)\cdot \dfrac{1-z^{n+1}}{1-z} =\\&=
        \sen \left(\dfrac{x}{2}\right)\cdot \dfrac{-2i\sen\left((n+1)\cdot \dfrac{x}{2}\right)\cdot u^{n+1}}{-2i\sen\left(\dfrac{x}{2}\right)\cdot u}
        =\\&=\sen\left(\dfrac{(n+1)x}{2}\right)\cdot u^n
        =\\&=\sen\left(\dfrac{(n+1)x}{2}\right)\left[\cos\left(\dfrac{nx}{2}\right)+i\sen\left(\dfrac{nx}{2}\right)\right]
    \end{align*}
    donde en $(\ast)$ hemos calculado la suma parcial, donde hemos supuesto que $z\neq 1$; es decir, que $x\notin 2\pi\bb{Z}$ (ya que, en dicho caso, ambas igualdades son triviales). Igualando las partes real e imaginaria, obtenemos las igualdades pedidas.
\end{ejercicio}
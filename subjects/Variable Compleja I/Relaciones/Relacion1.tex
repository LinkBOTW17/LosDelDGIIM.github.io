\section{Números complejos}

\begin{ejercicio}
    Probar que el conjunto de matrices
    \[
        M = \left\{ \begin{pmatrix} a & -b \\ b & a \end{pmatrix} \mid a,b \in \mathbb{R} \right\}
    \]
    con las operaciones de suma y producto de matrices, es un cuerpo isomorfo a $\mathbb{C}$.
\end{ejercicio}

\begin{ejercicio}
    Calcular la parte real, la parte imaginaria y el módulo de los siguientes números complejos:
    \begin{enumerate}
        \item $z_1 = \dfrac{i-\sqrt{3}}{1+i}$.
        \item $z_2 = \dfrac{1}{i\sqrt{3}-1}$.
    \end{enumerate}
\end{ejercicio}

\begin{ejercicio}
    Sea $U = \{z \in \mathbb{C} \mid |z| < 1\}$. Fijado $a \in U$, se considera la función $f : U \to \mathbb{C}$ dada por
    \[
        f(z) = \dfrac{z-a}{1-\ol{a}z} \qquad \forall z \in U.
    \]
    Probar que $f$ es una biyección de $U$ sobre sí mismo y calcular su inversa.
\end{ejercicio}

\begin{ejercicio}
    Dados $z_1, z_2, \ldots, z_n \in \mathbb{C}^*$, encontrar una condición necesaria y suficiente para que se verifique la siguiente igualdad:
    \[
        \left|\sum_{k=1}^n z_k\right| = \sum_{k=1}^n |z_k|.
    \]
\end{ejercicio}

\begin{ejercicio}
    Describir geométricamente los subconjuntos del plano dados por
    \begin{enumerate}
        \item $A = \{z \in \mathbb{C} \mid |z+i| = 2|z-i|\}$.
        \item $B = \{z \in \mathbb{C} \mid |z-i| + |z+i| = 4\}$.
    \end{enumerate}
\end{ejercicio}

\begin{ejercicio}
    Probar que se cumple la siguiente igualdad para todo $z \in \mathbb{C}^\ast \setminus \mathbb{R}^-$:
    \[
        \arg z = 2\arctan\left(\dfrac{\Im z}{\Re z + |z|}\right)\qquad z\in \mathbb{C}^\ast \setminus \mathbb{R}^-.
    \]
\end{ejercicio}

\begin{ejercicio}
    Probar que, si $z = x+iy \in \mathbb{C}^*$, con $x,y \in \mathbb{R}$, se tiene:
    \[
        \arg z = \begin{cases}
            \arctan\left(\nicefrac{y}{x}\right) & \text{si } x > 0, \\
            \arctan\left(\nicefrac{y}{x}\right) + \pi & \text{si } x < 0, y > 0, \\
            \arctan\left(\nicefrac{y}{x}\right) - \pi & \text{si } x < 0, y < 0, \\
            \nicefrac{\pi}{2} & \text{si } x = 0, y > 0, \\
            \nicefrac{-\pi}{2} & \text{si } x = 0, y < 0.
        \end{cases}
    \]
\end{ejercicio}

\begin{ejercicio}
    Probar las \emph{fórmulas de De Moivre}:
    \[
        \cos(n\theta) + i\sin(n\theta) = (\cos\theta + i\sin\theta)^n \qquad \forall \theta \in \mathbb{R}, \forall n \in \mathbb{N}.
    \]
\end{ejercicio}

\begin{ejercicio}
    Calcular las partes real e imaginaria del número complejo
    \[
        z=\left(1+i\sqrt{3}\right)^8.
    \]
\end{ejercicio}

\begin{ejercicio}
    Probar que, para todo $x \in \mathbb{R}$, se tiene:
    \begin{align}
        \sen\left(\frac{x}{2}\right)\sum_{k=0}^n \cos(kx) &= \cos \left(\dfrac{nx}{2}\right) \sen\left(\dfrac{(n+1)x}{2}\right), \\
        \sen\left(\frac{x}{2}\right)\sum_{k=0}^n \sen(kx) &= \sen \left(\dfrac{nx}{2}\right) \sen\left(\dfrac{(n+1)x}{2}\right)
    \end{align}
\end{ejercicio}
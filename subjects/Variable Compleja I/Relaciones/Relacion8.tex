\section{Equivalencia entre analiticidad y holomorfía}

\begin{ejercicio}
    Sea $\gamma$ un camino y $\varphi : \gamma^* \to \bb{C}$ una función continua. Se define $f : \bb{C}\setminus \gamma^* \to \bb{C}$ por:
    \Func{f}{\bb{C}\setminus \gamma^*}{\bb{C}}{z}{\displaystyle\int_{\gamma} \frac{\varphi(w)}{w-z}\ dw}

    Probar que $f$ es una función analítica en $\bb{C}\setminus \gamma^*$ y que:
    \begin{equation*}
        f^{(k)}(z) = k! \int_{\gamma} \frac{\varphi(w)}{(w-z)^{k+1}}dw \quad \forall z \in \bb{C}\setminus \gamma^*, \forall k \in \bb{N}
    \end{equation*}~

    Realizaremos un desarrollo similar al de la demostración del desarrollo en Serie de Taylor. Como $\gamma$ es continua y está definida en un compacto, $\gamma*$ es cerrado. Por tanto, sea le abierto $\Omega=\bb{C}\setminus \gamma^*$. Fijamos ahora $a\in \Omega$, y sea $R_a=d(a,\gamma^*)>0$. Entonces, para cada $w\in \gamma^*$ y $z\in D(a,R_a)$, se tiene que:
    \begin{align*}
        \sum_{n=0}^{\infty} \left(\frac{z-a}{w-a}\right)^n &= \frac{1}{1-\frac{z-a}{w-a}} = \frac{w-a}{w-z}
    \end{align*}

    Por lo tanto, para cada $z\in D(a,R_a)$, se tiene que:
    \begin{align*}
        f(z) &= \int_{\gamma} \frac{\varphi(w)}{w-z}\ dw = \int_{\gamma} \varphi(w)\cdot \dfrac{1}{w-a}\cdot \frac{w-a}{w-z}\ dw \\
        &= \int_{\gamma} \varphi(w)\cdot \dfrac{1}{w-a}\cdot \sum_{n=0}^{\infty} \left(\frac{z-a}{w-a}\right)^n\ dw \\
        &= \int_{\gamma}\ \sum_{n=0}^{\infty} \left(\varphi(w)\cdot \dfrac{1}{(w-a)^{n+1}}\cdot (z-a)^n\right)\ dw 
    \end{align*}

    Aplicamos ahora el Test de Weierstrass para demostrar la convergencia uniforme de la serie. Como $\varphi$ es continua y está definida en un compacto, $\exists M\in \bb{R}^+$ tal que $|\varphi(w)|\leq M$ para todo $w\in \gamma^*$. Entonces, para cada $n\in \bb{N}$ y $z\in D(a,R_a)$, se tienw que:
    \begin{align*}
        \left|\varphi(w)\cdot \dfrac{1}{(w-a)^{n+1}}\cdot (z-a)^n\right| &\leq M\cdot \frac{|z-a|^n}{|w-a|^{n+1}}
        \leq \dfrac{M}{R_a} \cdot \left(\frac{|z-a|}{R_a}\right)^n \qquad \forall w\in \gamma^*
    \end{align*}

    Como $|z-a|<R_a$, se tiene que la serie de término general la cota dada es convergente. Por lo tanto, por el Test de Weierstrass, la serie converge uniformemente en $\gamma^*$, y por lo tanto podemos cambiar el orden de integración y suma. Entonces, se tiene que:
    \begin{align*}
        f(z) &= \sum_{n=0}^{\infty} (z-a)^n \int_{\gamma} \dfrac{\varphi(w)}{(w-a)^{n+1}}\ dw
        \qquad \forall z\in D(a,R_a)
    \end{align*}

    Por tanto, como este razonamiento es válido para cualquier $a\in \Omega$, se tiene la analiticidad de $f$ en $\Omega$. Para la segunda parte, por el Teorema de Holomorfía de funciones dadas como suma de series de potencias, se tiene que:
    \begin{align*}
        f^{(k)}(z) &= \sum_{n=k}^{\infty} (z-a)^{n-k}\cdot \dfrac{n!}{(n-k)!}\cdot \int_{\gamma} \dfrac{\varphi(w)}{(w-a)^{n+1}}\ dw
        \qquad \forall z\in D(a,R_a),\ \forall k\in \bb{N}
    \end{align*}

    En particular, evaluando en $z=a$, se tiene que:
    \begin{align*}
        f^{(k)}(a) &= k!\cdot \int_{\gamma} \varphi(w)\cdot \dfrac{\varphi(w)}{(w-a)^{k+1}}\ dw\qquad \forall k\in \bb{N},\ \forall a\in \Omega
    \end{align*}
\end{ejercicio}

\begin{ejercicio}\label{ej:8.2}
    Para $\alpha \in \bb{C}$ se define:
    \begin{equation*}
        \binom{\alpha}{0} = 1 \quad \text{y} \quad \binom{\alpha}{n} = \frac{1}{n!} \prod_{j=0}^{n-1} (\alpha-j) = \frac{\alpha(\alpha-1)\cdots(\alpha-n+1)}{n!} \quad \forall n \in \bb{N}
    \end{equation*}
    Probar que:
    \begin{equation*}
        (1+z)^{\alpha} = \sum_{n=0}^{\infty} \binom{\alpha}{n} z^n \quad \forall z \in D(0,1)
    \end{equation*}~

    Definimos la función $f$ siguiente:
    \Func{f}{D(0,1)}{\bb{C}}{z}{(1+z)^{\alpha}}

    Por la definición de potencia principal, se tiene que:
    \begin{equation*}
        f(z) = e^{\alpha \log(1+z)}\qquad \forall z \in D(0,1)
    \end{equation*}

    Como $\log(1+z)$ es holomorfa en $D(0,1)$, se tiene que $f\in \cc{H}(D(0,1))$. Por lo tanto, por el Desarrollo en Serie de Taylor de $f$ centrado en el origen, se tiene que:
    \begin{equation*}
        f(z) = \sum_{n=0}^{\infty} \frac{f^{(n)}(0)}{n!} z^n \quad \forall z \in D(0,1)
    \end{equation*}

    Veamos por inducción sobre $n$ que:
    \begin{equation*}
        f^{(n)}(z) = \binom{\alpha}{n}\cdot n!\cdot (1+z)^{\alpha-n} \qquad \forall n \in \bb{N}, \forall z \in D(0,1)
    \end{equation*}
    \begin{itemize}
        \item Para $n=0$, se tiene que:
        \begin{equation*}
            f(z) = (1+z)^{\alpha} = \binom{\alpha}{0}\cdot 0!\cdot (1+z)^{\alpha-0}
        \end{equation*}

        \item Para $n=1$, se tiene que:
        \begin{equation*}
            f'(z) = \alpha(1+z)^{\alpha-1} = \binom{\alpha}{1}\cdot 1!\cdot (1+z)^{\alpha-1}
        \end{equation*}

        \item Supongamos que es cierto para $n$, y veamos que es cierto para $n+1$. Entonces, se tiene que:
        \begin{align*}
            f^{(n+1)}(z) &= \binom{\alpha}{n}\cdot n!\cdot (\alpha-n)(1+z)^{\alpha-n-1} \\
            &= \dfrac{1}{\cancel{n!}}\prod_{j=0}^{n-1} (\alpha-j)\cdot \cancel{n!}\cdot (\alpha-n)(1+z)^{\alpha-n-1}\cdot \dfrac{(n+1)!}{(n+1)!} \\
            &= \dfrac{1}{(n+1)!}\prod_{j=0}^{n} (\alpha-j)\cdot (n+1)!\cdot (1+z)^{\alpha-n-1} \\
            &= \binom{\alpha}{n+1}\cdot (n+1)!\cdot (1+z)^{\alpha-(n+1)} \qquad \forall z \in D(0,1)
        \end{align*}
        Por lo tanto, se ha probado por inducción. Evaluando en $z=0$, se tiene que:
        \begin{equation*}
            f^{(n)}(0) = \binom{\alpha}{n}\cdot n!\qquad \forall n \in \bb{N}
        \end{equation*}

        Por lo tanto, se tiene que:
        \begin{align*}
            (1+z)^{\alpha}=f(z) &= \sum_{n=0}^{\infty} \frac{f^{(n)}(0)}{n!} z^n \\
            &= \sum_{n=0}^{\infty} \binom{\alpha}{n}\cdot z^n\qquad \forall z \in D(0,1)
        \end{align*}
    \end{itemize}
\end{ejercicio}

\begin{ejercicio}
    Obtener el desarrollo en serie de Taylor de la función $f$, centrado en el origen, en cada uno de los siguientes casos:
    \begin{enumerate}
        \item $f(z) = \log(z^2 - 3z + 2) \quad \forall z \in D(0,1)$
        
        Fijado $z\in D(0,1)$, se tiene que:
        \begin{align*}
            \Re(z^2 - 3z + 2) &= \Re(z^2) - 3\Re(z) + 2
            = \Re(z)^2-\Im(z)^2 - 3\Re(z) + 2\\
            \Im(z^2 - 3z + 2) &= \Im(z^2) - 3\Im(z) = 2\Re(z)\Im(z) - 3\Im(z)
        \end{align*}

        Supongamos que $z^2 - 3z + 2 \in \bb{R}^-_0$. Entonces, se tiene que:
        \begin{align*}
            0 = \Im(z^2 - 3z + 2) \iff 2\Re(z)\Im(z) = 3\Im(z)\iff \left\{
            \begin{array}{c}
                \Im(z) = 0\\
                \lor\\
                \Re(z) = \frac{3}{2}
            \end{array}\right.
        \end{align*}
        
        Distinguimos dos casos:
        \begin{itemize}
            \item Supongamos $\Im(z) = 0$. Como la parte real es negativa, resolvemos la inecuación:
            \begin{equation*}
                \Re(z)^2-3\Re(z)+2<0
            \end{equation*}

            Las raíces de la ecuación son $1$ y $2$, y evaluando en $\Re(z)=0$ vemos que:
            \begin{equation*}
                \Re(z)^2-3\Re(z)+2>0\qquad \forall z\in D(0,1)
            \end{equation*}

            Por lo tanto, llegamos a una contradicción.

            \item Supongamos $\Re(z) = \frac{3}{2}$. En este caso, $z\notin D(0,1)$, y por lo tanto no es posible.
        \end{itemize}

        Por tanto, como el logaritmo principal es holomorfo en $\bb{C}^*\setminus \bb{R}^-$, se tiene que $f\in \cc{H}(D(0,1))$. Por el Desarollo en Serie de Taylor, se tiene que:
        \begin{equation*}
            f(z) = \sum_{n=0}^{\infty} \frac{f^{(n)}(0)}{n!} z^n \qquad \forall z \in D(0,1)
        \end{equation*}

        Obtener la derivada $n-$ésima de $f$ en el origen no es sencillo, por lo que optaremos por otro método para obtener el desarrollo. Tenemos que:
        \begin{equation*}
            f'(z) = \frac{2z-3}{z^2-3z+2}\qquad \forall z \in D(0,1)
        \end{equation*}

        Descomponemos la función en fracciones parciales:
        \begin{equation*}
            \frac{2z-3}{z^2-3z+2} = \frac{A}{z-1} + \frac{B}{z-2}= \frac{A(z-2)+B(z-1)}{(z-1)(z-2)}
        \qquad \forall z \in D(0,1)
        \end{equation*}
        \begin{itemize}
            \item Para $z=1$, se tiene que $-1=-A$, y por lo tanto $A=1$.
            \item Para $z=2$, se tiene que $1=B$, y por lo tanto $B=1$.
        \end{itemize}

        Por lo tanto, para cada $z\in D(0,1)$, se tiene que:
        \begin{align*}
            f'(z) &= \frac{1}{z-1} + \frac{1}{z-2}
            = -\left(\frac{1}{1-z} + \frac{1}{2}\cdot \frac{1}{1-\frac{z}{2}}\right) \\
            &= -\left(\sum_{n=0}^{\infty} z^n + \frac{1}{2}\cdot \sum_{n=0}^{\infty} \left(\frac{z}{2}\right)^n\right)
            = -\left(\sum_{n=0}^{\infty}\left(1+\frac{1}{2^{n+1}}\right)z^n\right)
        \end{align*}

        Integrando término a término, se tiene que:
        \begin{align*}
            f(z) &= -\left(\sum_{n=0}^{\infty}\left(1+\frac{1}{2^{n+1}}\right)\cdot \frac{z^{n+1}}{n+1}\right)+C =
            -\left(\sum_{n=0}^{\infty}\dfrac{2^{n+1}+1}{2^{n+1}(n+1)}\cdot z^{n+1}\right)+C \\
            &= -\left(\sum_{n=1}^{\infty}\dfrac{2^{n}+1}{n2^n}\cdot z^{n}\right)+C\qquad \forall z \in D(0,1)
        \end{align*}

        Como $f(0)=\log 2 = \ln 2$, se tiene que $C=\ln 2$ y el desarrollo en Serie de Taylor buscado es:
        \begin{align*}
            f(z) &= -\left(\sum_{n=1}^{\infty}\dfrac{2^{n}+1}{n2^n}\cdot z^{n}\right)+\ln 2 \qquad \forall z \in D(0,1)
        \end{align*} 
        \item $f(z) = \dfrac{z^2}{(z+1)^2} \quad \forall z \in \bb{C}\setminus\{-1\}$
        
        Por ser racional, sabemos que $f\in \cc{H}(\bb{C}\setminus\{-1\})$, por lo que solo podremos aspirar a un desarrollo en serie de Taylor en $D(0,1)$. Hay dos opciones:
        \begin{description}
            \item[Fracciones Simples] 
            
            Para evitar el cálculo de la derivada $n-$ésima, descomponemos en fracciones simples, pero antes hemos de realizar la división de polinomios:
            \begin{equation*}
                \polyset{vars=z}
                \polylongdiv[style=D]{z^2}{z^2+2z+1}
            \end{equation*}

            Por lo tanto, se tiene que:
            \begin{equation*}
                f(z) = 1 - \frac{2z+1}{(z+1)^2} = 1-\left(\dfrac{A}{z+1}+\dfrac{B}{(z+1)^2}\right)
                = 1-\left(\dfrac{A(z+1)+B}{(z+1)^2}\right)
            \end{equation*}
            \begin{itemize}
                \item Para $z=-1$, se tiene que $-1=B$.
                \item Para $z=0$, se tiene que $1=A+B$, y por lo tanto $A=2$.
            \end{itemize}

            Por tanto:
            \begin{equation*}
                f(z) = 1 - \frac{2}{z+1} + \frac{1}{(z+1)^2} \qquad \forall z \in D(0,1)
            \end{equation*}

            Viendo el tercer sumando como la derivada de una función que es la suma de una serie geométrica, se tiene que:
            \begin{align*}
                f(z) &= 1 - 2\cdot \left(\sum_{n=0}^{\infty} (-z)^n\right) + \left(-\sum_{n=0}^{\infty} (-z)^n\right)'
                =\\&= 1-2\cdot \left(\sum_{n=0}^{\infty} (-z)^n\right) + \left(-\sum_{n=1}^{\infty} (-1)^nn\ z^{n-1}\right) =\\
                &= 1-2\cdot \left(\sum_{n=0}^{\infty} (-1)^n\ z^n\right) + \left(-\sum_{n=0}^{\infty} (-1)^{n+1}(n+1)\ z^{n}\right)=\\
                &= 1-2\cdot \left(\sum_{n=0}^{\infty} (-1)^n\ z^n\right) + \left(\sum_{n=0}^{\infty} (-1)^{n}(n+1)\ z^{n}\right)= \\
                &= 1+\left(\sum_{n=0}^{\infty} (-1)^n\left(n-1\right)\ z^{n}\right)
                = \sum_{n=2}^{\infty} (-1)^n\left(n-1\right)\ z^{n} \qquad \forall z \in D(0,1)
            \end{align*}

            \item[Usando el Ejercicio~\ref{ej:8.2}]
            
            Ya hemos visto que solo podemos aspirar a un desarrollo en serie de Taylor en $D(0,1)$. En ese conjunto, tenemos que:
            \begin{align*}
                f(z) &= z^2(1+z)^{-2} = z^2\cdot \left(\sum_{n=0}^{\infty} \binom{-2}{n}\cdot z^n\right)
                =\\&= \sum_{n=0}^{\infty} \binom{-2}{n}\cdot z^{n+2}
                = \sum_{n=2}^{\infty} \binom{-2}{n-2}\cdot z^{n} \qquad \forall z \in D(0,1)
            \end{align*}
        \end{description}
        
        
        
        \item $f(z) = \arcsen z \quad \forall z \in D(0,1)$
        
        Para definir el arcoseno, hacemos uso de que:
        \begin{equation*}
            \sen\left(z+\frac{\pi}{2}\right) = \cos z \qquad \forall z \in \bb{C}
        \end{equation*}

        Por tanto, para cada $z,w\in \bb{C}$, se tiene que:
        \begin{align*}
            \sen\left(z\right) = \cos \left(z-\dfrac{\pi}{2}\right) = w& \iff z-\frac{\pi}{2} \in -i\Log(w\pm \sqrt{w^2-1})
            \iff\\&\iff z \in \frac{\pi}{2} - i\Log(w\pm \sqrt{w^2-1})
        \end{align*}

        Por tanto, definimos el arcoseno complejo como:
        \Func{\arcsen}{\bb{C}}{\bb{C}}{w}{\dfrac{\pi}{2}-i\log\left(w+\sqrt{w^2-1}\right)}

        Tenemos que:
        \begin{equation*}
            \arcsen(w)=\dfrac{\pi}{2}-i\log\left(w+\exp\left(\dfrac{\log(w^2-1)}{2}\right)\right)
        \end{equation*}

        En cualquier caso, se puede probar que $\arcsen$ es holomorfa en el conjunto $\bb{C}\setminus \{x\in \bb{R}\mid |x|\geq 1\}$ (y en particular $D(0,1)$), con:
        \begin{equation*}
            \arcsen'(w)=\dfrac{1}{\sqrt{1-w^2}}=(1-w^2)^{-\nicefrac{1}{2}}
            \qquad \forall w\in D(0,1) 
        \end{equation*}

        Usando el Ejercicio~\ref{ej:8.2}, tenemos que:
        \begin{equation*}
            \arcsen'(w)=\sum_{n=0}^{\infty} (-1)^n\binom{\nicefrac{-1}{2}}{n}{w}^{2n}\qquad \forall w\in D(0,1)
        \end{equation*}

        Integrando término a término, tenemos que:
        \begin{equation*}
            \arcsen(z)=C+\sum_{n=0}^{\infty} (-1)^n\binom{\nicefrac{-1}{2}}{n}\cdot \dfrac{1}{2n+1}\cdot {z}^{2n+1}\qquad \forall z\in D(0,1)
        \end{equation*}

        Como $\arcsen(0)=0$, se tiene que $C=0$, por lo que:
        \begin{equation*}
            \arcsen(z)=\sum_{n=0}^{\infty} (-1)^n\binom{\nicefrac{-1}{2}}{n}\cdot \dfrac{1}{2n+1}\cdot {z}^{2n+1}\qquad \forall z\in D(0,1)
        \end{equation*}

        
        \item $f(z) = \cos^2 z \quad \forall z \in \bb{C}$
        
        Como $f\in \cc{H}(\bb{C})$, se tiene que $f$ es analítica en $\bb{C}$. Como será centrado en el origen y el coseno y el seno complejos extienden a las funciones seno y coseno reales, podríamos usar el desarrollo en serie de Taylor de las correspondientes funciones trigonométricas reales, pero preferimos hacerlo directamente desde $0$ para así practicar.        
        Veamos por inducción sobre $n$ que:
        \begin{equation*}
            f^{(2n)}(z) = (-1)^{n}\cdot 2^{2n-1}\cdot \cos(2z) \qquad \forall n \in \bb{N}, \forall z \in \bb{C}
        \end{equation*}
        \begin{itemize}
            \item Para $n=1$, se tiene que:
            \begin{align*}
                f'(z) &= -2\cos(z)\sen(z) = -\sen(2z)\\
                f''(z) &= -2\cos(2z) = (-1)^1\cdot 2^{1}\cdot \cos(2z)
            \end{align*}

            \item Supuesto cierto para $n$, veamos que es cierto para $n+1$.
            \begin{align*}
                f^{(2n)}(z) &= (-1)^{n}\cdot 2^{2n-1}\cdot \cos(2z)\\
                f^{(2n+1)}(z) &= (-1)^{n+1}\cdot 2^{2n}\cdot \sen(2z)\\
                f^{(2(n+1))}(z) = f^{(2n+2)}(z) &= (-1)^{n+1}\cdot 2^{2n+1}\cdot \cos(2z) = (-1)^{n+1}\cdot 2^{2(n+1)-1}\cdot \cos(2z)
            \end{align*}
        \end{itemize}

        Por tanto, queda demostrado para todo $n\in \bb{N}$. Evaluando en $z=0$, se tiene que:
        \begin{align*}
            f^{(2n)}(0) &= (-1)^{n}\cdot 2^{2n-1}\cdot \cos(0) = (-1)^{n}\cdot 2^{2n-1}\qquad \forall n \in \bb{N}
        \end{align*}

        Demostremos ahora por inducción que:
        \begin{equation*}
            f^{(2n-1)}(z) = (-1)^{n-1}\cdot 2^{2(n-1)}\cdot \sen(2z) \qquad \forall n \in \bb{N}, \forall z \in \bb{C}
        \end{equation*}
        \begin{itemize}
            \item Para $n=1$, se tiene que:
            \begin{align*}
                f'(z) &= -\sen(2z)=(-1)^0\cdot 2^0\cdot \sen(2z)
            \end{align*}

            \item Supuesto cierto para $n$, veamos que es cierto para $n+1$.
            \begin{align*}
                f^{(2n-1)}(z) &= (-1)^{n-1}\cdot 2^{2(n-1)}\cdot \sen(2z)\\
                f^{(2n)}(z) &= (-1)^{n-1}\cdot 2^{2(n-1)+1}\cdot \cos(2z)\\
                f^{(2(n+1)-1)}(z) = f^{(2n+1)}(z) &= (-1)^{n}\cdot 2^{2(n-1)+2}\cdot \sen(2z) = (-1)^{n}\cdot 2^{2(n)}\cdot \sen(2z)
            \end{align*}
        \end{itemize}

        Por tanto, queda demostrado para todo $n\in \bb{N}$. Evaluando en $z=0$, se tiene que:
        \begin{align*}
            f^{(2n-1)}(0) &= (-1)^{n-1}\cdot 2^{2(n-1)}\cdot \sen(0) = 0\qquad \forall n \in \bb{N}
        \end{align*}

        Por tanto, tenemos que:
        \begin{align*}
            f(z) &= \sum_{n=0}^{\infty} \dfrac{f^{(n)}(0)}{n!}z^n
            = f(0) + \sum_{n=0}^{\infty} \dfrac{f^{(2n)}(0)}{(2n)!}z^{2n} + \sum_{n=0}^{\infty} \dfrac{f^{(2n+1)}(0)}{(2n+1)!}z^{2n+1}\\
            &= 1+\sum_{n=1}^{\infty} \dfrac{(-1)^{n}\cdot 2^{2n-1}}{(2n)!}z^{2n}\qquad \forall z \in \bb{C}
        \end{align*}
    \end{enumerate}
\end{ejercicio}

\begin{ejercicio}
    Dado $\alpha \in \bb{C}^*\setminus \bb{N}$, probar que existe una única función $f$ tal que $f \in \cc{H}(D(0,1))$ verificando que:
    \begin{equation*}
        z f'(z) - \alpha f(z) = \frac{1}{1+z} \quad \forall z \in D(0,1)
    \end{equation*}

    Como $f\in \cc{H}(D(0,1))$, se tiene que $f$ es analítica en $D(0,1)$. Consideramos su serie de Taylor centrada en el origen:
    \begin{equation*}
        f(z) = \sum_{n=0}^{\infty} \alpha_n z^n \qquad \forall z \in D(0,1)
    \end{equation*}

    Por el Teorema de Holomorfía de funciones dadas como suma de series de potencias, se tiene que:
    \begin{equation*}
        f'(z) = \sum_{n=0}^{\infty} \alpha_n\cdot n\cdot z^{n-1} \qquad \forall z \in D(0,1)
    \end{equation*}

    Por tanto, la ecuación dada se puede reescribir como:
    \begin{align*}
        z\cdot \sum_{n=0}^{\infty} \alpha_n\cdot n\cdot z^{n-1} - \alpha\cdot \sum_{n=0}^{\infty} \alpha_n z^n &= \sum_{n=0}^{\infty} (n-\alpha)\alpha_n z^n
        = \dfrac{1}{1+z}=\sum_{n=0}^{\infty} (-1)^n z^n \qquad \forall z \in D(0,1)
    \end{align*}

    Por el Principio de Identidad de funciones analíticas, se tiene que:
    \begin{equation*}
        (n-\alpha)\alpha_n = (-1)^n\Longrightarrow
        \alpha_n = \frac{(-1)^n}{n-\alpha} \qquad \forall n \in \bb{N}
    \end{equation*}

    Por lo tanto, se tiene que:
    \begin{align*}
        f(z) &= \sum_{n=0}^{\infty}\dfrac{(-1)^n}{n-\alpha}\cdot z^n\qquad \forall z \in D(0,1)
    \end{align*}

    Además, esta función está bien definida en $D(0,1)$. Esto prueba la existencia y unicidad de la función $f$.    
\end{ejercicio}

\begin{ejercicio}
    Probar que existe una única función $f$ tal que $f \in \cc{H}(D(0,1))$ verificando que $f(0) = 0$ y:
    \begin{equation*}
        \exp\left({-z} f'(z)\right) = 1-z \quad \forall z \in D(0,1)
    \end{equation*}

    Como $f\in \cc{H}(D(0,1))$, se tiene que $f$ es analítica en $D(0,1)$. Consideramos su serie de Taylor centrada en el origen:
    \begin{equation*}
        f(z) = \sum_{n=0}^{\infty} \alpha_n z^n \qquad \forall z \in D(0,1)
    \end{equation*}

    Por el Teorema de Holomorfía de funciones dadas como suma de series de potencias, se tiene que:
    \begin{equation*}
        f'(z) = \sum_{n=1}^{\infty} \alpha_n\cdot n\cdot z^{n-1}
        = \sum_{n=0}^{\infty} \alpha_{n+1}\cdot (n+1)\cdot z^{n} \qquad \forall z \in D(0,1)
    \end{equation*}

    Por lo tanto, la ecuación dada se puede reescribir como:
    \begin{align*}
        \exp\left(\sum_{n=1}^{\infty} -\alpha_{n}\cdot n\cdot z^{n}\right) &= 1-z\qquad \forall z \in D(0,1)
    \end{align*}

    Derivando la ecuación anterior, se tiene que:
    \begin{multline*}
        (1-z)\cdot \sum_{n=1}^{\infty} -\alpha_{n}\cdot n^2\cdot z^{n-1} = -1
        \iff\\\iff \sum_{n=0}^{\infty} \alpha_{n+1}\cdot (n+1)^2\cdot z^{n} = \frac{1}{1-z}
        = \sum_{n=0}^{\infty} z^{n} \qquad \forall z \in D(0,1)
    \end{multline*}

    Por el Principio de Identidad de funciones analíticas, se tiene que:
    \begin{equation*}
        \alpha_{n+1}\cdot (n+1)^2 = 1\Longrightarrow \alpha_{n+1} = \frac{1}{(n+1)^2} \qquad \forall n \in \bb{N}_0
    \end{equation*}

    Nos falta por determinar $\alpha_0$. Tenemos por el momento que:
    \begin{align*}
        f(z) &= \sum_{n=0}^{\infty} \alpha_n z^n = \alpha_0 + \sum_{n=1}^{\infty} \frac{1}{n^2} z^n\qquad \forall z \in D(0,1)
    \end{align*}

    Como $f(0)=0$, se tiene que $\alpha_0=0$. Por lo tanto, se tiene que:
    \begin{align*}
        f(z) &= \sum_{n=1}^{\infty} \frac{1}{n^2} z^n\qquad \forall z \in D(0,1)
    \end{align*}

    Además, esta función está bien definida en $D(0,1)$, puesto que la serie converge, puesto que su radio de convergencia es $1$.

    Esto prueba la existencia y unicidad de la función $f$.
\end{ejercicio}

\begin{ejercicio}
    Para $z \in \bb{C}$ con $1-z-z^2 \neq 0$ se define $f(z) = (1-z-z^2)^{-1}$. Sea $\sum\limits_{n>0} \alpha_n z^n$ la serie de Taylor de $f$ centrada en el origen. Probar que $\{\alpha_n\}$ es la sucesión de Fibonacci:
    \begin{equation*}
        \alpha_0 = \alpha_1 = 1 \quad \text{y} \quad \alpha_{n+2} = \alpha_n + \alpha_{n+1} \quad \forall n \in \bb{N}\cup\{0\}
    \end{equation*}
    Calcular en forma explícita dicha sucesión.\\

    En primer lugar, hemos de determinar en qué conjunto es holomorfa la función $f$. Para ello, resolvemos la ecuación:
    \begin{equation*}
        1-z-z^2 = 0\iff z^2+z-1=0\iff z=\frac{-1\pm\sqrt{1+4}}{2}=\frac{-1\pm\sqrt{5}}{2}
    \end{equation*}

    Por lo tanto, $f$ es holomorfa en $\bb{C}\setminus\left\{\frac{-1\pm\sqrt{5}}{2}\right\}$. Por tanto, podemos aspirar a un desarrollo en serie de Taylor centrado en el origen en $D\left(0,R\right)$,l donde:
    \begin{equation*}
        R=\min\left\{\left|\frac{-1+\sqrt{5}}{2}\right|,\left|\frac{-1-\sqrt{5}}{2}\right|\right\}
        =\frac{-1+\sqrt{5}}{2}
    \end{equation*}
    
    Descomponemos la función en fracciones simples:
    \begin{equation*}
        \frac{1}{1-z-z^2} = -\frac{A}{z-\frac{-1+\sqrt{5}}{2}} - \frac{B}{z-\frac{-1-\sqrt{5}}{2}}
        = \frac{A(z-\frac{-1-\sqrt{5}}{2})+B(z-\frac{-1+\sqrt{5}}{2})}{-(z-\frac{-1+\sqrt{5}}{2})(z-\frac{-1-\sqrt{5}}{2})}
    \end{equation*}

    \begin{itemize}
        \item Para $z=\frac{-1+\sqrt{5}}{2}$, se tiene que $1=A\cdot \sqrt{5}$, y por lo tanto $A=\frac{1}{\sqrt{5}}$.
        \item Para $z=\frac{-1-\sqrt{5}}{2}$, se tiene que $1=B\cdot (-\sqrt{5})$, y por lo tanto $B=-\frac{1}{\sqrt{5}}$.
    \end{itemize}

    Por tanto, para cada $z\in D\left(0,R\right)$, se tiene que:
    \begin{align*}
        f(z) &= -\frac{1}{\sqrt{5}}\cdot \left(\frac{1}{z-\frac{-1+\sqrt{5}}{2}}-\frac{1}{z-\frac{-1-\sqrt{5}}{2}}\right)
        =\\&= -\frac{1}{\sqrt{5}}\cdot \left(\frac{-1}{\frac{-1+\sqrt{5}}{2}}\cdot \frac{1}{1-\frac{z}{\frac{-1+\sqrt{5}}{2}}}-\frac{1}{\frac{1+\sqrt{5}}{2}}\cdot \frac{1}{1+\frac{z}{\frac{1+\sqrt{5}}{2}}}\right) \\
        &= -\frac{1}{\sqrt{5}}\cdot \left(\frac{-1}{\frac{-1+\sqrt{5}}{2}}\cdot \sum_{n=0}^{\infty} \left(\frac{z}{\frac{-1+\sqrt{5}}{2}}\right)^n-\frac{1}{\frac{1+\sqrt{5}}{2}}\cdot \sum_{n=0}^{\infty} \left(-\frac{z}{\frac{1+\sqrt{5}}{2}}\right)^n\right)
    \end{align*}
    donde podemos considerar dichas series convergentes, puesto que $|z|<R\leq \frac{-1+\sqrt{5}}{2}, \frac{1+\sqrt{5}}{2}$. Por lo tanto, para cada $z\in D\left(0,R\right)$ se tiene que:
    \begin{align*}
        f(z) &= \frac{1}{\sqrt{5}}\cdot \left(\frac{1}{\frac{-1+\sqrt{5}}{2}}\cdot \sum_{n=0}^{\infty} \left(\frac{z}{\frac{-1+\sqrt{5}}{2}}\right)^n+\frac{1}{\frac{1+\sqrt{5}}{2}}\cdot \sum_{n=0}^{\infty} \left(-\frac{z}{\frac{1+\sqrt{5}}{2}}\right)^n\right)
        =\\&= \frac{1}{\sqrt{5}}\cdot \left(\sum_{n=0}^{\infty} \left(\frac{2}{-1+\sqrt{5}}\right)^{n+1}z^n-\sum_{n=0}^{\infty} \left(-\frac{2}{1+\sqrt{5}}\right)^{n+1}z^n\right)
        =\\&= \sum_{n=0}^{\infty} \left(\frac{1}{\sqrt{5}}\cdot \left(\frac{2}{-1+\sqrt{5}}\right)^{n+1}-\frac{1}{\sqrt{5}}\cdot \left(-\frac{2}{1+\sqrt{5}}\right)^{n+1}\right)z^n
    \end{align*}

    Por lo tanto, para cada $n\in \bb{N}_0$, se tiene que:
    \begin{align*}
        \alpha_n &= \frac{1}{\sqrt{5}}\cdot \left(\frac{2}{-1+\sqrt{5}}\right)^{n+1}-\frac{1}{\sqrt{5}}\cdot \left(-\frac{2}{1+\sqrt{5}}\right)^{n+1}
    \end{align*}

    Por tanto, ya tenemos la sucesión $\{\alpha_n\}$ de forma explícita. Tenemos que:
    \begin{align*}
        \alpha_0 &= \frac{1}{\sqrt{5}}\cdot \left(\frac{2}{-1+\sqrt{5}}\right)+\frac{1}{\sqrt{5}}\cdot \left(\frac{2}{1+\sqrt{5}}\right) = \dfrac{2}{\sqrt{5}}\cdot \dfrac{1+\sqrt{5}-1+\sqrt{5}}{5-1} = 1\\
        \alpha_1 &= \frac{1}{\sqrt{5}}\cdot \left(\frac{2}{-1+\sqrt{5}}\right)^2-\frac{1}{\sqrt{5}}\cdot \left(-\frac{2}{1+\sqrt{5}}\right)^2 = \dfrac{4}{\sqrt{5}}\cdot \dfrac{(1+\sqrt{5})^2-(-1+\sqrt{5})^2}{(-1+\sqrt{5})^2(1+\sqrt{5})^2}
        =\\&= \dfrac{4}{\sqrt{5}}\cdot \dfrac{(1+\sqrt{5}-1+\sqrt{5})(1+\sqrt{5}+1-\sqrt{5})}{[(-1+\sqrt{5})(1+\sqrt{5})]^2}
        = \dfrac{4}{\sqrt{5}}\cdot \dfrac{4\sqrt{5}}{(5-1)^2}=1
    \end{align*}

    Calculemos ahora $\alpha_{n+1}+\alpha_{n}$ fijado un $n\in \bb{N}_0$:
    \begin{align*}
        \alpha_{n+1}+\alpha_{n} &= \frac{1}{\sqrt{5}}\cdot \left(\frac{2}{-1+\sqrt{5}}\right)^{n+2}-\frac{1}{\sqrt{5}}\cdot \left(-\frac{2}{1+\sqrt{5}}\right)^{n+2} +\\&\qquad +
        \frac{1}{\sqrt{5}}\cdot \left(\frac{2}{-1+\sqrt{5}}\right)^{n+1}-\frac{1}{\sqrt{5}}\cdot \left(-\frac{2}{1+\sqrt{5}}\right)^{n+1} =\\
        &= \frac{1}{\sqrt{5}}\cdot \left[\left(\frac{2}{-1+\sqrt{5}}\right)^{n+2}+\left(\frac{2}{-1+\sqrt{5}}\right)^{n+1} - \left(-\frac{2}{1+\sqrt{5}}\right)^{n+2}-\left(-\frac{2}{1+\sqrt{5}}\right)^{n+1}\right]
        =\\&= \frac{1}{\sqrt{5}}\cdot \left[\left(\frac{2}{-1+\sqrt{5}}\right)^{n+1}\left(\frac{2}{-1+\sqrt{5}}+1\right) - \left(-\frac{2}{1+\sqrt{5}}\right)^{n+1}\left(-\frac{2}{1+\sqrt{5}}+1\right)\right]
    \end{align*}

    Además, tenemos que:
    \begin{align*}
        \left(\frac{2}{-1+\sqrt{5}}\right)^2 &= \dfrac{4}{1+5-2\sqrt{5}} = \dfrac{4}{6-2\sqrt{5}} = \dfrac{2(3+\sqrt{5})}{9-5}=\dfrac{3+\sqrt{5}}{2}\\
        \dfrac{2}{-1+\sqrt{5}}+1 &= \dfrac{2-1+\sqrt{5}}{-1+\sqrt{5}} = \dfrac{1+\sqrt{5}}{-1+\sqrt{5}} = \frac{-(1+\sqrt{5})^2}{1-5}=\frac{1+5+2\sqrt{5}}{4} = \frac{3+\sqrt{5}}{2}\\
        \left(-\frac{2}{1+\sqrt{5}}\right)^2 &= \dfrac{4}{1+5+2\sqrt{5}} = \dfrac{4}{6+2\sqrt{5}} = \dfrac{2(3-\sqrt{5})}{9-5}=\dfrac{3-\sqrt{5}}{2}\\
        -\dfrac{2}{1+\sqrt{5}}+1 &= \dfrac{-2+1+\sqrt{5}}{1+\sqrt{5}} = \dfrac{-1+\sqrt{5}}{1+\sqrt{5}} = \frac{-(-1+\sqrt{5})^2}{1-5}=\frac{1+5-2\sqrt{5}}{4} = \frac{3-\sqrt{5}}{2}
    \end{align*}

    Por lo tanto, se tiene que:
    \begin{align*}
        \alpha_{n+1}+\alpha_{n} &= \frac{1}{\sqrt{5}}\cdot \left[\left(\frac{2}{-1+\sqrt{5}}\right)^{n+3}-\left(-\frac{2}{1+\sqrt{5}}\right)^{n+3}\right]
        = \alpha_{n+2}
        \qquad \forall n \in \bb{N}_0
    \end{align*}


\end{ejercicio}

\begin{ejercicio}
    En cada uno de los siguientes casos, decidir si existe una función $f$ tal que $f \in \cc{H}(\Omega)$ verificando que $f^{(n)}(0) = a_n$ para todo $n \in \bb{N}$:
    \begin{enumerate}
        \item $\Omega = \bb{C}$,\qquad $a_n = n$
        
        Supongamos que existe. Entonces, el desarrollo en serie de Taylor de $f$ centrado en el origen necesariamente tiene que ser:
        \begin{equation*}
            f(z) = \sum_{n=0}^{\infty} \dfrac{f^{(n)}(0)}{n!}z^n = \sum_{n=0}^{\infty} \dfrac{a_n}{n!}z^n = \sum_{n=0}^{\infty} \dfrac{n}{n!}z^n =\sum_{n=1}^{\infty} \dfrac{n}{n!}z^n =
            \sum_{n=1}^{\infty} \dfrac{1}{(n-1)!}z^n\qquad \forall z \in \bb{C}
        \end{equation*}

        Comprobemos que efectivamente $f\in \cc{H}(\bb{C})$. Calculamos su radio de convergencia:
        \begin{equation*}
            \left\{\dfrac{1}{n!}\cdot \dfrac{(n-1)!}{1}\right\}=\left\{\dfrac{1}{n}\right\} \to 0 \Longrightarrow \left\{\sqrt[n]{\dfrac{1}{(n-1)!}}\right\} \to 0 \Longrightarrow R=\infty
        \end{equation*}

        Por tanto, efectivamente $f\in \cc{H}(\bb{C})$ y verifica la condición dada.
        \item $\Omega = \bb{C}$,\qquad $a_n = (n+1)!$
        
        Supongamos que existe. Entonces, el desarrollo en serie de Taylor de $f$ centrado en el origen necesariamente tiene que ser:
        \begin{equation*}
            f(z) = \sum_{n=0}^{\infty} \dfrac{f^{(n)}(0)}{n!}z^n = \sum_{n=0}^{\infty} \dfrac{a_n}{n!}z^n = \sum_{n=0}^{\infty} (n+1)\ z^n\qquad 
            \forall z \in \bb{C}
        \end{equation*}

        Calculemos su radio de convergencia $R$:
        \begin{equation*}
            \left\{\sqrt[n]{n+1}\right\}\to 1\Longrightarrow R=\frac{1}{1} = 1
        \end{equation*}

        Por tanto, $f$ no converge en $\bb{C}\setminus \ol{D}(0,1)$, en contradicción con que $f\in \cc{H}(\bb{C})$. Por tanto, no existe tal función $f$.
        \item $\Omega = D(0,1)$,\qquad $a_n = 2^n n!$
        
        Supongamos que existe. Entonces, el desarrollo en serie de Taylor de $f$ centrado en el origen necesariamente tiene que ser:
        \begin{equation*}
            f(z) = \sum_{n=0}^{\infty} \dfrac{f^{(n)}(0)}{n!}z^n = \sum_{n=0}^{\infty} \dfrac{a_n}{n!}z^n = \sum_{n=0}^{\infty} 2^n\ z^n\qquad 
            \forall z \in D(0,1)
        \end{equation*}

        Calculemos su radio de convergencia $R$:
        \begin{equation*}
            \left\{\sqrt[n]{2^n}\right\}=\{2\}\to 2\Longrightarrow R=\frac{1}{2}
        \end{equation*}

        Por tanto, $f$ no converge en $\bb{C}\setminus \ol{D}(0,\nicefrac{1}{2})$, en contradicción con $f\in \cc{H}(D(0,1))$. Por tanto, no existe tal función $f$.
        \item $\Omega = D(0,\nicefrac{1}{2})$,\qquad $a_n = n^n$
        
        Supongamos que existe. Entonces, el desarrollo en serie de Taylor de $f$ centrado en el origen necesariamente tiene que ser:
        \begin{equation*}
            f(z) = \sum_{n=0}^{\infty} \dfrac{f^{(n)}(0)}{n!}z^n = \sum_{n=0}^{\infty} \dfrac{a_n}{n!}z^n = \sum_{n=0}^{\infty} \dfrac{n^n}{n!}\ z^n\qquad
            \forall z \in D(0,\nicefrac{1}{2})
        \end{equation*}

        Calculemos su radio de convergencia $R$:
        \begin{equation*}
            \left\{\dfrac{(n+1)^{n+1}}{(n+1)!}\cdot \dfrac{n!}{n^n}\right\}=\left\{\left(\dfrac{n+1}{n}\right)^{n}\right\}=\left\{\left(1+\dfrac{1}{n}\right)^{n}\right\} \to e\Longrightarrow R=\frac{1}{e}<\frac{1}{2}
        \end{equation*}

        Por tanto, $f$ no converge en $\bb{C}\setminus \ol{D}(0,\nicefrac{1}{e})$, en contradicción con $f\in \cc{H}(D(0,\nicefrac{1}{2}))$. Por tanto, no existe tal función $f$.
    \end{enumerate}
\end{ejercicio}

\begin{ejercicio}
    Dados $r \in \bb{R}^+$, $k \in \bb{N}$, y $a,b \in \bb{C}$ con $|b| < r < |a|$, calcular la siguiente integral:
    \begin{equation*}
        \int_{C(0,r)} \frac{dz}{(z-a)(z-b)^k}
    \end{equation*}

    Descomponemos en primer lugar en fracciones simples. Sabemos que existen $A,B_1,\ldots,B_k \in \bb{C}$ tales que:
    \begin{equation*}
        \frac{1}{(z-a)(z-b)^k} = \frac{A}{z-a} + \sum\limits_{i=1}^{k} \frac{B_i}{(z-b)^i} = \frac{A(z-b)^k + \sum\limits_{i=1}^{k} B_i(z-a)(z-b)^{k-i}}{(z-a)(z-b)^k}
    \end{equation*}

    Por el Ejercicio~\ref{ej:7.1}, como $r<|a|$, sabemos que:
    \begin{equation*}
        \int_{C(0,r)} \frac{dz}{(z-a)}=0
    \end{equation*}

    Para $i\in \{2,\ldots,k\}$, tenemos que:
    \begin{equation*}
        \int_{C(0,r)} \frac{dz}{(z-b)^i} = 0
    \end{equation*}
    puesto que el integrando admite una primitiva en $\bb{C}\setminus\{b\}$, $b\notin C(0,r)^*$ y $C(0,r)$ es un camino cerrado. Por tanto, el único término que no se anula es el de $i=1$. Por tanto, tenemos que:
    \begin{align*}
        \int_{C(0,r)} \frac{dz}{(z-a)(z-b)^k} &= B_1\int_{C(0,r)} \frac{dz}{z-b}
    \end{align*}

    Aplicamos ahora la Fórmula de Cauchy para la circunferencia, aplicado a la función constantemente igual a $1$ (que es entera). Como $b\in D(0,r)$, tenemos que:
    \begin{equation*}
        \int_{C(0,r)} \frac{dz}{(z-a)(z-b)^k} = B_1\cdot 2\pi i
    \end{equation*}

    Para obtener el valor de $B_1$, tras igualar los numeradores de la igualdad de fracciones simples, igualamos los coeficientes de $z^{k}$ y vemos que:
    \begin{equation*}
        0 = A+B_1
    \end{equation*}

    Por otro lado, tras igualar los denominadores evaluamos en $z=a$, y obtenemos:
    \begin{equation*}
        A=\frac{1}{(a-b)^k} \Longrightarrow B_1 = -\frac{1}{(a-b)^k}
    \end{equation*}

    Por tanto, tenemos que:
    \begin{align*}
        \int_{C(0,r)} \frac{dz}{(z-a)(z-b)^k} &= -\frac{2\pi i}{(a-b)^k}
    \end{align*}
\end{ejercicio}

\begin{ejercicio}
    Calcular la integral para cada una de las siguientes curvas:
    \begin{equation*}
        \int_{\gamma} \frac{e^z}{z^2(z-1)}dz
    \end{equation*}
    \begin{enumerate}
        \item $\gamma = C(\nicefrac{1}{4},\nicefrac{1}{2})$
        
        En primer lugar, descomponemos en fracciones simples:
        \begin{equation*}
            \frac{1}{z^2(z-1)} = \frac{A}{z} + \frac{B}{z^2} + \frac{C}{z-1} = \frac{Az(z-1)+B(z-1)+Cz^2}{z^2(z-1)}
        \end{equation*}
        \begin{itemize}
            \item Para $z=0$, se tiene que $1=B\cdot (-1)$, y por lo tanto $B=-1$.
            \item Para $z=1$, se tiene que $1=C\cdot 1^2$, y por lo tanto $C=1$.
            \item Igualando los coeficientes de $z^2$, se tiene que $0=A+C$, por lo que $A=-C=-1$.
        \end{itemize}

        Por tanto, tenemos que:
        \begin{equation*}
            \frac{1}{z^2(z-1)} = -\frac{1}{z} - \frac{1}{z^2} + \frac{1}{z-1}
        \end{equation*}

        Por tanto, tenemos que:
        \begin{align*}
            \int_{\gamma} \frac{e^z}{z^2(z-1)}dz &= -\int_{\gamma} \frac{e^z}{z}dz - \int_{\gamma} \frac{e^z}{z^2}dz + \int_{\gamma} \frac{e^z}{z-1}dz
        \end{align*}

        Centrándonos en el caso de $\gamma = C(\nicefrac{1}{4},\nicefrac{1}{2})$, como la exponencial es entera y $0\in D(\nicefrac{1}{4},\nicefrac{1}{2})$, por la Fórmula de Cauchy para la circunferencia, tenemos que:
        \begin{equation*}
            \int_{\gamma} \frac{e^z}{z}dz = 2\pi i e^0 = 2\pi i
        \end{equation*}

        Respecto a la segunda integral, como la exponencial es entera y $0\in D(\nicefrac{1}{4},\nicefrac{1}{2})$, por la Fórmula de Cauchy para las derivadas tenemos que:
        \begin{equation*}
            \int_{\gamma} \frac{e^z}{z^2}dz = 2\pi i e^0 = 2\pi i
        \end{equation*}

        Respecto a la tercera integral, sabemos que el integrando es holomorfo en el conjunto $\bb{C}\setminus \{1\}$, y en particular lo es en $D(0,1)$. Como $D(0,1)$ es estrellado, por el Teorema Local de Cauchy el integrando admite una primitiva en $D(0,1)$. Como $D(\nicefrac{1}{4},\nicefrac{1}{2})$ es un camino cerrado en $D(0,1)$, se tiene que:
        \begin{equation*}
            \int_{\gamma} \frac{e^z}{z-1}dz = 0
        \end{equation*}

        Por tanto, tenemos que:
        \begin{align*}
            \int_{\gamma} \frac{e^z}{z^2(z-1)}dz &= -\int_{\gamma} \frac{e^z}{z}dz - \int_{\gamma} \frac{e^z}{z^2}dz + \int_{\gamma} \frac{e^z}{z-1}dz\\
            &= -2\pi i - 2\pi i + 0 = -4\pi i
        \end{align*}


        \item $\gamma = C(1,\nicefrac{1}{2})$
        
        El integrando tanto de la primera como de la segunda integral es holomorfo en $\bb{C}\setminus\{0\}$, y en particular lo es en $D(1,1)$. Como dicho conjunto es estrellado, por el Teorema Local de Cauchy el integrando admite una primitiva en $D(1,1)$. Como $C(1,\nicefrac{1}{2})$ es un camino cerrado en $D(1,1)$, se tiene que:
        \begin{equation*}
            \int_{\gamma} \frac{e^z}{z}dz = \int_{\gamma} \frac{e^z}{z^2}dz = 0
        \end{equation*}

        Respecto a la tercera integral, como la exponencial es entera y $1\in D(1,\nicefrac{1}{2})$, por la Fórmula de Cauchy para la circunferencia, tenemos que:
        \begin{equation*}
            \int_{\gamma} \frac{e^z}{z-1}dz = 2\pi i e^1 = 2\pi e i
        \end{equation*}

        Por tanto, tenemos que:
        \begin{align*}
            \int_{\gamma} \frac{e^z}{z^2(z-1)}dz &= -\int_{\gamma} \frac{e^z}{z}dz - \int_{\gamma} \frac{e^z}{z^2}dz + \int_{\gamma} \frac{e^z}{z-1}dz\\
            &= 0 + 0 + 2\pi e i = 2\pi e i
        \end{align*}
        \item $\gamma = C(0,2)$
        
        Como $0\in D(0,2)$, por la Fórmula de Cauchy para la circunferencia, tenemos que:
        \begin{align*}
            \int_{\gamma} \frac{e^z}{z}dz &= 2\pi i e^0 = 2\pi i\\
            \int_{\gamma} \frac{e^z}{z-1}dz &= 2\pi i e^1 = 2\pi e i
        \end{align*}

        Por la Fórmula de Cauchy para las derivadas, tenemos que:
        \begin{align*}
            \int_{\gamma} \frac{e^z}{z^2}dz &= 2\pi i e^0 = 2\pi i
        \end{align*}

        Por tanto, tenemos que:
        \begin{align*}
            \int_{\gamma} \frac{e^z}{z^2(z-1)}dz &= -\int_{\gamma} \frac{e^z}{z}dz - \int_{\gamma} \frac{e^z}{z^2}dz + \int_{\gamma} \frac{e^z}{z-1}dz\\
            &= -2\pi i - 2\pi i + 2\pi e i = 2\pi(e-2)i
        \end{align*}
    \end{enumerate}
\end{ejercicio}

\begin{ejercicio}
    Dado $n \in \bb{N}$, calcular las siguientes integrales:
    \begin{enumerate}
        \item $\displaystyle\int_{C(0,1)} \frac{\sen z}{z^n}\ dz$
        
        Como el seno es una función entera y $0\in D(0,1)$, por la Fórmula de Cauchy para las derivadas se tiene que:
        \begin{equation*}
            \int_{C(0,1)} \frac{\sen z}{z^n}\ dz = \dfrac{2\pi i}{(n-1)!}\cdot \sen^{(n-1)}(0)
        \end{equation*}

        Calculando las derivadas sucesivas de la función seno (compruébese mediante una sencilla inducción), se tiene que:
        \begin{equation*}
            \sen^{(n)}(0) = \begin{cases}
                0 & \text{si } n\text{ es par}\\
                (-1)^{\frac{n-1}{2}} & \text{si } n\text{ es impar}
            \end{cases}
        \end{equation*}

        Por tanto, se tiene que:
        \begin{equation*}
            \int_{C(0,1)} \frac{\sen z}{z^n}\ dz = \begin{cases}
                0 & \text{si } n\text{ es impar}\\
                \dfrac{2\pi i}{(n-1)!}\cdot (-1)^{\frac{n}{2}-1} & \text{si } n\text{ es par}
            \end{cases}
        \end{equation*}
        \item $\displaystyle\int_{C(0,1)} \frac{e^z - e^{-z}}{z^n}\ dz$
        
        Multiplicamos y dividimos por $2$:
        \begin{equation*}
            \int_{C(0,1)} \frac{e^z - e^{-z}}{z^n}\ dz = 2\int_{C(0,1)} \frac{\senh z}{z^n}\ dz
        \end{equation*}

        Como la función seno hiperbólico es una función entera y $0\in D(0,1)$, por la Fórmula de Cauchy para las derivadas se tiene que:
        \begin{equation*}
            \int_{C(0,1)} \frac{\senh z}{z^n}\ dz = \dfrac{2\pi i}{(n-1)!}\cdot \senh^{(n-1)}(0)
        \end{equation*}

        Calculando las derivadas sucesivas de la función seno hiperbólico (compruébese mediante una sencilla inducción), se tiene que:
        \begin{equation*}
            \senh^{(n)}(z) = \begin{cases}
                \cosh z & \text{si } n\text{ es impar}\\
                \senh z & \text{si } n\text{ es par}
            \end{cases}
        \end{equation*}

        Evaluando en $0$, se tiene que:
        \begin{equation*}
            \senh^{(n)}(0) = \begin{cases}
                e^0=1 & \text{si } n\text{ es impar}\\
                0 & \text{si } n\text{ es par}
            \end{cases}
        \end{equation*}

        Por tanto, se tiene que:
        \begin{equation*}
            \int_{C(0,1)} \frac{e^z - e^{-z}}{z^n}\ dz = \begin{cases}
                \frac{4\pi i}{(n-1)!} & \text{si } n\text{ es par}\\
                0 & \text{si } n\text{ es impar}
            \end{cases}
        \end{equation*}
            
        \item $\displaystyle\int_{C(0,\nicefrac{1}{2})} \frac{\log(1+z)}{z^n}\ dz$
        
        Definimos la siguiente función auxiliar:
        \Func{f}{D(0,1)}{\bb{C}}{z}{\log(1+z)}

        Sabemos que $f\in \cc{H}(D(0,1))$. Demostramos por inducción que:
        \begin{align*}
            f^{(n)}(z) &= \frac{(-1)^{n-1}(n-1)!}{(1+z)^n} \qquad \forall z\in D(0,1),\ n\in \bb{N}
        \end{align*}
        \begin{itemize}
            \item Para $n=1$, se tiene que:
            \begin{align*}
                f^{(1)}(z) &= \frac{1}{1+z} = \frac{(-1)^{0}\cdot 0!}{(1+z)^1} \qquad \forall z\in D(0,1)
            \end{align*}

            \item Supongamos cierto para $n$, y veámoslo para $n+1$:
            \begin{align*}
                f^{(n+1)}(z) &= -\dfrac{(-1)^{n-1}(n-1)!\cdot n(1+z)^{n-1}}{(1+z)^{2n}} = \dfrac{(-1)^{n}\cdot n!}{(1+z)^{n+1}} \qquad \forall z\in D(0,1)
            \end{align*}
        \end{itemize}

        Por tanto, queda probado para todo $n\in \bb{N}$.        
        Por otro lado, por la Fórmula de Cauchy para las derivadas se tiene que:
        \begin{align*}
            \int_{C(0,\nicefrac{1}{2})} \frac{\log(1+z)}{z^n}\ dz &= \dfrac{2\pi i}{(n-1)!}\cdot f^{(n-1)}(0) =\\&= \begin{cases}
                0 & \text{si } n=1\\
                \dfrac{2\pi i}{(n-1)!}\cdot \dfrac{(-1)^{n}(n-2)!}{(1)^{n-1}} = (-1)^n\cdot \dfrac{2\pi i}{n-1}& \text{si } n\geq 2
            \end{cases}
        \end{align*}
    \end{enumerate}
\end{ejercicio}

\begin{ejercicio}[Fórmula de cambio de variable]\label{ej:8.11}
    Si $\Omega$ es un abierto del plano, $\gamma$ un camino en $\Omega$ y $\varphi \in \cc{H}(\Omega)$, entonces $\varphi \circ \gamma$ es un camino y, para cualquier función $f$ que sea continua en $(\varphi\circ \gamma)^*$ se tiene:
    \begin{equation*}
        \int_{\varphi\circ\gamma} f(z)dz = \int_{\gamma} f(\varphi(w)) \varphi'(w)dw
    \end{equation*}~

    Veamos en primer lugar que $\varphi\circ\gamma$ es un camino. Como $\gamma$ es un camino, existe una partición
    \begin{equation*}
        P=\{t_0,t_1,\ldots,t_n\}
    \end{equation*}
    del intervalo de definición de $\gamma$ tal que, para cada $k\in \{1,\dots,n\}$, se tiene que $\gamma_{\big| [t_{k-1},t_k]}\in C^1([t_{k-1},t_k])$. Como $\gamma$ es un camino en $\Omega$, la siguiente composición tiene sentido:
    \begin{equation*}
        \left(\varphi\circ\gamma\right)_{\big| [t_{k-1},t_k]} = \varphi\circ\gamma_{\big| [t_{k-1},t_k]}\in C^1([t_{k-1},t_k])
    \end{equation*}
    donde, para afirmar que es de clase $1$, hemos hecho uso de que $\varphi\in \cc{H}(\Omega)$ y, por tanto, es analítica en $\Omega$, lo que implica que es infinitamente derivable en $\Omega$. Por tanto, $\varphi\circ\gamma$ es un camino. Aplicamos ahora la definición de integral sobre un camino:
    \begin{align*}
        \int_{\varphi\circ\gamma} f(z)dz &= \sum_{k=1}^{n} \int_{\left(\varphi\circ\gamma\right)_{\big| [t_{k-1},t_k]}} f(z)dz = \sum_{k=1}^{n} \int_{t_{k-1}}^{t_k} f\left((\varphi\circ\gamma)(t)\right) \cdot \left(\varphi\circ\gamma\right)'(t)dt
        =\\&= \sum_{k=1}^{n} \int_{t_{k-1}}^{t_k} f\left(\varphi(\gamma(t))\right) \cdot \varphi'(\gamma(t))\cdot \gamma'(t)dt
        = \sum_{k=1}^{n} \int_{\gamma_{\big| [t_{k-1},t_k]}} f\left(\varphi(w)\right) \cdot \varphi'(w)\ dw
    \end{align*}
\end{ejercicio}

\begin{ejercicio}
    Usar el resultado del ejercicio anterior para calcular las siguientes integrales:
    \begin{enumerate}
        \item $\displaystyle\int_{C(0,2)} \frac{dz}{z^2(z-1)^2}$
        
        Buscamos una función $\varphi$ que nos permita transformar el integrando de forma que el denominador tan solo se anule en un punto. Notemos que, dado $a^*\in \bb{C}$, se tiene que:
        \begin{align*}
            |a|<r&\Longrightarrow \dfrac{1}{|a|}>\frac{1}{r}\\
            |a|>r&\Longrightarrow \dfrac{1}{|a|}<\frac{1}{r}
        \end{align*}

        Por tanto, en este caso tenemos $\varphi\circ \gamma=C(0,2)$, y buscamos $\gamma=C(0,\nicefrac{1}{2})$ (o similar). Esto nos permitirá que ambos puntos ($0$ y $1$) dejen de ser problemáticos, y podremos usar el Teorema Local de Cauchy. Una vez introducida la intuición, definimos:
        \Func{\varphi}{\bb{C}^*}{\bb{C}}{w}{\nicefrac{1}{w}}

        Supongamos $\gamma=C(0,\nicefrac{1}{2})$. Entonces, se tiene que:
        \begin{align*}
            (\varphi\circ\gamma)(t) &= \varphi\left(\gamma(t)\right) = \varphi\left(\frac{1}{2}e^{it}\right) = \frac{2}{e^{it}} = 2e^{-it}\qquad \forall t\in [-\pi,\pi]
        \end{align*}

        Por tanto, $\varphi\circ\gamma=-C(0,2)$. Por tanto, se tiene que:
        \begin{align*}
            \int_{C(0,2)} \frac{dz}{z^2(z-1)^2} &= -\int_{\varphi\circ\gamma}\frac{dz}{z^2(z-1)^2} \AstIg -\int_{\gamma}\frac{1}{\varphi(w)^2(\varphi(w)-1)^2} \cdot \varphi'(w)\ dw
        \end{align*}
        donde en $(\ast)$ hemos usado el resultado del ejercicio~\ref{ej:8.11}. Por tanto, tenemos que:
        \begin{align*}
            \int_{C(0,2)} \frac{dz}{z^2(z-1)^2} &= -\int_{C(0,\nicefrac{1}{2})}\frac{w^2}{\left(\frac{1}{w}-1\right)^2} \cdot \left(-\frac{1}{w^2}\right) dw
            =\\&= \int_{C(0,\nicefrac{1}{2})}\frac{1}{\left(\frac{1-w}{w}\right)^2} dw
            =\\&= \int_{C(0,\nicefrac{1}{2})}\frac{w^2}{(1-w)^2} dw
        \end{align*}

        El integrando es una función racional y holomorfa en $D(0,\nicefrac{3}{4})$ (estrellado), y por el Teorema Local de Cauchy el integrando admite una primitiva en $D(0,\nicefrac{3}{4})$. Como $C(0,\nicefrac{1}{2})$ es un camino cerrado en $D(0,\nicefrac{3}{4})$, se tiene que:
        \begin{equation*}
            \int_{C(0,2)} \frac{dz}{z^2(z-1)^2} = \int_{C(0,\nicefrac{1}{2})}\frac{w^2}{(1-w)^2} dw = 0
        \end{equation*}

        \item $\displaystyle\int_{C(0,2)} \frac{dz}{(z-1)^2(z+1)^2(z-3)}$
        
        De igual forma, consideremos como $\varphi$ la función:
        \Func{\varphi}{\bb{C}^*}{\bb{C}}{w}{\nicefrac{1}{w}}

        Consideramos de nuevo $\gamma=C(0,\nicefrac{1}{2})$, de forma que $\varphi\circ\gamma=-C(0,2)$. Esto provocará que, de los tres puntos problemáticos $(1,-1,3)$, dos dejen de ser problemáticos, y tan solo el $3$ se transformará en el $\nicefrac{1}{3}$, que será problemático pero podremos aplicar la Fórmula de Cauchy para la circunferencia. Por tanto, tenemos que:
        \begin{align*}
            \int_{C(0,2)} \frac{dz}{(z-1)^2(z+1)^2(z-3)} &= -\int_{\varphi\circ\gamma}\frac{dz}{(z-1)^2(z+1)^2(z-3)} \AstIg\\&\AstIg -\int_{\gamma}\frac{1}{(\varphi(w)-1)^2(\varphi(w)+1)^2(\varphi(w)-3)} \cdot \varphi'(w)\ dw
        \end{align*}
        donde en $(\ast)$ hemos usado el resultado del ejercicio~\ref{ej:8.11}. Por tanto, tenemos que:
        \begin{align*}
            \int_{C(0,2)} \frac{dz}{(z-1)^2(z+1)^2(z-3)} &= -\int_{C(0,\nicefrac{1}{2})}\frac{1}{\left(\frac{1-w}{w}\right)^2\left(\frac{1+w}{w}\right)^2\left(\frac{1-3w}{w}\right)} \cdot \left(-\frac{1}{w^2}\right) dw
            =\\&= \int_{C(0,\nicefrac{1}{2})}\frac{1}{\left(\frac{1-w}{w}\right)^2\left(\frac{1+w}{w}\right)^2\left(\frac{1-3w}{w}\right)}  \cdot \left(\frac{1}{w^2}\right) dw
            =\\&= \int_{C(0,\nicefrac{1}{2})}\frac{w^3}{(1-w)^2(1+w)^2(1-3w)} dw
            =\\&= -\dfrac{1}{3}\cdot \int_{C(0,\nicefrac{1}{2})}\frac{w^3}{(1-w)^2(1+w)^2\left(w-\nicefrac{1}{3}\right)} dw
        \end{align*}

        Definimos la siguiente función auxiliar:
        \Func{f}{\bb{C}\setminus \{-1,1\}}{\bb{C}}{z}{\dfrac{z^3}{(1-z)^2(1+z)^2}}

        Por ser $f$ racional, $f\in \cc{H}(\bb{C}\setminus \{-1,1\})$. Como $\ol{D}(0,\nicefrac{1}{2})\subset \bb{C}\setminus \{-1,1\}$, por la Fórmula de Cauchy para la circunferencia, se tiene que:
        \begin{equation*}
            \int_{C(0,\nicefrac{1}{2})} \dfrac{f(w)}{w-\nicefrac{1}{3}}\ dw = 2\pi i\cdot f\left(\nicefrac{1}{3}\right) = 2\pi i \cdot \dfrac{\left(\nicefrac{1}{3}\right)^3}{(1-\nicefrac{1}{3})^2(1+\nicefrac{1}{3})^2} = 2\pi i \cdot \dfrac{\left(\nicefrac{1}{3}\right)^3}{\left(\nicefrac{2}{3}\right)^2\left(\nicefrac{4}{3}\right)^2} = 2\pi i \cdot \dfrac{3}{64}
        \end{equation*}

        Por tanto, tenemos que:
        \begin{align*}
            \int_{C(0,2)} \frac{dz}{(z-1)^2(z+1)^2(z-3)} &= -\dfrac{1}{3}\cdot \int_{C(0,\nicefrac{1}{2})}\frac{w^3}{(1-w)^2(1+w)^2\left(w-\nicefrac{1}{3}\right)} dw\\
            &= -\dfrac{1}{3}\cdot 2\pi i \cdot \dfrac{3}{64} = -\dfrac{\pi i}{32}
        \end{align*}
    \end{enumerate}
\end{ejercicio}

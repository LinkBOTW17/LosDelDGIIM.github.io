\section{Equivalencia entre analiticidad y holomorfía}

\begin{ejercicio}
    Sea $\gamma$ un camino y $\varphi : \gamma^* \to \bb{C}$ una función continua. Se define $f : \bb{C}\setminus \gamma^* \to \bb{C}$ por:
    \Func{f}{\bb{C}\setminus \gamma^*}{\bb{C}}{z}{\displaystyle\int_{\gamma} \frac{\varphi(w)}{w-z}dw}

    Probar que $f$ es una función analítica en $\bb{C}\setminus \gamma^*$ y que:
    \begin{equation*}
        f^{(k)}(z) = k! \int_{\gamma} \frac{\varphi(w)}{(w-z)^{k+1}}dw \quad \forall z \in \bb{C}\setminus \gamma^*, \forall k \in \bb{N}
    \end{equation*}
\end{ejercicio}

\begin{ejercicio}
    Para $\alpha \in \bb{C}$ se define:
    \begin{equation*}
        \binom{\alpha}{0} = 1 \quad \text{y} \quad \binom{\alpha}{n} = \frac{1}{n!} \prod_{j=0}^{n-1} (\alpha-j) = \frac{\alpha(\alpha-1)\cdots(\alpha-n+1)}{n!} \quad \forall n \in \bb{N}
    \end{equation*}
    Probar que:
    \begin{equation*}
        (1+z)^{\alpha} = \sum_{k=0}^{\infty} \binom{\alpha}{n} z^n \quad \forall z \in D(0,1)
    \end{equation*}
\end{ejercicio}

\begin{ejercicio}
    Obtener el desarrollo en serie de Taylor de la función $f$, centrado en el origen, en cada uno de los siguientes casos:
    \begin{enumerate}
        \item $f(z) = \log(z^2 - 3z + 2) \quad \forall z \in D(0,1)$
        \item $f(z) = \dfrac{z^2}{(z+1)^2} \quad \forall z \in \bb{C}\setminus\{-1\}$
        \item $f(z) = \arcsen z \quad \forall z \in D(0,1)$
        \item $f(z) = \cos^2 z \quad \forall z \in \bb{C}$
    \end{enumerate}
\end{ejercicio}

\begin{ejercicio}
    Dado $\alpha \in \bb{C}^*\setminus \bb{N}$, probar que existe una única función $f$ tal que $f \in \cc{H}(D(0,1))$ verificando que:
    \begin{equation*}
        z f'(z) - \alpha f(z) = \frac{1}{1+z} \quad \forall z \in D(0,1)
    \end{equation*}
\end{ejercicio}

\begin{ejercicio}
    Probar que existe una única función $f$ tal que $f \in \cc{H}(D(0,1))$ verificando que $f(0) = 0$ y:
    \begin{equation*}
        \exp\left({-z} f'(z)\right) = 1-z \quad \forall z \in D(0,1)
    \end{equation*}
\end{ejercicio}

\begin{ejercicio}
    Para $z \in \bb{C}$ con $1-z-z^2 \neq 0$ se define $f(z) = (1-z-z^2)^{-1}$. Sea $\sum\limits_{n>0} \alpha_n z^n$ la serie de Taylor de $f$ centrada en el origen. Probar que $\{\alpha_n\}$ es la sucesión de Fibonacci:
    \begin{equation*}
        \alpha_0 = \alpha_1 = 1 \quad \text{y} \quad \alpha_{n+2} = \alpha_n + \alpha_{n+1} \quad \forall n \in \bb{N}\cup\{0\}
    \end{equation*}
    Calcular en forma explícita dicha sucesión.
\end{ejercicio}

\begin{ejercicio}
    En cada uno de los siguientes casos, decidir si existe una función $f$ tal que $f \in \cc{H}(\Omega)$ verificando que $f^{(n)}(0) = a_n$ para todo $n \in \bb{N}$:
    \begin{enumerate}
        \item $\Omega = \bb{C}$,\qquad $a_n = n$
        \item $\Omega = \bb{C}$,\qquad $a_n = (n+1)!$
        \item $\Omega = D(0,1)$,\qquad $a_n = 2^n n!$
        \item $\Omega = D(0,\nicefrac{1}{2})$,\qquad $a_n = n^n$
    \end{enumerate}
\end{ejercicio}

\begin{ejercicio}
    Dados $r \in \bb{R}^+$, $k \in \bb{N}$, y $a,b \in \bb{C}$ con $|b| < r < |a|$, calcular la siguiente integral:
    \begin{equation*}
        \int_{C(0,r)} \frac{dz}{(z-a)(z-b)^k}
    \end{equation*}
\end{ejercicio}

\begin{ejercicio}
    Calcular la integral para cada una de las siguientes curvas:
    \begin{equation*}
        \int_{\gamma} \frac{e^z}{z^2(z-1)}dz
    \end{equation*}
    \begin{enumerate}
        \item $\gamma = C(\nicefrac{1}{4},\nicefrac{1}{2})$
        \item $\gamma = C(1,\nicefrac{1}{2})$
        \item $\gamma = C(0,2)$
    \end{enumerate}
\end{ejercicio}

\begin{ejercicio}
    Dado $n \in \bb{N}$, calcular las siguientes integrales:
    \begin{enumerate}
        \item $\displaystyle\int_{C(0,1)} \frac{\sen z}{z^n}dz$
        \item $\displaystyle\int_{C(0,1)} \frac{e^z - e^{-z}}{z^n}dz$
        \item $\displaystyle\int_{C(0,\nicefrac{1}{2})} \frac{\log(1+z)}{z^n}dz$
    \end{enumerate}
\end{ejercicio}

\begin{ejercicio}[Fórmula de cambio de variable]
    Si $\Omega$ es un abierto del plano, $\gamma$ un camino en $\Omega$ y $\varphi \in \cc{H}(\Omega)$, entonces $\varphi \circ \gamma$ es un camino y, para cualquier función $f$ que sea continua en $(\varphi\circ \gamma)^*$ se tiene:
    \begin{equation*}
        \int_{\varphi\circ\gamma} f(z)dz = \int_{\gamma} f(\varphi(w)) \varphi'(w)dw
    \end{equation*}
\end{ejercicio}

\begin{ejercicio}
    Usar el resultado del ejercicio anterior para calcular las siguientes integrales:
    \begin{enumerate}
        \item $\displaystyle\int_{C(0,2)} \frac{dz}{z^2(z-1)^2}$
        \item $\displaystyle\int_{C(0,2)} \frac{dz}{(z-1)^2(z+1)^2(z-3)}$
    \end{enumerate}
\end{ejercicio}
\section{Funciones Elementales}

\begin{ejercicio}
    Sea $f : \mathbb{C} \to \mathbb{C}$ una función verificando que
    \[
        f(z + w) = f(z)f(w) \quad \forall z,w \in \mathbb{C}
    \]
    Probar que, si $f$ es derivable en algún punto del plano, entonces $f$ es entera. Encontrar todas las funciones enteras que verifiquen la condición anterior. Dar un ejemplo de una función que verifique dicha condición y no sea entera.\\

    Sea $z_0\in \bb{C}$, y supongamos que $f$ es derivable en $z_0$. Por ser derivable en $z_0$, tenemos que:
    \begin{equation*}
        f'(z_0) = \lim_{h \to 0} \frac{f(z_0 + h) - f(z_0)}{h} \AstIg \lim_{h \to 0} \frac{f(z_0)f(h) - f(z_0)}{h} = f(z_0)\lim_{h \to 0} \frac{f(h) - 1}{h}
    \end{equation*}
    donde en $(\ast)$ hemos usado la propiedad de $f$. Caben dos casos:
    \begin{itemize}
        \item \ul{Si $f(z_0) = 0$}, entonces:
        \begin{align*}
            f(z)=f(z_0 + (z-z_0)) &= f(z_0)f(z-z_0) = 0 \quad \forall z \in \mathbb{C}
        \end{align*}
        Por tanto, $f$ es la función nula, por lo que $f\in \mathcal{H}(\mathbb{C})$.

        \item \ul{Si $f(z_0) \neq 0$}, entonces:
        \begin{equation*}
            \lim_{h \to 0} \frac{f(h) - 1}{h} = \frac{f'(z_0)}{f(z_0)} \in \mathbb{C}
        \end{equation*}

        Veamos ahora que $f$ es entera. Fijado $z\in \mathbb{C}$, se tiene:
        \begin{align*}
            f'(z) &= \lim_{h \to 0} \frac{f(z + h) - f(z)}{h} = \lim_{h \to 0} \frac{f(z)f(h) - f(z)}{h}
            = f(z)\lim_{h \to 0} \frac{f(h) - 1}{h} = f(z)\cdot \frac{f'(z_0)}{f(z_0)}
        \end{align*}

        Por lo tanto, $f\in \mathcal{H}(\mathbb{C})$.
    \end{itemize}
    En cualquier caso, $f$ es entera. Veamos ahora $f\in \cc{H}(\bb{C})$ cumple la condición del enunciado si y sólo si, $f=0$ o $f(z) = e^{\lm z}$ para algún $\lm \in \bb{C}$.
    \begin{description}
        \item[$\Longrightarrow$)] Sea $f\in \cc{H}(\bb{C})$, y sea $z_0\in \bb{C}$, por lo que $f$ es derivable en $z_0$. Por lo visto anteriormente, o bien $f$ es la función nula, o bien $f(z_0) \neq 0$ y se tiene:
        \begin{equation*}
            f'(z)=f(z)\cdot \frac{f'(z_0)}{f(z_0)} \quad \forall z\in \bb{C}
        \end{equation*}

        Definimos $\lm = \frac{f'(z_0)}{f(z_0)}\in \bb{C}$, y sea la siguiente función:
        \Func{g}{\bb{C}}{\bb{C}}{z}{f(z)e^{-\lm z}}

        Sabemos que $f\in \cc{H}(\bb{C})$, con:
        \begin{equation*}
            g'(z) = f'(z)e^{-\lm z} - \lm f(z)e^{-\lm z} = e^{-\lm z}\left(f'(z) - \lm f(z)\right) = 0\qquad \forall z\in \bb{C}
        \end{equation*}

        Por tanto, $g$ es constante. Tenemos que:
        \begin{equation*}
            g(0)=f(0)e^{-\lm\cdot 0} = f(0)
        \end{equation*}
        
        Por lo que $g(z) = f(0)$ para todo $z\in \bb{C}$. Por tanto lado, de la ecuación del enunciado, tenemos que:
        \begin{equation*}
            f(z_0)=f(z_0+0)=f(z_0)f(0) \Longrightarrow f(0)=1
        \end{equation*}

        Por tanto, $g$ es la función constante $1$, y se tiene que:
        \begin{equation*}
            1=g(z) = f(z)e^{-\lm z} \Longrightarrow f(z) = e^{\lm z} \quad \forall z\in \bb{C}
        \end{equation*}

        \item[$\Longleftarrow$)] Sea $f(z) = e^{\lm z}$, con $\lm \in \bb{C}$. Entonces, se tiene que:
        \begin{align*}
            f(z+w) &= e^{\lm(z+w)} = e^{\lm z}e^{\lm w} = f(z)f(w) \quad \forall z,w\in \bb{C}
        \end{align*}
    \end{description}

    % // TODO: Ejemplo. Funcion dirichlet
\end{ejercicio}

\begin{ejercicio}
    Calcular la imagen por la función exponencial de una banda horizontal o vertical y del dominio cuya frontera es un rectángulo de lados paralelos a los ejes.\\

    Sean $a,b,c,d\in \bb{R}$ tal que $a<b$ y $c<d$. Consideramos:
    \begin{itemize}
        \item La banda vertical siguiente:
        \begin{equation*}
            \Omega_1 = \{z\in \bb{C} : a \leq \Re z \leq b\} = [a,b]\times \bb{R}
        \end{equation*}

        \item La banda horizontal siguiente:
        \begin{equation*}
            \Omega_2 = \{z\in \bb{C} : c \leq \Im z \leq d\} = \bb{R}\times [c,d]
        \end{equation*}

        \item El rectángulo siguiente:
        \begin{equation*}
            \Omega_3 = \{z\in \bb{C} : a \leq \Re z \leq b, c \leq \Im z \leq d\} = [a,b]\times [c,d] = \Omega_1\cap \Omega_2
        \end{equation*}
    \end{itemize}

    Definimos ahora la función siguiente:
    \Func{f}{\bb{C}}{\bb{C}^*}{z}{e^z}

    Calculamos la imagen de cada uno de los dominios anteriores:
    \begin{itemize}
        \item La imagen de la banda vertical $\Omega_1$ es:
        \begin{align*}
            f(\Omega_1) &= \{e^z : z\in \Omega_1\} = \{e^{x+iy} : a\leq x\leq b, y\in \bb{R}\} \\
            &= \{e^x e^{iy} : a\leq x\leq b, y\in \bb{R}\}
        \end{align*}

        Veamos por doble inclusión que:
        \begin{equation*}
            f(\Omega_1) = \{w\in \bb{C} : |w|\in [e^a,e^b]\}
        \end{equation*}
        \begin{description}
            \item[$\subseteq$)] Sea $w\in f(\Omega_1)$, entonces existe $x\in [a,b]$ y $y\in \bb{R}$ tal que $w = e^{x+iy}$. Por tanto, se tiene que:
            \begin{equation*}
                |w| = |e^{x+iy}| = e^x \in [e^a,e^b]
            \end{equation*}
            
            \item[$\supseteq$)] Sea $w\in \bb{C}$ tal que $|w|\in [e^a,e^b]$. Entonces, y definimos $x=\ln |w|$, y $y = \arg w$. Por tanto, se tiene que:
            \begin{equation*}
                f(x+iy) = e^{x+iy} = e^x e^{iy} = |w|e^{i\arg w} = w
            \end{equation*}
            Por tanto, $w\in f(\Omega_1)$.
        \end{description}

        Por tanto, se tiene que $f(\Omega_1)$ es el anillo del plano complejo delimitado por las circunferencias de radio $e^a$ y $e^b$ centradas en el origen.

        \item La imagen de la banda horizontal $\Omega_2$ es:
        \begin{align*}
            f(\Omega_2) &= \{e^z : z\in \Omega_2\} = \{e^{x+iy} : x\in \bb{R}, c\leq y\leq d\} \\
            &= \{e^x e^{iy} : x\in \bb{R}, c\leq y\leq d\}
            =\\&= \{e^x\left(\cos y + i\sen y\right) : x\in \bb{R}, c\leq y\leq d\}
        \end{align*}

        Vemos por tanto que $f(\Omega_2)$ son los puntos del plano complejo que forman un sector angular del plano complejo delimitado por el origen y los ángulos $c$ y $d$. Si $[c,d]$ parametriza toda la circunferencia (es decir, $l(c,d)=d-c\geq 2\pi$), entonces $f(\Omega_2)=\bb{C}^*$. En caso contrario, será el sector angular correspondiente a la parametrización realidada por $[c,d]$.

        % // TODO: De forma formal?

        \item La imagen del rectángulo $\Omega_3$ es:
        \begin{align*}
            f(\Omega_3) &= f(\Omega_1\cap \Omega_2) = f(\Omega_1)\cap f(\Omega_2)
        \end{align*}

        Por tanto, se trata de la región del plano compleja delimitada por las circunferencias de radio $e^a$ y $e^b$, y los ángulos $c$ y $d$. Si $l(c,d)\geq 2\pi$, entonces $f(\Omega_3) = f(\Omega_1)$. En caso contrario, se trata del sector angular delimitado por los ángulos $c$ y $d$ y las circunferencias de radio $e^a$ y $e^b$.
    \end{itemize}
\end{ejercicio}

\begin{ejercicio}
    Dado $\theta\in \left] -\pi, \pi \right]$, estudiar la existencia del límite en $+\infty$ de la función siguiente:
    \Func{\varphi}{\mathbb{R}^+}{\mathbb{C}}{r}{e^{re^{i\theta}}}

    Tenemos que:
    \begin{align*}
        |\varphi(r)| &= |e^{re^{i\theta}}| = e^{r\cos\theta}
    \end{align*}

    Distinguimos casos:
    \begin{itemize}
        \item Si $\theta\in \left] -\nicefrac{\pi}{2}, \nicefrac{\pi}{2} \right[$, entonces $\cos\theta > 0$, y se tiene que:
        \begin{equation*}
            \lim_{r\to +\infty} |\varphi(r)| = \lim_{r\to +\infty} e^{r\cos\theta} = +\infty
        \end{equation*}

        Por tanto, $\varphi(r) \to \infty$ cuando $r\to +\infty$.

        \item Si $\theta \in \left]-\pi, -\nicefrac{\pi}{2} \right[ \cup \left]\nicefrac{\pi}{2}, \pi \right]$, entonces $\cos\theta < 0$, y se tiene que:
        \begin{equation*}
            \lim_{r\to +\infty} |\varphi(r)| = \lim_{r\to +\infty} e^{r\cos\theta} = 0
        \end{equation*}

        Por tanto, $\lim\limits_{r\to +\infty} \varphi(r) = 0$.

        \item Si $\theta = \pm \nicefrac{\pi}{2}$, entonces $\cos\theta = 0$, y se tiene que:
        \begin{equation*}
            |\varphi(r)| = |e^{re^{i\theta}}| = e^{r\cos\theta} = e^0 = 1\qquad \forall r\in \bb{R}^+
        \end{equation*}

        Por tanto, hemos de estudiar la función completa para ver si tiene límite. En este caso, se tiene que:
        \begin{align*}
            \varphi(r) &= e^{re^{i\theta}} = e^{ir\sen\theta}\qquad \forall r\in \bb{R}^+
        \end{align*}

        Vemos fácilmente que no tendrá límite pues recorre la circunferencia de radio $1$ centrada en el origen en sentido antihorario, pero demostrémoslo. Consideramos las dos siguientes sucesiónes:
        \begin{equation*}
            \{r_n\} = \{2\pi n\} \quad \text{y} \quad \{s_n\} = \{(2n+1)\pi\}
        \end{equation*}

        Se tiene que $\{r_n\} \to +\infty$ y $\{s_n\} \to +\infty$. Además, para cada $n\in \bb{N}$, se tiene que:
        \begin{align*}
            \{\varphi(r_n)\} &= \{e^{ir_n\sen\theta}\} = 1\\
            \{\varphi(s_n)\} &= \{e^{is_n\sen\theta}\} = -1
        \end{align*}

        Por tanto, por la unicidad del límite, se tiene que $\varphi(r)$ no tiene límite cuando $r\to +\infty$.
    \end{itemize}
\end{ejercicio}

\begin{ejercicio}
    Probar que si $\{z_n\}$ y $\{w_n\}$ son sucesiones de números complejos, con $z_n \neq 0$ para todo $n \in \mathbb{N}$ y $\{z_n\} \to 1$, entonces
    \[
        \left\{w_n(z_n - 1)\right\} \to \lm \in \mathbb{C} \implies \left\{{z_n}^{w_n}\right\} \to e^{\lm}
    \]

    Para cada $n\in \bb{N}$, sabemos que:
    \begin{align*}
        z_n^{w_n} &= e^{w_n\log z_n}
    \end{align*}

    Calculamos por tanto el límite de la sucesión $\{w_n\log z_n\}$. Como $\{z_n\} \to 1$, $\exists n_0\in \bb{N}$ tal que, para $n\geq n_0$, se tiene que $z_n\in D(1,1)$. Además, se tiene que:
    \begin{align*}
        \log z_n &= \log 1 + \sum_{m=1}^\infty \frac{(-1)^{m+1}}{m}\left(z_n - 1\right)^m = \sum_{m=1}^\infty \frac{(-1)^{m+1}}{m}\left(z_n - 1\right)^m \qquad \forall n\geq n_0
    \end{align*}

    De esta forma, para cada $n\geq n_0$, se tiene que:
    \begin{align*}
        w_n\log z_n &= w_n\sum_{m=1}^\infty \frac{(-1)^{m+1}}{m}\left(z_n - 1\right)^m
        = w_n(z_n - 1)\sum_{m=1}^\infty \frac{(-1)^{m+1}}{m}\left(z_n - 1\right)^{m-1} \\ &= w_n(z_n - 1)\left(1+\sum_{m=2}^\infty \frac{(-1)^{m+1}}{m}\left(z_n - 1\right)^{m-1}\right)
        =\\&= w_n(z_n - 1)\left(1+\sum_{m=1}^\infty \frac{(-1)^{m+2}}{m+1}\left(z_n - 1\right)^{m}\right)     
    \end{align*}

    Calculamos ahora el radio de convergencia de dicha serie de potencias.
    \begin{equation*}
        \left\{\dfrac{(1)^{m+3}}{m+2}\cdot \dfrac{m+1}{(1)^{m+2}}\right\} = \left\{\dfrac{m+2}{m+1}\right\}\to 1
    \end{equation*}

    Por tanto, sabemos que dicha suma es continua en cada compacto $K\subset D(1,1)$, y por el carácter local de la continuidad se tiene que dicha suma es continua en $D(1,1)$. Por tanto, se tiene que:
    \begin{equation*}
        \lim_{n\to +\infty} \sum_{m=1}^\infty \frac{(-1)^{m+2}}{m+1}\left(z_n - 1\right)^{m} = \sum_{m=1}^\infty \frac{(-1)^{m+2}}{m+1}\left(\lim_{n\to \infty}(z_n) - 1\right)^{m}
        = \sum_{m=1}^\infty \frac{(-1)^{m+2}}{m+1}\cdot 0^{m} = 0
    \end{equation*}

    Por tanto, como el límite de dos sucesiones convegentes es el producto de sus límites, se tiene que:
    \begin{align*}
        \lim_{n\to +\infty} w_n\log z_n &= \lim_{n\to +\infty} w_n(z_n - 1)\cdot \lim_{n\to +\infty} \left(1+\sum_{m=1}^\infty \frac{(-1)^{m+2}}{m+1}\left(z_n - 1\right)^{m}\right) = \lm \cdot (1+0) = \lm
    \end{align*}

    Por tanto, como la exponencial es una función continua, se tiene que:
    \begin{align*}
        \lim_{n\to +\infty} z_n^{w_n} &= \lim_{n\to +\infty} e^{w_n\log z_n} = e\exp\left(\lim_{n\to +\infty} w_n\log z_n\right) = e^\lm
    \end{align*}
\end{ejercicio}

\begin{ejercicio}
    Estudiar la convergencia puntual, absoluta y uniforme de la serie de funciones
    \[
        \sum_{n\geq 0} e^{-nz^2}
    \]

    Definimos las funciones siguientes:
    \Func{f}{\mathbb{C}}{\mathbb{C}}{z}{e^{-nz^2}}
    \Func{\varphi}{\mathbb{C}}{\mathbb{C}}{z}{e^{-z^2}}

    De esta forma, tenemos que la serie pedida es:
    \[
        \sum_{n\geq 0} f_n(z) = \sum_{n\geq 0} e^{-nz^2} = \sum_{n\geq 0} \varphi(z)^n
    \]

    Por tanto, estamos considerando una serie geométrica de razón $\varphi(z)$. En primer lugar, sabemos que esta converge puntualmente en $D(0,1)$.
    \begin{align*}
        |\varphi(z)| &= |e^{-z^2}| = e^{\Re(-z^2)} = e^{-\Re(z^2)} = \dfrac{1}{\exp(\Im(z)^2 - \Re(z)^2)} <1\iff \Im(z)^2 - \Re(z)^2 > 0
    \end{align*}

    Definimos por tanto el siguiente conjunto:
    \begin{equation*}
        H=\{z\in \bb{C} : \Im(z)^2 - \Re(z)^2 > 0\}
    \end{equation*}

    Por lo conocido sobre la serie geométrica, sabemos que la serie converge absolutamente (y por tanto puntualmente) en $H$, y no converge (ni siquiera puntualmente) en $\bb{C}\setminus H$.\\

    Estudiamos ahora la convergencia uniforme de la serie. Razonemos en primer lugar sobre compactos. Sea $K\subset H$ compacto. Entonces, por ser $\varphi$ continua, se tiene que $\varphi(K)\subset D(0,1)$ es compacto. Por tanto, la serie converge uniformemente en $K$.\\

    Supongamos ahora $\emptyset\neq A\subset H$ no necesariamente compacto, y supongamos que la serie converge uniformemente en $A$.

    % // TODO: Qué hacer sobre no compactos?
\end{ejercicio}

\begin{ejercicio}
    Probar que si $a,b,c \in \mathbb{T}$ son vértices de un triángulo equilátero si, y sólo si, $a+b+c = 0$.
\end{ejercicio}

\begin{ejercicio}
    Sea $\Omega$ un subconjunto abierto no vacío de $\mathbb{C}^*$ y $\varphi \in \mathcal{C}(\Omega)$ tal que $\varphi(z)^2 = z$ para todo $z \in \Omega$. Probar que $\varphi \in \mathcal{H}(\Omega)$ y calcular su derivada.
\end{ejercicio}

\begin{ejercicio}
    Probar que, para todo $z \in D(0,1)$ se tiene:
    \begin{enumerate}
        \item $\sum\limits_{n= 1}^\infty \dfrac{(-1)^{n+1}}{n}z^n = \log(1+z)$
        \item $\sum\limits_{n= 1}^\infty \dfrac{z^{2n+1}}{n(2n+1)} = 2z - (1+z)\log(1+z) + (1-z)\log(1-z)$
    \end{enumerate}
\end{ejercicio}

\begin{ejercicio}
    Sea la siguiente función:
    \Func{f}{\mathbb{C}\setminus\{1,-1\}}{\mathbb{C}}{z}{\log\left(\frac{1+z}{1-z}\right)}
    Probar que $f$ es holomorfa en el dominio $W = \mathbb{C} \setminus \{x \in \mathbb{R} : |x| \geq 1\}$ y calcular su derivada. Probar también que
    \[
        f(z) = 2\sum_{n=0}^\infty \frac{z^{2n+1}}{2n+1} \quad \forall z \in D(0,1)
    \]
\end{ejercicio}

\begin{ejercicio}
    Sean $\alpha,\beta \in \left[ -\pi, \pi \right]$ con $\alpha < \beta$, y $\rho \in \mathbb{R}^+$ tal que $\rho\alpha,\rho\beta \in \left[ -\pi, \pi \right]$. Consideramos los siguientes dominios:
    \begin{align*}
        \Omega &= \{z \in \mathbb{C}^* : \alpha < \arg z < \beta\} \\
        \Omega_\rho &= \{z \in \mathbb{C}^* : \rho\alpha < \arg z < \rho\beta\}
    \end{align*}
    Probar que la siguiente función define una biyección de $\Omega$ sobre el dominio $\Omega_\rho$:
    \Func{f}{\Omega}{\Omega_\rho}{z}{z^\rho}
\end{ejercicio}

\begin{ejercicio}
    Probar que el seno, el coseno y la tangente son funciones simplemente periódicas.
\end{ejercicio}

\begin{ejercicio}
    Estudiar la convergencia de la serie
    \[
        \sum_{n\geq 0} \frac{\sen(nz)}{2^n}
    \]
\end{ejercicio}

\begin{ejercicio}
    Sea $\Omega = \mathbb{C} \setminus \{x \in \mathbb{R} : |x| \geq 1\}$. Probar que existe $f \in \mathcal{H}(\Omega)$ tal que $\cos f(z) = z$ para todo $z \in \Omega$ y $f(x) = \arccos x$ para todo $x \in \left] -1, 1 \right[$. Calcular la derivada de $f$.
\end{ejercicio}

\begin{ejercicio}
    Para $z \in D(0,1)$ con $\Re z \neq 0$, probar que
    \[
        \arctan\left(\frac{1}{z}\right) + \sum_{n=0}^\infty \frac{(-1)^n}{2n+1}z^{2n+1} = \begin{cases}
            \nicefrac{\pi}{2} & \text{si } \Re z > 0 \\
            \nicefrac{-\pi}{2} & \text{si } \Re z < 0
        \end{cases}
    \]
\end{ejercicio}
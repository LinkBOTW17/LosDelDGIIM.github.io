\section{Funciones Elementales}

\begin{ejercicio}
    Sea $f : \mathbb{C} \to \mathbb{C}$ una función verificando que
    \[
        f(z + w) = f(z)f(w) \quad \forall z,w \in \mathbb{C}
    \]
    Probar que, si $f$ es derivable en algún punto del plano, entonces $f$ es entera. Encontrar todas las funciones enteras que verifiquen la condición anterior. Dar un ejemplo de una función que verifique dicha condición y no sea entera.\\

    Sea $z_0\in \bb{C}$, y supongamos que $f$ es derivable en $z_0$. Por ser derivable en $z_0$, tenemos que:
    \begin{equation*}
        f'(z_0) = \lim_{h \to 0} \frac{f(z_0 + h) - f(z_0)}{h} \AstIg \lim_{h \to 0} \frac{f(z_0)f(h) - f(z_0)}{h} = f(z_0)\lim_{h \to 0} \frac{f(h) - 1}{h}
    \end{equation*}
    donde en $(\ast)$ hemos usado la propiedad de $f$. Caben dos casos:
    \begin{itemize}
        \item \ul{Si $f(z_0) = 0$}, entonces:
        \begin{align*}
            f(z)=f(z_0 + (z-z_0)) &= f(z_0)f(z-z_0) = 0 \quad \forall z \in \mathbb{C}
        \end{align*}
        Por tanto, $f$ es la función nula, por lo que $f\in \mathcal{H}(\mathbb{C})$.

        \item \ul{Si $f(z_0) \neq 0$}, entonces:
        \begin{equation*}
            \lim_{h \to 0} \frac{f(h) - 1}{h} = \frac{f'(z_0)}{f(z_0)} \in \mathbb{C}
        \end{equation*}

        Veamos ahora que $f$ es entera. Fijado $z\in \mathbb{C}$, se tiene:
        \begin{align*}
            f'(z) &= \lim_{h \to 0} \frac{f(z + h) - f(z)}{h} = \lim_{h \to 0} \frac{f(z)f(h) - f(z)}{h}
            = f(z)\lim_{h \to 0} \frac{f(h) - 1}{h} = f(z)\cdot \frac{f'(z_0)}{f(z_0)}
        \end{align*}

        Por lo tanto, $f\in \mathcal{H}(\mathbb{C})$.
    \end{itemize}
    En cualquier caso, $f$ es entera. Veamos ahora $f\in \cc{H}(\bb{C})$ cumple la condición del enunciado si y sólo si, $f=0$ o $f(z) = e^{\lm z}$ para algún $\lm \in \bb{C}$.
    \begin{description}
        \item[$\Longrightarrow$)] Sea $f\in \cc{H}(\bb{C})$, y sea $z_0\in \bb{C}$, por lo que $f$ es derivable en $z_0$. Por lo visto anteriormente, o bien $f$ es la función nula, o bien $f(z_0) \neq 0$ y se tiene:
        \begin{equation*}
            f'(z)=f(z)\cdot \frac{f'(z_0)}{f(z_0)} \quad \forall z\in \bb{C}
        \end{equation*}

        Definimos $\lm = \frac{f'(z_0)}{f(z_0)}\in \bb{C}$, y sea la siguiente función:
        \Func{g}{\bb{C}}{\bb{C}}{z}{f(z)e^{-\lm z}}

        Sabemos que $f\in \cc{H}(\bb{C})$, con:
        \begin{equation*}
            g'(z) = f'(z)e^{-\lm z} - \lm f(z)e^{-\lm z} = e^{-\lm z}\left(f'(z) - \lm f(z)\right) = 0\qquad \forall z\in \bb{C}
        \end{equation*}

        Por tanto, $g$ es constante. Tenemos que:
        \begin{equation*}
            g(0)=f(0)e^{-\lm\cdot 0} = f(0)
        \end{equation*}
        
        Por lo que $g(z) = f(0)$ para todo $z\in \bb{C}$. Por tanto lado, de la ecuación del enunciado, tenemos que:
        \begin{equation*}
            f(z_0)=f(z_0+0)=f(z_0)f(0) \Longrightarrow f(0)=1
        \end{equation*}

        Por tanto, $g$ es la función constante $1$, y se tiene que:
        \begin{equation*}
            1=g(z) = f(z)e^{-\lm z} \Longrightarrow f(z) = e^{\lm z} \quad \forall z\in \bb{C}
        \end{equation*}

        \item[$\Longleftarrow$)] Sea $f(z) = e^{\lm z}$, con $\lm \in \bb{C}$. Entonces, se tiene que:
        \begin{align*}
            f(z+w) &= e^{\lm(z+w)} = e^{\lm z}e^{\lm w} = f(z)f(w) \quad \forall z,w\in \bb{C}
        \end{align*}
    \end{description}

    Buscamos ahora un ejemplo de función que verifique la condición del enunciado y no sea entera.
    Definimos la función siguiente:
    \Func{f}{\bb{C}}{\bb{C}}{z}{e^{\ol{z}}}

    Veamos en primer lugar que $f$ verifica la condición del enunciado. Para ello, sea $z,w\in \bb{C}$, y tenemos que:
    \begin{align*}
        f(z+w) &= e^{\ol{z+w}} = e^{\ol{z}+\ol{w}} = e^{\ol{z}}e^{\ol{w}} = f(z)f(w)
    \end{align*}

    Supongamos ahora que $f\in \cc{H}(\bb{C})$, y definimos la exponencial:
    \Func{g}{\bb{C}}{\bb{C}}{z}{e^z}

    Como $f,g\in \cc{H}(\bb{C})$, se tiene que $fg\in \cc{H}(\bb{C})$. Veamos cuál es la función $fg$:
    \begin{align*}
        (fg)(z)=f(z)g(z) &= e^{\ol{z}}e^z = e^{\ol{z}+z} = e^{2\Re(z)}\qquad \forall z\in \bb{C}
    \end{align*}

    Como $\bb{C}$ es un dominio y $\Im(fg)=0$ constante, entonces $fg$ es constante. No obstante, vemos que:
    \begin{equation*}
        e^{2\cdot 2}=(fg)(2)\neq (fg)(0) = e^{2\cdot 0} = 1
    \end{equation*}

    Por tanto, $fg$ no es constante, llegando a una contradicción. Por tanto, $f$ no es entera, y se tiene lo buscado.
\end{ejercicio}

\begin{ejercicio}
    Calcular la imagen por la función exponencial de una banda horizontal o vertical y del dominio cuya frontera es un rectángulo de lados paralelos a los ejes.\\

    Sean $a,b,c,d\in \bb{R}$ tal que $a<b$ y $c<d$. Consideramos:
    \begin{itemize}
        \item La banda vertical siguiente:
        \begin{equation*}
            \Omega_1 = \{z\in \bb{C} : a \leq \Re z \leq b\} = [a,b]\times \bb{R}
        \end{equation*}

        \item La banda horizontal siguiente:
        \begin{equation*}
            \Omega_2 = \{z\in \bb{C} : c \leq \Im z \leq d\} = \bb{R}\times [c,d]
        \end{equation*}

        \item El rectángulo siguiente:
        \begin{equation*}
            \Omega_3 = \{z\in \bb{C} : a \leq \Re z \leq b, c \leq \Im z \leq d\} = [a,b]\times [c,d] = \Omega_1\cap \Omega_2
        \end{equation*}
    \end{itemize}

    Definimos ahora la función siguiente:
    \Func{f}{\bb{C}}{\bb{C}^*}{z}{e^z}

    Calculamos la imagen de cada uno de los dominios anteriores:
    \begin{itemize}
        \item La imagen de la banda vertical $\Omega_1$ es:
        \begin{align*}
            f(\Omega_1) &= \{e^z : z\in \Omega_1\} = \{e^{x+iy} : a\leq x\leq b, y\in \bb{R}\} \\
            &= \{e^x e^{iy} : a\leq x\leq b, y\in \bb{R}\}
        \end{align*}

        Veamos por doble inclusión que:
        \begin{equation*}
            f(\Omega_1) = \{w\in \bb{C} : |w|\in [e^a,e^b]\}
        \end{equation*}
        \begin{description}
            \item[$\subseteq$)] Sea $w\in f(\Omega_1)$, entonces existe $x\in [a,b]$ y $y\in \bb{R}$ tal que $w = e^{x+iy}$. Por tanto, se tiene que:
            \begin{equation*}
                |w| = |e^{x+iy}| = e^x \in [e^a,e^b]
            \end{equation*}
            
            \item[$\supseteq$)] Sea $w\in \bb{C}$ tal que $|w|\in [e^a,e^b]$. Entonces, y definimos $x=\ln |w|$, y $y = \arg w$. Por tanto, se tiene que:
            \begin{equation*}
                f(x+iy) = e^{x+iy} = e^x e^{iy} = |w|e^{i\arg w} = w
            \end{equation*}
            Por tanto, $w\in f(\Omega_1)$.
        \end{description}

        Por tanto, se tiene que $f(\Omega_1)$ es el anillo del plano complejo delimitado por las circunferencias de radio $e^a$ y $e^b$ centradas en el origen.

        \item La imagen de la banda horizontal $\Omega_2$ es:
        \begin{align*}
            f(\Omega_2) &= \{e^z : z\in \Omega_2\} = \{e^{x+iy} : x\in \bb{R}, c\leq y\leq d\} \\
            &= \{e^x e^{iy} : x\in \bb{R}, c\leq y\leq d\}
            =\\&= \{e^x\left(\cos y + i\sen y\right) : x\in \bb{R}, c\leq y\leq d\}
        \end{align*}

        Vemos por tanto que $f(\Omega_2)$ son los puntos del plano complejo que forman un sector angular del plano complejo delimitado por el origen y los ángulos $c$ y $d$. Si $[c,d]$ parametriza toda la circunferencia (es decir, $l(c,d)=d-c\geq 2\pi$), entonces $f(\Omega_2)=\bb{C}^*$. En caso contrario, será el sector angular correspondiente a la parametrización realidada por $[c,d]$.

        \item La imagen del rectángulo $\Omega_3$ es:
        \begin{align*}
            f(\Omega_3) &= f(\Omega_1\cap \Omega_2) = f(\Omega_1)\cap f(\Omega_2)
        \end{align*}

        Por tanto, se trata de la región del plano compleja delimitada por las circunferencias de radio $e^a$ y $e^b$, y los ángulos $c$ y $d$. Si $l(c,d)\geq 2\pi$, entonces $f(\Omega_3) = f(\Omega_1)$. En caso contrario, se trata del sector angular delimitado por los ángulos $c$ y $d$ y las circunferencias de radio $e^a$ y $e^b$.
    \end{itemize}
\end{ejercicio}

\begin{ejercicio}
    Dado $\theta\in \left] -\pi, \pi \right]$, estudiar la existencia del límite en $+\infty$ de la función siguiente:
    \Func{\varphi}{\mathbb{R}^+}{\mathbb{C}}{r}{e^{re^{i\theta}}}

    Tenemos que:
    \begin{align*}
        |\varphi(r)| &= |e^{re^{i\theta}}| = e^{r\cos\theta}
    \end{align*}

    Distinguimos casos:
    \begin{itemize}
        \item Si $\theta\in \left] -\nicefrac{\pi}{2}, \nicefrac{\pi}{2} \right[$, entonces $\cos\theta > 0$, y se tiene que:
        \begin{equation*}
            \lim_{r\to +\infty} |\varphi(r)| = \lim_{r\to +\infty} e^{r\cos\theta} = +\infty
        \end{equation*}

        Por tanto, $\varphi(r) \to \infty$ cuando $r\to +\infty$.

        \item Si $\theta \in \left]-\pi, -\nicefrac{\pi}{2} \right[ \cup \left]\nicefrac{\pi}{2}, \pi \right]$, entonces $\cos\theta < 0$, y se tiene que:
        \begin{equation*}
            \lim_{r\to +\infty} |\varphi(r)| = \lim_{r\to +\infty} e^{r\cos\theta} = 0
        \end{equation*}

        Por tanto, $\lim\limits_{r\to +\infty} \varphi(r) = 0$.

        \item Si $\theta = \pm \nicefrac{\pi}{2}$, entonces $\cos\theta = 0$, y se tiene que:
        \begin{equation*}
            |\varphi(r)| = |e^{re^{i\theta}}| = e^{r\cos\theta} = e^0 = 1\qquad \forall r\in \bb{R}^+
        \end{equation*}

        Por tanto, hemos de estudiar la función completa para ver si tiene límite. En este caso, se tiene que:
        \begin{align*}
            \varphi(r) &= e^{re^{i\theta}} = e^{ir\sen\theta}\qquad \forall r\in \bb{R}^+
        \end{align*}

        Vemos fácilmente que no tendrá límite pues recorre la circunferencia de radio $1$ centrada en el origen en sentido antihorario, pero demostrémoslo. Consideramos las dos siguientes sucesiónes:
        \begin{equation*}
            \{r_n\} = \{2\pi n\} \quad \text{y} \quad \{s_n\} = \{(2n+1)\pi\}
        \end{equation*}

        Se tiene que $\{r_n\} \to +\infty$ y $\{s_n\} \to +\infty$. Además, para cada $n\in \bb{N}$, se tiene que:
        \begin{align*}
            \{\varphi(r_n)\} &= \{e^{ir_n\sen\theta}\} = 1\\
            \{\varphi(s_n)\} &= \{e^{is_n\sen\theta}\} = -1
        \end{align*}

        Por tanto, por la unicidad del límite, se tiene que $\varphi(r)$ no tiene límite cuando $r\to +\infty$.
    \end{itemize}
\end{ejercicio}

\begin{ejercicio}
    Probar que si $\{z_n\}$ y $\{w_n\}$ son sucesiones de números complejos, con $z_n \neq 0$ para todo $n \in \mathbb{N}$ y $\{z_n\} \to 1$, entonces
    \[
        \left\{w_n(z_n - 1)\right\} \to \lm \in \mathbb{C} \implies \left\{{z_n}^{w_n}\right\} \to e^{\lm}
    \]

    Para cada $n\in \bb{N}$, sabemos que:
    \begin{align*}
        z_n^{w_n} &= e^{w_n\log z_n}
    \end{align*}

    Calculamos por tanto el límite de la sucesión $\{w_n\log z_n\}$. Como $\{z_n\} \to 1$, $\exists n_0\in \bb{N}$ tal que, para $n\geq n_0$, se tiene que $z_n\in D(1,1)$. Además, se tiene que:
    \begin{align*}
        \log z_n &= \log 1 + \sum_{m=1}^\infty \frac{(-1)^{m+1}}{m}\left(z_n - 1\right)^m = \sum_{m=1}^\infty \frac{(-1)^{m+1}}{m}\left(z_n - 1\right)^m \qquad \forall n\geq n_0
    \end{align*}

    De esta forma, para cada $n\geq n_0$, se tiene que:
    \begin{align*}
        w_n\log z_n &= w_n\sum_{m=1}^\infty \frac{(-1)^{m+1}}{m}\left(z_n - 1\right)^m
        = w_n(z_n - 1)\sum_{m=1}^\infty \frac{(-1)^{m+1}}{m}\left(z_n - 1\right)^{m-1} \\ &= w_n(z_n - 1)\left(1+\sum_{m=2}^\infty \frac{(-1)^{m+1}}{m}\left(z_n - 1\right)^{m-1}\right)
        =\\&= w_n(z_n - 1)\left(1+\sum_{m=1}^\infty \frac{(-1)^{m+2}}{m+1}\left(z_n - 1\right)^{m}\right)     
    \end{align*}

    Calculamos ahora el radio de convergencia de dicha serie de potencias.
    \begin{equation*}
        \left\{\dfrac{(1)^{m+3}}{m+2}\cdot \dfrac{m+1}{(1)^{m+2}}\right\} = \left\{\dfrac{m+2}{m+1}\right\}\to 1
    \end{equation*}

    Por tanto, sabemos que dicha suma es continua en cada compacto $K\subset D(1,1)$, y por el carácter local de la continuidad se tiene que dicha suma es continua en $D(1,1)$. Por tanto, se tiene que:
    \begin{equation*}
        \lim_{n\to +\infty} \sum_{m=1}^\infty \frac{(-1)^{m+2}}{m+1}\left(z_n - 1\right)^{m} = \sum_{m=1}^\infty \frac{(-1)^{m+2}}{m+1}\left(\lim_{n\to \infty}(z_n) - 1\right)^{m}
        = \sum_{m=1}^\infty \frac{(-1)^{m+2}}{m+1}\cdot 0^{m} = 0
    \end{equation*}

    Por tanto, como el límite de dos sucesiones convegentes es el producto de sus límites, se tiene que:
    \begin{align*}
        \lim_{n\to +\infty} w_n\log z_n &= \lim_{n\to +\infty} w_n(z_n - 1)\cdot \lim_{n\to +\infty} \left(1+\sum_{m=1}^\infty \frac{(-1)^{m+2}}{m+1}\left(z_n - 1\right)^{m}\right) = \lm \cdot (1+0) = \lm
    \end{align*}

    Por tanto, como la exponencial es una función continua, se tiene que:
    \begin{align*}
        \lim_{n\to +\infty} z_n^{w_n} &= \lim_{n\to +\infty} e^{w_n\log z_n} = \exp\left(\lim_{n\to +\infty} w_n\log z_n\right) = e^\lm
    \end{align*}
\end{ejercicio}

\begin{ejercicio}
    Estudiar la convergencia puntual, absoluta y uniforme de la serie de funciones
    \[
        \sum_{n\geq 0} e^{-nz^2}
    \]

    Definimos las funciones siguientes:
    \Func{f}{\mathbb{C}}{\mathbb{C}}{z}{e^{-nz^2}}
    \Func{\varphi}{\mathbb{C}}{\mathbb{C}}{z}{e^{-z^2}}

    De esta forma, tenemos que la serie pedida es:
    \[
        \sum_{n\geq 0} f_n(z) = \sum_{n\geq 0} e^{-nz^2} = \sum_{n\geq 0} \varphi(z)^n
    \]

    Por tanto, estamos considerando una serie geométrica de razón $\varphi(z)$. En primer lugar, sabemos que esta converge puntualmente en $D(0,1)$.
    \begin{align*}
        |\varphi(z)| &= \left|e^{-z^2}\right| = e^{\Re(-z^2)} = e^{-\Re(z^2)} = 
        \exp(\Im(z)^2 - \Re(z)^2) < 1
        \iff \Im(z)^2 - \Re(z)^2 < 0
    \end{align*}

    Definimos por tanto el siguiente conjunto:
    \begin{equation*}
        H=\{z\in \bb{C} : \Im(z)^2 - \Re(z)^2 < 0\}
    \end{equation*}

    Por lo conocido sobre la serie geométrica, sabemos que la serie converge absolutamente (y por tanto puntualmente) en $H$, y no converge (ni siquiera puntualmente) en $\bb{C}\setminus H$.\\

    Estudiamos ahora la convergencia uniforme de la serie. Razonemos en primer lugar sobre compactos. Sea $K\subset H$ compacto. Entonces, por ser $\varphi$ continua, se tiene que $\varphi(K)\subset D(0,1)$ es compacto. Por tanto, la serie converge uniformemente en $K$.\\

    Supongamos ahora $\emptyset\neq A\subset H$ no necesariamente compacto. Veamos que la serie converge uniformemente en $A$ si y solo si $r=\sup\{\Im(z)^2 - \Re(z)^2 : z\in A\} < 0$.
    \begin{description}
        \item[$\Longleftarrow$)] Supongamos que $r=\sup\{\Im(z)^2 - \Re(z)^2 : z\in A\} < 0$. 
        \begin{align*}
            |\varphi(z)|^n &= \left(\exp(\Im(z)^2 - \Re(z)^2)\right)^n\leq \left(e^r\right)^n \quad \forall z\in A
        \end{align*}

        Como $r<0$, se tiene que $e^r<1$, y por tanto la serie geométrica $\sum\limits_{n\geq 0} \left(e^r\right)^n$ converge. Por tanto, por el Test de Weierstrass, se tiene que la serie converge uniformemente en $A$.

        \item[$\Longrightarrow$)] Como $A\subset H$, no es posible que $r>0$. Por tanto, supongamos $r=0$. Por tanto, para cada $n\in \bb{N}$, existe $z_n\in A$ tal que:
        \begin{align*}
            \Im(z_n)^2 - \Re(z_n)^2 > -\frac{1}{n}
        \end{align*}

        Por tanto, se tiene que:
        \begin{align*}
            |\varphi(z_n)| &= \exp(\Im(z_n)^2 - \Re(z_n)^2) > \exp\left(-\frac{1}{n}\right)\qquad \forall n\in \bb{N}
        \end{align*}

        Como $\left\{\exp\left(\nicefrac{-1}{n}\right)\right\}\to 1$, se tiene que $\{|\varphi(z_n)|\}$ no converge a $0$. Por tanto, la serie $\sum\limits_{n\geq 0} \varphi(z_n)^n$ no converge uniformemente en $A$.
    \end{description}

    A partir de lo anterior, vemos que la serie no converge uniformemente en $H$. En efecto, para cada $n\in \bb{N}$, consideramos la sudeción siguiente:
    \begin{equation*}
        \{z_n\} = \left\{\dfrac{1}{n}\right\} \subset H
    \end{equation*}
    Por un lado, tenemos que $0$ es un mayorante de $\{\Im(z)^2 - \Re(z)^2 : z\in H\}$ y, además, se tiene que  $\left\{\Im(z_n)^2 - \Re(z_n)^2\right\} = \left\{\nicefrac{-1}{n^2}\right\} \to 0$, entonces:
    \begin{equation*}
        \sup\{\Im(z)^2 - \Re(z)^2 : z\in H\} = 0
    \end{equation*}
    Por tanto, la serie no converge uniformemente en $H$.
\end{ejercicio}

\begin{ejercicio}
    Dados $a,b,c \in \mathbb{T}$, probar que son vértices de un triángulo equilátero si, y sólo si, $a+b+c = 0$.
    \begin{description}
        \item[$\Longrightarrow$)] Supongamos que $a,b,c\in \mathbb{T}$ son vértices de un triángulo equilátero.
        El origen del plano complejo es el centro $\bb{T}$, y por tanto también el centro del triángulo equilátero.
        Como el ángulo interior del triángulo equilátero es de $\nicefrac{2\pi}{3}$, sabemos que:
        \begin{align*}
            \arg a + \frac{2\pi}{3} &\in \Arg b\\
            \arg a + \frac{4\pi}{3} &\in \Arg c
        \end{align*}

        Por tanto:
        \begin{align*}
            a+b+c &= e^{i\arg a} + e^{i\left(\arg a + \frac{2\pi}{3}\right)} + e^{i\left(\arg a + \frac{4\pi}{3}\right)}\\
            &= e^{i\arg a}\left(1 + e^{i\frac{2\pi}{3}} + e^{i\frac{4\pi}{3}}\right)\\
            &= e^{i\arg a}\left(1 + \left(-\frac{1}{2} + i\frac{\sqrt{3}}{2}\right) + \left(-\frac{1}{2} - i\frac{\sqrt{3}}{2}\right)\right)\\
            &= e^{i\arg a}\cdot 0 = 0
        \end{align*}

        \item[$\Longleftarrow$)] Supongamos que $a+b+c = 0$, y buscamos $\alpha\in \bb{C}$ tal que $a,b,c$ sean raíces cúbicas. Es decir, buscamos $\alpha\in \bb{C}$ tal que:
        \begin{equation*}
            a^3=\alpha=b^3=c^3
        \end{equation*}

        Por tanto, para cada $z\in \{a,b,c\}$, buscamos $\alpha\in \bb{C}$ tal que:
        \begin{equation*}
            (z-a)(z-b)(z-c) = z^3 - \alpha
        \end{equation*}

        Calculamos el polinomio de la izquierda. Para cada $z\in \bb{C}$, se tiene que:
        \begin{align*}
            (z-a)(z-b)(z-c) &= z^3 - (a+b+c)z^2 + (ab+bc+ca)z - abc\\
            &= z^3 + (ab+bc+ca)z - abc
        \end{align*}

        Como $a,b,c\in \mathbb{T}$, se tiene que $|a|=|b|=|c|=1$. Elevando al cuadrado, vemos que $\ol{a}a=\ol{b}b=\ol{c}c=1$, y por tanto:
        \begin{equation*}
            ab+bc+ca = \ol{c}cab + \ol{a}abc + \ol{b}abc = \left(\ol{a} + \ol{b} + \ol{c}\right)abc = \ol{a+b+c}\cdot abc = 0\cdot abc = 0
        \end{equation*}

        Por tanto, se tiene que:
        \begin{align*}
            (z-a)(z-b)(z-c) &= z^3 - abc\qquad \forall z\in \bb{C}
        \end{align*}

        Evaluando en $z\in \{a,b,c\}$, se tiene que:
        \begin{equation*}
            0=a^3 - abc = b^3 - abc = c^3 - abc \Longrightarrow a^3 = b^3 = c^3 = abc
        \end{equation*}

        De esta forma, hemos visto que $a,b,c$ son raíces cúbicas de $\alpha = abc$. Por tanto, como $[(abc)^{\nicefrac{1}{3}}]$ es finito con tres elementos, se tiene que:
        \begin{equation*}
            [(abc)^{\nicefrac{1}{3}}] = \{a,b,c\}
        \end{equation*}

        Además, sabemos que $[(abc)^{\nicefrac{1}{3}}]$ son los vértices de un triángulo equilátero inscrito en la circunferencia de radio $|abc|=1$. Por tanto, forman un triángulo equilátero.
    \end{description}
\end{ejercicio}

\begin{ejercicio}
    Sea $\Omega$ un subconjunto abierto no vacío de $\mathbb{C}^*$ y $\varphi \in \mathcal{C}(\Omega)$ tal que $\varphi(z)^2 = z$ para todo $z \in \Omega$. Probar que $\varphi \in \mathcal{H}(\Omega)$ y calcular su derivada.\\

    Sea $a\in \Omega$. Buscamos calcular la derivada de $\varphi$ en $a$, pero veamos antes que $\varphi(a)\neq 0$. En efecto, si $\varphi(a) = 0$, entonces se tiene que:
    \begin{align*}
        a = \varphi(a)^2 &= \varphi(a)\cdot \varphi(a) = 0\cdot 0 = 0\Longrightarrow a = 0\notin \Omega\subset \bb{C}^*
    \end{align*}

    Por tanto, $\varphi(a)\neq 0$. Tenemos por tanto que:
    \begin{align*}
        \varphi'(a) &= \lim_{z\to a}\dfrac{\varphi(z) - \varphi(a)}{z-a} \AstIg \lim_{z\to a}\dfrac{\varphi(z)-\varphi(a)}{\varphi(z)^2 - \varphi(a)^2} =\\&= \lim_{z\to a}\dfrac{\varphi(z)-\varphi(a)}{(\varphi(z) - \varphi(a))(\varphi(z) + \varphi(a))} = \lim_{z\to a}\dfrac{1}{\varphi(z) + \varphi(a)} = \dfrac{1}{2\varphi(a)}
    \end{align*}
    donde en $(\ast)$ hemos usado que $z\in \Omega\setminus \{a\}$ y $a\in \Omega$. Por tanto, $\varphi\in \mathcal{H}(\Omega)$ y:
    \begin{equation*}
        \varphi'(z) = \dfrac{1}{2\varphi(z)}\qquad \forall z\in \Omega
    \end{equation*}
\end{ejercicio}

\begin{ejercicio}\label{ej:serie_log_1masZ}
    Probar que, para todo $z \in D(0,1)$ se tiene:
    \begin{enumerate}
        \item $\sum\limits_{n= 1}^\infty \dfrac{(-1)^{n+1}}{n}z^n = \log(1+z)$
        
        \begin{description}
            \item[Haciendo uso del Desarrollo en Serie ya conocido]~
            
            En primer lugar, por el desarrollo en serie del logaritmo, sabemos que:
            \begin{align*}
                \log w = \log 1 + \sum_{n=1}^\infty \frac{(-1)^{n+1}}{n}\left(w - 1\right)^n = \sum_{n=1}^\infty \frac{(-1)^{n+1}}{n}\left(w - 1\right)^n\qquad \forall w\in D(1,1)
            \end{align*}

            Sea ahora $z\in D(0,1)$, y definimos $w = 1+z\in D(1,1)$. Entonces, se tiene que:
            \begin{align*}
                \log(1+z) &= \log w = \sum_{n=1}^\infty \frac{(-1)^{n+1}}{n}\left(w - 1\right)^n = \sum_{n=1}^\infty \frac{(-1)^{n+1}}{n}z^n\qquad \forall z\in D(0,1)
            \end{align*}

            \item[Otra opción]~
            
            Definimos las siguientes funciones:
            \Func{f}{D(0,1)}{\mathbb{C}}{z}{\log(1+z)}
            \Func{g}{D(0,1)}{\mathbb{C}}{z}{\sum\limits_{n= 1}^\infty \dfrac{(-1)^{n+1}}{n}z^n}

            En primer lugar, veamos que $g$ está bien definida, es decir, que la serie converge. Tenemos que:
            \begin{equation*}
                \left\{\dfrac{1^{n+2}}{n+1}\cdot \dfrac{n}{1^{n+1}}\right\} = \left\{\dfrac{n}{n+1}\right\}\to 1
            \end{equation*}
            Por tanto, la serie converge absolutamente en $D(0,1)$.\\

            Como $\log\in \cc{H}(\bb{C}^*\setminus\bb{R}^-)$, en particular $f\in \cc{H}(D(0,1))$. Por otro lado, por el Teorema de Holomorfía de las funciones dadas como suma de series de potencias, se tiene que $g\in \cc{H}(D(0,1))$. Calculamos ambas derivadas para cada $z\in D(0,1)$:
            \begin{align*}
                f'(z) &= \frac{1}{1+z} \\
                g'(z) &= \sum_{n=1}^\infty \frac{(-1)^{n+1}}{n}nz^{n-1} = \sum_{n=1}^\infty (-1)^{n+1}z^{n-1} = \sum_{n=0}^\infty (-1)^{n+2}z^{n} = \sum_{n=0}^\infty (-1)^{n}z^{n}
                =\\&= \sum_{n=0}^\infty (-z)^{n}
            \end{align*}

            Como $|z|=|-z|<1$, se tiene que $-z\in D(0,1)$, y por tanto dicha serie geométrica converge. Es decir:
            \begin{align*}
                g'(z) = \sum_{n=0}^\infty (-z)^{n}
                = \dfrac{1}{1-(-z)} = \dfrac{1}{1+z} = f'(z)\qquad \forall z\in D(0,1)
            \end{align*}

            Como $D(0,1)$ es un dominio, entonces $\exists \lm\in \bb{C}$ tal que $f=g+\lm$. Evaluando en $0$, tenemos que $f(0)=0=g(0)$, luego $\lm=0$. Por tanto, se tiene que $f=g$ en $D(0,1)$ como queríamos probar.
        \end{description}
        \item $\sum\limits_{n= 1}^\infty \dfrac{z^{2n+1}}{n(2n+1)} = 2z - (1+z)\log(1+z) + (1-z)\log(1-z)$
            
        Fijamos $z\in D(0,1)$. Entonces, por el primer apartado:
        \begin{align*}
            \log(1+z) &= \sum_{n=1}^\infty \frac{(-1)^{n+1}}{n}z^n
        \end{align*}

        Como $-z\in D(0,1)$, se tiene que:
        \begin{align*}
            \log(1-z) &= \sum_{n=1}^\infty \frac{(-1)^{n+1}}{n}(-z)^n = \sum_{n=1}^\infty \frac{(-1)^{2n+1}}{n}z^n
            = \sum_{n=1}^\infty -\frac{z^n}{n}
        \end{align*}

        Definimos las siguientes funciones:
        \Func{f}{D(0,1)}{\mathbb{C}}{z}{2z-(1+z)\log(1+z)+(1-z)\log(1-z)}
        \Func{g}{D(0,1)}{\mathbb{C}}{z}{\sum\limits_{n= 1}^\infty \dfrac{z^{2n+1}}{n(2n+1)}}

        En primer lugar, veamos que $g$ está bien definida, es decir, que la serie converge. Tenemos que:
        \begin{equation*}
            \left\{\dfrac{1}{(n+1)(2(n+1)+1)}\cdot \dfrac{n(2n+1)}{1}\right\} \to 1
        \end{equation*}

        Por tanto, la serie converge absolutamente en $D(0,1)$.\\

        Como $\log\in \cc{H}(D(0,1))$, en particular $f\in \cc{H}(D(0,1))$. Por otro lado, por el Teorema de Holomorfía de las funciones dadas como suma de series de potencias, se tiene que $g\in \cc{H}(D(0,1))$. Calculamos ambas derivadas para cada $z\in D(0,1)$:
        \begin{align*}
            f'(z) &= 2 - \log(1+z) - \frac{1+z}{1+z} - \log(1-z) - \frac{1-z}{1-z} = -\left(\log(1+z) + \log(1-z)\right)
            =\\&= -\sum_{n=1}^\infty \left(\frac{(-1)^{n+1}}{n}z^n - \dfrac{z^n}{n}\right) = -\sum_{n=1}^\infty \frac{z^n}{n}\left((-1)^{n+1}-1\right)\\
            g'(z) &= \sum_{n=1}^\infty \frac{(2n+1)z^{2n}}{n(2n+1)} = \sum_{n=1}^\infty \dfrac{z^{2n}}{n}
        \end{align*}

        Buscamos ahora simplificar $f'(z)$, y para ello razonamos según la paridad de $n$. Si $n$ es impar, entonces $n+1$ es par y dicho sumando se anula. Por tanto, tan solo quedan los sumandos pares. Es decir:
        \begin{align*}
            f'(z) = -\sum_{n=1}^\infty \frac{z^{2n}}{2n}\left((-1)^{2n+1}-1\right) = -\sum_{n=1}^\infty \frac{z^{2n}}{2n}\left(-2\right) = \sum_{n=1}^\infty \frac{z^{2n}}{n}
        \end{align*}

        Por tanto, $f'(z)=g'(z)$ para todo $z\in D(0,1)$. Como $D(0,1)$ es un dominio, entonces $\exists \lm\in \bb{C}$ tal que $f=g+\lm$. Evaluando en $0$, tenemos que $f(0)=0=g(0)$, luego $\lm=0$. Por tanto, se tiene que $f=g$ en $D(0,1)$ como queríamos probar.
    \end{enumerate}
\end{ejercicio}

\begin{ejercicio}
    Sea la siguiente función:
    \Func{f}{\mathbb{C}\setminus\{1,-1\}}{\mathbb{C}}{z}{\log\left(\frac{1+z}{1-z}\right)}
    Probar que $f$ es holomorfa en el dominio $W = \mathbb{C} \setminus \{x \in \mathbb{R} : |x| \geq 1\}$ y calcular su derivada. Probar también que
    \[
        f(z) = 2\sum_{n=0}^\infty \frac{z^{2n+1}}{2n+1} \quad \forall z \in D(0,1)
    \]

    Definimos las siguientes funciones auxiliares:
    \Func{g}{W}{\bb{C}^*\setminus \bb{R}^-}{z}{\dfrac{1+z}{1-z}}
    \Func{h}{\bb{C}^*\setminus \bb{R}^-}{\bb{C}}{w}{\log w}

    Veamos en primer lugar que $g$ está bien definida. Por reducción al absurdo, sea $z=x+iy\in W$ tal que $\exists \lm\in \bb{R}^+_0$ tal que $g(z)=-\lm$. Entonces, se tiene que:
    \begin{align*}
        \dfrac{1+x+iy}{1-x-iy} &= -\lm \Longrightarrow 1+x+iy = -\lm(1-x-iy)
        \Longrightarrow \begin{cases}
            1+x = -\lm(1-x) \\
            y = \lm y
        \end{cases}
    \end{align*}
    \begin{itemize}
        \item Si $y\neq 0$, entonces $\lm = 1$, y se tiene que:
        \begin{equation*}
            1+x = x-1 \Longrightarrow 1 = -1\qquad \text{(absurdo)}
        \end{equation*}

        \item Si $y=0$, entonces se tiene que $z=x\in \bb{R}$, y como $z\in W$, se tiene que $|x|<1$. Por tanto, se tiene que:
        \begin{align*}
            \lm = \frac{1+x}{x-1}<0\qquad \text{(absurdo)}
        \end{align*}
    \end{itemize}

    En cualquier caso, vemos que $g$ no puede tomar valores reales negativos. Por tanto, se tiene que $g(z)\in \bb{C}^*\setminus \bb{R}^-$, y por tanto $h$ está bien definida. Sabemos que $g\in \cc{H}(W)$ por ser racional y $h\in \cc{H}(\bb{C}^*\setminus \bb{R}^-)$ por tratarse del logaritmo principal. Por tanto, por la regla de la cadena, tenemos que $f=h\circ g\in \cc{H}(W)$, con:
    \begin{align*}
        f'(z) &= h'(g(z))\cdot g'(z) = \frac{1}{g(z)}\cdot g'(z)
        = \dfrac{1-z}{1+z}\cdot \dfrac{(1-z)+(1+z)}{(1-z)^2}
        =\\&=\dfrac{1-z}{1+z}\cdot \dfrac{2}{(1-z)^2}
        = \dfrac{2}{(1+z)(1-z)}
        = \dfrac{2}{1-z^2}\qquad \forall z\in W
    \end{align*}

    Para comprobar ahora esa expresión como suma de serie de potencias, definimos la siguiente función:
    \Func{g}{D(0,1)}{\bb{C}}{z}{2\sum\limits_{n=0}^\infty \frac{z^{2n+1}}{2n+1}}

    En primer lugar, veamos que $g$ está bien definida, es decir, que la serie converge. Tenemos que:
    \begin{equation*}
        \left\{\dfrac{1}{2(n+1)+1}\cdot \dfrac{2n+1}{1}\right\}\to 1
    \end{equation*}

    Por tanto, la serie converge absolutamente en $D(0,1)$, y por tanto $g$ está bien definida. Por el Teorema de Holomorfía de las funciones dadas como suma de series de potencias, se tiene que $g\in \cc{H}(D(0,1))$. Calculamos su derivada para cada $z\in D(0,1)$:
    \begin{align*}
        g'(z) &= 2\sum_{n=0}^\infty \frac{(2n+1)z^{2n}}{2n+1} = 2\sum_{n=0}^\infty z^{2n} = 2\sum_{n=0}^\infty (z^2)^{n}
        \AstIg \dfrac{2}{1-z^2}\qquad \forall z\in D(0,1)
    \end{align*}
    donde en $(\ast)$ hemos usado que, como $|z|<1$, se tiene que $|z^2|<1$.\\

    Por tanto, como $D(0,1)\subset W$ es un dominio, entonces $\exists \lm\in \bb{C}$ tal que $f=g+\lm$. Evaluando en $0$, tenemos que $f(0)=0=g(0)$, luego $\lm=0$. Por tanto, se tiene que $f=g$ en $D(0,1)$ como queríamos probar.
\end{ejercicio}

\begin{ejercicio}
    Sean $\alpha,\beta \in \left[ -\pi, \pi \right]$ con $\alpha < \beta$, y $\rho \in \mathbb{R}^+$ tal que $\rho\alpha,\rho\beta \in \left[ -\pi, \pi \right]$. Consideramos los siguientes dominios:
    \begin{align*}
        \Omega &= \{z \in \mathbb{C}^* : \alpha < \arg z < \beta\} \\
        \Omega_\rho &= \{z \in \mathbb{C}^* : \rho\alpha < \arg z < \rho\beta\}
    \end{align*}
    Probar que la siguiente función define una biyección de $\Omega$ sobre el dominio $\Omega_\rho$:
    \Func{f}{\Omega}{\Omega_\rho}{z}{z^\rho}

    Veamos en primer lugar que está bien definida. En efecto, sea $z\in \Omega$, entonces se tiene que:
    \begin{align*}
        f(z) &= z^\rho = \exp(\rho\log z) = \exp\left(\rho\left(\log |z| + i\arg z\right)\right) = \exp\left(\rho\log |z| + i\rho\arg z\right)
    \end{align*}
    Por tanto, $\rho\arg z \in \Arg f(z)$. Como $\alpha < \arg z < \beta$, se tiene que:
    \begin{align*}
        -\pi\leq \rho\alpha < \rho\arg z < \rho\beta\leq \pi
    \end{align*}
    Por tanto, $\rho\arg z=\arg f(z)$. Por tanto, se tiene que:
    \begin{align*}
        \rho\alpha < \arg f(z) < \rho\beta
        \Longrightarrow f(z)\in \Omega_\rho
    \end{align*}
    Por tanto, $f$ está bien definida.
    \begin{itemize}
        \item \ul{Inyectividad}~
        
        Sea $z_1,z_2\in \Omega$ tales que $f(z_1)=f(z_2)$. Entonces, se tiene que:
        \begin{align*}
            z_1^\rho = z_2^\rho &\Longrightarrow \log(z_1^\rho) = \log(z_2^\rho) \Longrightarrow \ln|z_1^\rho| + i\arg(z_1^\rho) = \ln|z_2^\rho| + i\arg(z_2^\rho)\\
            &\Longrightarrow \rho\ln|z_1| + i\arg(z_1^\rho) = \rho\ln|z_2| + i\arg(z_2^\rho)
        \end{align*}

        Igualando las partes reales, obtenemos:
        \begin{align*}
            |z_1^\rho| &= |z_2^\rho| \Longrightarrow e^{\Re(\rho\log z_1)} = e^{\Re(\rho\log z_2)} \Longrightarrow \rho\ln|z_1| = \rho\ln|z_2|\Longrightarrow |z_1| = |z_2|
        \end{align*}
        
        Por otro lado, igualando las partes imaginarias, se tiene que:
        \begin{align*}
            \arg(z_1^\rho) &= \arg(z_2^\rho) \Longrightarrow \Arg(z_1^\rho) = \Arg(z_2^\rho) \Longrightarrow \rho\Arg(z_1) = \rho\Arg(z_2)\Longrightarrow \Arg(z_1) = \Arg(z_2)
        \end{align*}

        Por tanto, se tiene que $z_1=z_2$. Por tanto, $f$ es inyectiva.

        \item \ul{Sobreyectividad}~
        
        Dado $w\in \Omega_\rho$, consideramos $z=w^{1/\rho}$. Entonces, se tiene que:
        \begin{equation*}
            z=w^{1/\rho} = \exp\left(\frac{\log w}{\rho}\right) = \exp\left(\frac{\ln |w| + i\arg w}{\rho}\right)\Longrightarrow
            \dfrac{\arg w}{\rho}\in \Arg z
        \end{equation*}

        Además, como $\rho\alpha < \arg w < \rho\beta$, se tiene que:
        \begin{align*}
            -\pi\leq \alpha < \frac{\arg w}{\rho} < \beta\leq \pi
            \Longrightarrow \arg z = \frac{\arg w}{\rho}\Longrightarrow z\in \Omega
        \end{align*}

        Veamos ahora que $f(z)=w$. En efecto, se tiene que:
        \begin{align*}
            f(z) &= z^\rho = \left(w^{1/\rho}\right)^\rho = w
        \end{align*}
        Por tanto, $f$ es sobreyectiva.
    \end{itemize}

    Por tanto, $f$ es biyectiva.
\end{ejercicio}

\begin{ejercicio}
    Probar que el seno, el coseno y la tangente son funciones simplemente periódicas.\\

    Comprobemos que el seno y el coseno son funciones simplemente periódicas, con periodo fundamental $2\pi$. En primer lugar, tenemos que $\bb{C}+2\pi=\bb{C}$. Además, para todo $z\in \bb{C}$, se tiene que:
    \begin{align*}
        \sen(z+2\pi) &= \sen z\cos(2\pi)+\cos z\sen(2\pi) = \sen z\cdot 1 + \cos z\cdot 0 = \sen z\\
        \cos(z+2\pi) &= \cos z\cos(2\pi)-\sen z\sen(2\pi) = \cos z\cdot 1 - \sen z\cdot 0 = \cos z
    \end{align*}

    Por tanto, $2\pi$ es un periodo de $\sen$ y $\cos$. Sea ahora $w\in \bb{C}$ otro periodo. Entonces, para todo $z\in \bb{C}$, se tiene que:
    \begin{align*}
        \sen z &= \sen(z+w) = \sen z\cos w + \cos z\sen w\\
        \cos z &= \cos(z+w) = \cos z\cos w - \sen z\sen w
    \end{align*}

    Considerando las restricciones a $\bb{R}$, como $\{\sen z,\cos z\}$ son linealmente independientes, se tiene que:
    \begin{align*}
        \cos w &= 1 \\
        \sen w &= 0
    \end{align*}

    Como $\sen w=0$, entonces:
    \begin{align*}
        e^{iw} = e^{-iw}&\Longrightarrow e^{2iw} = 1\Longrightarrow \exp(\Re(2iw))=1
        \Longrightarrow \Re(2iw) = 0\Longrightarrow \\ &\Longrightarrow -2\Im(w) = 0\Longrightarrow \Im(w) = 0
        \Longrightarrow w\in \bb{R}
    \end{align*}

    Por tanto, como $w\in \bb{R}$, conocemos sin problema las soluciones de dicho sistema. Se tiene que $w=2k\pi$ para algún $k\in \bb{Z}$. Por tanto, $2\pi$ es el periodo fundamental de $\sen$ y $\cos$, y por tanto son funciones simplemente periódicas.\\

    Estudiamos ahora la función tangente. Sea $\Omega$ el dominio de la función tangente:
    \begin{align*}
        \Omega &= \bb{C}\setminus\left\{z\in \bb{C} : \cos z = 0\right\} = \bb{C}\setminus\left\{(2k+1)\frac{\pi}{2} : k\in \bb{Z}\right\}
        = \bb{C}\setminus\left\{\pi k +\frac{\pi}{2} : k\in \bb{Z}\right\}
    \end{align*}
    
    De la última igualdad, se deduce que $\Omega+\pi=\Omega$. Además, para todo $z\in \bb{C}$, se tiene que:
    \begin{align*}
        \tan(z+\pi) &= \frac{\sen(z+\pi)}{\cos(z+\pi)} = \frac{\sen z\cos(\pi)+\cos z\sen(\pi)}{\cos z\cos(\pi)-\sen z\sen(\pi)} = \frac{\sen z\cdot (-1)+\cos z\cdot 0}{\cos z\cdot (-1)-\sen z\cdot 0} = \tan z
    \end{align*}

    Por tanto, $\pi$ es un periodo de $\tan$. Sea ahora $w\in \bb{C}$ otro periodo. Entonces, para todo $z\in \bb{C}$, se tiene que:
    \begin{align*}
        \tan z &= \tan(z+w) = \frac{\sen(z+w)}{\cos(z+w)} = \frac{\sen z\cos w + \cos z\sen w}{\cos z\cos w - \sen z\sen w}
        \Longrightarrow\\&\Longrightarrow
        \sen z \left(\cos z \cos w - \sen z\sen w\right) = \cos z \left(\sen z \cos w + \cos z\sen w\right)
        \Longrightarrow\\&\Longrightarrow
        \cancel{\sen z \cos z \cos w} - \sen^2 z \sen w = \cancel{\sen z \cos z \cos w} + \cos^2 z \sen w
        \Longrightarrow\\&\Longrightarrow
        0 = \sen w\left(\cos^2 z + \sen^2 z\right) = \sen w
    \end{align*}
    
    Como $\sen w=0$, vimos anteriormente que $w\in \bb{R}$. Por tanto, las soluciones de $\sen w=0$ son $w=k\pi$ para algún $k\in \bb{Z}$. Por tanto, $\pi$ es el periodo fundamental de $\tan$, y por tanto es una función simplemente periódica.
\end{ejercicio}

\begin{ejercicio}
    Estudiar la convergencia de la serie
    \[
        \sum_{n\geq 0} \frac{\sen(nz)}{2^n}
    \]

    Definimos la siguiente función para cada $n\in \bb{N}$:
    \Func{f_n}{\bb{C}}{\bb{C}}{z}{\dfrac{\sen(nz)}{2^n}}

    Para todo $z\in \bb{C}$, se tiene que:
    \begin{align*}
        f_n(z) &= \frac{\sen(nz)}{2^n} = \frac{e^{inz}-e^{-inz}}{2i\cdot 2^n} = \frac{1}{2i}\left(\frac{e^{inz}}{2^n}-\frac{e^{-inz}}{2^n}\right) = \frac{1}{2i}\left[\left(\frac{e^{iz}}{2}\right)^{n}-\left(\frac{e^{-iz}}{2}\right)^{n}\right]
    \end{align*}

    Buscamos ahora calcular $|f_n(z)|$. Para ello, antes vemos que:
    \begin{align*}
        \left|\frac{e^{iz}}{2}\right| &= \frac{|e^{iz}|}{2} = \dfrac{\exp(\Re(iz))}{2} = \dfrac{\exp(-\Im(z))}{2}\\
        \left|\frac{e^{-iz}}{2}\right| &= \frac{|e^{-iz}|}{2} = \dfrac{\exp(\Re(-iz))}{2} = \dfrac{\exp(\Im(z))}{2}
    \end{align*}

    Por tanto, empleando la desigualdad triangular en ambos sentidos, se tienen para todo $z\in \bb{C}$ las siguientes desigualdades:
    \begin{align*}
        \dfrac{1}{2}\left|\left(\dfrac{\exp(-\Im(z))}{2}\right)^{n}-\left(\dfrac{\exp(\Im(z))}{2}\right)^{n}\right| &\leq |f_n(z)| \leq \dfrac{1}{2}\left[\left(\dfrac{\exp(-\Im(z))}{2}\right)^{n}+\left(\dfrac{\exp(\Im(z))}{2}\right)^{n}\right]
    \end{align*}

    Antes de estudiar la convergencia absoluta y puntual, veremos cuándo convergen ambas series geométricas presentes en la desigualdad anterior. En efecto, tenemos que:
    \begin{align*}
        \dfrac{\exp(-\Im(z))}{2}<1 \iff \Im(z) > -\ln(2)\\
        \dfrac{\exp(\Im(z))}{2}<1 \iff \Im(z) < \ln(2)
    \end{align*}

    Veamos ahora que la serie converge absolutamente en el siguiente conjunto, mientras que no converge (ni siquiera puntualmente) en el resto de $\bb{C}$:
    \begin{equation*}
        H=\left\{z\in \bb{C} : -\ln(2)<\Im(z)<\ln(2)\right\}=\left\{z\in \bb{C} : |\Im(z)|<\ln(2)\right\}
    \end{equation*}
    \begin{itemize}
        \item Sea $z\in H$. Entonces, las siguientes series geométricas convergen, pues tienen razon cuyo valor absoluto es menor que $1$:
        \begin{equation*}
            \sum_{n\geq 0}\left(\dfrac{\exp(-\Im(z))}{2}\right)^{n} \text{ y } \sum_{n\geq 0}\left(\dfrac{\exp(\Im(z))}{2}\right)^{n}
        \end{equation*}

        Por tanto, la serie con término general la cota superior de $|f_n(z)|$ converge, y por el criterio de comparación de series en $\bb{R}$, se tiene que la serie de término de general $|f_n(z)|$ converge. Por tanto, la serie converge absolutamente (y en particular puntualmente) en $H$.

        \item Sea $z\in \bb{C}\setminus H$. Entonces, puede darse $\Im(z) > \ln(2)$ o $\Im(z) < -\ln(2)$. En ambos casos, una de las dos sucesiones geométricas de la cota inferior de $|f_n(z)|$ converge a $0$, mientras que la otra converge a $1$ o diverge positivamente. Por tanto, la cota inferior de $|f_n(z)|$ no converge a $0$, y por tanto:
        \begin{equation*}
            \{|f_n(z)|\} \not\to 0 \Longrightarrow \{f_n(z)\}\not\to 0
        \end{equation*}

        Por tanto, la serie de término general $f_n(z)$ no converge, y por tanto no hay convergencia puntual (y por tanto tampoco absoluta) en $\bb{C}\setminus H$.
    \end{itemize}

    Por tanto, hemos probado que la serie converge absolutamente (y en particular puntualmente) en $H$, y no converge (ni siquiera puntualmente) en $\bb{C}\setminus H$.\\

    Estudiamos ahora la convergencia uniforme de la serie. Sea $\emptyset\neq \Omega \subset H$ un conjunto. Entonces, la serie converge uniformemente en $\Omega$ si y solo si
    \begin{equation*}
        r=\sup\{|\Im(z)| : z\in \Omega\} < \ln(2)
    \end{equation*}
    \begin{description}
        \item[$\Longleftarrow)$] Buscamos aplicar el Test de Weierstrass:
        \begin{align*}
            |f_n(z)| &\leq 
             \dfrac{1}{2}\left[\left(\dfrac{\exp(-\Im(z))}{2}\right)^{n}+\left(\dfrac{\exp(\Im(z))}{2}\right)^{n}\right]
            \leq\\&\leq \dfrac{1}{2}\left[\left(\dfrac{\exp(r)}{2}\right)^{n}+\left(\dfrac{\exp(r)}{2}\right)^{n}\right] \qquad \forall z\in \Omega
        \end{align*}

        Como $0\leq r<\ln(2)<1$, se tiene que ambas series geométricas de la cota superior (que ya no depende de $z$) convergen. Por el Test de Weierstrass, se tiene que la serie converge uniformemente en $\Omega$.

        \item[$\Longrightarrow)$] Como $\Omega\subset H$, no es posible $r>\ln(2)$. Supongamos $r=\ln(2)$. Entonces, para cada $n\in \bb{N}$, existe $z_n\in \Omega$ tal que:
        \begin{align*}
            |\Im(z_n)| > \ln(2)-\frac{1}{n}
        \end{align*}

        Por tanto, para $n>1$, como $\ln(2)>\nicefrac{1}{2}$, se tiene que:
        \begin{align*}
            |f_n(z_n)| &\geq \dfrac{1}{2}\left|\left(\dfrac{\exp(-\Im(z_n))}{2}\right)^{n}-\left(\dfrac{\exp(\Im(z_n))}{2}\right)^{n}\right|
            >\\& > \dfrac{1}{2}\left|\left(\dfrac{\exp(-\left(\ln(2)-\nicefrac{1}{n}\right))}{2}\right)^{n}-\left(\dfrac{\exp(\ln(2)-\nicefrac{1}{n})}{2}\right)^{n}\right|
            =\\&= \dfrac{1}{2}\left[\left(\dfrac{\exp(\ln(2)-\nicefrac{1}{n})}{2}\right)^{n}-\left(\dfrac{1}{2\exp(\ln(2)-\nicefrac{1}{n})}\right)^{n}\right]
            =\\&= \dfrac{1}{2}\left[\left(\dfrac{\exp(\ln(2))}{2\exp\left(\nicefrac{1}{n}\right)}\right)^{n}-\left(\dfrac{\exp\left(\nicefrac{1}{n}\right)}{2\exp(\ln(2))}\right)^{n}\right]
            =\\&= \dfrac{1}{2}\left[\left(\dfrac{1}{\exp\left(\nicefrac{1}{n}\right)}\right)^{n}-\left(\dfrac{\exp\left(\nicefrac{1}{n}\right)}{4}\right)^{n}\right]
            = \dfrac{1}{2}\left[\dfrac{1}{e}-\dfrac{e}{4^n}\right]
        \end{align*}

        Por tanto, vemos que $\{|f_n(z_n)|\}\not\to 0$. Por tanto, la serie no converge uniformemente en $\Omega$.
    \end{description}

    A partir de esto, vemos varios resultados. En primer lugar, veamos que la serie no converge uniformemente en $H$. Definimos la siguiente sucesión:
    \begin{align*}
        \{z_n\} &= \left\{i\left(1-\frac{1}{n}\right)\ln(2)\right\}\subset H
    \end{align*}
    Entonces, se tiene que:
    \begin{equation*}
        \{|\Im(z_n)|\} = \{\Im(z_n)\} = \left\{\left(1-\frac{1}{n}\right)\ln(2)\right\}\to \ln(2)
    \end{equation*}

    Como $\ln(2)$ es un mayorante de $\{|\Im(z)| : z\in H\}$ y $\{|\Im(z_n)|\}\to \ln(2)$, se tiene que $r=\ln(2)$. Por tanto, la serie no converge uniformemente en $H$.\\

    Por otro lado, si $K\subset H$ es compacto, veamos que $r<\ln(2)$. Como $K\subset H$, no puede darse $r>\ln(2)$. Supongamos $r=\ln(2)$, y llegaremos a una contradicción. Como $r=\ln(2)$, existe una sucesión $\{z_n\}\subset K$ tal que:
    \begin{align*}
        \{|\Im(z_n)|\}\to \ln(2)
    \end{align*}

    Como $K$ es compacto y $\Im(z)$, $|\cdot|$ son continuas en $\bb{C}$, se tiene que $|\Im(K)|$ es compacto, y en particular cerrado. Por tanto, $\ln(2)\in |\Im(K)|$, y entonces existe $z_0\in K$ tal que $|\Im(z_0)|=\ln(2)$, por lo que $z_0\notin H$, en contradicción con que $K\subset H$. Por tanto, se tiene que $r<\ln(2)$ y la serie converge uniformemente en $K$.
\end{ejercicio}

\begin{ejercicio}
    Sea $\Omega = \mathbb{C} \setminus \{x \in \mathbb{R} : |x| \geq 1\}$. Probar que existe $f \in \mathcal{H}(\Omega)$ tal que $\cos f(z) = z$ para todo $z \in \Omega$ y $f(x) = \arccos x$ para todo $x \in \left] -1, 1 \right[$. Calcular la derivada de $f$.\\

    Sea $z\in \Omega$, y veamos las soluciones en $\bb{C}$ de la ecuación $\cos w = z$. \begin{align*}
        \cos w = z&\iff \dfrac{e^{iw}+e^{-iw}}{2} = z\iff e^{iw} + e^{-iw} = 2z\iff e^{2iw} - 2ze^{iw} + 1 = 0
        \iff\\&\iff \left(e^{iw}-z\right)^2 = z^2-1\iff e^{iw}-z\in \left[(z^2-1)^{\nicefrac{1}{2}}\right]
    \end{align*}

    Como buscamos una función holomorfa, tan solo usaremos en la construcción funciones holomorfas, por lo que buscamos un elemento de dicho conjunto que sea holomorfo. Dado $z\in \Omega$, veamos que $z^2-1\in \bb{C}^*\setminus\bb{R}^+$. Supongamos que $z^2-1\in \bb{R}^+_0$. Entonces, $z^2\in \bb{R}^+$, y por tanto:
    \begin{equation*}
        2\pi\bb{Z}=\Arg(z^2)=2\Arg(z)\Longrightarrow \Arg(z)=\pi\bb{Z}\Longrightarrow z\in \bb{R}
    \end{equation*}
    Por tanto, $z\in \bb{R}$ y $|z|<1$. Entonces $z^2-1=|z|^2-1<1-1=0$, por lo que llegamos a una contradicción y $z^2-1\in \bb{C}^*\setminus\bb{R}^+$. Como la raíz cuadrada principal sabemos que es holomorfa en $\bb{C}^*\setminus\bb{R}^-$, usamos que:
    \begin{align*}
        e^{iw}-z & \in \left[(z^2-1)^{\nicefrac{1}{2}}\right]\iff e^{iw}-z\in \left[(-1)^{\nicefrac{1}{2}}\right]\left[(1-z^2)^{\nicefrac{1}{2}}\right]
    \end{align*}

    Consideramos ahora las raíces principales, sabiendo que $(-1)^{\nicefrac{1}{2}}=\exp\left(\frac{\pi i}{2}\right)=i$. Como $1-z^2\in \bb{C}^*\setminus\bb{R}^-$ para todo $z\in \Omega$, esta es holomorfa en $\Omega$. Por tanto:
    \begin{align*}
        e^{iw}&= z + i\left(1-z^2\right)^{\nicefrac{1}{2}}
        = z+i\cdot \exp\left(\frac{\log(1-z^2)}{2}\right)
        = z+i\cdot \exp\left(\frac{\ln |1-z^2| + i\arg(1-z^2)}{2}\right)
    \end{align*}

    Por tanto, encontrar la raíz holomorfa se resume en encontrar un logaritmo holomorfo en $\bb{C}^*\setminus\bb{R}^+$, y esto se tendrá si conseguimos un argumento continuo en dicho conjunto. Para ello, consideramos la función $\varphi\in C(\bb{C}^*\setminus \bb{R}^+)$ que nos da el Ejercio~\ref{ej:2.2}, de forma que $\varphi(z)\in \Arg(z)$ para todo $z\in \bb{C}^*\setminus \bb{R}^+$. A partir de él, definimos:
    \Func{\log_{\varphi}}{\bb{C}^*\setminus \bb{R}^+}{\bb{C}}{z}{\ln |z| + i\varphi(z)}
    Entonces, se tiene que $\log_{\varphi}\in \cc{C}(\bb{C}^*\setminus \bb{R}^+)$ verificando $e^{\log_\varphi(z)}=z$ para cada valor de $z\in \bb{C}^*\setminus \bb{R}^+$, y por tanto $\log_{\varphi}\in \cc{H}(\bb{C}^*\setminus \bb{R}^+)$. Por último, definimos la función:
    \Func{\left(\cdot\right)^{\nicefrac{1}{2}}_{\varphi}}{\bb{C}^*\setminus \bb{R}^+}{\bb{C}}{z}{\exp\left(\frac{\log_{\varphi}(z)}{2}\right)}
    Entonces, se tiene que $\left(z\right)^{\nicefrac{1}{2}}_{\varphi}\in [z^{\nicefrac{1}{2}}]$ para cada $z\in \bb{C}^*\setminus \bb{R}^+$ verificando además lo buscado, es decir, $\left(\cdot\right)^{\nicefrac{1}{2}}_{\varphi}\in \cc{H}(\bb{C}^*\setminus \bb{R}^+)$. Por tanto, elegimos dicha raíz.
    \begin{align*}
        e^{iw} -z = \left(z^2-1\right)^{\nicefrac{1}{2}}_{\varphi}
        \iff e^{iw} &= z + \left(z^2-1\right)^{\nicefrac{1}{2}}_{\varphi}
        = z+\exp\left(\frac{\log_{\varphi}(z^2-1)}{2}\right)
        =\\&= z+\exp\left(\frac{\ln |z^2-1| + i\varphi(z^2-1)}{2}\right)
        \iff \\&\iff iw\in \Log\{z + \left(z^2-1\right)^{\nicefrac{1}{2}}_{\varphi}\}
        \iff\\&\iff w\in -i\Log\left\{z + \left(z^2-1\right)^{\nicefrac{1}{2}}_{\varphi}\right\}
    \end{align*}

    Llegados a este punto, hemos de seleccionar un logaritmo holomorfo, pero sabemos que para ello basta con encontrar un argumento continuo de la función $z + \left(z^2-1\right)^{\nicefrac{1}{2}}_{\varphi}$ en $\Omega$.\\

    % // TODO: Continuar


    \begin{comment}
     = \{\pm (z^2-1)^{\nicefrac{1}{2}}\}
        \iff\\&\iff e^{iw} = z \pm (z^2-1)^{\nicefrac{1}{2}}
        \iff iw\in \Log\left(z \pm (z^2-1)^{\nicefrac{1}{2}}\right)
        \iff\\&\iff w\in -i\Log\left(z \pm (z^2-1)^{\nicefrac{1}{2}}\right)
    

    Definimos por tanto la siguiente función (eligiendo la raíz principal y el logaritmo principal):
    \Func{f}{\Omega}{\bb{C}}{z}{-i\log\left(z + (z^2-1)^{\nicefrac{1}{2}}\right)}

    Podríamos demostrar que está bien definida (comprobando para ello que las composiciones son correctas). No obstante, comprobaremos directamente que es holomorfa. En primer lugar, hemos de ver que la raíz lo es.
    \begin{equation*}
        (z^2-1)^{\nicefrac{1}{2}} = \exp\left(\dfrac{\log(z^2-1)}{2}\right)
    \end{equation*}

    Como $\exp\in \cc{H}(\bb{C})$, y $\log\in \cc{H}(\bb{C}^*\setminus\bb{R}^-)$, veamos que $z^2-1\in \bb{C}^*\setminus\bb{R}^-$. Supongamos $z^2-1\in \bb{R}^-_0$, y entonces $z\in \bb{R}$. Además, $z^2\leq 1$, por lo que $|z|\leq 1$. Esto contradice que $z\in \Omega$, por lo que $z^2-1\in \bb{C}^*\setminus\bb{R}^-$. Por tanto, $(z^2-1)^{\nicefrac{1}{2}}$ es holomorfa en $\Omega$.\\
    
    Para demostrar que $f\in \cc{H}(\Omega)$, hemos de ver que el logaritmo es holomorfmo, para lo cual hemos de ver que $z + (z^2-1)^{\nicefrac{1}{2}} \in \bb{C}^*\setminus\bb{R}^-$.
\end{comment} 
\end{ejercicio}

\begin{ejercicio}
    Para $z \in D(0,1)$ con $\Re z \neq 0$, probar que
    \[
        \arctan\left(\frac{1}{z}\right) + \sum_{n=0}^\infty \frac{(-1)^n}{2n+1}z^{2n+1} = \begin{cases}
            \nicefrac{\pi}{2} & \text{si } \Re z > 0 \\
            \nicefrac{-\pi}{2} & \text{si } \Re z < 0
        \end{cases}
    \]

    Por el desarrollo en serie de potencias de la función $\arctan$, vista en clase, hemos de probar que:
    \begin{align*}
        \arctan\left(\frac{1}{z}\right) + \arctan(z) &= \begin{cases}
            \nicefrac{\pi}{2} & \text{si } \Re z > 0 \\
            \nicefrac{-\pi}{2} & \text{si } \Re z < 0
        \end{cases}
    \end{align*}

    Definimos los siguientes conjuntos:
    \begin{align*}
        H_{-} &= \{z\in \bb{C} : \Re z < 0\} \\
        H_{+} &= \{z\in \bb{C} : \Re z > 0\} \\
        H&= H_{+}\cup H_{-}
    \end{align*}

    Definimos ahora la siguiente función:
    \Func{f}{H}{\bb{C}}{z}{\arctan\left(\frac{1}{z}\right) + \arctan(z)}

    Sea $z\in H=H_+\cup H_-$, y supongamos que $\exists y\in \bb{R}$, con $|y|\geq 1$, tal que $\nicefrac{1}{z}=iy$. Entonces, se tiene que:
    \begin{align*}
        \dfrac{1}{z}=iy\Longrightarrow z = \dfrac{1}{iy} = -\dfrac{i}{y}
        \Longrightarrow \Re z = 0\Longrightarrow z\notin H
    \end{align*}

    Por tanto, tenemos que $f\in \cc{H}(H)$. Calculamos su derivada:
    \begin{align*}
        f'(z) &= \dfrac{1}{1+\left(\frac{1}{z}\right)^2}\cdot \left(-\frac{1}{z^2}\right) + \dfrac{1}{1+z^2} = -\dfrac{1}{z^2+1}+\dfrac{1}{1+z^2} = 0\qquad \forall z\in H
    \end{align*}

    Por tanto, $f$ es constante en cada una de las componentes conexas de $H$; es decir, en $H_+$ y $H_-$. Por tanto:
    \begin{align*}
        f(z) &= f(1)=2\arctan(1)=\nicefrac{\pi}{2} \quad \forall z\in H_+\\
        f(z) &= f(-1)=2\arctan(-1)=\nicefrac{-\pi}{2} \quad \forall z\in H_-
    \end{align*}

    Por tanto, se tiene demostrado lo pedido.
\end{ejercicio}
\section{Funciones Elementales}

\begin{ejercicio}
    Sea $f : \mathbb{C} \to \mathbb{C}$ una función verificando que
    \[
        f(z + w) = f(z)f(w) \quad \forall z,w \in \mathbb{C}
    \]
    Probar que, si $f$ es derivable en algún punto del plano, entonces $f$ es entera. Encontrar todas las funciones enteras que verifiquen la condición anterior. Dar un ejemplo de una función que verifique dicha condición y no sea entera.
\end{ejercicio}

\begin{ejercicio}
    Calcular la imagen por la función exponencial de una banda horizontal o vertical y del dominio cuya frontera es un rectángulo de lados paralelos a los ejes.
\end{ejercicio}

\begin{ejercicio}
    Dado $\theta\in \left] -\pi, \pi \right]$, estudiar la existencia del límite en $+\infty$ de la función siguiente:
    \Func{\varphi}{\mathbb{R}^+}{\mathbb{C}}{r}{e^{re^{i\theta}}}
\end{ejercicio}

\begin{ejercicio}
    Probar que si $\{z_n\}$ y $\{w_n\}$ son sucesiones de números complejos, con $z_n \neq 0$ para todo $n \in \mathbb{N}$ y $\{z_n\} \to 1$, entonces
    \[
        \left\{w_n(z_n - 1)\right\} \to \lm \in \mathbb{C} \implies \left\{{z_n}^{w_n}\right\} \to e^{\lm}
    \]
\end{ejercicio}

\begin{ejercicio}
    Estudiar la convergencia puntual, absoluta y uniforme de la serie de funciones
    \[
        \sum_{n\geq 0} e^{-nz^2}
    \]
\end{ejercicio}

\begin{ejercicio}
    Probar que si $a,b,c \in \mathbb{T}$ son vértices de un triángulo equilátero si, y sólo si, $a+b+c = 0$.
\end{ejercicio}

\begin{ejercicio}
    Sea $\Omega$ un subconjunto abierto no vacío de $\mathbb{C}^*$ y $\varphi \in \mathcal{C}(\Omega)$ tal que $\varphi(z)^2 = z$ para todo $z \in \Omega$. Probar que $\varphi \in \mathcal{H}(\Omega)$ y calcular su derivada.
\end{ejercicio}

\begin{ejercicio}
    Probar que, para todo $z \in D(0,1)$ se tiene:
    \begin{enumerate}
        \item $\sum\limits_{n= 1}^\infty \dfrac{(-1)^{n+1}}{n}z^n = \log(1+z)$
        \item $\sum\limits_{n= 1}^\infty \dfrac{z^{2n+1}}{n(2n+1)} = 2z - (1+z)\log(1+z) + (1-z)\log(1-z)$
    \end{enumerate}
\end{ejercicio}

\begin{ejercicio}
    Sea la siguiente función:
    \Func{f}{\mathbb{C}\setminus\{1,-1\}}{\mathbb{C}}{z}{\log\left(\frac{1+z}{1-z}\right)}
    Probar que $f$ es holomorfa en el dominio $W = \mathbb{C} \setminus \{x \in \mathbb{R} : |x| \geq 1\}$ y calcular su derivada. Probar también que
    \[
        f(z) = 2\sum_{n=0}^\infty \frac{z^{2n+1}}{2n+1} \quad \forall z \in D(0,1)
    \]
\end{ejercicio}

\begin{ejercicio}
    Sean $\alpha,\beta \in \left[ -\pi, \pi \right]$ con $\alpha < \beta$, y $\rho \in \mathbb{R}^+$ tal que $\rho\alpha,\rho\beta \in \left[ -\pi, \pi \right]$. Consideramos los siguientes dominios:
    \begin{align*}
        \Omega &= \{z \in \mathbb{C}^* : \alpha < \arg z < \beta\} \\
        \Omega_\rho &= \{z \in \mathbb{C}^* : \rho\alpha < \arg z < \rho\beta\}
    \end{align*}
    Probar que la siguiente función define una biyección de $\Omega$ sobre el dominio $\Omega_\rho$:
    \Func{f}{\Omega}{\Omega_\rho}{z}{z^\rho}
\end{ejercicio}

\begin{ejercicio}
    Probar que el seno, el coseno y la tangente son funciones simplemente periódicas.
\end{ejercicio}

\begin{ejercicio}
    Estudiar la convergencia de la serie
    \[
        \sum_{n\geq 0} \frac{\sen(nz)}{2^n}
    \]
\end{ejercicio}

\begin{ejercicio}
    Sea $\Omega = \mathbb{C} \setminus \{x \in \mathbb{R} : |x| \geq 1\}$. Probar que existe $f \in \mathcal{H}(\Omega)$ tal que $\cos f(z) = z$ para todo $z \in \Omega$ y $f(x) = \arccos x$ para todo $x \in \left] -1, 1 \right[$. Calcular la derivada de $f$.
\end{ejercicio}

\begin{ejercicio}
    Para $z \in D(0,1)$ con $\Re z \neq 0$, probar que
    \[
        \arctan\left(\frac{1}{z}\right) + \sum_{n=0}^\infty \frac{(-1)^n}{2n+1}z^{2n+1} = \begin{cases}
            \nicefrac{\pi}{2} & \text{si } \Re z > 0 \\
            \nicefrac{-\pi}{2} & \text{si } \Re z < 0
        \end{cases}
    \]
\end{ejercicio}
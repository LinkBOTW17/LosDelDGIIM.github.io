\section{Ceros de las funciones holomorfas}

\begin{ejercicio}
    Sea $f\in \cc{H}(D(0,1))$ tal que:
    \begin{equation*}
        |f(z)|\leq \frac{1}{1-|z|} \quad \forall z\in D(0,1).
    \end{equation*}
    Probar que $|f^{(n)}(0)|\leq e(n+1)!$ para todo $n\in \bb{N}$.
\end{ejercicio}

\begin{ejercicio}
    Sea $f$ una función entera verificando que existen constantes $\alpha,\beta,\rho\in~\bb{R}^+$ tales que:
    \begin{equation*}
        z\in \bb{C}, |z|>\rho \implies |f(z)|\leq \alpha |z|^\beta.
    \end{equation*}
    Probar que $f$ es una función polinómica de grado menor o igual que $\beta$.
\end{ejercicio}

\begin{ejercicio}
    Sea $f$ una función entera verificando que:
    \begin{equation*}
        f(z)=f(z+1)=f(z+i) \quad \forall z\in \bb{C}.
    \end{equation*}
    Probar que $f$ es constante.
\end{ejercicio}


\begin{ejercicio}
    Sea $f$ una función entera verificando que:
    \begin{equation*}
        f\left(f(z)\right)=f(z) \quad \forall z\in \bb{C}.
    \end{equation*}
    ¿Qué se puede afirmar sobre $f$?
\end{ejercicio}

\begin{ejercicio}
    En cada uno de los siguientes casos, decidir si existe una función $f$, holomorfa en un entorno del origen, y verificando que $f(\nicefrac{1}{n})=a_n$ para todo $n\in \bb{N}$ suficientemente grande:
    \begin{enumerate}
        \item $a_{2n}=0$,\ $a_{2n-1}=1$ \qquad$\forall n\in \bb{N}$.
        \item $a_{2n}=a_{2n-1}=\dfrac{1}{2n}$\qquad $\forall n\in \bb{N}$.
        \item $a_n=\dfrac{n}{n+1}$\qquad $\forall n\in \bb{N}$.
    \end{enumerate}
\end{ejercicio}

\begin{ejercicio}
    Enunciar y demostrar un resultado referente al orden de los ceros de una suma, producto o cociente de funciones holomorfas.
\end{ejercicio}

\begin{ejercicio}
    Dado un abierto $\Omega$ del plano, probar que el anillo $\cc{H}(\Omega)$ es un dominio de integridad si, y sólo si, $\Omega$ es conexo.
\end{ejercicio}

\begin{ejercicio}
    ¿Qué se puede afirmar sobre dos funciones enteras cuya composición es constante?
\end{ejercicio}

\begin{ejercicio}
    Sea $f$ una función entera verificando que $f(z)\to \infty$ cuando $z\to \infty$. Probar que $f$ es una función polinómica.
\end{ejercicio}

\begin{ejercicio}
    Sea $f$ una función entera verificando que:
    \begin{equation*}
        z\in \bb{C}, |z|=1 \implies |f(z)|=1.
    \end{equation*}
    Probar que existen $\alpha\in \bb{C}$ con $|\alpha|=1$ y $n\in \bb{N}\cup\{0\}$ tales que:
    \begin{equation*}
        f(z)=\alpha z^n \quad \forall z\in \bb{C}.
    \end{equation*}
\end{ejercicio}
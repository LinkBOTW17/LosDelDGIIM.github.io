\section{Ceros de las funciones holomorfas}

\begin{ejercicio}
    Sea $f\in \cc{H}(D(0,1))$ tal que:
    \begin{equation*}
        |f(z)|\leq \frac{1}{1-|z|} \quad \forall z\in D(0,1).
    \end{equation*}
    Probar que $|f^{(n)}(0)|\leq e(n+1)!$ para todo $n\in \bb{N}$.\\

    Por la fórmula de Cauchy para las derivadas, tenemos que:
    \begin{equation*}
        f^{(n)}(0)=\frac{n!}{2\pi i} \int_{C(0,r)} \frac{f(w)}{w^{n+1}} dw\qquad r\in \left]0,1\right[.
    \end{equation*}

    Acotamos ahora dicha integral. Para cada $r\in \left]0,1\right[$, se tiene que:
    \begin{align*}
        |f^{(n)}(0)| &\leq \frac{n!}{2\pi} \left| \int_{C(0,r)} \frac{f(w)}{w^{n+1}} dw \right| \\
        &\leq \frac{n!}{2\pi}\cdot 2\pi r\cdot \max\left\{ \left| \frac{f(w)}{w^{n+1}} \right| : w\in C(0,r) \right\} \\
        &\leq n!\cdot r\cdot \dfrac{1}{r^{n+1}} \cdot \max\left\{ |f(w)| : w\in C(0,r) \right\} \\
        &\leq \frac{n!}{r^n}\cdot \max\left\{ \dfrac{1}{1-|w|} : w\in C(0,r) \right\} \\
        &\leq \frac{n!}{r^n}\cdot \frac{1}{1-r}
    \end{align*}

    Para cada $n\in \bb{N}$, consideramos la siguiente función:
    \Func{f_n}{\left]0,1\right[}{\bb{R}}{r}{\dfrac{n!}{r^n(1-r)}=\dfrac{n!}{r^{n} - r^{n+1}}}

    Tenemos que $f_n$ es de clase infinito, luego podemos hallarle el mínimo absoluto. Como diverge positivamente en $0$ y en $1$, el mínimo absoluto se alcanza en un punto interior.
    \begin{equation*}
        f_n'(r)=-n!\cdot \left(\frac{nr^{n-1} - (n+1)r^n}{r^{2n}(1-r)^2}\right)
        = -n!\cdot \left(\frac{n - (n+1)r}{r^{n+1}(1-r)^2}\right)
        = 0\iff r=\frac{n}{n+1}
    \end{equation*}

    Por tanto, el mínimo absoluto se alcanza en $r=\frac{n}{n+1}$. Por tanto:
    \begin{align*}
        |f^{(n)}(0)| &\leq f_n\left(\frac{n}{n+1}\right)  = \frac{n!}{\left(\frac{n}{n+1}\right)^n\left(1-\frac{n}{n+1}\right)}
        = (n+1)!\cdot \left(\frac{n+1}{n}\right)^n
        = (n+1)!\cdot \left(1+\frac{1}{n}\right)^n
    \end{align*}

    Sabemos que $\left\{\left(1+\frac{1}{n}\right)^n\right\}_{n\in \bb{N}}$ es una sucesión estrictamente creciente y convergente a $e$, luego podemos acotarla por dicho valor, llegando así a la conclusión de que:
    \begin{equation*}
        |f^{(n)}(0)|\leq e(n+1)! \quad \forall n\in \bb{N}.
    \end{equation*}
\end{ejercicio}

\begin{ejercicio}\label{ej:9.2}
    Sea $f$ una función entera verificando que existen constantes $\alpha,\beta,\rho\in~\bb{R}^+$ tales que:
    \begin{equation*}
        z\in \bb{C}, |z|>\rho \implies |f(z)|\leq \alpha |z|^\beta.
    \end{equation*}
    Probar que $f$ es una función polinómica de grado menor o igual que $\beta$.\\

    Como $f$ es entera, en particular admite un desarrollo de Taylor en $\bb{C}$ alrededor del origen:
    \begin{equation*}
        f(z)=\sum_{n=0}^{\infty} \dfrac{f^{(n)}(0)}{n!} z^n \quad \forall z\in \bb{C}.
    \end{equation*}

    Buscamos obtener dichos coeficientes. Para ello, aplicamos la fórmula de Cauchy para las derivadas, válida para todo $r\in \bb{R}^+$:
    \begin{equation*}
        f^{(n)}(0)=\frac{n!}{2\pi i} \int_{C(0,r)} \frac{f(w)}{w^{n+1}} dw
    \end{equation*}

    Acotamos ahora dicha integral, considerando ahora tan solo $r>\rho$:
    \begin{align*}
        |f^{(n)}(0)| &\leq \frac{n!}{2\pi} \left| \int_{C(0,r)} \frac{f(w)}{w^{n+1}} dw \right| \\
        &\leq \frac{n!}{2\pi}\cdot 2\pi r\cdot \max\left\{ \left| \frac{f(w)}{w^{n+1}} \right| : w\in C(0,r) \right\} \\
        &\leq n!\cdot r\cdot \dfrac{\alpha r^\beta}{r^{n+1}} = n!\cdot \alpha r^{\beta-n}.
    \end{align*}

    Sea ahora $k\in \bb{N}$ el mayor entero tal que $k\leq \beta$. Entonces, para todo $n\in \bb{N}$ tal que $n>k$, tenemos que (tomando $r > \max\left\{1,\rho \right\}$):
    \begin{equation*}
        |f^{(n)}(0)|\leq n!\cdot \alpha r^{\beta-n}
    \end{equation*}

    Como es válido para todo $r> \max \left\{1,\rho \right\}$, tomando límite con $r\to \infty$, y haciendo uso de que $\beta-n<0$, obtenemos que:
    \begin{equation*}
        |f^{(n)}(0)| \leq n!\cdot \alpha \cdot \lim_{r\to \infty} r^{\beta-n}=0 \quad \forall n>k.
    \end{equation*}

    Por tanto, $f^{(n)}(0)=0$ para todo $n>k$, luego el desarrollo de Taylor de $f$ es:
    \begin{equation*}
        f(z)=\sum_{n=0}^{k} \dfrac{f^{(n)}(0)}{n!} z^n,
    \end{equation*}
    que es un polinomio de grado menor o igual que $k\leq \beta$ (no podemos garantizar que sea de grado $k$). Por tanto, $f$ es una función polinómica de grado menor o igual que $\beta$.
\end{ejercicio}

\begin{ejercicio}
    Sea $f$ una función entera verificando que:
    \begin{equation*}
        f(z)=f(z+1)=f(z+i) \quad \forall z\in \bb{C}.
    \end{equation*}
    Probar que $f$ es constante.\\

    Sea $A$ el cuadrado unitario:
    \begin{align*}
        A &= \left\{ z\in \bb{C} : 0\leq \Re(z),\Im(z)<1 \right\} \\
        &= \left[0,1\right[ \times \left[0,1\right[.
    \end{align*}

    Veamos que, para cada $w\in \bb{C}$, existe un único $z\in A$ tal que $f(z)=f(w)$. Haremos uso de que, para todo $x\in \bb{R}$, existe un único $y\in \left[0,1\right[$ tal que $x=k+y$ para algún $k\in \bb{Z}$. Aplicando ese concepto a la parte real e imaginaria de $w$, podemos escribir:
    \begin{equation*}
        w=k_1 + k_2 i + y_1 + y_2 i,
    \end{equation*}
    donde $k_1,k_2\in \bb{Z}$ y $y_1,y_2\in \left[0,1\right[$. Entonces, definimos $z=y_1 + y_2 i\in A$. Veamos que $f(z)=f(w)$:
    \begin{align*}
        f(z) &= f(y_1 + y_2 i) \\
        &= f(k_1 + k_2 i + y_1 + y_2 i) \\
        &= f(w).
    \end{align*}
    donde hemos usado que $f$ es periódica con periodo $1$ y $i$. Por tanto, hemos demostrado que, para cada $w\in \bb{C}$, existe un único $z\in A$ tal que $f(z)=f(w)$.\\

    Como $f$ es continua y $\ol{A}$ es compacto, $f\left(\ol{A}\right)$ es compacto, y en particular $A\subset \ol{A}$ es acotado. Como $f(A)=f(\bb{C})$, tenemos que $f(\bb{C})$ es acotado. Por el teorema de Liouville, $f$ es constante.
\end{ejercicio}


\begin{ejercicio}
    Sea $f$ una función entera verificando que:
    \begin{equation*}
        f\left(f(z)\right)=f(z) \quad \forall z\in \bb{C}.
    \end{equation*}
    ¿Qué se puede afirmar sobre $f$?\\

    Sea $w\in f(\bb{C})\subset \bb{C}$. Entonces, existe $z\in \bb{C}$ tal que $f(z)=w$. Por tanto:
    \begin{align*}
        w &= f(z) = f\left(f(z)\right) = f(w).
    \end{align*}

    Por tanto, $f_{\big| f(\bb{C})}$ es la identidad sobre $f(\bb{C})$.
    Distinguimos casos:
    \begin{itemize}
        \item Si $f$ es constante, entonces $\exists \alpha\in \bb{C}$ tal que $f(z)=\alpha$ para todo $z\in \bb{C}$.
        \item Si $f$ no es constante, por el Teorema de Liouville $\bb{C}=\ol{f(\bb{C})}$, por lo que se tiene que $f(\bb{C})'=\bb{C}$. Consideramos ahora la función identidad en $\bb{C}$, $Id_{\bb{C}}$, y por lo demostrado anteriormente tenemos que:
        \begin{equation*}
            Id_{\bb{C}}(z) = f(z) \quad \forall z\in f(\bb{C}).
        \end{equation*}

        Como $f(\bb{C})'\cap \bb{C}=\bb{C}\neq \emptyset$, por el Principio de Identidad tenemos que $f=Id_{\bb{C}}$. Es decir:
        \begin{equation*}
            f(z)=z \quad \forall z\in \bb{C}.
        \end{equation*}
        En este caso, $f$ es la función identidad.
    \end{itemize}
    Por tanto, $f$ es constante o la función identidad.
\end{ejercicio}

\begin{ejercicio}
    En cada uno de los siguientes casos, decidir si existe una función $f$, holomorfa en un entorno del origen, y verificando que $f(\nicefrac{1}{n})=a_n$ para todo $n\in \bb{N}$ suficientemente grande:
    \begin{enumerate}
        \item $a_{2n}=0$,\ $a_{2n-1}=1$ \qquad$\forall n\in \bb{N}$.
        
        Supongamos que existe una función $f\in \cc{H}(D(0,r))$ tal que $f(\nicefrac{1}{n})=a_n$ para todo $n\in \bb{N}$ suficientemente grande. Consideramos el siguiente conjunto:
        \begin{equation*}
            A=\left\{ \frac{1}{2n} : n\in \bb{N} \right\} \cap D(0,r).
        \end{equation*}

        Veamos además que la función constantemente nula coincide con $f$ en $A$. Dado $z\in A$, tenemos que:
        \begin{align*}
            f(z) &= f\left(\frac{1}{2n}\right) = a_{2n} = 0.
        \end{align*}

        Como $A'=\{0\}$ y $A'\cap D(0,r)=\{0\}\neq \emptyset$, por el Principio de Identidad, tenemos que $f$ es la función constantemente nula en $D(0,r)$. Por tanto, $f(z)=0$ para todo $z\in D(0,r)$. Dado ahora $n\in \bb{N}$ tal que $\frac{1}{2n-1}\in D(0,r)$, tenemos que:
        \begin{align*}
            f\left(\frac{1}{2n-1}\right) &= a_{2n-1} = 1,
        \end{align*}
        No obstante, esto es una contradicción, ya que $f$ es la función constantemente nula en $D(0,r)$, y por tanto no puede tomar el valor $1$. Por tanto, la contradicción viene de suponer que existe una función $f\in \cc{H}(D(0,r))$ tal que $f(\nicefrac{1}{n})=a_n$ para todo $n\in \bb{N}$ suficientemente grande. Por tanto, no existe tal función.

        \item $a_{2n}=a_{2n-1}=\dfrac{1}{2n}$\qquad $\forall n\in \bb{N}$.
        
        Sea $f\in \cc{H}(D(0,r))$ tal que $f(\nicefrac{1}{n})=a_n$ para todo $n\in \bb{N}$ suficientemente grande. Consideramos el conjunto:
        \begin{equation*}
            A=\left\{ \frac{1}{2n} : n\in \bb{N} \right\} \cap D(0,r).
        \end{equation*}

        Vemos que $f$ coincide en $A$ con la función identidad en $\bb{C}$, ya que dado $z\in A$, tenemos que:
        \begin{align*}
            f(z) &= f\left(\frac{1}{2n}\right) = a_{2n} = \frac{1}{2n} = z.
        \end{align*}
        Como $A'=\{0\}$ y $A'\cap D(0,r)=\{0\}\neq \emptyset$, por el Principio de Identidad, tenemos que $f$ es la función identidad en $D(0,r)$. Por tanto, $f(z)=z$ para todo $z\in D(0,r)$. Dado ahora $n\in \bb{N}$ tal que $\frac{1}{2n-1}\in D(0,r)$, tenemos que:
        \begin{align*}
            f\left(\frac{1}{2n-1}\right) &= a_{2n-1} = \frac{1}{2n}\neq \frac{1}{2n-1} = f\left(\frac{1}{2n-1}\right),
        \end{align*}
        No obstante, esto es una contradicción. Por tanto, la contradicción viene de suponer que existe una función $f\in \cc{H}(D(0,r))$ tal que $f(\nicefrac{1}{n})=a_n$ para todo $n\in \bb{N}$ suficientemente grande. Por tanto, no existe tal función.
        \item $a_n=\dfrac{n}{n+1}$\qquad $\forall n\in \bb{N}$.
        
        Sea la siguiente función:
        \Func{f}{D(0,1)}{\bb{C}}{z}{\frac{1}{1+z}}

        Tenemos que $f\in \cc{H}(D(0,1))$, y además:
        \begin{align*}
            f\left(\frac{1}{n}\right) &= \frac{1}{1+\frac{1}{n}} = \frac{n}{n+1} = a_n \quad \forall n\in \bb{N}.
        \end{align*}

        Por tanto, existe una función $f\in \cc{H}(D(0,1))$ tal que $f(\nicefrac{1}{n})=a_n$ para todo $n\in \bb{N}$ suficientemente grande.
    \end{enumerate}
\end{ejercicio}

\begin{ejercicio}
    Enunciar y demostrar un resultado referente al orden de los ceros de una suma, producto o cociente de funciones holomorfas.\\

    Fijado un dominio $\Omega\subset \bb{C}$, para cada $f\in \cc{H}(\Omega)$ y cada $z_0\in \Omega$, notaremos por $\ord_{z_0}(f)$ el orden del cero de $f$ en $z_0$ (en el caso de que $z_0\in Z(f)$), o $0$ en caso contrario. Enunciamos los siguientes resultados:
    \begin{description}
        \item[Orden del producto:] Sea $f,g\in \cc{H}(\Omega)$ y $z_0\in \Omega$. Entonces:
        \begin{equation*}
            \ord_{z_0}(f\cdot g) = \ord_{z_0}(f) + \ord_{z_0}(g).
        \end{equation*}
        \begin{proof}
            Sea $f,g\in \cc{H}(\Omega)$ y $z_0\in \Omega$. Por la caracterización de los ceros de una función holomorfa, tenemos que:
            \begin{align*}
                f(z) &= (z-z_0)^{\ord_{z_0}(f)}\cdot h(z), \quad h\in \cc{H}(\Omega),\ h(z_0)\neq 0, \\
                g(z) &= (z-z_0)^{\ord_{z_0}(g)}\cdot k(z), \quad k\in \cc{H}(\Omega),\ k(z_0)\neq 0.
            \end{align*}
            Por tanto, tenemos que:
            \begin{align*}
                f(z)\cdot g(z) &= (z-z_0)^{\ord_{z_0}(f)}\cdot h(z)\cdot (z-z_0)^{\ord_{z_0}(g)}\cdot k(z) \\
                &= (z-z_0)^{\ord_{z_0}(f)+\ord_{z_0}(g)}\cdot h(z)\cdot k(z).
            \end{align*}
            Como $h,k\in \cc{H}(\Omega)$ y $h(z_0)\neq 0$, $k(z_0)\neq 0$, tenemos que $h(z_0)\cdot k(z_0)\neq 0$. Por tanto, queda demostrado lo que queríamos.
        \end{proof}

        \item[Orden de la suma:] Sea $f,g\in \cc{H}(\Omega)$ y $z_0\in \Omega$. Entonces:
        \begin{equation*}
            \ord_{z_0}(f+g) \geq \min\left\{ \ord_{z_0}(f),\ord_{z_0}(g) \right\}.
        \end{equation*}
        \begin{proof}
            Sea $f,g\in \cc{H}(\Omega)$ y $z_0\in \Omega$. Por la caracterización de los ceros de una función holomorfa, tenemos que:
            \begin{align*}
                f(z) &= (z-z_0)^{\ord_{z_0}(f)}\cdot h(z), \quad h\in \cc{H}(\Omega),\ h(z_0)\neq 0, \\
                g(z) &= (z-z_0)^{\ord_{z_0}(g)}\cdot k(z), \quad k\in \cc{H}(\Omega),\ k(z_0)\neq 0.
            \end{align*}

            Sin pérdida de generalidad, supongamos que $\ord_{z_0}(f)\leq \ord_{z_0}(g)$. Entonces, tenemos que:
            \begin{align*}
                f(z)+g(z) &= (z-z_0)^{\ord_{z_0}(f)}\cdot h(z) + (z-z_0)^{\ord_{z_0}(g)}\cdot k(z) \\
                &= (z-z_0)^{\ord_{z_0}(f)}\left[h(z) + k(z)(z-z_0)^{\ord_{z_0}(g)-\ord_{z_0}(f)}\right].
            \end{align*}

            Como $h,k\in \cc{H}(\Omega)$ entonces $h(z) + k(z)(z-z_0)^{\ord_{z_0}(g)-\ord_{z_0}(f)}$ es una función holomorfa en $\Omega$. Además, puesto que $h(z_0)\neq 0$, entonces la función $h(z) + k(z)(z-z_0)^{\ord_{z_0}(g)-\ord_{z_0}(f)}$ no se anula en $z_0$. Por tanto, queda demostrado lo que queríamos.
        \end{proof}

        \item[Orden del cociente:] Sea $f,g\in \cc{H}(\Omega)$ y $z_0\in \Omega$ tal que $g(z)\neq 0$ para todo $z\in \Omega$ (para poder definir el cociente). Entonces:
        \begin{equation*}
            \ord_{z_0}\left(\frac{f}{g}\right) = \ord_{z_0}(f)
        \end{equation*}

        \begin{proof}
            Por la caracterización de los ceros de una función holomorfa, tenemos que:
            \begin{align*}
                f(z) &= (z-z_0)^{\ord_{z_0}(f)}\cdot h(z), \quad h\in \cc{H}(\Omega),\ h(z_0)\neq 0, \\
            \end{align*}
            y como $g(z)\neq 0$ para todo $z\in \Omega$, entonces podemos escribir:
            \begin{align*}
                \left(\frac{f}{g}\right)(z) &= \frac{(z-z_0)^{\ord_{z_0}(f)}\cdot h(z)}{g(z)} \\
                &= (z-z_0)^{\ord_{z_0}(f)}\cdot \frac{h(z)}{g(z)}.
            \end{align*}
            Como $h,g\in \cc{H}(\Omega)$, entonces $\nicefrac{h}{g}\in \cc{H}(\Omega)$, y además $\nicefrac{h(z_0)}{g(z_0)}\neq 0$. Por tanto, queda demostrado lo que queríamos.
        \end{proof}
    \end{description}

\end{ejercicio}

\begin{ejercicio}
    Dado un abierto $\Omega$ del plano, probar que el anillo $\cc{H}(\Omega)$ es un dominio de integridad si, y sólo si, $\Omega$ es conexo.
\end{ejercicio}

\begin{ejercicio}
    ¿Qué se puede afirmar sobre dos funciones enteras cuya composición es constante?

    Sean $f,g\in \cc{H}(\bb{C})$ tales que $f\circ g$ es constante. Hay dos opciones:
    \begin{itemize}
        \item Si $g$ es constante (por ejemplo, $g(z)=\alpha$ para algún $\alpha\in \bb{C}$), entonces:
        \begin{equation*}
            (f\circ g)(z) = f(g(z)) = f(\alpha) = \beta \quad \forall z\in \bb{C},
        \end{equation*}
        Por tanto, la composición es constante e igual a $\beta=f(\alpha)$. No hay restricciones sobre $f$.

        \item Si $g$ no es constante, entonces por el recíproco del Teorema de Liouville, $g$ no es acotada y, de hecho, $\ol{g(\bb{C})}=\bb{C}$. Por tanto, $\ol{g(\bb{C})}'=\bb{C}$.
        
        Sea ahora $\beta\in \bb{C}$ tal que $(f\circ g)(z)=\beta$ para todo $z\in \bb{C}$. Veamos que $f$ coincide con la función constantemente $\beta$ en $g(\bb{C})$. Sea $w\in g(\bb{C})$, entonces existe $z\in \bb{C}$ tal que $g(z)=w$. Por tanto:
        \begin{align*}
            f(w) &= f(g(z)) = (f\circ g)(z) = \beta,
        \end{align*}

        Por tanto, $f$ coincide con la función constantemente $\beta$ en $g(\bb{C})$. Como $g(\bb{C})'\cap \bb{C}=\bb{C}\neq \emptyset$, por el Principio de Identidad, tenemos que $f$ es la función constantemente $\beta$ en $\bb{C}$. Es decir:
        \begin{equation*}
            f(z)=\beta \quad \forall z\in \bb{C}.
        \end{equation*}
    \end{itemize}

    Por tanto, si $f\circ g$ es constante, entonces $f$ o $g$ (o ambas) son constantes.
\end{ejercicio}

\begin{ejercicio}
    Sea $f\in \cc{H}(\bb{C})$ verificando que $f(z)\to \infty$ cuando $z\to \infty$. Probar que $f$ es una función polinómica.\\

    Fijado $N\in \bb{R}^+$, como $f(z)\to \infty$ cuando $z\to \infty$, tenemos que $\exists R\in \bb{R}^+$ tal que:
    \begin{equation*}
        |z|>R \implies |f(z)|>N>0
    \end{equation*}

    Entonces sabemos que $Z(f)$ es un conjunto acotado, puesto que:
    \begin{equation*}
        Z(f)\subset \ol{D}(0,R).
    \end{equation*}

    Supongamos que $Z(f)$ es un conjunto infinito. Como $Z(f)$ es numerable, entonces podemos enumerarlo como:
    \begin{equation*}
        Z(f)=\{z_n\}_{n\in \bb{N}}\subset \ol{D}(0,R).
    \end{equation*}

    Por el Teorema de Bolzano-Weierstrass, sabemos que $\{z_n\}_{n\in \bb{N}}$ admite una parcial convergente, es decir, existe una sucesión $\{z_{n_k}\}_{k\in \bb{N}}$ tal que:
    \begin{equation*}
        \left\{z_{n_k}\right\}\to z_0\in \ol{D}(0,R)
    \end{equation*}
    Por tanto, $Z(f)'\neq \emptyset$. Por tanto:
    \begin{equation*}
        Z(f)'\cap \bb{C}\neq \emptyset
    \end{equation*}

    Por tanto, tenemos que $f$ es idénticamente nula en $\bb{C}$. No obstante, esto contradice que $f(z)\to \infty$ cuando $z\to \infty$. Por tanto, $Z(f)$ no puede ser un conjunto infinito, por lo que:
    \begin{equation*}
        Z(f)=\{z_1,\ldots,z_n\} \quad n\in \bb{N}
    \end{equation*}

    Notemos por $m_i$ al orden de $z_i\in Z(f)$, para cada $i\in \{1,\ldots,n\}$. Entonces, podemos escribir:
    \begin{equation*}
        f(z) = \prod_{i=1}^{n} (z-z_i)^{m_i}\cdot g(z),\qquad g\in \cc{H}(\bb{C}),\ g(z)\neq 0 \quad \forall z\in \bb{C}.
    \end{equation*}

    Para $z\in \bb{C}$ tal que $|z|>R$, tenemos que:
    \begin{align*}
        \left|\dfrac{1}{g(z)}\right| &= \left|\dfrac{\prod\limits_{i=1}^{n} (z-z_i)^{m_i}}{f(z)}\right|
        \leq \frac{\prod\limits_{i=1}^{n} \left(|z|+|z_i|\right)^{m_i}}{|f(z)|}
        \leq \frac{\prod\limits_{i=1}^{n} \left(|z|+R\right)^{m_i}}{|f(z)|}
    \end{align*}

    Definiendo $m=\sum\limits_{i=1}^{n} m_i$, tenemos que:
    \begin{align*}
        \left|\dfrac{1}{g(z)}\right| &\leq \frac{\left(|z|+R\right)^{m}}{|f(z)|}
    \end{align*}

    Además, puesto que $|z|>R$, entonces $|f(z)|>N$, por lo que:
    \begin{align*}
        \left|\dfrac{1}{g(z)}\right| &\leq \frac{\left(|z|+R\right)^{m}}{N}
    \end{align*}

    Como buscamos emplear el Ejercicio~\ref{ej:9.2}, realizamos la siguiente acotación:
    \begin{align*}
        \left|\dfrac{1}{g(z)}\right| &\leq \frac{\left(|z|+R\right)^{m}}{N}\leq \frac{\left(|z|+|z|\right)^{m}}{N}\leq \dfrac{2^m}{N}\cdot |z|^m
    \end{align*}

    Por tanto, $\nicefrac{1}{g}$ es polinómica. Como además no se anula en ningún punto de $\bb{C}$, tenemos que es constante; es decir, $\exists \alpha\in \bb{C}$ tal que:
    \begin{equation*}
        \dfrac{1}{g(z)}=\alpha \quad \forall z\in \bb{C}.
    \end{equation*}

    Por tanto, $g(z)=\nicefrac{1}{\alpha}$ es una constante. Por tanto:
    \begin{equation*}
        f(z) = \dfrac{1}{\alpha} \prod_{i=1}^{n} (z-z_i)^{m_i}
    \end{equation*}

    Por tanto, $f$ es un polinomio de grado $m=\sum\limits_{i=1}^{n} m_i$.
\end{ejercicio}

\begin{ejercicio}
    Sea $f$ una función entera verificando que:
    \begin{equation*}
        z\in \bb{C}, |z|=1 \implies |f(z)|=1.
    \end{equation*}
    Probar que existen $\alpha\in \bb{C}$ con $|\alpha|=1$ y $n\in \bb{N}\cup\{0\}$ tales que:
    \begin{equation*}
        f(z)=\alpha z^n \quad \forall z\in \bb{C}.
    \end{equation*}

    Hacemos uso de que:
    \begin{equation*}
        z = \dfrac{1}{\ol{z}}\qquad \forall z\in \bb{T}
    \end{equation*}

    Definimos la función siguiente:
    \Func{g}{\bb{C}^*}{\bb{C}}{z}{f(z)\ol{f\left(\dfrac{1}{\ol{z}}\right)}}

    Por el Ejercicio~\ref{ej:3.6}, sabemos que $g\in \cc{H}(\bb{C}^*)$. Para cada $z\in \bb{T}$, tenemos que:
    \begin{equation*}
        g(z) = f(z)\ol{f\left(\dfrac{1}{\ol{z}}\right)} = f(z)\ol{f(z)} = |f(z)|^2 = 1.
    \end{equation*}

    Por tanto, $g\in \cc{H}(\bb{C}^*)$ cumple que $g(z)=1$ para todo $z\in \bb{T}$. Por el Principio de Identidad, como $\bb{T}$ no es numerable, $g(z)=1$ para todo $z\in \bb{C}^*$. Por ser $\bb{C}$ un dominio de integridad, tenemos que $f(z)\neq 0$ para todo $z\in \bb{C}^*$. Por tanto, podemos distinguir casos:
    \begin{itemize}
        \item Si $Z(f)=\emptyset$, buscamos ver si $f$ es acotada o no. Como $f$ es entera, tenemos que $f\left(\ol{D}(0,r)\right)$ es compacto para todo $r\in \bb{R}^+$, luego es acotado. Para ver si $f$ es acotada, hemos de considerar entonces tan solo el límite en el infinito.
        \begin{align*}
            \lim_{z\to \infty} f(z) &= \lim_{z\to \infty} \dfrac{1}{\ol{f\left(\dfrac{1}{\ol{z}}\right)}} = \dfrac{1}{\ol{f(0)}}
        \end{align*}

        Por tanto, tenemos que $f$ es acotada, y por el Teorema de Liouville, $f$ es constante. Por tanto, $\exists \alpha\in \bb{C}$ tal que $f(z)=\alpha$ para todo $z\in \bb{C}$. Consideramos por tanto $\alpha=f(0)$ y $n=0$.
        
        \item Si $Z(f)=\{0\}$, entonces:
        \begin{align*}
            f(z) = z\cdot h(z),\qquad h\in \cc{H}(\bb{C}),\ h(z)\neq 0 \quad \forall z\in \bb{C}.
        \end{align*}

        Dado $z\in \bb{T}$, tenemos que:
        \begin{align*}
            \left|h(z)\right| &= \left| \dfrac{f(z)}{z} \right| = 1
        \end{align*}

        Además, $Z(h)=\emptyset$. Estamos por tanto en el caso anterior, por lo que $h$ es constante. Consideramos por tanto $\alpha=h(0)$ y $n=1$.
    \end{itemize}
\end{ejercicio}

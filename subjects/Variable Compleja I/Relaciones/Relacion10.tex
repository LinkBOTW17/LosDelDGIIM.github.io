\section{Teorema de Morera y sus consecuencias}

\begin{ejercicio}
    Sea $\{f_n\}$ una sucesión de funciones holomorfas en un abierto $\Omega$ del plano complejo y sea $f:\Omega\to\bb{C}$ una función continua. Probar que las siguientes afirmaciones son equivalentes:
    \begin{enumerate}
        \item $\{f_n\}$ converge uniformemente a $f$ en cada subconjunto compacto de $\Omega$.
        \item Para toda sucesión $\{z_n\}$ de puntos de $\Omega$ que converge a un punto $z\in\Omega$, se tiene que $\{f_n(z_n)\}\to f(z)$.
    \end{enumerate}
\end{ejercicio}

\begin{ejercicio}
    Sea $\{f_n\}$ la sucesión de funciones enteras definida para cada $n\in\bb{N}$ por:
    \Func{f_n}{\bb{C}}{\bb{C}}{z}{z\exp\left(-\dfrac{n^2 z^2}{2}\right)}
    Probar que $\{f_n\}$ converge uniformemente en $\bb{R}$ pero no converge uniformemente en ningún entorno del origen.
\end{ejercicio}

\begin{ejercicio}
    Probar que la serie de funciones $\sum\limits_{n\geq 1} \frac{z^n}{1-z^n}$ converge en $D(0,1)$ y que su suma es una función holomorfa en $D(0,1)$.
\end{ejercicio}

\begin{ejercicio}
    Sea $f\in \cc{H}(D(0,1))$ tal que $f(0)=0$. Probar que la serie $\sum\limits_{n\geq 1} f\left(\frac{z^n}{n}\right)$ converge en $D(0,1)$ y que su suma es una función holomorfa en $D(0,1)$.
\end{ejercicio}

\begin{ejercicio}
    Probar que la sucesión de funciones enteras definida para cada $n\in\bb{N}$ por:
    \Func{f_n}{\bb{C}}{\bb{C}}{z}{\dfrac{1}{n}\sen(nz)}
    converge uniformemente en $\bb{R}$ pero no converge uniformemente en ningún subconjunto de $\bb{C}$ que tenga interior no vacío.
\end{ejercicio}

\begin{ejercicio}
    Probar que la serie de funciones $\sum_{n\geq 0} \frac{\sen(nz)}{3^n}$ converge en la banda $\Omega = \{z\in\bb{C} : |\Im z| < \log 3\}$ y que su suma es una función $f\in \cc{H}(\Omega)$. Calcular $f'(0)$.
\end{ejercicio}

\begin{ejercicio}
    Sea $\varphi : [0,1]\to\bb{C}$ una función continua. Probar que definiendo:
    \Func{f}{\bb{C}}{\bb{C}}{z}{\int_0^1 \varphi(t)e^{it z} dt}
    se obtiene una función entera y calcular el desarrollo en serie de Taylor de $f$ centrado en el origen.
\end{ejercicio}

\begin{ejercicio}
    Para cada $n\in\bb{N}$, se considera la función:
    \Func{f_n}{\bb{C}}{\bb{C}}{z}{\int_0^n \sqrt{t} e^{-tz} dt}
    \begin{enumerate}
        \item Probar que $f_n$ es una función entera y calcular su desarrollo en serie de Taylor centrado en el origen.
        \item Estudiar la convergencia de la sucesión $\{f_n\}$ en $\Omega = \{z\in\bb{C} : \Re z > 0\}$.
        \item Deducir que $f\in \cc{H}(\Omega)$, donde $f$ es la función definida por:
        \Func{f}{\Omega}{\bb{C}}{z}{\int_0^{+\infty} \sqrt{t} e^{-tz} dt}
    \end{enumerate}
\end{ejercicio}
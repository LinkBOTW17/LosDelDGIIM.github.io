\section{Teorema de Morera y sus consecuencias}

\begin{ejercicio}
    Sea $\{f_n\}$ una sucesión de funciones holomorfas en un abierto $\Omega$ del plano complejo y sea $f:\Omega\to\bb{C}$ una función continua. Probar que las siguientes afirmaciones son equivalentes:
    \begin{enumerate}
        \item $\{f_n\}$ converge uniformemente a $f$ en cada subconjunto compacto de $\Omega$.
        \item Para toda sucesión $\{z_n\}$ de puntos de $\Omega$ que converge a un punto $z\in\Omega$, se tiene que $\{f_n(z_n)\}\to f(z)$.
    \end{enumerate}
\end{ejercicio}

\begin{ejercicio}
    Sea $\{f_n\}$ la sucesión de funciones enteras definida para cada $n\in\bb{N}$ por:
    \Func{f_n}{\bb{C}}{\bb{C}}{z}{z\exp\left(-\dfrac{n^2 z^2}{2}\right)}
    Probar que $\{f_n\}$ converge uniformemente en $\bb{R}$ pero no converge uniformemente en ningún entorno del origen.
\end{ejercicio}

\begin{ejercicio}
    Probar que la serie de funciones $\sum\limits_{n\geq 1} \frac{z^n}{1-z^n}$ converge en $D(0,1)$ y que su suma es una función holomorfa en $D(0,1)$.\\

    Sea $K\subset D(0,1)$ un compacto. Entonces, como la función módulo es continua, se tiene que:
    \[
    \exists r\in \bb{R}\ \text{tal que}\ r=\max\{|z| : z\in K\}<1.
    \]

    Por tanto, para todo $z\in K$, se tiene que:
    \[
    \left|\dfrac{z^n}{1-z^n}\right| \leq \dfrac{r^n}{\left|1-|z|^n\right|} = \dfrac{r^n}{1-|z|^n} \leq \dfrac{r^n}{1-r^n}\qquad \forall n\in \bb{N}\cup \{0\}
    \]

    Para ver si la serie cuyo término general es la cota es convergente, empleamos el criterio del cociente para series de términos positivos:
    \begin{equation*}
        \lim_{n\to\infty} \dfrac{\dfrac{r^{n+1}}{1-r^{n+1}}}{\dfrac{r^n}{1-r^n}}
        = \lim_{n\to\infty} r \cdot \dfrac{1-r^n}{1-r^{n+1}} = r\cdot \lim_{n\to\infty} \dfrac{1-r^n}{1-r^{n+1}} \AstIg r\cdot 1 = r < 1.
    \end{equation*}
    donde, en $(\ast)$, hemos usado que $r<1$ y, por tanto, $\{r^n\}\to 1$. Por tanto, la serie cuyo término general es la cota es convergente, y por tanto, la serie dada converge uniformemente en $K$. Por el Test de Weierstrass, sabemos que la serie converge absoluta y uniformemente en $K$.\\

    Como la serie converge uniformemente en compactos de $D(0,1)$, por el Teorema de Convergencia de Weierstrass, la suma de la serie es una función holomorfa en $D(0,1)$. 
\end{ejercicio}

\begin{ejercicio}
    Sea $f\in \cc{H}(D(0,1))$ tal que $f(0)=0$. Probar que la serie $\sum\limits_{n\geq 1} f\left(z^n\right)$ converge en $D(0,1)$ y que su suma es una función holomorfa en $D(0,1)$.\\

    Para cada $n\in\bb{N}$, definimos la función:
    \Func{f_n}{D(0,1)}{\bb{C}}{z}{f\left(z^n\right)}.

    Como $z\mapsto z^n$ es entera, y $f\in \cc{H}(D(0,1))$, se tiene que $f_n\in \cc{H}(D(0,1))$ para todo $n\in\bb{N}$. Ahora es necesario ver que $\{f_n\}$ converge uniformemente sobre compactos de $D(0,1)$. Previamente, puesto que $f(0)=0$, $0\in Z(f)$. Sea $m$ el orden de $0$ como raíz de $f$, por lo que:
    \begin{align*}
        f(z) &= z^m\cdot g(z)\qquad \text{con } g\in \cc{H}(D(0,1)),\ g(0)\neq 0\\
        f_n(z) &= f\left(z^n\right) = z^{mn}\cdot g\left(z^n\right).
    \end{align*}

    Sea ahora $K\subset D(0,1)$ un compacto. Entonces:
    \begin{equation*}
        \left|f_n(z)\right| = |z|^{mn}\cdot |g(z^n)|
    \end{equation*}

    Como $K$ es compacto y la función módulo es continua, existe $r\in\bb{R}$ tal que:
    \[
    r = \max\{|z| : z\in K\} < 1.
    \]

    Como la función módulo es continua, $g$ también, y $z \mapsto z^n$ también, existe $M\in\bb{R}$ tal que:
    \[
    M = \max\{|g(z^n)| : z\in K\}
    \]

    Por tanto, para todo $z\in K$, $n\in\bb{N}$, se tiene que:
    \[
    \left|f_n(z)\right| \leq r^{mn} M.
    \]

    Para ver si la serie cuyo término general es la cota es convergente, como esta se trata de una serie geométrica de razón $r^m<1$, entonces converge. Por tanto, por el Test de Weierstrass, la serie $\sum\limits_{n\geq 1} f_n$ converge uniformemente en $K$.\\

    Como la serie converge uniformemente en compactos de $D(0,1)$, por el Teorema de Convergencia de Weierstrass, la suma de la serie es una función holomorfa en $D(0,1)$.
\end{ejercicio}

\begin{ejercicio}
    Probar que la sucesión de funciones enteras definida para cada $n\in\bb{N}$ por:
    \Func{f_n}{\bb{C}}{\bb{C}}{z}{\dfrac{1}{n}\sen(nz)}
    converge uniformemente en $\bb{R}$ pero no converge uniformemente en ningún subconjunto de $\bb{C}$ que tenga interior no vacío.
\end{ejercicio}

\begin{ejercicio}
    Probar que la serie de funciones $\sum\limits_{n\geq 0} \frac{\sen(nz)}{3^n}$ converge en la banda $\Omega = \{z\in\bb{C} : |\Im z| < \log 3\}$ y que su suma es una función $f\in \cc{H}(\Omega)$. Calcular $f'(0)$.\\

    Sea $K\subset \Omega$ un compacto. Para todo $z\in K$, $n\in\bb{N}$, se tiene que:
    \begin{align*}
        \left|\dfrac{\sen(nz)}{3^n}\right| &= \left|\dfrac{e^{inz} - e^{-inz}}{2i\cdot 3^n}\right|\leq \dfrac{1}{2\cdot 3^n} \left(|e^{inz}| + |e^{-inz}|\right)
        = \dfrac{e^{-n\Im z} + e^{n\Im z}}{2\cdot 3^n}
        \leq \dfrac{e^{n|\Im z|}}{3^n}
        = \left(\dfrac{e^{|\Im z|}}{3}\right)^n.
    \end{align*}

    Como $K$ es compacto y la función módulo y parte imaginaria son continuas, existe $M\in\bb{R}$ tal que:
    \[
    M = \max\{|\Im z| : z\in K\} < \log 3.
    \]

    Por tanto, para todo $z\in K$, se tiene que:
    \[
    \left|\dfrac{\sen(nz)}{3^n}\right| \leq \left(\dfrac{e^M}{3}\right)^n
    \]

    Como $M<\log 3$, se tiene que $e^M < 3$, por lo que la cota se trata de una serie geométrica de razón menor que $1$. Por tanto, la serie cuyo término general es la cota es convergente. Por el Test de Weierstrass, la serie dada converge uniformemente en $K$.\\

    Como la serie converge uniformemente en compactos de $\Omega$, por el Teorema de Convergencia de Weierstrass, la suma de la serie es una función holomorfa en $\Omega$.\\

    Para calcular $f'(0)$, hacemos uso de que:
    \begin{equation*}
        f'(z) = \sum_{n\geq 0} \left(\dfrac{\sen(nz)}{3^n}\right)' = \sum_{n\geq 0} \dfrac{n\cos(nz)}{3^n}.
    \end{equation*}

    Evaluando en $z=0$, se tiene que:
    \begin{equation*}
        f'(0) = \sum_{n\geq 0} \dfrac{n\cos(0)}{3^n} = \sum_{n\geq 0} \dfrac{n}{3^n}.
    \end{equation*}

    Para calcular esta suma, usamos el hecho de que:
    \begin{equation*}
        \sum_{n\geq 0} x^n = \dfrac{1}{1-x}\qquad \forall x\in\bb{R},\ |x|<1.
    \end{equation*}
    Derivando ambos lados, se tiene que:
    \begin{equation*}
        \sum_{n\geq 1} n x^{n-1} = \dfrac{1}{(1-x)^2}\qquad \forall x\in\bb{R},\ |x|<1.
    \end{equation*}
    Multiplicando ambos lados por $x$, se obtiene:
    \begin{equation*}
        \sum_{n\geq 1} n x^n = \dfrac{x}{(1-x)^2}\qquad \forall x\in\bb{R},\ |x|<1.
    \end{equation*}

    Evaluando en $x=\frac{1}{3}$, se tiene que:
    \begin{equation*}
        \sum_{n\geq 0}\dfrac{n}{3^n} = \sum_{n\geq 1} n \left(\dfrac{1}{3}\right)^n = \dfrac{\frac{1}{3}}{\left(1-\frac{1}{3}\right)^2} = \dfrac{\frac{1}{3}}{\left(\frac{2}{3}\right)^2} = \dfrac{\nicefrac{1}{3}}{\nicefrac{4}{9}} = \dfrac{3}{4}.
    \end{equation*}

    Por tanto, se tiene que $f'(0) = \nicefrac{3}{4}$.
\end{ejercicio}

\begin{ejercicio}
    Sea $\varphi : [0,1]\to\bb{C}$ una función continua. Probar que definiendo:
    \Func{f}{\bb{C}}{\bb{C}}{z}{\int_0^1 \varphi(t)e^{it z} dt}
    se obtiene una función entera y calcular el desarrollo en serie de Taylor de $f$ centrado en el origen.\\

    Consideramos el camino $\gamma=[0,1]$ y el abierto $\Omega=\bb{C}$. Definimos la siguiente función:
    \Func{\Phi}{[0,1]\times \bb{C}}{\bb{C}}{(t,z)}{\varphi(t)e^{itz}}

    En primer lugar, tenemos que $\varphi$ es continua por ser producto de funciones continuas. Además, fijado $t\in[0,1]$, la función $z\mapsto \Phi(t,z)$ es holomorfa en $\bb{C}$ por ser el producto de una constante por la exponencial. Por el Teorema de Holomorfía de la Integral dependiente de un parámetro, se tiene que $f\in \cc{H}(\bb{C})$. Calculamos ahora el desarrollo de Taylor de $f$ centrado en el origen.
    \begin{align*}
        f(z) &= \sum_{n\geq 0} \dfrac{f^{(n)}(0)}{n!} z^n\qquad \forall z\in\bb{C}
    \end{align*}

    Para cada $n\in\bb{N}$, calculamos $f^{(n)}(0)$:
    \begin{align*}
        f^{(n)}(0) &= \int_0^1 \left(\dfrac{\partial^{n}}{\partial z^n} \varphi(t)e^{itz}\right)_{z=0} dt
        = \int_0^1 \varphi(t) \left(it\right)^n e^{it\cdot 0} dt
        = \int_0^1 \varphi(t) (it)^n dt\\
        &= i^n \int_0^1 t^n \varphi(t) dt.
    \end{align*}

    Por tanto, el desarrollo de Taylor de $f$ centrado en el origen es:
    \begin{align*}
        f(z) &= \sum_{n\geq 0} \dfrac{i^n}{n!} \left(\int_0^1 t^n \varphi(t) dt\right) z^n\qquad \forall z\in\bb{C}\\
    \end{align*}
\end{ejercicio}

\begin{ejercicio}
    Para cada $n\in\bb{N}$, se considera la función:
    \Func{f_n}{\bb{C}}{\bb{C}}{z}{\int_0^n \sqrt{t} e^{-tz} dt}
    \begin{enumerate}
        \item Probar que $f_n$ es una función entera y calcular su desarrollo en serie de Taylor centrado en el origen.\\
        
        Consideramos el camino $\gamma = [0,n]$ y el abierto $\Omega = \bb{C}$. Definimos la siguiente función:
        \Func{\Phi}{[0,n]\times \bb{C}}{\bb{C}}{(t,z)}{\sqrt{t} e^{-tz}}

        En primer lugar, tenemos que $\Phi$ es continua por ser producto de funciones continuas. Además, fijado $t\in[0,n]$, la función $z\mapsto \Phi(t,z)$ es holomorfa en $\bb{C}$ por ser el producto de una constante por la exponencial. Por el Teorema de Holomorfía de la Integral dependiente de un parámetro, se tiene que $f_n\in \cc{H}(\bb{C})$. Calculamos ahora el desarrollo de Taylor de $f_n$ centrado en el origen.
        \begin{align*}
            f_n(z) &= \sum_{k\geq 0} \dfrac{f_n^{(k)}(0)}{k!} z^k\qquad \forall z\in\bb{C}
        \end{align*}
        
        Para cada $k\in\bb{N}$, calculamos $f_n^{(k)}(0)$:
        \begin{align*}
            f_n^{(k)}(0) &= \int_0^n \left(\dfrac{\partial^{k}}{\partial z^k} \sqrt{t} e^{-tz}\right)_{z=0} dt
            = \int_0^n \sqrt{t} (-t)^k e^{-t\cdot 0} dt
            = (-1)^k \int_0^n t^{k+\nicefrac{1}{2}} dt\\
            &= (-1)^k \cdot \dfrac{n^{k+\nicefrac{3}{2}}}{k+\nicefrac{3}{2}}.
        \end{align*}

        Por tanto, el desarrollo de Taylor de $f_n$ centrado en el origen es:
        \begin{align*}
            f_n(z) &= \sum_{k\geq 0} \dfrac{(-1)^k}{k!} \cdot \dfrac{n^{k+\nicefrac{3}{2}}}{k+\nicefrac{3}{2}} \cdot z^k\qquad \forall z\in\bb{C}.
        \end{align*}
        

        \item Estudiar la convergencia de la sucesión $\{f_n\}$ en $\Omega = \{z\in\bb{C} : \Re z > 0\}$.
        
        Estudiar la convergencia de una sucesión de funciones no es directo, y es por ello que la convertiremos en la suma de una serie de funciones. Para cada $k\in \bb{N}$, $k\leq n$, definimos la función:
        \Func{g_k}{\bb{C}}{\bb{C}}{z}{\int_{k-1}^k \sqrt{t} e^{-tz} dt}

        Entonces, se tiene que:
        \begin{align*}
            f_n(z) &= \sum_{k=1}^n g_k(z)
        \end{align*}

        Definiendo ahora $g_k=0$ para todo $k>n$ y para $k=0$, se tiene que:
        \begin{align*}
            f_n(z) &= \sum_{k\geq 0} g_k(z).
        \end{align*}

        Estamos por tanto en las condiciones de aplicar el Test de Weierstrass. Dado $K\subset \Omega$ un compacto, puesto que la función parte real es continua, existe $r\in\bb{R}$ tal que:
        \[
        r= \min\{\Re z : z\in K\} > 0.
        \]

        Por tanto, para todo $z\in K$ y $k\in\bb{N}$, se tiene que:
        \begin{align*}
            \left|g_k(z)\right| &\leq \left|\int_{k-1}^k \sqrt{t} e^{-tz} dt\right| \leq \sup\left\{\left|\sqrt{t} e^{-tz}\right| : t\in [k-1,k]\right\}
            =\\&= \sqrt{k} \cdot \sup\left\{e^{-t\Re z} : t\in [k-1,k]\right\}
            \leq \sqrt{k} \cdot e^{-(k-1)\Re z}
            \leq \sqrt{k} \cdot e^{-(k-1)r}.
        \end{align*}

        Para ver si la serie cuyo término general es la cota es convergente, empleamos el criterio del cociente para series de términos positivos:
        \begin{equation*}
            \lim_{k\to\infty} \dfrac{\sqrt{k+1} e^{-kr}}{\sqrt{k} e^{-(k-1)r}} = \lim_{k\to\infty} \dfrac{\sqrt{k+1}}{\sqrt{k}} e^{-r} = 1\cdot e^{-r}
        \end{equation*}

        Puesto que $r>0$, se tiene que $e^{-r}<1$. Por tanto, la serie cuyo término general es la cota es convergente, y por tanto, la serie $\sum\limits_{k\geq 0} g_k=f_n$ converge uniformemente en $K$.


        \item Deducir que $f\in \cc{H}(\Omega)$, donde $f$ es la función definida por:
        \Func{f}{\Omega}{\bb{C}}{z}{\int_0^{+\infty} \sqrt{t} e^{-tz} dt}

        Calculemos el límite de la sucesión $\{f_n\}$ en $\Omega$. Para todo $z\in\Omega$, se tiene que:
        \begin{align*}
            \lim_{n\to\infty} f_n(z) &= \lim_{n\to\infty} \int_0^n \sqrt{t} e^{-tz} dt
            = \int_0^{+\infty} \sqrt{t} e^{-tz} dt
            = f(z).
        \end{align*}

        Por tanto, la sucesión $\{f_n\}$ converge uniformemente a $f$ en cada compacto de $\Omega$. Por el Teorema de Convergencia de Weierstrass, se tiene que $f\in \cc{H}(\Omega)$.
    \end{enumerate}
\end{ejercicio}

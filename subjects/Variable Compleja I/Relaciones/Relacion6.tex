\section{Integral Curvilínea}

\begin{ejercicio}
    Sean $\alpha \in \bb{C}$ y $r \in \bb{R}^+$. Probar que
    \[
        \int_{[\alpha,\alpha+r]} f(z)dz = \int_{0}^{r} f(\alpha+s)ds
    \]
    para cualquier función $f \in C\left([\alpha,\alpha+r]^*\right)$.
    ¿Cual es la igualdad análoga para un segmento vertical?\\

    El segmento horizontal $[\alpha,\alpha+r]$ viene dado por el siguiente arco:
    \Func{\sigma}{[0,1]}{\bb{C}}{t}{(1-t)\alpha+t(\alpha+r) = \alpha + tr}

    Por tanto, por la definición de integral sobre un arco, tenemos que:
    \[
        \int_{[\alpha,\alpha+r]} f(z)dz = \int_{0}^{1} f(\sigma(t))\sigma'(t)dt = \int_{0}^{1} f(\alpha + tr)r dt
    \]

    Aplicamos ahora el siguiente cambio de variable:
    \begin{equation*}
        \MetInt{s=rt}{ds=rdt}\Longrightarrow \begin{cases}
            s=0 & \text{si } t=0 \\
            s=r & \text{si } t=1
        \end{cases}
    \end{equation*}

    Por tanto, la integral queda:
    \begin{equation*}
        \int_{[\alpha,\alpha+r]} f(z)dz = \int_{0}^{r} f(\alpha+s)ds
    \end{equation*}~\\

    El segmento vertical $[\alpha,\alpha+ir]$ viene dado por el siguiente arco:
    \Func{\sigma}{[0,1]}{\bb{C}}{t}{(1-t)\alpha+t(\alpha+ir) = \alpha + it\cdot r}

    Por tanto, por la definición de integral sobre un arco, tenemos que:
    \[
        \int_{[\alpha,\alpha+ir]} f(z)dz = \int_{0}^{1} f(\sigma(t))\sigma'(t)dt = \int_{0}^{1} f(\alpha + it\cdot r)ir dt
    \]

    De nuevo, aplicamos ahora el mismo cambio de variable. Notemos que no es válido $s=itr$, puesto que la función de cambio de variable debe ser una función real de variable real. Por tanto, usando el mismo cambio de variable que antes, tenemos que:
    \begin{equation*}
        \int_{[\alpha,\alpha+ir]} f(z)dz = \int_{0}^{r} f(\alpha+is)ids
        = i\int_{0}^{r} f(\alpha+is)ds
    \end{equation*}
\end{ejercicio}

\begin{ejercicio}
    Para $r \in \left]1,+\infty\right[$ se define
    \[
        I(r) = \int_{\gamma_r} \frac{z}{z^3+1}dz
        \quad \text{y} \quad
        J(r) = \int_{\sigma_r} \frac{z^2 e^z}{z+1}dz
    \]
    donde $\gamma_r = C(0,r)$ y $\sigma_r = [-r,-r+i]$. Probar que
    \[
        \lim_{r\to+\infty} I(r) = 0
        \quad \text{y} \quad
        \lim_{r\to+\infty} J(r) = 0.
    \]

    Trabajamos en primer lugar con la integral $I(r)$.
    \begin{align*}
        I(r) &= \int_{\gamma_r} \frac{z}{z^3+1}dz
        = \int_{-\pi}^{\pi} \frac{re^{it}}{(re^{it})^3+1}\cdot ire^{it}dt
        = i\int_{-\pi}^{\pi} \frac{(re^{it})^2}{(re^{it})^3+1} dt \\
    \end{align*}

    Buscamos acotar ahora dicha integral:
    \begin{align*}
        |I(r)| &\leq \int_{-\pi}^{\pi} \left|\frac{(re^{it})^2}{(re^{it})^3+1}\right|dt
        \leq \int_{-\pi}^{\pi} \frac{|re^{it}|^2}{\left||re^{it}|^3-|1|\right|}dt 
        \AstIg \int_{-\pi}^{\pi} \frac{r^2}{|r^3-1|}dt
    \end{align*}
    donde en $(\ast)$ se ha empleado que $|e^{it}|=1$ para todo $t \in \bb{R}$. Como $r>1$ entonces $r^3>1$, por lo que:
    \begin{align*}
        |I(r)| &\leq \int_{-\pi}^{\pi} \frac{r^2}{r^3-1}dt = 2\pi\cdot \frac{r^2}{r^3-1}
    \end{align*}

    Por el Lema del Sándwhich tomando límite cuando $r \to +\infty$, deducimos que:
    \begin{equation*}
        \lim_{r\to+\infty} |I(r)| = 0
        \Longrightarrow \lim_{r\to+\infty} I(r) = 0
    \end{equation*}

    Trabajaremos de forma análoga con la integral $J(r)$:
    \begin{align*}
        J(r) &= \int_{\sigma_r} \frac{z^2 e^z}{z+1}dz
        = \int_{0}^{1} \dfrac{(-r+ti)^2e^{-r+ti}}{-r+ti+1}i dt
        = ie^{-r}\int_{0}^{1} \frac{(-r+ti)^2}{-r+ti+1}e^{it}dt
    \end{align*}

    Podríamos ahora intentar resolver dicha integral, algo que no es directo. Para acotar la integral, definimos la siguiente función:
    \Func{\varphi}{[0,1]}{\bb{C}}{t}{\frac{(-r+ti)^2}{-r+ti+1}}

    Como $r>1$, la parte real del denominador no se anula, y por tanto la función $\varphi$ es continua por ser racional. Como $[0,1]$ es compacto, $\varphi\left(\left[0,1\right]\right)$ es compacto, y por tanto acotado. Por tanto, $\exists M\in \bb{R}^+$ tal que:
    \begin{equation*}
        \left|\varphi(t)\right| \leq M\qquad \forall t \in [0,1]
    \end{equation*}

    Por tanto, tenemos que:
    \begin{align*}
        |J(r)| &= e^{-r}\int_{0}^{1} \left|\frac{(-r+ti)^2}{-r+ti+1}\right|dt
        = e^{-r}\int_{0}^{1} \left|\varphi(t)\right|dt
        \leq e^{-r}\int_{0}^{1} M dt
        = e^{-r}M
    \end{align*}

    Por tanto, por el Lema del Sándwhich, deducimos que:
    \begin{equation*}
        \lim_{r\to+\infty} |J(r)| = 0
        \Longrightarrow \lim_{r\to+\infty} J(r) = 0
    \end{equation*}
\end{ejercicio}

\begin{ejercicio}
    Para todo $r \in \left]0,1\right[$, probar que
    \[
        \int_{C(0,r)} \frac{\log(1+z)}{z}dz = 0
    \]
    y deducir que
    \[
        \int_{0}^{\pi} \log(1+r^2 + 2r\cos t)dt = 0.
    \]

    \begin{description}
        \item[Opción Rutinaria:]~
        
        Usando el arco de la circunferencia $C(0,r)$, tenemos que:
        \begin{align*}
            \int_{C(0,r)} \frac{\log(1+z)}{z}dz &= \int_{-\pi}^{\pi} \frac{\log(1+re^{it})}{re^{it}}ire^{it}dt
            = i\int_{-\pi}^{\pi} \log(1+re^{it})\ dt
            =\\&= i\left(\int_{-\pi}^{\pi} \ln|1+re^{it}|\ dt + i\int_{-\pi}^{\pi} \arg(1+re^{it})\ dt\right)
            =\\&= -\int_{-\pi}^{\pi} \arg(1+re^{it})\ dt + i\int_{-\pi}^{\pi} \ln|1+re^{it}|\ dt
        \end{align*}

        Como vemos, calcular dicha integral no es sencillo, por lo que descartamos esta opción.

        \item[Desarrollo en Serie:]~
        
        Por el Ejercicio~\ref{ej:serie_log_1masZ}, sabemos que:
        \begin{equation*}
            \log(1+z) = \sum\limits_{n= 1}^\infty \dfrac{(-1)^{n+1}}{n}z^n\qquad \forall z \in D(0,1)
        \end{equation*}

        Definimos ahora la función:
        \Func{f}{D(0,1)\setminus \{0\}}{\bb{C}}{z}{\dfrac{\log(1+z)}{z}}

        Por dicho desarrollo, tenemos que:
        \begin{align*}
            f(z) &= \frac{\log(1+z)}{z} = \sum\limits_{n= 1}^\infty \dfrac{(-1)^{n+1}}{n}z^{n-1}\\
            &= \sum\limits_{n=0}^\infty \dfrac{(-1)^{n}}{n+1}z^{n}\qquad \forall z \in D(0,1)\setminus \{0\}
        \end{align*}

        Definimos ahora la siguiente función:
        \Func{F}{D(0,1)\setminus \{0\}}{\bb{C}}{z}{\sum\limits_{n=0}^\infty \dfrac{(-1)^{n}}{(n+1)^2}z^{n+1}}

        Por el Teorema de Holomorfía de las funciones dadas como suma de serie de potencias, $F\in \cc{H}(D(0,1)\setminus \{0\})$, con:
        \begin{align*}
            F'(z) &= f(z)\qquad \forall z \in D(0,1)\setminus \{0\}
        \end{align*}

        Por tanto, $F$ es una primitiva de $f$. Como $C(0,r)$ es un camino cerrado en $D(0,1)\setminus \{0\}$ ($0<r<1$), por el Teorema de Caracterización de existencia de primitivas, tenemos que:
        \begin{equation*}
            \int_{C(0,r)} f(z)dz = 0
        \end{equation*}

        \item[Opción Mezclada:]~
        
        Como vimos en la primera opción, tenemos que:
        \begin{align*}
            \int_{C(0,r)} \frac{\log(1+z)}{z}dz &= \int_{-\pi}^{\pi} i\log(1+re^{it})\ dt
        \end{align*}

        Como $r\in \left]0,1\right[$ y $e^{it} \in \bb{T}$ para todo $t \in \bb{R}$, tenemos que $re^{it} \in D(0,1)$ para todo $t \in \bb{R}$, por lo que:
        \begin{align*}
            \int_{C(0,r)} \frac{\log(1+z)}{z}dz &= \int_{-\pi}^{\pi} i\log(1+re^{it})\ dt = \int_{-\pi}^{\pi} \left(i\cdot\sum\limits_{n=0}^\infty \dfrac{(-1)^{n+1}}{n}(re^{it})^{n}\right)\ dt
            =\\&= \int_{-\pi}^{\pi} \left(\sum\limits_{n=0}^\infty \dfrac{(-1)^{n+1}}{n^2}\cdot nr^ne^{int}\right)\ dt
        \end{align*}

        Por lo demostrado en la Subsección~\ref{sec:converge_uniforme_integral_cauchy}, como sabemos que la serie converge uniformemente, podemos permutar la integral con la suma de la serie:
        \begin{align*}
            \int_{C(0,r)} \frac{\log(1+z)}{z}dz &= \sum\limits_{n=0}^\infty\left(\int_{-\pi}^{\pi} \dfrac{(-1)^{n+1}}{n^2}\cdot inr^ne^{int}\ dt\right)
            =\\&= \sum\limits_{n=0}^\infty\left(\dfrac{(-1)^{n+1}}{n^2}\cdot r^n\cdot \int_{-\pi}^{\pi} ine^{int}\ dt\right)
            =\\&= \sum\limits_{n=0}^\infty\left(\dfrac{(-1)^{n+1}}{n^2}\cdot r^n\cdot \left(e^{i\pi n}-e^{-i\pi n}\right)\right)
            = \sum\limits_{n=0}^\infty\left(\dfrac{(-1)^{n+1}}{n^2}\cdot r^n\cdot 0\right)=0
        \end{align*}

        
    \end{description}

    En resumen, por cualquier opción hemos demostrado que:
    \begin{equation*}
        \int_{C(0,r)} \frac{\log(1+z)}{z}dz = 0
    \end{equation*}

    Por la Opción Rutinaria, vemos por tanto que:
    \begin{align*}
        0&= \int_{-\pi}^{\pi} \ln|1+re^{it}|\ dt
        = \int_{-\pi}^{\pi} \ln\left(\sqrt{(1+r\cos t)^2+(r\sen t)^2}\right)\ dt
        =\\&= \int_{-\pi}^{\pi} \ln\left(\sqrt{1+r^2+2r\cos t}\right)\ dt
        = \dfrac{1}{2}\int_{-\pi}^{\pi} \ln\left(1+r^2+2r\cos t\right)\ dt
    \end{align*}

    Como la función $t\mapsto \ln\left(1+r^2+2r\cos t\right)$ es par, tenemos que:
    \begin{align*}
        0&= \dfrac{1}{2}\int_{-\pi}^{\pi} \ln\left(1+r^2+2r\cos t\right)\ dt
        = \int_{0}^{\pi} \ln\left(1+r^2+2r\cos t\right)\ dt
    \end{align*}
    como se quería demostrar. Vemos que, habiendo uso de variable compleja, hemos logrado resolver una integral real que, de por si, no es trivial de resolver.
\end{ejercicio}

\begin{ejercicio}
    Sea $f \in \cc{H}(D(0,1))$ verificando que $|f(z)-1| < 1$ para todo $z \in D(0,1)$. Admitiendo que $f'$ es continua, probar que
    \[
        \int_{C(0,r)} \frac{f'(z)}{f(z)}dz = 0\qquad \forall r \in \left]0,1\right[.
    \]

    Definimos la función:
    \Func{F}{D(0,1)}{\bb{C}}{z}{\log(f(z))}

    Supongamos por reducción al absurdo que $\exists z_0 \in D(0,1)$ tal que $f(z_0)\in \bb{R}_0^-$. Entonces, $f(z_0)-1\in \bb{R}$, con $f(z_0)-1<-1$, luego $|f(z_0)-1|>1$, lo que es una contradicción. Por tanto, $f(z)\in \bb{C}^\ast\setminus \bb{R}^-$ para todo $z \in D(0,1)$. Como se tiene $\log\in \cc{H}(\bb{C}^\ast\setminus \bb{R}^-)$, tenemos que $F\in \cc{H}(D(0,1))$, con:
    \begin{align*}
        F'(z) &= \frac{f'(z)}{f(z)}\qquad \forall z \in D(0,1)
    \end{align*}

    Por tanto, $F$ es una primitiva de $f'$. Como $C(0,r)$ es un camino cerrado en $D(0,1)$ para todo $r \in \left]0,1\right[$, por el Teorema de Caracterización de existencia de primitivas, tenemos que:
    \begin{equation*}
        \int_{C(0,r)} f(z)dz = 0\qquad \forall r \in \left]0,1\right[
    \end{equation*}
\end{ejercicio}


\begin{ejercicio}
    Sean $\Omega = \bb{C}\setminus\{i,-i\}$ y $f \in \cc{H}(\Omega)$ dada por:
    \Func{f}{\Omega}{\bb{C}}{z}{\frac{1}{1+z^2}}
    Probar que $f$ no admite una primitiva en $\Omega$.\\

    Por el Teorema de Caracterización de existencia de primitivas, como $\Omega$ es un abierto no vacío y $f\in \cc{C}(\Omega)$, hemos de encontrar un camino cerrado $\sigma$ en $\Omega$ tal que:
    \begin{equation*}
        \int_{\sigma} f(z)dz \neq 0
    \end{equation*}

    En primer lugar, por el método de descomposición en fracciones simples, tenemos que:
    \begin{equation*}
        \frac{1}{1+z^2} = \frac{A}{z-i} + \frac{B}{z+i} = \frac{A(z+i)+B(z-i)}{(z-i)(z+i)}
    \end{equation*}
    \begin{itemize}
        \item \ul{Si $z=i$}: $1=2iA\Longrightarrow A=\nicefrac{1}{2i}$.
        \item \ul{Si $z=-i$}: $1=-2iB\Longrightarrow B=-\nicefrac{1}{2i}$.
    \end{itemize}

    Por tanto, tenemos que:
    \begin{equation*}
        \int_{\sigma} f(z)dz = \dfrac{1}{2i}\left(\int_{\sigma} \frac{dz}{z-i} - \int_{\sigma} \frac{dz}{z+i}\right)
    \end{equation*}

    Con vistas a calcular dicha integral, consideramos como camino cerrado la circunferencia $C(i,1)$:
    \Func{\sigma}{[-\pi,\pi]}{\Omega}{t}{i+e^{it}}

    En primer lugar, tenemos que:
    \begin{align*}
        \int_{C(i,1)} \frac{dz}{z-i} &= \int_{-\pi}^{\pi} \frac{\sigma'(t)}{\sigma(t)-i}dt
        = \int_{-\pi}^{\pi} \frac{ie^{it}}{e^{it}}dt
        = 2\pi i
    \end{align*}

    En segundo lugar, tenemos que:
    \begin{align*}
        \int_{C(i,1)} \frac{dz}{z+i} &= \int_{-\pi}^{\pi} \frac{\sigma'(t)}{\sigma(t)+i}dt
        = \int_{-\pi}^{\pi} \frac{ie^{it}}{e^{it}+2i}dt
        = \int_{-\pi}^{\pi} \frac{ie^{it}\cdot \ol{e^{it}+2i}}{(e^{it}+2i)\cdot \ol{e^{it}+2i}}dt
        =\\&= \int_{-\pi}^{\pi} \frac{ie^{it}\cdot \left(\ol{e^{it}}+\ol{2i}\right)}{|e^{it}+2i|^2}dt
        = \int_{-\pi}^{\pi} \frac{i-2i^2e^{it}}{{\cos^2t + (2+\sen t)^2}}dt
        =\\&= \int_{-\pi}^{\pi} \frac{2e^{it}+i}{{\cos^2t + 4+\sen^2t+4\sen t}}dt
        = \int_{-\pi}^{\pi} \frac{2e^{it}+i}{{5+4\sen t}}dt
        =\\&= \left(\int_{-\pi}^{\pi} \frac{2\cos t}{{5+4\sen t}}dt\right) + i\left(\int_{-\pi}^{\pi} \frac{1+2\sen t}{{5+4\sen t}}dt\right)
    \end{align*}

    Por un lado, tenemos que:
    \begin{align*}
        \int_{-\pi}^{\pi} \frac{2\cos t}{{5+4\sen t}}dt
        &= \dfrac{1}{2}\cdot \int_{-\pi}^{\pi} \frac{4\cos t}{{5+4\sen t}}dt
        = \dfrac{1}{2}\cdot \left[\ln(5+4\sen t)\right]_{-\pi}^{\pi}
        = 0
    \end{align*}

    Calcular la segunda integral no es directo, por lo que vamos a intentar evitarlo. Tenemmos que:
    \begin{align*}
        0 = \int_{\sigma} f(z)dz &\iff  \dfrac{1}{2i}\left(\int_{\sigma} \frac{dz}{z-i} - \int_{\sigma} \frac{dz}{z+i}\right) = 0\iff
        \\&\iff \dfrac{1}{2i}\left(2\pi i-\int_{\sigma} \frac{dz}{z+i}\right) = 0\iff
        \\&\iff \int_{\sigma} \frac{dz}{z+i} = i\left(\int_{-\pi}^{\pi} \frac{1+2\sen t}{{5+4\sen t}}dt\right) = 2\pi i
    \end{align*}

    Por tanto, esto solo se dará si:
    \begin{equation*}
        2\pi = \int_{-\pi}^{\pi} \frac{1+2\sen t}{{5+4\sen t}}dt
    \end{equation*}

    Para todo $t \in [-\pi,\pi]$, tenemos que:
    \begin{align*}
        \frac{1+2\sen t}{{5+4\sen t}}=\dfrac{1}{2}\cdot \frac{2+4\sen t}{{5+4\sen t}}\leq \dfrac{1}{2}
        \iff \frac{2+4\sen t}{{5+4\sen t}}\leq 1\stackrel{(\ast)}{\iff} 2+4\sen t\leq 5+4\sen t\iff 2\leq 5
    \end{align*}
    donde en $(\ast)$ hemos aplicado que $5\geq -4\sen t$ para todo $t \in \bb{R}$.
    Por tanto:
    \begin{align*}
        \int_{-\pi}^{\pi} \frac{1+2\sen t}{{5+4\sen t}}dt &\leq \int_{-\pi}^{\pi} \frac{1}{2}dt
        = \dfrac{1}{2}\cdot 2\pi = \pi
    \end{align*}

    Por tanto:
    \begin{align*}
        2\pi \neq \int_{-\pi}^{\pi} \frac{1+2\sen t}{{5+4\sen t}}dt
        \Longrightarrow
        \int_{\sigma} f(z)dz \neq 0
    \end{align*}

    Por tanto, se ha encontrado un camino cerrado $\sigma$ en $\Omega$ tal que su integral no es nula. Por el Teorema de Caracterización de existencia de primitivas, deducimos que $f$ no admite una primitiva en $\Omega$.
\end{ejercicio}

\begin{ejercicio}
    Probar que
    \[
        \int_{\sigma} \frac{dz}{1+z^2} = 0,
    \]
    donde $\sigma(t) = \cos t + \nicefrac{i}{2}\cdot \sen t$ para todo $t \in [0,2\pi]$.
    \begin{description}
        \item[Método Rutinario:]~
        
        Por definición de integral sobre un arco, tenemos que:
        \begin{align*}
            \int_{\sigma} \frac{dz}{1+z^2} &= \int_{0}^{2\pi} \frac{\sigma'(t)}{1+\sigma(t)^2}dt
            = \int_{0}^{2\pi} \frac{\left(-\sen t + \nicefrac{i}{2}\cdot \cos t\right)}{1+\left(\cos t + \nicefrac{i}{2}\cdot \sen t\right)^2}dt
            =\\&= \int_{0}^{2\pi} \frac{-\sen t + \nicefrac{i}{2}\cdot \cos t}{1+\cos^2t -\nicefrac{1}{4}\sen^2t+i\sen t\cos t}dt
            =\\&= \int_{0}^{2\pi} \frac{-\sen t + \nicefrac{i}{2}\cdot \cos t}{2-\nicefrac{5}{4}\sen^2t+i\sen t\cos t}dt
        \end{align*}
        Como preveíamos, la integral no es trivial de resolver. Por tanto, descartamos esta opción.

        \item[Método Alternativo:]~
        
        Sea $U$ el siguiente conjunto:
        \begin{equation*}
            U=\bb{C}\setminus \{iy\mid y \in \bb{R},|y|\geq 1\}
        \end{equation*}

        Definimos la siguiente función:
        \Func{F}{U}{\bb{C}}{z}{\arctan z}

        Sabemos que $F\in \cc{H}(U)$, con:
        \begin{align*}
            F'(z) &= \frac{1}{1+z^2}\qquad \forall z \in U
        \end{align*}

        Por tanto, $F$ es una primitiva de $\frac{1}{1+z^2}$. Vemos además que $\sigma$ es un camino cerrado, por lo que hemos de ver que $\sigma\left([0,2\pi]\right) \subset U$.
        \begin{itemize}
            \item Si $t \in [0,2\pi]\setminus \left\{\frac{\pi}{2},\frac{3\pi}{2}\right\}$, entonces $\cos t\neq 0$, y por tanto $\Re(\sigma(t))\neq 0$. Por tanto, $\sigma(t) \in U$.
            
            \item Si $t\in \left\{\frac{\pi}{2},\frac{3\pi}{2}\right\}$, entonces $|\Im(\sigma(t))|=|\sen t|=\left|\nicefrac{1}{2}\right|<1$, y por tanto $\sigma(t) \in U$.            
        \end{itemize}

        Por tanto, $\sigma$ es un camino cerrado en $U$. Como el integrando es continuo en $\Omega$, por el Teorema de Caracterización de existencia de primitivas, tenemos que:
        \begin{equation*}
            \int_{\sigma} \frac{dz}{1+z^2} = 0
        \end{equation*}
        como se quería demostrar.
    \end{description}
\end{ejercicio}




\newpage
    
% ##############################################################
% Anexo: Sobre la convergencia uniforme y la integral de Cauchy 
% ##############################################################
\subsection{Sobre la convergencia uniforme de la integral de Cauchy}\label{sec:converge_uniforme_integral_cauchy}
Al igual que ocurría en el caso real, la convergencia uniforme de
sucesiones de funciones continuas definidas en intervalos compactos
nos permiten permutar la integral con el límite puntual de una sucesión
de funciones. En particular, podemos permutar una integral con la suma
de una serie de funciones continuas siempre que tengamos convergencia
uniforme en un intervalo compacto:
\begin{teo}
    Dados $a,b \in \mathbb{R}$ con $a < b$, supongamos que $\{f_n\}$ es una sucesión de funciones
    que converge uniformemente en $[a,b]$ a $f : [a,b] \rightarrow \mathbb{C}$ y que
    $f_n \in \mathcal{C}([a,b])$ para todo $n \in \mathbb{N}$. Entonces, se tiene que
    \begin{equation*}
        \left\{\displaystyle\int_{a}^{b} f_n(t) \,dt\right\} \rightarrow
        \displaystyle\int_{a}^{b} f(t) \,dt
    \end{equation*}
    En particular, si $\{f_n\}$ es una sucesión de funciones tal que $\displaystyle\sum_{n \geq 1} f_n$
    converge uniformemente en $[a,b]$ y $f_n \in \mathcal{C}([a,b])$ para todo $n \in \mathbb{N}$,
    se tiene que
    \begin{equation*}
        \displaystyle\sum_{n=1}^{\infty} \left(\int_{a}^{b} f_n(t) \,dt\right)=
        \displaystyle\int_{a}^{b} \left(\displaystyle\sum_{n=1}^{\infty} f_n(t) \right) \,dt
    \end{equation*}
\end{teo}

\begin{proof}
    En primer lugar, observamos que la integral del miembro derecho tiene perfecto sentido,
    ya que la convergencia uniforme de $\{f_n\}$ en $[a,b]$ nos garantiza la continuidad de $f$
    en dicho intervalo.
    \newline
    \newline
    Dado $\varepsilon > 0$, la convergencia uniforme de $\{f_n\}$ nos
    dice que existe un $m \in \mathbb{N}$ tal que
    \begin{equation*}
        n \geq m \Longrightarrow |f_n(t)-f(t)| < \frac{\varepsilon}{b-a} ~\forall t \in [a,b]
    \end{equation*}
    Por tanto, para $n \geq m$ se tiene que
    \begin{equation*}
        \left|\int_{a}^{b} f_n(t) \,dt - \int_{a}^{b} f(t) \,dt\right| =
        \left|\int_{a}^{b} (f_n(t) - f(t)) \,dt\right| \leq \int_{a}^{b} |f_n(t)-f(t)| \,dt < \varepsilon
    \end{equation*}
    lo que demuestra que $\left\{\displaystyle\int_{a}^{b} f_n(t) \,dt\right\} \rightarrow
    \displaystyle\int_{a}^{b} f(t) \,dt$.
    \newline
    \newline
    Por otra parte, si $\displaystyle\sum_{n \geq 1} f_n$ converge uniformemente en $[a,b]$,
    escribiendo $S_n=\displaystyle\sum_{k=1}^{n} f_k$ para todo $n \in \mathbb{N}$, tenemos
    que $\{S_n\}$ es una sucesión de funciones que converge uniformemente en $[a,b]$, por
    lo que aplicando lo que acabamos de demostrar:
    \begin{equation*}
        \displaystyle\int_{a}^{b} \left(\displaystyle\sum_{n=1}^{\infty} f_n(t)\right) \,dt=
        \displaystyle\int_{a}^{b} \left(\lim_{n \to \infty} S_n(t) \,dt\right) =
        \lim_{n \to \infty} \displaystyle\sum_{k=1}^{n} \displaystyle\int_{a}^{b} f_k(t) \,dt =
        \displaystyle\sum_{n=1}^{\infty} \left(\int_{a}^{b} f_n(t) \,dt\right)
    \end{equation*}
    como se quería demostrar.
    \end{proof}
\section{Funciones analíticas}

\begin{ejercicio}
    Calcular el radio de convergencia de las siguientes series de potencias:
    \begin{enumerate}
        \item $\displaystyle \sum_{n \geq 1} \dfrac{n!}{n^n}z^n$
        
        Para cada $n\in \mathbb{N}$, definimos:
        \[
            \alpha_n = \dfrac{n!}{n^n}
        \]

        Calculamos el siguiente límite:
        \begin{align*}
            \lim_{n \to \infty} \dfrac{\alpha_{n+1}}{\alpha_n} &= \lim_{n \to \infty} \dfrac{(n+1)!}{(n+1)^{n+1}} \cdot \dfrac{n^n}{n!}
            =\\ &= \lim_{n \to \infty} \dfrac{(n+1)n^n}{(n+1)^{n+1}}
            = \lim_{n \to \infty} \left(\dfrac{n}{n+1}\right)^n
        \end{align*}

        Como la base tiende a $1$ y el exponente diverge positivamente, aplicamos el criterio de Euler, y tenemos:
        \begin{align*}
            \lim_{n \to \infty} \dfrac{\alpha_{n+1}}{\alpha_n} &=
            \exp\left[\lim_{n \to \infty} n\left(\dfrac{n}{n+1}-1\right)\right]
            = \exp\left[\lim_{n \to \infty} n\left(\dfrac{n-n-1}{n+1}\right)\right]
            =\\&= \exp\left[\lim_{n \to \infty} \dfrac{-n}{n+1}\right]
            = e^{-1} = \dfrac{1}{e}
        \end{align*}

        Por tanto, por el criterio del cociente para sucesiones y por la Fórmula de Cauchy-Hadamard, tenemos que:
        \begin{equation*}
            \{\sqrt[n]{\alpha_n}\} \to \dfrac{1}{e} \implies R = \dfrac{1}{\nicefrac{1}{e}} = e
        \end{equation*}
        \item $\displaystyle \sum_{n \geq 0} z^{2n}$
        
        En primer lugar, vemos que no se trata de forma directa de una serie de potencias. No obstante, definimos la siguiente sucesión $\{\alpha_n\}$:
        \[
            \alpha_n = \begin{cases}
                1 & \text{si } n\ \text{es par}\\
                0 & \text{si } n\ \text{es impar}
            \end{cases}
        \]

        De esta forma, tenemos que:
        \begin{equation*}
            \sum_{n \geq 0} z^{2n} = \sum_{n \geq 0} \alpha_nz^n
        \end{equation*}

        Estudiamos por tanto la sucesión $\{\sqrt[n]{\alpha_n}\}=\{\alpha_n\}$. Tenemos en primer lugar que no es convergente, por lo que no podemos considerar su límite. No obstante, tenemos que está acotada, por lo que consideramos su límite superior:
        \begin{equation*}
            \limsup\{\sqrt[n]{\alpha_n}\} = \limsup\{\alpha_n\} = \lim_{n \to \infty} \sup\{\alpha_k\mid k \geq n\} =
            \lim_{n \to \infty} \sup\{1,0\} = \sup\{1,0\} = 1
        \end{equation*}

        Por tanto, por la Fórmula de Cauchy-Hadamard, tenemos que:
        \begin{equation*}
            R = \dfrac{1}{\limsup\{\sqrt[n]{\alpha_n}\}} = \dfrac{1}{1} = 1
        \end{equation*}
        \item $\displaystyle \sum_{n \geq 0} 2^nz^{n!}$
        
        De nuevo, no está en la forma de una serie de potencias. No obstante, definimos en primer lugar el siguiente conjunto $M$:
        \[
            M = \{n! \mid n \in \mathbb{N}\} = \{0,1,2,6,24,\ldots\}
        \]
        que claramente es infinito.

        De esta forma, definimos la sucesión $\{\alpha_n\}$:
        \[
            \alpha_n = \begin{cases}
                2^n & \text{si } n \in M\\
                0 & \text{si } n \notin M
            \end{cases}
        \]

        Por tanto, tenemos que:
        \begin{equation*}
            \sqrt[n]{\alpha_n} = \begin{cases}
                2 & \text{si } n \in M\\
                0 & \text{si } n \notin M
            \end{cases}
        \end{equation*}

        Por tanto, para cada $n\in \bb{N}$, tenemos:
        \begin{equation*}
            \sup\{\sqrt[k]{\alpha_k} \mid k \geq n\} = \sup\{0,2\} = 2\qquad \forall n\in \bb{N}
        \end{equation*}

        Por tanto, el límite superior de la sucesión $\{\sqrt[n]{\alpha_n}\}$ es:
        \begin{equation*}
            \limsup\{\sqrt[n]{\alpha_n}\} = \lim_{n \to \infty} \sup\{\sqrt[k]{\alpha_k} \mid k \geq n\} = \lim_{n \to \infty} 2 = 2
        \end{equation*}

        Por tanto, por la Fórmula de Cauchy-Hadamard, tenemos que:
        \begin{equation*}
            R = \dfrac{1}{\limsup\{\sqrt[n]{\alpha_n}\}} = \dfrac{1}{2}
        \end{equation*}
        
        \item $\displaystyle \sum_{n \geq 0} \left(3+(-1)^n\right)^n z^n$
        
        Definimos la sucesión $\{\alpha_n\}$:
        \[
            \alpha_n = \left(3+(-1)^n\right)^n
        \]

        Por tanto, la sucesión $\{\sqrt[n]{\alpha_n}\}$ es:
        \[
            \sqrt[n]{\alpha_n} = \sqrt[n]{\left(3+(-1)^n\right)^n} = 3+(-1)^n\qquad \forall n\in \bb{N}
        \]

        Vemos que no es convergente, pero sí está acotada, puesto que:
        \begin{equation*}
            \sqrt[n]{\alpha_n} = \begin{cases}
                4 & \text{si } n\ \text{es par}\\
                2 & \text{si } n\ \text{es impar}
            \end{cases}
        \end{equation*}

        Por tanto, podemos considerar el límite superior de la sucesión $\{\sqrt[n]{\alpha_n}\}$:
        \begin{equation*}
            \limsup\{\sqrt[n]{\alpha_n}\} = \lim_{n \to \infty} \sup\{\sqrt[k]{\alpha_k} \mid k \geq n\} = \lim_{n \to \infty} \sup\{2,4\} = 4
        \end{equation*}

        Por tanto, por la Fórmula de Cauchy-Hadamard, tenemos que:
        \begin{equation*}
            R = \dfrac{1}{\limsup\{\sqrt[n]{\alpha_n}\}} = \dfrac{1}{4}
        \end{equation*}

        \item $\displaystyle \sum_{n \geq 0} \left(n+a^n\right)z^n$ con $a \in \mathbb{R}^+$
        
        Definimos la sucesión $\{\alpha_n\}$:
        \[
            \alpha_n = n+a^n
        \]

        Es deirecto ver que $|\alpha_n|=\alpha_n$ para todo $n\in \bb{N}$. Para estudiar la sucesión $\{\sqrt[n]{\alpha_n}\}$ empleamos el criterio del cociente para sucesiones:
        \begin{align*}
            \dfrac{\alpha_{n+1}}{\alpha_n} &= \dfrac{n+1+a^{n+1}}{n+a^n}\qquad \forall n\in \bb{N}
        \end{align*}
        \item $\displaystyle \sum_{n \geq 0} a^{n^2}z^n$ con $a \in \mathbb{C}$
    \end{enumerate}
\end{ejercicio}

\begin{ejercicio}
    Conocido el radio de convergencia $R$ de la serie $\displaystyle \sum_{n \geq 0} \alpha_nz^n$, calcular el de las siguientes:
    \begin{enumerate}
        \item $\displaystyle \sum_{n \geq 0} n^k\alpha_nz^n$ con $k \in \mathbb{N}$ fijo.
        \item $\displaystyle \sum_{n \geq 0} \dfrac{\alpha_n}{n!}z^n$
    \end{enumerate}
\end{ejercicio}

\begin{ejercicio}
    Caracterizar las series de potencias que convergen uniformemente en todo el plano.
\end{ejercicio}

\begin{ejercicio}
    Estudiar la convergencia puntual, absoluta y uniforme, de la serie $\displaystyle \sum_{n \geq 0} f_n$ donde:
    \[
        f_n(z) = \left(\dfrac{z-1}{z+1}\right)^n \quad \text{para todo } z \in \mathbb{C}\setminus\{-1\}
    \]
\end{ejercicio}
\section{Funciones analíticas}

\begin{ejercicio}
    Calcular el radio de convergencia de las siguientes series de potencias:
    \begin{enumerate}
        \item $\displaystyle \sum_{n \geq 1} \dfrac{n!}{n^n}z^n$
        \item $\displaystyle \sum_{n \geq 0} z^{2n}$
        \item $\displaystyle \sum_{n \geq 0} 2^nz^{n!}$
        \item $\displaystyle \sum_{n \geq 0} \left(3+(-1)^n\right)^n z^n$
        \item $\displaystyle \sum_{n \geq 0} \left(n+a^n\right)z^n$ con $a \in \mathbb{R}^+$
        \item $\displaystyle \sum_{n \geq 0} a^{n^2}z^n$ con $a \in \mathbb{C}$
    \end{enumerate}
\end{ejercicio}

\begin{ejercicio}
    Conocido el radio de convergencia $R$ de la serie $\displaystyle \sum_{n \geq 0} \alpha_nz^n$, calcular el de las siguientes:
    \begin{enumerate}
        \item $\displaystyle \sum_{n \geq 0} n^k\alpha_nz^n$ con $k \in \mathbb{N}$ fijo.
        \item $\displaystyle \sum_{n \geq 0} \dfrac{\alpha_n}{n!}z^n$
    \end{enumerate}
\end{ejercicio}

\begin{ejercicio}
    Caracterizar las series de potencias que convergen uniformemente en todo el plano.
\end{ejercicio}

\begin{ejercicio}
    Estudiar la convergencia puntual, absoluta y uniforme, de la serie $\displaystyle \sum_{n \geq 0} f_n$ donde:
    \[
        f_n(z) = \left(\dfrac{z-1}{z+1}\right)^n \quad \text{para todo } z \in \mathbb{C}\setminus\{-1\}
    \]
\end{ejercicio}
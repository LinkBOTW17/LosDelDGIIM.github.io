\section{Funciones analíticas}

\begin{ejercicio}
    Calcular el radio de convergencia de las siguientes series de potencias:
    \begin{enumerate}
        \item $\displaystyle \sum_{n \geq 1} \dfrac{n!}{n^n}z^n$
        
        Para cada $n\in \mathbb{N}$, definimos:
        \[
            \alpha_n = \dfrac{n!}{n^n}
        \]

        Con vistas a aplicar el criterio del cociente para sucesiones, consideramos el siguiente cociente:
        \begin{align*}
             \dfrac{\alpha_{n+1}}{\alpha_n} &=  \dfrac{(n+1)!}{(n+1)^{n+1}} \cdot \dfrac{n^n}{n!}
            =  \dfrac{(n+1)n^n}{(n+1)^{n+1}}
            =  \left(\dfrac{n}{n+1}\right)^n
        \end{align*}

        Como la base tiende a $1$ y el exponente diverge positivamente, aplicamos el criterio de Euler, y tenemos:
        \begin{align*}
            \lim_{n \to \infty} \dfrac{\alpha_{n+1}}{\alpha_n} &=
            \exp\left[\lim_{n \to \infty} n\left(\dfrac{n}{n+1}-1\right)\right]
            = \exp\left[\lim_{n \to \infty} n\left(\dfrac{n-n-1}{n+1}\right)\right]
            =\\&= \exp\left[\lim_{n \to \infty} \dfrac{-n}{n+1}\right]
            = e^{-1} = \dfrac{1}{e}
        \end{align*}

        Por tanto, por el criterio del cociente para sucesiones y por la Fórmula de Cauchy-Hadamard, tenemos que:
        \begin{equation*}
            \{\sqrt[n]{\alpha_n}\} \to \dfrac{1}{e} \implies R = \dfrac{1}{\nicefrac{1}{e}} = e
        \end{equation*}
        \item $\displaystyle \sum_{n \geq 0} z^{2n}$
        
        En primer lugar, vemos que no se trata de forma directa de una serie de potencias. No obstante, definimos la siguiente sucesión $\{\alpha_n\}$:
        \[
            \alpha_n = \begin{cases}
                1 & \text{si } n\ \text{es par}\\
                0 & \text{si } n\ \text{es impar}
            \end{cases}
        \]

        De esta forma, tenemos que:
        \begin{equation*}
            \sum_{n \geq 0} z^{2n} = \sum_{n \geq 0} \alpha_nz^n
        \end{equation*}

        Estudiamos por tanto la sucesión $\{\sqrt[n]{\alpha_n}\}=\{\alpha_n\}$. Tenemos en primer lugar que no es convergente, por lo que no podemos considerar su límite. No obstante, tenemos que está acotada, por lo que consideramos su límite superior:
        \begin{equation*}
            \limsup\{\sqrt[n]{\alpha_n}\} = \limsup\{\alpha_n\} = \lim_{n \to \infty} \sup\{\alpha_k\mid k \geq n\} =
            \lim_{n \to \infty} \sup\{1,0\} = \sup\{1,0\} = 1
        \end{equation*}

        Por tanto, por la Fórmula de Cauchy-Hadamard, tenemos que:
        \begin{equation*}
            R = \dfrac{1}{\limsup\{\sqrt[n]{\alpha_n}\}} = \dfrac{1}{1} = 1
        \end{equation*}
        \item $\displaystyle \sum_{n \geq 0} 2^nz^{n!}$
        
        De nuevo, no está en la forma de una serie de potencias. No obstante, definimos en primer lugar el siguiente conjunto $M$:
        \[
            M = \{n! \mid n \in \bb{N}_0\}
        \]
        Además, para cada $n\in M$, sea $k\in \bb{N}$ tal que $k! = n$. Por tanto, podemos escribir la sucesión $\{\alpha_n\}$ como:
        \[
            \alpha_n = \begin{cases}
                2^k & \text{si } n \in M\\
                0 & \text{si } n \notin M
            \end{cases}
        \]

        Por tanto, tenemos que:
        \begin{equation*}
            \sqrt[n]{\alpha_n} = \begin{cases}
                \sqrt[k!]{2^k} & \text{si } n \in M\\
                0 & \text{si } n \notin M
            \end{cases}
        \end{equation*}

        Entonces, tenemos que:
        \begin{align*}
            \sqrt[n]{\alpha_n} &= 2^{\frac{k}{k!}} = 2^{\frac{1}{(k-1)!}}\qquad \forall n\in M
        \end{align*}

        Por tanto, el límite superior de la sucesión $\{\sqrt[n]{\alpha_n}\}$ es:
        \begin{equation*}
            \limsup\{\sqrt[n]{\alpha_n}\} = \lim_{n \to \infty} \sup\{\sqrt[p]{\alpha_p} \mid p \geq n\} = \lim_{k \to \infty} 2^{\frac{1}{(k-1)!}} = 2^0 = 1
        \end{equation*}

        Por tanto, por la Fórmula de Cauchy-Hadamard, tenemos que:
        \begin{equation*}
            R = \dfrac{1}{\limsup\{\sqrt[n]{\alpha_n}\}} = 1
        \end{equation*}
        
        \item $\displaystyle \sum_{n \geq 0} \left(3+(-1)^n\right)^n z^n$
        
        Definimos la sucesión $\{\alpha_n\}$:
        \[
            \alpha_n = \left(3+(-1)^n\right)^n
        \]

        Por tanto, la sucesión $\{\sqrt[n]{\alpha_n}\}$ es:
        \[
            \sqrt[n]{\alpha_n} = \sqrt[n]{\left(3+(-1)^n\right)^n} = 3+(-1)^n\qquad \forall n\in \bb{N}
        \]

        Vemos que no es convergente, pero sí está acotada, puesto que:
        \begin{equation*}
            \sqrt[n]{\alpha_n} = \begin{cases}
                4 & \text{si } n\ \text{es par}\\
                2 & \text{si } n\ \text{es impar}
            \end{cases}
        \end{equation*}

        Por tanto, podemos considerar el límite superior de la sucesión $\{\sqrt[n]{\alpha_n}\}$:
        \begin{equation*}
            \limsup\{\sqrt[n]{\alpha_n}\} = \lim_{n \to \infty} \sup\{\sqrt[k]{\alpha_k} \mid k \geq n\} = \lim_{n \to \infty} \sup\{2,4\} = 4
        \end{equation*}

        Por tanto, por la Fórmula de Cauchy-Hadamard, tenemos que:
        \begin{equation*}
            R = \dfrac{1}{\limsup\{\sqrt[n]{\alpha_n}\}} = \dfrac{1}{4}
        \end{equation*}

        \item $\displaystyle \sum_{n \geq 0} \left(n+a^n\right)z^n$ con $a \in \mathbb{R}^+$
        
        Definimos la sucesión $\{\alpha_n\}$:
        \[
            \alpha_n = n+a^n
        \]

        Es directo ver que $|\alpha_n|=\alpha_n$ para todo $n\in \bb{N}$. Para estudiar la sucesión $\{\sqrt[n]{\alpha_n}\}$ empleamos el criterio del cociente para sucesiones:
        \begin{align*}
            \dfrac{\alpha_{n+1}}{\alpha_n} &= \dfrac{n+1+a^{n+1}}{n+a^n}
            = \dfrac{n+a\cdot a^{n}}{n+a^n}+\dfrac{1}{n+a^n}\qquad \forall n\in \bb{N}
        \end{align*}

        Independientemente del valor de $a\in \bb{R}^+$, tenemos que:
        \begin{equation*}
            \left\{\dfrac{1}{n+a^n}\right\}\to 0
        \end{equation*}

        Para el otro sumando, distinguimos en función de los valores de $a$:
        \begin{itemize}
            \item Si $a=1$, tenemos que:
            \begin{align*}
                \left\{\dfrac{n+a\cdot a^{n}}{n+a^n}\right\} &= \left\{\dfrac{n+1}{n+1}\right\} = \left\{1\right\}\to 1
            \end{align*}

            Por tanto, por el Criterio del Cociente para sucesiones, tenemos que:
            \begin{equation*}
                \dfrac{\alpha_{n+1}}{\alpha_n}\to 1+0 = 1
                \Longrightarrow
                \{\sqrt[n]{\alpha_n}\}\to 1
            \end{equation*}

            Por tanto, por la Fórmula de Cauchy-Hadamard, tenemos que:
            \begin{equation*}
                R = \dfrac{1}{\limsup\{\sqrt[n]{\alpha_n}\}} = \dfrac{1}{1} = 1
            \end{equation*}

            \item Si $a<1$, tenemos que:
            \begin{align*}
                \left\{\dfrac{n+a\cdot a^{n}}{n+a^n}\right\} &= \left\{\dfrac{1+a\cdot \dfrac{a^{n}}{n}}{1+\dfrac{a^n}{n}}\right\}\to 1
            \end{align*}

            Por tanto, por el Criterio del Cociente para sucesiones, tenemos que:
            \begin{equation*}
                \left\{\dfrac{\alpha_{n+1}}{\alpha_n}\right\}\to 1
                \Longrightarrow
                \{\sqrt[n]{\alpha_n}\}\to 1
            \end{equation*}

            Por tanto, por la Fórmula de Cauchy-Hadamard, tenemos que:
            \begin{equation*}
                R = \dfrac{1}{\limsup\{\sqrt[n]{\alpha_n}\}} = 1
            \end{equation*}

            \item Si $a>1$, tenemos que:
            \begin{align*}
                \left\{\dfrac{n+a\cdot a^{n}}{n+a^n}\right\} &= \left\{\dfrac{\dfrac{n}{a^n}+a}{\dfrac{n}{a^n}+1}\right\}\to a
            \end{align*}
            puesto que $\left\{\frac{n}{a^n}\right\} \to 0$.            
            Por tanto, por el Criterio del Cociente para sucesiones, tenemos que:
            \begin{equation*}
                \left\{\dfrac{\alpha_{n+1}}{\alpha_n}\right\}\to a
                \Longrightarrow
                \{\sqrt[n]{\alpha_n}\}\to a
            \end{equation*}

            Por tanto, por la Fórmula de Cauchy-Hadamard, tenemos que:
            \begin{equation*}
                R = \dfrac{1}{\limsup\{\sqrt[n]{\alpha_n}\}} = \dfrac{1}{a}
            \end{equation*}            
        \end{itemize}
        \item $\displaystyle \sum_{n \geq 0} a^{n^2}z^n$ con $a \in \mathbb{C}$
        
        Definimos la sucesión $\{\alpha_n\}$:
        \[
            \alpha_n = a^{n^2}
        \]

        Tenemos que:
        \begin{equation*}
            \sqrt[n]{|\alpha_n|} = \sqrt[n]{|a|^{n^2}} = |a|^n\qquad \forall n\in \bb{N}
        \end{equation*}

        Por tanto, distinguimos en función de los valores de $|a|$:
        \begin{itemize}
            \item Si $|a|<1$, tenemos que:
            \begin{equation*}
                \lim_{n \to \infty} \sqrt[n]{|\alpha_n|} = \lim_{n \to \infty} |a|^n = 0
            \end{equation*}

            Por tanto, por la Fórmula de Cauchy-Hadamard, tenemos que:
            \begin{equation*}
                R = \infty
            \end{equation*}

            \item Si $|a|=1$, tenemos que:
            \begin{equation*}
                \lim_{n \to \infty} \sqrt[n]{|\alpha_n|} = \lim_{n \to \infty} 1 = 1
            \end{equation*}

            Por tanto, por la Fórmula de Cauchy-Hadamard, tenemos que:
            \begin{equation*}
                R = 1
            \end{equation*}

            \item Si $|a|>1$, tenemos que:
            \begin{equation*}
                \left\{\sqrt[n]{|\alpha_n|}\right\} = \left\{|a|^n\right\}
            \end{equation*}

            Supongamos que dicha sucesión está mayorada; es decir, que existe $M\in \bb{R}^+$ tal que $|a|^n\leq M$ para todo $n\in \bb{N}$. Entonces:
            \begin{align*}
                |a|^n \leq M \iff n\ln|a|\leq \ln M &\iff n\leq \dfrac{\ln M}{\ln|a|}
            \end{align*}
            Tomando $N=\left\lceil\dfrac{\ln M}{\ln|a|}+1\right\rceil$, tenemos que $|a|^N\geq M$, lo que contradice la suposición. Por tanto, la sucesión no está mayorada.

            Por tanto, por la Fórmula de Cauchy-Hadamard, tenemos que:
            \begin{equation*}
                R = 0
            \end{equation*}
        \end{itemize}
    \end{enumerate}
\end{ejercicio}

\begin{ejercicio}
    Conocido el radio de convergencia $R$ de la serie $\displaystyle \sum_{n \geq 0} \alpha_nz^n$, calcular el de las siguientes:
    \begin{enumerate}
        \item $\displaystyle \sum_{n \geq 0} n^k\alpha_nz^n$ con $k \in \mathbb{N}$ fijo.
        
        \begin{description}
            \item[Opción 1. Distinguir casos]~
            
            Definimos la sucesión $\{\beta_n\}$:
        \[
            \beta_n = n^k\alpha_n\qquad \forall n\in \bb{N}
        \]

        Sea $\wt{R}$ el radio de convergencia de la serie a estudiar. Distinguimos en función de los valores de $R$:
        \begin{itemize}
            \item Si $R=0$, tenemos que la sucesión $\left\{\sqrt[n]{|\alpha_n|}\right\}$ no está mayorada. Por tanto, la sucesión:
            \begin{equation*}
                \left\{\sqrt[n]{|\beta_n|}\right\} = \left\{\sqrt[n]{n^k|\alpha_n|}\right\}
                = \left\{\sqrt[n]{n^k}\sqrt[n]{|\alpha_n|}\right\}
            \end{equation*}

            Supongamos ahora que la sucesión $\left\{\sqrt[n]{|\beta_n|}\right\}$ está mayorada; es decir, que existe $M\in \bb{R}^+$ tal que $\sqrt[n]{n^k|\alpha_n|}\leq M$ para todo $n\in \bb{N}$. Entonces, para todo $n\geq 1$:
            \begin{align*}
                \sqrt[n]{n^k|\alpha_n|}\leq M &\iff \sqrt[n]{|\alpha_n|}\leq \dfrac{M}{\sqrt[n]{n^k}}\leq M\iff \sqrt[n]{n^k}\geq 1\iff n^k\geq 1
            \end{align*}

            Por tanto, llegamos a que la sucesión $\left\{\sqrt[n]{|\alpha_n|}\right\}$ está mayorada, lo que contradice la hipótesis. Por tanto, la sucesión $\left\{\sqrt[n]{|\beta_n|}\right\}$ no está mayorada, por lo que:
            \begin{equation*}
                \wt{R} = R=0
            \end{equation*}

            \item Si $R=\infty$, tenemos que $\left\{\sqrt[n]{|\alpha_n|}\right\}\to 0$. Calculemos en primer lugar el límite de la sucesión $\left\{\sqrt[n]{n^k}\right\}$ usando el criterio del cociente para sucesiones:
            \begin{align*}
                \left\{\dfrac{(n+1)^k}{n^k}\right\} &= \left\{\left(1+\dfrac{1}{n}\right)^k\right\}\to 1^k = 1
            \end{align*}

            Por tanto, tenemos que $\left\{\sqrt[n]{n^k}\right\}\to 1$. Por tanto, la sucesión $\left\{\sqrt[n]{|\beta_n|}\right\}$ es:
            \begin{equation*}
                \left\{\sqrt[n]{|\beta_n|}\right\} = \left\{\sqrt[n]{n^k}\sqrt[n]{|\alpha_n|}\right\}\to 1\cdot 0 = 0
            \end{equation*}

            Por tanto, por la Fórmula de Cauchy-Hadamard, tenemos que:
            \begin{equation*}
                \wt{R} = \infty = R
            \end{equation*}

            \item Si $R\in \bb{R}^+$, tenemos que $\limsup\left\{\sqrt[n]{|\alpha_n|}\right\} = \nicefrac{1}{R}$.
            
            Aunque sí bien es cierto que el límite superior del producto de dos sucesiones acotadas no tiene por qué ser el producto de los límites superiores, si una de las sucesiones es convergente, entonces sí se cumple\footnote{Concepto que no demostramos por ser materia de Cálculo I.}. Por tanto, tenemos que:
            \begin{align*}
                \limsup\left\{\sqrt[n]{|\beta_n|}\right\} &= \limsup\left\{\sqrt[n]{n^k}\sqrt[n]{|\alpha_n|}\right\}
                =\\&= \lim_{n \to \infty} \sqrt[n]{n^k}\cdot \limsup\left\{\sqrt[n]{|\alpha_n|}\right\} = 1\cdot \dfrac{1}{R} = \dfrac{1}{R}
            \end{align*}

            Por tanto, por la Fórmula de Cauchy-Hadamard, tenemos que:
            \begin{equation*}
                \wt{R} = \dfrac{1}{\limsup\left\{\sqrt[n]{|\beta_n|}\right\}} = R
            \end{equation*}
        \end{itemize}

        \item[Opción 2. Emplear un lema teórico]~
        
        Mediante inducción, demostraremos que, para cada $k\in \bb{N}\cup \{0\}$, la siguiente serie tiene radio de convergencia $R$:
        \begin{align*}
            \sum_{n \geq 1} n^k\alpha_nz^n
        \end{align*}
       
        Comprobemos entonces dicha inducción:
        \begin{itemize}
            \item \ul{Caso base:} $k=0$. La serie a estudiar es la de partida (a excepción del primer término), por lo que el radio de convergencia es $R$.
            \item \ul{Hipótesis de inducción:} Supongamos que la siguiente serie tiene radio de convergencia $R$:
            \begin{align*}
                \sum_{n \geq 1} n^k\alpha_nz^n
            \end{align*}

            \item \ul{Paso inductivo:} Demostrémoslo para $k+1$. Por el Lema del Radio de Convergencia de la Serie derivada término a término, tenemos que el radio de la serie siguiente es $R$:
            \begin{equation*}
                \sum_{n \geq 1} n\cdot n^k\alpha_nz^{n-1} = \sum_{n \geq 1} n^{k+1}\alpha_nz^{n-1}
            \end{equation*}

            Al multiplicar el término general de una serie por un número $z\in \bb{C}^*$, el radio de convergencia se mantiene, puesto que:
            \begin{align*}
                \{\rho\in \bb{R}^+ : \{|\alpha_n|\rho^n\}\ \text{está acotada}\} &= \{\rho\in \bb{R}^+ : \{\red{|z|}|\alpha_n|\rho^n\}\ \text{está acotada}\}
            \end{align*}

            Por tanto, el radio de convergencia de la siguiente serie es $R$:
            \begin{equation*}
                \sum_{n \geq 1} n^{k+1}\alpha_nz^n
            \end{equation*}
        \end{itemize}
        
        Por tanto, por inducción, tenemos que el radio de convergencia de la serie a estudiar es $R$ para todo $k\in \bb{N}\cup \{0\}$.
        \end{description}

        \item $\displaystyle \sum_{n \geq 0} \dfrac{\alpha_n}{n!}z^n$
        
        Definimos la sucesión $\{\beta_n\}$:
        \[
            \beta_n = \dfrac{\alpha_n}{n!}\qquad \forall n\in \bb{N}
        \]

        Sea $\wt{R}$ el radio de convergencia de la serie a estudiar. Distinguimos en función de los valores de $R$:
        \begin{itemize}
            \item Si $R\in \bb{R}^+\cup \{\infty\}$, tenemos que la sucesión $\left\{\sqrt[n]{|\alpha_n|}\right\}$ está mayorada. Calculamos ahora el siguiente límite:
            \begin{align*}
                \lim_{n \to \infty} \dfrac{\nicefrac{1}{(n+1)!}}{\nicefrac{1}{n!}}
                &= \lim_{n \to \infty} \dfrac{n!}{(n+1)!}
                = \lim_{n \to \infty} \dfrac{1}{n+1} = 0
            \end{align*}

            Por tanto, por el Criterio del Cociente para sucesiones, tenemos que:
            \begin{equation*}
                \left\{\sqrt[n]{\dfrac{1}{n!}}\right\}\to 0
            \end{equation*}

            Por tanto, tenemos que:
            \begin{equation*}
                \left\{\sqrt[n]{|\beta_n|}\right\} = \left\{\sqrt[n]{\dfrac{|\alpha_n|}{n!}}\right\}
                = \left\{\sqrt[n]{|\alpha_n|}\sqrt[n]{\dfrac{1}{n!}}\right\}\to 0
            \end{equation*}

            Por tanto, por la Fórmula de Cauchy-Hadamard, tenemos que:
            \begin{equation*}
                \wt{R} = \infty
            \end{equation*}

            \item Si $R=0$, tenemos que la sucesión $\left\{\sqrt[n]{|\alpha_n|}\right\}$ no está mayorada. Por tanto, no podemos garantizar nada sobre $\wt{R}$, puesto que pueden darse las tres casuísticas. Veámoslo:
            \begin{itemize}
                \item Si $\alpha_n=(n!)^2$, tenemos que:
                \begin{equation*}
                    \sqrt[n]{|\beta_n|} = \sqrt[n]{\dfrac{(n!)^2}{n!}} = \sqrt[n]{n!}
                \end{equation*}

                Empleamos ahora el criterio del cociente para sucesiones:
                \begin{equation*}
                    \left\{\dfrac{(n+1)!}{n!}\right\}= \left\{(n+1)\right\}
                \end{equation*}
                Como la sucesión $\left\{(n+1)\right\}$ diverge positivamente, entonces la sucesión $\left\{\sqrt[n]{\beta_n}\right\}= \left\{\sqrt[n]{n!}\right\}$ también diverge positivamente. Por tanto, por la Fórmula de Cauchy-Hadamard, tenemos que:
                \begin{equation*}
                    \wt{R} = 0
                \end{equation*}

                \item Fijado $\lm\in \bb{R}^+$, si $\alpha_n=\lm^nn!$, tenemos que:
                \begin{equation*}
                    \left\{\sqrt[n]{|\beta_n|}\right\} = \left\{\sqrt[n]{\dfrac{\lm^nn!}{n!}}\right\} = \left\{\sqrt[n]{\lm^n}\right\} = \left\{\lm\right\}\to \lm
                \end{equation*}

                Por tanto, por la Fórmula de Cauchy-Hadamard, tenemos que:
                \begin{equation*}
                    \wt{R} = \dfrac{1}{\lm}
                \end{equation*}

                \item Si $\alpha_n=\sqrt{n!}$, tenemos que:
                \begin{equation*}
                    \left\{\sqrt[n]{|\beta_n|}\right\} = \left\{\sqrt[n]{\dfrac{\sqrt{n!}}{n!}}\right\} = \left\{\sqrt[n]{\dfrac{1}{\sqrt{n!}}}\right\}
                    = \left\{\dfrac{1}{\sqrt[2n]{n!}}\right\}\to 0
                \end{equation*}

                Por tanto, por la Fórmula de Cauchy-Hadamard, tenemos que:
                \begin{equation*}
                    \wt{R} = \infty
                \end{equation*}
            \end{itemize}
        \end{itemize}
    \end{enumerate}
\end{ejercicio}

\begin{ejercicio}
    Caracterizar las series de potencias que convergen uniformemente en todo el plano.\\

    Fijado $a\in \bb{C}$, definimos la siguiente función para cada $n\in \bb{N}\cup \{0\}$:
    \Func{f_n}{\bb{C}}{\bb{C}}{z}{\alpha_n(z-a)^n}

    Consideramos ahora la serie de potencias $\displaystyle \sum_{n \geq 0}f_n$. Definimos el siguiente conjunto:
    \begin{equation*}
        A=\left\{n\in \bb{N}\cup \{0\}\mid \alpha_n\neq 0\right\}
    \end{equation*}

    Veamos que la serie converge uniformemente en todo el plano si y solo si $A$ es finito.
    \begin{description}
        \item[$\Longrightarrow)$] Por el recíproco, supongamos que $A$ es infinito; y veamos que la serie no converge uniformemente en todo el plano. Para ello, comprobaremos que el término general de la serie no converge uniformemente a la función nula en todo el plano.
        
        Por reducción al absurdo, supongamos que el término general de la serie converge uniformemente a la función $f$ nula en todo el plano.
        Consideramos la siguiente sucesión:
        \begin{equation*}
            \begin{cases}
                z_n = 0 & \text{si } n\notin A\\
                z_n = a+\left(\dfrac{1}{\alpha_n}\right)^{\nicefrac{1}{n}} & \text{si } n\in A
            \end{cases}
        \end{equation*}

        Por tanto, tenemos que:
        \begin{equation*}
            f_n(z_n)=\alpha_n(z_n-a)^n = \alpha_n\left(\left(\dfrac{1}{\alpha_n}\right)^{\nicefrac{1}{n}}\right)^n = \alpha_n\left(\dfrac{1}{\alpha_n}\right) = 1 \qquad \forall n\in A
        \end{equation*}

        Por tanto, para todo $n\in A$, tenemos que:
        \begin{equation*}
            f_n(z_n)-f(z_n) = 1-0 = 1
        \end{equation*}

        Como $A$ es infinito, entonces tenemos que $\{f_n(z_n)-f(z_n)\}$ no converge puntualmente a la función nula en todo el plano, por lo que hemos llegado a una contradicción y el término general de la serie no converge uniformemente a la función nula en todo el plano. Por tanto, la serie no converge uniformemente en todo el plano.

        \item[$\Longleftarrow)$] Supongamos ahora que $A$ es finito. Si $A=\emptyset$, entonces se tiene trivialmente la convergencia uniforme de la serie en todo el plano (a la función nula). Supongamos ahora que $A\neq \emptyset$. Sea entonces $m=1+\max A$ (podemos considerar el máximo, puesto que es finito). Por tanto:
        \begin{equation*}
            \forall \veps\in \bb{R}^+\qquad n\geq m\Longrightarrow \left|\sum_{k=n}^\infty f_k(z)\right| = \left|\sum_{k=n}^\infty 0\cdot (z-a)^k\right| = 0 < \veps\qquad \forall z\in \bb{C}
        \end{equation*}

        Por tanto, la serie converge uniformemente en todo el plano.
    \end{description}
\end{ejercicio}

\begin{ejercicio}
    Estudiar la convergencia puntual, absoluta y uniforme, de la serie $\displaystyle \sum_{n \geq 0} f_n$ donde:
    \[
        f_n(z) = \left(\dfrac{z-1}{z+1}\right)^n \quad \text{para todo } z \in \mathbb{C}\setminus\{-1\}
    \]

    %\begin{description}
    %    \item[Opción 1. Uso del Conocimiento de la Serie Geométrica]~
        
        Definimos la siguiente función auxiliar:
        \Func{\varphi}{\bb{C}\setminus\{-1\}}{\bb{C}}{z}{\dfrac{z-1}{z+1}}

        Por tanto, tenemos que nuestra serie a estudiar es:
        \begin{equation*}
            \sum_{n \geq 0} \left(\varphi(z)\right)^n
        \end{equation*}

        Estudiamos en primer lugar la convergencia absoluta de la serie geométrica de razón $\varphi(z)$. Sabemos que converge absolutamente (y por tanto puntualmente) en cualquier $z\in \bb{C}$ tal que $\varphi(z)\in D(0,1)$, mientras que no converge (ni puntualmente) en cualquier $z\in \bb{C}$ tal que $\varphi(z)\notin D(0,1)$. Tenemos que:
        \begin{align*}
            \varphi(z)\in D(0,1) &\iff |\varphi(z)|<1
            \iff |z-1|<|z+1|\iff |z-1|^2<|z+1|^2\iff\\&\iff (z-1)(\ol{z}-1)<(z+1)(\ol{z}+1)
            \iff\\&\iff |z|^2+1 -2\Re(z)<|z|^2+1+2\Re(z)\iff \Re(z)>0
        \end{align*}

        Por tanto, definimos el siguiente conjunto:
        \begin{equation*}
            H = \{z\in \bb{C}\mid \Re(z)>0\}
        \end{equation*}

        Por tanto, la serie converge absolutamente (y por tanto puntualmente) en $H$, y no converge (ni puntualmente) en $\bb{C}\setminus H$.\\

        Estudiamos ahora la convergencia uniforme de la serie, pudiendo hacerlo de dos formas distintas: 
        \begin{description}
            \item [Opción 1.] 
                Razonemos en primer lugar sobre compactos. Sea $K\subset H$ compacto. Por ser $\varphi$ continua, tenemos que $\varphi(K)\subset D(0,1)$ es compacto, por lo que la serie converge uniformemente en $K$.\\
                
                Supongamos ahora $\emptyset\neq A\subset H$ no necesariamente compacto, y supongamos que la serie converge uniformemente en $A$. Por ser condición necesaria, tenemos que la sucesión $\left\{(\varphi(z))^n\right\}$ converge uniformemente a la función nula en $A$. Por los conocimientos sobre el término general de una serie geométrica, como esta converge uniformemente a la función nula en $A$ tenemos que $r<1$, donde $r$ se define como:
                \begin{equation*}
                    r=\sup\{|\varphi(z)|\mid z\in A\}<1
                \end{equation*}

                Por tanto, $\varphi(A)\subset \ol{D}(0,r)$. Veamos ahora que $\varphi$ es inyectiva en $H$. Para ello, consideramos dos elementos $z_1,z_2\in H$ tales que $\varphi(z_1)=\varphi(z_2)$. Entonces:
                \begin{align*}
                    \varphi(z_1)=\varphi(z_2) &\iff 
                    \dfrac{z_1-1}{z_1+1} = \dfrac{z_2-1}{z_2+1}
                    \iff (z_1-1)(z_2+1) = (z_2-1)(z_1+1)
                    \iff\\&\iff  z_1z_2+z_1-z_2-1 = z_1z_2+z_2-z_1-1
                    \iff\\&\iff z_1-z_2 = z_2-z_1
                    \iff z_1 = z_2
                \end{align*}

                Por tanto, tomámos imágenes inversas, y llegamos a que:
                \begin{equation*}
                    A\subset \varphi^{-1}\left(\ol{D}(0,r)\right) = \left\{z\in \bb{C}\mid |\varphi(z)|\leq r\right\}
                \end{equation*}

                Veamos cómo es este conjunto. Tenemos que:
                \begin{align*}
                    |\varphi(z)|\leq r &\iff \left|\dfrac{z-1}{z+1}\right|\leq r
                    \iff |z-1|^2\leq r^2|z+1|^2
                    \iff\\&\iff |z|^2+1-2\Re(z)\leq r^2(|z|^2+1+2\Re(z))
                    \iff\\&\iff (1-r^2)|z|^2+(1-r^2)-2\Re(z)(1+r^2)\leq 0
                    \iff\\&\iff |z|^2+1-2\Re(z)\cdot \dfrac{1+r^2}{1-r^2}\leq 0
                \end{align*}

                Consideramos ahora $z=x+iy\in \bb{C}$, y tenemos que:
                \begin{align*}
                    |\varphi(x+iy)|\leq r &\iff x^2+y^2+1-2x\cdot \dfrac{1+r^2}{1-r^2}\leq 0
                    \iff\\&\iff x^2-2x\cdot \dfrac{1+r^2}{1-r^2} + \left(\dfrac{1+r^2}{1-r^2}\right)^2-\left(\dfrac{1+r^2}{1-r^2}\right)^2+y^2+1\leq 0
                    \iff\\&\iff \left(x-\dfrac{1+r^2}{1-r^2}\right)^2+y^2\leq \left(\dfrac{1+r^2}{1-r^2}\right)^2-1
                \end{align*}

                Por tanto, tenemos que:
                \begin{equation*}
                    A\subset D:=\ol{D}\left(\dfrac{1+r^2}{1-r^2},\sqrt{\left(\dfrac{1+r^2}{1-r^2}\right)^2-1}\right)
                \end{equation*}
                Este conjunto está bien definido puesto que:
                \begin{equation*}
                    \left(\dfrac{1+r^2}{1-r^2}\right)^2-1>0\iff 1+r^2>1-r^2\iff r>0
                \end{equation*}

                Veamos ahora que $D\subset H$. Para ello, consideramos $z\in D$, y tenemos que:
                \begin{align*}
                    z\in D &\Longrightarrow
                    \left|z-\dfrac{1+r^2}{1-r^2}\right|<\sqrt{\left(\dfrac{1+r^2}{1-r^2}\right)^2-1}
                    \Longrightarrow \\ &\Longrightarrow
                    |z|^2+\cancel{\left(\dfrac{1+r^2}{1-r^2}\right)^2}-2\cdot \dfrac{1+r^2}{1-r^2}\cdot \Re(z)<\cancel{\left(\dfrac{1+r^2}{1-r^2}\right)^2}-1
                    \Longrightarrow \\ &\Longrightarrow
                    \Re(z)>\left(|z|^2+1\right)\cdot \dfrac{1-r^2}{1+r^2}\cdot \dfrac{1}{2}>0
                    \Longrightarrow z\in H
                \end{align*}

                Por tanto, hemos llegado a que $A\subset D\subset H$, siendo $D$ compacto. Por tanto, supuesto que la serie converge uniformemente en $A$, hemos llegado a que $A$ está contenido en un compacto (en el cual ya sabíamos que la serie converge uniformemente). Por tanto, tenemos que, dado $A\subset H$:
                \begin{equation*}
                    \sum_{n \geq 0} f_n\ \text{converge uniformemente en } A\iff \exists K\subset H\ \text{compacto tal que } A\subset K
                \end{equation*}
            \item [Opción 2.] Como sabemos que:
                \begin{equation*}
                    \sum_{n\geq 0} z^n \text{\ converge uniformemente en\ } B\subseteq D(0,1) \Longleftrightarrow \rho = \sup\{|z| \mid z\in B\} < 1
                \end{equation*}
                Vamos a tratar de probar que:
                \begin{equation*}
                    \sum_{n\geq 0} f_n \text{\ converge uniformemente en\ } B \subseteq H \Longleftrightarrow \rho = \sup\{|\varphi(z)| \mid z\in B\} < 1
                \end{equation*}
                \begin{description}
                    \item [$\Longleftarrow)$] Supuesto que $\rho < 1$, es fácil probar la convergencia uniforme de la serie en $B$:
                        \begin{equation*}
                            |f_n(z)| = {|\varphi(z)|}^{n} \leq \rho^n \qquad \forall z\in B, n\in \mathbb{N}
                        \end{equation*}
                        Y sabemos que $\sum\limits_{n\geq 0}\rho^n$ converge por ser $\rho < 1$. Por el Test de Weierstrass, tenemos que $\sum\limits_{n\geq 0}f_n$ converge uniformemente en $B$.
                    \item [$\Longrightarrow)$] Supuesto que $\sum\limits_{n\geq 0}f_n$ converge uniformemente en un conjunto $B\subseteq H$, por reducción al absurdo, supongamos que $\rho \geq 1$, en cuyo caso (por la definición de supremo), tendremos la existencia de una sucesión $\{|\varphi(z_n)|\}$ con $z_n \in B$ para todo $n\in \mathbb{N}$ de forma que:
                        \begin{equation*}
                            \{|\varphi(z_n)|\} \to \rho \geq 1
                        \end{equation*}
                        En cuyo caso, para dicha sucesión tendremos que:
                        \begin{equation*}
                            |f_n(z_n)| = {|\varphi(z_n)|}^{n}
                        \end{equation*}
                        Y esta sucesión no podrá converger a 0, por ser $\rho \geq 1$, lo que contradice que la serie $\sum\limits_{n\geq 0} f_n$ converja uniformemente en $B$, por no converger uniformemente su término general a 0 en $B$.
                \end{description}
        \end{description}

        %\item[Opción 2. Uso del Test de Weierstrass]~
    %\end{description}
\end{ejercicio}

\section{Singularidades}

\begin{ejercicio}
    Dados $a, b \in \bb{C}$ con $0 < |a| < |b|$, obtener el desarrollo en serie de Laurent de la función $f$ definida por:
    \Func{f}{\bb{C} \setminus \{a, b\}}{\bb{C}}{z}{\frac{1}{(z - a)(z - b)}}
    en cada uno de los anillos siguientes:
    \begin{enumerate}
        \item $A(0; |a|, |b|)$
        
        En primer lugar, descomponemos la función en fracciones simples:
        \begin{align*}
            \dfrac{1}{(z - a)(z - b)} &= \dfrac{A}{z - a} + \dfrac{B}{z - b}
            = \dfrac{A(z - b) + B(z - a)}{(z - a)(z - b)}
        \end{align*}
        \begin{itemize}
            \item Para $z=a$, tenemos que $1=A(a - b)$.
            \item Para $z=b$, tenemos que $1=B(b - a)$.
        \end{itemize}

        Por tanto:
        \begin{align*}
            \dfrac{1}{(z - a)(z - b)} &= \dfrac{1}{a - b} \left( \dfrac{1}{z - a} - \dfrac{1}{z - b} \right)
        \end{align*}

        Notemos que, en este caso, la serie debe ser convergente para todo $z\in \bb{C}$ tal que $|a| < |z| < |b|$. Trabajamos con cada fracción por separado, buscando en expresarla de forma que pueda ser la suma de una serie geométrica (de razón con módulo menor que 1).

        Para la primera fracción, puesto que $|a| < |z|$, buscamos que la razón de la serie geométrica sea $\nicefrac{a}{z}$:
        \begin{align*}
            \dfrac{1}{z - a} &= \dfrac{1}{z(1 - \nicefrac{a}{z})} = \dfrac{1}{z} \cdot \sum_{n=0}^{\infty} \left( \frac{a}{z} \right)^n = \sum_{n=0}^{\infty} \frac{a^n}{z^{n+1}} = \sum_{n=1}^{\infty} \frac{a^{n-1}}{z^n}
        \end{align*}

        Para la segunda fracción, puesto que $|z| < |b|$, buscamos que la razón de la serie geométrica sea $\nicefrac{z}{b}$:
        \begin{align*}
            -\dfrac{1}{z - b} &= -\dfrac{1}{b(\nicefrac{z}{b} - 1)} = \dfrac{1}{b(1 - \nicefrac{z}{b})} = \dfrac{1}{b}\cdot \sum_{n=0}^{\infty} \left( \frac{z}{b} \right)^n = \sum_{n=0}^{\infty} \frac{z^n}{b^{n+1}}
        \end{align*}

        Por tanto, el desarrollo en serie de Laurent de $f$ en el anillo $A(0; |a|, |b|)$ es:
        \begin{align*}
            f(z) &= \dfrac{1}{a - b} \left( \sum_{n=1}^{\infty} \frac{a^{n-1}}{z^n} + \sum_{n=0}^{\infty} \frac{z^n}{b^{n+1}} \right) =\\&= \sum_{n\in \bb{Z}} c_n z^n
            \quad \text{con } c_n = \begin{cases}
                \dfrac{1}{a - b} \cdot a^{1-n} & \text{si } n < 0\\
                \dfrac{1}{a - b} \cdot \dfrac{1}{b^{n+1}} & \text{si } n \geq 0
            \end{cases}
        \end{align*}

        \item $A(0; |b|, \infty)$
        
        En este caso, tenemos que $|z| > |b|>|a|$, por lo que:
        \begin{align*}
            \dfrac{1}{z-a} &= \sum_{n=1}^{\infty} \frac{a^{n-1}}{z^n}\\
            \dfrac{1}{z-b} &= \sum_{n=1}^{\infty} \frac{b^{n-1}}{z^n}
        \end{align*}

        Por tanto, el desarrollo en serie de Laurent de $f$ en el anillo $A(0; |b|, \infty)$ es:
        \begin{align*}
            f(z) &= \dfrac{1}{a - b} \left( \sum_{n=1}^{\infty} \frac{a^{n-1}}{z^n} - \sum_{n=1}^{\infty} \frac{b^{n-1}}{z^n} \right)
            = \dfrac{1}{a - b} \sum_{n=1}^{\infty} \left( a^{n-1} - b^{n-1} \right) z^{-n}
            =\\&= \sum_{n\in \bb{Z}} c_n z^n
            \quad \text{con } c_n = \begin{cases}
                \dfrac{1}{a - b} \cdot (a^{n-1} - b^{n-1}) & \text{si } n < 0\\
                0 & \text{si } n\geq 0
            \end{cases}
        \end{align*}
        \item $A(a; 0, |b - a|)$
        
        En este caso, notemos que la primera fracción ya está de forma directa en la forma que buscamos. Para la segunda fracción, tenemos que $|z - a| < |b - a|$, por lo que:
        \begin{align*}
            \dfrac{1}{z-b} &= \dfrac{1}{z-a+a-b}
            = \dfrac{1}{(a-b)\left(\dfrac{z-a}{a-b} + 1\right)}
            = \dfrac{1}{(a-b)\left(1-\dfrac{z-a}{b-a}\right)}
            =\\&= \sum_{n=0}^{\infty} \frac{1}{(a-b)} \left(\frac{z-a}{b-a}\right)^n
        \end{align*}

        Por tanto, el desarrollo en serie de Laurent de $f$ en el anillo $A(a; 0, |b - a|)$ es:
        \begin{align*}
            f(z) &= \dfrac{1}{a - b} \left( \dfrac{1}{z - a} + \sum_{n=0}^{\infty} \frac{1}{(a-b)} \left(\frac{z-a}{b-a}\right)^n \right)
            =\\&=
            \dfrac{1}{(a-b)(z - a)} - \sum_{n=0}^{\infty} \frac{1}{(b-a)^{n+2}} \left(z-a\right)^n
            =\\&= \sum_{n\in \bb{Z}} c_n (z-a)^n
            \quad \text{con } c_n = \begin{cases}
                \dfrac{1}{(a-b)} & \text{si } n=-1\\
                0 & \text{si } n < -1\\
                -\dfrac{1}{(b-a)^{n+2}} & \text{si } n \geq 0
            \end{cases}
        \end{align*}
        \item $A(a; |b - a|, \infty)$
        
        En este caso, notemos que la primera fracción ya está de forma directa en la forma que buscamos. Para la segunda fracción, tenemos que $|z - a| > |b - a|$, por lo que:
        \begin{align*}
            \dfrac{1}{z-b} &= \dfrac{1}{z-a+a-b}
            = \dfrac{1}{(z-a)\left(1+\dfrac{a-b}{z-a}\right)}
            = \dfrac{1}{(z-a)} \cdot \dfrac{1}{1-\dfrac{b-a}{z-a}}
            =\\&= \dfrac{1}{(z-a)} \cdot \sum_{n=0}^{\infty} \left(\dfrac{b-a}{z-a}\right)^n
            = \sum_{n=0}^{\infty} \frac{(b-a)^n}{(z-a)^{n+1}}
            = \sum_{n=1}^{\infty} \frac{(b-a)^{n-1}}{(z-a)^n}
        \end{align*}

        Por tanto, el desarrollo en serie de Laurent de $f$ en el anillo $A(a; |b - a|, \infty)$ es:
        \begin{align*}
            f(z) &= \dfrac{1}{a - b} \left( \dfrac{1}{z - a} - \sum_{n=1}^{\infty} \frac{(b-a)^{n-1}}{(z-a)^n} \right)
            =\\&= \dfrac{1}{(a-b)(z - a)} + \sum_{n=1}^{\infty} \frac{(b-a)^{n-2}}{(z-a)^{n}} = \sum_{n=2}^{\infty} \frac{(b-a)^{n-2}}{(z-a)^{n}}
            =\\&= \sum_{n\in \bb{Z}} c_n (z-a)^n
            \quad \text{con } c_n = \begin{cases}
                0 & \text{si } n \geq -1\\
                (b-a)^{-n-2} & \text{si } n \leq -2
            \end{cases}
        \end{align*}
    \end{enumerate}
\end{ejercicio}

\begin{ejercicio}
    Obtener el desarrollo en serie de Laurent de la función
    \Func{f}{\bb{C} \setminus \{1, -1\}}{\bb{C}}{z}{\frac{1}{(z^2 - 1)^2}}
    en los anillos:
    \begin{enumerate}
        \item $A(1; 0, 2)$
        
        Como $f$ es holomorfa en dicho anillo, admite un desarrollo en Serie de Laurent en dicho anillo. Descomponemos $f$ en fracciones simples:
        \begin{align*}
            \dfrac{1}{(z^2 - 1)^2} &= \dfrac{A}{z - 1} + \dfrac{B}{(z - 1)^2} + \dfrac{C}{z + 1} + \dfrac{D}{(z + 1)^2}
            =\\&= \dfrac{A(z -1)(z + 1)^2 + B(z + 1)^2 + C(z +1)(z - 1)^2 + D(z - 1)^2}{(z - 1)^2(z + 1)^2}
        \end{align*}
        \begin{itemize}
            \item Para $z=1$, tenemos que $1 = 4B$.
            \item Para $z=-1$, tenemos que $1 = 4D$.
            \item Igualando los coeficientes de $z^3$, tenemos que $A + C = 0$.
            \item Para $z=0$, tenemos que $1 = -A + B + C + D$.
        \end{itemize}

        Resolvemos por tanto el sistema:
        \begin{align*}
            \left.\begin{array}{rcl}
                A + C &=& 0\\
                -4A + 1 + 4C + 1 &=& 4
            \end{array}\right\}
            \Rightarrow
            \left\{\begin{array}{rcl}
                A + C &=& 0\\
                -A + C &=& \nicefrac{1}{2}
            \end{array}\right\}
            \Rightarrow
            \left\{\begin{array}{rcl}
                A &=& -\nicefrac{1}{4}\\
                C &=& \nicefrac{1}{4}
            \end{array}\right.
        \end{align*}

        Por tanto:
        \begin{align*}
            f(z) = \dfrac{1}{4}\left[-\dfrac{1}{z-1} + \dfrac{1}{(z-1)^2} + \dfrac{1}{z+1} + \dfrac{1}{(z+1)^2}\right]
        \end{align*}

        Las dos primeras fracciones ya están en la forma que buscamos. Para la segunda, hacemos uso de que $|z-1| < 2$ para escribir:
        \begin{align*}
            \dfrac{1}{z+1} &= \dfrac{1}{z-1+1+1}
            = \dfrac{1}{2\left(1 + \dfrac{z-1}{2}\right)}
            = \dfrac{1}{2} \cdot \sum_{n=0}^{\infty} \left(-\frac{z-1}{2}\right)^n
        \end{align*}

        Para la última fracción, realizamos antes los siguientes cálculos. Para cada $w\in D(0,1)$, tenemos que:
        \begin{align*}
            \dfrac{1}{1-w} &= \sum_{n=0}^{\infty} w^n
            \Longrightarrow
            \dfrac{1}{(1-w)^2} = \sum_{n=1}^{\infty} nw^{n-1}
        \end{align*}

        En nuestro caso particular, como $|z-1| < 2$, tenemos que:
        \begin{align*}
            \dfrac{1}{(z+1)^2} &= \dfrac{1}{(z-1+1+1)^2}
            = \dfrac{1}{4\left(1 + \dfrac{z-1}{2}\right)^2}
            = \dfrac{1}{4} \cdot \sum_{n=1}^{\infty} n \left(-\frac{z-1}{2}\right)^{n-1}
            =\\&= \dfrac{1}{4} \sum_{n=0}^{\infty} (n+1) \left(-\frac{z-1}{2}\right)^{n}
        \end{align*}

        Por tanto, el desarrollo en serie de Laurent de $f$ en el anillo $A(1; 0, 2)$ es:
        \begin{align*}
            f(z) &= \dfrac{1}{4}\left[-\dfrac{1}{z-1} + \dfrac{1}{(z-1)^2} + \dfrac{1}{2} \sum_{n=0}^{\infty} \left(-\frac{z-1}{2}\right)^n + \dfrac{1}{4} \sum_{n=0}^{\infty} (n+1) \left(-\frac{z-1}{2}\right)^{n}\right]
            =\\&= \dfrac{1}{4}\left[-\dfrac{1}{z-1} + \dfrac{1}{(z-1)^2} + \sum_{n=0}^{\infty} \left(\dfrac{1}{2} + \left(-\frac{1}{2}\right)^n + \frac{n+1}{4}\left(-\frac{1}{2}\right)^n\right)(z-1)^n\right]
            =\\&= \dfrac{1}{4}\left[-\dfrac{1}{z-1} + \dfrac{1}{(z-1)^2} + \sum_{n=0}^{\infty} \left(\dfrac{1}{2} + \frac{n+5}{4}\left(-\frac{1}{2}\right)^n\right)(z-1)^n\right]
            =\\&= \sum_{n\in \bb{Z}} c_n (z-1)^n
            \quad \text{con } c_n = \begin{cases}
                \nicefrac{-1}{4} & \text{si } n = -1\\
                \nicefrac{1}{4} & \text{si } n = -2\\
                \dfrac{1}{8} + \dfrac{n+5}{16}\left(-\dfrac{1}{2}\right)^n & \text{si } n \geq 0
            \end{cases}
        \end{align*}
        \item $A(1; 2, \infty)$
        
        Las dos primeras fracciones ya están en la forma que buscamos.
        Para la tercera, tenemos que $|z-1| > 2$, por lo que:
        \begin{align*}
            \dfrac{1}{z+1} &= \dfrac{1}{z-1+1+1}
            = \dfrac{1}{(z-1)\left(1+\dfrac{2}{z-1}\right)}
            = \dfrac{1}{(z-1)} \cdot \sum_{n=0}^{\infty} \left(-\frac{2}{z-1}\right)^n
            =\\&= \sum_{n=0}^{\infty} \frac{(-2)^n}{(z-1)^{n+1}}
            = \sum_{n=1}^{\infty} \frac{(-2)^{n-1}}{(z-1)^n}
        \end{align*}

        Para la última fracción, tenemos que:
        \begin{align*}
            \dfrac{1}{(z+1)^2} &= \dfrac{1}{(z-1+1+1)^2}
            = \dfrac{1}{(z-1)^2\left(1+\dfrac{2}{z-1}\right)^2}
            = \dfrac{1}{(z-1)^2} \cdot \sum_{n=0}^{\infty} (n+1)\left(-\frac{2}{z-1}\right)^n
            =\\&= \sum_{n=0}^{\infty} (n+1)(-2)^n(z-1)^{-n-2}
            = \sum_{n=2}^{\infty} (n-1)(-2)^{n-2}(z-1)^{-n}
        \end{align*}

        Por tanto, el desarrollo en serie de Laurent de $f$ en el anillo $A(1; 2, \infty)$ es:
        \begin{align*}
            f(z) &= \dfrac{1}{4}\left[-\dfrac{1}{z-1} + \dfrac{1}{(z-1)^2} + \sum_{n=1}^{\infty} \frac{(-2)^{n-1}}{(z-1)^n} + \sum_{n=2}^{\infty} (n-1)(-2)^{n-2}(z-1)^{-n}\right]
            =\\&= \dfrac{1}{4}\left[-\cancel{\dfrac{1}{z-1}} + \dfrac{1}{(z-1)^2} + \cancel{\frac{1}{z-1}} + \sum_{n=2}^{\infty} \left((-2)^{n-1} + (n-1)(-2)^{n-2}\right)(z-1)^{-n}\right]
            =\\&= \dfrac{1}{4}\left[\dfrac{1}{(z-1)^2} + \sum_{n=2}^{\infty} \left((-2)^{n-1}\left(1 + \frac{n-1}{2}\right)\right)(z-1)^{-n}\right]
            =\\&= \dfrac{1}{4}\left[\dfrac{1}{(z-1)^2} + \sum_{n=2}^{\infty} \left((-2)^{n-1}\left(\frac{n+1}{2}\right)\right)(z-1)^{-n}\right]
            =\\&= \dfrac{1}{4}\left[\dfrac{1}{(z-1)^2} + (-2)\cdot \frac{3}{2} \cdot \frac{1}{(z-1)^2} + \sum_{n=3}^{\infty} \left((-2)^{n-1}\left(\frac{n+1}{2}\right)\right)(z-1)^{-n}\right]
            =\\&= \dfrac{1}{4}\left[\frac{-2}{(z-1)^2} + \sum_{n=3}^{\infty} \left((-2)^{n-1}\left(\frac{n+1}{2}\right)\right)(z-1)^{-n}\right]
            =\\&= \sum_{n\in \bb{Z}} c_n (z-1)^n
            \quad \text{con } c_n = \begin{cases}
                0 & \text{si } n \geq -1\\
                \nicefrac{-1}{2} & \text{si } n = -2\\
                (-2)^{-n-1}\left(\frac{-n+1}{8}\right) & \text{si } n < -2
            \end{cases}
        \end{align*}
        

    \end{enumerate}
\end{ejercicio}

\begin{ejercicio}
    En cada uno de los siguientes casos, clasificar las singularidades de la función $f$ y determinar la parte singular de $f$ en cada una de sus singularidades:
    \begin{enumerate}
        \item $f(z) = \dfrac{1 - \cos z}{z^n}$, $\forall z \in \bb{C}^*\qquad (n \in \bb{N})$
        
        % Como $f\in \cc{H}(\bb{C} \setminus \{0\})$, la única singularidad de $f$ es $z=0$.
        \item $f(z) = z^n \sen(\nicefrac{1}{z})$, $\forall z \in \bb{C}^*\qquad (n \in \bb{N})$
        \item $f(z) = \dfrac{\log(1 + z)}{z^2}$, $\forall z \in \bb{C} \setminus \{0, -1\}$
        \item $f(z) = \dfrac{1}{z(1 - e^{2\pi i z})}$, $\forall z \in \bb{C} \setminus \bb{Z}$
        \item $f(z) = z \tg\left(\dfrac{2\pi z}{2}\right)$, $\forall z \in \bb{C} \setminus \bb{Z}$
    \end{enumerate}
\end{ejercicio}

\begin{ejercicio}
    Sea $\Omega$ un abierto del plano, $a \in \Omega$ y $f \in \cc{H}(\Omega \setminus \{a\})$. ¿Qué relación existe entre las posibles singularidades en el punto $a$ de las funciones $f$ y $f'$?\\

    Consideramos la única descomposición de $f$ en parte regular y parte singular:
    \begin{align*}
        f(z) &= g(z) + h(z)\qquad \forall z \in
        \Omega \setminus \{a\}
    \end{align*}
    con $g\in \cc{H}(\Omega)$ y $h \in \cc{H}(\bb{C} \setminus \{a\})$ con:
    \begin{equation*}
        h(z) = \varphi\left(\dfrac{1}{z - a}\right) \qquad \forall z \in \Omega \setminus \{a\}
    \end{equation*}
    donde $\varphi\in \cc{H}(\bb{C})$ y $\varphi(0) = 0$.\\

    Entonces, la derivada de $f$ es:
    \begin{align*}
        f'(z) &= g'(z) + h'(z) \qquad \forall z \in \Omega \setminus \{a\}
    \end{align*}
    con $g' \in \cc{H}(\Omega)$ y $h' \in \cc{H}(\bb{C} \setminus \{a\})$ con:
    \begin{equation*}
        h'(z) = \varphi'\left(\dfrac{1}{z - a}\right) \cdot \left(-\dfrac{1}{(z - a)^2}\right) \qquad \forall z \in \Omega \setminus \{a\}
    \end{equation*}
    con $\varphi' \in \cc{H}(\bb{C})$. Definimos por tanto $\Phi\in \cc{H}(\Omega)$ como sigue:
    \begin{equation*}
        \Phi(w) = -\varphi'\left(w\right) \cdot w^2 \qquad \forall w \in \bb{C}
    \end{equation*}

    De esta forma, se tiene que:
    \begin{align*}
        h'(z) &= \Phi\left(\dfrac{1}{z - a}\right) \qquad \forall z \in \Omega \setminus \{a\}
    \end{align*}
    Además, $\Phi(0) = 0$, puesto que $\varphi'$ está definida en el origen.
    \begin{enumerate}
        \item Supongamos que $f$ tiene un polo de orden $m > 0$ en $a$ (notemos que $\varphi$ no puede ser constante).
        
        Entonces, $\varphi$ es una función polinómica de grado $m$, y por tanto $\varphi'$ es una función polinómica de grado $m-1$. Por tanto, $\Phi$ es una función polinómica de grado $m+1$, y por tanto $h'$ tiene un polo de orden $m+1$ en $a$.

        \item Supongamos que $f$ tiene una singularidad esencial en $a$.
        
        Entonces, $\varphi$ es una función entera no polinómica, y por tanto $\varphi'$ es una función entera. Sabemos que $\varphi'$ no puede ser polinómica, puesto que si lo fuese, por el Principio de Identidad, $\varphi$ también lo sería. Por tanto, $\Phi$ es una función entera no polinómica, y por tanto $h'$ tiene una singularidad esencial en $a$.
    \end{enumerate}
\end{ejercicio}

\begin{ejercicio}
    Sea $\Omega$ un dominio, $a \in \Omega$ y $f \in \cc{H}(\Omega \setminus \{a\})$ tal que $f(z) \neq 0$ para todo $z \in \Omega \setminus \{a\}$. ¿Qué relación existe entre las posibles singularidades en $a$ de las funciones $f$ y $\nicefrac{1}{f}$?

    \begin{description}
        \item[Definición]~
        
        Consideramos la única descomposición de $f$ en parte regular y parte singular:
        \begin{align*}
            f(z) &= g(z) + h(z)\qquad \forall z \in \Omega \setminus \{a\}
        \end{align*}
        con $g\in \cc{H}(\Omega)$ y $h \in \cc{H}(\bb{C} \setminus \{a\})$ con:
        \begin{equation*}
            h(z) = \varphi\left(\dfrac{1}{z - a}\right) \qquad \forall z \in \Omega \setminus \{a\}
        \end{equation*}
        donde $\varphi\in \cc{H}(\bb{C})$ y $\varphi(0)=0$.\\

        Entonces, la función $\nicefrac{1}{f}$ es:
        \begin{align*}
            \dfrac{1}{f(z)} &= \dfrac{1}{g(z) + h(z)} \qquad \forall z \in \Omega \setminus \{a\}
        \end{align*}

        Como vemos, obetener las partes regular y singular de $\nicefrac{1}{f}$ no es tan sencillo como en el caso de $f'$, por lo que descartamos esta opción.

        \item[Caracterización]~
        
        \begin{enumerate}
            \item Supongamos que $f$ tiene un punto regular en $a$.
            
            Entonces, por la caracterización de los puntos regulares, $\exists g \in \cc{H}(\Omega)$ tal que $f(z) = g(z)$ para todo $z \in \Omega \setminus \{a\}$. Hay dos opciones:
            \begin{itemize}
                \item Si $g(a) \neq 0$, por la continuidad de $g$ tenemos que:
                \begin{equation*}
                    \lim_{z \to a} f(z) = \lim_{z \to a} g(z) = g(a) \neq 0
                \end{equation*}

                Por tanto, tenemos que:
                \begin{equation*}
                    \lim_{z \to a} \dfrac{1}{f(z)} = \lim_{z \to a} \dfrac{1}{g(z)} = \dfrac{1}{g(a)} \neq 0
                \end{equation*}

                Por tanto, $\nicefrac{1}{f}$ tiene un punto regular en $a$.

                \item Si $g(a) = 0$, como $\Omega$ es un dominio, existe $k\in \bb{N}$ tal que $g(z) = (z - a)^k \cdot h(z)$, con $h \in \cc{H}(\Omega)$ y $h(a) \neq 0$. Por tanto, tenemos que:
                \begin{align*}
                    \dfrac{1}{f(z)} &= \dfrac{1}{g(z)} = \dfrac{1}{(z - a)^k \cdot h(z)}\qquad \forall z \in \Omega \setminus \{a\}
                \end{align*}

                Definimos por tanto la siguiente función $\Psi=\nicefrac{1}{h}$, y tenemos que:
                \begin{align*}
                    \dfrac{1}{f(z)} &= \dfrac{\Psi(z)}{(z - a)^k}\qquad \forall z \in \Omega \setminus \{a\}
                \end{align*}
                con $\Psi \in \cc{H}(\Omega)$ y $\Psi(a) \neq 0$. Por tanto, $\nicefrac{1}{f}$ tiene un polo de orden $k$ (el orden de $a$ como cero de $g$) en $a$.
            \end{itemize}

            \item Supongamos que $f$ tiene un polo de orden $m > 0$ en $a$.
            

            Entonces, por la caracterización de los polos, $\exists \Psi \in \cc{H}(\Omega)$ tal que:
            \begin{equation*}
                f(z) = \dfrac{\Psi(z)}{(z - a)^m}\qquad \forall z \in \Omega \setminus \{a\}
            \end{equation*}
            con $\Psi(a) \neq 0$. Por tanto, tenemos que:
            \begin{align*}
                \dfrac{1}{f(z)} &= \dfrac{(z - a)^m}{\Psi(z)}\qquad \forall z \in \Omega \setminus \{a\}
            \end{align*}
            Definimos por tanto $g=\nicefrac{1}{\Psi}$, y tenemos que:
            \begin{align*}
                \dfrac{1}{f(z)} &= g(z) \cdot (z - a)^m\qquad \forall z \in \Omega \setminus \{a\}
            \end{align*}
            con $g \in \cc{H}(\Omega)$ y $g(a) \neq 0$. Por tanto, $\nicefrac{1}{f}$ tiene un cero de orden $m$ (el orden de $a$ como polo de $f$) en $a$.


            \item Supongamos que $f$ tiene una singularidad esencial en $a$.
            \begin{itemize}
                \item Supongamos que $\nicefrac{1}{f}$ tiene un punto regular en $a$. Por lo anteriormente demostrado, $f=\frac{1}{\nicefrac{1}{f}}$ tiene un punto regular en $a$ o un polo de orden $m > 0$ en $a$. En ambos casos, $f$ no puede tener una singularidad esencial en $a$, por lo que no puede ser el caso que $\nicefrac{1}{f}$ tenga un punto regular en $a$.
                
                \item Supongamos que $\nicefrac{1}{f}$ tiene un polo de orden $m > 0$ en $a$. Por lo anteriormente demostrado, $f=\frac{1}{\nicefrac{1}{f}}$ tiene un cero de orden $m$ en $a$. Por tanto, $f$ no puede tener una singularidad esencial en $a$, por lo que no puede ser el caso que $\nicefrac{1}{f}$ tenga un polo de orden $m > 0$ en $a$.
            \end{itemize}

            Por tanto, $\nicefrac{1}{f}$ tiene una singularidad esencial en $a$.
        \end{enumerate}
    \end{description}
\end{ejercicio}

\begin{ejercicio}
    Sea $\Omega$ un abierto del plano, $a \in \Omega$ y $f, g \in \cc{H}(\Omega \setminus \{a\})$. Estudiar el comportamiento en el punto $a$ de las funciones $f + g$ y $f g$, suponiendo conocido el de $f$ y $g$.

\begin{comment}
    Consideramos la única descomposición de $f$ y $g$ en parte regular y parte singular:
    \begin{align*}
        f(z) &= g_f(z) + h_f(z)\qquad \forall z \in \Omega \setminus \{a\}\\
        g(z) &= g_g(z) + h_g(z)\qquad \forall z \in \Omega \setminus \{a\}
    \end{align*}
    con $g_f, g_g \in \cc{H}(\Omega)$ y $h_f, h_g \in \cc{H}(\bb{C} \setminus \{a\})$ con:
    \begin{equation*}
        h_f(z) = \varphi_f\left(\dfrac{1}{z - a}\right) \qquad \forall z \in \Omega \setminus \{a\}
        \qquad
        h_g(z) = \varphi_g\left(\dfrac{1}{z - a}\right) \qquad \forall z \in \Omega \setminus \{a\}
    \end{equation*}
    donde $\varphi_f, \varphi_g \in \cc{H}(\bb{C})$ y $\varphi_f(0) = 0$, $\varphi_g(0) = 0$.\\

    Entonces, las funciones $f + g$ y $f g$ son:
    \begin{align*}
        f(z) + g(z) &= \left(g_f(z) + h_f(z)\right) + \left(g_g(z) + h_g(z)\right) = (g_f + g_g)(z) + (h_f + h_g)(z)\qquad \forall z \in \Omega \setminus \{a\}\\
        f(z) g(z) &= \left(g_f(z) + h_f(z)\right)\left(g_g(z) + h_g(z)\right) =\\&= g_f(z) g_g(z) + g_f(z) h_g(z) + g_g(z) h_f(z) + h_f(z) h_g(z)
        \qquad \forall z \in \Omega \setminus \{a\}
    \end{align*}


    Trabajamos en primer lugar con la función $f + g$. Sabemos que $g_f + g_g \in \cc{H}(\Omega)$ y $(h_f + h_g) \in \cc{H}(\bb{C} \setminus \{a\})$ con:
    \begin{equation*}
        (h_f + h_g)(z) = \varphi_f\left(\dfrac{1}{z - a}\right) + \varphi_g\left(\dfrac{1}{z - a}\right) \qquad \forall z \in \Omega \setminus \{a\}
    \end{equation*}

    Definimos la función $\varphi_{f+g} \in \cc{H}(\bb{C})$ como sigue:
    \Func{varphi_{f+g}}{\bb{C}}{\bb{C}}{w}{\varphi_f(w) + \varphi_g(w)}
    Entonces, tenemos que:
    \begin{align*}
        (h_f + h_g)(z) &= \varphi_{f+g}\left(\dfrac{1}{z - a}\right) \qquad \forall z \in \Omega \setminus \{a\}
    \end{align*}
    Además, $\varphi_{f+g}(0) = \varphi_f(0) + \varphi_g(0) = 0 + 0 = 0$. Por tanto, hemos obtenido la descomposición de $f + g$ en parte regular y parte singular:
    \begin{align*}
        f(z) + g(z) &= (g_f + g_g)(z) + (h_f + h_g)(z)\qquad \forall z \in \Omega \setminus \{a\}
    \end{align*}
\end{comment}
\end{ejercicio}

\begin{ejercicio}
    La función $f$ es holomorfa en un entorno del punto $a$ y otra función $g$ tiene un polo de orden $m$ en el punto $f(a)$. ¿Cómo se comporta en $a$ la composición $g \circ f$? ¿Qué ocurre si $a$ es una singularidad esencial en $g$?\\

    Sea $\Omega$ un dominio del plano tal que $a\in \Omega$ y $f \in \cc{H}(\Omega)$. Sea ahora $g$ una función que tiene un polo de orden $m > 0$ en el punto $f(a)$. Es decir, $\exists \Psi \in \cc{H}(\Omega)$ tal que:
    \begin{equation*}
        g(z) = \dfrac{\Psi(z)}{(z - f(a))^m}\qquad \forall z \in \Omega \setminus \{f(a)\}
    \end{equation*}
    con $\Psi(f(a)) \neq 0$. Entonces, la composición $g \circ f$ es:
    \begin{align*}
        g(f(z)) &= \dfrac{\Psi(f(z))}{(f(z) - f(a))^m}\qquad \forall z \in \Omega \setminus f^{-1}(f(a))
    \end{align*}

    Como $f(a)-f(a) = 0$, tenemos que la aplicación $z\mapsto z - f(a)$ tiene un cero en $f(a)$. Sea $k \in \bb{N}$ el orden de $f(a)$ como cero de $z - f(a)$. Entonces, podemos escribir:
    \begin{align*}
        z - f(a) &= (z - a)^k \cdot h(z)\qquad \forall z \in \Omega
    \end{align*}
    con $h \in \cc{H}(\Omega)$ y $h(f(a)) \neq 0$. Por tanto, tenemos que:
    \begin{align*}
        g(f(z)) &= \dfrac{\Psi(f(z))}{\left((z - a)^k \cdot h(f(z))\right)^m}
        = \dfrac{\Psi(f(z))}{(z - a)^{km} \cdot h(f(z))^m}\qquad \forall z \in \Omega \setminus f^{-1}(f(a))
    \end{align*}
    Definimos por tanto la función $\Phi \in \cc{H}(\Omega)$ como sigue:
    \begin{equation*}
        \Phi(z) = \dfrac{\Psi(f(z))}{h(f(z))^m}\qquad \forall z \in \Omega
    \end{equation*}
    Entonces, tenemos que:
    \begin{align*}
        g(f(z)) &= \dfrac{\Phi(z)}{(z - a)^{km}}\qquad \forall z \in \Omega \setminus \{a\}
    \end{align*}
    Con $\Phi \in \cc{H}(\Omega)$ y $\Phi(a) \neq 0$. Por tanto, $g \circ f$ tiene un polo de orden $km$ en $a$.\\

    Supongamos ahora que $g$ tiene una singularidad esencial en $f(a)$. Entonces, por la caracterización, para todo $\delta\in \bb{R}^+$ con $D(f(a), \delta) \subseteq f(\Omega)$, tenemos que:
    \begin{equation*}
        \ol{g(D(f(a), \delta)\setminus \{f(a)\})} = \bb{C}
    \end{equation*}

    Veamos que también se cumple que, para cada $\rho \in \bb{R}^+$ con $D(a, \rho) \subseteq \Omega$, se cumple que:
    \begin{equation*}
        \ol{g(f(D(a, \rho)\setminus \{a\}))} = \bb{C}
    \end{equation*}
    Sea $\rho \in \bb{R}^+$ tal que $D(a, \rho) \subseteq \Omega$. Hay dos opciones:
    \begin{itemize}
        \item Supongamos que $f$ no es constante.
        
        Entonces, por el Teorema de la Aplicación Abierta, tenemos que $f(D(a, \rho))$ es un abierto de $\bb{C}$, y por tanto $\exists \rho' \in \bb{R}^+$ tal que $D(f(a), \rho') \subseteq f(D(a, \rho))\subset f(\Omega)$. Por tanto, como $g$ tiene una singularidad esencial en $f(a)$, tenemos que:
        \begin{equation*}
            \ol{g(D(f(a), \rho')\setminus \{f(a)\})} = \bb{C}
        \end{equation*}

        Por otro lado como $D(f(a), \rho') \subseteq f(D(a, \rho))$, tenemos que:
        \begin{align*}
            D(f(a), \rho') \setminus \{f(a)\} &\subseteq f(D(a, \rho)\setminus \{a\})
        \end{align*}

        Tomando en primer lugar $g$ en ambos lados de la inclusión, y en segundo lugar el cierre, tenemos que:
        \begin{align*}
            \bb{C} = \ol{g(D(f(a), \rho')\setminus \{f(a)\})} &\subseteq \ol{g(f(D(a, \rho)\setminus \{a\}))}\subset \bb{C}
        \end{align*}

        Por tanto, tenemos que:
        \begin{equation*}
            \ol{g(f(D(a, \rho)\setminus \{a\}))} = \bb{C}
        \end{equation*}

        Por tanto, dicho conjunto es denso en $\bb{C}$, y por tanto $g \circ f$ tiene una singularidad esencial en $a$.

        \item Supongamos que $f$ es constante. Entonces:
        \begin{equation*}
            f(z) = f(a) \qquad \forall z \in \Omega
        \end{equation*}
        Por tanto, como $g$ no está definida en $f(a)$, no se puede considerar la composición $g \circ f$ en ningún punto de $\Omega$.
    \end{itemize}
    
    
    
\end{ejercicio}

\begin{ejercicio}
    Sea $U$ un entorno reducido de un punto $a \in \bb{C}$ y supongamos que $f \in \cc{H}(U)$ tiene un polo en $a$. Probar que existe $R\in \bb{R}^+$ tal que $\bb{C} \setminus D(0,R) \subseteq f(U)$.\\

    Como $f$ tiene un polo en $a$, entonces:
    \begin{equation*}
        \lim_{z \to a} f(z) = \infty
    \end{equation*}

    Por tanto, por la continuidad de $f$ en $a$, $\exists \delta\in \bb{R}^+$ con $D(a, \delta) \subseteq U$ tal que:
    \begin{equation*}
        f(z)\neq 0 \qquad \forall z \in D(a, \delta) \setminus \{a\}
    \end{equation*}

    Definimos por tanto:
    \Func{\wt{g}}{D(a, \delta) \setminus \{a\}}{\bb{C}}{z}{\frac{1}{f(z)}}

    Como $\wt{g}\in \cc{H}(D(a, \delta) \setminus \{a\})$, por el Teorema de Extesión de Riemman existe una función $g \in \cc{H}(D(a, \delta))$ tal que:
    \begin{equation*}
        g(z) = \frac{1}{f(z)}\qquad \forall z \in D(a, \delta) \setminus \{a\}
    \end{equation*}

    Veamos el valor de $g(a)$. Por continuidad de $g$ en $a$, tenemos que:
    \begin{align*}
        g(a) &= \lim_{z \to a} g(z) = \lim_{z \to a} \frac{1}{f(z)} = 0
    \end{align*}

    Como $f$ tiene un polo en $a$, en particular no está acotada y por tanto no es constante, por lo que $g$ tampoco lo es. Por el Teorema de la Aplicación Abierta, $g(D(a, \delta))$ es un abierto de $\bb{C}$, y como $g(a) = 0$, tenemos que $\exists \rho \in \bb{R}^+$ tal que:
    \begin{equation*}
        D(0, \rho) \subseteq g(D(a, \delta))
    \end{equation*}

    Definimos $R= \nicefrac{2}{\rho} \in \bb{R}^+$, y veamos que:
    \begin{align*}
        \bb{C} \setminus D(0, R) \subseteq f\left(D(a, \delta)\setminus \{a\}\right)\subset f(U)
    \end{align*}
    \begin{description}
        \item[$\subseteq$)] Sea $w \in \bb{C} \setminus D(0, R)$. Entonces:
        \begin{align*}
            |w| \geq R &= \frac{2}{\rho}>\frac{1}{\rho}
            \Longrightarrow \frac{1}{|w|} < \rho
            \Longrightarrow \frac{1}{w}\in D(0, \rho)\subset g(D(a, \delta))
        \end{align*}

        Por tanto, $\exists z \in D(a, \delta)\subset U$ tal que:
        \begin{align*}
            g(z) = \frac{1}{w}\neq 0 &\Longrightarrow \frac{1}{f(z)} = \frac{1}{w}
            \Longrightarrow f(z) = w
        \end{align*}
        Además, como $g(z)\neq 0$, deducimos que $z\neq a$. Por tanto, tenemos que:
        \begin{align*}
            w \in f\left(D(a, \delta)\setminus \{a\}\right) \subset f(U)
        \end{align*}
    \end{description}
\end{ejercicio}


\begin{ejercicio}
    Sea $a$ una singularidad de una función $f$. Probar que la función $\Re f$ no puede estar acotada en un entorno reducido de $a$.
\end{ejercicio}

\begin{ejercicio}
    Sea $\Omega$ un abierto del plano, $a \in \Omega$ y $\{a_n\}$ una sucesión de puntos de $\Omega \setminus \{a\}$ tal que $\{a_n\} \to a$. Consideremos el conjunto $K = \{a_n : n \in \bb{N}\} \cup \{a\}$, sea $f \in \cc{H}(\Omega \setminus K)$ y supongamos que $f$ tiene un polo en $a_n$ para todo $n \in \bb{N}$. Probar que, para todo $\varepsilon > 0$ que verifique $D(a, \varepsilon) \subseteq \Omega$, el conjunto $f^{-1}(D(a, \varepsilon) \setminus K)$ es denso en $\bb{C}$.
\end{ejercicio}
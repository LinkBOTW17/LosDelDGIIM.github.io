\section{El teorema general de Cauchy}

\begin{ejercicio}
    Enunciar con detalle y demostrar que el índice de un punto respecto a un camino cerrado se conserva por giros, homotecias y traslaciones.\\

    Sea $\gamma : [a,b] \to \bb{C}$ un camino cerrado y $z \in \bb{C}$. Sean también $a\in \bb{C}^*$ y $b\in \bb{C}$, y definimos la siguiente transformación afín:
    \Func{T}{\bb{C}}{\bb{C}}{z}{az+b}

    Probar que:
    \begin{equation*}
        \Ind_{\gamma}(z) = \Ind_{T\circ\gamma}(T(z)) \quad \forall z \in \bb{C}\setminus \gamma^*
    \end{equation*}
    \begin{proof}
        Veamos en primer que dicha definición tiene sentido. Para ello, hemos de comprobar que $T\circ\gamma$ es un camino cerrado y que $T(z) \notin (T\circ\gamma)^*$. 
        \begin{itemize}
            \item Veamos que $T\circ\gamma$ es un camino. Como $\gamma$ es un camino, existe una partición $a = t_0 < t_1 < \ldots < t_n = b$ tal que $\gamma$ es de clase $C^1$ en cada intervalo $[t_i,t_{i+1}]$. Entonces, como $T$ es una transformación afín, $T\circ\gamma$ es de clase $C^1$ en cada intervalo $[t_i,t_{i+1}]$, y por tanto, $T\circ\gamma$ es un camino. Además, como $\gamma(a) = \gamma(b)$, tenemos que
            \begin{equation*}
                T\circ\gamma(a) = T(\gamma(a)) = T(\gamma(b)) = T\circ\gamma(b),
            \end{equation*}
            por lo que $T\circ\gamma$ es un camino cerrado.


            \item Veamos ahora que $T(z) \notin (T\circ\gamma)^*$. Supongamos por el contrario que $T(z) \in (T\circ\gamma)^*$. Entonces, existe $t \in [a,b]$ tal que $T(z) = T(\gamma(t))$. Apicando la inversa de $T$ (puesto que sabemos que $T$ es biyectiva), tenemos que:
            \begin{equation*}
                z = \gamma(t)\Longrightarrow z \in \gamma^*
            \end{equation*}
            lo cual contradice la hipótesis de que $z \notin \gamma^*$. Por tanto, $T(z) \notin (T\circ\gamma)^*$.
        \end{itemize}

        Por tanto, hemos comprobado que $T\circ\gamma$ es un camino cerrado y que $T(z) \notin (T\circ\gamma)^*$, por lo que la igualdad del enunciado tiene sentido. Veamos ahora que efectivamente se cumple la igualdad:
        \begin{align*}
            \Ind_{T\circ\gamma}(T(z)) &= \frac{1}{2\pi i} \int_{T\circ\gamma} \frac{1}{w - T(z)} dw
            = \frac{1}{2\pi i} \int_{a}^{b} \frac{1}{(T\circ \gamma)(t) - T(z)} \cdot (T\circ \gamma)'(t) dt
            =\\&= \frac{1}{2\pi i} \int_{a}^{b} \frac{1}{T(\gamma(t)) - T(z)} \cdot T'(\gamma(t)) \cdot \gamma'(t) dt.
            = \frac{1}{2\pi i} \int_{a}^{b} \frac{1}{\cancel{a}(\gamma(t)-z)} \cdot \cancel{a} \cdot \gamma'(t) dt
            =\\&= \frac{1}{2\pi i} \int_{a}^{b} \frac{1}{\gamma(t)-z} \cdot \gamma'(t) dt.
            = \int_{\gamma} \frac{1}{w - z} dw
            = \Ind_{\gamma}(z).
        \end{align*}

        Como $z \notin \gamma^*$ era arbitrario, hemos probado que la igualdad se cumple para todo $z \in \bb{C} \setminus \gamma^*$. Veamos ahora que los giros, homotecias y traslaciones son casos particulares de la transformación $T$ que hemos definido.
        \begin{itemize}
            \item Si $a = e^{i\theta}$ y $b = 0$, entonces $T(z) = e^{i\theta}z$ es un giro de ángulo $\theta$ alrededor del origen.
            \item Si $a = k$ y $b = 0$, con $k\in \bb{R}^+$, entonces $T(z) = kz$ es una homotecia de centro el origen y razón $k$.
            \item Si $a = 1$ y $b = b_0$, con $b_0 \in \bb{C}$, entonces $T(z) = z + b_0$ es una traslación de vector $b_0$.
        \end{itemize}
    \end{proof}
\end{ejercicio}

\begin{ejercicio}
    Sea $\rho : [-\pi,\pi] \to \bb{R}^+$ una función de clase $C^1$, con $\rho(-\pi)=\rho(\pi)$, y sea $\sigma$ el siguiente arco:
    \Func{\sigma}{[-\pi,\pi]}{\bb{C}}{t}{\rho(t)e^{it}}
    Calcular $\Ind_\sigma(z)$ para todo $z \in \bb{C}\setminus\sigma^*$.\\

    Gracias a la condición de que $\rho(-\pi) = \rho(\pi)$, sabemos que $\sigma$ es un camino cerrado. Buscamos comprender cómo es el arco $\sigma$ y qué puntos recorre, para poder así obtener intuitivamente el índice de $z$ respecto a dicho arco.
    \begin{align*}
        |\sigma(t)| &= |\rho(t)e^{it}| = \rho(t) \quad \forall t \in [-\pi,\pi],\\
        \Arg(\sigma(t)) &= \Arg(\rho(t)e^{it}) = \Arg(e^{it}) = t + 2\pi\bb{Z} \quad \forall t \in [-\pi,\pi].
    \end{align*}

    Por tanto, para cada $t \in \left]-\pi,\pi\right]$, $\exists! z\in \bb{C}$ tal que:
    \begin{equation*}
        z = \sigma(t)
    \end{equation*}

    Por tanto, se trata de una especie de ``círculo'' de centro en el origen, tan solo que el radio de dicho círculo es variable ($\rho(t)$). Veamos cuál es el conjunto de puntos que recorre el arco $\sigma$:
    \begin{align*}
        \sigma([- \pi, \pi]) &= \left\{ z \in \bb{C} : \exists t \in [-\pi,\pi] \text{ tal que } z = \rho(t)e^{it} \right\}
        =\\&= \left\{ z \in \bb{C} : \exists t \in [-\pi,\pi] \text{ tal que } |z| = \rho(t) \text{ y } \Arg(z) = t + 2\pi\bb{Z} \right\}.
        = \\&= \left\{ z \in \bb{C} : \exists t \in [-\pi,\pi] \text{ tal que } |z| = \rho(t)\right\}
        = \left\{ z \in \bb{C} : |z|\in \rho([- \pi, \pi]) \right\}.
    \end{align*}

    Por tanto:
    \begin{equation*}
        \bb{C}\setminus\sigma^* = \left\{ z \in \bb{C} : |z| \notin \rho([- \pi, \pi]) \right\}.
    \end{equation*}
    
    Calculemos ahora el índice de $z$ respecto al arco $\sigma$.
    Sabemos que el círculo tan solo se recorre una vez y que se recorre en el sentido positivo (sentido antihorario), por lo que:
    \begin{equation*}
        \Ind_\sigma(z) = \begin{cases}
            1 & \text{si } |z| < \rho(t) \text{ para todo } t \in [-\pi,\pi],\\
            0 & \text{si } |z| > \rho(t) \text{ para todo } t \in [-\pi,\pi].
        \end{cases}
    \end{equation*}

    Formalicemos este hecho. Para cada $z \in \bb{C}\setminus\sigma^*$, tenemos que:
    \begin{align*}
        \Ind_\sigma(z) &= \frac{1}{2\pi i} \int_{\sigma} \frac{1}{w - z} dw
        = \frac{1}{2\pi i} \int_{-\pi}^{\pi} \frac{1}{\rho(t)e^{it} - z} \cdot (\rho'(t)e^{it} + \rho(t)ie^{it})\ dt
    \end{align*}
    
    Como desconocemos el valor de $\rho'(t)$, no podemos calcular la integral directamente, por lo que nos quedaremos con la intuición anterior.
    % // TODO: Calcular
\end{ejercicio}

\begin{ejercicio}
    Sean $\gamma_1, \gamma_2 : [a,b] \to \bb{C}$ caminos cerrados y $z \in \bb{C}$ verificando que
    \begin{equation*}
        | \gamma_2(t) - \gamma_1(t) | < | \gamma_1(t) - z | \quad \forall t \in [a,b].
    \end{equation*}
    Probar que $\Ind_{\gamma_1}(z) = \Ind_{\gamma_2}(z)$.
\end{ejercicio}

\begin{ejercicio}
    Sea $\alpha : \bb{R}^+_0 \to \bb{C}$ una función continua, tal que:
    \begin{equation*}
        \alpha(0) = 0 \quad \text{y} \quad \lim_{t \to +\infty} \alpha(t) = +\infty
    \end{equation*}
    y sea $\Omega = \bb{C} \setminus \alpha(\bb{R}^+_0)$. Probar que $\Omega$ es abierto y que existe $f \in \cc{H}(\Omega)$ tal que $e^{f(z)} = z$ para todo $z \in \Omega$.\\

    Veamos que $\alpha(\bb{R}^+_0)$ es un conjunto cerrado. Sea una sucesión $\{z_n\}_{n \in \bb{N}} \subseteq \alpha(\bb{R}^+_0)$ tal que $\{z_n\}\to z \in \bb{C}$. Veamos que $z \in \alpha(\bb{R}^+_0)$. Para cada $n \in \bb{N}$, existe $t_n \in \bb{R}^+_0$ tal que $z_n = \alpha(t_n)$. Veamos que $\{t_n\}$ es acotada.
    \begin{itemize}
        \item Como $\{z_n\}\to z$, tenemos que $\exists R \in \bb{N}$ tal que $|z_n|<R$ para todo $n$ suficiente grande.
        \item Supongamos que $\{t_n\}$ no es acotada. Entonces:
        \begin{equation*}
            \left\{|z_n|\right\} = \left\{|\alpha(t_n)|\right\} \to \infty
        \end{equation*}
        en contradicción con que $\{z_n\}$ es acotada. Por tanto, $\{t_n\}$ es acotada.
    \end{itemize}
    Por tanto, como $\{t_n\}$ es acotada, admite una parcial convergente $\{t_{n_k}\}$ tal que $\{t_{n_k}\} \to t \in \bb{R}^+_0$. Como $\alpha$ es continua, tenemos que:
    \begin{equation*}
        \lim_{k \to +\infty} z_{n_k} = \lim_{k \to +\infty} \alpha(t_{n_k}) = \alpha(t) = z.
    \end{equation*}

    Como toda parcial de una sucesión convergente converge al mismo límite, tenemos que $\alpha(t) = z$, por lo que $z \in \alpha(\bb{R}^+_0)$. Por tanto, $\alpha(\bb{R}^+_0)$ es cerrado, consiguiendo que su complemento es abierto:
    \begin{equation*}
        \Omega = \bb{C} \setminus \alpha(\bb{R}^+_0) \text{ es abierto.}
    \end{equation*}

    Veamos ahora que $\Omega$ es homológicamente conexo. Su complemento es $\alpha(\bb{R}^+_0)$, que es un conjunto conexo (pues es la imagen de un intervalo conexo por una función continua). Por tanto, $\alpha(\bb{R}^+_0)$ tan solo tiene una componente conexa, y no está acotada (pues $\lim\limits_{t \to +\infty} \alpha(t) = +\infty$). Por tanto, $\Omega$ es homológicamente conexo.

    Usamos ahora la caracterización de los abiertos homológicamente conexos del plano. Como $\alpha(0)=0$, sabemos que $0\notin \Omega$. Por tanto, como la identidad en $\Omega$ no es anula y es holomorfa, tenemos que $\exists f \in \cc{H}(\Omega)$ tal que $f(z) \in \Log(z)$ para todo $z \in \Omega$. Esto es:
    \begin{equation*}
        e^{f(z)} = z \quad \forall z \in \Omega.
    \end{equation*}
\end{ejercicio}

\begin{ejercicio}
    Sea $\Omega$ un abierto homológicamente conexo del plano tal que $\bb{R}^+ \subseteq \Omega \subseteq \bb{C}^*$. Probar que existe $f \in \cc{H}(\Omega)$ tal que
    \begin{equation*}
        f(x) = x^x \quad \forall x \in \bb{R}^+.
    \end{equation*}

    Por la caracterización de los abiertos homológicamente conexos del plano, sabemos que para todo $g\in \cc{H}(\Omega)$ tal que $0\notin g(\Omega)$, existe $f \in \cc{H}(\Omega)$ tal que:
    \begin{equation*}
        g(z) = e^{f(z)} \quad \forall z \in \Omega.
    \end{equation*}

    Podríamos intentar lo siguiente. Definimos $g : \Omega \to \bb{C}^*$ como:
    \Func{g}{\Omega}{\bb{C}^*}{z}{e^{(z^z)}=e^{\left(e^{z\log(z)}\right)}}

    No obstante, no podemos garantizar que $g$ sea holomorfa, ya que la función logaritmo principal no tiene por qué ser holomorfa en $\Omega$. Por tanto, no nos sirve dicha función. No obstante, como $0\notin \Omega$ y la función identidad es holomorfa en $\Omega$, existe un logaritmo holomorfo $\varphi\in \cc{H}(\Omega)$ de la identidad:
    \begin{equation*}
        e^{\varphi(z)} = z \quad \forall z \in \Omega.
    \end{equation*}

    Definimos por tanto la función $g$ como sigue:
    \Func{g}{\Omega}{\bb{C}^*}{z}{e^{\left(e^{z\varphi(z)}\right)}}

    Por ser composición de funciones holomorfas, $g\in \cc{H}(\Omega)$. Además, $0\notin g(\Omega)$, puesto que la exponencial no se anula. Por tanto, por la caracterización de los abiertos homológicamente conexos del plano, existe $f \in \cc{H}(\Omega)$ tal que:
    \begin{equation*}
        e^{f(z)} = g(z) = e^{\left(e^{z\varphi(z)}
        \right)}
        = e^{z^z}
        \quad \forall z \in \Omega.
    \end{equation*}

    Como $\bb{R}^+ \subseteq \Omega$, tenemos que:
    \begin{equation*}
        e^{f(x)} = e^{\left(x^x\right)}
        \quad \forall x \in \bb{R}^+.
    \end{equation*}

    Como la exponencial real sí sabemos que es inyectiva, tenemos que:
    \begin{equation*}
        f(x) = x^x
        \quad \forall x \in \bb{R}^+.
    \end{equation*}
    Por tanto, hemos probado lo pedido.
\end{ejercicio}
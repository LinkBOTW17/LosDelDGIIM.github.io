\section{Topología del plano complejo}

\begin{ejercicio}
    Estudiar la continuidad de la función argumento principal; esta es, $\arg : \mathbb{C}^\ast \to \mathbb{R}$.\\

    Por el Ejercicio~\ref{ej:1.6}, sabemos que:
    \begin{equation*}
        \arg z = 2\arctan\left(\frac{\Im z}{\Re z + |z|}\right) \quad \forall z \in \mathbb{C}^\ast\setminus \bb{R}^-
    \end{equation*}

    Consideramos $\Omega = \mathbb{C}^\ast\setminus \bb{R}^-$. Como la función $Id$ es continua, tenemos que $\Re z,\Im z, |z|$ son continuas en $\cc{C}$. Además, como el denominador tan solo se anula en $\bb{R}^-_0$, el argumento de la arcotangente restringido a $\Omega$ es una función continua. Por ser la arcotangente continua en $\bb{R}$ y serlo el producto de funciones continuas, concluimos que $\arg_{\big| \Omega}$ es continua. Como $\Omega$ es abierto, por el carácter local de la continuidad, $\arg$ es continua en $\Omega=\mathbb{C}^\ast\setminus \bb{R}^-$.\\

    Tan falta por estudiar la continuidad en $\bb{R}^-$. Para ello, sea $z\in \bb{R}^-$, del que sabemos que $\arg z = \pi$. Sea la sucesión $\{\theta_n\}$ que recorre los ángulos desde $0$ en sentido horario hasta $-\pi$, límite de la sucesión:
    \begin{equation*}
        \{\theta_n\} = \left\{-\pi\left(1+\frac{1}{n}\right)\right\}\to -\pi
    \end{equation*}

    A partir de dicha sucesión, definimos $\{z_n\}$ como los números complejos de módulo $|z|$ y argumento $\theta_n$; que recorren los puntos de la circunferencia unitaria desde el eje positivo en sentido horario hasta el eje negativo.
    \begin{equation*}
        \{z_n\} = \left\{|z|\left(\cos\left(\theta_n\right)+i\sin\left(\theta_n\right)\right)\right\}\to |z|\left(\cos(-\pi)+i\sin(-\pi)\right) = -|z| = z
    \end{equation*}

    Por último, tenemos que:
    \begin{equation*}
        \{\arg z_n\} = \{\theta_n\} \to -\pi\neq \pi = \arg z
    \end{equation*}

    Por tanto, hemos encontrado una sucesión $\{z_n\}$ con $z_n\in \bb{C}^*~\forall n\in \mathbb{N}$, con $\{z_n\}\to z$ pero $\{\arg z_n\}\nrightarrow \arg z$. Por tanto, $\arg$ no es continua en $z$. Como $z$ era arbitrario, concluimos que $\arg$ no es continua en $\bb{R}^-$.\\

    Por tanto, concluimos que $\arg$ es continua en $\mathbb{C}^\ast\setminus \bb{R}^-$, pero no lo es en $\bb{R}^-$.
\end{ejercicio}

\begin{ejercicio}\label{ej:2.2}
    Dado $\theta \in \mathbb{R}$, se considera el conjunto $S_\theta = \{z \in \mathbb{C}^\ast \mid \theta \notin \Arg z\}$. Probar que existe una función $\varphi \in \cc{C}(S_\theta)$ que verifica $\varphi(z) \in \Arg (z)$ para todo $z \in S_\theta$.\\

    La elección del argumento principal de un número complejo realizada provoca que haya una discontinuidad en $\bb{R}^-=S_{\pi}$. Este ejercicio nos pide encontrar una función que, dado un argumento $\theta$, sea continua en $\bb{C}^*$ excepto en los puntos $z$ para los cuales $\theta\in \Arg z$.\\

    Dado $z\in S_{\theta}$, como $\arg$ es continua en $\mathbb{C}^\ast\setminus \bb{R}^-$, en primer lugar definiremos una función $g_{\theta}:S_{\theta}\to C^{\ast}\setminus \bb{R}^-$ que nos lleve $z$ a un punto $w\notin \bb{R}^-$ (esto lo haremos girando $z$ un ángulo de $\pi-\theta$); para poder aplicar luego $\arg$ y modificar el valor de forma que $\varphi(z)\in \Arg z$ (esto lo haremos restando $\pi-\theta$). Vamos a ello.\\

    Definimos en primer lugar $w_\theta=\cos(\pi-\theta) + i\sin(\pi-\theta)\in \mathbb{C}$, de forma que $|w_\theta|=1$ y $\pi-\theta\in \Arg w_\theta$. Definimos $g_{\theta}$ como:
    \Func{g_{\theta}}{S_{\theta}}{\mathbb{C}^{\ast}\setminus \bb{R}^-}{z}{zw_{\theta}}

    En primer lugar, como $g_{\theta}$ es polinómica, tenemos que $g_\theta\in \cc{C}(S_{\theta})$. Además, dado $z\in S_{\theta}$, tenemos que:
    \begin{align*}
        \Arg g_{\theta}(z) &= \Arg(zw_\theta) = \Arg z + \Arg w_{\theta} = (\arg z + \pi-\theta)+2\pi\bb{Z}
    \end{align*}

    Veamos que $g_{\theta}(z)\notin \bb{R}^-$. Supongamos que $g_{\theta}(z)\in \bb{R}^-$. Entonces, $\exists k\in \bb{Z}$ tal que $\arg z + \pi-\theta = 2k\pi$. Por tanto, $\arg z = 2k\pi - \pi + \theta = (2k-1)\pi + \theta$. Por tanto, $\theta\in \Arg z$, lo cual es una contradicción. Por tanto, $g_{\theta}(z)\notin \bb{R}^-$.\\

    A continuación, definimos $\varphi$ como sigue:
    \Func{\varphi}{S_{\theta}}{\bb{R}}{z}{\arg (g_{\theta}(z)) - (\pi-\theta)}

    De esta forma, tenemos que $\varphi$ es continua en $S_{\theta}$, puesto que $\arg$ es continua en $\mathbb{C}^{\ast}\setminus \bb{R}^-$ y $g_{\theta}$ es continua en $S_{\theta}$. Además, dado $z\in S_{\theta}$, tenemos que:
    \begin{equation*}
        \varphi(z) \in \Arg g_{\theta}(z) - \Arg w_{\theta} = \Arg g_{\theta}(z) + \Arg\frac{1}{w_{\theta}} = \Arg\left(\frac{g_{\theta}(z)}{w_{\theta}}\right) = \Arg\left(\frac{zw_{\theta}}{w_{\theta}}\right) = \Arg z
    \end{equation*}
\end{ejercicio}

\begin{ejercicio}
    Probar que no existe ninguna función $\varphi \in \cc{C}(\mathbb{C}^\ast)$ de forma que $\varphi(z) \in \Arg z$ para todo $z \in \mathbb{C}^\ast$, y que el mismo resultado es cierto, sustituyendo $\mathbb{C}^\ast$ por $\bb{T} = \{z \in \mathbb{C} \mid |z| = 1\}$.\\

    Por reducción al absurdo, supongamos que existe una función $\varphi\in \cc{C}(\mathbb{C}^{\ast})$ tal que $\varphi(z)\in \Arg z~\forall z\in \mathbb{C}^{\ast}$. Definimos la siguiente función auxiliar:
    \Func{f}{\mathbb{C}^{\ast}}{\bb{R}}{z}{\varphi(z)-\varphi(-z)}

    Por ser $\varphi$ continua, $f$ es continua. Además, dado $z\in \mathbb{C}^{\ast}$, tenemos que:
    \begin{align*}
        f(z) &= \varphi(z)-\varphi(-z)\\
        f(-z) &= \varphi(-z)-\varphi(z) = -(\varphi(z)-\varphi(-z)) = -f(z)
    \end{align*}

    Por tanto, fijado $w\in \mathbb{C}^{\ast}$, hay dos opciones:
    \begin{itemize}
        \item Si $f(w)=0$, entonces sea $z_0=w$, y se tiene que $f(z_0)=0$.
        \item Si $f(w)\neq 0$, entonces $f(w)f(-w)<0$. Como $\bb{C}^{\ast}$ es conexo, por el Teorema del Valor Intermedio $\exists z_0\in \mathbb{C}^{\ast}$ tal que $f(z_0)=0$.
    \end{itemize}
    En cualquier caso, $\exists z_0\in \mathbb{C}^{\ast}$ tal que $f(z_0)=0$. Por tanto, $\varphi(z_0)=\varphi(-z_0)$. Esto implica que $\Arg z_0 = \Arg (-z_0)$, lo cual es una contradicción ya que:
    \begin{align*}
        \Arg -z_0 &= (\arg z_0 + \pi) + 2\pi\bb{Z}
    \end{align*}

    Por tanto, no puede existir una función $\varphi\in \cc{C}(\mathbb{C}^{\ast})$ tal que $\varphi(z)\in \Arg z~\forall z\in \mathbb{C}^{\ast}$.\\

    Por otro lado, consideramos el caso para $\bb{T}$. Hay diversas formas de probarlo:
    \begin{itemize}
        \item De forma análoga, haciendo uso ahora de que $\bb{T}$ es conexo.
        \item Aplicando de forma directa el Teorema de Borsuk-Ulam a $\varphi$ (esto es lo que en realidad hacemos en la opción anterior).
        \item Haciendo uso de lo anteriormente demostrado.
    \end{itemize}

    Desarrollaremos la tercera opción, por ser aquella que difiere de lo anterior. De nuevo, supongamos por reducción al absurdo que existe una función $\varphi\in \cc{C}(\bb{T})$ tal que $\varphi(z)\in \Arg z~\forall z\in \bb{T}$. Definimos la siguiente función auxiliar:
    \Func{f}{\bb{C}^{\ast}}{\bb{R}}{z}{\varphi\left(\dfrac{z}{|z|}\right)}

    Tenemos que $f$ es continua, y verifica que:
    \begin{equation*}
        f(z) = \varphi\left(\frac{z}{|z|}\right) \in \Arg\left(\frac{z}{|z|}\right) = \Arg z - \Arg (|z|) = \Arg z - 2\pi\bb{Z} = \Arg z
    \end{equation*}
    No obstante, hemos demostrado que no puede existir una función $f\in \cc{C}(\mathbb{C}^{\ast})$ tal que $f(z)\in \Arg z~\forall z\in \mathbb{C}^{\ast}$. Por tanto, hemos llegado a una contradicción, y concluimos que no puede existir una función $\varphi\in \cc{C}(\bb{T})$ tal que $\varphi(z)\in \Arg z~\forall z\in \bb{T}$.
\end{ejercicio}

\begin{ejercicio}
    Probar que la función $\Arg : \mathbb{C}^\ast \to \mathbb{R}/2\pi\mathbb{Z}$ es continua, considerando en $\mathbb{R}/2\pi\mathbb{Z}$ la topología cociente. Más concretamente, se trata de probar que, si $\{z_n\}$ es una sucesión de números complejos no nulos, tal que $\{z_n\} \to z \in \mathbb{C}^\ast$ y $\theta \in \Arg z$, se puede elegir $\theta_n \in \Arg z_n$ para todo $n \in \mathbb{N}$, de forma que $\{\theta_n\} \to \theta$.

    \begin{description}
        \item[Usando sucesiones:] Usaremos la caracterización que en el mismo enunciado describen. Dada una sucesión $\{z_n\}$ de números complejos no nulos, tal que $\{z_n\}\to z\in \mathbb{C}^{\ast}$ y $\theta\in \Arg z$, definimos $\theta_n$ como sigue:
        \begin{itemize}
            \item \ul{Si $z\notin \bb{R}^-$}:
            
            Como $\arg z\in \Arg z$, tenemos que $\exists k\in \bb{Z}$ tal que $\theta=2k\pi+\arg z$. Por tanto, definimos $\theta_n$ como:
            \begin{equation*}
                \theta_n = \arg z_n + 2k\pi \in \Arg z_n\qquad \forall n\in \mathbb{N}
            \end{equation*}

            Además, tenemos que:
            \begin{equation*}
                \left\{\theta_n\right\} = \left\{\arg z_n + 2k\pi\right\}\to \arg z + 2k\pi = \theta
            \end{equation*}
            donde hemos usado que, al ser $\arg$ continua en $z\in \mathbb{C}^{\ast}\setminus \bb{R}^-$, como se tiene que $\{z_n\}\to z$, entonces $\{\arg z_n\}\to \arg z$.
            
            \item \ul{Si $z\in \bb{R}^-$}:
            
            Por el Ejercicio~\ref{ej:2.2}, $\exists \varphi\in \cc{C}(S_{0})$ tal que $\varphi(w)\in \Arg w~\forall w\in S_{0}$. En particular, $\varphi(z)\in \Arg z$, por lo que $\exists k\in \bb{Z}$ tal que $\varphi(z)=\theta+2k\pi$.\\

            Como $\{z_n\}\to z\in S_0=S_0^\circ$ abierto, $\exists N\in \mathbb{N}$ tal que $\forall n\geq N$ se tiene que $z_n\in S_{0}$. Por tanto, definimos $\theta_n$ como:
            \begin{equation*}
                \begin{cases}
                    \theta_n = \arg z_n & \text{si } n<N\\
                    \theta_n = \varphi(z_n)-2k\pi & \text{si } n\geq N
                \end{cases}
            \end{equation*}

            De esta forma, tenemos que $\theta_n\in \Arg z_n~\forall n\in \mathbb{N}$, y además:
            \begin{equation*}
                \left\{\theta_n\right\} \to \varphi(z)-2k\pi = \theta
            \end{equation*}
            donde hemos usado que, al ser $\varphi$ continua en $z\in S_0$, como $\{z_n\}\to z$, se tiene que $\{\varphi(z_n)\}\to \varphi(z)$.
        \end{itemize}
        \begin{observacion}
            Notemos que podríamos haber generalizado todo en el segundo caso, considerando $S_{\theta+\pi}$. No obstante, se ha optado por hacerlo de forma más explícita para facilitar la comprensión, ya que el primer caso seguramente sea más intuitivo.
        \end{observacion}
        
        \item[Usando el punto de vista topoógico:]~\\
        
        Definimos la función proyección:
        \Func{\pi}{\mathbb{R}}{\mathbb{R}/2\pi\mathbb{Z}}{x}{x+2\pi\mathbb{Z}}

        Tenemos la siguiente descomposición de $\bb{C}^*$:
        \begin{equation*}
            \bb{C}^* = \left(\bb{C}^*\setminus \bb{R}^-\right)\cup \left(\bb{C}^*\setminus \bb{R}^+\right)
        \end{equation*}
        Tenemos que:
        \begin{itemize}
            \item \ul{En $\bb{C}^*\setminus \bb{R}^-$}:
            \begin{equation*}
                \Arg\left(z\right)=\left(\pi\circ \arg\right)(z)\qquad \forall z\in \bb{C}^*\setminus \bb{R}^-
            \end{equation*}

            Por tanto, $\Arg$ es continua en $\bb{C}^*\setminus \bb{R}^-$.

            \item \ul{En $\bb{C}^*\setminus \bb{R}^+$}:
            
            Por el Ejercicio~\ref{ej:2.2}, sabemos que $\exists \varphi\in \cc{C}(S_{0})$ tal que:
            \begin{equation*}
                \Arg\left(z\right)=\left(\pi\circ \varphi\right)(z)\qquad \forall z\in S_0=\bb{C}^*\setminus \bb{R}^+
            \end{equation*}

            Por tanto, $\Arg$ es continua en $\bb{C}^*\setminus \bb{R}^+$.
        \end{itemize}

        Por el carácter local de la continuidad, $\Arg$ es continua en $\bb{C}^*$.
    \end{description}
\end{ejercicio}

\begin{ejercicio}
    Dado $z \in \mathbb{C}$, probar que la sucesión $\left\{\left(1 + \dfrac{z}{n}\right)^n\right\}$ es convergente y calcular su límite.\\

    Para facilitar la notación, sea:
    \begin{equation*}
        z_n = \left(1 + \dfrac{z}{n}\right)^n\qquad \forall n\in \mathbb{N}
    \end{equation*}

    En primer lugar, vamos a estudiar el límite de la sucesión $\{|z_n|\}$:
    \begin{align*}
        |z_n| &= \left|\left(1 + \dfrac{z}{n}\right)^n\right| = \left|1 + \dfrac{z}{n}\right|^n = \left(\sqrt{\left(1+\dfrac{\Re z}{n}\right)^2 + \left(\dfrac{\Im z}{n}\right)^2}\right)^n
        =\\&= \sqrt{\left(1 + \dfrac{\Re^2z}{n^2} + \dfrac{2\Re z}{n} + \dfrac{\Im^2 z}{n^2}\right)^{n}}
        = \sqrt{\left(1 + \dfrac{\dfrac{\Re^2z + \Im^2z}{n}+2\Re z}{n}\right)^{n}}
    \end{align*}

    Por tanto, tenemos que:
    \begin{align*}
        \lim_{n\to \infty}|z_n| &= \sqrt{\lim_{n\to \infty}\left(1 + \dfrac{\dfrac{\Re^2z + \Im^2z}{n}+2\Re z}{n}\right)^{n}}
        =\\&= \sqrt{\exp\left(\lim_{n\to \infty}\dfrac{\Re^2z + \Im^2z + 2n\Re z}{n}+2\Re z\right)}
        = \sqrt{\exp(2\Re z)} = e^{\Re z}
    \end{align*}
    donde en la primera igualdad hemos usado que la raíz es una función continua, y en la segunda igualdad hemos usado el Criterio de Euler. A continuación, estudiamos los argumentos de $z_n$. Para ello, definimos:
    \begin{equation*}
        w_n = 1 + \dfrac{z}{n}\qquad \forall n\in \mathbb{N}
    \end{equation*}

    Como $\{w_n\}\to 1$, $\exists N\in \mathbb{N}$ tal que $\forall n\geq N$ se tiene que $\Re w_n>0$. Por tanto, $\forall n\geq N$ se tiene que:
    \begin{equation*}
        \arg w_n = \arctan\left(\dfrac{\Im w_n}{\Re w_n}\right) = \arctan\left(\dfrac{\Im z}{n+\Re z}\right)
    \end{equation*}

    Como $\Arg(zw)=\Arg z + \Arg w$ para todo $z,w\in \mathbb{C}^{\ast}$, tenemos que:
    \begin{equation*}
        \Arg z_n = \Arg \left((w_n)^n\right) = n\Arg w_n \Longrightarrow
        n\arctan\left(\dfrac{\Im z}{n+\Re z}\right)\in \Arg z_n\qquad \forall n\geq N
    \end{equation*}

    Por tanto, definimos la sucesión $\{\theta_n\}$ como sigue:
    \begin{equation*}
        \theta_n = \begin{cases}
            \arg z_n & \text{si } n<N\\
            n\arctan\left(\dfrac{\Im z}{n+\Re z}\right) & \text{si } n\geq N
        \end{cases}
    \end{equation*}

    Por tanto, para todo $n\in \mathbb{N}$, tenemos que $\theta_n\in \Arg z_n$. 
    Calculemos el límite de la sucesión $\{\theta_n\}$:
    \begin{align*}
        \lim_{n\to \infty} \theta_n &= \lim_{n\to \infty}n\arctan\left(\dfrac{\Im z}{n+\Re z}\right) = \lim_{n\to \infty}\dfrac{\arctan\left(\dfrac{\Im z}{n+\Re z}\right)}{\dfrac{1}{n}}
        =\\&= \lim_{n\to \infty}\dfrac{-n^2}{1+\left(\dfrac{\Im z}{n+\Re z}\right)^2}\cdot \dfrac{-\Im z}{(n+\Re z)^2}
        = \lim_{n\to \infty}\dfrac{n^2\Im z}{(n+\Re z)^2+\Im^2 z} = \Im z
    \end{align*}
    
    
    Uniendo ambos resultados, tenemos que:
    \begin{align*}
        z_n = |z_n|\left(\cos(\theta_n) + i\sen(\theta_n)\right) \qquad \forall n\in \mathbb{N}
    \end{align*}

    Tomando límite, y como las funciones seno y coseno son continuas, tenemos que:
    \begin{equation*}
        \lim_{n\to \infty}z_n = \lim_{n\to \infty}|z_n|\left(\cos\left(\lim_{n\to \infty}\theta_n\right) + i\sen\left(\lim_{n\to \infty}\theta_n\right)\right) = e^{\Re z}\left(\cos(\Im z) + i\sen(\Im z)\right)
    \end{equation*}
\end{ejercicio}
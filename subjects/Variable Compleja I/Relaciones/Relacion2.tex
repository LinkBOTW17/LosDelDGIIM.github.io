\section{Topología del plano complejo}

\begin{ejercicio}
    Estudiar la continuidad de la función argumento principal; esta es, $\arg : \mathbb{C}^\ast \to \mathbb{R}$.
\end{ejercicio}

\begin{ejercicio}
    Dado $\theta \in \mathbb{R}$, se considera el conjunto $S_\theta = \{z \in \mathbb{C}^\ast \mid \theta \notin \Arg z\}$. Probar que existe una función $\varphi \in \cc{C}(S_\theta)$ que verifica $\varphi(z) \in \Arg (z)$ para todo $z \in S_\theta$.
\end{ejercicio}

\begin{ejercicio}
    Probar que no existe ninguna función $\varphi \in \cc{C}(\mathbb{C}^\ast)$ tal que $\varphi(z) \in \Arg z$ para todo $z \in \mathbb{C}^\ast$, y que el mismo resultado es cierto, sustituyendo $\mathbb{C}^\ast$ por $\bb{T} = \{z \in \mathbb{C} \mid |z| = 1\}$.
\end{ejercicio}

\begin{ejercicio}
    Probar que la función $\Arg : \mathbb{C}^\ast \to \mathbb{R}/2\pi\mathbb{Z}$ es continua, considerando en $\mathbb{R}/2\pi\mathbb{Z}$ la topología cociente. Más concretamente, se trata de probar que, si $\{z_n\}$ es una sucesión de números complejos no nulos, tal que $\{z_n\} \to z \in \mathbb{C}^\ast$ y $\theta \in \Arg z$, se puede elegir $\theta_n \in \Arg z_n$ para todo $n \in \mathbb{N}$, de forma que $\{\theta_n\} \to \theta$.
\end{ejercicio}

\begin{ejercicio}
    Dado $z \in \mathbb{C}$, probar que la sucesión $\left\{\left(1 + \dfrac{z}{n}\right)^n\right\}$ no es convergente y calcular su límite.
\end{ejercicio}
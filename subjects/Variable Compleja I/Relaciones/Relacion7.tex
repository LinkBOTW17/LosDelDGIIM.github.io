\section{Teorema local de Cauchy}

\begin{ejercicio}
    Sean $a \in \bb{C}$ y $r \in \bb{R}^+$. Probar que, para cada $z \in \bb{C}$ con $|z-a| > r$, se tiene:
    \[
        \int_{C(a,R)} \frac{dw}{w-z} = 0
    \]
\end{ejercicio}

\begin{ejercicio}[Versión más general de la fórmula de Cauchy]
    Sean $a \in \bb{C}$, $R \in \bb{R}^+$ y $f : \ol{D}(a,R) \to \bb{C}$ una función continua en $\ol{D}(a,R)$ y holomorfa en $D(a,R)$. Se tiene entonces:
    \[
        f(z) = \frac{1}{2\pi i} \int_{C(a,R)} \frac{f(w)}{w-z}dw \qquad \forall z \in D(a,R).
    \]
\end{ejercicio}

\begin{ejercicio}
    Dados $a \in \bb{C}$, $r \in \bb{R}^+$ y $b,c \in \bb{C}\setminus C(a,r)^\ast$, calcular todos los posibles valores de la integral
    \[
        \int_{C(a,r)} \frac{dz}{(z-b)(z-c)}
    \]
    dependiendo de la posición relativa de $b,c$ respecto de la circunferencia $C(a,r)^\ast$.
\end{ejercicio}

\begin{ejercicio}
    Calcular las siguientes integrales:
    \begin{enumerate}
        \item $\displaystyle\int_{C(0,r)} \frac{z+1}{z(z^2+4)}\ dz$ \qquad ($r \in \bb{R}^+$, $r \neq 2$)
        \item $\displaystyle\int_{C(0,1)} \frac{\cos z}{(a^2+1)z - a(z^2+1)}\ dz$ \qquad ($a \in \bb{C}$, $|a| \neq 1$)
    \end{enumerate}
\end{ejercicio}

\begin{ejercicio}
    Dados $a,b \in \bb{C}$ con $a \neq b$, sea $R \in \bb{R}^+$ tal que $R > \max\{|a|,|b|\}$. Probar que, si $f$ es una función entera, se tiene:
    \[
        \int_{C(0,R)} \frac{f(z)}{(z-a)(z-b)}dz = 2\pi i \cdot \frac{f(b)-f(a)}{b-a}.
    \]
    Deducir que toda función entera y acotada es constante.
\end{ejercicio}
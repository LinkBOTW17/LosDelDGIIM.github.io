\section{Teorema local de Cauchy}

\begin{ejercicio}\label{ej:7.1}
    Sean $a \in \bb{C}$ y $r \in \bb{R}^+$. Probar que, para cada $z \in \bb{C}$ con $|z-a| > r$, se tiene:
    \[
        \int_{C(a,r)} \frac{dw}{w-z} = 0
    \]
    \begin{description}
        \item[Opción 1] Consideramos el conjunto $\Omega=D(a,|z-a|)$, que es un abierto que contiene a $C(a,r)^*$. Veamos en primer lugar que $z-w$ no se anula en $\Omega$. Para ello, supongamos que $w \in \Omega$, entonces:
        \begin{equation*}
            |w-a| < |z-a|=|z-w+w-a| \leq |z-w|+|w-a| \Longrightarrow 0<|z-w|\Longrightarrow z-w\neq 0.
        \end{equation*}

        Por tanto, definimos la siguiente función:
        \Func{f}{\Omega}{\bb{C}}{w}{\frac{1}{w-z}}

        Como $f$ es racional, entonces $f\in \cc{H}(\Omega)$ y $\Omega$ es estrellado, entonces admite una primitiva $F:\Omega\to \bb{C}$, que es holomorfa en $\Omega$. Como $C(a,r)$ es un camino cerrado contenido en $\Omega$, se tiene que:
        \begin{equation*}
            \int_{C(a,r)} f(w) \ dw = \int_{C(a,r)} \dfrac{dw}{w-z} = 0
        \end{equation*}
        

        \item[Opción 2] Como se trata de una integral sobre un camino cerrado igualada a $0$, podríamos buscar una primitiva del integrando holomorfa en un abierto que contenga a $C(a,r)^*$. Sea dicho abierto el conjunto $\Omega=D(a,|z-a|)$, de forma que $C(a,r)^* \subset \Omega$, y veamos que $z-w$ admite un argumento continuo en $\Omega$. Veamos en primer lugar que $z-w$ no se anula en $\Omega$. Para ello, supongamos que $w \in \Omega$, entonces:
        \begin{equation*}
            |w-a| < |z-a|=|z-w+w-a| \leq |z-w|+|w-a| \Longrightarrow 0<|z-w|\Longrightarrow z-w\neq 0.
        \end{equation*}

        Por tanto, podemos considerar un argumento de $z-w$ en $\Omega$. Veamos que $z-w$ admite un argumento continuo en $\Omega$. Para ello, y en vistas de usar el Ehercicio~\ref{ej:2.2}, veamos que o bien nunca toma valores en $\bb{R}^+$, o bien nunca toma valores en $\bb{R}^-$. Supongamos que toma valores en ambos conjuntos; es decir, supongamos que $\exists w,w'\in \Omega$ tal que $w-z\in \bb{R}^+$ y $w'-z\in \bb{R}^-$. Entonces:
        \begin{align*}
            w-z\in \bb{R}^+ & \Longrightarrow \left\{\begin{array}{l}
                \Re w > \Re z\\
                \Im w = \Im z
            \end{array}\right.\\
            w'-z\in \bb{R}^- & \Longrightarrow \left\{\begin{array}{l}
                \Re w' < \Re z\\
                \Im w' = \Im z
            \end{array}\right.
        \end{align*}

        Por tanto, se tiene que:
        \begin{align*}
            \Re w' < \Re z < \Re w\\
            \Im w' = \Im z = \Im w
        \end{align*}

        Llegados a este punto, uno ya puede ver gráficamente que hemos llegado a una contradicción, puesto que $w$ y $w'$ no pueden estar ammbos en $D(a,|z-a|)$, pero veámoslo formalmente.
        Como $w\in \Omega$, se tiene que:
        \begin{align*}
            |w-a| < |z-a|\Longrightarrow \left(\Re w-\Re a\right)^2+\left(\Im w-\Im a\right)^2 < \left(\Re z-\Re a\right)^2+\left(\Im z-\Im a\right)^2
        \end{align*}

        Como $\Im w=\Im z$, se tiene que:
        \begin{multline*}
            \left(\Re w-\Re a\right)^2<\left(\Re z-\Re a\right)^2
            \Longrightarrow |\Re w-\Re a|<|\Re z-\Re a|\Longrightarrow \\ \Longrightarrow
            \Re a - |\Re z -\Re a| < \Re w < \Re a + |\Re z - \Re a|
        \end{multline*}

        Realizamos el procedimiento análogo para $w'$, llegando por tanto a que:
        \begin{align*}
            \Re a - |\Re z -\Re a| < \Re w,\Re w' < \Re a + |\Re z - \Re a|
        \end{align*}
        \begin{itemize}
            \item Si $|\Re z -\Re a| \geq 0$, se tiene que $\Re w<\Re z$.
            \item Si $|\Re z -\Re a| < 0$, se tiene que $\Re z<\Re w'$.
        \end{itemize}

        En cualquier caso, se llega a una contradicción, por lo que $z-w$ no puede tomar valores en ambos conjuntos. Por tanto, $z-w$ admite un argumento continuo en $\Omega$, por lo que se puede considerar un logaritmo continuo en $\Omega$ de $z-w$ (función holomorfa en $\Omega$). Por tanto, existe un logaritmo holomorfo de $z-w$ en $\Omega$ (llamémosle $F:\Omega \to \bb{C}$), de forma que:
        \begin{equation*}
            F'(w) = \frac{1}{w-z} \qquad \forall w \in \Omega.
        \end{equation*}

        Como $F\in \cc{H}(\Omega)$ es una primitiva del integrando (siendo este una función continua en $\Omega$ por no anularse el denominador) y $C(a,r)$ es un camino cerrado contenido en $\Omega$, se tiene que:
        \begin{equation*}
            \int_{C(a,r)} \frac{dw}{w-z} = 0
        \end{equation*}

        \item[Opción 3] Como $z\neq a$, tenemos que:
        \begin{equation*}
            \dfrac{1}{w-z} = \dfrac{1}{w-a+a-z}\cdot \dfrac{a-z}{a-z}
            = \dfrac{1}{a-z}\cdot \dfrac{1}{1+\frac{w-a}{a-z}}
            = \dfrac{1}{a-z}\cdot \dfrac{1}{1-\left(-\frac{w-a}{a-z}\right)}\quad \forall w\neq z
        \end{equation*}

        Por tando, para cada $w\in C(a,r)^*$, consideramos la siguiente serrie geométrica:
        \begin{equation*}
            \sum_{n\geq 0} \left(-\frac{w-a}{a-z}\right)^n
        \end{equation*}

        Tenemos que:
        \begin{equation*}
            \left|\left(-\frac{w-a}{a-z}\right)^n\right| = \left(\frac{r}{|a-z|}\right)^n\qquad \forall w\in C(a,r)^*
        \end{equation*}

        Por tanto, como $|a-z|>r$, por el Test de Weierstrass, la serie converge uniformemente en $C(a,r)^*$. Por tanto, podemos intercambiar la integral y la suma:
        \begin{align*}
            \int_{C(a,r)} \frac{dw}{w-z} &= \int_{C(a,r)} \frac{1}{a-z}\cdot \sum_{n=0}^\infty \left(-\frac{w-a}{a-z}\right)^n dw\\
            &= \frac{1}{a-z}\cdot \sum_{n=0}^\infty \int_{C(a,r)} \left(-\frac{w-a}{a-z}\right)^n dw\\
            &= \frac{1}{a-z}\cdot \sum_{n=0}^\infty \left(-\frac{1}{a-z}\right)^n \int_{C(a,r)} (w-a)^n \ dw\\
            &\AstIg \frac{1}{a-z}\cdot \sum_{n=0}^\infty \left(-\frac{1}{a-z}\right)^n \cdot 0= 0
        \end{align*}
        donde $(\ast)$ se debe a que la función $w\mapsto (w-a)^n$ es entera y admite primitiva en $\bb{C}$, por lo que la integral sobre un camino cerrado es $0$.

        \item[Opción 4] Buscamos aplicar la Fórmula de Cauchy para la circunferencia. Buscamos un abierto $\Omega\subset \bb{C}$ de forma que $\ol{D}(a,r)\subset \Omega$, por lo que sea $\Omega=D(a,|z-a|)$. Como $z\notin D(a,r)$, hemos de buscar una función $f:\Omega\to \bb{C}$ de forma que:
        \begin{equation*}
            \dfrac{1}{w-z} = \dfrac{f(w)}{w-a}\qquad \forall w\in D(a,r)
        \end{equation*}

        Vemos por tanto la expresión de $f(w)$. Por tanto, formalmente, definimos:
        \Func{f}{\Omega}{\bb{C}}{w}{\dfrac{w-a}{w-z}}

        Como $f$ es racional y el denominador no se anula en $\Omega$, se tiene que $f\in \cc{H}(\Omega)$. Como $\ol{D}(a,r)\subset \Omega$, estamos en las condiciones de aplicar la Fórmula de Cauchy para la circunferencia:
        \begin{align*}
            \int_{C(a,r)} \frac{dw}{w-z} &= \int_{C(a,r)} \dfrac{f(w)}{w-a}\ dw
            = 2\pi i \cdot f(a)\\
            &= 2\pi i \cdot \dfrac{a-a}{a-z} = 0
        \end{align*}
    \end{description}
\end{ejercicio}

\begin{ejercicio}[Versión más general de la fórmula de Cauchy]\label{ej:7.2}
    Sean $a \in \bb{C}$, $R \in \bb{R}^+$ y $f : \ol{D}(a,R) \to \bb{C}$ una función continua en $\ol{D}(a,R)$ y holomorfa en $D(a,R)$. Se tiene entonces:
    \[
        f(z) = \frac{1}{2\pi i} \int_{C(a,R)} \frac{f(w)}{w-z}dw \qquad \forall z \in D(a,R).
    \]

    % // TODO: Probar
\end{ejercicio}

\begin{ejercicio}
    Dados $a \in \bb{C}$, $r \in \bb{R}^+$ y $b,c \in \bb{C}\setminus C(a,r)^\ast$, calcular todos los posibles valores de la integral
    \[
        \int_{C(a,r)} \frac{dz}{(z-b)(z-c)}
    \]
    dependiendo de la posición relativa de $b,c$ respecto de la circunferencia $C(a,r)^\ast$.\\

    Caben varias posibilidades:
    \begin{enumerate}
        \item \ul{Si $b=c$}: En este caso, la integral queda:
        \begin{equation*}
            \int_{C(a,r)} \frac{dz}{(z-b)^2}
        \end{equation*}

        El integrando es una función holomorfa en $\bb{C}\setminus \{b\}$, que admite igualmente primitiva en $\bb{C}\setminus \{b\}$. Como $C(a,r)$ es un camino cerrado con $b\notin C(a,r)$, se tiene que:
        \begin{equation*}
            \int_{C(a,r)} \frac{dz}{(z-b)^2} = 0
        \end{equation*}

        \item{\ul{Si $b\neq c$}}: En este caso, descomponemos el integrando en fracciones simples:
        \begin{equation*}
            \dfrac{1}{(z-b)(z-c)} = \dfrac{A}{z-b} + \dfrac{B}{z-c} = \dfrac{A(z-c)+B(z-b)}{(z-b)(z-c)}
        \end{equation*}
        \begin{itemize}
            \item Para $z=b$: $1=A(b-c)$.
            \item Para $z=c$: $1=B(c-b)$.
        \end{itemize}

        Por tanto, se tiene que:
        \begin{equation*}
            \frac{1}{(z-b)(z-c)} = \frac{1}{b-c}\left(\frac{1}{z-b}-\frac{1}{z-c}\right)
        \end{equation*}

        Por tanto, la integral queda:
        \begin{align*}
            \int_{C(a,r)} \frac{dz}{(z-b)(z-c)} &= \frac{1}{b-c}\left(\int_{C(a,r)} \frac{dz}{z-b}-\int_{C(a,r)} \frac{dz}{z-c}\right)
        \end{align*}

        Distinguimos casos, teniendo en cuenta que $b,c\notin C(a,r)*$:
        \begin{enumerate}
            \item \ul{Si $b,c\in \bb{C}\setminus \ol{D}(a,r)$}: En este caso, por el Ejercicio~\ref{ej:7.1}, se tiene que:
            \begin{align*}
                \int_{C(a,r)} \frac{dz}{(z-b)(z-c)} &= \frac{1}{b-c}\left(\int_{C(a,r)} \frac{dz}{z-b}-\int_{C(a,r)} \frac{dz}{z-c}\right)\\
                &= \frac{1}{b-c}\left(0-0\right) = 0
            \end{align*}

            \item \ul{Si $b,c\in D(a,r)$}: En este caso, aplicamos la Fórmula de Cauchy para la circunferencia con $\Omega=\bb{C}$ y con función $f$ constantemente igual a $1$. Por tanto, se tiene que:
            \begin{align*}
                \int_{C(a,r)} \frac{dz}{(z-b)(z-c)} &= \frac{1}{b-c}\left(\int_{C(a,r)} \frac{dz}{z-b}-\int_{C(a,r)} \frac{dz}{z-c}\right)\\
                &= \frac{1}{b-c}\left(2\pi i - 2\pi i\right) = 0
            \end{align*}

            \item \ul{Si $b\in D(a,r)$ y $c\in \bb{C}\setminus \ol{D}(a,r)$}: En este caso, por lo visto en los dos primeros casos, se tiene que:
            \begin{align*}
                \int_{C(a,r)} \frac{dz}{(z-b)(z-c)} &= \frac{1}{b-c}\left(\int_{C(a,r)} \frac{dz}{z-b}-\int_{C(a,r)} \frac{dz}{z-c}\right)\\
                &= \frac{1}{b-c}\left(2\pi i - 0\right) = \frac{2\pi i}{b-c}
            \end{align*}

            \item \ul{Si $b\in \bb{C}\setminus \ol{D}(a,r)$ y $c\in D(a,r)$}: En este caso, por lo visto en los dos primeros casos, se tiene que:
            \begin{align*}
                \int_{C(a,r)} \frac{dz}{(z-b)(z-c)} &= \frac{1}{b-c}\left(\int_{C(a,r)} \frac{dz}{z-b}-\int_{C(a,r)} \frac{dz}{z-c}\right)\\
                &= \frac{1}{b-c}\left(0 - 2\pi i\right) = -\frac{2\pi i}{b-c}
            \end{align*}
        \end{enumerate}
    \end{enumerate}
\end{ejercicio}

\begin{ejercicio}
    Calcular las siguientes integrales:
    \begin{enumerate}
        \item $\displaystyle\int_{C(0,r)} \frac{z+1}{z(z^2+4)}\ dz$ \qquad ($r \in \bb{R}^+$, $r \neq 2$)
        
        Descomponemos el integrando en fracciones simples, y podemos hacer el ejercicio de dos formas distintas:
        \begin{description}
            \item [Opción 1: Calculando los valores de las constantes.] 
                \begin{equation*}
                    \frac{z+1}{z(z^2+4)} = \frac{A}{z} + \frac{B}{z-2i} + \frac{C}{z+2i}
                    = \frac{A(z^2+4) + Bz(z+2i) + Cz(z-2i)}{z(z^2+4)}
                \end{equation*}
                \begin{itemize}
                    \item Para $z=0$: $1=A4\Longrightarrow A=\nicefrac{1}{4}$.
                    \item Para $z=2i$: $2i+1 = B\cdot 2i(2i+2i)=-8B\Longrightarrow B=\nicefrac{-1}{8}-\nicefrac{i}{4}$.
                    \item Para $z=-2i$: $-2i+1 = -C\cdot 2i(-2i-2i)=-8C\Longrightarrow C=\nicefrac{-1}{8}+\nicefrac{i}{4}$.
                \end{itemize}

                Por tanto:
                \begin{multline*}
                    \int_{C(0,r)} \frac{z+1}{z(z^2+4)}\ dz = \dfrac{1}{4}\int_{C(0,r)} \frac{dz}{z} + \left(-\frac{1}{8}-\frac{i}{4}\right)\int_{C(0,r)} \frac{dz}{z-2i} \\ + \left(-\frac{1}{8}+\frac{i}{4}\right)\int_{C(0,r)} \frac{dz}{z+2i}
                \end{multline*}

                Distinguimos en función del valor de $r$:
                \begin{enumerate}
                    \item \ul{Si $r\in \left]0,2\right[$}: En este caso, se tiene que $|2i|=|-2i|=2>r$, y por el Ejercicio~\ref{ej:7.1}, las últimas dos integrales son $0$. Para la primera, considerando como función la constantemente igual a $1$, se tiene que:
                    \begin{align*}
                        \int_{C(0,r)} \frac{z+1}{z(z^2+4)}\ dz &= \dfrac{1}{4}\int_{C(0,r)} \frac{dz}{z} = \dfrac{1}{4}\cdot 2\pi i = \frac{\pi i}{2}
                    \end{align*}

                    \item \ul{Si $r>2$}: En este caso, se tiene que $|2i|=|-2i|=2<r$, y podemos usar de forma directa la Fórmula de Cauchy para la circunferencia considerando como función la constantemente igual a $1$. Por tanto, se tiene que:
                    \begin{align*}
                        \int_{C(0,r)} \frac{z+1}{z(z^2+4)}\ dz &= \dfrac{1}{4}\cdot 2\pi i + \left(-\frac{1}{8}-\frac{i}{4}\right)2\pi i + \left(-\frac{1}{8}+\frac{i}{4}\right)2\pi i\\
                        &= \dfrac{2\pi i}{4} -2\cdot \dfrac{2\pi i}{8} - \dfrac{2\pi i}{4} + \dfrac{2\pi i}{4} = 0
                    \end{align*}
                \end{enumerate}
            \item [Opción 2: Sin calcular los valores de las constantes.]~\\
                Sabemos que es posible encontrar valores $A,B,C\in \mathbb{C}$ de forma que:
                \begin{equation*}
                    \dfrac{z+1}{z(z^2+4)} = \dfrac{A}{z} + \dfrac{B}{z-2i} + \dfrac{C}{z+2i} = \dfrac{A(z^2+4)+Bz(z+2i) + Cz(z-2i)}{z(z^2+4)}
                \end{equation*}
                Igualando grado a grado (haciendo uso de que $\{1,z,z^2\}$ es una base del espacio de polinomios $\bb{P}_2$), llegamos a unas ecuaciones:
                \begin{equation*}
                    \left\{\begin{array}{rcl}
                            0 & = & A + B + C \\
                            1 & = & 2iB - 2iC \\
                            1 & = & 4A
                    \end{array}\right.
                \end{equation*}
                Que no nos molestamos en calcular. De esta forma, tenemos que la integral de partida se puede expresar como:
                \begin{equation*}
                    \int_{C(0,r)} \dfrac{z+1}{z(z^2+4)}~dz = A \int_{C(0,r)} \dfrac{dz}{z} + B\int_{C(0,r)} \dfrac{dz}{z-2i} + C\int_{C(0,r)} \dfrac{dz}{z+2i}
                \end{equation*}
                Y calculamos el valor de cada una de estas integrales:
                \begin{enumerate}
                    \item Para la primera, usando la Fórmula de Cauchy para la circunferencia considerando como función la constantemente igual a $1$, se tiene que:
                        \begin{equation*}
                            \int_{C(0,r)} \dfrac{dz}{z} = 2\pi i
                        \end{equation*}
                    \item Para la segunda y tercera, tenemos que: 
                        \begin{itemize}
                            \item Si $r < 2$, tenemos que $|-2i|=|2i| = 2 > r$, y por el Ejercicio~\ref{ej:7.1} sabemos que la integral es $0$.
                            \item Si $r > 2$, tenemos que $|-2i|=|2i| = 2 < r$, y por la Fórmula de Cauchy para la circunferencia considerando como función la constantemente igual a $1$, se tiene que la integral vale $2\pi i$.
                        \end{itemize}
                \end{enumerate}
                En resumen, tenemos que:
                \begin{enumerate}
                    \item \underline{Si $r<2$:} tenemos que las integrales de $B$ y $C$ son cero, por lo que solo calculamos el valor de $A$, que sabemos que cumple $1 = 4A$, de donde $A = \nicefrac{1}{4}$ y:
                        \begin{equation*}
                            \int_{C(0,r)} \dfrac{z+1}{z(z^2+4)}~dz = A\int_{C(0,r)} \dfrac{dz}{z} = 2\pi i A = \dfrac{\pi i}{2}
                        \end{equation*}
                    \item \underline{Si $r>2$:} tendremos:
                        \begin{equation*}
                            \int_{C(0,r)} \dfrac{z+1}{z(z^2+4)}~dz = 2\pi i (A + B + C) = 0
                        \end{equation*}
                \end{enumerate}
        \end{description}
        \item $\displaystyle\int_{C(0,1)} \frac{\cos z}{(a^2+1)z - a(z^2+1)}\ dz$ \qquad ($a \in \bb{C}$, $|a| \neq 1$)
        
        Le hallamos las raíces al denominador:
        \begin{align*}
            (a^2+1)z - a(z^2+1) = 0 &\iff -az^2 + (a^2+1)z - a=0\iff\\&\iff z=
            \frac{-(a^2+1)\pm\sqrt{(a^2+1)^2-4(-a)(-a)}}{2(-a)} \iff\\
            &\iff z=\frac{(a^2+1)\pm\sqrt{(a^2-1)^2}}{2a} = \frac{(a^2+1)\pm(a^2-1)}{2a} \iff\\
            &\iff z\in \left\{a,\nicefrac{1}{a}\right\}
        \end{align*}
        
        \begin{description}
            \item [Opción 1: Descomponiendo en fracciones simples]~\\
                Tenemos que:
                \begin{align*}
                    \frac{1}{(a^2+1)z - a(z^2+1)} &= -\dfrac{1}{a}\left(\frac{A}{z-a} + \frac{B}{z-\nicefrac{1}{a}}\right) = \frac{A(z-\nicefrac{1}{a}) + B(z-a)}{-a(z-a)(z-\nicefrac{1}{a})}
                \end{align*}
                \begin{itemize}
                    \item Para $z=a$: $1=A\left(a-\frac{1}{a}\right)\Longrightarrow A=\frac{a}{a^2-1}$.
                    \item Para $z=\nicefrac{1}{a}$: $1=B\left(\frac{1}{a}-a\right)\Longrightarrow B=\frac{a}{1-a^2}=\frac{-a}{a^2-1}$.
                \end{itemize}

                Por tanto, la integral queda:
                \begin{align*}
                    \int_{C(0,1)} \frac{\cos z}{(a^2+1)z - a(z^2+1)}\ dz &= -\frac{1}{a^2-1}\left(\int_{C(0,1)} \frac{\cos z}{z-a}\ dz-\int_{C(0,1)} \frac{\cos z}{z-\nicefrac{1}{a}}\ dz\right)
                \end{align*}

                Distinguimos casos:
                \begin{enumerate}
                    \item \ul{Si $|a|<1$}: En este caso, como $a\in D(0,1)$ para la primera integral podemos aplicar la Fórmula de Cauchy para la circunferencia considerando como función el coseno, que es una función entera:
                    \begin{equation*}
                        \int_{C(0,1)} \frac{\cos z}{z-a}\ dz = 2\pi i \cdot \cos(a)
                    \end{equation*}

                    Por otro lado, para la segunda integral, consideramos el disco $D(0,\nicefrac{1}{|a|})$, que contiene a $C(0,1)^*$. Como dicho disco es estrellado y la aplicación $z\mapsto \frac{\cos z}{z-\nicefrac{1}{a}}$ es holomorfa en $D(0,\nicefrac{1}{|a|})$, se tiene que:
                    \begin{equation*}
                        \int_{C(0,1)} \frac{\cos z}{z-\nicefrac{1}{a}}\ dz = 0
                    \end{equation*}

                    Por tanto, se tiene que:
                    \begin{align*}
                        \int_{C(0,1)} \frac{\cos z}{(a^2+1)z - a(z^2+1)}\ dz &= -\frac{1}{a^2-1}\left(2\pi i \cdot \cos(a) - 0\right)\\
                        &= -\frac{2\pi i \cdot \cos(a)}{a^2-1}
                    \end{align*}

                    \item \ul{Si $|a|>1$}: En este caso, como $\nicefrac{1}{a}\in D(0,1)$ para la segunda integral podemos aplicar la Fórmula de Cauchy para la circunferencia considerando como función el coseno, que es una función entera:
                    \begin{equation*}
                        \int_{C(0,1)} \frac{\cos z}{z-\frac{1}{a}}\ dz = 2\pi i \cdot \cos\left(\frac{1}{a}\right)
                    \end{equation*}

                    Por otro lado, para la primera integral, consideramos el disco $D(0,|a|)$, que contiene a $C(0,1)^*$. Como dicho disco es estrellado y la aplicación $z\mapsto \frac{\cos z}{z-a}$ es holomorfa en $D(0,|a|)$, se tiene que:
                    \begin{equation*}
                        \int_{C(0,1)} \frac{\cos z}{z-a}\ dz = 0
                    \end{equation*}

                    Por tanto, se tiene que:
                    \begin{align*}
                        \int_{C(0,1)} \frac{\cos z}{(a^2+1)z - a(z^2+1)}\ dz &= -\frac{1}{a^2-1}\left(0 - 2\pi i \cdot \cos\left(\frac{1}{a}\right)\right)\\
                        &= \frac{2\pi i \cdot \cos\left(\nicefrac{1}{a}\right)}{a^2-1}
                    \end{align*}
                \end{enumerate}
            \item [Opción 2: Sin descomponer en fracciones simples]~\\
                Como hemos visto anteriormente que:
                \begin{equation*}
                    (a^2 + 1)z - a(z^2+1) = 0 \Longleftrightarrow z\in \{a,\nicefrac{1}{a}\}
                \end{equation*}
                Podemos escribir:
                \begin{equation*}
                    (a^2 + 1)z - a(z^2+1) = 0 = (z-a)(z-\nicefrac{1}{a})
                \end{equation*}
                Con lo que buscamos calcular la integral:
                \begin{equation*}
                    \int_{C(0,1)} \dfrac{\cos z}{(z-a)(z-\nicefrac{1}{a})}~dz
                \end{equation*}
                Notemos que la condición $|a|\neq 1$ hace que $a\in D(0,1)\Longleftrightarrow \nicefrac{1}{a}\notin D(0,1)$, por lo que distinguiendo casos tenemos el ejercicio resuelto:
                \begin{enumerate}
                    \item \underline{Si $|a|<1$:} Tendremos que $a\in D(0,1)$ y que $\nicefrac{1}{a}\notin D(0,1)$, por lo que podemos definir la aplicación $f:\overline{D}(0,1)\to\mathbb{C}$ dada por:
                        \begin{equation*}
                            f(z) = \dfrac{\cos z}{(z-\nicefrac{1}{a})}
                        \end{equation*}
                        Que claramente es continua en $\overline{D}(0,1)$ y holomorfa en $D(0,1)$, por ser producto de una función racional de denominador no nulo por la función coseno. Aplicando una versión más relajada de la Fórmula de Cauchy para la circunferencia que probamos en el Ejercicio~\ref{ej:7.2}, tenemos que:
                        \begin{equation*}
                            \int_{C(0,1)} \dfrac{\cos z}{(z-a)(z-\nicefrac{1}{a})}~dz = \int_{C(0,1)} \dfrac{f(z)}{(z-a)}~dz = 2\pi i f(a) = \dfrac{2\pi i \cos a}{a-\nicefrac{1}{a}}
                        \end{equation*}
                    \item \underline{Si $|a|>1$:} Tendremos que $\nicefrac{1}{a}\in D(0,1)$ y que $a\in \overline{D}(0,1)$, por lo que podemos definir la aplicación $g:\overline{D}(0,1)\to \mathbb{C}$ dada por:
                        \begin{equation*}
                            g(z) = \dfrac{\cos z}{(z-a)}
                        \end{equation*}
                        Que de la misma forma es continua en $\overline{D}(0,1)$ y holomorfa en $D(0,1)$, por lo que podemos aplicar la versión del Teorema de Cauchy para la circunferencia hya comentada, teniendo que:
                        \begin{equation*}
                            \int_{C(0,1)} \dfrac{\cos z}{(z-a)(z-\nicefrac{1}{a})}~dz = \int_{C(0,1)} \dfrac{g(z)}{(z-\nicefrac{1}{a})}~dz = 2\pi i g(\nicefrac{1}{a}) = \dfrac{2\pi i \cos(\nicefrac{1}{a})}{\nicefrac{1}{a}-a}
                        \end{equation*}
                \end{enumerate}
        \end{description}

    \end{enumerate}
\end{ejercicio}

\begin{ejercicio}
    Dados $a,b \in \bb{C}$ con $a \neq b$, sea $R \in \bb{R}^+$ tal que $R > \max\{|a|,|b|\}$. Probar que, si $f$ es una función entera, se tiene:
    \[
        \int_{C(0,R)} \frac{f(z)}{(z-a)(z-b)}dz = 2\pi i \cdot \frac{f(b)-f(a)}{b-a}.
    \]
    Deducir que toda función entera y acotada es constante.\\

    Descomponemos el integrando en fracciones simples:
    \begin{equation*}
        \frac{1}{(z-a)(z-b)} = \frac{A}{z-a} + \frac{B}{z-b} = \frac{A(z-b)+B(z-a)}{(z-a)(z-b)}
    \end{equation*}
    \begin{itemize}
        \item Para $z=a$: $1=A(a-b)\Longrightarrow A=\frac{1}{a-b}$.
        \item Para $z=b$: $1=B(b-a)\Longrightarrow B=\frac{1}{b-a}=-\frac{1}{a-b}$.
    \end{itemize}

    Por tanto, la integral queda:
    \begin{align*}
        \int_{C(0,R)} \frac{f(z)}{(z-a)(z-b)}dz &= \frac{1}{a-b}\left(\int_{C(0,R)} \frac{f(z)}{z-a}dz - \int_{C(0,R)} \frac{f(z)}{z-b}dz\right)\\
        &\AstIg \frac{1}{a-b}\left(2\pi i f(a) - 2\pi i f(b)\right)\\
        &= 2\pi i \cdot \frac{f(b)-f(a)}{b-a}
    \end{align*}
    donde $(\ast)$ se debe a que la función $f(z)$ es entera y que $a,b\in D(0,R)$, por lo que se puede aplicar la Fórmula de Cauchy para la circunferencia considerando como función $f(z)$.\\

    Sea ahora $f$ entera y acotada. Entonces, $\exists M\in \bb{R}^+$ tal que $|f(z)|\leq M$ para todo $z\in \bb{C}$. Por tanto, se tiene que:
    \begin{align*}
        \left|\dfrac{f(z)}{(z-a)(z-b)}\right| &\leq \frac{M}{|z-a||z-b|}
        \leq \frac{M}{\left||z| - |a|\right|\left||z| - |b|\right|}
        \leq \frac{M}{\left|R - |a|\right|\left|R - |b|\right|}
        \leq\\&\leq  \frac{M}{(R-|a|)(R-|b|)}\qquad \forall z\in C(0,R)^*\\
    \end{align*}
    donde hemos usado que $z\in C(0,R)^*$, por lo que $|z|=R$; y que $R>\max\{|a|,|b|\}$, por lo que $R-|a|>0$ y $R-|b|>0$. Por tanto, se tiene que:
    \begin{align*}
        \left|\int_{C(0,R)} \frac{f(z)}{(z-a)(z-b)}dz\right| &\leq \int_{C(0,R)} \left|\frac{f(z)}{(z-a)(z-b)}\right|dz\\
        &\leq 2\pi R \cdot \frac{M}{(R-|a|)(R-|b|)} = \frac{2\pi M R}{(R-|a|)(R-|b|)}
    \end{align*}

    Como la anterior expresión es válida para todo $R\in \bb{R}^+$ tal que $R>\max\{|a|,|b|\}$, podemos hacer tender $R\to \infty$. Por el Lema del Sándwich, se tiene que:
    \begin{equation*}
        \lim_{R\to \infty} \int_{C(0,R)} \frac{f(z)}{(z-a)(z-b)}dz = 0
    \end{equation*}

    Por la expresión anterior a la que habíamos llegado, se tiene que:
    \begin{equation*}
        \lim_{R\to \infty} \int_{C(0,R)} \frac{f(z)}{(z-a)(z-b)}dz = \lim_{R\to \infty} 2\pi i \cdot \frac{f(b)-f(a)}{b-a} = 2\pi i \cdot \frac{f(b)-f(a)}{b-a}
    \end{equation*}

    Por la unicidad del límite, se tiene que:
    \begin{equation*}
        f(b)=f(a) \qquad \forall a,b\in \bb{C}
    \end{equation*}

    Por tanto, $f$ es constante.
\end{ejercicio}
